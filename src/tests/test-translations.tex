% arara: pdflatex
% arara: bib2gls: {group: on}
% arara: pdflatex
% arara: pdflatex
\documentclass{article}

\usepackage[T1]{fontenc}
\usepackage[utf8]{inputenc}
\usepackage[british,brazilian]{babel}
\usepackage[colorlinks]{hyperref}
\usepackage[record,
 nomain,
 nostyles,
 stylemods=bookindex,
 style=bookindex
]{glossaries-extra}
\usepackage{glossaries-prefix}

\newglossary*{en}{English Terms}
\newglossary*{pt}{Portuguese Terms}

\GlsXtrLoadResources[
  type=en,
  src=entries-en,
  sort=en-GB,
  category=en,
  field-aliases={translations-pt=user1},
  dependency-fields={user1},
  sort-label-list={user1:pt-BR:glsentryname}
]
\GlsXtrLoadResources[
  type=pt,
  src=entries-pt,
  sort=pt-BR,
  category=pt,
  field-aliases={translations-en=user1},
  dependency-fields={user1},
  sort-label-list={user1:en-GB:glsentryname}
]

\apptoglossarypreamble[en]{\selectlanguage{british}}
\apptoglossarypreamble[pt]{\selectlanguage{brazilian}}

\begin{document}
\selectlanguage{british}

The \gls{cat} sat on the \gls{mat}.

\selectlanguage{brazilian}

O \gls{gato} sentou-se no \gls{tapete}.

\renewcommand*{\glsxtrbookindexname}[1]{%
 \glsxtrifhasfield{prefix}{#1}{\xmakefirstuc\glscurrentfieldvalue\space}{}%
 \glossentryname{#1}%
 \glsxtrifhasfield{useri}{#1}{; translations: \glsxtrseelist\glscurrentfieldvalue}{}%
}
\printunsrtglossaries
\end{document}
