% arara: pdflatex
% arara: bib2gls
% arara: pdflatex
\documentclass{book}

\usepackage[colorlinks]{hyperref}

\usepackage
 [
   shortcuts,% provide shortcut commands like \ac
   section,% use section headings
   subentrycounter,% enable numbered sub-entries
   nostyles,stylemods={tree,long,bookindex},% only need these styles
   style=tree,
   abbreviations,index,
   record=nameref
 ]
 {glossaries-extra}


 % Define some convenient commands

\newcommand*{\principleloc}[1]{\textbf{#1}}
\newcommand*{\glossaryloc}[1]{\textit{#1}}
\newcommand*{\appname}[1]{\texttt{#1}}

 % Add custom fields

\glsaddkey{ed}% key
 {}% default value assigned with assign-fields resource option
 {\entryed}% no link cs
 {\Entryed}% no link Cs
 {\glsed}% link cs
 {\Glsed}% link Cs
 {\GLSed}% link CS

\glsaddkey{ing}% key
 {}%
 {\entrying}% no link cs
 {\Entrying}% no link Cs
 {\glsing}% link cs
 {\Glsing}% link Cs
 {\GLSing}% link CS

\glsaddkey{altplural}% key
 {}% 
 {\entryaltpl}% no link cs
 {\Entryaltpl}% no link Cs
 {\glsaltpl}% link cs
 {\Glsaltpl}% link Cs
 {\GLSaltpl}% link CS

 % Make a notation glossary labelled 'notation' with given title

\newglossary*{notation}{Notation}

\GlsXtrLoadResources[
 src={sample-gidx-glossary},
 type=main,% default glossary
 post-description-dot=check,
 assign-fields =
  {
     ed = text + "ed",
     ing = text + "ing"
  },
 name-case-change=firstuc, % make the first letter of the name upper case
 identical-sort-action=def % order entries with identical sort values according to their definition
]

\GlsXtrLoadResources[src={sample-gidx-index},
 type=index,
 assign-fields =
  {
     ed = text + "ed",
     ing = text + "ing"
  }
]

\GlsXtrLoadResources[src={sample-gidx-notation},
 type=notation,
 post-description-dot=check,
 set-widest, % needed for alttree style
 name-case-change=firstuc, % make the first letter of the name upper case
 sort=none % don't sort
]

\setabbreviationstyle{long-short-sc}
\GlsXtrLoadResources[src={sample-gidx-acronyms},
 type=\glsxtrabbrvtype,
 post-description-dot=check,
 name-case-change=firstuc, % make the first letter of the name upper case
]

\title{Sample Document Using the glossaries-extra Package}
\author{Nicola L. C. Talbot}

\begin{document}
\pagenumbering{alph}% Stop hyperref complaining about duplicate page identifiers
\maketitle
\thispagestyle{empty}%

\frontmatter

\tableofcontents

\chapter{Summary}

This is a modification sample document illustrating the use of the
\texttt{datagidx} package to create document \glspl{index},
\glspl{glossarylist} and \glspl{acronymlist} without the use of
external \glsing{ind-index} % custom command defined earlier
applications, such as \gls{makeindex} or
\gls{xindy}. Please read the user guide (datatool-user.pdf) for
further guidance. This modified document has been adapted to use
\texttt{glossaries-extra} and \appname{bib2gls} instead of using
\texttt{datagidx}. The definitions were converted with the aid of
\appname{datatool2bib}.

\mainmatter

\chapter{Introduction}
%\renewcommand{\thepage}{\numberstring{page}}

Words can be \glsed{ind-index}. % custom command defined earlier.

A \gls{glossarylist}\glsadd[format=principleloc]{ind-glossary} is a useful addition to any technical document,
although a \gls{glossarycol} can also simply be a collection of
glosses, which is another thing entirely. Some documents have
multiple \glspl{glossarylist}.

A \gls{bravocry} is a cry of approval (plural \glspl{bravocry}) but a 
\gls{bravokiller} can also be a hired ruffian or killer (plural
\glspl{bravokiller}).

\section{Characters}

When defining entries be careful of \glspl{comma} and \glspl{equals}
so they don't interfere with the key=value mechanism. The sort and
label keys get expanded so be careful of special characters, such as
\gls{amp}, \gls{underscore}, \gls{circum}, \gls{dollar} and \gls{tilde}.

Since we're not using \gls{makeindex}, we don't need to worry about
\gls{makeindex}'s special characters, such as \gls{double-quote}, \gls{!} and
\gls{|}. (Unless they've been made active by packages such as
\texttt{ngerman} or \texttt{babel}.)

Non-alphabetical characters are usually grouped at the start of an
index, and are usually followed by the number group. That is, the
group of entries that are numerical, such as \gls{0}, \gls{1},
\gls{2} and \gls{3}.

\section{Custom Fields}

You can add custom fields. For example, this document has added
three custom fields to the `index' database.

\section{Plurals}

The \gls{plural} of \gls{glossarylist} is
\glspl{glossarylist}. The \gls{plural} of \gls{index} is
\glspl{index}. Some words have an \gls{altplural}. For example,
an alternative to \glspl{index} is 
\glsaltpl{ind-index}.% custom command defined earlier

\section{Sorting}

There are lots of different types of sorting available. 
Word sorting is the default with \appname{bib2gls} as well as with
\gls{makeindex} and \gls{xindy}.
The difference between letter sorting and word sorting can
be demonstrated by the ordering of ``\gls{sea-lion}'' and ``\gls{seal}''.

The default sort is case-insensitive so \gls{kite} before
\gls{knuth} and \gls{knuth} before \gls{koala}.

\section{Using without indexing}

Here's a defined entry that won't get into the glossary.
Name: \glsentryname{pglist}.
Description: \glsentrydesc{pglist}.
(Unless I~later reference it using a command like \verb|\gls|.)
However, if it isn't indexed it won't be selected by default and so
won't be defined. This means that the above commands will expand to
nothing. An alternative is to index it with an ignored location or
select all entries and apply filtering to \verb|\printunsrtglossary|
to omit entries without locations.

\section{Links to Entries}

You can reference and index entries using \verb|\gls|, \verb|\Gls|,
\verb|\glspl|, \verb|\Glspl|, \verb|\glssymbol| and \verb|\Glssymbol|.
(Note, the \texttt{datagidx} and \texttt{glossaries} packages have
different syntax for these type of commands.)

Or you can reference a particular field using the corresponding
command, such as \verb|\glsdesc| or
\verb|\Glsdesc|. So here's the description for \gls{seal}:
\glsdesc{seal}.

If the \texttt{hyperref} package has been loaded, commands like
\verb|\gls| will link to the relevant entry in the glossary or
index. Referencing using commands like \verb|\glsentrytext| and
\verb|\Glsentrytext| won't have hyperlinks.

\subsection{Enabling and Disabling Hyperlinks}

If the \texttt{hyperref} package has been loaded, hyperlinks can be
enabled and disabled. Either globally
\glsdisablehyper
(here's a reference to \gls{seal} without a hyperlink 
\glsenablehyper
 and here's a reference to \gls{seal} with a hyperlink)
or locally
({%
  \glsdisablehyper
  here's a reference to \gls{seal} without a hyperlink
}%
and here's a reference to \gls{seal} with a hyperlink).

\section{Acronyms}

\glsadd{firstuse}Here's an \gls{acronym} referenced using \verb|\ac|: \ac{html}. And here
it is again: \ac{html}. That command requires the \texttt{shortcuts}
package option. Unlike \texttt{datagidx}, this is the same as using
\verb|\gls|: \gls[prereset]{html}.  And again: \gls{html}.

In order to display abbreviations with a footnote, it's necessary to
set a footnote style prior to
defining all abbreviations with the associated category.

Some more examples
First use: \gls{xml}. Next use: \gls{xml}.
First use: \gls{css}. Next use: \gls{css}.

\glsadd{reset}Reset: \glsresetall[\glsxtrabbrvtype]% just the abbreviations
Here are the acronyms again:
\gls{html}, \gls{xml} and \gls{css}.
Next use:
\gls{html}, \gls{xml} and \gls{css}.
Full form:
\gls{html}, \gls{xml} and \gls{css}.

Reset again. \glsresetall[\glsxtrabbrvtype]
Start with a capital. \Gls{html}.
Next: \Gls{html}. Full: \Glsxtrfull{html}.

\Gls{css}. Next: \gls{css}. Full: \glsxtrfull{css}.

\Gls{xml}. Next: \gls{xml}. Full: \glsxtrfull{xml}.

\section{Conditionals}

You can test if a term has been defined using \verb|\ifglsentryexists|.
For example: \ifglsentryexists{seal}{seal exists}{seal doesn't exist}.
Another example: \ifglsentryexists{jabberwocky}{jabberwocky
exists}{jabberwocky doesn't exist}.

You can test if a term has been used via \verb|\ifglsused|.
For example: \ifglsused{seal}{seal has been used}{seal hasn't been
used}.
Another example: \ifglsused{cardinality}{cardinality has been
used}{cardinality
hasn't been used}.

Example of a term that isn't selected (because it hasn't
been indexed) and so
won't be defined: \ifglsused{pglist}{pglist has been used}{pglist
hasn't been used}.

With \appname{bib2gls}, a more reliable command is
\verb|\GlsXtrIfUnusedOrUndefined|
For example, \GlsXtrIfUnusedOrUndefined{pglist}{pglist
has either not been used or is undefined}{pglist is defined
and has been used}.
Another example, \GlsXtrIfUnusedOrUndefined{seal}{seal
has either not been used or is undefined}{seal is defined
and has been used}.
Another example, \GlsXtrIfUnusedOrUndefined{jabberwocky}{jabberwocky
has either not been used or is undefined}{jabberwocky is defined
and has been used}.

\section{Symbols}

Terms may have an associated symbol. The symbol can be accessed
using \verb|\glssymbol| or if you don't want to add information to the
location list you can use \verb|\glsentrysymbol|. Here's the symbol
associated with the \gls{cardinality} entry:
\glsentrysymbol{cardinality}.

A \gls{set} (denoted \glssymbol{set}) is a collection of objects.
The \gls{universal} is the set of everything.
The \gls{empty} contains no elements.
The \gls{cardinality} of a set (denoted \glssymbol{cardinality}) is the
number of elements in the set.

\section{Location Ranges}

A range is formed if a location sequence contains more than 2
locations. Here's \gls{seal} again.

\backmatter
 % suppress section numbering
\setcounter{secnumdepth}{-1}

\chapter{Glossaries}

\printunsrtglossary
 [% the default glossary
   style=treenoname,% use 'treenoname' style (don't print sub item names)
%   nonumberlist% hide the location list
 ]

\printunsrtglossary
 [
   type=\glsxtrabbrvtype,% 'abbreviations' list
   style=long,% use the 'long' style
   subentrycounter=false,
%   nonumberlist% hide the location list
 ]

\printunsrtglossary
 [
   type=notation,% 'notation' list
   subentrycounter=false,
   style=alttree
 ]

\renewcommand{\glsxtrbookindexprelocation}[1]{%
  \glsxtrifhasfield{location}{#1}%
  {\dotfill}%
  {\glsxtrprelocation}%
}

% don't balance columns:
%\renewcommand*{\glsxtrbookindexmulticolsenv}{multicols*}

\setglossarysection{chapter}

\printunsrtglossary
 [
   type=index,% 'index' list
   subentrycounter=false,
   style=bookindex,
   preamble={Locations in bold indicate primary reference.
    Locations in italic indicate definitions in the
    glossaries.}
 ]


\end{document}
