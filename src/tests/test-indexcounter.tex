% arara: pdflatex
% arara: bib2gls
% arara: pdflatex
\documentclass{article}

\usepackage{amsmath}
\usepackage{lipsum}
\usepackage[colorlinks]{hyperref}
\usepackage[record,indexcounter]{glossaries-extra}

\newcommand{\primary}[1]{\hyperbf{#1}}

\renewcommand{\glslinkpresetkeys}{%
 \ifmmode \setkeys{glslink}{counter=equation}\fi}
\renewcommand{\glsaddpresetkeys}{%
 \ifmmode \setkeys{glossadd}{counter=equation}\fi}

\newcommand{\wrglossaryname}{page}
\newcommand{\wrglossarysname}{pages}

\newcommand{\countertag}[1]{\ifcsdef{#1name}{\csuse{#1name}}{#1}}
\newcommand{\counterpltag}[1]{\ifcsdef{#1sname}{\csuse{#1sname}}{#1s}}

\newcommand*{\bibglslocationgroup}[3]{%
\ifnum#1=1
  \countertag{#2}:
\else
  \counterpltag{#2}:
\fi
#3%
}

\newcommand*{\bibglslocationgroupsep}{; }

\GlsXtrLoadResources[src={test-entries},
 save-index-counter=primary,
 loc-counters={wrglossary,equation}
]

\renewcommand{\glsnamefont}[1]{\GlsXtrIndexCounterLink{#1}{\glscurrententrylabel}}

\begin{document}

A \gls{sample}. \lipsum*[1] A \gls{duck}.

An equation:
\begin{equation}
a = \gls{pi}r^2\label{eq:area}
\end{equation}
And some more equations:
\begin{align}
A &= b\\
f(x) &= \gls[counter=equation]{pi}x
\end{align}

\lipsum[2]

Another \gls[format=primary]{sample}. \lipsum*[3] Another \gls{duck}.

\gls{pi}. \lipsum[4]

A \gls{sample}. \lipsum*[5] A \gls{duck} and \gls[format=primary]{pi}.

\lipsum*[6] A \gls[format=primary]{duck}.

\printunsrtglossaries
\end{document}
