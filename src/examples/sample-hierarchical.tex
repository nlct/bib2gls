% This file is public domain. See the "Examples" chapter
% in the bib2gls user manual for a more detailed description
% of this file.

\documentclass[12pt,a4paper]{article}

\usepackage[T1]{fontenc}
\usepackage[colorlinks]{hyperref}

\usepackage[record,% use bib2gls
 nostyles,% don't load default styles
 postdot,% add a full stop after the description
% load glossary-tree.sty and patch styles:
 stylemods={tree},
 style=treegroup]{glossaries-extra}

\GlsXtrLoadResources[
  src={terms,animals,minerals,vegetables},% data these .bib files
  field-aliases={identifier=parent}
]

\begin{document}
Some sample terms: \gls{duck}, \gls{sealion}, \gls{armadillo},
\gls{seal}, \gls{aardvark}, \gls{amethyst}, \gls{aquamarine},
\gls{diamond}, \gls{dolomite}, \gls{chalcedony}, \gls{citrine},
\gls{quetzalcoatlite}, \gls{cabbage}, \gls{cauliflower},
\gls{artichoke}, \gls{courgette}.

\GlsXtrSetDefaultNumberFormat{glsignore}% ignore records in the glossary
\printunsrtglossary
\end{document}
