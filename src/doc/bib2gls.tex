% arara: lualatex
% arara: bib2gls: {group: on} if missing("glstex") || changed("bib") || found("log", "Warning: Glossary entry")
% arara: bibtex
% arara: lualatex if changed("glstex") || missing("toc")
% arara: bib2gls: {group: on, packages: [mfirstuc-english]}
% arara: lualatex
% arara: lualatex
\documentclass[titlepage=false,index=totoc,bibliography=totoc,
 fontsize=12pt,captions=tableheading]{scrreprt}

\usepackage[autooneside=false]{scrlayer-scrpage}
\usepackage{pifont}

% Need support for less common extended characters
% and need a mono font that supports hyphenation
\usepackage[no-math]{fontspec}
\setmainfont{Linux Libertine O}

\newfontface\cyrillicmono{FreeMono}[Scale=MatchLowercase]
\newcommand{\textcyrillicmono}[1]{{\cyrillicmono #1}}

\newfontface\greekmono{FreeMono}[Scale=MatchLowercase]
\newfontface\freeserif{FreeSerif}
\newcommand{\devone}{{\freeserif\char"0967}}
\newcommand{\devtwo}{{\freeserif\char"0968}}
\newcommand{\devsix}{{\freeserif\char"096C}}
\newcommand{\insularG}{{\freeserif\char"A77D}}
\newcommand{\insularg}{{\freeserif\char"1D79}}
\newcommand{\longs}{\char"017F}
\newcommand{\Wynn}{\char"01F7}
\newcommand{\wynn}{\char"01BF}

\usepackage[x11names]{xcolor}
\usepackage{alltt}
\usepackage{array}
\usepackage{upquote}
\usepackage{etoolbox}

\usepackage{hologo}
\usepackage{amsmath}
\usepackage{accents}
\usepackage{tikz}
\usepackage{tcolorbox}
\usepackage{datetime2}
\usepackage{mfirstuc-english}
\usepackage{listings}
\usepackage{scrhack}
\usepackage[hidelinks]{hyperref}
\usepackage{cleveref}
\usepackage[% v1.47+
   record,% use bib2gls
   index,% create index glossary
   entrycounter,% enable entry counting
   subentrycounter,% enable entry counting
   %debug=showtargets,% debugging information
   %debug=showwrgloss,% debugging information
   stylemods={mcols,bookindex},% adjust predefined styles and load glossary-mcols.sty and glossary-bookindex.sty
   style=bookindex
 ]
 {glossaries-extra}

\lstset{language={[LaTeX]TeX},upquote,basicstyle={\ttfamily\small},
 commentstyle={\color{gray}}}

\renewcommand*{\glsshowtarget}[1]{\texttt{\small [#1]}}

\glsxtrprovidestoragekey{note}{}{}
\glsxtrprovidestoragekey{package}{}{}
\glsxtrprovidestoragekey{nametitle}{}{}

\providecommand{\omicron}{o}

% put break before dash to avoid confusion with a hyphen
\newcommand{\dhyphen}{%
 \texorpdfstring
 {\discretionary{}{}{}\texttt{-}}%
 {-}%
}

% Redefine \- (used in entry names) so that hyphenation in command names, 
% switches options etc uses a serif hyphen to reduce confusion with an 
% actual hyphen in the code.
\renewrobustcmd{\-}{%
 \discretionary
 {{\rmfamily\char\ifnum\hyphenchar\font<0
  \defaulthyphenchar\else\hyphenchar\font\fi
 }}%
 {}{}%
}

\renewcommand{\glsentrycounterlabel}{}
\renewcommand{\glssubentrycounterlabel}{}
\renewcommand\glstreeitem{\par\hangindent10pt}
\renewcommand\glstreesubitem{\par\hangindent30pt\hspace*{15pt}}

\newcommand{\bibglsseealsosep}{\glstreesubitem}

\renewcommand*{\glsseesep}{, \glstreesubitem\quad}
\renewcommand*{\glsseelastsep}{\glstreesubitem\quad\strut\llap{\andname\space}}

\newcommand{\glsxtrpostnameabbreviation}{%
 \space(\glsentrylong{\glscurrententrylabel})%
}

\newcommand{\glsxtrpostnameenvironment}{\space environment}
\newcommand{\glsxtrpostnamecounter}{\space counter}

\newcommand*{\csfmtfont}[1]{\texttt{#1}}
\newcommand*{\csfmtcolourfont}[1]{\texttt{\textcolor{cs}{#1}}}

\newcommand*{\csfmt}[1]{%
 \texorpdfstring
 {\csfmtfont{\char`\\ #1}}%
 {\string\\#1}%
}

\newcommand{\texparserdefprefix}{}
\newcommand{\texparserdefnotetarget}{%
  \raisebox{2ex}{\hypertarget{\texparserdefprefix texparserdef}{}}%
  \textsuperscript{*}Indicates command is
  recognised by \bibgls's interpreter although it may have
  a slightly different implementation.
}
\newcommand{\texparserdefnote}{\hyperlink{\texparserdefprefix texparserdef}{*}}

\setabbreviationstyle{long-short-sc}
\setabbreviationstyle[common]{short-sc-nolong}

\glsxtrprovidestoragekey{unsortedprogeny}{}{}

\newcommand*{\longargfmt}[1]{%
 \texorpdfstring{\texttt{\longswitch #1}}%
 {\string-\string-#1}%
}

\GlsXtrLoadResources[
  entry-type-aliases={
   bibglscommand=dualindexentry,
   mainglscommand=spawndualindexentry,
   glscommand=spawnentry,
   abbrvstylecommand=spawnentry,
   glostylecommand=spawnentry,
   resourceoption=dualindexentry,
   packageoption=dualindexentry,
   field=dualindexentry,
   entrytype=dualindexentry,
   switch=dualindexentry,
   samplefile=dualindexentry,
   examplecommand=dualindexentry,
   examplesymbol=symbol,
   exampleabbreviation=abbreviation,
   exampleentry=entry
  },
  max-loc-diff={2},
  see=omit,seealso=omit,alias=omit,
  field-aliases={topics=adoptparents,annote=user2},
  replicate-fields={name=nametitle,progeny=unsortedprogeny},
  word-boundaries={white space,cs space,dash},
  field-case-change={nametitle=title},
  interpret-fields={nametitle},
  no-case-change-cs={fieldfmt,abbrstylefmt,glostylefmt},
  short-case-change={lc},
  dual-short-case-change={lc},
  unknown-entry-alias={index},
  save-original-entrytype,
  label-prefix={idx.},
  save-loclist=false,
  sort={letternumber-nocase},
  sort-replace={{\glshex2423}{ },{\string\\([a-zA-Z])}{\glscapturedgroup1}},
  type={index},
  dual-field={dualid},
  match-action={add},
  match-op={or},
  match={
   {category}={.*field},
   {original entrytype}={.*(gls|style)command},
   {original entrytype}={.*option},
   {original entrytype}={abbreviationstyle},
   {original entrytype}={glossarystyle},
  },
  dual-prefix={},
  dual-type={main},
  ext-prefixes={},
  combine-dual-locations={primary},
  selection={recorded and deps and see not also},
  symbol-sort-fallback={name},
  sort-label-list={progeny:letter-nocase:glsxtrentryparentname},
  strip-missing-parents,
  save-child-count
]

\newglossary*{terms}{Glossary}
\GlsXtrLoadResources[src=bib2gls-terms,
 name-case-change=title,replicate-fields={name=text},
 type=terms,save-locations=false,category=term]

\newcommand{\igls}[2][]{%
 \glsxtrifhasfield{useri}{#2}{\glsadd{\glscurrentfieldvalue}}{}%
 \gls[#1]{#2}%
}
\newcommand{\iglspl}[2][]{%
 \glsxtrifhasfield{useri}{#2}{\glsadd{\glscurrentfieldvalue}}{}%
 \glspl[#1]{#2}%
}
\newcommand{\primary}{\glsdisp{primaryentry}{primary}}
\newcommand{\dual}{\glsdisp{dualentry}{dual}}
\newcommand{\parent}{\glsdisp{parententry}{parent}}
\newcommand{\child}{\glsdisp{childentry}{child}}
\newcommand{\children}{\glsdisp{childentry}{children}}
\newcommand{\hierarchicallevel}{\glsdisp{hierarchicalglossary}{hierarchical level}}
\newcommand{\hierarchical}{\glsdisp{hierarchicalglossary}{hierarchical}}
\newcommand{\hierarchically}{\glsdisp{hierarchicalglossary}{hierarchically}}
\newcommand{\hierarchy}{\glsdisp{hierarchicalglossary}{hierarchy}}
\newcommand{\mainglossary}{\glsdisp{mainglossary}{\code{main} glossary}}
\newcommand{\supplementallocation}{\glsdisp{supplementalrecord}{supplemental
location}}
\newcommand{\supplementallocations}{\glsdisp{supplementalrecord}{supplemental
locations}}
\newcommand{\supplementarylocation}{\glsdisp{supplementalrecord}{supplementary
location}}
\newcommand{\supplementarylocations}{\glsdisp{supplementalrecord}{supplementary
locations}}
\newcommand{\supplementarydocument}{\glsdisp{supplementaldocument}{supplementary
document}}
\newcommand{\supplementarydocuments}{\glsdisp{supplementaldocument}{supplementary
documents}}
\newcommand{\supplemental}{\glsdisp{supplementaldocument}{supplemental}}
\newcommand{\supplementary}{\glsdisp{supplementaldocument}{supplementary}}

\glsxtraddlabelprefix{}
\glsxtraddlabelprefix{idx.}

\DTMsavetimestamp{creation}{2017-01-20T15:39:00Z}

\IfFileExists{../java/common/BibGlsTeXApp}
{
  \DTMsavefilemoddate{moddate}{../java/common/BibGlsTeXApp.java}
}
{
  \DTMsavenow{moddate}
}

\newif\ifmainmatter

% ignore locations outside of main matter
\GlsXtrSetDefaultNumberFormat{glsignore}

\GlsXtrSetPlusModifier{format=glsnumberformat}
\GlsXtrSetAltModifier{!}{format=glsignore}

\newcommand{\frontmatter}{%
 \clearpage\pagenumbering{roman}%
 \pagestyle{scrheadings}%
 \automark[section]{chapter}%
 \mainmatterfalse
}
\newcommand{\mainmatter}{%
 \clearpage\pagenumbering{arabic}%
 \GlsXtrSetDefaultNumberFormat{glsnumberformat}%
 \mainmattertrue
}

\newcommand{\rightleftmark}[2]{\ifstrequal{#1}{#2}{#1}{#1: #2}}
\makeatletter
\newcommand{\setsecdepth}{\@ifstar\s@setsecdepth\@setsecdepth}
\newcommand{\s@setsecdepth}[1]{%
 \setcounter{secnumdepth}{#1}%
 \automark[section]{chapter}%
 \let\@rightmark\rightleftmark
}
\newcommand{\@setsecdepth}[1]{%
 \setcounter{secnumdepth}{#1}%
 \ifnum#1 > 0
  \automark[section]{chapter}%
 \else
  \automark{chapter}%
 \fi
 \let\@rightmark\@secondoftwo
}
\makeatother

\newtcolorbox{important}{colback=red!5!white,colframe=red}

\newcommand{\bibgls}{\appfmt{bib2gls}}

\newcommand*{\BibTeX}{\hologo{BibTeX}}
\newcommand*{\eTeX}{\hologo{eTeX}}
\newcommand*{\XeLaTeX}{\hologo{XeLaTeX}}
\newcommand*{\LuaLaTeX}{\hologo{LuaLaTeX}}
\newcommand*{\pdfLaTeX}{\hologo{pdfLaTeX}}

\newcommand{\langxml}{\hyperref[sec:lang.xml]{language resource file}}
\newcommand{\texparserlib}{\hyperref[sec:texparserlib]{\TeX\ Parser
Library}}

\newcommand{\hex}[1]{0x#1}
\newcommand{\hexsb}[1]{\textsubscript{\scriptsize\rmfamily#1}}
\newcommand{\subfigfmt}[1]{(\emph{#1})}

\newcommand{\qt}[1]{``#1''}

\newcommand{\qtt}[1]{\qt{\,\texttt{#1}\,}}

\newcommand{\qtdelim}[1]{\idx{doublequotechardelim}#1\idx{doublequotechardelim}}

\newcommand{\ifnumhex}[1]{\cs{ifnum}\idx{doublequotecharhex}#1}

\newcommand{\charhex}[1]{\cs{char}\idx{doublequotecharhex}#1}
\newcommand{\icharhex}[1]{\ics{char}\idx[format=glsnumberformat,noindex=false]{doublequotecharhex}#1}

\newcommand{\era}[1]{\texorpdfstring{\textsc{\MakeLowercase{#1}}}{#1}}

\newcommand{\dequals}{%
 \texorpdfstring
 {\discretionary{}{}{}\texttt{=}\discretionary{}{}{}}%
 {=}%
}

\newcommand{\dcomma}{%
 \texorpdfstring
 {\texttt{,}\discretionary{}{}{}}%
 {,}%
}

\newcommand{\dcolon}{%
 \texorpdfstring
 {\texttt{:}\discretionary{}{}{}}%
 {:}%
}

\pdfstringdefDisableCommands{%
  \def\dhyphen{-}%
  \def\dcolon{:}%
  \def\dcomma{,}%
  \def\dequals{,}%
  \let\-\empty
}

% Provide commands that work like \gls[#1]{idx.#2}[#3]

\glsxtrnewglslike{idx.}{\idx}{\idxpl}{\Idx}{\Idxpl}
\newcommand{\idxlink}[3][]{\glslink[#1]{idx.#2}{#3}}

\newcommand{\stringu}{\idx{cs.string}\gls{uhex}}

\newcommand*{\filefmt}[1]{\texorpdfstring{\texttt{#1}}{#1}}
\newcommand*{\metafilefmt}[3]{%
  \filefmt{#1}\discretionary{}{}{}\meta{#2}\discretionary{}{}{}\filefmt{#3}%
}
\newcommand*{\metametafilefmt}[5]{%
  \filefmt{#1}\discretionary{}{}{}\meta{#2}\discretionary{}{}{}\filefmt{#3}%
  \discretionary{}{}{}\meta{#4}\discretionary{}{}{}\filefmt{#5}%
}

\newcommand*{\primaryresourcefmt}{%
 \texorpdfstring
  {\texttt{\char`\\ jobname.glstex}}%
  {\string\\jobname.glstex}%
}

\newcommand*{\suppresourcefmt}[1]{%
 \texorpdfstring
  {\texttt{\char`\\ jobname-#1.glstex}}%
  {\string\\jobname-#1.glstex}%
}

\definecolor{field}{named}{DarkSlateGray4}
\definecolor{cs}{named}{DarkSeaGreen4}
\definecolor{styopt}{named}{DarkOrchid4}
\definecolor{entry}{named}{SteelBlue4}
\definecolor{comment}{named}{gray}
\definecolor{attribute}{named}{Purple4}
\definecolor{style}{named}{Blue4}

\newcommand{\extstyopt}{\textsuperscript{\textdagger}}

\newcommand*{\appfmt}[1]{\texorpdfstring{\texttt{#1}}{#1}}
\newcommand*{\styfmt}[1]{\texorpdfstring{\textsf{#1}}{#1}}
\newcommand*{\envfmt}[1]{\texorpdfstring{\textsf{#1}}{#1}}
\newcommand*{\optfmt}[1]{\texorpdfstring{\texttt{#1}}{#1}}
\newcommand*{\csoptfmt}[1]{\texorpdfstring{\textcolor{cs}{\optfmt{#1}}}{#1}}
\newcommand*{\styoptfmt}[1]{\texorpdfstring{\textcolor{styopt}{\optfmt{#1}}}{#1}}
\newcommand*{\fieldfmt}[1]{\texorpdfstring{\texttt{\color{field}#1}}{#1}}
\newcommand*{\entryfmt}[1]{\texorpdfstring{\texttt{\color{entry}#1}}{#1}}
\newcommand*{\atentryfmt}[1]{\entryfmt{@#1}}
\newcommand*{\abbrstylefmt}[1]{\texorpdfstring{\textsf{#1}}{#1}}
\newcommand*{\glostylefmt}[1]{\texorpdfstring{\textsf{#1}}{#1}}
\newcommand*{\catattrfmt}[1]{\texorpdfstring{\textsf{#1}}{#1}}
\newcommand*{\counterfmt}[1]{\texorpdfstring{\textsf{#1}}{#1}}
\newcommand*{\idprefixfmt}[1]{%
 \texorpdfstring{\texttt{#1\spacefactor999 .}}{#1.}}
\newcommand*{\xmltagfmt}[1]{\texorpdfstring{\texttt{#1}}{#1}}

\newcommand*{\argor}{\texorpdfstring{\protect\textbar}{|}}

\newrobustcmd*{\texmeta}[1]{{\normalfont\normalcolor$\langle$\emph{#1}$\rangle$}}

\newcommand*{\meta}[1]{%
 \texorpdfstring{\ifmmode\text{\texmeta{#1}}\else\texmeta{#1}\fi}{#1}%
}

\newcommand*{\oarg}[1]{\discretionary{}{}{}[#1]}
\newcommand*{\oargm}[1]{\oarg{\meta{#1}}}

\newcommand*{\marg}[1]{\texorpdfstring
 {\discretionary{}{}{}\char`\{#1\char`\} }%
 {\{#1\}}%
}

\newcommand*{\margm}[1]{\marg{\meta{#1}}}

\newcommand*{\file}[1]{%
 \texorpdfstring
 {\idx{file.#1}}%
 {#1}%
}

\newcommand{\texmfcnffmt}[1]{\texttt{#1}}

\glsxtrnewgls{idx.texmfcnf.}{\texmfcnf}

\newcommand*{\extfmt}[1]{\filefmt{.#1}}%

\glsxtrnewgls[format=glsignore]{idx.ext.}{\ext}

\newcommand*{\iext}[1]{%
 \glsxtrtitleorpdforheading{\idx{ext.#1}}{.#1}{\extfmt{#1}}%
}

\glsxtrnewgls[format=glsignore]{idx.}{\sty}

\newcommand*{\isty}[1]{%
  \texorpdfstring{\idx[noindex=false]{#1}}{#1}%
}

\newcommand*{\env}[1]{%
  \texorpdfstring{\idx{env.#1}}{#1}%
}

\newcommand{\glsxtrpostdescenvironment}{environment}

\newcommand*{\abbrstyle}[1]{%
  \texorpdfstring{\idx{abbrstyle.#1}}{#1}%
}

\newcommand*{\topicabbrstyle}[1]{%
  %\texorpdfstring{\idx[prefix=topic.]{abbrstyle.#1}}{#1}%
  \texorpdfstring{\idx{abbrstyle.#1}}{#1}%
}

\newcommand*{\glostyle}[1]{%
  \texorpdfstring{\idx{glostyle.#1}}{#1}%
}

\newcommand*{\topicglostyle}[1]{%
 %\texorpdfstring{\idx[prefix=topic.]{glostyle.#1}}{#1}%
 \texorpdfstring{\idx{glostyle.#1}}{#1}%
}

\newcommand*{\iglostyle}[1]{%
  \texorpdfstring{\idx[noindex=false]{glostyle.#1}}{#1}%
}

\newcommand*{\catattr}[1]{%
  \texorpdfstring{\idx{catattr.#1}}{#1}%
}

\newcommand*{\counter}[1]{%
  \texorpdfstring{\idx{ctr.#1}}{#1}%
}

\newcommand{\glsxtrpostdesccounter}{counter}

\newcommand*{\idprefix}[1]{%
 \texorpdfstring{\idx{idprefix.#1}}{#1}%
}


\newcommand*{\styopt}[2][]{%
  \texorpdfstring%
  {%
    \gls{styopt.#2}\styoptfmt{\ifblank{#1}{}{\dequals\marg{#1}}}%
  }%
  {#2\ifblank{#1}{}{=#1}}%
}

\newcommand*{\istyopt}[2][]{%
  \texorpdfstring%
  {%
    \gls[noindex=false]{styopt.#2}\styoptfmt{\ifblank{#1}{}{\dequals\marg{#1}}}%
  }%
  {#2\ifblank{#1}{}{=#1}}%
}

\newcommand*{\styoptref}[1]{\gls[format=glsignore]{styopt.#1}}

\newcommand*{\keyval}{%
 \texorpdfstring
 {\meta{key}\dequals \meta{value}}%
 {key=value}%
}

\newcommand*{\keyvallist}{%
 \texorpdfstring
 {key\dequals value list}%
 {key=value list}%
}

\newcommand{\cssectioncmd}{\subsection}

\newcommand{\cssection}[2][\cssectioncmd]{%
 #1{\texorpdfstring{\glsxtrglossentry{#2}}{\string\\#2}}%
}

\newcommand{\nameparen}[1]{ (#1)}
\newcommand{\prefixedcssection}[3][\cssectioncmd]{%
 {%
   \renewcommand{\nameparen}[1]{}%
   #1{\texorpdfstring{\glsxtrglossentry{#2.#3}}{\string\\#3}}%
 }%
}

\newcommand{\format}[1]{%
  \gls{#1}\glsentryuseri{#1}%
}

\newcommand{\formatdef}[1]{%
  \begin{definition}%
  \gls[hyper=false]{#1}\glsentryuseri{#1}%
  \end{definition}\ignorespaces
}

\newcommand{\nosecformatdef}[1]{%
  \begin{definition}
   \renewcommand{\GlsXtrStandaloneEntryName}[1]{\glstarget{##1}{\strut}\cs{##1}}%
   \glsadd{#1}\glsxtrglossentry{#1}%
   \glsentryuseri{#1}%
  \end{definition}\ignorespaces
}

\newcommand{\nosecdef}[1]{%
  \begin{flushleft}\ttfamily
   \renewcommand{\GlsXtrStandaloneEntryName}[1]{\glstarget{##1}{\strut}\cs{##1}}%
   \glsadd{#1}\glsxtrglossentry{#1}%
   \glsentryuseri{#1}%
  \end{flushleft}\ignorespaces
}

\newcommand{\inlinedef}[1]{%
  \code
  {%
   \glsadd{#1}\glsxtrglossentry{#1}%
   \glsentryuseri{#1}%
  }%
}

\newcommand*{\csref}{\gls}

\newcommand*{\cs}[1]{%
  \glsdoifexists{idx.#1}%
  {\glsdohyperlink{\glolinkprefix cs.idx.#1}{\glsentryname{idx.#1}}}%
}

\glsxtridentifyglslike{idx.}{\cs}

\newcommand*{\ics}[2][]{\glsadd[#1]{idx.#2}\cs{#2}}

\newcommand*{\csdisp}[2]{%
 \texorpdfstring
 {\glsdohyperlink{\glolinkprefix cs.idx.#1}{\csfmt{#2}}}%
 {\string\\#2}%
}

\newcommand*{\icsdisp}[3][]{%
 \glsadd[#1]{idx.#2}\glsdohyperlink{\glolinkprefix cs.idx.#2}{\csfmt{#3}}%
}

\newcommand*{\encapdisp}[2]{%
 \texorpdfstring
 {\glsdohyperlink{\glolinkprefix cs.idx.#1}{\code{#2}}}%
 {#2}%
}

\glsxtridentifyglslike{idx.}{\ics}

\newcommand*{\icswithargs}[2][]{\glsadd[#1]{idx.#2}\cs{#2}\glsentryuseri{#2}}

\newcommand*{\postdeschook}[1]{%
 \glsdohyperlink{\glolinkprefix cs.idx.glsxtrpostdesccategory}{\csfmt{glsxtrpostdesc#1}}}
\newcommand*{\postnamehook}[1]{%
 \glsdohyperlink{\glolinkprefix cs.idx.glsxtrpostnamecategory}{\csfmt{glsxtrpostname#1}}}
\newcommand*{\postlinkhook}[1]{%
 \glsdohyperlink{\glolinkprefix cs.idx.glsxtrpostlinkcategory}{\csfmt{glsxtrpostlink#1}}}

\newcommand*{\encap}[2][]{\glsadd[#1]{idx.#2}%
 {%
   \let\csfmt\code
   \glsdohyperlink{\glolinkprefix cs.idx.#2}{\glsentryname{idx.#2}}%
 }%
}

\newcommand{\optsection}[2][\subsection]{%
 \glsxtrifhasfield*{useri}{opt.#2}%
 {%
   #1[\texorpdfstring
       {\glsxtrheadname{opt.#2}\csoptfmt{=\glscurrentfieldvalue}}%
       {\glsentryname{opt.#2}=\glscurrentfieldvalue}]
     {\glsadd{opt.#2}\glsxtrglossentry{opt.#2}%
       \csoptfmt{=\glscurrentfieldvalue}% 
     }%
 }%
 {%
   #1[\texorpdfstring{\glsxtrheadname{opt.#2}}{\glsentryname{opt.#2}}]
     {\glsadd{opt.#2}\glsxtrglossentry{opt.#2}}%
 }%
}

\glsxtrnewgls{file.}{\exfile}

\newcommand{\filesection}[2][\section]{%
 #1[\texorpdfstring{\glsxtrheadname{file.#2}}{\glsentryname{file.#2}}]
   {\glsadd{file.#2}\glsxtrglossentry{file.#2}}%
}

\newcommand*{\csopt}[2][]{\gencsopt{#1}{opt}{#2}}%
\newcommand*{\glsopt}[2][]{\gencsopt{#1}{idx.gls}{#2}}%
\newcommand*{\glsaddopt}[2][]{\gencsopt{#1}{idx.glsadd}{#2}}%
\newcommand*{\printglossopt}[2][]{\gencsopt{#1}{idx.printgloss}{#2}}%
\newcommand*{\iprintglossopt}[2][]{\glsadd{idx.printgloss.#2}\gencsopt{#1}{idx.printgloss}{#2}}%

\newcommand*{\gencsopt}[3]{%
  \texorpdfstring%
  {%
    \gls{#2.#3}%
    \csoptfmt{\ifblank{#1}{}{\dequals\marg{#1}}}%
  }%
  {#3\ifblank{#1}{}{=#1}}%
}

\newcommand*{\field}[1]{%
 \texorpdfstring
 {\gls{field.#1}}%
 {#1}%
}

\newcommand*{\fielddisp}[2]{%
 \texorpdfstring
 {\glsdisp{field.#1}{\fieldfmt{#2}}}%
 {#2}%
}

\newcommand*{\fielddef}[1]{%
 \glsadd{field.#1}%
 \glsxtrglossentry{field.#1}{}%
}

\newcommand*{\fieldref}[1]{\gls[format=glsignore]{field.#1}}

\newglossarystyle{fieldstyle}{%
  \renewenvironment{theglossary}%
   {%
     \setlength{\glsdescwidth}{\dimexpr\linewidth-\maxnamewidth-4\tabcolsep}%
     \begin{longtable}{l>{\raggedright}p{\glsdescwidth}}%
   }%
   {\end{longtable}}%
  \renewcommand*{\glossaryheader}{%
    \noalign{\xdef\dofieldtabcaption{%
      \noexpand\caption\expandonce\fieldtablelotcaption
        {\expandonce\fieldtablecaption}}}%
    \dofieldtabcaption
    \fieldtablecaptionlabel
    \fieldtablepostcaption\tabularnewline
    \bfseries Field & \bfseries Description\tabularnewline
    \endfirsthead
    \caption*{\fieldtablecaption\ (Continued)}\tabularnewline
    \bfseries Field & \bfseries Description\tabularnewline
    \endhead
  }%
  \renewcommand*{\glsgroupheading}[1]{}%
  \renewcommand{\glossentry}[2]{%
    \glsadd{##1}%
    \glsentryitem{##1}\glstarget{##1}{\glossentryname{##1}} &
    \glossentrydesc{##1}\tabularnewline
  }%
  \renewcommand{\subglossentry}[3]{\glossentry{##2}{##3}}%
  \renewcommand*{\glsgroupskip}{}%
}

\newcommand*{\fieldhook}[1]{%
  \edef\currentcategory{\glscategory{#1}}%
  \glsxtriflabelinlist{\currentcategory}{\fieldcategories}%
  {%
     \settowidth{\dimen0}{\glsentryname{#1}}%
     \ifdim\dimen0>\maxnamewidth \setlength{\maxnamewidth}{\dimen0}\fi
  }%
  {\printunsrtglossaryskipentry}%
}

\newlength\maxnamewidth

%\printfields[postcaption]{category list}[lot caption]{caption}{label}
\newcommand*{\printfields}[2][]{%
  \ifstrempty{#1}%
  {%
    \def\fieldtablepostcaption{}%
  }%
  {%
    \def\fieldtablepostcaption{%
      \tabularnewline
      \multicolumn{2}{p{\linewidth}}{#1}%
      \tabularnewline
    }%
  }%
  \def\fieldcategories{#2}%
  \innerprintfields
}

\newcommand*{\innerprintfields}[3][]{%
 \printunsrtglossary*[style=fieldstyle]{%
   \renewcommand*\glossarysection[2][]{}%
   \ifstrempty{#1}%
   {%
     \def\fieldtablelotcaption{}%
   }%
   {%
     \def\fieldtablelotcaption{[#1]}%
   }%
   \def\fieldtablecaption{#2}%
   \def\fieldtablecaptionlabel{\label{#3}}%
   \setlength{\maxnamewidth}{0pt}%
   \let\printunsrtglossaryentryprocesshook\fieldhook
 }%
}

\renewcommand*{\glsxtrbookindexname}[1]{% 
 \glsxtrifhasfield{dualid}{#1}%
 {%
   \glshyperlink[\glossentryname{#1}]{\glscurrentfieldvalue}%
 }
 {%
   \GlsXtrIfFieldEqStr{category}{#1}{command}%
   {\glsdohyperlink{\glolinkprefix cs.#1}{\glossentryname{#1}}}%
   {\glossentryname{#1}}%
 }%
}

\newcommand{\summarysubheaderfont}[1]{\textbf{\sffamily #1}}

\newcommand{\summaryoptionhandler}[1]{%
 \par\smallskip\nopagebreak
 \par\noindent\hspace{1em}%
 \code{\gls{#1}\glsxtrifhasfield{useri}{#1}{=\glscurrentfieldvalue}{}}%
 \glsxtrifhasfield{note}{#1}%
 {\nolinebreak\hfill\mbox{\footnotesize\glscurrentfieldvalue}}%
 {}%
 \nopagebreak\par\hangindent2em\noindent\hspace{2em}\Glossentrydesc{#1}%
}

\newrobustcmd{\donote}[1]{%
 \glsxtrifhasfield{note}{#1}%
 {\nolinebreak\hfill\mbox{\normalfont\footnotesize\glscurrentfieldvalue}}%
 {}%
}

\newglossarystyle{commandsummary}%
{%
  \setglossarystyle{index}%
  \renewcommand*{\glossaryheader}{\raggedright}%
  \renewcommand*{\glsgroupheading}[1]{%
    \glsxtrgetgrouptitle{##1}{\thisgrptitle}%
    \glsxtrbookindexbookmark{\thisgrptitle}{summary.##1}%
    \glsxtrbookindexformatheader{\thisgrptitle}%
    \nopagebreak\indexspace\nopagebreak\csuse{@afterheading}%
  }%
  \renewcommand*{\glossentry}[2]{%
   \glstreeitem \glstarget{cs.##1}{\strut}%
   \glsxtrifhasfield{dualid}{##1}%
   {%
     {%
       \let\csfmtfont\csfmtcolourfont
       \glshyperlink{\glscurrentfieldvalue}%
     }%
     \glsxtrifhasfield{useri}{##1}{\glscurrentfieldvalue}{}%
     \donote{##1}%
     \nopagebreak\par\hspace{10pt}%
     \Glossentrydesc{\glscurrentfieldvalue}%
     \ifglshasdesc{\glscurrentfieldvalue}{\@. }{}%
   }%
   {%
     {\let\csfmtfont\csfmtcolourfont\gls[hyper=false]{##1}}%
     \glsxtrifhasfield{useri}{##1}{\glscurrentfieldvalue}{}%
     \donote{##1}%
     \nopagebreak\par\hspace{10pt}%
     \Glossentrydesc{##1}%
     \ifglshasdesc{##1}{\@. }{}%
   }%
   \GlsXtrIfFieldEqStr{originalentrytype}{##1}{glscommand}%
   {%
     % Does this entry have options?
     \GlsXtrIfHasNonZeroChildCount{##1}%
     {%
       \glspar\medskip\glspar 
       \summarysubheaderfont{Options:} 
       \glsxtrfieldforlistloop{##1}{childlist}{\summaryoptionhandler}%
     }%
     {}%
   }%
   {%
     \GlsXtrIfHasNonZeroChildCount{##1}%
     {This command has \glscurrentfieldvalue\ forms:}{}%
   }%
   %\listtopics{##1}%
   \par\medskip
  }%
  \renewcommand*{\subglossentry}[3]{%
   \glstreesubitem \glstarget{cs.##2}{\strut}%
   \glsxtrifhasfield{dualid}{##2}%
   {%
     {%
       \let\csfmtfont\csfmtcolourfont
       \glshyperlink{\glscurrentfieldvalue}%
     }%
     \glsxtrifhasfield{useri}{##2}{\glscurrentfieldvalue}{}%
     \donote{##2}%
     \nopagebreak\par\hspace{20pt}%
     \ifglshasdesc{\glscurrentfieldvalue}%
     {\Glossentrydesc{\glscurrentfieldvalue}.}{}%
   }%
   {%
     {\let\csfmtfont\csfmtcolourfont\gls[hyper=false]{##2}}%
     \glsxtrifhasfield{useri}{##2}{\glscurrentfieldvalue}{}%
     \donote{##2}%
     \nopagebreak\par\hspace{20pt}%
     \ifglshasdesc{##2}{\Glossentrydesc{##2}.}{}%
   }%
   \par\medskip
  }%
}

\newcommand*{\commandsummaryhook}[1]{%
  \edef\currentcategory{\glscategory{#1}}%
  \ifdefstring{\currentcategory}{command}%
  {%
    \glsxtrifhasfield*{progenitor}{#1}%
    {\printunsrtglossaryskipentry}{}%
  }%
  {\printunsrtglossaryskipentry}%
}

\newcommand*{\printcommandsummary}{%
  \printunsrtglossary*[type=index,
    style=commandsummary,
    title={General Command Summary},
  ]%
  {%
   \def\glossarypreamble{This is an alphabetical summary of 
    commands referenced in this document. See the
    relevent user guides for further details.\endgraf\medskip
    \texparserdefnotetarget}%
   \let\printunsrtglossaryentryprocesshook\commandsummaryhook
   \let\csref\cs
   \glsxtrlocalsetgrouptitle{glssymbols}{}%
  }%
}

\newcommand{\topicxrhandler}[1]{%
 \glsxtrifhasfield{parent}{#1}%
 {%
   \let\field\fieldfmt
   \hyperlink
   {topic.\glscurrentfieldvalue}%
   {\glsentryname{\glscurrentfieldvalue}}%
 }%
 {%
   \hyperlink{topic.#1}{\glsentryname{#1}}%
 }% 
}

% \iffirstprogeny{entry label}{topic label}{true}{false}
\makeatletter
\newcommand{\iffirstprogeny}[2]{%
  \let\doiffirstprogeny\@secondoftwo
  \def\firstprogeny{}%
  \glsxtrifhasfield*{unsortedprogeny}{#1}%
  {%
    \edef\suppliedprogeny{#2}%
    \@for\thisprogeny:=\glscurrentfieldvalue\do{%
      \let\firstprogeny\thisprogeny
      \ifdefequal{\thisprogeny}{\suppliedprogeny}%
      {\let\doiffirstprogeny\@firstoftwo}%
      {\let\doiffirstprogeny\@secondoftwo}%
      \@endfortrue
    }%
  }%
  {}%
  \doiffirstprogeny
}
\makeatother

\newcommand{\ifstylecommand}[1]{%
  \letcs\doifstylecommand{@firstoftwo}%
  \GlsXtrIfFieldEqStr{originalentrytype}{#1}{abbrvstylecommand}%
  {}%
  {%
    \GlsXtrIfFieldEqStr{originalentrytype}{#1}{glostylecommand}%
    {}%
    {\global\letcs\doifstylecommand{@secondoftwo}}%
  }%
  \doifstylecommand
}

\newcommand{\listtopics}[1]{%
 \glsxtrifhasfield{progeny}{#1}% 
 {%
   \ifdefequal\glscurrentfieldvalue\glscurrententrylabel
   {}%
   {%
     \par\nopagebreak\medskip\par\footnotesize
     \sloppy
     \ifstylecommand{#1}%
     {\summarysubheaderfont{Styles:}}%
     {\summarysubheaderfont{Topics:}}%
     \space
     \let\DTLlistformatitem\topicxrhandler
     \DTLformatlist{\glscurrentfieldvalue}.\par
   }%
 }%
 {}%
}

\newglossarystyle{commandtopic}%
{%
  \setglossarystyle{index}%
  %\renewcommand*{\glossaryheader}{\raggedright}%
  \renewcommand*{\glsgroupheading}[1]{}%
  \renewcommand*{\glossentry}[2]{%
   \glsfieldfetch{##1}{nametitle}{\thisvalue}%
   \section*{%
     \raisebox{2ex}%
     {\pdfbookmark[1]{\thisvalue}{topic.##1}\hypertarget{topic.##1}{}}%
     \glsdisp{##1}{\thisvalue}\donote{##1}%
   }%
   \expandafter\sectionmark\expandafter{\thisvalue}%
   \ifglshasdesc{##1}{\Glossentrydesc{##1}.}{}%
   \glsxtrifhasfield{seealso}{##1}%
   {%
    \par\smallskip
     \emph{See also:} 
     \let\DTLlistformatitem\topicxrhandler
     \DTLformatlist{\glscurrentfieldvalue}.%
     \par\medskip\par
   }%
   {}%
  }%
  \renewcommand*{\subglossentry}[3]{%
   \glsifcategory{##2}{command}%
   {%
     \glsdohypertarget{topic.##2}{\strut}%
     \glsxtrifhasfield*{progenitor}{##2}%
     {\let\originallabel\glscurrentfieldvalue}%
     {\def\originallabel{##2}}%
%
     \ifstylecommand{##2}%
     {%
        \iffirstprogeny{\originallabel}{##2}%
        {\letcs\dothisentry{@firstofone}}%
        {%
          \glspar
          \ifdefempty\firstprogeny
          {%
            {\let\csfmtfont\csfmtcolourfont\gls[noindex]{\originallabel}}%
            \glsxtrifhasfield{useri}{##2}{\glscurrentfieldvalue}{}%
          }%
          {%
            {\let\csfmtfont\csfmtcolourfont
              \hyperlink{topic.\firstprogeny}{\glsentryname{\firstprogeny}}}%
            \glsxtrifhasfield{useri}{##2}{\glscurrentfieldvalue}{}%
            \glsxtrifhasfield{parent}{\firstprogeny}%
            {\hfill
             {\footnotesize\emph{see} \hyperlink{topic.\glscurrentfieldvalue}%
             {\glsentryname{\glscurrentfieldvalue}}.}}
            {}%
          }%
          \letcs\dothisentry{@gobble}%
        }%
     }%
     {\letcs\dothisentry{@firstofone}}%
     \dothisentry
     {%
       \par\smallskip\par 
       % hyperlink to index:
       {\let\csfmtfont\csfmtcolourfont\gls[format=glsignore]{\originallabel}}%
       \glsxtrifhasfield{useri}{##2}{\glscurrentfieldvalue}{}%
       \glsxtrifhasfield{note}{##2}%
       {\nolinebreak\hfill\mbox{\footnotesize\glscurrentfieldvalue}}%
       {}%
       \nopagebreak\par\smallskip\par\nopagebreak
       \ifglshasdesc{##2}{\Glossentrydesc{##2}.}{}%
       % Topics
       \listtopics\originallabel
       \par\smallskip
     }%
   }%
   {%
     \glsfieldfetch{##2}{nametitle}{\thisvalue}%
     \subsection*{%
       \raisebox{2ex}%
       {\hypertarget{topic.##2}{}\pdfbookmark[2]{\thisvalue}{topic.##2}}%
       \glsdisp{##2}{\thisvalue}\donote{##2}%
     }%
     \expandafter\subsectionmark\expandafter{\thisvalue}%
     \ifglshasdesc{##2}{\Glossentrydesc{##2}.\glspar\nopagebreak\smallskip}{}%
   }%
  }%
}

\newcommand*{\commandtopichook}[1]{%
  \glsxtrifhasfield*{progenitor}{#1}%
  {}%
  {%
    \GlsXtrIfFieldEqStr*{originalentrytype}{#1}{topic}%
    {}%
    {%
      \glsxtrifhasfield*{parent}{#1}%
      {%
        \GlsXtrIfFieldEqStr*{originalentrytype}{\glscurrentfieldvalue}{topic}%
        {}%
        {\printunsrtglossaryskipentry}%
      }%
      {\printunsrtglossaryskipentry}%
    }%
  }%
}

\newcommand*{\printcommandtopic}{%
  \setsecdepth*{0}%
  \automark[subsection]{section}%
  \printunsrtglossary*[type=index,
    style=commandtopic,
    title={Summary by Topic of Glossary Commands},
    label={sec:topics},
    nogroupskip,
    nonumberlist
  ]%
  {%
   \def\texparserdefprefix{topic.}%
   \def\glossarypreamble{This is a summary of commands provided
    by \styfmt{glossaries}, \styfmt{glossaries-extra} and
    their supplementary packages that may be useful to 
    \bibgls\ users. Commands specific to other indexing methods
    aren't listed unless they've been referenced in this manual.
    See the appropriate user guides for further details.
    \endgraf\medskip
    \texparserdefnotetarget}%
   \let\abbrstyle\topicabbrstyle
   \let\glostyle\topicglostyle
   \let\printunsrtglossaryentryprocesshook\commandtopichook
   \let\csref\cs
  }%
  \setsecdepth{0}%
}

\newglossarystyle{styoptsummary}%
{%
  \setglossarystyle{mcolindex}%
  \renewcommand{\glossarypreamble}{%
    Most options are in the form \meta{option}=\meta{value} and
    may have a default if \meta{value} is omitted, 
    but some options don't have values and should not have
    one assigned.  For boolean options, 
    if the value is omitted \styoptfmt{true} is assumed. 
    \extstyopt Indicates a value that's only provided by
    \styfmt{glossaries-extra} and not by the base
    \styfmt{glossaries} package. \par
  }%
  \renewcommand*{\glossaryheader}{}%
  \renewcommand*{\glsgroupheading}[1]{%
    \glsxtrgetgrouptitle{##1}{\thisgrptitle}%
    \glsxtrbookindexbookmark{\thisgrptitle}{styopt.##1}%
    \glsxtrbookindexformatheader{\thisgrptitle}%
    \nopagebreak\indexspace\nopagebreak\csuse{@afterheading}%
  }%
  \renewcommand*{\glossentry}[2]{%
   \item \glstarget{##1}{\strut}%
   \glsxtrifhasfield{dualid}{##1}%
   {%
     \gls{\glscurrentfieldvalue}%
   }%
   {%
     \gls[hyper=false]{##1}%
   }%
   \glsxtrifhasfield{useri}{##1}{=\glscurrentfieldvalue}{}%
   \nopagebreak\par\hspace{10pt}%
   \Glossentrydesc{##1}.%
   \glsxtrifhasfield{package}{##1}%
   {\nopagebreak\par \ding{43}Provided by 
     \ifdefstring{\glscurrentfieldvalue}{glossaries,glossaries-extra}
     {\styfmt{glossaries} and modified by \styfmt{glossaries-extra}}%
     {\styfmt{\glscurrentfieldvalue}}.}%
   {}%
   \glsxtrifhasfield{note}{##1}%
   {\space \xmakefirstuc{\glscurrentfieldvalue}.}%
   {}%
   \par\medskip
  }%
}

\newcommand*{\styoptsummaryhook}[1]{%
  \edef\currentcategory{\glscategory{#1}}%
  \ifdefstring{\currentcategory}{packageoption}%
  {}%
  {\printunsrtglossaryskipentry}%
}

\newcommand*{\printstyoptsummary}{%
  \printunsrtglossary*[type=main,
    style=styoptsummary,
    title={Package Option Summary}
  ]%
  {%
   \let\printunsrtglossaryentryprocesshook\styoptsummaryhook
   \glsxtrlocalsetgrouptitle{glssymbols}{@}%
  }%
}

\newcommand{\entrysection}[2][\subsection]{%
 #1[\texorpdfstring{\glsxtrheadname{entry.#2}}{\glsentryname{entry.#2}}]
   {\glsadd{entry.#2}\glsxtrglossentry{entry.#2}}%
}

\newcommand{\entrydef}[1]{\glsadd{entry.#1}\glsxtrglossentry{entry.#1}}

\newcommand*{\atentry}[2][]{%
 \texorpdfstring
 {\gls[#1]{entry.#2}}%
 {#2}%
}

\newcommand*{\atentrypageref}[1]{%
 \atentry{#1} (page~\glsxtrpageref{entry.#1})}%

\newcommand*{\atentryref}[1]{\gls[format=glsignore]{entry.#1}}

\newcommand*{\entryref}[1]{\glslink[format=glsignore]{entry.#1}{\entryfmt{#1}}}

\newcommand{\commonarg}[2][\subsection]{\argsection[#1]{common.#2}}

\newcommand{\converterarg}[2][\subsection]{\argsection[#1]{converter.#2}}

\newcommand{\convertglsbibarg}[2][\subsubsection]{\argsection[#1]{gls2bib.#2}}

\newcommand{\datatoolbibarg}[2][\subsubsection]{\argsection[#1]{datatool2bib.#2}}

\newcommand{\argsection}[2][\subsection]{%
 \def\switcharg{}%
 \def\switchalt{}%
 \def\switchaltpdf{}%
 \glsxtrifhasfield*{useri}{switch.#2}%
 {%
   \edef\switcharg{ \expandonce\glscurrentfieldvalue}%
 }%
 {}%
 \ifglshassymbol{switch.#2}%
 {%
   \def\switchalt{ (or \protect\glsentrysymbol{switch.#2}\switcharg)}%
 }%
 {}%
 #1[\texorpdfstring
     {\glsxtrheadname{switch.#2}}%
     {\glsentryname{switch.#2}}%
     \switcharg\switchalt
   ]
   {\glsadd{switch.#2}%
    \glsxtrglossentry{switch.#2}\switcharg\switchalt}%
}

\newrobustcmd{\longswitch}{\string-{}\string-}

\newcommand*{\shortargfmt}[1]{%
 \texorpdfstring{\texttt{\string-#1}}%
 {\string-#1}%
}

\newcommand*{\longarg}[1]{%
  \texorpdfstring
  {\gls{switch.#1}}%
  {\string-\string-#1}%
}

\newcommand*{\commonlongarg}[1]{%
  \texorpdfstring
  {\gls{switch.common.#1}}%
  {\string-\string-#1}%
}

\newcommand*{\converterlongarg}[1]{%
  \texorpdfstring
  {\gls{switch.converter.#1}}%
  {\string-\string-#1}%
}

\newcommand*{\convertglsbiblongarg}[1]{%
  \texorpdfstring
  {\gls{switch.gls2bib.#1}}%
  {\string-\string-#1}%
}

\newcommand*{\datatoolbiblongarg}[1]{%
  \texorpdfstring
  {\gls{switch.datatool2bib.#1}}%
  {\string-\string-#1}%
}

\newcommand*{\longargpageref}[1]{%
 \longarg{#1} (page~\glsxtrpageref{switch.#1})}%

\newcommand*{\longargorshort}[1]{%
  \texorpdfstring
  {\gls{switch.#1}\ifglshassymbol{switch.#1}{ or \glssymbol{switch.#1}}{}}%
  {\string-\string-#1}%
}

\newcommand*{\unicodecategoryfmt}[1]{\qt{\textsf{#1}}}

\definecolor{defbackground}{rgb}{1,1,0.75}
\newsavebox\defsbox
\newlength\defwidth

\newenvironment{definition}%
{%
  \setlength{\fboxsep}{4pt}\setlength{\fboxrule}{1.25pt}%
  \begin{lrbox}{\defsbox}%
   \setlength\defwidth\linewidth
   \addtolength\defwidth{-2\fboxrule}%
   \addtolength\defwidth{-2\fboxsep}%
   \begin{minipage}{\defwidth}
   \flushleft\ttfamily\ignorespaces
}%
{%
   \end{minipage}%
  \end{lrbox}\par\medskip\noindent
  \fcolorbox{black}{defbackground}{\usebox\defsbox}%
  \medskip\par\noindent
  \ignorespacesafterend
}

\newcommand{\introguide}{\href{bib2gls-begin.pdf}{\qt{\styfmt{glossaries-extra}
and \bibgls: An Introductory Guide} (\filefmt{bib2gls-begin.pdf})}}

\setcounter{tocdepth}{4}

\newcommand*{\sectionref}[1]{\cref{#1}}
\newcommand*{\Sectionref}[1]{\Cref{#1}}
\newcommand*{\figureref}[1]{\hyperlink{#1-top}{figure~\ref*{#1}}}
\newcommand*{\Figureref}[1]{\hyperlink{#1-top}{Figure~\ref*{#1}}}
\newcommand*{\tableref}[1]{\cref{#1}}
\newcommand*{\Tableref}[1]{\Cref{#1}}

\newcommand{\figcontents}[3]{%
  \hypertarget{#3-top}{}%
  \centering
  #1\par
  #2\label{#3}%
}

\newlength\imagewidth
\newcount\tmpctr

\newcommand*{\pdftwocol}[3][1]{%
  \renewcommand{\tabcolsep}{2pt}%
  \setlength{\imagewidth}{\dimexpr.5\linewidth-2\tabcolsep}%
  \def\tabcontents{\begin{tabular}{@{}cc@{}}}%
  \tmpctr\numexpr#1-1\relax
  \loop
  \advance\tmpctr by 1\relax
  \eappto\tabcontents{\noexpand\frame{%
   \noexpand\includegraphics[page=\the\tmpctr,width=\noexpand\imagewidth]{#2}}}%
  \ifodd\tmpctr
    \appto\tabcontents{&}%
  \else
    \appto\tabcontents{\\}%
  \fi
  \ifnum\tmpctr<#3
  \repeat
  \appto\tabcontents{\end{tabular}}%
  \tabcontents
}

\makeatletter
\newcommand*{\tablelistcs}[2]{%
 \tmpctr=0\relax
 \def\tabcontents{}%
 \loop
   \advance\tmpctr by 1\relax
   \appto\tabcontents{l}%
 \ifnum\tmpctr<#1
 \repeat
%
 \edef\tabcontents{\noexpand\begin{tabular}{@{}\tabcontents @{}}}%
 \def\thislist{#2}%
 \dtlsortlist{\thislist}{\dtlicompare}%
 \tmpctr=1\relax
 \@for\thislabel:=\thislist\do{%
  \eappto\tabcontents{\noexpand\ics{\thislabel}}%
  \ifnum\tmpctr<#1\relax
   \appto\tabcontents{&}%
   \advance\tmpctr by 1\relax
  \else
   \appto\tabcontents{\\}%
   \tmpctr=1\relax
  \fi
 }%
 \appto\tabcontents{\end{tabular}}%
 \tabcontents
}
\makeatother

% This is a bit fiddly. Need to represent \vec{v} in typewriter
% font for the interpreter examples. ($\mathtt{\vec{v}}$ doesn't
% work)
\newsavebox\varrow
\sbox\varrow{$\mathtt{\accentset{\to}{v}}$}

\newcommand*{\mtx}[1]{\boldsymbol{#1}}
\newcommand*{\set}[1]{\mathcal{#1}}
\newcommand*{\card}[1]{|\set{#1}|}
\newcommand*{\imaginary}{i}

\newcommand{\addr}[1]{\\\href{https://www.#1/}{\nolinkurl{#1}}}
\title{\bibgls: a command line Java application to convert
\filefmt{.bib} files to \filefmt{glossaries-extra.sty} resource
files}
\author{Nicola Talbot\addr{dickimaw-books.com}}
\input{version}

\makeatletter
\renewcommand{\fps@figure}{htbp}

\newcommand{\code}[1]{\texorpdfstring{{\ttfamily\obeyspaces #1}}{#1}}

\newcommand{\setupcodeenvfmts}{%
   \def\cmd{\char`\\}%
   \def\comment##1{\mbox{\textcolor{comment}{\idx{commentchar}\ ##1}}}%
   \renewcommand*{\styfmt}[1]{##1}%
   \renewcommand*{\counterfmt}[1]{##1}%
   \renewcommand*{\catattrfmt}[1]{\textcolor{attribute}{##1}}%
   \renewcommand*{\abbrstylefmt}[1]{\textcolor{style}{##1}}%
   \renewcommand*{\csfmtfont}[1]{\textcolor{cs}{##1}}%
}

\newenvironment{codeenv}
 {%
   \GlsXtrStartUnsetBuffering
   \renewcommand{\glslinkpresetkeys}{\setkeys{glslink}{noindex}}%
   \setupcodeenvfmts
   \begin{flushleft}\ttfamily\obeylines\frenchspacing\@vobeyspaces
   \parindent\z@\parfillskip\@flushglue\parskip\z@skip
 }
 {\end{flushleft}\GlsXtrDiscardUnsetBuffering\ignorespacesafterend}

\newenvironment{codeenv*}
 {%
   \GlsXtrStartUnsetBuffering
   \setupcodeenvfmts
   \begin{flushleft}\ttfamily\obeylines\frenchspacing\@vobeyspaces
   \parindent\z@\parfillskip\@flushglue\parskip\z@skip
 }
 {\end{flushleft}\GlsXtrDiscardUnsetBuffering\ignorespacesafterend}

\begingroup
 \renewcommand{\addr}[1]{}
 \let\texorpdfstring\@secondoftwo
 \DTMsetstyle{pdf}
 \protected@edef\x{\endgroup
   \noexpand\hypersetup{%
     pdfinfo={
       Title={\@title},
       Author={\@author},
       CreationDate={\DTMuse{creation}},
       ModDate={\DTMuse{moddate}},
     }%
   }%
 }\x

\makeatother

\renewcommand{\titlepagestyle}{empty}

\begin{document}
\maketitle
\pagenumbering{alph}
\pagestyle{empty}

\begin{abstract}
The \bibgls\ command line application can be used to extract
glossary information stored in a \ext{bib} file and convert it
into glossary entry definitions that can be read using
\styfmt{glossaries-extra}'s \gls{GlsXtrLoadResources} command. When used
in combination with the \styoptfmt{record} package option, \bibgls\
can select only those entries that have been used in the document,
as well as any dependent entries, which reduces the \TeX\ resources
required by not defining unnecessary commands.

Since \bibgls\ can also sort and collate the recorded \glspl{location}
present in the \ext{aux} file, it can simultaneously by-pass the
need to use \idx!{makeindex} or \idx!{xindy}, although \bibgls\ 
can be used together with an external indexing application if required. (For
example, if a custom \idx{xindy} rule is needed.)

An additional build may be required to ensure the \glspl{location} are
up-to-date as the page-breaking may be slightly different on the
first \LaTeX\ run due to the unknown references being replaced with
?? which can be significantly shorter than the actual text produced
when the reference is known. 

Note that \bibgls\ is a Java application, and requires at least
Java~8. Additionally, \sty{glossaries-extra} must be at least
version 1.12.  These are minimum requirements, but the latest
versions are recommended. This
application was developed in response to the question
\qt{\href{http://tex.stackexchange.com/q/342544}{Is there a program
for managing glossary tags?}} on \TeX\ on
StackExchange~\cite{tex.sx.2016}. The \ext{bib} file can be managed in an
application such as JabRef.

If you already have a \ext{tex} file containing 
entry definitions using commands like \gls{newglossaryentry}
then you can use the supplementary tool \idx{convertgls2bib}
to convert the entries to the \ext{bib} format required by
\bibgls. See \sectionref{sec:gls2bib} for further details.

The supplementary file \introguide\ is an
introductory guide to the \isty{glossaries-extra} package, which you
may prefer to start with if you are unfamiliar with the
\sty{glossaries} and \sty{glossaries-extra} packages.
\end{abstract}

Additional resources:
\begin{itemize}
\item 
\href{https://www.dickimaw-books.com/gallery/#bib2gls}{\bibgls\ gallery}.
\item
\href{https://www.dickimaw-books.com/faq.php?category=bib2gls}{\bibgls\ FAQ}
\end{itemize}
TUGboat articles:
\begin{itemize}
\item
\href{http://tug.org/TUGboat/tb40-1/tb124talbot-bib2gls.pdf}{Glossaries
with \bibgls}, issue 40:1, 2019.
\item
\href{http://tug.org/TUGboat/tb41-3/tb129talbot-bib2gls-more.pdf}{\bibgls:
selection, cross-references and locations}, issue 41:3, 2020.
\item \bibgls: sorting, issue \href{http://tug.org/TUGboat/Contents/contents42-2.html}{42:2, 2021}.
\end{itemize}

\frontmatter

\tableofcontents
\listoftables
\listoffigures
\printunsrtglossary*[type=terms,style=altlist,groups=false]
{%
  \renewcommand{\glslistitem}[1]{%
    \item[\glsentryitem{#1}%
          \glstarget{#1}{\glossentryname{#1}}]%
    \glsxtrifhasfield{useri}{#1}%
     {\glsadd[format=glsnumberformat]{\glscurrentfieldvalue}}{}%
  }
}

\mainmatter
\chapter{Introduction}

If you have extensively used the \styfmt{glossaries}~\cite{glossaries} or
\styfmt{glossaries-extra}~\cite{glossaries-extra} package, 
you may have found yourself creating a large \iext{tex} file
containing many definitions that you frequently use in documents.
This file can then simply be loaded using \ics{input} or
\gls{loadglsentries}, but a large file like this can be difficult to
maintain and if the document only actually uses a small proportion
of those entries, the document build is unnecessarily slow due to
the time and resources taken on defining the unwanted entries.

The aim of \bibgls\ is to allow the entries to be stored in a
\iext{bib} file, which can be maintained using a reference system
such as JabRef. The document build process
can now be analogous to that used with \idx{bibtex} (or
\appfmt{biber}), where only those entries that have been recorded in the
document (and possibly their dependent entries) will be extracted
from the \ext{bib} file. Since \bibgls\ can also perform
\hierarchical\ sorting and can collate \glspl{locationlist}, it doubles as
an indexing application, which means that the \idx{makeglossaries}
step can be skipped.
Note that \bibgls\ doesn't warn you if an entry that's referenced
in the document doesn't exist in any of the supplied \ext{bib}
files, but instead relies on the \styfmt{glossaries-extra} package
to generate the warning. So at the end of the document build check
the \ext{log} file for warnings.

You can't use \ics{glsaddall} with \bibgls\ as that command works
by iterating over all defined entries and calling
\code{\ics{glsadd}\margm{label}}. On the first \LaTeX\ run there are no
entries defined, so \ics{glsaddall} does nothing. If you want to
select all entries, just use \csopt[all]{selection} instead (which
has the advantage over \ics{glsaddall} in that it doesn't create a
redundant \gls{location} for each entry).

Note that \bibgls\ requires the extension package
\isty{glossaries-extra} and can't be used with just the base
\isty{glossaries} package, since it requires some of the extension
commands. See the \styfmt{glossaries-extra} user
manual~\cite{glossaries-extra} for information
on the differences between the basic package and the extended
package, as some of the default settings are different.

Since information required by \bibgls\ is
written to the \iext{aux} file, it's not possible to run \bibgls\
through \TeX's shell escape while the \ext{aux} file is open for
write access. (The \ext{aux} file is closed \emph{after} the end
document hook, so it can't be deferred with \ics{AtEndDocument}.) 
This means that if you really want to run \bibgls\
through \ics{write18} it must be done in the preamble with
\ics{immediate}. For example:
\begin{verbatim}
\immediate\write18{bib2gls \jobname}
\end{verbatim}
As from version 1.14 of \styfmt{glossaries-extra}, this can be done
automatically with the \styopt{automake} option if the \ext{aux}
file exists. (Remember that this will require the shell escape to be
enabled.)

\section{Default Encoding}
\label{sec:defencoding}

Both \XeLaTeX\ and \LuaLaTeX\ default to UTF-8 \igls{encoding}. With modern \TeX\
distributions, \pdfLaTeX\ also defaults to UTF-8 but may be changed
with the \sty{inputenc} package.
 
The default encoding for Java applications, such as \bibgls, is the
default encoding of the \idx{JVM}. This typically matches the
operating system's default, but can be changed (see below).  If you don't want
to alter the \idx{JVM}['s] default, you can set the \bibgls\ default
with \longarg{default-encoding}.

In general, UTF-8 works best with \bibgls, but you need to be careful if your
\idx{JVM} isn't set up to use UTF-8 by default as you can end
up with encoding mismatches. This can happen with some versions of
Windows, so it's a good idea to double-check the \bibgls\ transcript
file to make sure all the encoding information is correct.

The default encoding is written at the start of the \ext{glg} file.
For example:
\begin{verbatim}
Default encoding: UTF-8
\end{verbatim}
When a file is opened, the associated encoding is written to the \ext{glg}
file. For example, when the \ext{aux} file is opened:
\begin{verbatim}
Reading myDoc.aux
Encoding: UTF-8
\end{verbatim}
If the document encoding is detected in the \ext{aux} file, it will
be written to the transcript. For example:
\begin{verbatim}
TeX character encoding: UTF-8
\end{verbatim}
When a \ext{bib} file is read, the \csopt{charset} setting, 
the detected encoding (from the encoding comment line,
see \sectionref{sec:bibencoding}), if found, and the encoding
actually used are written. For example, where \csopt{charset} hasn't
been set but an encoding comment line has been found:
\begin{verbatim}
Parsing bib files for resource myDoc.glstex.
Default encoding: not set
Detected encoding: UTF-8
Reading symbols.bib
Encoding: UTF-8
\end{verbatim}

The file encodings used by \bibgls\ are as follows:
\begin{itemize}
\item Writing the \iext{glg} transcript file: default encoding.

\item Reading the document \iext{log} file: \longarg{log-encoding}
setting, if supplied, otherwise the default encoding. 

Note that the \ext{log} file may not have the same encoding as the
\ext{tex} file~\cite{tex.sx.2013}.  In the case of the T1 font
encoding, the encoding will be close enough to ISO-8859-1 for that
to be used with \bibgls.  Any problematic character will trigger a
warning and \bibgls\ will quit reading the file. This will most
likely be in an overfull warning, by which point \bibgls\ should
have gathered all the information it requires.

\item Reading the \iext{aux} file: the \longarg{tex-encoding}
setting, if supplied, or UTF-8 if \sty{fontspec} is detected in the
\ext{log} file, otherwise the default encoding.

\item Reading the \iext{bib} files: the \csopt{charset} resource
option, if supplied, or the encoding specified by the encoding
comment line in the \ext{bib} file (see
\sectionref{sec:bibencoding}), otherwise the default encoding.

\item Writing the \iext{glstex} files: the \longarg{tex-encoding}
setting, if supplied, or UTF-8 if \sty{fontspec} detected in the
\ext{log} file, or the document encoding picked up from the
\ext{aux} file, otherwise the default encoding.
\end{itemize}
For example:
\begin{verbatim}
bib2gls --log-encoding ISO-8859-1 --default-encoding UTF-8 myDoc
\end{verbatim}

To change the default \igls{encoding} for the \idx{JVM} set the 
\code{JAVA\_TOOL\_OPTIONS} environment variable to include
\code{-Dfile.encoding=\meta{encoding}} where \meta{encoding} is the
desired default encoding (such as UTF-8). Note that this will
affect all your installed Java applications, not just \bibgls, (for
example, JabRef).

If you have a problem with non-ASCII characters not displaying
correctly in your document:
\begin{itemize}
\item Check that the file \gls{encoding} of your document \ext{tex}
and \ext{bib} files have been correctly set by your text editor.
\item Check that your document supports that \gls{encoding} (for example,
through the \isty{inputenc} package).
\item Check \bibgls's transcript file for the encoding information
to ensure that the settings are correct.
\end{itemize}

\section{Example Use}
The glossary entries are stored in a \ext{bib} file. For
example, the file \filefmt{entries.bib} might contain:
\begin{codeenv}
\atentry{entry}\marg{bird,
  \field{name}=\marg{bird},
  \field{description} = \marg{feathered animal}
}
\strut
\atentry{abbreviation}\marg{html,
  \field{short}=\qtdelim{html},
  \field{long}=\marg{hypertext markup language}
}
\strut
\atentry{symbol}\marg{v,
  \field{name}=\marg{\idx!{mshiftchar}\cs{vec}\marg{v}\idx!{mshiftchar}},
  \field{text}=\marg{\cs{vec}\marg{v}},
  \field{description}=\marg{a vector}
}
\strut
\atentry{index}\marg{goose,\field{plural}=\qtdelim{geese}}
\end{codeenv}
Here's an example document that uses this data:
\begin{codeenv}
\cmd{documentclass}\marg{article}
\strut
\cs{usepackage}\oarg{\styoptref{record}}\marg{glossaries-extra}
\strut
\gls{GlsXtrLoadResources}\oarg{
  \csopt[entries]{src},\comment{data in entries.bib}
  \csopt[en-GB]{sort}\comment{sort according to 'en-GB' locale}
}
\strut
\cmd{begin}\marg{document}
\cs{Gls}\marg{bird} and \cs{gls}\marg{goose}.
Symbol: \idx{mshiftchar}\cs{gls}\marg{v}\idx{mshiftchar}.
Abbreviation: \cs{gls}\marg{html}.
\strut
\cs{printunsrtglossaries}
\cmd{end}\marg{document}
\end{codeenv}
If this document is called \filefmt{myDoc.tex}, the build 
process is:
\begin{verbatim}
pdflatex myDoc
bib2gls myDoc
pdflatex myDoc
\end{verbatim}
(This manual assumes \appfmt{pdflatex} for simplicity. Replace
with \appfmt{latex}, \appfmt{xelatex} or \appfmt{lualatex} as
appropriate.) If you want \idxpl{lettergroup} (either headed, with
styles like \glostyle{indexgroup}, or just a blank line separator
with \styopt[false]{nogroupskip}) then you need to use the
\longarg{group} switch:
\begin{verbatim}
pdflatex myDoc
bib2gls --group myDoc
pdflatex myDoc
\end{verbatim}

You can have multiple instances of \gls{GlsXtrLoadResources}. For
example:
\begin{codeenv}
\cmd{documentclass}\marg{article}
\strut
\cmd{usepackage}\oarg{\styopt{record},\styopt{index},\styopt{abbreviations},\styopt{symbols}}\marg{glossaries-extra}
\strut
\gls{GlsXtrLoadResources}\oarg{
  \csopt[entries]{src},\comment{data in entries.bib}
  \csopt[en-GB]{sort},\comment{sort according to 'en-GB' locale}
  \csopt[entrytype=\marg{entry}]{match},\comment{only select \atentry{entry}}
  \csopt[main]{type}\comment{put these entries in the 'main' glossary}
}
\strut
\gls{GlsXtrLoadResources}\oarg{
  \csopt[entries]{src},\comment{data in entries.bib}
  \csopt[en-GB]{sort},\comment{sort according to 'en-GB' locale}
  \csopt[entrytype=\marg{abbreviation}]{match},\comment{only select \atentry{abbreviation}}
  \csopt[abbreviations]{type}\comment{put these in the 'abbreviations' glossary}
}
\strut
\gls{GlsXtrLoadResources}\oarg{
  \csopt[entries]{src},\comment{data in entries.bib}
  \csopt[letter-case]{sort},\comment{case-sensitive letter sort}
  \csopt[entrytype=\marg{symbol}]{match},\comment{only select \atentry{symbol}}
  \csopt[symbols]{type}\comment{put these entries in the 'symbols' glossary}
}
\strut
\gls{GlsXtrLoadResources}\oarg{
  \csopt[entries]{src},\comment{data in entries.bib}
  \csopt[en-GB]{sort},\comment{sort according to 'en-GB' locale}
  \csopt[entrytype=\marg{index}]{match},\comment{only select \atentry{index}}
  \csopt[index]{type}\comment{put these entries in the 'index' glossary}
}
\strut
\cmd{begin}\marg{document}
\cs{Gls}\marg{bird} and \cs{gls}\marg{goose}.
Symbol: \idx!{mshiftchar}\cs{gls}\marg{v}\idx!{mshiftchar}. Abbreviation: \cs{gls}\marg{html}.
\cs{printunsrtglossaries}
\cmd{end}\marg{document}
\end{codeenv}
There are more examples provided in \sectionref{sec:examples} and
also in the \href{https://www.dickimaw-books.com/gallery/#bib2gls}{\bibgls\ gallery}.

Note that there's no need to called \idx!{xindy} or \idx!{makeindex}
since \bibgls\ automatically sorts the entries and collates the \glspl{location}
after selecting the required entries from the \ext{bib} file and
before writing the temporary file that's input with \gls{glsxtrresourcefile}
(or the more convenient shortcut
\gls{GlsXtrLoadResources}).\footnote{This document will mostly use
the more convenient \gls{GlsXtrLoadResources}.}  This
means the entries are already defined in the correct order, and only
those entries that are required in the document are defined, so
\ics{printunsrtglossary} (or \ics{printunsrtglossaries}) may be used.
(The \qtt{unsrt} part of the command name indicates that all
defined entries should be listed in the order of definition from
\styfmt{glossaries-extra}'s point of view, see the supplementary
document \introguide\ for further details.)

If you don't provide a value with the \styopt{record} option, then
\styopt[only]{record} is assumed. This saves the same indexing
information that's used with the \ics{cs.makeglossaries} and
\ics{makenoidxglossaries} methods (described in the main
\styfmt{glossaries} user manual~\cite{glossaries}). As from
\styfmt{glossaries-extra} version 1.37, you can instead use
\styopt[nameref]{record}, which saves some extra information for
each \gls{location} that's not available for the other indexing methods. 
See \longarg{merge-nameref-on} for further details.

If you additionally want to use an indexing application, such
as \idx{xindy}, you need the package option
\styopt[alsoindex]{record} and use \ics{cs.makeglossaries}
and \ics{printglossary} (or the iterative \ics{printglossaries}) as usual.
This requires a more complicated build process:
\begin{verbatim}
pdflatex myDoc
bib2gls myDoc
pdflatex myDoc
makeglossaries myDoc
pdflatex myDoc
\end{verbatim}
(The entries aren't defined until the second \LaTeX\ run, so the
indexing files required by \idx!{xindy} or \idx!{makeindex} can't be
created until then.) In this case, \bibgls\ is simply being used to
fetch the entry definitions from one or more \ext{bib} files, with
the sorting and collating performed by the other indexing
application (so the resource option list would need
\csopt[none]{sort} and \csopt[false]{save-locations}).  In general, 
it's best to avoid this hybrid method unless you have a particular
set of \idx{xindy} rules that can't be replicated with \bibgls.

\section{Logical Divisions: \field{type} vs \field{group} vs \field{parent}}
\label{sec:logicaldivisions}

If you have a document with many terms that need listing, it's
likely that you may want to divide the terms into separate blocks or
units for easier reading. There are three fields that are used
for this.
\begin{description}
\item[\field{type}] The highest division is the glossary to which
the entry belongs. The glossary must first be defined (see
\sectionref{sec:newglossary}) with an associated label used to
identify it. The title is assigned to the glossary when it is
defined or it can be overridden with the \printglossopt{title} key.
The glossary is displayed using \cs{printunsrtglossary} and the
title is placed in a sectioning command by default.

\begin{important}
\bibgls\ does not provide any means of sorting glossary types. If
you use \ics{printunsrtglossaries} the order will be according to
the order in which the glossaries were defined. You may use
\cs{printunsrtglossary} to list individual glossaries in your own
preferred order.
\end{important}

\item[\field{group}] The entries within a glossary can form \idxpl{group}
as a by-product of the sorting method. This must be enabled with the
\longarg{group} switch and isn't available for the sort methods
listed in \tableref{tab:sortoptionsnosort}. The group label is
stored in the \field{group} field. This is an internal field that
typically shouldn't be set in the \ext{bib} file.

You can specify your own \idxpl{customgroup} but if you do so you must
ensure that the terms are ordered in such a way that they are
gathered according to group. This is typically done by splitting the
glossary into blocks using a separate \gls{GlsXtrLoadResources} with
the \csopt{group} option set. You control the order of the groups by
your ordering of \gls{GlsXtrLoadResources}. The group title can be
assigned using \gls{glsxtrsetgrouptitle} within the document.

\begin{important}
\bibgls\ does not sort by group title. At most it can sort by the
group label (by changing the \csopt{sort-field}) but this is usually
an indication that you actually have a \gls{hierarchicalglossary} and you
ought to be using the \field{parent} field instead. (Compare
\exfile{sample-textsymbols.tex} and
\exfile{sample-textsymbols2.tex}.)
\end{important}

\item[\field{parent}] An entry may have one or more \glspl{sub-entry}.
Most of the sort methods will produce a \hierarchical\ ordering that
ensures that the \glspl{sub-entry} are listed immediately after their
\gls{parententry}. The \gls{parententry} is identified by the \field{parent}
field which should contain the \glsdisp{parententry}{parent's} label.

\begin{important}
\bibgls\ sorts the \parent\ and \child\ entries using the same
comparator. The sort methods listed in
\tableref{tab:sortoptionsnosort} disregard the \hierarchicallevel,
which can result in \glspl{childentry} becoming detached from their
\gls{parententry}. The other methods sort \hierarchically\ using the
same comparator but take the \hierarchicallevel\ into account.
\end{important}

\end{description}

Suppose you have a mixture of terms, abbreviations and symbols, then
you might want to have three glossaries that are listed in the table
of contents. In this case, you use the \field{type} field or the
\field{type} resource option. The ordering of the glossaries is
determined by the ordering of the \cs{printunsrtglossary}
commands within the document. For example:
\begin{codeenv}
\cs{printunsrtglossary}
\cs{printunsrtglossary}\oarg{\printglossopt[abbreviations]{type}}
\cs{printunsrtglossary}\oarg{\printglossopt[symbols]{type}}
\end{codeenv}

Suppose that your list of terms spans many pages and you feel it
would be helpful to the reader to split it up into letter groups
then you would need to run \bibgls\ with the \longarg{group} switch
and use a glossary style that supports \idxpl{lettergroup} for that
glossary. For example:
\begin{codeenv}
\cs{printunsrtglossary}\oarg{\printglossopt[indexgroup]{style}}
\end{codeenv}

Suppose that your list of symbols consists of pictographs, Latin characters
and Greek characters and you want them grouped together
in that order. Then you would use a separate
\gls{GlsXtrLoadResources} for each block and assign your own custom 
group. This means ensuring that each \igls{resourceset} only selects
the terms for that group. The simplest way of doing this is to
have a separate \ext{bib} file for each set. For example:
\begin{codeenv}
\gls{glsxtrsetgrouptitle}\marg{pictographs}\marg{Pictographs}
\gls{glsxtrsetgrouptitle}\marg{latinsymbols}\marg{Latin Characters}
\gls{glsxtrsetgrouptitle}\marg{greeksymbols}\marg{Greek Characters}
\gls{GlsXtrLoadResources}\oarg{
  \csopt[generalsymbols]{src},\comment{data in generalsymbols.bib}
  \csopt[pictographs]{group},
  \csopt[symbols]{type}
}
\gls{GlsXtrLoadResources}\oarg{
  \csopt[latinsymbols]{src},\comment{data in latinsymbols.bib}
  \csopt[latin]{group},
  \csopt[symbols]{type}
}
\gls{GlsXtrLoadResources}\oarg{
  \csopt[greeksymbols]{src},\comment{data in greeksymbols.bib}
  \csopt[greek]{group},
  \csopt[symbols]{type}
}
\end{codeenv}

Suppose instead that you have many of these logical blocks and you
want them ordered according to the block title. In this case you
have a \gls{hierarchicalglossary} and you need to use the \field{parent}
field. You then need to select an appropriate \idx{glossarystyle}.

If you only want to have a single \ext{bib} file that contains all
your entries and you want to share it across multiple documents then
the most flexible approach is to use custom fields and \glspl{entrytype}
that can be aliased according to the needs of the \glspl{resourceset}.

For example, the file \filefmt{entries.bib}:
\begin{codeenv}
\comment{Encoding: UTF-8}
\strut
\atentryfmt{indexplural}\marg{latin,\field{text}=\marg{Latin character}}
\atentryfmt{indexplural}\marg{greek,\field{text}=\marg{Greek character}}
\atentryfmt{indexplural}\marg{pictograph}
\strut
\atentry{symbol}\marg{fx,
  \field{name}=\marg{\cs{ensuremath}\marg{f(x)}},
  \field{description}=\marg{function of \idx{mshiftchar}x\idx{mshiftchar}},
  \fieldfmt{identifier}=\marg{latin}
}
\strut
\atentry{symbol}\marg{f\idx{aposchar}x,
  \field{name}={\cs{ensuremath}\marg{f\idx{aposchar}(x)}},
  \field{description}=\marg{derivative of \cs{gls}\marg{fx}},
  \fieldfmt{identifier}=\marg{latin}
}
\strut
\atentry{symbol}\marg{pi,
  \field{name}=\marg{\cs{ensuremath}\marg{\cs{pi}}},
  \field{description}=\marg{ratio of circumference to diameter},
  \fieldfmt{identifier}=\marg{greek}
}
\strut
\atentry{symbol}\marg{heart,
  \field{name}=\marg{\cs{ensuremath}\marg{\ics{heartsuit}}},
  \field{description}=\marg{heart},
  \fieldfmt{identifier}=\marg{pictograph}
}
\strut
\atentry{symbol}\marg{diamond,
  \field{name}=\marg{\cs{ensuremath}\marg{\ics{diamondsuit}}},
  \field{description}=\marg{diamond},
  \fieldfmt{identifier}=\marg{pictograph}
}
\strut
\atentry{abbreviation}\marg{html,
  \field{short}=\marg{html},
  \field{long}=\marg{hypertext markup language},
  \fieldfmt{identifier}=\marg{markuplanguage}
}
\strut
\atentry{abbreviation}\marg{xml,
  \field{short}=\marg{xml},
  \field{long}=\marg{extensible markup language},
  \fieldfmt{identifier}=\marg{markuplanguage}
}
\strut
\atentry{entry}\marg{duck,
  \field{name}=\marg{duck},
  \field{description}=\marg{a waterbird with webbed feet},
  \fieldfmt{identifier}=\marg{animal}
}
\strut
\atentry{entry}\marg{parrot,
  \field{name}=\marg{parrot},
  \field{description}=\marg{mainly tropical bird with bright plumage},
  \fieldfmt{identifier}=\marg{animal}
}
\end{codeenv}
This has a custom field \fieldfmt{identifier}. This will be ignored
by \bibgls\ unless defined or aliased in the document.

Here's an example document that creates three glossary types (the
default \mainglossary\ and the glossaries created with the
\styopt{abbreviations} and \styopt{symbols} options). They are
listed in the order of \cs{printunsrtglossary} and their titles are
added to the table of contents.

The custom \fieldfmt{identifier} fields are ignored for the
\glsdisp{mainglossary}{main} and abbreviation glossaries, but they are aliased for the
symbols to the \field{group} field. Since I've split the symbols
glossary into blocks with each block only containing entries that
have the same \field{group} value, this isn't a problem. It also
won't trigger a warning with \longarg{warn-non-bib-fields} as it's
being aliased rather than set in the \ext{bib} file. The blocks
appear in the same order as the corresponding \gls{GlsXtrLoadResources} 
commands. The title for each block is provided in the document 
using \gls{glsxtrsetgrouptitle}.
\begin{codeenv}
\cmd{documentclass}\marg{article}
\strut
\cmd{usepackage}\oarg{\styopt{record},\styopt{abbreviations},\styopt{symbols}}\marg{glossaries-extra}
\strut
\cs{renewcommand}\marg{\gls{GlsXtrDefaultResourceOptions}}\marg{
 \csopt[all]{selection},\csopt[entries]{src},\csopt[false]{save-locations}}
\strut
\gls{GlsXtrLoadResources}\oarg{\csopt[main]{type},\csopt[entrytype=entry]{match}}
\gls{GlsXtrLoadResources}\oarg{\csopt[abbreviations]{type},
 \csopt[entrytype=abbreviation]{match}}
\strut
\gls{glsxtrsetgrouptitle}\marg{pictograph}\marg{Pictographs}
\gls{GlsXtrLoadResources}\oarg{\csopt[symbols]{type},
 \csopt[identifier=\field{group}]{field-aliases},
 \csopt[\field{group}=pictograph]{match}}
\strut
\gls{glsxtrsetgrouptitle}\marg{latin}\marg{Latin Characters}
\gls{GlsXtrLoadResources}\oarg{\csopt[symbols]{type},
 \csopt[identifier=\field{group}]{field-aliases},
 \csopt[\field{group}=latin]{match}}
\strut
\gls{glsxtrsetgrouptitle}\marg{greek}\marg{Greek Characters}
\gls{GlsXtrLoadResources}\oarg{\csopt[symbols]{type},
 \csopt[identifier=\field{group}]{field-aliases},
 \csopt[\field{group}=greek]{match}}
\strut
\cmd{begin}\marg{document}
\cs{tableofcontents}
\cs{printunsrtglossary}\oarg{\printglossopt[abbreviations]{type}}
\cs{printunsrtglossary}
\cs{printunsrtglossary}\oarg{\printglossopt[symbols]{type},\printglossopt[\glostyle{treegroup}]{style}}
\cmd{end}\marg{document}
\end{codeenv}

In the above example document, the symbols list is divided into
three groups, listed in the order: Pictographs, Latin characters and
Greek characters. If you want these titles ordered alphabetically
then you need a \hierarchical\ structure instead. This can be obtained
by aliasing the custom \fieldfmt{identifier} field to
\field{parent}:
\begin{codeenv}
\cmd{documentclass}\marg{article}
\strut
\cmd{usepackage}\oarg{\styopt{record},\styopt[topic]{stylemods},\styopt{abbreviations},\styopt{symbols}}\marg{glossaries-extra}
\strut
\cs{renewcommand}\marg{\gls{GlsXtrDefaultResourceOptions}}\marg{\comment{}
 \csopt[all]{selection},\csopt[entries]{src},\csopt[false]{save-locations}}
\strut
\gls{GlsXtrLoadResources}\oarg{\csopt[main]{type},\csopt[entrytype=entry]{match}}
\gls{GlsXtrLoadResources}\oarg{\csopt[abbreviations]{type},
 \csopt[entrytype=abbreviation]{match}}
\strut
\gls{GlsXtrLoadResources}\oarg{\csopt[symbols]{type},
 \csopt[identifier=\field{parent}]{field-aliases},
 \csopt[entrytype=symbol,entrytype=indexplural]{match}}
\strut
\cmd{begin}\marg{document}
\cs{tableofcontents}
\cs{printunsrtglossary}\oarg{\printglossopt[abbreviations]{type}}
\cs{printunsrtglossary}
\cs{printunsrtglossary}\oarg{\printglossopt[symbols]{type},\printglossopt[\glostyle{topic}]{style}}
\cmd{end}\marg{document}
\end{codeenv}
The style used for the symbols list is now \glostyle{topic} rather
than \glostyle{treegroup}. This results in a slightly different
appearance. You can select the most appropriate style according to
your needs (see the gallery of predefined
styles~\cite{glossarystylesgallery}). The topic ordering is now:
Greek characters, Latin characters and Pictographs.

\section{Defining a New Glossary}
\label{sec:newglossary}

Some of the examples in this manual use \ics{newglossary*} to define
a new glossary type and some use \ics{newignoredglossary} or
\ics{newignoredglossary*}. You may be wondering why the starred
forms and why define an \igls{ignoredglossary}?

The base \styfmt{glossaries} package was originally designed to work
with \idx!{makeindex}. Support for \idx!{xindy} was later added, but
both require three files per glossary type: the transcript file
(created by the indexing application), the file written by \LaTeX\
(and input by the indexing application) and the file input by
\LaTeX\ (and written by the indexing application). So when a new glossary
is defined with \ics{newglossary}, this not only defines internal
control sequences that store the list of entry labels associated
with that glossary, the title and the entry format but also has to
define internal control sequences that store the three file
extensions. The starred form \cs{newglossary*} is just a shortcut
that forms the extensions from the glossary label. For the purposes
of \bibgls, this is simpler than the unstarred version since the
extensions are now irrelevant as they are only applicable to
\idx!{makeindex} and \idx!{xindy}. (Unless, of course, you are using
a hybrid method with \styopt[alsoindex]{record}.)

Since some users wanted the ability to define entries that were
common enough to not be worth including in any glossary lists, the
concept of an \igls{ignoredglossary} was introduced, defined with
\cs{newignoredglossary}. This only requires an internal control
sequence to store the list of entry labels associated with that
glossary\footnote{All entries must be assigned to a glossary. If you
don't use the \field{type} field the default is used.}\ and the
associated internal command that governs the way that
commands like \cs{gls} are displayed for that glossary type. 
Since this type of glossary has no associated files, it can't be
used with \cs{printglossary} and therefore isn't included in the
list of glossary labels that's iterated over by commands like
\cs{printglossaries}. Since there's no glossary list (and therefore
no targets), \cs{newignoredglossary} additionally disables hyperlinks for that
glossary type, but it doesn't disable indexing.  The indexing macro
is still called, but because there's no associated file to write to,
it has no effect. With \bibgls, the indexing is written to the
\ext{aux} file and so does have an effect.

Although \iglspl{ignoredglossary} can't be used with
\cs{printglossary}, they can be used with \cs{printunsrtglossary},
which is designed to work without any indexing, but you need to
explicitly set the title in the optional argument to override the
default. \Idxpl{ignoredglossary} still can't be used in
\cs{printunsrtglossaries}, since they're not included in the list
that this command iterates over.

So \cs{newignoredglossary} (or \cs{provideignoredglossary}) is
useful with \bibgls\ if you're happy to use \cs{printunsrtglossary}
with the \printglossopt{type} and \printglossopt{title} options as it reduces the overall
number of internal control sequences.  \Idxpl{ignoredglossary} are
also useful for stand-alone definitions (\ics{glsxtrglossentry}) or
with \ics{printunsrtinnerglossary} as no title is required in those
cases (see \exfile{sample-nested.tex} for an example). 

Since there is now the possibility of targets (created within \cs{printunsrtglossary} or
\cs{printunsrtinnerglossary} or \cs{glsxtrglossentry}), it's convenient to
have an \igls{ignoredglossary} that doesn't suppress the hyperlinks,
which can be obtained with the starred form \ics{newignoredglossary*}
provided by \styfmt{glossaries-extra} (or \cs{provideignoredglossary*}).

Some resource options, such as \csopt{master}, \csopt{secondary},
and \csopt{trigger-type}, need to ensure that a
required glossary is defined. In this case, \bibgls\ uses
\ics{provideignoredglossary*} in the \iext{glstex} file 
even if \longarg{no-provide-glossaries} is set.
Note that only \csopt{ignored-type} uses the unstarred
\ics{provideignoredglossary}.

If you haven't already defined that glossary in the document with \cs{newglossary*},
you'll need to set the title in the optional argument of
\cs{printunsrtglossary} if you don't want the default. The glossary
won't be defined on the first run (if the definition is only
provided in the \ext{glstex} file) but \cs{printunsrtglossary} will
just give a warning if the type is undefined so it won't interrupt
the document build.

If you want \bibgls\ to automatically provide unknown glossaries for
all entries that have the \field{type} field set (unrelated to the
\csopt{master}, \csopt{secondary}, \csopt{trigger-type}
and \csopt{ignored-type} options)
then use the \longarg{provide-glossaries} switch.

The base \sty{glossaries} package provides a command that can be
used to test the existence of a glossary:
\begin{codeenv}
\ics{ifglossaryexists}\margm{label}\margm{true}\margm{false}
\end{codeenv}
The unstarred version considers \glspl{ignoredglossary} as non-existent
(and so will do \meta{false} for an \gls{ignoredglossary}). As from
v4.46, this command now has a starred version
\ics{ifglossaryexists*} that considers \glspl{ignoredglossary} as
existing (and so will do \meta{true} for an \gls{ignoredglossary}). In
the event that you have an older version of \sty{glossaries}, the
\sty{glossaries-extra} package (v1.44+) will provide the starred
form if it hasn't been defined. (In general, it's best to have
up-to-date versions of both \sty{glossaries} and
\sty{glossaries-extra}.)

\section{Resource Sets}
\label{sec:resourcesets}

Each instance of \gls{glsxtrresourcefile} or
\gls{GlsXtrLoadResources} in the document represents a \igls{resourceset}. Each
\igls{resourceset} has one or more associated \iext{bib} files
that provides the data for that set. Command line switches
(\sectionref{sec:switches}) are applied to all \iglspl{resourceset}.
Resource options (\sectionref{sec:resourceopts}) are only applied to
that specific \igls{resourceset}. Each \igls{resourceset} is processed
in stages:
\begin{description}

\item[Stage 1 (Initialisation)] Occurs after the \ext{aux} file has
been read, this stage parses the resource option list and ensures
options are valid and don't cause a conflict. The transcript will
show the message
\begin{alltt}
Initialising resource \meta{resource-name}
\end{alltt}
at this point.

\item[Stage 2 (Parsing)] All the \iext{bib} files associated with
the \igls{resourceset} are parsed. Entry aliases (identified by
\csopt{entry-type-aliases}) are performed. The
\hyperref[sec:multientry]{multi-entry types}, such as
\atentry{bibtexentry} and \atentry{progenitor}, spawn their
associated \glspl{primaryentry}. Preamble information (provided by
\atentry{preamble}) is saved but is not interpreted at this stage.
The transcript will show the message
\begin{alltt}
Parsing bib files for resource \meta{resource-name}
\end{alltt}
at this point.

\item[Stage 3 (Processing Entries)] 
The transcript will show the message
\begin{alltt}
Processing resource \meta{resource-name}
\end{alltt}
at this point.
For each entry that was found in
the corresponding set of \ext{bib} files: 
\begin{itemize}
\item Records are transferred to aliases if required (\csopt{alias-loc}).
\item Field checks and modifications are performed:
  \begin{itemize}
  \item field aliases are performed (\csopt{field-aliases}); 

  \item known fields identified with \csopt{save-original-id} and
  \csopt{save-original-entrytype} are set (internal fields that
  don't have a corresponding key for use with \gls!{newglossaryentry} aren't 
  set until the \ext{glstex} file is written);

  \item ignored fields (identified by \csopt{ignore-fields}, not by \csopt{omit-fields}) are
  removed;

  \item case-changes (for example, \csopt{short-case-change}) are
  performed, except for the \field{name} field and fields identified
  with \csopt{field-case-change};

  \item suffixes are appended if required (for example, with
  \csopt{short-plural-suffix});

  \item field replications are made (\csopt{replicate-fields}),
    and any of the above case-change or suffixes required
    on the replicated fields are performed;

  \item the \field{group} field is assigned if \csopt[\meta{label}]{group} 
    is set;

  \item any variables (identified by \atentry{string}) are expanded
  (if not already done in any of the previous steps);

  \item any fields that have been identified by
  \csopt{bibtex-contributor-fields} are converted;

  \item any fields that have been identified with
  \csopt{encapsulate-fields} are converted;

  \item any fields that have been identified with
  \csopt{encapsulate-fields*} are converted;

  \item any fields that must be converted into a label form
  (\csopt{labelify} or \csopt{labelify-list}) are processed;

  \item any fields identified by \csopt{dependency-fields} are parsed for
  dependent entries;

  \item any fields whose value must be a label are interpreted
  if \csopt{interpret-label-fields} is set;

  \item the \field{parent} field is adjusted according to the label prefix 
  settings (\csopt{label-prefix} etc);

  \item \ics{makefirstuc} protection is applied according to 
  \longarg{mfirstuc-protection} and \longarg{mfirstuc-math-protection};

  \item fields are parsed for commands like \cs{gls} or
  \cs{glshyperlink} and also checked for nested links if
  \longarg{nested-link-check} is set;

  \item the \field{description} field is adjusted according to
  \csopt{strip-trailing-nopost};

  \item end punctuation is checked according to
  \csopt{check-end-punctuation};

  \item field assignments are made (\csopt{assign-fields}),
    and any of the above case-change or suffixes required
    by the destination fields are performed;

  \item \field{name} adjustment is performed if
   \csopt{compound-adjust-name} is set (and the criteria is met);

  \item \field{name} case-change is performed if
  \csopt{name-case-change} is set;

  \item if \csopt[true]{copy-alias-to-see} the \field{alias} is copied 
  to the \field{see} field;

  \item general field case changes identified by
  \csopt{field-case-change} are performed;

  \item any fields that have been identified with
  \csopt{interpret-fields} are replaced with their interpreted
  values;

  \item any fields that have been identified with
  \csopt{hex-unicode-fields} will have Unicode characters replaced;

  \item check for \field{nonumberlist}.
  \end{itemize}
\item The dual version (if appropriate) is created.
\item \Glspl{record} are added to the entry's \igls{locationlist} (or transferred
to the \dual\slash\primary\ according to \csopt{combine-dual-locations}).
\item The \field{type}, \field{category} and \field{counter} fields
are set according to \csopt{type}, \csopt{dual-type},
\csopt{category}, \csopt{dual-category}, \csopt{counter} and
\csopt{dual-counter}.
\item Filtering is applied (according to options like \csopt{match}
but not \csopt{selection} or \csopt{limit}).
\item Required fields are checked for existence.
\item Dependencies are registered (if
\csopt[recorded and deps]{selection} or \csopt[recorded and deps and
see]{selection}). 
\item Any fields that have been identified by
\csopt{date-time-fields}, \csopt{date-fields} or \csopt{time-fields}
are converted.
\end{itemize}
If \csopt[recorded and deps and see]{selection} then any
\glspl{recordedentry} that have been cross-referenced by an
\gls{unrecordedentry}, will register a dependency with the
\gls{unrecordedentry}.

The \gls{compoundentry} options \csopt{compound-dependent} and
\csopt{compound-add-hierarchy} are implemented, if enabled.

Finally, \glspl{supplementalrecord} are added to entries.

\item[Stage 4 (Selection, Sorting, Writing)] Entries are selected
from the list according to the \csopt{selection} setting, sorting is
performed (if required), truncation is applied (if \csopt{limit} is
set) and the \iext{glstex} file is written.
The transcript will show the message
\begin{alltt}
Selecting entries for resource \meta{resource-name}
\end{alltt}
or (if \csopt{master})
\begin{alltt}
Processing master \meta{resource-name}
\end{alltt}
at this point.

Options such as \csopt{copy-to-glossary} and \csopt{omit-fields} are
implemented when each entry has its definition written to the \ext{glstex}
file.  This means that the omitted fields will still be available for actions
such as sorting, establishing dependencies, or field assignments.

\end{description}

Parent entries must always be in the same \igls{resourceset} as their
\glspl{childentry}. (They may be defined in different \ext{bib} files as
long as all those \ext{bib} files are listed in the same \csopt{src}.) 
Other forms of dependencies may be in a
different \igls{resourceset} under certain circumstances. These types
of dependencies are instances of commands such as \cs{gls} being
found (for example, in the \field{description} field), or the
\glspl{cross-referencefield} (\field{see}, \field{seealso} or
\field{alias} or fields identified with \csopt{dependency-fields}) 
in \glspl{recordedentry} that reference \glspl{unrecordedentry}.

The \qt{cross-referenced by} dependencies enabled with
\csopt[recorded and deps and see]{selection} (where an
\gls{unrecordedentry} references a \gls{recordedentry} through the
\glspl{cross-referencefield}) \emph{aren't supported} across
\iglspl{resourceset} (even with \longarg{force-cross-resource-refs}).

A \igls{crossresourceref} is a reference from a \gls{recordedentry}
provided in one \igls{resourceset} to an \gls{unrecordedentry} in
another \igls{resourceset}.  Since the contents of each
\igls{resourceset}['s] preamble must be processed before fields can
be interpreted and one \igls{resourceset}['s] preamble may contain
definitions that override another, \iglspl{crossresourceref} can't
be supported if fields containing cross-referencing information need
to be interpreted.

The \igls{crossresourceref} mode determines whether or not \bibgls\
can support \iglspl{crossresourceref}. If enabled, the message
\begin{verbatim}
Cross-resource references allowed.
\end{verbatim}
will be written to the transcript otherwise the message is
\begin{verbatim}
Cross-resource references disabled.
\end{verbatim}
The mode can only be enabled if the following condition is satisfied:
\begin{itemize}
\item the interpreter is off (\longarg{no-interpret}), or
\item every \igls{resourceset} either doesn't have a preamble
(\atentry{preamble}) or has \csopt[false]{interpret-preamble} set.
\end{itemize}
If you know the preamble contents won't cause a problem, you
can force the \iglspl{crossresourceref} mode on with 
\longarg{force-cross-resource-refs}.

If you don't use either \csopt[recorded and
deps]{selection} or \csopt[recorded and deps and see]{selection}
then the dependencies aren't picked up for that \igls{resourceset}
(and so can't be cross-referenced from another \igls{resourceset}).

Trails don't work with \iglspl{crossresourceref}. For example, if
entry $A$ has been recorded and depends on entry $B$ that hasn't been
recorded, then $B$ can be picked up from a different
\igls{resourceset}, but if $A$ and $B$ are in the same
\igls{resourceset} and $B$ is dependent on $C$ which is in a
different \igls{resourceset} then $C$ won't be picked up if it hasn't
been recorded because $B$ hasn't been recorded and is in a different
\igls{resourceset}.

If the \igls{crossresourceref} mode is enabled then stage~3 and
stage~4 are processed in separate \idxpl{loop}, otherwise they are
processed in the same \idx{loop}.

\section{\bibgls\ Quarks}
\label{sec:quarks}

A \bibgls\ \idx{quark} is similar in principle to a \LaTeX3 quark,
in that it is a token that looks like a control sequence but isn't
intended to be interpreted as a \LaTeX\ command. Unlike \LaTeX3
quarks, their name isn't prefixed with \csfmt{q\_} and can
coincidentally look the same as a \LaTeX\ command. This is
particularly the case with regular expressions that have escaped
characters to indicate a literal character. For example, in a
regular expression a pipe or vertical bar character \idx{regexpor}
indicates \qt{or}. If you want to match a literal pipe, you need to
identify this with \cs{quark.pipe}. This is distinct from, but
visually identical to, the \LaTeX\ command used to create a double
vertical bar in maths mode.

The resource options provided in \gls{GlsXtrLoadResources} expand as
they are written to the \ext{aux} file. This allows commands to be
used within the resource options that expand to a complex option
that may be required multiple times. For example,
\ics{GlsXtrBibTeXEntryAliases} or \ics{glsxtrhyphenrules}.
Unfortunately, this means that quarks must be prevented from
expansion as they form part of the option syntax and are not
intended for use in the document.

This means that, unless they happen to coincidentally be robust
commands, they must be preceded by either \ics{protect} or
\idx{cs.string}.  Since \cs{protect} adds a space afterwards,
\idx{cs.string} is usually better if the syntax requires that spaces
after quarks are significant.

This can lead to cumbersome expressions, but it's possible to
redefine \ics{glsxtrresourceinit} to locally redefine these quarks
to expand to detokenized forms of themselves. For example:
\begin{codeenv}
\cs{renewcommand}*\marg{\cs{glsxtrresourceinit}}\marg{\cmd{let}\cmd{u}\cs{glshex}}
\end{codeenv}
Since there are a number of these quarks, as from v1.51,
\sty{glossaries-extra-bib2gls} (which is automatically loaded with
\styopt{record}) provides \ics{GlsXtrResourceInitEscSequences},
so you can change the above to the following:
\begin{codeenv}
\cs{renewcommand}*\marg{\cs{glsxtrresourceinit}}\marg{\comment{}
  \cs{GlsXtrResourceInitEscSequences}
}
\end{codeenv}
Note that if new quarks, such as \gls{INTERPRETNOREPL} and \gls{REPLACESPCHARS},
are added to \bibgls, they may not be included in \cs{GlsXtrResourceInitEscSequences} if they
were introduced to \bibgls\ after the version date of the \sty{glossaries-extra}
package installed on your system. In this case, you will need to add them. For example:
\begin{codeenv}
\cs{renewcommand}*\marg{\cs{glsxtrresourceinit}}\marg{\comment{}
  \cs{GlsXtrResourceInitEscSequences}
  \cmd{def}\gls{INTERPRETNOREPL}\marg{\idx{cs.string}\gls{INTERPRETNOREPL}}\comment{}
  \cmd{def}\gls{REPLACESPCHARS}\marg{\idx{cs.string}\gls{REPLACESPCHARS}}\comment{}
}
\end{codeenv}


This will locally define the quarks listed below. Since
\cs{glsxtrresourceinit} is used in a scoped context, the
definitions only have an effect within the protected write, and so this
shouldn't interfere with the corresponding commands that are
required in the document. Note that these quarks should only be used
in their designated contexts.

\begin{description}

\item[General] \inlinedef{uhex} is recognised in certain resource options
(such as \csopt{field-concat-sep}) as indicating the Unicode
character with the given hexadecimal code.

\item[Regular expressions] The following indicate a literal
character: 
\glsadd{idx.periodchar}\inlinedef{quark.dot} 
\glsadd{idx.backslashchar}\inlinedef{quark.backslash}
\glsadd{idx.slashchar}\inlinedef{quark.slash} 
\glsadd{idx.pipechar}\inlinedef{quark.pipe} 
\glsadd{idx.ampchar}\inlinedef{quark.amp}
\glsadd{idx.pluschar}\inlinedef{quark.plus} 
\glsadd{idx.ltchar}\inlinedef{quark.lt} 
\glsadd{idx.gtchar}\inlinedef{quark.gt} 
\glsadd{idx.starchar}\inlinedef{quark.star}
\glsadd{idx.dollarchar}\inlinedef{quark.dollar} 
\glsadd{idx.circumchar}\inlinedef{quark.circum} 
\glsadd{idx.tildechar}\inlinedef{quark.tilde}
\glsadd{idx.openparenchar}\inlinedef{quark.openparen} 
\glsadd{idx.closeparenchar}\inlinedef{quark.closeparen}
\glsadd{idx.opensqchar}\inlinedef{quark.opensq} 
\glsadd{idx.closesqchar}\inlinedef{quark.closesq}
\glsadd{idx.doublequotechar}\inlinedef{quark.doublequote} 
\glsadd{idx.hyphenchar}\inlinedef{quark.hyphen}
\glsadd{idx.questionchar}\inlinedef{quark.question} 
\glsadd{idx.colonchar}\inlinedef{quark.colon} 
\glsadd{idx.hashchar}\inlinedef{quark.hash}.
Note that \idxpl{regex} in resource options are typically
\gls{anchored}, so there shouldn't be any need to use \idx{regex.circum}
or \idx{regex.dollar} to denote the start and end.

\item[Field assignments] The following commands may be used in the
\meta{element-list} syntax of \csopt{assign-fields}:
\gls{CS}, \gls{MGP}, \gls{LEN}, \gls{TRIM}, \gls{INTERPRET},
\gls{INTERPRETNOREPL}, \gls{REPLACESPCHARS},
\gls{LC}, \gls{UC}, \gls{FIRSTLC}, \gls{FIRSTUC}, and \gls{TITLE}.

\item[Conditionals] The \meta{condition} part of the \csopt{assign-fields}
syntax recognises \gls{LEN}, \gls{CAT}, \gls{IN}, \gls{NIN}, \gls{PREFIXOF},
\gls{NOTPREFIXOF}, \gls{SUFFIXOF}, \gls{NOTSUFFIXOF} and \gls{NULL}.
\end{description}

Finally, this isn't actually a quark, but \inlinedef{cs} is defined
to expand to the literal string \csfmt{\meta{csname}} so you can use it for any
other escape sequences that aren't covered above. For example, 
\code{\gls{cs}\marg{n}} for a newline \ics{n}.

\section{Indexing}

The dual index entries such as \atentry{dualindexentry} (described in
\sectionref{sec:dualentry}) are designed to provide a way of
including an entry in a glossary (with a description) and also
include the term (without the description) in an index.  Additional
terms that should only appear in the index can be defined with
\atentry{index}.  (See, for example, the \exfile{sample-multi1.tex}
and \exfile{sample-multi2.tex} sample files.)

Although \bibgls\ is designed to create indexes as well as glossary
lists using the same interface (\cs{gls} etc), it is
possible to have a mixture of \bibgls\ and \ics{index}. For example:
\begin{codeenv}
\cmd{documentclass}\marg{report}
\strut
\cmd{usepackage}\marg{makeidx}
\cmd{usepackage}\oarg{\styopt{record}}\marg{glossaries-extra}
\strut
\cmd{makeindex}
\gls{GlsXtrLoadResources}\oarg{\csopt[entries]{src}}
\strut
\ics{glssetcategoryattribute}\marg{general}\marg{dualindex}\marg{true}
\cs{glssetcategoryattribute}\marg{symbol}\marg{dualindex}\marg{true}
\cs{glssetcategoryattribute}\marg{abbreviation}\marg{dualindex}\marg{true}
\strut
\cs{glssetcategoryattribute}\marg{general}\marg{indexname}\marg{hyperbf}
\cs{glssetcategoryattribute}\marg{symbol}\marg{indexname}\marg{hyperbf}
\cs{glssetcategoryattribute}\marg{abbreviation}\marg{indexname}\marg{hyperbf}
\strut
\cmd{begin}\marg{document}
\cmd{chapter}\marg{Example}
\cs{gls}\marg{bird}, \cs{gls}\marg{html}, \idx{mshiftchar}\cs{gls}\marg{v}\idx{mshiftchar} and \cs{glspl}\marg{goose}.
\strut
\cs{printunsrtglossaries}
\cmd{printindex}
\cmd{end}\marg{document}
\end{codeenv}
If the document is called \filefmt{myDoc.tex} then the document
build is:
\begin{verbatim}
pdflatex myDoc
bib2gls myDoc
pdflatex myDoc
makeindex myDoc.idx
pdflatex myDoc
\end{verbatim}
This requires an additional \LaTeX\ call between \bibgls\ and
\idx{makeindex} since the entries must be defined before they can be
indexed (and they can't be defined until \bibgls\ creates the
associated \ext{glstex} files).

Note that this method will use the \field{sort} value obtained by \bibgls\
as the \meta{sort} part within
\code{\cs{index}\marg{\meta{sort}@\meta{actual}}}.
Be careful if you use \idx{makeindex} as this can result in Unicode
characters appearing in the sort value, which \idx{makeindex}
doesn't support.
The \meta{actual} part is given by
\code{\ics{glsentryname}\margm{label}}.
(You can change the \meta{sort} and \meta{actual} parts by
redefining \ics{glsxtrautoindexassignsort} and
\ics{glsxtrautoindexentry}. See the \sty{glossaries-extra} manual
for further details.)

\section{Security}

\TeX\ Live come with security settings
\texmfcnf{openinany} and \texmfcnf{openoutany} that, respectively,
govern read and write file access (in addition to the operating
system's file permissions). \bibgls\ uses \idx{kpsewhich} to
determine these values and honours them. MikTeX doesn't use these
settings, so if these values are unset, \bibgls\ will default to
\code{a} (any) for \texmfcnf{openinany} and \code{p} (paranoid) for
\texmfcnf{openoutany}.

The only external processes that are run by \bibgls\ are calls to
\idx{kpsewhich} to check the security settings and locate files on
\TeX's path. These are started with Java's \code{ProcessBuilder}
class so there should be no issues with spaces or shell special characters
in the argument.  The \commonlongarg{debug} switch will write the process
call in the transcript file and will delimit the argument in the log
with single quote characters for convenience, but the process isn't
actually called in that way.

\bibgls\ creates files with the extension \ext{glstex}, which are
input by \gls{glsxtrresourcefile} (and therefore by the shortcut
\gls{GlsXtrLoadResources}). This extension is fixed and is imposed
by both \bibgls\ and \gls{glsxtrresourcefile}. \bibgls\ also creates
a transcript file with the default extension \ext{glg}. This may be
overridden by the \longarg{log-file} switch, but \bibgls\ always
forbids write access to any file with the following extensions:
\extfmt{tex}, \extfmt{ltx}, \extfmt{sty}, \extfmt{cls},
\extfmt{bib}, \extfmt{dtx}, \extfmt{ins}, \extfmt{def} and
\extfmt{ldf}.

\section{Localisation}
\label{sec:lang.xml}

The messages produced by \bibgls\ are fetched from a resource file
called \metafilefmt{bib2gls-}{lang}{.xml}, where \meta{lang} is a
valid \idx{IETF} language tag.

The appropriate file is searched for in the following order, where
\meta{locale} is the \gls{Java-locale} or the value supplied
by the \commonlongarg{locale} switch:
\begin{enumerate}
\item \meta{lang} exactly matches \meta{locale}.
For example, my locale is \code{en-GB}, so \bibgls\ will first search
for \filefmt{bib2gls-en-GB.xml}. This file doesn't exist, so it will
try again.

\item If \meta{locale} has an associated script, the
next try is with \meta{lang} set to \meta{lang
code}\code{-}\meta{script} where \meta{lang code} is the two
letter ISO language code and \meta{script} is the script code.
For example, if \meta{locale} is \code{sr-RS-Latn}
then \bibgls\ will search for \filefmt{bib2gls-sr-Latn.xml} if
\filefmt{bib2gls-sr-RS-Latn.xml} doesn't exist.

\item The final attempt is with \meta{lang} set to just the two
letter ISO language code. For example, \filefmt{bib2gls-sr.xml}.
\end{enumerate}

If there is no match, \bibgls\ will fallback on the English resource file
\file{bib2gls-en.xml}.
(Currently only \file{bib2gls-en.xml} exists as my language skills aren't up
to translating it. Any volunteers who want to provide other language
resource files would be much appreciated.)

In addition to the main language file, it's possible to have
supplementary files that provide text that matches the
\gls{resource-locale}. These are in files called
\metafilefmt{bib2gls-extra-}{lang}{.xml}, which has the same format
as \metafilefmt{bib2gls-}{lang}{.xml}. These supplementary files
will be loaded automatically if they exist and if you have
\styfmt{glossaries-extra} v1.51+ (which will save a list of all
tracked languages for the document).

Note that if you use the \csopt[true]{loc-prefix} option, the
textual labels (\qt{Page} and \qt{Pages} in English) will be 
first be attempted from the supplementary file with the tags
\code{tag.\meta{lang}.page} and \code{tag.\meta{lang}.pages}
(where \meta{lang} is the language code) and then, if not
found, from the main resource file using the tags \code{tag.page}
and \code{tag.pages}. In the event that the loaded resource file
doesn't match the document language and there's no supplementary
file, you will have to manually set
the correct translation (in English, this would be
\csopt[Page\dcomma Pages]{loc-prefix}). The default definition of
\gls!{bibglspassim} is also obtained from the resource file in a
similar manner.

There are also keys in the resource file to assist case-conversion.
Currently, there's only support for the Dutch \qt{IJ} case.

\section{Conditional Document Build}

If you are using a document build method that tries to determine
whether or not \bibgls\ should be run, you can find the information
by searching the \iext{aux} file for instances of
\nosecdef{glsxtr@resource}
Each instance corresponds to an instance of \gls{glsxtrresourcefile}
where \meta{filename} is the base name of the \iext{glstex} file
that \bibgls\ needs to create for this resource set. If the
\meta{options} part is missing the \csopt{src} option, then
\meta{filename} also indicates the base name for the \iext{bib} file.

So the simplest check to determine if \bibgls\ needs to be run is to test
if the \iext{aux} file contains \gls{glsxtr@resource}. For
example, with \idx{arara} version 4.0:
\begin{codeenv}
\% arara: bib2gls if found("aux", "glsxtr@resource")
\end{codeenv}
A sophisticated method could check if
\meta{filename}\ext{glstex} is missing or is older than the document
\ext{tex} file for each instance of \gls{glsxtr@resource} found in
the \ext{aux} file.

It might also be possible, although far more complex, to
parse the \meta{options} part in each instance of \gls{glsxtr@resource}
for \csopt{src} and determine if the corresponding \ext{bib} file or
files are newer than the \ext{tex} file.

It's not possible to determine if the \glspl{locationlist} require
updating, just as it's not possible to do this for the
\idx{toc}, list of figures, list of tables etc. (Or, if it could
be implemented, the required code would make the document build far
more complicated.)

In general, the basic algorithm is:
\begin{enumerate}
\item Run \LaTeX\ (or PDF\LaTeX\ etc).

\item If \gls{glsxtr@resource} is found in the \ext{aux} file then:

   \begin{enumerate}
   \item run \bibgls;
   \item run \LaTeX\ (or PDF\LaTeX\ etc).
   \end{enumerate}

\item If \ics{@istfilename} is found in the \ext{aux} file then:

   \begin{enumerate}
   \item run \idx{makeglossaries} (or \idx{makeglossaries-lite});
   \item run \LaTeX\ (or PDF\LaTeX\ etc).
   \end{enumerate}

\end{enumerate}
This allows for the \styopt[alsoindex]{record} package option.  See
also
\href{https://www.dickimaw-books.com/latex/buildglossaries}{\qt{Incorporating
\idx[hyper=false,noindex]{makeglossaries} or
\idx[hyper=false,noindex]{makeglossaries-lite} or \bibgls\ into
the document build}}~\cite{buildglossaries}.

\section{Manual Installation}

In general it's best to install \bibgls\ via your \TeX\ package manager.
However, if you are unable to do this or if you are testing a development version,
you can install manually using the instructions below.
Replace \meta{TEXMF} with the path to your local or home TEXMF tree 
(for example, \filefmt{\glstildechar/texmf}).

Copy the files provided to the following locations:
\begin{itemize}
\item \meta{TEXMF}\filefmt{/scripts/bib2gls/bib2gls.jar}
(Java application.)
\item \meta{TEXMF}\filefmt{/scripts/bib2gls/convertgls2bib.jar}
(Java application.)
\item \meta{TEXMF}\filefmt{/scripts/bib2gls/datatool2bib.jar}
(Java application.)
\item \meta{TEXMF}\filefmt{/scripts/bib2gls/bibglscommon.jar}
(Java library.)
\item \meta{TEXMF}\filefmt{/scripts/bib2gls/texparserlib.jar}
(Java library.)
\item \meta{TEXMF}\filefmt{/scripts/bib2gls/resources/bib2gls-en.xml}
(English resource file.)
\item \meta{TEXMF}\filefmt{/scripts/bib2gls/resources/bib2gls-extra-en.xml}
(Extra English resource file.)
\item \meta{TEXMF}\filefmt{/scripts/bib2gls/resources/bib2gls-extra-nl.xml}
(Extra Dutch resource file.)
\item \meta{TEXMF}\filefmt{/doc/support/bib2gls/bib2gls.pdf}
(This document.)
\item \meta{TEXMF}\filefmt{/doc/support/bib2gls/bib2gls-begin.pdf}
(Introductory guide.)
\end{itemize}
If you use the Unix \appfmt{man} command, copy the
\filefmt{bib2gls.1} and \filefmt{convertgls2bib.1} files to the
appropriate location.

If you are using a Unix-like system, there are also bash scripts
provided called \file{bib2gls.sh}, \file{convertgls2bib.sh} and \file{datatool2bib.sh}. 
Either copy them directly to somewhere on your path without the \iext{sh} 
extension, for example:
\begin{verbatim}
cp bib2gls.sh ~/bin/bib2gls
cp convertgls2bib.sh ~/bin/convertgls2bib
cp datatool2bib.sh ~/bin/datatool2bib
\end{verbatim}
or copy the files to
\meta{TEXMF}\filefmt{/scripts/bib2gls/} and create a
symbolic link to them called just \filefmt{bib2gls},
\filefmt{convertgls2bib} and \filefmt{datatool2bib} from somewhere on
your path, for example:
\begin{verbatim}
cp bib2gls.sh ~/texmf/scripts/bib2gls/
cp convertgls2bib.sh ~/texmf/scripts/bib2gls/
cp datatool2bib.sh ~/texmf/scripts/bib2gls/
cd ~/bin
ln -s ~/texmf/scripts/bib2gls/bib2gls.sh bib2gls
ln -s ~/texmf/scripts/bib2gls/convertgls2bib.sh convertgls2bib
ln -s ~/texmf/scripts/bib2gls/datatool2bib.sh datatool2bib
\end{verbatim}
The \file{texparserlib.jar} file isn't an application but is a
library used by both \file{bib2gls.jar} and
\file{convertgls2bib.jar}, and so needs to be in the same class path. 
(The library is in a separate
\href{https://github.com/nlct/texparser}{GitHub
repository}~\cite{texparser} as it's
also used by some of my other applications.)

Windows users can create a \iext{bat} file that works in a
similar way to the bash scripts. To do this, create a file called
\file{bib2gls.bat} that contains the following:
\begin{verbatim}
@ECHO OFF
FOR /F "tokens=*" %%I IN ('kpsewhich --progname=bib2gls --format=texmfscripts
bib2gls.jar') DO SET JARPATH=%%I
java -jar "%JARPATH%" %*
\end{verbatim}
Save this file to somewhere on your system's path.
(Similarly for \filefmt{convertgls2bib} and \filefmt{datatool2bib}.) Note that \TeX\
distributions for Windows usually convert \iext{jar} files to
executables.

You may need to refresh \TeX's database to ensure that
\idx{kpsewhich} can find the \iext{jar} files.
To test that the application has been successfully installed, open a
command prompt or terminal and run the following command:
\begin{verbatim}
bib2gls --version
convertgls2bib --version
datatool2bib --version
\end{verbatim}
This should display the version information for both applications.

\chapter{\texorpdfstring{\TeX}{TeX}\ Parser Library}
\label{sec:texparserlib}

The \bibgls\ application requires the \TeX\ Parser Library
\file{texparserlib.jar}\footnote{\url{https://github.com/nlct/texparser}}
which is used to parse the \iext{aux} and \iext{bib} files.

With the \longarg{interpret} switch on (default), this library is
also used to interpret the sort value when it contains a backslash
\idx{escchar} or a tilde \idx{nbspchar} or a dollar symbol 
\idx{mshiftchar} or braces \idx{bgroupchar} \idx{egroupchar}
(and when the \csopt{sort} option is not
\optfmt{unsrt} or \optfmt{none} or \optfmt{use}).\footnote{The other
special characters are omitted from the check: the comment symbol
\idx{commentchar} is best avoided in field values, the subscript and
superscript characters \idx{sbchar} and \idx{spchar} should
either be encapsulated by \idx{mshiftchar} or by \cs{ensuremath}, which
will be picked up by the check for \idx{mshiftchar} or
\idx{escchar}, and the other special characters would
indicate something too complex for the interpreter to handle.}

The other cases that the interpreter is used for are:
\begin{itemize}
\item when \csopt{set-widest} is used to determine the width of
the \field{name} field;
\item if \csopt{labelify} or \csopt{labelify-list} are set the
identified field values are first interpreted (if they contain
\idx{escchar} \idx{bgroupchar} \idx{egroupchar} 
\idx{nbspchar} or \idx{mshiftchar}) before being converted to labels;
\item if \csopt[true]{interpret-label-fields} is set and the
\field{parent}, \field{category}, \field{type}, \field{group},
\field{seealso} or \field{alias} fields contain \idx{escchar}
or \idx{bgroupchar} or \idx{egroupchar} the interpreter is used 
since these fields must be just a label
(other special characters aren't checked as they won't expand to
characters allowed in a label).
\end{itemize}

Information in the \ext{aux} file is parsed for specific commands
but the arguments of those commands are not interpreted so, for
example, UTF-8 characters that occur in any resource options will
need to be detokenized when using \sty{inputenc} to prevent
expansion when they are written to the \ext{aux} file. (In some
options, such as \csopt{sort-rule}, you can use \cs{glshex}\meta{hex}
syntax to specify a UTF-8 character.) Note that newer \LaTeX\
kernels have better support for UTF-8 and this issue is less likely
to occur.

The \longarg{no-interpret} switch will turn off the interpreter, but
the library will still be used to parse the \ext{aux} and \ext{bib}
files. Note that the \field{see} field doesn't use the interpreter
with \csopt[true]{interpret-label-fields} as
it may legitimately contain \LaTeX\ code in the optional tag part
(such as \cs{seealsoname} or \cs{alsoname}).

The parser has a different concept of expansion to \TeX\
and will expand some things that aren't expanded by \LaTeX\
(such as \ics{MakeUppercase} and \ics{char}) and won't expand other commands
that would be expanded by \LaTeX\ (such as commands defined
in terms of complicated internals).

If you get a \idx{StackOverflowError} while a field is being
interpreted (with a long stack trace that contains repeated file names and
line numbers) then it's likely you have an infinite \idx{loop}. For
example, this can be triggered if a field contains \csfmt{foo} that
has been defined as:
\begin{codeenv}
\cs{def}\cmd{foo}\marg{\cmd{foo}}
\end{codeenv}
This will obviously also cause an error in the \LaTeX\ document as
well (unless the document has a different definition that doesn't
have this unbounded recursion).

The \file{texparserlib.jar} library is not a \TeX\ engine and there
are plenty of situations where it doesn't work. In particular, with
\bibgls, it's being used in a fragmented context without knowing
most of the packages used by the document or any custom commands or
environments provided within the document.

\bibgls\ can detect from the log file a small
number of packages that the parser recognises. Note that in
some cases there's only very limited support. For example, 
\isty{siunitx}'s \ics{si} and \ics{unit} commands are recognised but other
commands from that package aren't. See
\longargpageref{list-known-packages} for further details.

Since the parser doesn't have a full set of commands available
within the \LaTeX\ document, when it encounters \ics{renewcommand} it
won't check if the command is undefined. If the command isn't
defined, it will simply behave like \cs{newcommand}. Whereas
with \ics{providecommand} the parser will only define the command
if it's unrecognised.

The interpreter has its own internal implementation of the
glossary-related commands listed in \tableref{tab:bibglsdefs}. These
may be overridden by custom packages provided with the
\longarg{custom-packages} switch. Note that commands that reference
an entry, such as \cs{glsentryname}, aren't guaranteed to work
across \iglspl{resourceset} and will only be able to look up field
values that are known to \bibgls. (For example, the \field{name}
field for abbreviations is typically set by the associated
abbreviation style, which isn't available to \bibgls.)

\begin{table}[htbp]
\caption{Glossary-Related Commands Implemented by the \bibgls\ Interpreter}
\label{tab:bibglsdefs}
\setlength{\tabcolsep}{4pt}%
\centering
\tablelistcs{3}{glsxtrprovidecommand,bibglsuppercase,bibglslowercase,bibglsfirstuc,%
bibglstitlecase,GlsXtrEnableInitialTagging,bibglscontributorlist,%
bibglshyperlink,glshyperlink,bibglscontributor,bibglsdatetime,%
bibglsdate,bibglstime,glsentryname,glsentrytext,glsentryshort,%
glsentrylong,glsentryfirst,glsentrysymbol,glsentryuseri,%
glsentryuserii,glsentryuseriii,glsentryuseriv,glsentryuserv,%
glsentryuservi,glsentryplural,glsentryfirstplural,glsentryshortpl,%
glsentrylongpl,glsentrysymbolplural,Glsentryname,Glsentrytext,%
Glsentryshort,Glsentrylong,Glsentryfirst,Glsentrysymbol,%
Glsentryuseri,Glsentryuserii,Glsentryuseriii,Glsentryuseriv,%
Glsentryuserv,Glsentryuservi,Glsentryplural,Glsentryfirstplural,Glsentryshortpl,%
Glsentrylongpl,Glsentrysymbolplural,bibglshashchar,bibglsunderscorechar,%
bibglsdollarchar,bibglsampersandchar,bibglscircumchar,glsbackslash,glstildechar,glsopenbrace,glsclosebrace,glspercentchar,%
glsxtrusefield,Glsxtrusefield,GLSxtrusefield,glsentrytitlecase,%
glsxtrhiernamesep,glsxtrhiername,Glsxtrhiername,GlsXtrhiername,%
GLSxtrhiername,GLSXTRhiername}
\end{table}

If a command isn't recognised, you can provide it in the
\atentry{preamble} and use \ics{char} to map a symbol to the most
appropriate Unicode character. For example, suppose your document
loads a package that provides symbols for use on maps, such as
\csfmt{Harbour}, \csfmt{Battlefield} and \csfmt{Stadium}, then you
can provide versions of these commands just for \bibgls's
use:\footnote{These commands won't work with PDF\LaTeX, as the
\cs{char} values are too large, but they're fine for \bibgls.}
\begin{codeenv}
\atentry{preamble}\marg{\qtdelim{\cs{providecommand}\marg{\cmd{Harbour}}\marg{\charhex{2693}}
\cs{providecommand}\marg{\cmd{Battlefield}}\marg{\charhex{2694}}
\cs{providecommand}\marg{\cmd{Stadium}}\marg{\charhex{26BD}}}}
\end{codeenv}
Since these use \cs{providecommand}, they won't overwrite the
document's version (provided these commands have been defined before
\gls!{GlsXtrLoadResources}). Alternatively, you can instruct \bibgls\
to not write the \atentry{preamble} contents to the resource file
using \csopt[false]{write-preamble}. Now you can either sort these
symbols by
their Unicode values (\csopt[letter-case]{sort}) or provide a custom
rule that recognises these Unicode characters (for example,
\csopt[custom]{sort},
\csopt[\cs{glshex}2694 \string< \cs{glshex}2693 \string< \cs{glshex}26BD]{sort-rule}).

Another approach is to use \ics{IfTeXParserLib}, which is defined by
the \texparserlib\ to expand to its first argument. The
\sty{glossaries-extra-bib2gls} package provides a definition that
expands to its second argument, so that command may be used to
provide alternative code. For example:
\begin{codeenv}
\atentry{preamble}\marg{\qtdelim{\cs{providecommand}\marg{\cmd{Ord}}[1]\marg{\comment{}
 \cs{IfTeXParserLib}
  \marg{\ics{bibglspaddigits}\marg{2}\marg{\idx{param}1}}\comment{interpreter}
  \marg{\cs{MakeUppercase}\marg{\cmd{romannumeral} \idx{param}1}}\comment{document}
}}}
\atentry{index}\marg{John-IV,
 \field{name}=\marg{John\idx{nbspchar}\cmd{Ord}\marg{4}}
}
\atentry{index}\marg{John-VI,
 \field{name}=\marg{John\idx{nbspchar}\cmd{Ord}\marg{6}}
}
\atentry{index}\marg{John-IX,
 \field{name}=\marg{John\idx{nbspchar}\cmd{Ord}\marg{9}}
}
\atentry{index}\marg{John-XII,
 \field{name}=\marg{John\idx{nbspchar}\cmd{Ord}\marg{12}}
}
\end{codeenv}
The sort values for these entries will be: \qt{John 04}, \qt{John
06}, \qt{John 09} and \qt{John 12}, but in the document text they
will be typeset as \qt{John~IV}, \qt{John~VI}, \qt{John~IX} and
\qt{John~XII}. Note that \ics{bibglspaddigits} is only recognised by
the \bibgls\ interpreter. Alternatively, you can use the
\csopt{sort-number-pad} option to pad the numbers (or use 
\ics{dtlpadleadingzeros} which is also recognised by the
\texparserlib\ and \sty{datatool-base} v3.0+).

There is a similar command with reversed syntax \ics{IfNotBibGls},
which is defined by \sty{glossaries-extra-bib2gls} to expand to its
first argument. The \bibgls\ interpreter defines this command to
expand to its second argument.

\TeX\ syntax can be quite complicated and, in some cases, far too
complicated for simple \idxpl!{regex}. The \texparserlib\ performs
better than a simple pattern match, and that's the purpose of
\file{texparserlib.jar} and why it's used by \bibgls\ (and by
\idx!{convertgls2bib}). When the \commonlongarg{debug} mode is on, any
warnings or errors triggered by the interpreter will be written to
the transcript prefixed with \code{texparserlib:} (the results of
the conversions will be included in the transcript as informational
messages prefixed with \code{texparserlib:} even with
\commonlongarg{no-debug}).

For example, suppose the \ext{bib} file includes:
\begin{codeenv}
\atentry{preamble}\marg{
\qtdelim{\cs{providecommand}\marg{\cmd{mtx}}[1]\marg{\cs{boldsymbol}\marg{\idx{param}1}}
\cs{providecommand}\marg{\cmd{set}}[1]\marg{\ics{mathcal}\marg{\idx{param}1}}
\cs{providecommand}\marg{\cmd{card}}[1]\marg{|\cmd{set}\marg{\idx{param}1}|}
\cs{providecommand}\marg{\cmd{imaginary}}\marg{i}}}
\strut
\atentry{entry}\marg{M,
  \field{name}=\marg{\marg{}\idx{mshiftchar}\cmd{mtx}\marg{M}\idx{mshiftchar}},
  \field{text}=\marg{\cmd{mtx}\marg{M}},
  \field{description}=\marg{a matrix}
}
\strut
\atentry{entry}\marg{v,
  \field{name}=\marg{\marg{}\idx{mshiftchar}\cs{vec}\marg{v}\idx{mshiftchar}},
  \field{text}=\marg{\cs{vec}\marg{v}},
  \field{description}=\marg{a vector}
}
\strut
\atentry{entry}\marg{S,
  \field{name}=\marg{\marg{}\idx{mshiftchar}\cmd{set}\marg{S}\idx{mshiftchar}},
  \field{text}=\marg{\cmd{set}\marg{S}},
  \field{description}=\marg{a set}
}
\strut
\atentry{entry}\marg{card,
  \field{name}=\marg{\marg{}\idx{mshiftchar}\cmd{card}\marg{S}\idx{mshiftchar}},
  \field{text}=\marg{\cmd{card}\marg{S}},
  \field{description}=\marg{the cardinality of the set \idx{mshiftchar}\cmd{set}\marg{S}\idx{mshiftchar}}
}
\strut
\atentry{entry}\marg{i,
  \field{name}=\marg{\marg{}\idx{mshiftchar}\cmd{imaginary}\idx{mshiftchar}},
  \field{text}=\marg{\cmd{imaginary}},
  \field{description}=\marg{square root of minus one (\idx{mshiftchar}\cmd{sqrt}\marg{-1}\idx{mshiftchar})}
}
\end{codeenv}
(The empty group at the start of the \field{name} fields
protects against the possibility that the \catattr{glossname}
category attribute might be set to \optfmt{firstuc}, which
automatically converts the first letter of the name to 
\idx{uppercase} when displaying the glossary. See also
\longarg{mfirstuc-protection} and
\longarg{mfirstuc-math-protection}.)

None of these entries have a \field{sort} field so the \field{name}
is used (see \sectionref{sec:fallbacks}). If the \gls{entrytype} had
been \atentry{symbol} instead, the fallback would be the entry's
label.  This means that with \atentry{symbol} instead of
\atentry{entry}, and the default \csopt[sort]{sort-field}, and with
\csopt[letter-case]{sort}, these entries will be defined in the
order: \code{M}, \code{S}, \code{card}, \code{i}, \code{v} (since
this is the case-sensitive letter order of the labels) whereas with
\csopt[letter-nocase]{sort-field}, the order will be: \code{card},
\code{i}, \code{M}, \code{S}, \code{v} (since this is the
case-insensitive letter order of the labels).

However, with \atentry{entry}, the fallback field will be taken from
the \field{name} which in the above example contains \TeX\ code, so
\bibgls\ will use \file{texparserlib.jar} to interpret this code.
The library has several different ways of writing the processed
code. For simplicity, \bibgls\ uses the library's HTML output and
then strips the HTML markup and trims any leading or trailing
spaces. The library method that writes \idx{non-ASCII} characters
using \qtt{\&x\meta{hex};} markup is overridden by \bibgls\ to just
write the actual Unicode character, which means that the letter-based
sorting options will sort according to the integer value \meta{hex}
rather than the string \qtt{\&x\meta{hex};}.

The interpreter is first passed the code provided with
\atentry{preamble}:
\begin{codeenv}
\cs{providecommand}\marg{\cmd{mtx}}[1]\marg{\cs{boldsymbol}\marg{\idx{param}1}}
\cs{providecommand}\marg{\cmd{set}}[1]\marg{\cs{mathcal}\marg{\idx{param}1}}
\cs{providecommand}\marg{\cmd{card}}[1]\marg{|\cmd{set}\marg{\idx{param}1}|}
\cs{providecommand}\marg{\cmd{imaginary}}\marg{i}
\end{codeenv}
(unless \csopt[false]{interpret-preamble}).
This means that the provided commands are now recognised by the
interpreter when it has to parse the fields later.

In the case of the \code{M} entry in the example above, the 
code that's passed to the interpreter is:
\begin{verbatim}
{}$\mtx{M}$
\end{verbatim}
The transcript (\iext{glg}) file will show the results of the
conversion:
\begin{verbatim}
texparserlib: {}$\mtx{M}$ -> M
\end{verbatim}
So the \field{sort} value for this entry is set to \qtt{M}. The font
change (caused by math-mode and \ics{boldsymbol}) has been 
ignored. The sort value therefore consists of a single Unicode
character \hex{4D} (Latin \idx!{uppercase} letter \qtt{M}, decimal value 77).

For the \code{v} entry, the code is:
\begin{verbatim}
{}$\vec{v}$
\end{verbatim}
The transcript shows:
\begin{alltt}
texparserlib: \marg{}\$\ics{vec}\marg{v}\$ -> \usebox\varrow
\end{alltt}
So the \field{sort} value for this entry is set to \qtt{\usebox\varrow},
which consists of two Unicode characters \hex{76}
(Latin \idx!{lowercase} letter \qtt{v}, decimal value 118) and 
\hex{20D7} (combining right arrow above, decimal value 8407).

For the \code{set} entry, the code is:
\begin{verbatim}
{}$\set{S}$
\end{verbatim}
The transcript shows:
\begin{verbatim}
texparserlib: {}$\set{S}$ -> S
\end{verbatim}
So the \field{sort} value for this entry is set to \qtt{S} (again
ignoring the font change). This consists of a single Unicode
character \hex{53} (Latin \idx!{uppercase} letter \qtt{S}, decimal value~83).

For the \code{card} entry, the code is:
\begin{verbatim}
{}$\card{S}$
\end{verbatim}
The transcript shows:
\begin{verbatim}
texparserlib: {}$\card{S}$ -> |S|
\end{verbatim}
So the \field{sort} value for this entry is set to \qtt{|S|}
(the \textbar\ characters from the definition of \csfmt{card} provided
in \atentry{preamble} have been included, but the
font change has been discarded). In this case the sort value
consists of three Unicode characters \hex{7C} (vertical line, decimal
value 124),
\hex{53} (Latin \idx!{uppercase} letter \qtt{S}, decimal value 83) and
\hex{7C} again.
If \csopt[false]{interpret-preamble} had been used, \csfmt{card}
wouldn't be recognised and would be discarded leaving just \qtt{S}
as the sort value.

(Note that if \ics{vert} is used instead of \textbar\ then it would
be converted into the mathematical operator \hex{2223} and result in
a different order.)

For the \code{i} entry, the code is:
\begin{verbatim}
{}$\imaginary$
\end{verbatim}
The transcript shows:
\begin{verbatim}
texparserlib: {}$\imaginary$ -> i
\end{verbatim}
So the \field{sort} value for this entry is set to \qtt{i}.
If \csopt[false]{interpret-preamble} had been used, \csfmt{imaginary}
wouldn't be recognised and would be discarded, leaving an empty sort
value.

This means that in the case of the default \csopt[sort]{sort-field} with
\csopt[letter-case]{sort}, these entries will be defined in
the order: \code{M} ($\mtx{M}$), \code{S} ($\set{S}$),
\code{i} ($\imaginary$), \code{v} ($\vec{v}$) and
\code{card} ($\card{S}$).
In this case, the entries have been sorted according to
the character codes. If you run \bibgls\ with \commonlongarg{verbose}
the decimal character codes will be included in the transcript.
For this example:
\begin{alltt}
i -> 'i' [105]
card -> '|S|' [124 83 124]
M -> 'M' [77]
S -> 'S' [83]
v -> '\usebox\varrow' [118 8407]
\end{alltt}

The \longarg{group} option (in addition to \commonlongarg{verbose}) 
will place the letter group in 
parentheses before the character code list:
\begin{alltt}
i -> 'i' (i) [105]
card -> '|S|' [124 83 124]
M -> 'M' (M) [77]
S -> 'S' (S) [83]
v -> '\usebox\varrow' (v) [118 8407]
\end{alltt}
(Note that the \code{card} entry doesn't have a letter
group since the vertical bar character isn't considered a 
letter.)

If \csopt[letter-nocase]{sort} is used instead then, after conversion
by the interpreter, the sort values will all be changed to
\idx{lowercase}. The order is now: \code{i} ($\imaginary$), 
\code{M} ($\mtx{M}$), \code{S} ($\set{S}$),
\code{v} ($\vec{v}$) and \code{card} ($\card{S}$).
The transcript (with \commonlongarg{verbose}) now shows
\begin{alltt}
i -> 'i' [105]
card -> '|s|' [124 115 124]
M -> 'm' [109]
S -> 's' [115]
v -> '\usebox\varrow' [118 8407]
\end{alltt}
With \longarg{group} (in addition to \commonlongarg{verbose}) 
the letter groups are again included:
\begin{alltt}
i -> 'i' (I) [105]
card -> '|s|' [124 115 124]
M -> 'm' (M) [109]
S -> 's' (S) [115]
v -> '\usebox\varrow' (V) [118 8407]
\end{alltt}
Note that the letter groups are \idx{uppercase} not \idx{lowercase}.
Again the \code{card} entry doesn't have an associated
letter group.

If a locale-based sort is used, the ordering will follow the
locale's alphabet rules. For example, with \csopt[en]{sort}
(English, no region or variant), the order becomes:
\code{card} ($\card{S}$), \code{i} ($\imaginary$), 
\code{M} ($\mtx{M}$), \code{S} ($\set{S}$) and
\code{v} ($\vec{v}$). The transcript (with \commonlongarg{verbose}) 
shows the collation keys instead:
\begin{alltt}
i -> 'i' [0 92 0 0 0 0]
card -> '|S|' [0 66 0 102 0 66 0 0 0 0]
M -> 'M' [0 96 0 0 0 0]
S -> 'S' [0 102 0 0 0 0]
v -> '\usebox\varrow' [0 105 0 0 0 0]
\end{alltt}
Again the addition of the \longarg{group} switch will show
the letter groups.\footnote{For more information on collation keys see the
\href{http://docs.oracle.com/javase/8/docs/api/java/text/CollationKey.html}{CollationKey}
class in Java's API~\cite{collationkey}.}

Suppose I add a new symbol to my \ext{bib} file:
\begin{codeenv}
\atentry{symbol}\marg{angstrom,
  \field{name}=\marg{\ics{AA}},
  \field{description}=\marg{\cs{AA} ngstr\ics{umlaut}om}
}
\end{codeenv}
and I also use this entry in the document.\footnote{A better method
is to use \sty{siunitx} instead.} Then with
\csopt[en]{sort}, the order is: \code{card} ($\card{S}$),
\code{angstrom} (\AA), \code{i} ($\imaginary$), \code{M}
($\mtx{M}$), \code{S} ($\set{S}$), and \code{v} ($\vec{v}$).
The \longarg{group} switch shows that the \code{angstrom} entry
(\AA) has been placed in the \qt{A} letter group.

However, if I change the locale to \csopt[sv]{sort}, the
\code{angstrom} entry is moved to the end of the list and the
\longarg{group} switch shows that it's been placed in the \qt{\AA}
letter group.

\label{locale.provider}If you are using Java~8, you can set the
\code{java.locale.providers} property~\cite{javacldr} to use
the \idx{CLDR} \idx{localeprovider}, which has more extensive
support for locales than the native \idx{JRE}. For example:
\begin{verbatim}
java.locale.providers=CLDR,JRE
\end{verbatim}
This should be enabled by default for Java~9. The property
can either be set in a script that runs \bibgls, for example,
\begin{verbatim}
java -Djava.locale.providers=CLDR,JRE,SPI -jar "$jarpath" "$@"
\end{verbatim}
(where \verb|$jarpath| is the path to the \file{bib2gls.jar}
file and \verb|"$@"| is the argument list) or you can set the
property as the default for all Java applications by adding
the definition to the \code{JAVA\_TOOL\_OPTIONS} environment
variable~\cite{javaoptions}.
For example, in a bash shell:
\begin{verbatim}
export JAVA_TOOL_OPTIONS='-Djava.locale.providers=CLDR,JRE,SPI'
\end{verbatim}
or in Windows:
\begin{verbatim}
set JAVA_TOOL_OPTIONS=-Djava.locale.providers=CLDR,JRE,SPI
\end{verbatim}
Note that newer versions of Java support \idx{CLDR} by 
default, and the \idx{JRE} synonym for the \code{COMPAT} provider is now 
deprecated.\footnote{https://www.oracle.com/java/technologies/javase/jdk21-suported-locales.html}

\chapter{Command Line Options}
\label{sec:switches}
\setsecdepth{1}

The syntax of \bibgls\ is:
\begin{alltt}
bib2gls \oargm{options} \meta{filename}
\end{alltt}
where \meta{filename} is the name of the \ext{aux} file. (The
extension may be omitted.) Only one \meta{filename} is permitted.
Available options are listed below.

If you are using an automated build system that makes it difficult
to change the command line options and you have at least version
1.54 of \sty{glossaries-extra} and at least version 4.0 of \bibgls,
then most (but not all) switches can be specified within the
argument of 
\nosecformatdef{BibGlsOptions}
This command may be placed anywhere within the preamble, but the
options will always be processed before the resource commands. This
command may be used multiple times.
Unlike the resource options, which are local to the resource set, the options 
described here are global and are applied to all resource sets,
where applicable.
For example:
\begin{codeenv}
\gls{BibGlsOptions}\marg{replace-quotes=true}
\gls{GlsXtrLoadResources}
\gls{BibGlsOptions}\marg{collapse-same-location-range=true}
\end{codeenv}
This is equivalent to:
\begin{codeenv}
\gls{BibGlsOptions}\marg{replace-quotes=true,collapse-same-location-range=true}
\gls{GlsXtrLoadResources}
\end{codeenv}
The \meta{options} list should be a \keyvallist\ where the \meta{key}
is the same as the long switch without the preceding
\code{\longswitch} and any \code{\longswitch no-\meta{option}} should 
be specified as \code{\meta{option}=false} within
\gls{BibGlsOptions}.
For example, to set global options via the command line:
\begin{verbatim}
bib2gls --group --no-replace-quotes myDoc
\end{verbatim}
Alternatively, the document may contain:
\begin{codeenv}
\gls{BibGlsOptions}\marg{group=true,replace-quotes=false}
\end{codeenv}
You can omit the value if it is \code{true}, so the above can also
be written:
\begin{codeenv}
\gls{BibGlsOptions}\marg{group,replace-quotes=false}
\end{codeenv}
 
\begin{important}
Options that must be set before the \ext{log} and \ext{aux} file are read can
only be set via the command line switch.
\end{important}

\section{Common Options}
\label{sec:commonswitches}

These command line switches are common to \bibgls\ and the
supplementary command line conversion tools

\commonarg{help}

Display the help message and quit.
This option cannot be set in \gls{BibGlsOptions}.

\commonarg{version}

Display the version information and quit. As from v2.5, this now
includes the version number of the \file{texparserlib.jar} library.
This option cannot be set in \gls{BibGlsOptions}.

\commonarg{verbose}

Switches on the verbose mode. This writes extra information to the
terminal and transcript file.
This option cannot be set in \gls{BibGlsOptions}.

\commonarg{no-verbose}

Switches off the verbose mode. This is the default behaviour.
Some messages are written to the terminal. To completely suppress
all messages (except errors), switch on the silent mode.
For additional information messages, switch on the verbose mode.
This option cannot be set in \gls{BibGlsOptions}.

\commonarg{quiet}

Suppresses all messages except for errors that would normally be
written to the terminal. Warnings and informational messages are
written to the transcript file, which can be inspected afterwards.
This option cannot be set in \gls{BibGlsOptions}.

\commonarg{silent}

Synonym of \commonlongarg{quiet}.
This option cannot be set in \gls{BibGlsOptions}.

\commonarg{locale}

Specify the preferred \langxml, where \meta{lang} is a valid \idx{IETF} language tag.
This option requires an appropriate \metafilefmt{bib2gls-}{lang}{.xml}
resource file otherwise \bibgls\ will fallback on English.
This also sets the default \gls{document-locale} when the \optfmt{doc}
keyword (in options such as \csopt[doc]{sort}) is
used and the document doesn't have any language support.
Note that the \optfmt{locale} keyword (in options such as \csopt[locale]{sort})
uses the \gls{Java-locale} and is not governed by this switch.
This option cannot be set in \gls{BibGlsOptions}.

If a document doesn't have any locale support or has support
for more than one language then it's best to explicitly set
the required locale in the appropriate \idx{resourceset} using
the \csopt{locale} resource option, to specify the default
\gls{resource-locale}, or set the locale for individual options,
such as \csopt{sort}.

\commonarg{debug}

Sometimes when things go wrong it can be hard to diagnose the problem
from the normal messages. If you report an issue, you may be asked
to switch on debugging mode to help identify a non-reproducible
error and provide the transcript file.

The \commonlongarg{debug} optional value can be used to adjust the level of
debugging information.  If \meta{n} is present, it must be a non-negative
integer indicating the debugging mode. If omitted, 1 is assumed. This option
also switches on the verbose mode. A value of 0 is equivalent to
\commonlongarg{no-debug}.  This option cannot be set in \gls{BibGlsOptions}.

The value of \meta{n} determines how much extra information is
provided. If \meta{n} is greater than 0 then all \bibgls\ debugging
information is written. The amount of debugging information provided
by the \texparserlib\ is determined by a bitwise operation on
\meta{n}. For example, if \meta{n} is 1 then I/O information is
included. If \meta{n} is 2 then information is included when an
object is popped off a stack. If \meta{n} is 3 then both I/O and
popped information is provided.

Note that messages such as \qt{Can't find language resource} or
about a failed \idx{kpsewhich} call are informational and don't
necessarily mean an error has occurred. Error messages will always
be written to the transcript regardless of the debug or verbose
setting. An error message will start with \qt{Error:~} and a warning
message will start with \qt{Warning:~}. Unknown commands will throw
an exception with a stack trace in debug mode.

\commonarg{debug-mode}

This option is an alternative to \commonlongarg{debug} where the value
of \meta{n} needs to be calculated. 
This option cannot be set in \gls{BibGlsOptions}.
Debugging mode requires a
transcript file, which is automatically created with \bibgls, but is
optional for the converter tools (see \sectionref{sec:tex2bib}). The
\meta{setting} is required and should be a comma-separated list of
any of the following keywords.

\begin{itemize}
\item\optfmt{all}: enable all debugging information
(likely to result in a very large transcript file).

\item\optfmt{catcode}: \texparserlib\ category code changes.

\item\optfmt{cs}: \texparserlib\ command definitions.

\item\optfmt{decl}: information about declarations.

\item\optfmt{expansion}: \texparserlib\ expansions 
(may result in a large transcript file).

\item\optfmt{expansion-list}: \texparserlib\ stack expansions 
(may result in a large transcript file).

\item\optfmt{expansion-once}: \texparserlib\ one-level expansions 
(may result in a large transcript file).

\item\optfmt{expansion-once-list}: \texparserlib\ one-level list expansions 
(may result in a large transcript file).

\item\optfmt{io}: I/O information, such as opening or closing files
and fetching tokens.

\item\optfmt{popped}: information about objects popped from stacks.

\item\optfmt{process}: \texparserlib\ macro process 
(may result in a large transcript file).

\item\optfmt{process-generic-cs}: \texparserlib\ generic command process.

\item\optfmt{process-stack}: \texparserlib\ stack process 
(may result in a large transcript file).

\item\optfmt{process-stack-list}: \texparserlib\ stack process with
list detail (may result in a large transcript file).

\item\optfmt{read}: \texparserlib\ file codepoint read
(likely to result in a very large transcript file).

\item\optfmt{settings}: \texparserlib\ settings information.

\item\optfmt{sty-data}: data associated with packages used to store
information that may not exactly correspond to the way the
information is stored in \LaTeX. In the case of \bibgls, this will
typically just be data read from recognised \iext{aux} commands.

\end{itemize}

For example:
\begin{codeenv}
bib2gls \commonlongarg{debug-mode} catcode,sty-data \meta{filename}
\end{codeenv}

\commonarg{no-debug}

Switches off the debugging mode.
This option cannot be set in \gls{BibGlsOptions}.

\section{File Options}

\argsection{dir}

This option cannot be set in \gls{BibGlsOptions}.

By default \bibgls\ assumes that the output files should be written
in the current working directory. The input \iext{bib} files are assumed to be
either in the current working directory or on \TeX's path (in which
case \idx{kpsewhich} will be used to find them).

If your \iext{aux} file isn't in the current working directory (for
example, you have run \TeX\ with \shortargfmt{output-directory})
then you need to take care how you invoke \bibgls.

Suppose I have a file called \filefmt{test-entries.bib} that
contains my entry definitions and a document called
\filefmt{mydoc.tex} that selects the \ext{bib} file using:
\begin{codeenv}
\gls{GlsXtrLoadResources}\oarg{\csopt[test-entries]{src}}
\end{codeenv}
(\filefmt{test-entries.bib} is in the same directory as
\filefmt{mydoc.tex}).
If I compile this document using
\begin{verbatim}
pdflatex -output-directory tmp mydoc
\end{verbatim}
then the auxiliary file \filefmt{mydoc.aux} will be written to the
\filefmt{tmp} sub-directory. The resource information is listed in
the \ext{aux} file as
\begin{codeenv}
\gls{glsxtr@resource}\marg{\csopt[test-entries]{src}}\marg{mydoc}
\end{codeenv}
If I run \bibgls\ from the \filefmt{tmp} directory, then it won't
be able to find the \filefmt{test-entries.bib} file (since it's in
the parent directory).

If I run \bibgls\ from the same directory as \filefmt{mydoc.tex}
using
\begin{verbatim}
bib2gls tmp/mydoc
\end{verbatim}
then the \ext{aux} file is found and the transcript file is
\filefmt{tmp/mydoc.glg} (since the default path name is the same as the
\ext{aux} file but with the extension changed to \iext{glg}) but the
output file \filefmt{mydoc.glstex} will be written to the current
directory.

This works fine from \TeX's point of view as it can find the
\iext{glstex} file, but it may be that you'd rather the \ext{glstex}
file was tidied away into the \filefmt{tmp} directory along with all
the other files.  In this case you need to invoke \bibgls\ with the
\longargorshort{dir} option:
\begin{verbatim}
bib2gls -d tmp mydoc
\end{verbatim}

\argsection{log-file}

Sets the name of the \bibgls\ transcript file. By default, the name is the
same as the \iext{aux} file but with a \iext{glg} extension. Note that
if you use \bibgls\ in combination with \idx{xindy} or
\idx{makeindex}, you will need to change the transcript file name to
prevent conflict.
This option cannot be set in \gls{BibGlsOptions}.

The transcript file encoding is governed by
\longarg{log-encoding}.

\argsection{tex-encoding}

This option cannot be set in \gls{BibGlsOptions}.

In general, it's best to have all your files (\ext{aux}, \ext{bib}
and \iext{glstex}) in the same \igls{encoding} that matches your
default encoding (see \sectionref{sec:defencoding}). However, if
your \iext{aux} and \iext{glstex} files have a different encoding to
your default, you can use \longarg{tex-encoding} to specify the
\TeX\ encoding. If omitted the default encoding is used. See
\sectionref{sec:defencoding}.

Note that \bibgls\ will try to detect the document encoding from the
\ext{aux} file  to ensure that the \iext{glstex} files match it.
However, at that point, it's too late to establish the encoding of
the \ext{aux} file, which has already been opened. So if the
\ext{aux} file encoding doesn't match the default encoding, you can
specify the correct encoding to use with \longarg{tex-encoding}.

If you are using \sty{fontspec}, \bibgls\ can detect this
from the \ext{log} file instead and will assume UTF-8.

\argsection{log-encoding}

The \igls{encoding} of the \iext{log} file. If omitted, the default
encoding will be used.  See \sectionref{sec:defencoding}.  (Note
that the \ext{log} file may not have the same encoding as the
\ext{tex} file~\cite{tex.sx.2013}.)
This option cannot be set in \gls{BibGlsOptions}.

\argsection{default-encoding}

The default \igls{encoding} used by \bibgls\ to read and write files is
governed by the \idx{JVM}. This typically matches your operating
system's default encoding. If this is incorrect, you can either
globally change the encoding for the \idx{JVM}, which will affect
all Java applications installed on your device, or you can use 
\longarg{default-encoding} just to set the default for \bibgls.
See \sectionref{sec:defencoding}.
This option cannot be set in \gls{BibGlsOptions}.

\argsection{date-in-header}

The comment header block at the start of the \iext{glstex} files
will include the file modification date in the first line (after the
version information). This setting can interfere with
the document build process or version control if you are testing for
file differences rather than file modification dates when only the
timestamp changes.

\argsection{no-date-in-header}

The comment header block at the start of the \iext{glstex} files
won't include the file modification date (default).
If used in \gls{BibGlsOptions}, this option should be specified as
\begin{codeenv}
\gls{BibGlsOptions}\marg{date-in-header=false}
\end{codeenv}

\section{Interpreter Options}

\argsection{break-space}

The interpreter treats a tilde character \idx{nbspchar} as a normal
space. Similarly \ics{nobreakspace} just produces a space.

\argsection{no-break-space}

The interpreter treats a tilde character \idx{nbspchar} as a non-breakable
space (default). Similarly the interpreter will define
\ics{nobreakspace} to produce a non-breakable space character (\hex{00A0}).
If used in \gls{BibGlsOptions}, this option should be specified as
\begin{codeenv}
\gls{BibGlsOptions}\marg{break-space=false}
\end{codeenv}

\argsection{custom-packages}

Instruct the interpreter to parse the package files identified in
\meta{list}. The package files need to be quite simple. When this
switch is used, the interpreter can recognise \ics{ProvidesPackage},
\ics{DeclareOption} (and \ics{DeclareOption*}),
\ics{ProcessOptions}, \ics{PackageError} and \ics{RequirePackage},
but it can't deal with complicated code.  In the case of
\ics{RequirePackage}, support will also be governed by
\longarg{custom-packages}.  This option has a cumulative action.

Multiple instances of this switch can occur on the command line.
If used in \gls{BibGlsOptions}, the nature of the \keyvallist\
parser means that multiple instances within the same option list
will override each other. Instead, you will need a comma-separated
list as the argument.
For example, from the command line:
\begin{verbatim}
bib2gls --custom-packages 'pkg1,pkg2,pkg3' myDoc
\end{verbatim}
This is equivalent to:
\begin{verbatim}
bib2gls --custom-packages pkg1 --custom-packages pkg2 --custom-packages pkg3 myDoc
\end{verbatim}
Alternatively, within the document:
\begin{codeenv}
\gls{BibGlsOptions}\marg{custom-packages=\marg{pkg1,pkg2,pkg3}}
\end{codeenv}

\argsection{datatool-sort-markers}

The \sty{datatool-base} package provides some marker commands
designed for use with \ics{DTLsortwordlist}: \ics{datatoolasciistart},
\ics{datatoolpersoncomma}, \ics{datatoolplacecomma},
\ics{datatoolsubjectcomma}, \ics{datatoolparenstart},
\ics{datatoolctrlboundary}, \ics{datatoolasciiend}, and
\ics{datatoolparen}. These commands by default will be defined by the
interpreter to match their normal \sty{datatool-base} behaviour  (see the
\sty{datatool} documentation for further details). Additionally,
\ics{dtltexorsort} will be redefined to expand to its \emph{first} argument.

Note that you don't need to request the \sty{datatool-base} package,
unless you require support for other commands provided by that
package.

The \longarg{datatool-sort-markers} switch will instead define these
commands to match their localised definitions within
\cs{DTLsortwordlist}. This means that \ics{dtltexorsort} will be
defined to expand to its \emph{second} argument and the marker
commands will expand to content that includes special control codes.
Note that language-sensitive sort methods typically ignore control
codes, so these would either need to be used with a character-code
comparator or a custom sort method would need to be used.
For example:
\begin{codeenv}
\csopt[\ics{glsxtrcontrolIrules}
 ;\cs{glshex} 0
 ;\cs{glsxtrspacerules}
 ;\cs{glsxtrnonprintablerules}
 ;\cs{glsxtrcombiningdiacriticrules}
 ,\cs{glsxtrhyphenrules}
 <\ics{glsxtrcontrolIIrules}
 <\cs{glsxtrgeneralpuncrules}
 <\cs{glsxtrdigitrules}
 <\cs{glsxtrfractionrules}
 <\ics{glsxtrGeneralLatinIrules}
 <\cs{glshex} 7F
]{sort-rule}
\end{codeenv}

Instead of \ics{dtltexorsort} (which varies according to this setting), you may
prefer to use \ics{IfTeXParserLib} or \ics{IfNotBibGls}.

\argsection{no-datatool-sort-markers}

Define the \sty{datatool-base} marker commands 
(including \cs{dtltexorsort}) to match their normal definition.

\argsection{ignore-packages}

This option is cumulative. When the document \iext{log} file is
parsed for known packages, \bibgls\ will skip the check for any
listed in \meta{list}. Note that this option simply instructs
\bibgls\ to ignore the package information in the log file. Any packages
that are identified with \longarg{packages} will be passed to the
interpreter if support is available, even if the package is also
listed in \longarg{ignore-packages}.  Note that
unknown packages can't be included in the ignored \meta{list}.
This option cannot be set in \gls{BibGlsOptions}.

\argsection{interpret}

Switch on the interpreter mode (default). See \sectionref{sec:texparserlib}
for more details.

\argsection{no-interpret}

Switch off the interpreter mode. See \sectionref{sec:texparserlib}
for more details about the interpreter. 
If used in \gls{BibGlsOptions}, this option should be specified as
\begin{codeenv}
\gls{BibGlsOptions}\marg{interpret=false}
\end{codeenv}

\argsection{list-known-packages}

This option will list all the packages supported by the \texparserlib\
and will then exit \bibgls. 
This option cannot be set in \gls{BibGlsOptions}.

The results are divided into two
sections: those packages that are searched for in the \iext{log}
file and those packages that aren't searched for in the \iext{log}
file but have some support available. Some of the support is very
limited. Package options aren't detected.
The transcript file is always searched for \isty{glossaries-extra}
to ensure that the version is new enough to support \bibgls. 

Packages that fall into the first category are:
\isty{amsmath}, \isty{amssymb}, \isty{bpchem}, \isty{fontenc},
\isty{fontspec}, \isty{fourier}, 
\isty{hyperref}, \isty{lipsum}, \isty{MnSymbol}, \isty{mhchem}, \isty{natbib},
\isty{pifont}, \isty{siunitx} (limited), \isty{stix},
\isty{textcase}, \isty{textcomp}, \isty{tipa}, \isty{upgreek} and
\isty{wasysym}.  (You can omit checking for specific packages with
\longarg{ignore-packages}.) These are packages that provide commands
that might be needed within entry fields. The check for
\isty{fontspec} is to simply determine whether or not UTF-8
characters are allowed in labels (for \csopt{labelify} and 
\csopt{labelify-list}). (Now that there is better support for UTF-8
with \pdfLaTeX, UTF-8 characters will be allowed in labels if the
detected versions of \isty{glossaries} and \isty{glossaries-extra}
are new enough, but note that you will also need a relatively new
\LaTeX\ kernel as well.)

Packages that fall into the second category are:
\isty{booktabs}, \isty{color}, \isty{datatool-base} (very limited),
\isty{datatool} (very limited), \isty{etoolbox} (very limited), \isty{graphics},
\isty{graphicx}, \isty{ifthen}, \isty{jmlrutils},
\isty{mfirstuc-english}, \isty{probsoln}, \isty{shortvrb},
and \isty{xspace}. These are less likely to be needed within fields
and so aren't checked for by default. If they are needed then you
can instruct \bibgls\ to support them with \longarg{packages}.

Note that \sty{mfirstuc} is always automatically loaded, but
\sty{mfirstuc-english} is not implemented unless explicitly
requested with \code{\longarg{packages} mfirstuc-english}.

If you're wondering about the selection, the \file{texparserlib.jar}
library was originally written for another application that required
support for some of them.

\argsection{packages}

Instruct the interpreter to assume the packages listed 
in \meta{list} have been used by the document.
This option has a cumulative action so \code{\longarg{packages}
"wasysym,pifont"} is the same as \code{\longarg{packages} wasysym 
\longarg{packages} pifont}.  

This option may also be used in \gls{BibGlsOptions} but, as with 
\longarg{custom-packages}, multiple instances of the \code{packages}
key in the same option list will override each other.

Note that there's only a limited number of packages supported by the
\texparserlib. This option is provided for cases where you're
using a command from a package that the interpreter doesn't support
but it happens to have the same name and meaning as a command from
a package that the interpreter does support. You can also use it to
provide support for known packages that aren't checked for when the
\ext{log} file is parsed. If you want \bibgls\ to parse an
unsupported package use \longarg{custom-packages}.

\argsection{support-unicode-script}

Text superscript (\ics{textsuperscript}) and subscript
(\ics{textsubscript}) will use Unicode super/subscript characters
if available (default). For example,
\begin{codeenv}
\cs{textsuperscript}\marg{(2)}
\end{codeenv}
will be converted to \code{\textsuperscript{(2)}}, which consists
of: \hex{207D} (superscript left parenthesis)
\hex{00B2} (superscript two) \hex{207E} (superscript right
parenthesis). If the entire contents of the argument can't be
represented by Unicode characters, the interpreter uses \verb|<sup>|
and \verb|<sub>| markup, which is then stripped by \bibgls. For
example,
\begin{codeenv}
\cs{textsuperscript}\marg{(2,3)}
\end{codeenv}
will be converted to
\begin{verbatim}
<sup>(2,3)</sup>
\end{verbatim}
(since there's no superscript comma). The markup is stripped leaving
just \code{(2,3)}.

Superscripts and subscripts in maths mode always use markup
regardless of this setting. Some supported packages that use
\idx{spchar} or \idx{sbchar} as shortcuts within an encapsulating
command may internally use the same code as \cs{textsuperscript} and
\cs{textsubscript}, in which case they will be sensitive to this
setting.

\argsection{no-support-unicode-script}

Text superscript (\cs{textsuperscript}) and subscript
(\cs{textsubscript}) won't use Unicode super/subscript characters.
Note that if other commands are provided that expand to Unicode
superscript or subscript characters, then they won't be affected by
this setting. For example, if \csfmt{superiortwo} is defined as
\begin{codeenv}
\cs{providecommand}\marg{\cmd{superiortwo}}\marg{\charhex{B2}}
\end{codeenv}
then it will be interpreted as \hex{00B2} (superscript two) even if
this setting is on.

If used in \gls{BibGlsOptions}, this option should be specified as
\begin{codeenv}
\gls{BibGlsOptions}\marg{support-unicode-script=false}
\end{codeenv}

\argsection{obey-aux-catcode}

By default, the \ext{aux} parser ignores category code changing
commands. This option will instruct the parser to implement the
category code, but note that it can only do this for known commands
that the parser is able to implement.
This option cannot be set in \gls{BibGlsOptions}.

\argsection{no-obey-aux-catcode}

Instructs the \ext{aux} parser to ignore category code changing
commands. (Default.)
This option cannot be set in \gls{BibGlsOptions}.

\section{Record Options}

\argsection{cite-as-record}

Treat instances of \code{\ics{citation}\margm{label}} found in the 
\ext{aux} file as though it was actually an \idx{ignoredrecord}:
\begin{codeenv}
\gls{glsxtr@record}\margm{label}\marg{}\marg{\counter{page}}\marg{\encap{glsignore}}\marg{}
\end{codeenv}
Note that \code{\cs{citation}\marg{*}} will always be skipped. Use
\csopt[all]{selection} to select all entries.
This switch is most useful in conjunction with
\atentrypageref{bibtexentry}.

\argsection{no-cite-as-record}

Don't check for instances of \ics{citation} in the \ext{aux} file
(default).
If used in \gls{BibGlsOptions}, this option should be specified as
\begin{codeenv}
\gls{BibGlsOptions}\marg{cite-as-record=false}
\end{codeenv}

\argsection{collapse-same-location-range}

Collapse any \idx{explicit-range} into a normal \gls{record} if
the start and end \glspl{location} are the same (default).
This record will be treated as a normal \gls{location} that can be
merged with neighbouring \glspl{location}, regardless of
\csopt{merge-ranges}.

\argsection{no-collapse-same-location-range}

Don't collapse any \idx{explicit-range} into a normal
\gls{record} if the start and end \glspl{location} are the same.
The \idx{explicit-range} will only be able to merge with
neighbouring \glspl{location} if \csopt[true]{merge-ranges}.
If used in \gls{BibGlsOptions}, this option should be specified as
\begin{codeenv}
\gls{BibGlsOptions}\marg{collapse-same-location-range=false}
\end{codeenv}

\argsection{map-format}

This sets up the rule of precedence for partial \gls{location}
matches (see \sectionref{sec:locationopts}). The argument may be
a comma-separated list of \meta{map}\code{:}\meta{value} pairs.
Alternatively, you can have multiple instances of 
\longarg{map-format} \meta{map}\dcolon\meta{value} which have a
cumulative effect on the command line. You can also use
\code{map-format} as an option within \gls{BibGlsOptions}, but
multiple instances of this key in the same option list will override
each other.

For example, 
\begin{verbatim}
bib2gls --map-format "emph:hyperbf" mydoc
\end{verbatim}
This essentially means that if there's a \gls{record} conflict
involving \encap{emph}, try replacing \encap{emph} with
\encap{hyperbf} and see if that resolves the conflict.

Note that if the conflict includes a \idx{range} formation, the
\idx{range} takes precedence.
The mapping tests are applied as the records are
read. For example, suppose the records are listed in the \ext{aux}
file as:
\begin{codeenv}
\gls{glsxtr@record}\marg{gls.sample}\marg{}\marg{\counter{page}}\marg{\encap{emph}}\marg{3}
\gls{glsxtr@record}\marg{gls.sample}\marg{}\marg{\counter{page}}\marg{\encap{hypersf}}\marg{3}
\gls{glsxtr@record}\marg{gls.sample}\marg{}\marg{\counter{page}}\marg{\encap{hyperbf}}\marg{3}
\end{codeenv}
and \bibgls\ is invoked with
\begin{verbatim}
bib2gls --map-format "emph:hyperbf,hypersf:hyperit" mydoc
\end{verbatim}
or
\begin{verbatim}
bib2gls --map-format emph:hyperbf --map-format hypersf:hyperit mydoc
\end{verbatim}
then \bibgls\ will process these \glspl{record} as follows:
\begin{enumerate}
\item Accept the first \gls{record} (\encap{emph}) since there's
currently no conflict. (This is the first \gls{record} for page~3 for the
entry given by \code{gls.sample}.)
\item The second \gls{record} (\encap{hypersf}) conflicts 
with the existing \gls{record} (\encap{emph}). Neither has
the format \encap{glsnumberformat} or \encap{glsignore} so \bibgls\ consults
the mappings provided by \longarg{map-format}.
\begin{itemize}
\item The \encap{hypersf} format (from the new record) is mapped to
\encap{hyperit},
so \bibgls\ checks if the existing \gls{record}
has this format. In this case it doesn't (the format is
\code{emph}). So \bibgls\ moves on to the next test:

\item The \encap{emph} format (from the existing record) is mapped
to \encap{hyperbf},
so \bibgls\ checks if the new record has this format.
In this case it doesn't (the format is \encap{hypersf}).

Since the provided mappings haven't resolved this conflict,
the new record is discarded with a warning. Note that there's 
no look ahead to the next record. (There may be other 
records for other entries also used on page~3 interspersed between these records.)
\end{itemize}
\item The third \gls{record} (\encap{hyperbf}) conflicts 
with the existing \gls{record} (\encap{emph}). Neither has
the format \encap{glsnumberformat} or \encap{glsignore} so \bibgls\ again consults
the mappings provided by \longarg{map-format}.
\begin{itemize}
\item  The new \gls{record}['s] \encap{hyperbf} format has no mapping provided,
so \bibgls\ moves on to the next test:

\item The existing \gls{record}['s] \encap{emph} format has a mapping
provided (\encap{hyperbf}). This matches the new \gls{record}['s] format,
so the new \gls{record} takes precedence.

This means that the \gls{locationlist} ends up with the \encap{hyperbf}
location for page~3.
\end{itemize}
\end{enumerate}

If, on the other hand, the mappings are given as
\begin{verbatim}
--map-format "emph:hyperit,hypersf:hyperit,hyperbf:hyperit"
\end{verbatim}
then all the three conflicting \glspl{record} (\encap{emph},
\encap{hypersf} and \encap{hyperbf}) will end up being replaced
by a single \gls{record} with \encap{hyperit} as the format.

Multiple conflicts will typically be rare as there's usually little
reason for more than two or three different \gls{location} formats within
the same list. (For example, \encap{glsnumberformat} as the default
and \encap{hyperbf} or \encap{hyperit} for a
\gls{principallocation}.)

\argsection{merge-nameref-on}

The \styopt[nameref]{record} package option (introduced to 
\sty{glossaries-extra} version 1.37) provides extra information
in the record when indexing, obtained from \ics{@currentlabelname},
\ics{@currentHref} and \ics{theHentrycounter}. Instead of writing the record as:
\begin{codeenv}
\format{glsxtr@record}
\end{codeenv}
the record is written as:
\nosecdef{glsxtr@record@nameref}
If \isty{hyperref} hasn't been loaded \meta{title} and \meta{href}
will always be empty. The most reliable target is given by
\code{\meta{counter}.\meta{hcounter}}, where \meta{counter} is the
associated counter name and \meta{hcounter} is
obtained from \cs{theHentrycounter}, which is set to the hyper
target command \csfmt{theH}\meta{counter} during indexing. Since
this information can't be included in the \gls{location} when indexing
with \idx!{makeindex} or \idx!{xindy}, the base \sty{glossaries}
package tries to obtain a prefix from which the target name can be
formed. This doesn't work if \csfmt{theH}\meta{counter} can't be
formed from \meta{prefix}\csfmt{the}\meta{counter}, which results in
broken links. Since \bibgls\ doesn't have the same restrictions, the
actual target can be included in the record. You can then customize
the document to choose whether to use \meta{href} (to link to the
nearest anchor) or \meta{hcounter} to link to the place where the
indexing counter was incremented.

The \code{nameref} record will be written to the \igls{locationlist} using:
\nosecdef{glsxtrdisplaylocnameref}
The \meta{file} part will be empty for normal internal \glspl{location},
and will be set to the corresponding file name for 
\glsdisp{supplementalrecord}{supplemental locations}.

With \sty{hyperref}, \meta{title} is initially empty. The \meta{href} will be
\code{Doc-Start} at the start of the document and is updated
globally on every instance of \ics{refstepcounter}. The 
\meta{title} is updated locally by certain commands, such as 
\ics{section} or \ics{caption}. This means that the \meta{href}
may not always correspond to the \meta{title}, so using
the \styopt[nameref]{record} package option can have unpredictable
results if the \meta{title} is used as link text with \meta{href} as
the target. 

For compactness, \bibgls\ tries to merge duplicate or near
duplicate records. There are four possible rules that it will
use for \code{nameref} records, identified by \meta{rule} in the
\longarg{merge-nameref-on} switch:
\begin{itemize}
\item \optfmt{location}: merge records that match on the
\meta{prefix}, \meta{counter} and \meta{location} parts (as regular
records);
\item \optfmt{title}: merge records that match on the \meta{counter}
and \meta{title} parts;
\item \optfmt{href}: merge records that match on the \meta{counter}
and \meta{href} parts;
\item \optfmt{hcounter}: merge records that match on the \meta{counter}
and \meta{hcounter} parts.
\end{itemize}
The default \meta{rule} is \optfmt{hcounter}. Note that for all
rules the \meta{counter} must match. See the \qt{Nameref Record}
section of the \sty{glossaries-extra} user manual for further
details.

\argsection{merge-wrglossary-records}

For use with the \styopt{indexcounter} package option
(\styfmt{glossaries-extra} v1.29+), this switch merges an entry's
\counter{wrglossary} records for the same page \gls{location}. This is the
default setting. (See also \csopt{save-index-counter}.)

\argsection{no-merge-wrglossary-records}

Don't merge an entry's \counter{wrglossary} records. This means that you
may end up with duplicate page numbers in the entry's \gls{locationlist},
but they will link to different parts of the page.
If used in \gls{BibGlsOptions}, this option should be specified as
\begin{codeenv}
\gls{BibGlsOptions}\marg{merge-wrglossary-records=false}
\end{codeenv}

\argsection{record-count}

Switch on record counting. This will ensure that when each entry
is written to the \iext{glstex} file, \bibgls\ will additionally
set the following fields
\begin{itemize}
\item \field{recordcount}: set to the total
number of records found for the entry;
\item \field{recordcount.counter}: set to the total
number of records found for the entry for the given counter.
\end{itemize}
These fields can then be used with the \gls{rgls}-like commands.

This option is governed by the \longarg{record-count-rule}, which
can be used to exclude certain types of records from the count. The
default rule is \optfmt{all}, which includes all
\idxpl{ignoredrecord}.

The default behaviour of
\nosecdef{rgls}
is to check the \field{recordcount} field against the \catattr{recordcount}
attribute value. This attribute can be set with
\nosecdef{GlsXtrSetRecordCountAttribute}
where \meta{category list} is a comma-separated list
of category labels and \meta{value} is a positive integer.
If the value of the \field{recordcount} field is greater than
\meta{value} then \gls{rgls} behaves like \ics{gls}, otherwise
it does
\nosecdef{rglsformat}
instead.
If the use of \gls{rglsformat} is triggered in this way,
then \gls{rgls} writes a \gls{record} to the \iext{aux} file
with the \glsopt{format} set to \encap{glstriggerrecordformat}.
This ensures that the \gls{recordcount} is correct on the next run,
but the \gls{record} isn't added to the \gls{locationlist} as
\bibgls\ recognises it as a special \igls{ignoredrecord}.
Note that the entry will still appear in the usual glossary unless
you assign it to a different one with \csopt{trigger-type}.

If the \catattr{recordcount} attribute hasn't been set
\gls{rgls} behaves like \ics{gls}. (That is, \gls{rgls}
uses the same internal command used by \ics{gls}.) You can use
\ics{glsxtrenablerecordcount} to redefine \ics{gls}
to \gls{rgls}, so that you can continue to use \ics{gls}
without having to switch command name.

For example:
\begin{codeenv}
\gls{GlsXtrLoadResources}\oarg{
 \csopt[abbrevs]{src},\comment{entries defined in abbrevs.bib}
 \csopt[ignored]{trigger-type},
 \csopt[abbreviation]{category}
}
\cs{glsxtrenablerecordcount}
\gls{GlsXtrSetRecordCountAttribute}\marg{abbreviation}\marg{1}
\end{codeenv}
See the \sty{glossaries-extra} user manual~\cite{glossaries-extra} 
for further details.

\begin{important}
Take care not to confuse the \field{recordcount} field with the \field{indexed}
field. The \field{indexed} field keeps a running total of the number of times an
entry has been recorded \emph{so far}, and is updated every time the entry is
indexed during the current \LaTeX\ run. The \field{recordcount} field stores the total number of records
obtained by \bibgls\ from the \ext{aux} file.
\end{important}

\argsection{no-record-count}

Switch off record counting. (Default.)
If used in \gls{BibGlsOptions}, this option should be specified as
\begin{codeenv}
\gls{BibGlsOptions}\marg{record-count=false}
\end{codeenv}

\argsection{record-count-unit}

Automatically implements \longarg{record-count} and additionally
sets the \field{recordcount.counter.location} fields.
These fields can then be used with the \gls{rgls}-like
commands. This option is governed by \longarg{record-count-rule},
to determine which records should be counted.

\argsection{no-record-count-unit}

Switches off unit record counting. (Default.)
Note that you need \longarg{no-record-count} to completely
switch off record counting.
If used in \gls{BibGlsOptions}, this option should be specified as
\begin{codeenv}
\gls{BibGlsOptions}\marg{record-count-unit=false}
\end{codeenv}

\argsection{record-count-rule}

Automatically implements \longarg{record-count} and sets the rule
that determines which records should contribute to the count.
The \meta{rule} may be one of:
\begin{itemize}
\item \code{all} or \code{a}: these keywords indicate that all
records should be included in the count (default).
\item \code{non-ignored} or \code{n}: these keywords indicate that 
\idxpl{ignoredrecord} should be excluded in the count.
\item \code{c/\meta{regex}/}: only records where the associated
counter name matches the regular expression \meta{regex} should be
included in the count.
\item \code{f/\meta{regex}/}: only records where the associated
format matches the regular expression \meta{regex} should be
included in the count.
\item \code{f/\meta{format-regex}/c/\meta{counter-regex}/\meta{op}}: 
this combines the format and counter name match. The trailing
\meta{op} is optional. If present, it should be one of the keywords: 
\code{and} (boolean AND) or \code{or} (boolean OR). If omitted, 
\code{and} is assumed.
\end{itemize}

For example:
\begin{verbatim}
bib2gls --record-count-rule 'f/.*(bf|it)/c/(sub)?section/or' myDoc
\end{verbatim}
This will only count records where the format matches the regular
expression \code{.*(bf|it)} (for example, \code{hyperbf} or
\code{hyperit}) or the counter name matches \code{section} or
\code{subsection} (but not \code{subsubsection}, since the
expressions are \gls{anchored}).

This syntax doesn't permit the use of the sequence \code{/c/}
appearing in the regular expressions, but both the format and
counter name are either control sequence names or are a substring of
a control sequence name, so they should typically just be
alphabetical strings.
\argsection{retain-formats}

It's possible that you may not want to lose certain \gls{location}
formats, even if it means having duplicate \glspl{location}. For example,
if you want to move a \gls{principallocation} using
\csopt[remove]{save-principal-locations}. In which case, use this
switch with a comma-separated list of formats that should be
retained. Note that exact duplicates will still be merged.
This switch has a cumulative effect.

Take care if you use this switch and you have an \idx{explicit-range} with
coincident start and end locations. If the
\glsdisp{principallocation}{principal record} is between
the start and end format markers then the \idx{range} can't collapse
to an ordinary \gls{record}. (You may need to use
\csopt[true]{merge-ranges}.)

\argsection{no-retain-formats}

Normal \gls{location} merging rules apply (default).
If used in \gls{BibGlsOptions}, this option should be specified as
\begin{codeenv}
\gls{BibGlsOptions}\marg{retain-formats=false}
\end{codeenv}

\section{Bib File Options}

\argsection{warn-non-bib-fields}

If any internal fields are found in the \ext{bib} file, this setting
will issue a warning as their use can cause unexpected results.
The fields checked for are those listed in Tables~\ref{tab:internalfields}
and \ref{tab:baseinternalfields} with a few exceptions, notably
\field{type} and \field{sort}. Ideally you shouldn't need to use
\field{sort} as there should be an appropriate fallback set up to
use if \field{sort} isn't set, such as the label for symbols or the
name for terms or the short form for abbreviations (see \sectionref{sec:fallbacks}).

This is the default setting and was added as some users were
confused over which fields could be used in the \ext{bib} file.
The use of these fields can break \bibgls's normal behaviour and
cause unexpected results.

The check is performed before field aliasing, so it's possible to
alias a field to an internal field, such as \field{group}, without
triggering this warning. If you do this you need to make sure you
have taken appropriate precautions to avoid unexpected results.

\argsection{no-warn-non-bib-fields}

Switches off the check for non-bib fields. If you use this option
you need to make sure you have taken appropriate precautions to
avoid unexpected results.
If used in \gls{BibGlsOptions}, this option should be specified as
\begin{codeenv}
\gls{BibGlsOptions}\marg{warn-non-bib-fields=false}
\end{codeenv}

\argsection{warn-unknown-entry-types}

If any unknown \glspl{entrytype} are found in the \ext{bib} file, \bibgls\
will issue a warning with this option set (default).

\argsection{no-warn-unknown-entry-types}

This option will suppress the warning if an unknown \glspl{entrytype} are
found in the \ext{bib} file.
If used in \gls{BibGlsOptions}, this option should be specified as
\begin{codeenv}
\gls{BibGlsOptions}\marg{warn-unknown-entry-types=false}
\end{codeenv}

\section{Field Options}

\argsection{group}

The \styfmt{glossaries-extra} \styopt{record} package option
automatically creates a new internal field called \field{group}. If the
\longarg{group} switch is used with the default \csopt[auto]{group}
option then, when sorting, \bibgls\ will try
to determine the \idx{group} for each entry and assign it to the
\field{group} field.  (Some \csopt{sort} options ignore this
setting.) This value will be picked up by \ics{printunsrtglossary}
if group headings are required (for example with the
\glostyle{indexgroup} style) or if group separators are required
(for example, the \glostyle{index} style with the default 
\styopt[false]{nogroupskip}). If you don't require grouping within
the glossary, there's no need to use this switch. Note that this
switch doesn't automatically select an appropriate glossary style.

If you want \idxpl{sub-group}, you will need to use the \csopt{group-level}
resource option and ensure you have \sty{glossaries-extra} v1.49+.
\Idxpl{smallgroup} can be merged with the \csopt{merge-small-groups}
resource option.

\begin{important}
The \field{group} field should typically not be set in the \ext{bib}
file and will trigger a warning if found. The explicit use of the
\field{group} key will override \bibgls's normal group formation
behaviour, which can cause unexpected results. The custom use of the
\field{group} field requires some care. As a general rule, if you
find yourself wanting to use the \field{group} field in the
\ext{bib} file, then the chances are that what you actually have is
a \gls{hierarchicalglossary} (list of topics) and what you really
need is the \field{parent} field.  Compare the example files
\exfile{sample-textsymbols.tex} and
\exfile{sample-textsymbols2.tex}. See also
\sectionref{sec:logicaldivisions}.
\end{important}
There are eight types of groups:
\begin{description}
\item[\idx{lettergroup}] The first non-ignored character of the
sort value is alphabetic. This type of group occurs when using the
alphabetic sort methods listed in \tableref{tab:sortoptionsrule} or
with the letter sort methods listed in
\tableref{tab:sortoptionsletter} or with the letter-number sort
methods listed in \tableref{tab:sortoptionsletternumber}. The group
label is obtained from \gls!{bibglslettergroup}.

\item[\idx{nonlettergroup} (or \idx{symbolgroup})]
The first non-ignored character of all the sort values within this group
are non-alphabetical. This type of group occurs when using the
alphabetic sort methods listed in \tableref{tab:sortoptionsrule} or
with the letter sort methods listed in
\tableref{tab:sortoptionsletter} or with the letter-number sort
methods listed in \tableref{tab:sortoptionsletternumber}.
The alphabetic sort methods ignore many punctuation
characters, so an entry that has a non-alphabetic initial
character in the sort value may actually be placed in a
\idx{lettergroup}.
The group label is obtained from \gls!{bibglsothergroup}.

\item[\idx{emptygroup}] The sort value is empty when sorting with an
alphabetical, letter or letter-number method, typically a
result of the original value consisting solely of commands that \bibgls\
can't interpret. The group label is obtained from \gls!{bibglsemptygroup}.

\item[\idx{numbergroup}] The entries were sorted by one of the
numeric comparisons listed in \tableref{tab:sortoptionsnumerical}.
The group label is obtained from \gls!{bibglsnumbergroup}.

\item[\idx{datetimegroup}] The entries were sorted by one of the date-time
comparisons listed in \tableref{tab:sortoptionsdatetime} (where both
date and time are present). 
The group label is obtained from \gls!{bibglsdatetimegroup}.

\item[\idx{dategroup}] The entries were sorted by one of the date
comparisons (where the time is omitted). 
The group label is obtained from \gls!{bibglsdategroup}.

\item[\idx{timegroup}] The entries were sorted by one of the time
comparisons (where the date is omitted). 
The group label is obtained from \gls!{bibglstimegroup}.

\item[\idx{customgroup}] The group label is explicitly set either by
aliasing a field (with \csopt{field-aliases}) or by using the
\csopt[\meta{label}]{group} resource option. You will need to use
\gls{glsxtrsetgrouptitle} in the document to provide an associated title if the
\meta{label} isn't the same as the title.  Remember that with older
\LaTeX\ kernels, the label can't contain any active characters, so
you can't use non-ASCII characters in \meta{label} with
\sty{inputenc} (but you can use non-ASCII alphanumerics with
\sty{fontspec}). To ensure better support for UTF-8 with \pdfLaTeX, make sure you
have a recent \TeX\ distribution and up-to-date versions of
\sty{glossaries} and \sty{glossaries-extra}.

\end{description}

The \idx{lettergroup} titles will typically have the first character
converted to \idx{uppercase} for the alphabet sort methods
(\tableref{tab:sortoptionsrule}). A \qt{letter} may not necessarily
be a single character (depending on the sort rule), but may be
composed of multiple characters, such as a \idx{digraph} (two
characters) or \idx{trigraph} (three characters).

For example, if the sort rule recognises the digraph \qt{dz} as a
letter, then it will be converted to \qt{Dz} for the group title. 
There are some exceptions to this. For example, the Dutch digraph 
\qt{ij} should be \qt{IJ} rather than \qt{Ij}. This is indicated 
by the following line in the \langxml:
\begin{verbatim}
<entry key="grouptitle.case.ij">IJ</entry>
\end{verbatim}
If there isn't a \idx{grouptitle.case.lc} key (where 
\meta{lc} is the \idx!{lowercase} version), then only the first character
will be converted to \idx{uppercase} otherwise the value supplied by the
resource file is used. This resource key is only checked for
the alphabetical comparisons listed in
\tableref{tab:sortoptionsrule}. If the initial part of the sort
value isn't recognised as a letter according to the sort rule, then
the entry will be in a \idx{nonlettergroup} (even if the character is
alphabetical).

The letter (\tableref{tab:sortoptionsletter}) and letter-number
(\tableref{tab:sortoptionsletternumber}) methods only select the
first character of the sort value for the group. If the character is
alphabetical\footnote{according to Java's
\code{Character.isAlphabetic(int)} method} then it will be a 
\idx{lettergroup} otherwise it's a \idx{nonlettergroup}.  The case-insensitive
ordering (such as \csopt[letter-nocase]{sort}) will convert the
letter group character to \idx{uppercase}. The case-sensitive ordering
(such as \csopt[letter-case]{sort}) won't change the case.

Glossary styles with navigational links to groups (such as
\glostyle{indexhypergroup}) require an extra run for the ordinary
\cs{cs.makeglossaries} and \cs{makenoidxglossaries} methods. For
example, for the document \filefmt{myDoc.tex}:
\begin{verbatim}
pdflatex myDoc
makeglossaries myDoc
pdflatex myDoc
pdflatex myDoc
\end{verbatim}
On the first \appfmt{pdflatex} call, there's no glossary. On the second
\appfmt{pdflatex}, there's a glossary but the glossary must be
processed to find the group information, which is written to the
\iext{aux} file as
\nosecdef{@gls@hypergroup}
The third \appfmt{pdflatex} reads this information and is then able
to create the navigation links.

With \bibgls, if the \optfmt{type} is provided (through the
\field{type} field or via options such as \csopt{type} and
\csopt{dual-type}) then this information can be determined when
\bibgls\ is ready to write the \iext{glstex} file, which means that
the extra \LaTeX\ run isn't necessary. If \bibgls\ doesn't know
the glossary type then it will fallback on the original method
which requires an extra \LaTeX\ run.

For example:
\begin{codeenv}
\cmd{documentclass}\marg{article}
\cmd{usepackage}\oarg{colorlinks}\marg{hyperref}
\cmd{usepackage}\oarg{\styopt{record},\styopt{abbreviations},\styopt[indexhypergroup]{style}}\marg{glossaries-extra}
\strut
\gls{GlsXtrLoadResources}\oarg{\csopt[entries]{src},\comment{data in entries.bib}
  \csopt[main]{type}\comment{put these entries in the 'main' glossary}
}
\strut
\gls{GlsXtrLoadResources}\oarg{\csopt[abbrvs]{src},\comment{data in abbrvs.bib}
  \csopt[abbreviations]{type}\comment{put entries in the 'abbreviations' glossary}
}
\end{codeenv}
Here the \csopt{type} is set and \bibgls\ can detect that
\isty{hyperref} has been loaded, so if the \longarg{group} switch
is used, then the group hyperlinks can be set (using
\gls!{bibglshypergroup}). This means that the build process is
just:
\begin{verbatim}
pdflatex myDoc
bibtex --group myDoc
pdflatex myDoc
\end{verbatim}
Note that this requires \isty{glossaries} v4.53+ and
\isty{glossaries-extra} v1.53. If your version of \sty{glossaries}
or \sty{glossaries-extra} is too old, an extra \LaTeX\ run is
required.

If \isty{hyperref} isn't loaded or the \longarg{group} switch
isn't used or the \field{type} isn't set or your version of
\sty{glossaries} is too old, then the information can't be saved in
the \ext{glstex} file.

For example:
\begin{codeenv}
\cmd{documentclass}\marg{article}
\cmd{usepackage}\oarg{colorlinks}\marg{hyperref}
\cmd{usepackage}\oarg{\styopt{record},\styopt{abbreviations},\styopt[indexhypergroup]{style}}\marg{glossaries-extra}
\strut
\gls{GlsXtrLoadResources}\oarg{\csopt[entries]{src}}\comment{data in entries.bib}
\gls{GlsXtrLoadResources}\oarg{\csopt[abbrvs]{src}}\comment{data in abbrvs.bib}
\end{codeenv}
This requires the build process:
\begin{verbatim}
pdflatex myDoc
bibtex --group myDoc
pdflatex myDoc
pdflatex myDoc
\end{verbatim}
because the group hyperlink information can't be determined by
\bibgls, so it's best to always set the \optfmt{type} if you want
hyper-group styles, and make sure you have an up-to-date version of 
\styfmt{glossaries} (and \styfmt{glossaries-extra}).

\argsection{no-group}

Don't automatically set the \field{group} field with
\csopt[auto]{group} (default). The glossary won't have groups even if a
group style, such as \glostyle{indexgroup}, is used (unless the
\field{group} field is set to a custom value).
If used in \gls{BibGlsOptions}, this option should be specified as
\begin{codeenv}
\gls{BibGlsOptions}\marg{group=false}
\end{codeenv}

\argsection{no-expand-fields}

By default, \gls{newglossaryentry} and similar commands expand field values
(except for \field{name}, \field{symbol} and \field{description}).
This is useful if constructing field values programmatically
(for example in a \idx!{loop}) but can cause a problem if certain fragile
commands are included in the field. 

The switch \longarg{no-expand-fields} makes \bibgls\ write
\gls{glsnoexpandfields} to the \iext{glstex} file, which
switches off the expansion.  Since \bibgls\ is simply
fetching the data from \iext{bib} files, it's unlikely 
that this automatic expansion is required and since it can
also be problematic this option is on by default. You can
switch it off with \longarg{expand-fields}.

If used in \gls{BibGlsOptions}, this option should be specified as
\begin{codeenv}
\gls{BibGlsOptions}\marg{expand-fields=false}
\end{codeenv}

\argsection{expand-fields}

Don't write \gls{glsnoexpandfields} to the \iext{glstex} file,
allowing fields to expand when the entries are defined.
Remember that this doesn't include the \field{name}, \field{symbol}
or \field{description} fields, which need to have their
expansion switched on with \gls{glssetexpandfield}
before the entries are defined (that is, before using
\gls!{GlsXtrLoadResources}).

\argsection{mfirstuc-protection}

If you have \isty{mfirstuc} v2.08+, \isty{glossaries} v4.50+
and \isty{glossaries-extra} v1.49+ then this setting shouldn't be
required any more as there's now better \idx{sentencecase} handling.
If these versions are detected in the \ext{log} file then the
default will switch to \longarg{no-mfirstuc-protection}
otherwise the default is \longarg{mfirstuc-protection}. If this causes any
problems, use \longarg{mfirstuc-protection} to re-enable this
setting. The information below relates to older versions.

Commands like \ics{Gls} use \ics{makefirstuc} provided by the
\isty{mfirstuc} package. This command has limitations and one of the
things that can break it is the use of a referencing command 
at the start of its argument. The \sty{glossaries-extra} package has
more detail about the problem in the \qt{Nested Links} section of
the user manual~\cite{glossaries-extra}. If a glossary field starts
with one of these problematic commands, the recommended method (if
the command can't be replaced) is to insert an empty group in front
of it.

For example, the following definition
\begin{codeenv}
\gls{newabbreviation}\marg{shtml}\marg{shtml}\marg{\ics{glsps}\marg{ssi} enabled \cs{glsps}\marg{short}\marg{html}}
\end{codeenv}
will cause a problem for \code{\cs{Gls}\marg{shtml}} on first use.
The above example would be written in a \ext{bib} file as:
\begin{codeenv}
\atentry{abbreviation}\marg{shtml,
  \field{short}=\marg{shtml},
  \field{long}=\marg{\cs{glsps}\marg{ssi} enabled \cs{glsps}\marg{html}}
}
\end{codeenv}
The default \sty{mfirstuc} protection will automatically insert an empty
group before \code{\cs{glsps}\marg{ssi}} when writing the definition
in the \ext{glstex} file.

The argument for this switch should either be a comma-separated list
of fields or the keyword \code{all} (which indicates all fields).
\bibgls\ will automatically insert an empty group at the start of
the listed fields that start with a problematic command, and a
warning will be written to the transcript. Unknown fields are
skipped even if they're included in the list. An empty argument is
equivalent to \longarg{no-mfirstuc-protection}. The default value is
\code{all}.

\argsection{no-mfirstuc-protection}

Switches off the \isty{mfirstuc} protection mechanism described
above.
If used in \gls{BibGlsOptions}, this option should be specified as
\begin{codeenv}
\gls{BibGlsOptions}\marg{mfirstuc-protection=false}
\end{codeenv}

\argsection{mfirstuc-math-protection}

If you have \isty{mfirstuc} v2.08+, \isty{glossaries} v4.50+ and
\isty{glossaries-extra} v1.49+ then this setting shouldn't be
required any more as there's now better \idx{sentencecase} handling.  If
these versions are detected in the \ext{log} file then the default
will switch to \longarg{no-mfirstuc-math-protection}. If this causes
any problems, use \longarg{mfirstuc-math-protection} to re-enable
this setting. The information below relates to older versions.

This setting works in the same way as \longarg{mfirstuc-protection} but
guards against fields starting with inline maths
(\idx{mshiftchar}\ldots\idx{mshiftchar}). For example, if the
\field{name} field starts with
\code{\idx{mshiftchar}x\idx{mshiftchar}} and the glossary style
automatically tries to convert the first letter of the name to
\idx{uppercase}, then this will cause a problem.

With \longarg{mfirstuc-math-protection} set, \bibgls\ will
automatically insert an empty group at the start of the field and
write a warning in the transcript. This setting is on by default.

\argsection{no-mfirstuc-math-protection}

Switches off the above.
If used in \gls{BibGlsOptions}, this option should be specified as
\begin{codeenv}
\gls{BibGlsOptions}\marg{mfirstuc-math-protection=false}
\end{codeenv}

\argsection{nested-link-check}

By default, \bibgls\ will parse certain fields for potential nested links.
(See the section \qt{Nested Links} in the \sty{glossaries-extra}
user manual~\cite{glossaries-extra}.)

The default set of fields to check are: \field{name}, \field{text},
\field{plural}, \field{first}, \field{firstplural}, \field{long},
\field{longplural}, \field{short}, \field{shortplural} and
\field{symbol}.

You can change this set of fields using
\longarg{nested-link-check} \meta{value} where \meta{value} may be
\optfmt{none} (don't parse any of the fields) or a comma-separated
list of fields to be checked.

\argsection{no-nested-link-check}

Equivalent to \longarg{nested-link-check} \optfmt{none}.
If used in \gls{BibGlsOptions}, this option should be specified as
\begin{codeenv}
\gls{BibGlsOptions}\marg{nested-link-check=none}
\end{codeenv}
or
\begin{codeenv}
\gls{BibGlsOptions}\marg{nested-link-check=false}
\end{codeenv}

\argsection{shortcuts}

Some entries may reference another entry within a field, using
commands like \ics{gls}, so \bibgls\ parses the fields for these
commands to determine dependent entries to allow them to be selected
even if they haven't been used within the document.
The \styopt{shortcuts} package option provided by
\styfmt{glossaries-extra} defines various synonyms, such as \ics{ac}
which is equivalent to \ics{gls}. By default the value of the
\styopt{shortcuts} option will be picked up by \bibgls\ when parsing the
\iext{aux} file. This then allows \bibgls\ to additionally search for
those shortcut commands while parsing the fields.

You can override the \styopt{shortcuts} setting using
\longarg{shortcuts} \meta{value} (where \meta{value} may take
any of the allowed values for the \styopt{shortcuts} package option), 
but in general there is little need to use this switch.

\argsection{trim-fields}

Trim leading and trailing spaces from all field values. For example,
if the \ext{bib} file contains:
\begin{codeenv}
\atentry{entry}\marg{sample,
  \field{name} = \marg{sample},
  \field{description} = \marg{
    an example
  }
}
\end{codeenv}
This will cause spurious spaces in the \field{description} field. 
Using \longarg{trim-fields} will automatically trim the values
before writing the \iext{glstex} file.

Note that even without this trimming option on, fields that are
set as keys within \cs{longnewglossaryentry} or the optional
argument of \cs{newabbreviation} will automatically have the leading and
trailing spaces internally trimmed by the \isty{xkeyval} package, so
this trimming action only affects fields that aren't set in this
way, such as the \field{description}, \field{long} and \field{short}
fields. If you specifically require a space at the start or end of a
field then use a spacing command, such as \cs{cs.space} or \cs{space} 
or \idx{nbspchar}.

\argsection{trim-only-fields}

Only trim leading and trailing spaces from the fields identified in
the comma-separated \meta{list}. This option has a cumulative effect
but is cancelled by \longarg{no-trim-fields} (which switches off all
trimming) and by \longarg{trim-fields} (which switches on trimming
for all fields). This option may not be used with
\longarg{trim-except-fields}.

For example, to only trim the \field{description} field:
\begin{verbatim}
bib2gls --trim-only-fields description myDoc
\end{verbatim}

\argsection{trim-except-fields}

Trim all leading and trailing spaces from fields except those identified in
the comma-separated \meta{list}. This option has a cumulative effect
but is cancelled by \longarg{no-trim-fields} (which switches off all
trimming) and by \longarg{trim-fields} (which switches on trimming
for all fields). This option may not be used with
\longarg{trim-only-fields}. See the above note about \sty{xkeyval}.

For example, to trim all fields except \field{short} and
\field{long}:
\begin{verbatim}
bib2gls --trim-except-fields short,long myDoc
\end{verbatim}
Or
\begin{verbatim}
bib2gls --trim-except-fields short --trim-except-fields long myDoc
\end{verbatim}

\argsection{no-trim-fields}

Don't trim any leading or trailing spaces from field values (but see
the above note about \sty{xkeyval}). This is the default setting.
If used in \gls{BibGlsOptions}, this option should be specified as
\begin{codeenv}
\gls{BibGlsOptions}\marg{trim-fields=false}
\end{codeenv}

\section{Other Options}

\argsection{force-cross-resource-refs}

Force \igls{crossresourceref} mode on (see
\sectionref{sec:resourcesets}).

\argsection{no-force-cross-resource-refs}

Don't force \igls{crossresourceref} mode on (default). 
The mode will be enabled if applicable (see
\sectionref{sec:resourcesets}).
If used in \gls{BibGlsOptions}, this option should be specified as
\begin{codeenv}
\gls{BibGlsOptions}\marg{force-cross-resource-refs=false}
\end{codeenv}

\argsection{provide-glossaries}

This setting will make \bibgls\ add the line
\begin{codeenv}
\ics{provideignoredglossary*}\margm{type}
\end{codeenv}
to the \ext{glstex} file before an entry is defined where that entry
has the \field{type} field set to an unknown glossary type (\bibgls\
can detect from the \ext{aux} file all glossaries that have been
defined with \cs{newglossary} but not those defined with \cs{newignoredglossary}).

This ensures that the glossary exists, but the use of
\ics{provideignoredglossary} (rather than \cs{newignoredglossary})
will prevent an error if the glossary has already been defined.

\argsection{no-provide-glossaries}

This setting prevents \bibgls\ from providing unknown glossaries,
except in a few documented situations (the \csopt{master},
\csopt{trigger-type}, \csopt{ignored-type} and \csopt{secondary} options). This is the
default since it's a useful way of detecting misspelt glossary
labels. It's harder to detect the problem if a misspelt label has
caused an entry to be added to a hidden glossary.
If used in \gls{BibGlsOptions}, this option should be specified as
\begin{codeenv}
\gls{BibGlsOptions}\marg{provide-glossaries=false}
\end{codeenv}

\argsection{replace-quotes}

Single and double-quote characters (\idx{aposchar} and
\idx{doublequotechar}) will be written as \cs{bibglsaposchar} and
\cs{bibglsdoublequotechar} in field values and group information 
written to the \ext{glstex} file.

\argsection{no-replace-quotes}

Single and double-quote characters (\idx{aposchar} and
\idx{doublequotechar}) will be written as those actual characters
(default).
If used in \gls{BibGlsOptions}, this option should be specified as
\begin{codeenv}
\gls{BibGlsOptions}\marg{replace-quotes=false}
\end{codeenv}

\chapter{\iext{bib} Format}
\label{sec:bib}
\setsecdepth{1}

\bibgls\ recognises certain \glspl{entrytype}. Any unrecognised types will
be ignored and a warning will be written to the transcript file.
Entries are defined in the usual \ext{bib} format:
\begin{codeenv*}
\idx{atchar}\meta{entry-type}\marg{\meta{id},
  \meta{field-name-1} = \margm{text},
  \ldots
  \meta{field-name-n} = \margm{text}
}
\end{codeenv*}
where \meta{entry-type} is the \gls{entrytype} (listed below),
\meta{field-name-1}, \ldots, \meta{field-name-n} are the field names and
\meta{id} is a unique label. The label can't contain any spaces or
commas, and most special characters are forbidden. The hyphen character and
some other punctuation characters are allowed by \bibgls, but you need to make 
sure that your document hasn't made them active. In general it's best to
stick with alpha-numeric labels.  The field values may be delimited by braces
\margm{text} or double-quotes \code{\qtdelim{\meta{text}}}.

The \csopt{label-prefix} option can be used to instruct \bibgls\ to
insert prefixes to the labels (\meta{id}) when the data is read.
Remember to use these prefixes when you reference the entries in the
document, but don't include them when you reference them in the
\ext{bib} file.  There are some special prefixes that have a
particular meaning to \bibgls: \qt{\idprefix{dual}} and
\qt{\idprefix{extn}} where \meta{n} is a positive integer.
In the first case, \idprefix{dual} references the dual element of a
dual entry (see \atentry{dualentry}). This prefix will be
replaced by the value of the \csopt{dual-prefix} option. The
\idprefix{extn} prefix is used to reference an entry from a
different set of resources (loaded by another
\gls{GlsXtrLoadResources} command). This prefix is replaced by the
corresponding element of the list supplied by \csopt{ext-prefixes},
but this is only supported if the \igls{crossresourceref} mode is enabled
(see \sectionref{sec:resourcesets}).

In the event that the \field{sort} value falls back on the label,
the original label supplied in the \ext{bib} file is used, not the
prefixed label.

\section{Encoding}
\label{sec:bibencoding}

If you are using \XeLaTeX\ or \LuaLaTeX\ (which are natively UTF-8)
or if you are using a modern \TeX\ distribution \pdfLaTeX\ with
UTF-8 support, then you can have UTF-8 characters in the \meta{id}
of your entries. (Avoid \TeX\ special characters, active characters
or characters that are part of the \ext{bib} syntax.)

You can set the character \igls{encoding} in the \ext{bib} file using:
\begin{codeenv}
\idx{commentchar} Encoding: \meta{encoding-name}
\end{codeenv}
where \meta{encoding-name} is the name of the character \gls{encoding}.
For example:
\begin{verbatim}
% Encoding: UTF-8
\end{verbatim}
You can also set the \gls{encoding} using the \csopt{charset} option,
but it's simpler to include the above comment on the first line of
the \ext{bib} file. (This comment is also searched for by JabRef
to determine the \gls{encoding}, so it works for both applications.) 
If you don't use either method \bibgls\ will
have to search the entire \ext{bib} file, which is inefficient and
you may end up with a mismatched \gls{encoding}.

\begin{important}
The encoding comment line must come before any non-ASCII content
otherwise a malformed input error may occur while parsing the file
for the comment line.
\end{important}

If there is no encoding line in the \ext{bib} file and the
\csopt{charset} option hasn't been used, then the default encoding
will be assumed (see \sectionref{sec:defencoding}).

\section{Comments}
\label{sec:bibcomments}

The original \ext{bib} file format as defined by \BibTeX\ doesn't
have a designated comment character, but instead treats anything
outside of \code{@\meta{entry}\margm{data}} as unwanted material
that's ignored. This can catch out users who try to do something
like:
\begin{verbatim}
%@misc{sample, title={Sample} }
\end{verbatim}
In this case, the percent character is simply discarded and the
line is treated as:
\begin{verbatim}
@misc{sample, title={Sample} }
\end{verbatim}
Some applications that parse \ext{bib} files are less tolerant of
unwanted material. In the case of \bibgls, the percent character is
treated as a comment character and other unwanted material
should be omitted.  Avoid using comments within field values.
Comments are best placed outside of entry definitions.

The most common type of comment is the \gls{encoding} comment, described
above.  \BibTeX's \entrydef{comment} is also supported by \bibgls\
for general comments, but not for the \gls{encoding}.

\section{Fields}
\label{sec:fields}

Each \gls{entrytype} may have required fields, optional fields and
ignored fields. These are set using a \keyvallist\ within
\code{@\meta{entry-type}\marg{\meta{id},\meta{fields}}} in the
\ext{bib} file. Most keys recognised by
\gls{newglossaryentry} may be used as a field unless \bibgls\
considers them an internal field (see below). In general, you
shouldn't need to use the \field{sort} field.

If an optional field is missing and \bibgls\ needs to access it for
some reason, \bibgls\ will try to fallback on another value. The
actual fallback value depends on the \gls{entrytype}. The most common
fallback is that used if the \field{sort} field is missing, which is
typically the case. This approach allows different \glspl{entrytype} to
have different fields used for sorting (see \sectionref{sec:fallbacks}).

Predefined fields for use in \ext{bib} files are listed in
Tables~\ref{tab:fields}, \ref{tab:bib2glsfields},
\ref{tab:prefixfields} and~\ref{tab:accsuppfields}.  If you add any
custom keys in your document using \ics{glsaddkey} or
\ics{glsaddstoragekey}, those commands must be placed before the
first use of \gls{GlsXtrLoadResources} to ensure that \bibgls\
recognises them as a valid field name. 

\begin{important}
If you define your own custom keys, ensure that they don't contain
spaces, commas (\idx{commasep}), equal signs (\idx{equalsassign}) or 
any other character that isn't supported by the \ext{bib} format.
Additionally, if you want to use \csopt{assign-fields}, ensure
that you don't use any of the assignment special characters, such as
plus (\idx{concat-plus}), within any
field names.
\end{important}

Internal fields that may be assigned within the document (the
\LaTeX\ assignment code having been written by \bibgls\ in the
\iext{glstex} file) are listed in Table~\ref{tab:internalfields}.
These typically shouldn't be used in the \ext{bib} file. Some of
these fields can be set for a particular document using a resource
option, such as \csopt{type} or \csopt{group}. With
\longarg{warn-non-bib-fields} set, \bibgls\ will check for internal
fields that can cause interference with its normal operations and
will warn if any are found in the \ext{bib} file.

There are also some fields that are set and used by 
\styfmt{glossaries} or \styfmt{glossaries-extra} listed in
Table~\ref{tab:baseinternalfields} that aren't recognised by \bibgls.
In most cases these fields don't have a designated key and are only
intended for internal use by \bibgls\ or by the \styfmt{glossaries} or
\styfmt{glossaries-extra} package. Note that the value of the \field{sort} field
written to the \ext{bib} file doesn't always exactly match the sort
value used by \bibgls\ (which is stored in \field{bib2gls@sort}).
Any special characters found in the sort value are always substituted
before writing the \ext{bib} file to avoid syntax errors.

Any unrecognised fields will be ignored by \bibgls. This is more
convenient than using \ics{input} or \gls{loadglsentries}, which
requires all the keys used in the file to be defined, regardless of
whether or not you actually need them in the document.

Other entries can be cross-referenced using the \field{see},
\field{seealso} or \field{alias} fields or
by using commands like \ics{gls} or \ics{glsxtrp} in any of the
recognised fields. These will automatically be selected if the
\csopt{selection} setting includes dependencies, but you may need to
rebuild the document to ensure the \glspl{locationlist} are correct.  Use
of the \ics{glssee} command will create an \igls{ignoredrecord} and the
\field{see} field will be set to the relevant information.  If an
entry has the \field{see} field already set, any instance of
\ics{glssee} in the document for that entry will be appended to the
\field{see} field (provided you have at least v1.14 of
\sty{glossaries-extra}). In general, it's best just to use the
\field{see} field and not use \ics{glssee}.

The \field{seealso} key was only added to \sty{glossaries-extra}
v1.16, but this field may be used with \bibgls\ even if you only
have version 1.14 or 1.15. If the key isn't available,
\code{\field{seealso}=\margm{xr-list}} will be treated as
\code{\field{see}=\marg{[\ics{seealsoname}]\meta{xr-list}}} (the resource
option \csopt{seealso} won't have an effect).  You can't use both
\field{see} and \field{seealso} for the same entry with \bibgls.
Note that the \field{seealso} field doesn't allow for the optional
\oargm{tag} part. If you need a different tag, either use \field{see}
or change the definition of \ics{seealsoname} or
\ics{glsxtruseseealsoformat}.  Note that, unless you are using
\idx!{xindy}, \ics{glsxtrindexseealso} just does
\ics{glssee}\oarg{\ics{seealsoname}}, and so will be treated as \field{see}
rather than \field{seealso} by \bibgls. Again, it's better to just
use the \field{seealso} field directly.

You can identify an arbitrary field as containing a list of dependent entry 
labels with \csopt{dependency-fields}. This instructs \bibgls\ to parse the 
listed fields for dependencies in a similar manner to the \field{see} field, 
but it doesn't add any information to the cross-referencing part of the 
\gls{locationlist}. The option may be used in combination with the \field{see} or
\field{seealso} fields.

\clearpage

\printfields
 {basefield,extrafield}
 {Fields Provided by \styfmt{glossaries-extra}}
 {tab:fields}

\clearpage

\printfields
 {bib2glsfield}
 {Fields Provided by \bibgls}
 {tab:bib2glsfields}

\printfields
 {prefixfield}%
 [Fields Provided by \styfmt{glossaries-prefix}]%
 {Fields Provided by \isty{glossaries-prefix}}
 {tab:prefixfields}

\printfields
 [%
   Don't load \isty{glossaries-accsupp} directly (with \ics{usepackage}) 
   when using \styfmt{glossaries-extra}. Load using the 
   \glsadd{idx.accsupp}\styopt{accsupp} 
   package option instead.
%\begin{codeenv}
%\cs{usepackage}\oarg{\styopt{record},\styopt{accsupp}}\marg{glossaries-extra}
%\end{codeenv}
 ]%
 {accessfield}%
 [Fields Provided by \styfmt{glossaries-accsupp}]%
 {Fields Provided by \isty{glossaries-accsupp}}
 {tab:accsuppfields}

\printfields
 [%
  You may define and assign \field{bibtextype} as a key (although it's more
  likely to be aliased). Don't define any of the others listed in this
  table, and don't use any of them in the \ext{bib} file. A possible
  exception is the \field{type} field, but it's more flexible to set
  that through a resource option. The explicit use of \field{group} 
  within a \ext{bib} file can cause unpredictable
  results and is best set through a resource option or by \bibgls.
  In general, you shouldn't need to set the \field{sort} field as
  appropriate fallbacks should produce useful sort values
  (see \sectionref{sec:fallbacks}).%
 ]%
 {internalfield}%
 [Fields Set by \bibgls]%
 {Fields Sometimes Set by \bibgls\ in the \iext{glstex} File}%
 {tab:internalfields}

\printfields
 [%
   Don't define any of these as keys and don't use any of them in the \ext{bib}
   file.%
 ]%
 {baseinternalfield}%
 [Internal Fields Set by \styfmt{glossaries} or \styfmt{glossaries-extra} 
  or \bibgls]%
 {Internal Fields Set by \isty{glossaries} or \isty{glossaries-extra} 
  or \bibgls}%
 {tab:baseinternalfields}

\printfields
 [%
   Only available for \atentry{compoundset}. These correspond to the
   arguments of \ics{multiglossaryentry}.%
 ]%
 {compoundsetfield}%
 {Compound Set Fields}%
 {tab:compoundsetfields}

\clearpage

\section{String Concatenation}
\label{sec:bibstringconcat}

The \ext{bib} format allows you to perform string \gls{concatenation}.
That is, join fragments together to form a single value.
The \gls{concatenation} operator in \ext{bib} files is
\idx{stringconcat}.
For example, if the following string is defined:
\begin{codeenv}
\atentry{string}\marg{markuplang=\marg{markup language}}
\end{codeenv}
Then values can be obtained by concatenating this string with other
strings. For example:
\begin{codeenv}
\atentry{abbreviation}\marg{xml,
  \field{short}=\marg{XML},
  \field{long}=\marg{extensible } \idx{stringconcat} markuplang
}
\atentry{abbreviation}\marg{html,
  \field{short}=\marg{HTML},
  \field{long}=\marg{hypertext } \idx{stringconcat} markuplang
}
\end{codeenv}
This is equivalent to:
\begin{codeenv}
\atentry{abbreviation}\marg{xml,
  \field{short}=\marg{XML},
  \field{long}=\marg{extensible markup language}
}
\atentry{abbreviation}\marg{html,
  \field{short}=\marg{HTML},
  \field{long}=\marg{hypertext markup language}
}
\end{codeenv}

Note that some resource options allow string \gls{concatenation} in
their syntax. That uses a different operator. See
\sectionref{sec:optstringconcat} for further details.

\section{Standard Entry Types}
\label{sec:standardentry}

\entrysection{string}

The standard \atentry{string} is available and can be used to define
variables that may be used in field values. Don't include braces or
double-quote delimiters when referencing a variable. You can use
\idx{stringconcat} to concatenate strings.
For example:
\begin{codeenv}
\atentry{string}\marg{ssi=\marg{server-side includes}}
\atentry{string}\marg{html=\marg{hypertext markup language}}
\strut
\atentry{abbreviation}\marg{shtml,
  \field{short}=\qtdelim{shtml},
  \field{long}=ssi \idx{stringconcat} \qtdelim{ enabled } \idx{stringconcat} html,
  \field{see}=\marg{ssi,html}
}
\strut
\atentry{abbreviation}\marg{html,
  \field{short}=\qtdelim{html},
  \field{long}=html
}
\strut
\atentry{abbreviation}\marg{ssi,
  \field{short}=\qtdelim{ssi},
  \field{long}=ssi
}
\end{codeenv}
Note the difference between \code{=\qtdelim{ssi}} (a field value delimited by
double-quotes), the undelimited \code{=ssi} (a reference to the
variable), the grouped \code{=\marg{ssi,html}} (a field value
delimited by braces) and \code{ssi} the entry label.

\entrysection{preamble}

The standard \atentry{preamble} is available and can be used to
provide command definitions used within field values.
For example:
\begin{codeenv}
\atentry{preamble}\marg{\qtdelim{\cs{providecommand}\marg{\cmd{mtx}}[1]\marg{\cs{boldsymbol}\marg{\idx{param}1}}}}
\strut
\atentry{entry}\marg{matrix,
  \field{name}=\marg{matrix},
  \field{plural}=\marg{matrices},
  \field{description}=\marg{rectangular array of values, denoted \idx{mshiftchar}\cmd{mtx}\marg{M}\idx{mshiftchar}}
}
\end{codeenv}
Alternatively you can use \ics{glsxtrprovidecommand} which behaves
the same as \ics{providecommand} within the document but behaves
like \ics{renewcommand} within \bibgls, which allows you to change
\bibgls's internal definition of a command without affecting the
definition within the document (if it's already been defined before
the resource file is input). In general, it's best to just use
\cs{providecommand}.

The \texparserlib\ used by \bibgls\ will parse the contents of
\atentry{preamble} before trying to interpret the field value used
as a \hyperref[sec:fallbacks]{fallback} when \field{sort} is omitted (unless
\csopt[false]{interpret-preamble} is set in the resource options).
For example:
\begin{codeenv}
\atentry{preamble}\marg{\qtdelim{\cs{providecommand}\marg{\cmd{set}}[1]\marg{\cs{mathcal}\marg{\idx{param}1}}
\cs{providecommand}\marg{\cmd{card}}[1]\marg{|\cmd{set}\marg{\idx{param}1}|}}}
\strut
\atentry{entry}\marg{S,
  \field{name}=\marg{\marg{}\idx{mshiftchar}\cmd{set}\marg{S}\idx{mshiftchar}},
  \field{text}=\marg{\cmd{set}\marg{S}},
  \field{description}=\marg{a set}
}
\atentry{entry}\marg{card,
  \field{name}=\marg{\marg{}\idx{mshiftchar}\cmd{card}\marg{S}\idx{mshiftchar}},
  \field{text}=\marg{\cmd{card}\marg{S}},
  \field{description}=\marg{the cardinality of \cs{gls}\marg{S}}
}
\end{codeenv}
Neither entry has the \field{sort} field, so \bibgls\ has to fall
back on the \field{name} field and, since this contains the special
characters \idx{escchar} (backslash), \idx{mshiftchar} (maths
shift), \idx{bgroupchar} (begin group) and
\idx{egroupchar} (end group), the \texparserlib\ is used to interpret it.
The definitions provided by \atentry{preamble} allow \bibgls\ to
deduce that the \field{sort} value of the \code{S} entry is just
\code{S} and the \field{sort} value of the \code{card} entry is
\verb"|S|" (see \sectionref{sec:texparserlib}).

What happens if you also need to use these commands in the document?
The definitions provided in \atentry{preamble} won't be available
until the \iext{glstex} file has been created, which means the
commands won't be defined on the first \LaTeX\ run.

There are several approaches:
\begin{enumerate}
\item Just define the commands in the document. This means the
commands are available, but \bibgls\ won't be able to correctly
interpret the \field{name} fields.

\item Define the commands in both the document and in
\atentry{preamble}. For example:
\begin{codeenv}
\cs{newcommand}\marg{\cmd{set}}[1]\marg{\cs{mathcal}\marg{\idx{param}1}}
\cs{newcommand}\marg{\cmd{card}}[1]\marg{|\cmd{set}\marg{\idx{param}1}|}
\gls{GlsXtrLoadResources}\oarg{\csopt[my-data]{src}}
\end{codeenv}
Alternatively:
\begin{codeenv}
\gls{GlsXtrLoadResources}\oarg{\csopt[my-data]{src}}
\cs{providecommand}\marg{\cmd{set}}[1]\marg{\cs{mathcal}\marg{\idx{param}1}}
\cs{providecommand}\marg{\cmd{card}}[1]\marg{|\cmd{set}\marg{\idx{param}1}|}
\end{codeenv}
If the provided definitions match those given in the \ext{bib} file,
there's no difference. If they don't match then in the first example 
the document definitions will take precedence (but the interpreter
will use the \atentry{preamble} definitions) and in the second
example the \atentry{preamble} definitions will take precedence.
For example, the document may define \csfmt{card} as:
\begin{codeenv}
\cs{newcommand}\marg{\cmd{card}}[1]\marg{\ics{vert}\cmd{set}\marg{\idx{param}1}\cs{vert}}
\end{codeenv}

\item Make use of \gls{glsxtrfmt} provided by \styfmt{glossaries-extra} which
allows you to store the name of the formatting command in a field.
The default is the \field{user1} field, but this can be changed to 
another field by redefining \ics{GlsXtrFmtField}.

The \ext{bib} file can now look like this:
\begin{codeenv}
\atentry{preamble}\marg{\qtdelim{\cs{providecommand}\marg{\cmd{set}}[1]\marg{\cs{mathcal}\marg{\idx{param}1}}
\cs{providecommand}\marg{\cmd{card}}[1]\marg{|\cmd{set}\marg{\idx{param}1}|}}}
\strut
\atentry{symbol}\marg{S,
  \field{name}=\marg{\marg{}\idx{mshiftchar}\cmd{set}\marg{S}\idx{mshiftchar}},
  \field{text}=\marg{\cmd{set}\marg{S}},
  \field{user1}=\marg{set},
  \field{description}=\marg{a set}
}
\atentry{symbol}\marg{cardS,
  \field{name}=\marg{\marg{}\idx{mshiftchar}\cmd{card}\marg{S}\idx{mshiftchar}},
  \field{text}=\marg{\cmd{card}\marg{S}},
  \field{user1}=\marg{card},
  \field{description}=\marg{the cardinality of \cs{gls}\marg{S}}
}
\end{codeenv}
Within the document, you can format \meta{text} using the formatting
command provided in the \field{user1} field with:
\nosecformatdef{glsxtrfmt}
(which internally uses \ics{glslink}) or
\nosecformatdef{glsxtrentryfmt}
which just applies the appropriate formatting command to
\meta{text}. Version 1.23+ of \sty{glossaries-extra} also provides a
starred form of the linking command:
\nosecformatdef{glsxtrfmt*}
which inserts additional material inside the link text but outside
the formatting command.

If the entry given by \meta{label} hasn't been defined,
then \gls!{glsxtrfmt} just does \meta{text} (followed by \meta{insert} for
the starred version) and a warning is issued. (There's no warning 
if the entry is defined but the field hasn't been set.)
The \meta{options} are as for \ics{glslink} but \ics{glslink} will
actually be using:
\begin{codeenv}
\ics{glslink}\oarg{\meta{def-options},\meta{options}}\margm{label}\marg{\cmd{}\meta{csname}\margm{text}\meta{insert}}
\end{codeenv}
where the default options \meta{def-options} are given by
\ics{GlsXtrFmtDefaultOptions}. The default definition of this is
just \code{noindex} which suppresses the automatic indexing or
recording action. (See the \sty{glossaries-extra}
manual~\cite{glossaries-extra} for further details.) The
\meta{insert} part is omitted for the unstarred form.

This means that the document doesn't need to actually provide
\verb|\set| or \verb|\card| but can instead use, for example,
\begin{codeenv}
\gls{glsxtrfmt}\marg{S}\marg{A}
\gls{glsxtrentryfmt}\marg{cardS}\marg{B}
\end{codeenv}
instead of:
\begin{verbatim}
\set{A}
\card{B}
\end{verbatim}
The first \LaTeX\ run will simply ignore the formatting and produce
a warning.

Since this is a bit cumbersome to write, you can provide shortcut
commands. For example:
\begin{codeenv}
\gls{GlsXtrLoadResources}\oarg{\csopt[my-data]{src}}
\cs{newcommand}\marg{\cmd{gset}}[2][]\marg{\gls{glsxtrfmt}\oarg{\idx{param}1}\marg{S}\marg{\idx{param}2}}
\cs{newcommand}\marg{\cmd{gcard}}[2][]\marg{\gls{glsxtrfmt}\oarg{\idx{param}1}\marg{cardS}\marg{\idx{param}2}}
\end{codeenv}
Whilst this doesn't seem a great deal different from simply
providing the definitions of \csfmt{set} and \csfmt{card} in the
document, this means you don't have to worry about remembering 
the names of the actual commands provided in the \ext{bib} file
(just the entry labels) and the use of \gls{glsxtrfmt} will
automatically produce a hyperlink to the glossary entry if the
\isty{hyperref} package has been loaded.
\end{enumerate}

Here's an alternative \ext{bib} that defines entries with a term, a
description and a symbol:
\begin{codeenv}
\atentry{preamble}\marg{\qtdelim{\cs{providecommand}\marg{\cmd{setfmt}}[1]\marg{\cs{mathcal}\marg{\idx{param}1}}
\cs{providecommand}\marg{\cmd{cardfmt}}[1]\marg{|\cmd{setfmt}\marg{\idx{param}1}|}}}
\strut
\atentry{entry}\marg{set,
  \field{name}=\marg{set},
  \field{symbol}=\marg{\cmd{setfmt}\marg{S}},
  \field{user1}=\marg{setfmt},
  \field{description}=\marg{collection of values}
}
\atentry{entry}\marg{cardinality,
  \field{name}=\marg{cardinality},
  \field{symbol}=\marg{\cmd{cardfmt}\marg{S}},
  \field{user1}=\marg{cardfmt},
  \field{description}=\marg{the number of elements in the \cs{gls}\marg{set} \idx{mshiftchar}\cs{glssymbol}\marg{set}\idx{mshiftchar}}
}
\end{codeenv}
I've changed the entry labels and the names of the formatting commands.
The definitions in the document need to reflect the change in label
but not the change in the formatting commands:
\begin{codeenv}
\cs{newcommand}\marg{\cmd{gset}}[2][]\marg{\gls{glsxtrfmt}\oarg{\idx{param}1}\marg{set}\marg{\idx{param}2}}
\cs{newcommand}\marg{\cmd{gcard}}[2][]\marg{\gls{glsxtrfmt}\oarg{\idx{param}1}\marg{cardinality}\marg{\idx{param}2}}
\end{codeenv}

Here's another approach that allows for a more complicated argument
for the cardinality. (For example, if the argument is an expression
involving set unions or intersections.)
The \ext{bib} file is now:
\begin{codeenv}
\atentry{preamble}\marg{\qtdelim{\cs{providecommand}\marg{\cmd{setfmt}}[1]\marg{\cs{mathcal}\marg{\idx{param}1}}
\cs{providecommand}\marg{\cmd{cardfmt}}[1]\marg{|\idx{param}1|}}}
\strut
\atentry{entry}\marg{set,
  \field{name}=\marg{set},
  \field{symbol}=\marg{\cmd{setfmt}\marg{S}},
  \field{user1}=\marg{setfmt},
  \field{description}=\marg{collection of values}
}
\atentry{entry}\marg{cardinality,
  \field{name}=\marg{cardinality},
  \field{symbol}=\marg{\cmd{cardfmt}\marg{\cmd{setfmt}\marg{S}}},
  \field{user1}=\marg{cardfmt},
  \field{description}=\marg{the number of elements in the \cs{gls}\marg{set} \idx{mshiftchar}\cs{glssymbol}\marg{set}\idx{mshiftchar}}
}
\end{codeenv}
This has removed the \csfmt{setfmt} command from the definition of
\csfmt{cardfmt}. Now the definitions in the document are:
\begin{codeenv}
\cs{newcommand}\marg{\cmd{gset}}[1]\marg{\gls{glsxtrentryfmt}\marg{set}\marg{\idx{param}1}}
\cs{newcommand}\marg{\cmd{gcard}}[2][]\marg{\gls{glsxtrfmt}\oarg{\idx{param}1}\marg{cardinality}\marg{\idx{param}2}}
\end{codeenv}
This allows for code such as:
\begin{verbatim}
\[ \gcard{\gset{A} \cap \gset{B}} \]
\end{verbatim}
which will link back to the \code{cardinality} entry in the
glossary and avoids any hyperlinking with \csfmt{gset}.
Alternatively to avoid links with \csfmt{gcard} as well:
\begin{codeenv}
\cs{newcommand}\marg{\cmd{gset}}[1]\marg{\gls{glsxtrentryfmt}\marg{set}\marg{\idx{param}1}}
\cs{newcommand}\marg{\cmd{gcard}}[1]\marg{\gls{glsxtrentryfmt}\marg{cardinality}\marg{\idx{param}1}}
\end{codeenv}
Now \csfmt{gset} and \csfmt{gcard} are simply formatting commands,
but their actual definitions are determined in the \ext{bib} file.

\section{Single Entry Types}
\label{sec:singleentry}

The \glspl{entrytype} described in this section create a single glossary
definition per entry (from \styfmt{glossaries-extra}'s point of view). 
For example:
\begin{codeenv}
\atentry{entry}\marg{matrix,
  \field{name}=\marg{matrix},
  \field{plural}=\marg{matrices},
  \field{description}=\marg{rectangular array of values}
}
\end{codeenv}
is analogous to:
\begin{codeenv}
\gls{newglossaryentry}\marg{matrix}\comment{label}
\marg{\comment{fields}
  \field{name}=\marg{matrix},
  \field{plural}=\marg{matrices},
  \field{description}=\marg{rectangular array of values}
}
\end{codeenv}
The \csopt{secondary} option allows the creation of a fake glossary
with the entry labels in its internal list in a different order.
This means that the same data can be displayed in two separate lists
without duplicating the resources required by each glossary entry.

\Sectionref{sec:dualentry} describes \bibgls\ \glspl{entrytype}
that create two separate (but related) \styfmt{glossaries-extra}
definitions per \ext{bib} entry.

\entrysection{entry}

Regular terms are defined by the \atentry{entry} field. This requires the
\field{description} field and either \field{name} or \field{parent}.
For example:
\begin{codeenv}
\atentry{preamble}\marg{\qtdelim{\cs{providecommand}\marg{\cmd{mtx}}[1]\marg{\cs{boldsymbol}\marg{\idx{param}1}}}}
\strut
\atentry{entry}\marg{matrix,
  \field{name}=\marg{matrix},
  \field{plural}=\marg{matrices},
  \field{description}=\marg{rectangular array of values, denoted \cs{gls}\marg{M}},
  \field{seealso}=\marg{vector}
}
\strut
\atentry{entry}\marg{M,
  \field{name}=\marg{\cs{ensuremath}\marg{M}},
  \field{description}=\marg{a \cs{gls}\marg{matrix}}
}
\strut
\atentry{entry}\marg{vector,
  \field{name} = \qtdelim{vector},
  \field{description} = \marg{column or row of values, denoted \cs{gls}\marg{v}},
  \field{seealso}=\marg{matrix}
}
\strut
\atentry{entry}\marg{v,
  \field{name}=\marg{\cs{ensuremath}\marg{\cs{vec}\marg{v}}},
  \field{description}=\marg{a \cs{gls}\marg{vector}}
}
\end{codeenv}

If the \field{sort} field is missing the default is obtained from
the \field{name} field (unless overridden by options like
\csopt{entry-sort-fallback}). For \hierarchical\ entries, if the
\field{name} field is omitted it will be obtained from the 
\glsdisp{parententry}{parent's} \field{name}.
See \sectionref{sec:fallbacks}.

Terms defined using \atentry{entry} will be written to the output
(\ext{glstex}) file using the command \gls!{bibglsnewentry}.

\entrysection{symbol}

The \atentry{symbol} \gls{entrytype} is much like \atentry{entry}, but it's
designed specifically for symbols, so in the previous example, the
\code{M} and \code{v} terms would be better defined using the
\atentry{symbol} \gls{entrytype} instead. For example:
\begin{codeenv}
\atentry{symbol}\marg{M,
  \field{name}=\marg{\cs{ensuremath}\marg{M}},
  \field{description}=\marg{a \cs{gls}\marg{matrix}}
}
\end{codeenv}

The required fields are \field{name} or \field{parent}. The
\field{description} field is required if the \field{name} field is
missing. If the \field{sort} field is omitted, the default fallback is
given by the entry label (unless overridden by options like
\csopt{symbol-sort-fallback}). Note that this is different from
\atentry{entry} where the sort defaults to \field{name} if omitted.
See \sectionref{sec:fallbacks}.

Terms that are defined using \atentry{symbol} will be written to
the output file using the command \gls!{bibglsnewsymbol}.

\entrysection{number}

The \atentry{number} \gls{entrytype} is like \atentry{symbol}, but it's for
numbers. The numbers don't have to be explicit digits and may have a
symbolic representation. There's no real difference between the
behaviour of \atentry{number} and \atentry{symbol} except that terms
defined using \atentry{number} will be written to the output file
using the command \gls!{bibglsnewnumber}.

For example, the file \filefmt{constants.bib} might define
mathematical constants like this:
\begin{codeenv}
\atentry{number}\marg{pi,
   \field{name}=\marg{\cs{ensuremath}\marg{\ics{pi}}},
   \field{description}=\marg{the ratio of the length of the circumference
    of a circle to its diameter},
   \field{user1}=\marg{3.14159}
}
\strut
\atentry{number}\marg{e,
  \field{name}=\marg{\cs{ensuremath}\marg{e}},
  \field{description}=\marg{base of natural logarithms},
  \field{user1}=\marg{2.71828}
}
\end{codeenv}
This stores the approximate value in the \field{user1} field. This
can be used to sort the entries in numerical order according to the
values rather than the symbols:
\begin{codeenv}
\gls{GlsXtrLoadResources}\oarg{
  \csopt[constants]{src},\comment{constants.bib}
  \csopt[number]{category},\comment{set the category for all selected entries}
  \csopt[double]{sort},\comment{numerical double-precision sort}
  \csopt[user1]{sort-field}\comment{sort according to 'user1' field}
}
\end{codeenv}
The \csopt[number]{category} option makes it easy to adjust the
glossary format to include the \field{user1} field:
\begin{codeenv}
\ics{glsdefpostdesc}\marg{number}\marg{\comment{}
  \cs{ifglshasfield}\marg{useri}\marg{\cs{glscurrententrylabel}}\comment{}
  \marg{ (approximate value: \cs{glscurrentfieldvalue})}\comment{}
  \marg{}\comment{}
}
\end{codeenv}

\entrysection{index}

The \atentry{index} \gls{entrytype} is designed for entries that don't
have a description. Only the label is required. If \field{name} is
omitted, it's assumed to be the same as the label, even if
\field{parent} is present. (Note this is different to the fallback
behaviour of \atentry{entry}, which fetches the name from the 
\gls{parententry}.) If the name contains any characters that can't be used in
the label, you must use the \field{name} field. If the \field{sort}
field is missing the default fallback is obtained from the \field{name}.
Note that the \atentry{index} \gls{entrytype} is \emph{not} governed by
\csopt{entry-sort-fallback} (but it is governed by
\csopt{custom-sort-fallbacks}). This allows \atentry{index} and
\atentry{entry} to have different fallbacks if the \field{sort}
field is missing.
See \sectionref{sec:fallbacks}.

Example:
\begin{codeenv}
\atentry{index}\marg{duck}
\atentry{index}\marg{goose,\field{plural}=\marg{geese}}
\atentry{index}\marg{sealion,\field{name}=\marg{sea lion}}
\atentry{index}\marg{facade,\field{name}=\marg{fa\ics{c}\marg{c}ade}}
\end{codeenv}

Terms that are defined using \atentry{index} will be written to the output
file using the command \gls!{bibglsnewindex}.

\entrysection{indexplural}

The \atentry{indexplural} \gls{entrytype} is similar to the
\atentry{index} \gls{entrytype} except that the \field{name} field, if
missing, is obtained from the \field{plural} field. If the
\field{plural} field is missing it's obtained from the \field{text}
field with the plural suffix appended. If the \field{text} field is
missing, it's obtained from the original entry label. If the
\field{sort} field is missing the default is obtained from the
\field{name} field. (As with \atentry{index}, \atentry{indexplural}
is \emph{not} governed by \csopt{entry-sort-fallback}, but it is
governed by \csopt{custom-sort-fallbacks}.) 
See \sectionref{sec:fallbacks}.
All fields are optional. For example:
\begin{codeenv}
\atentry{indexplural}\marg{goose,
  \field{plural} = \marg{geese}
}
\strut
\atentry{indexplural}\marg{duck}
\strut
\atentry{indexplural}\marg{chateau,
  \field{text} = \marg{ch\ics{cs.circum}ateau},
  \field{plural} = \marg{ch\cs{cs.circum}ateaux}
}
\end{codeenv}
This is equivalent to:
\begin{codeenv}
\atentry{indexplural}\marg{goose,
  \field{name} = \marg{geese},
  \field{text} = \marg{goose},
  \field{plural} = \marg{geese}
}
\strut
\atentry{indexplural}\marg{duck,
  \field{name} = \marg{ducks},
  \field{text} = \marg{duck},
  \field{plural} = \marg{ducks}
}
\strut
\atentry{indexplural}\marg{chateau,
  \field{name} = \marg{ch\cs{cs.circum}ateaux},
  \field{text} = {ch\cs{cs.circum}ateau},
  \field{plural} = {ch\cs{cs.circum}ateaux}
}
\end{codeenv}

Terms that are defined using \atentry{indexplural} will be written to the output
file using the command \gls!{bibglsnewindexplural}.

\entrysection{abbreviation}

The \atentry{abbreviation} \gls{entrytype} is designed for abbreviations.
The required fields are \field{short} and \field{long}. If the
\field{sort} key is missing, \bibgls\ will use the field given by
\csopt{abbreviation-sort-fallback}, which defaults to the \field{short} field.
(If you want an equivalent of \gls!{newdualentry}, use
\atentry{dualabbreviationentry} instead.)

If you use \csopt[name]{sort-field} (rather than the default
\csopt[sort]{sort-field}), then the fallback for the
\field{name} field is always the \field{short} field, regardless of
the \csopt{abbreviation-sort-fallback} setting, unless you use
\csopt{abbreviation-name-fallback} to change the fallback for the
\field{name} field.
See \sectionref{sec:fallbacks}.

Note that you must set the abbreviation style before loading the
resource file to ensure that the abbreviations are defined
correctly, however \bibgls\ has no knowledge of the abbreviation
style so it doesn't know if the \field{description} field must be
included or if the default \field{sort} value isn't simply the value
of the \field{short} field.

You can instruct \bibgls\ to sort by the \field{long} field instead
using \csopt[long]{abbreviation-sort-fallback}. You can also tell \bibgls\ to
ignore certain fields using \csopt{ignore-fields}, so you can
include a \field{description} field in the \ext{bib} file if
you sometimes need it, and then instruct \bibgls\ to ignore it when
you don't want it.

For example:
\begin{codeenv}
\atentry{abbreviation}\marg{html,
  \field{short} = \marg{html},
  \field{long}  = \marg{hypertext markup language},
  \field{description} = \marg{a markup language for creating web pages}
}
\end{codeenv}
If you want the \abbrstyle{long-noshort-desc} style, then you can put
the following in your document (where the \ext{bib} file is called
\filefmt{entries-abbrv.bib}):
\begin{codeenv}
\ics{setabbreviationstyle}\marg{long-noshort-desc}
\gls{GlsXtrLoadResources}\oarg{\csopt[entries-abbrv]{src},
 \csopt[long]{abbreviation-sort-fallback}}
\end{codeenv}
Whereas, if you want the \abbrstyle{long-short-sc} style, then you can
instead do:
\begin{codeenv}
\cs{setabbreviationstyle}\marg{long-short-sc}
\gls{GlsXtrLoadResources}\oarg{\csopt[entries-abbrv]{src},\csopt[description]{ignore-fields}}
\end{codeenv}
or to convert the short value to \idx{uppercase} and use the
\abbrstyle{long-short-sm} style instead:
\begin{codeenv}
\cs{setabbreviationstyle}\marg{long-short-sm}
\gls{GlsXtrLoadResources}\oarg{\csopt[entries-abbrv]{src},
 \csopt[uc]{short-case-change},\comment{convert short value to upper case}
 \csopt[description]{ignore-fields}}
\end{codeenv}

Case-changing can be applied with \csopt{short-case-change} to
convert the case of the \field{short} field, as illustrated above. 
If you use a style that obtains the \field{description} from the
\field{long} form, but you want to apply a case-change to the
\field{description} field with \csopt{description-case-change}, then
you can copy the \field{long} field to the \field{description} with
\csopt[\field{long}=\field{description}]{replicate-fields}.

For example, if \filefmt{entries-abbrv.bib} contains:
\begin{codeenv}
\atentry{abbreviation}\marg{html,
  \field{short} = \marg{html},
  \field{long}  = \marg{hypertext markup language}
}
\end{codeenv}
then the document may include:
\begin{codeenv}
\cs{setabbreviationstyle}\marg{long-short-sc}
\gls{GlsXtrLoadResources}\oarg{\csopt[entries-abbrv]{src},
 \csopt[firstuc]{description-case-change},
 \csopt[\field{long}=\field{description}]{replicate-fields}}
\end{codeenv}
Note that this can cause a problem for styles that set the
\field{description} field to the \field{long} form encapsulated by a
style command (such as with the \abbrstyle{long-em-short-em} style)
as this will override the style setting.

Similarly, if you want to change the case of the \field{name} field:
\begin{codeenv}
\cs{setabbreviationstyle}\marg{long-short-sc}
\gls{GlsXtrLoadResources}\oarg{\csopt[entries-abbrv]{src},
 \csopt[firstuc]{description-case-change},
 \csopt[uc]{name-case-change},
 \csopt[\field{long}=\field{description},\field{short}=\field{name}]{replicate-fields}}
\end{codeenv}
Again, this will lose any custom formatting command that would
usually be applied by the abbreviation style to the \field{name}
field (and \field{description}, if applicable).

Terms defined using \atentry{abbreviation} will be written to the output
file using the command \gls!{bibglsnewabbreviation}.

\entrysection{acronym}

The \atentry{acronym} \gls{entrytype} is like \atentry{abbreviation} except that
the term is written to the output file using the command
\gls!{bibglsnewacronym}.

\entrysection{contributor}

The \atentry{contributor} \gls{entrytype} is primarily provided for
use by the \atentry{bibtexentry} type. You may use it explicitly if
you want, but you need to take care that it doesn't clash with
\atentry{bibtexentry}. It behaves much like \atentry{index} except
that the term is written to the \ext{glstex} file using the command
\gls!{bibglsnewcontributor}. There are no required fields. As with
\atentry{index}, if the \field{name} field is missing, the fallback
value is the entry's label (see \sectionref{sec:fallbacks}). When
this \gls{entrytype} is automatically created by
\atentry{bibtexentry}, the \field{name} is set to 
\begin{codeenv}
\gls{bibglscontributor}\margm{forenames}\margm{von}\margm{surname}\margm{suffix}
\end{codeenv}

If you do explicitly use \atentry{contributor} you need to make sure it's
defined \emph{before} the first instance of \atentry{bibtexentry} that
tries to access it, but within the same resource set. If you ensure that 
the label of \atentry{contributor} matches the contributor label generated by
\atentry{bibtexentry} then they can have their dependency lists
updated, and
the \field{bibtexentry} and \field{bibtexentry@entrytype} internal
fields can be set for the
\atentry{contributor} entry.  For example:
\begin{codeenv}
\atentry{contributor}\marg{KnuthDonaldE,
  \field{name}=\marg{\gls{bibglscontributor}\marg{Donald E.}\marg{}\marg{Knuth}\marg{}},
  \field{description}=\marg{Famous mathematician and computer scientist who
   created \cmd{TeX}}
}
\strut
@book\marg{texbook,
  title = \marg{The {\cmd{TeX} book}},
  author = \marg{Donald E. Knuth},
  publisher = \marg{Addison-Wesley},
  year = 1986
}
\end{codeenv}
The resource options then need to include:
\begin{codeenv}
 \csopt[\ics{GlsXtrBibTeXEntryAliases}]{entry-type-aliases},
 \csopt[
  \marg{[ \cs{cs.string}\csfmt{-}\cs{cs.string}\csfmt{.}]}\marg{}
 ]{labelify-replace}
\end{codeenv}

If the \atentry{contributor} entry is deferred until after the
corresponding \atentry{bibtexentry} then you will end up with a label clash.

\section{Dual Entry Types}
\label{sec:dualentry}

The \glspl{entrytype} described in this section create two separate (but
related) \styfmt{glossaries-extra} entry definitions per \ext{bib}
entry. The first of these entries is considered the
\igls{primaryentry}, and the second is the \igls{dualentry}.
The naming scheme is \code{@dual}\meta{entry-type} where both the
\primary\ and \dual\ are considered to have the same type of entry
(such as \atentry{dualsymbol} where both the \primary\ and \dual\
are functionally like \atentry{symbol}) or
\code{@dual}\meta{primary}\meta{dual} where the \primary\ is
functionally like \code{@}\meta{primary} and the \dual\ is
functionally like \code{@}\meta{dual}.

If you need a field to store the \glsdisp{dualentry}{dual}
description in (and you're not simply swapping known fields around),
then you can use the special \field{dualdescription} field and add
it to your map.

If the fields provided by the \isty{glossaries-prefix} are defined,
there will be additional mappings for the special internal fields
\field{dualprefix}, \field{dualprefixfirst},
\field{dualprefixplural}, and \field{dualprefixfirstplural}.

For example:
\begin{codeenv}
\atentry{dualabbreviationentry}\marg{svm,
  \field{short} = \marg{SVM},
  \field{long} = \marg{support vector machine},
  \field{description} = \marg{statistical pattern recognition technique}
}
\end{codeenv}
is like:
\begin{codeenv}
\atentry{abbreviation}\marg{svm,
  \field{short} = \marg{SVM},
  \field{long} = \marg{support vector machine},
}
\atentry{entry}\marg{dual.svm,
  \field{text} = \marg{SVM},
  \field{name} = \marg{support vector machine},
  \field{description} = \marg{statistical pattern recognition technique}
}
\end{codeenv}
and is analogous to:
\begin{codeenv}
\gls{newabbreviation}\marg{svm}\marg{SVM}\marg{support vector machine}
\gls{newglossaryentry}\marg{dual.svm}\marg{\field{name}=\marg{support vector machine},\field{text}=\marg{SVM},
 \field{description}=\marg{statistical pattern recognition technique}}
\end{codeenv}
but both entries are considered dependent on each other. This means
that if you only reference the \gls{primaryentry} (using \ics{gls} etc)
then the \gls{dualentry} will still be selected if the \csopt{selection}
setting includes dependencies.

The creation of the \gls{dualentry} involves mapping or copying
fields from the \gls{primaryentry}. Each \gls{dualentry} type has a
set of mappings.  If a field in the set of mappings is missing, its
fallback value is used (see \sectionref{sec:fallbacks}).  Any fields
that aren't listed in the mappings are simply copied, except for the
\field{alias} field, which will never be copied to the
\gls{dualentry}, nor can it be mapped. The alias will only apply to
the \gls{primaryentry}.  The \gls{dualentry} is given the label
\meta{prefix}\meta{id} where \meta{prefix} is set by the
\csopt{dual-prefix} option and \meta{id} is the label supplied in
the \ext{bib} file.

If \csopt[combine]{dual-sort} then the \glspl{dualentry} will be sorted
along with the \glspl{primaryentry}, otherwise the \csopt{dual-sort}
indicates how to sort the dual entries and the dual entries will be
appended to the end of the \ext{glstex} file. The
\csopt{dual-sort-field} determines what field to use for the sort
value if the dual entries should be sorted separately.

Take care if you have a mixture of \glspl{entrytype} (such as 
\atentry{dualindexentry}, \atentry{dualindexsymbol} 
and \atentry{index}) and you're not
using the default \csopt[combine]{dual-sort}. Remember that the
\glspl{primaryentry} are all sorted together along with the single
entries types described in \sectionref{sec:dualentry} (but they may
be assigned to different glossary types), and then the dual entries
are sorted together (but may be assigned to different glossary
types). This may result in an odd ordering if some of the primaries
and some of the duals are assigned to the same glossary. For
example, don't mix \atentry{dualindexabbreviation} (duals are
abbreviations) with \atentry{dualabbreviationentry} (primaries
are abbreviations) when you aren't using \csopt[combine]{dual-sort}
(unless you have two different glossaries for the \primary\ vs
\dual\ abbreviations).

Remember that \bibgls\ is designed to take advantage of
\ics{printunsrtglossary}, which simply iterates over all defined
entries in the order in which they were defined (or, more precisely,
the order of the internal list of entry labels associated with that
glossary). The aim of \bibgls\ is to write the entry definitions to
the \ext{glstex} file so that the internal list of labels is in the
appropriate order.

For example, suppose the file \filefmt{entries.bib} contains:
\begin{codeenv}
\atentry{index}\marg{aardvark}
\atentry{index}\marg{mouse}
\atentry{index}\marg{zebra}
\atentry{dualindexabbreviation}\marg{xml,
 \field{short}=\marg{XML},
 \field{long}=\marg{extensible markup language}
}
\atentry{dualabbreviationentry}\marg{ssi,
  \field{short}=\marg{SSI},
  \field{long}=\marg{server-side includes},
  \field{description}=\marg{directives placed in \cs{gls}\marg{html} pages
   evaluated by the server}
}
\atentry{dualindexabbreviation}\marg{html,
 \field{short}=\marg{HTML},
 \field{long}=\marg{hypertext markup language}
}
\atentry{dualabbreviationentry}\marg{css,
 \field{short}=\marg{CSS},
 \field{long}=\marg{cascading stylesheets},
 \field{description}=\marg{a language that describes the style of an
  \cs{gls}\marg{html} document}
}
\end{codeenv}
This contains a mixture of \glspl{entrytype}, including
\atentry{dualindexabbreviation} (where the \dual\ is the
abbreviation) and \atentry{dualabbreviationentry} (where the
\primary\ is the abbreviation).

Now consider the following document:
\begin{codeenv}
\cmd{documentclass}\marg{article}
\strut
\cmd{usepackage}[\styopt{record},\styopt{abbreviations}]\marg{glossaries-extra}
\strut
\gls{GlsXtrLoadResources}\oarg{\csopt[all]{selection},\csopt[entries]{src}}
\strut
\cmd{begin}\marg{document}
\cs{printunsrtglossaries}
\cmd{end}\marg{document}
\end{codeenv}
This uses the default \csopt[combine]{sort}, so all the entries are
sorted together, resulting in the order: 
\code{aardvark},
\code{dual.css},
\code{css},
\code{html},
\code{dual.html},
\code{mouse},
\code{dual.ssi},
\code{ssi},
\code{xml},
\code{dual.xml},
\code{zebra}.

The \LaTeX\ code written to the \iext{glstex} file is essentially
(but not exactly):
\begin{codeenv}
\comment{from \atentry{index}\marg{aardvark}:}
\gls{newglossaryentry}\marg{aardvark}\marg{\field{name}=\marg{aardvark},\field{description}=\marg{}}
\strut
\comment{\dual\ of \atentry{dualabbreviationentry}\marg{css,\ldots}:}
\gls{newglossaryentry}\marg{dual.css}\marg{\field{name}=\marg{cascading stylesheets},\marg{text}=\marg{CSS},
 \field{description}=\marg{a language that describes the style of an
  \cs{glsxtrshort}\marg{html} document}}
\strut
\comment{\primary\ of \atentry{dualabbreviationentry}\marg{css,\ldots}:}
\gls{newabbreviation}\marg{css}\marg{CSS}\marg{cascading stylesheets}
\strut
\comment{\primary\ of \atentry{dualindexabbreviation}\marg{html,\ldots}:}
\gls{newglossaryentry}\marg{html}\marg{\field{name}=\marg{HTML},\field{description}=\marg{}}
\strut
\comment{\dual\ of \atentry{dualindexabbreviation}\marg{html,\ldots}:}
\gls{newabbreviation}\marg{dual.html}\marg{HTML}\marg{hypertext markup language}
\strut
\comment{from \atentry{index}\marg{mouse}:}
\gls{newglossaryentry}\marg{mouse}\marg{\marg{name}=\marg{mouse},\field{description}=\marg{}}
\strut
\comment{\dual\ of \atentry{dualabbreviationentry}\marg{ssi,\ldots}:}
\gls{newglossaryentry}\marg{dual.ssi}\marg{\field{name}=\marg{server-side includes},\field{text}=\marg{SSI},
 \field{description}=\marg{directives placed in \cs{glsxtrshort}\marg{html} pages
 evaluated by the server}}
\strut
\comment{\primary\ of \atentry{dualabbreviationentry}\marg{ssi,\ldots}:}
\gls{newabbreviation}\marg{ssi}\marg{SSI}\marg{server-side includes}
\strut
\comment{\primary\ of \atentry{dualindexabbreviation}\marg{xml,\ldots}:}
\gls{newglossaryentry}\marg{xml}\marg{\field{name}=\marg{XML},\field{description}=\marg{}}
\strut
\comment{\dual\ of \atentry{dualindexabbreviation}\marg{xml,\ldots}:}
\gls{newabbreviation}\marg{dual.xml}\marg{XML}\marg{extensible markup language}
\strut
\comment{from \atentry{index}\marg{zebra}:}
\gls{newglossaryentry}\marg{zebra}\marg{\field{name}=\marg{zebra},\field{description}=\marg{}}
\end{codeenv}
Since the document uses the \styopt{abbreviations} package option,
\gls{newabbreviation} automatically assigns the abbreviation 
to the \code{abbreviations} glossary (created through that package
option). This means that the \glsdisp{mainglossary}{\code{main} (default) glossary}
contains the entries (in order): 
\begin{itemize}
\item \code{aardvark} (name: aardvark),
\item \code{dual.css} (name: cascading stylesheets), 
\item \code{html} (name: HTML),
\item \code{mouse} (name: mouse), 
\item \code{dual.ssi} (name: server-side includes), 
\item \code{xml} (name: XML),
\item \code{zebra} (name: zebra).
\end{itemize}
The \code{abbreviations} glossary contains:
\begin{itemize}
\item \code{css} (short: CSS), 
\item \code{dual.html} (short: HTML),
\item \code{ssi} (short: SSI), 
\item \code{dual.xml} (short: XML).
\end{itemize}

Since all the entries were combined and sorted together, the
resulting glossaries are both ordered alphabetically (using
\field{short} for the abbreviations and \field{name} for the rest),
but note that you need to take care when referencing the
abbreviations if you want to make use of the abbreviation style. You
need \verb|\gls{css}| and \verb|\gls{ssi}| for the \primary\ 
abbreviations created with \atentry{dualabbreviationentry} and
\verb|\gls{dual.html}| and \verb|\gls{dual.xml}| for the \dual\ 
abbreviations created with \atentry{dualindexabbreviation}. Also
the \field{name} of the \primary\slash \dual\ alternative of the
abbreviations is also inconsistent (short form for \code{html} and
\code{xml} and long form for \code{dual.css} and \code{dual.ssi}), 
as different field mappings are used.

If the document is changed so that the \glspl{dualentry} are now
sorted and written after all the \glspl{primaryentry} have been dealt
with:
\begin{codeenv}
\gls{GlsXtrLoadResources}\oarg{
 \csopt[entries]{src},
 \csopt[letter-nocase]{dual-sort},
 \csopt[all]{selection}
}
\end{codeenv}
then \bibgls\ first orders the \glsdisp{primaryentry}{primaries}:
\begin{itemize}
\item \code{aardvark} (name: aardvark),
\item \code{css} (short: CSS),
\item \code{html} (name: HTML),
\item \code{mouse} (name: mouse),
\item \code{ssi} (short: SSI),
\item \code{xml} (name: XML),
\item \code{zebra} (name: zebra)
\end{itemize}
and writes them to the \ext{glstex} file
(functionally like):
\begin{codeenv}
\comment{from \atentry{index}\marg{aardvark}:}
\gls{newglossaryentry}\marg{aardvark}\marg{\field{name}=\marg{aardvark},\field{description}=\marg{}}
\strut
\comment{\primary\ of \atentry{dualabbreviationentry}\marg{css,\ldots}:}
\gls{newabbreviation}\marg{css}\marg{CSS}\marg{cascading stylesheets}
\strut
\comment{\primary\ of \atentry{dualindexabbreviation}\marg{html,\ldots}:}
\gls{newglossaryentry}\marg{html}\marg{\field{name}=\marg{HTML},\field{description}=\marg{}}
\strut
\comment{from \atentry{index}\marg{mouse}:}
\gls{newglossaryentry}\marg{mouse}\marg{\field{name}=\marg{mouse},\field{description}=\marg{}}
\strut
\comment{\primary\ of \atentry{dualabbreviationentry}\marg{ssi,\ldots}:}
\gls{newabbreviation}\marg{ssi}\marg{SSI}\marg{server-side includes}
\strut
\comment{\primary\ of \atentry{dualindexabbreviation}\marg{xml,\ldots}:}
\gls{newglossaryentry}\marg{xml}\marg{\field{name}=\marg{XML},\field{description}=\marg{}}
\strut
\comment{from \atentry{index}\marg{zebra}:}
\gls{newglossaryentry}\marg{zebra}\marg{\field{name}=\marg{zebra},\field{description}=\marg{}}
\end{codeenv}
Then \bibgls\ orders the \glsdisp{dualentry}{duals}:
\begin{itemize}
\item \code{dual.css} (name: cascading stylesheets),
\item \code{dual.html} (short: HTML),
\item \code{dual.ssi} (name: server-side includes),
\item \code{dual.xml} (short: XML)
\end{itemize}
and writes them to the \ext{glstex} file
(functionally like):
\begin{codeenv}
\comment{\dual\ of \atentry{dualabbreviationentry}\marg{css,\ldots}:}
\gls{newglossaryentry}\marg{dual.css}\marg{\field{name}=\marg{cascading stylesheets},\field{text}=\marg{CSS},
 \field{description}=\marg{a language that describes the style of an
  \cs{glsxtrshort}\marg{html} document}}
\strut
\comment{\dual\ of \atentry{dualindexabbreviation}\marg{html,\ldots}:}
\gls{newabbreviation}\marg{dual.html}\marg{HTML}\marg{hypertext markup language}
\strut
\comment{\dual\ of \atentry{dualabbreviationentry}\marg{ssi,\ldots}:}
\gls{newglossaryentry}\marg{dual.ssi}\marg{\field{name}=\marg{server-side includes},\field{text}=\marg{SSI},
 \field{description}=\marg{directives placed in \cs{glsxtrshort}\marg{html} pages
 evaluated by the server}}
\strut
\comment{\dual\ of \atentry{dualindexabbreviation}\marg{xml,\ldots}:}
\gls{newabbreviation}\marg{dual.xml}\marg{XML}\marg{extensible markup language}
\end{codeenv}
When the \ext{glstex} file is input (during the next \LaTeX\ run) 
the entries are defined in the order:
\begin{enumerate}
\item \code{aardvark} (type: \code{main}),
\item \code{css} (type: \code{abbreviations}),
\item \code{html} (type: \code{main}),
\item \code{mouse} (type: \code{main}),
\item \code{ssi} (type: \code{abbreviations}),
\item \code{xml} (type: \code{main}),
\item \code{zebra} (type: \code{main}),
\item \code{dual.css} (type: \code{main}),
\item \code{dual.html} (type: \code{abbreviations}),
\item \code{dual.ssi} (type: \code{main}),
\item \code{dual.xml} (type: \code{abbreviations}).
\end{enumerate}
This means that the \glsdisp{mainglossary}{\code{main} glossary's} internal list is in the
order:
\begin{itemize}
\item \code{aardvark} (aardvark),
\item \code{html} (HTML),
\item \code{mouse} (mouse),
\item \code{xml} (XML),
\item \code{zebra} (zebra), 
\item \code{dual.css} (cascading stylesheets), 
\item \code{dual.ssi} (server-side includes)
\end{itemize}
and the \code{abbreviations} glossary's internal list is in the order:
\begin{itemize}
\item \code{css} (CSS),
\item \code{ssi} (SSI),
\item \code{dual.html} (HTML), 
\item \code{dual.xml} (XML).
\end{itemize}
The lists are no longer in alphabetical order as they have a mixture
of \primary\ and \dual\ entries that were separated before sorting.

The above is a fairly contrived example as it wouldn't make sense
in a real document to have glossary terms (that include a
description) mixed with index terms (that don't include a
description). 
A better solution would be to use 
\atentry{tertiaryindexabbreviationentry} instead of
\atentry{dualabbreviationentry}.

\entrysection{dualentry}

The \atentry{dualentry} \gls{entrytype} is similar to \atentry{entry} but
actually defines two entries.
The \gls{dualentry} contains the same information as the \gls{primaryentry}
but some of the fields are swapped around.
The default mappings are:
\begin{itemize}
\item \field{name} $\mapsto$ \field{description}
\item \field{plural} $\mapsto$ \field{descriptionplural}
\item \field{description} $\mapsto$ \field{name}
\item \field{descriptionplural} $\mapsto$ \field{plural}
\end{itemize}
If the prefix fields are defined, then the default mappings
additionally include:
\begin{itemize}
\item \field{prefix} $\mapsto$ \field{dualprefix}
\item \field{prefixplural} $\mapsto$ \field{dualprefixplural}
\item \field{prefixfirst} $\mapsto$ \field{dualprefixfirst}
\item \field{prefixfirstplural} $\mapsto$ \field{dualprefixfirstplural}
\item \field{dualprefix} $\mapsto$ \field{prefix}
\item \field{dualprefixplural} $\mapsto$ \field{prefixplural}
\item \field{dualprefixfirst} $\mapsto$ \field{prefixfirst}
\item \field{dualprefixfirstplural} $\mapsto$ \field{prefixfirstplural}
\end{itemize}
The required fields are as for \atentry{entry}.

For example:
\begin{codeenv}
\atentry{dualentry}\marg{child,
  \field{name}=\marg{child},
  \field{plural}=\marg{children},
  \field{description}=\marg{enfant}
}
\end{codeenv}
is like:
\begin{codeenv}
\atentry{entry}\marg{child,
  \field{name}=\marg{child},
  \field{plural}=\marg{children},
  \field{description}=\marg{enfant}
  \field{descriptionplural}=\marg{enfants}
}
\strut
\atentry{entry}\marg{dual.child,
  \field{description}=\marg{child},
  \field{descriptionplural}=\marg{children},
  \field{name}=\marg{enfant}
  \field{plural}=\marg{enfants}
}
\end{codeenv}
where \idprefix{dual} is replaced by the value of the
\csopt{dual-prefix} option. However, instead of defining the entries
with \csfmt{bibglsnewentry} both the \primary\ and \dual\ entries are
defined using \gls!{bibglsnewdualentry}. The \field{category} 
and \field{type} fields can be set for the \gls{dualentry} using the
\csopt{dual-category} and \csopt{dual-type} options.

For example:
\begin{codeenv}
\cs{newglossary*}\marg{english}\marg{English}
\cs{newglossary*}\marg{french}\marg{French}
\strut
\gls{GlsXtrLoadResources}\oarg{
 \csopt[entries-dual]{src},\comment{data in entries-dual.bib}
 \csopt[english]{type},\comment{put \glspl{primaryentry} in glossary 'english'}
 \csopt[french]{dual-type},\comment{put \glspl{dualentry} in glossary 'french'}
 \csopt[dictionary]{category},\comment{set the \primary\ category to 'dictionary'}
 \csopt[dictionary]{dual-category},\comment{set the \dual\ category to 'dictionary'}
 \csopt[en]{sort},\comment{sort \glspl{primaryentry} according to language 'en'}
 \csopt[fr]{dual-sort}\comment{sort \glspl{dualentry} according to language 'fr'}
}
\end{codeenv}

If you need to keep the same name but have different descriptions
then you can use \field{dualdescription} and set up a mapping to use
it. For example:
\begin{codeenv}
\atentry{dualentry}\marg{sample,
  \field{name}=\marg{sample},
  \field{description}=\marg{\primary\ sample description},
  \field{dualdescription}=\marg{\dual\ sample description}
}
\end{codeenv}
The mapping can then be:
\begin{codeenv}
\csopt[\marg{\field{description}},
  \marg{\field{dualdescription}}]{dual-entry-map}
\end{codeenv}

\entrysection{dualindexentry}

There are no required fields.  The \gls{primaryentry} behaves like
\atentry{index} and the \gls{dualentry} behaves
like \atentry{entry}.  The default field mapping is:
\begin{itemize}
\item \field{name} $\mapsto$ \field{name}
\end{itemize}
If the prefix fields are defined, then the default mappings
additionally include:
\begin{itemize}
\item \field{prefix} $\mapsto$ \field{dualprefix}
\item \field{prefixplural} $\mapsto$ \field{dualprefixplural}
\item \field{prefixfirst} $\mapsto$ \field{dualprefixfirst}
\item \field{prefixfirstplural} $\mapsto$ \field{dualprefixfirstplural}
\item \field{dualprefix} $\mapsto$ \field{prefix}
\item \field{dualprefixplural} $\mapsto$ \field{prefixplural}
\item \field{dualprefixfirst} $\mapsto$ \field{prefixfirst}
\item \field{dualprefixfirstplural} $\mapsto$ \field{prefixfirstplural}
\end{itemize}
This doesn't actually perform any swapping of fields, but it
provides the field used for backlinks (if
\csopt{dual-indexentry-backlink} is set). The reason that the
\primary\ (rather than the \dual) is like \atentry{index} is to allow the
primaries to merge with any \atentry{index} entries found in the
resource set, since glossary entries with descriptions are likely to 
be a subset of all indexed entries.

If no \field{name} is given, the \gls{dualentry} is assigned the
(unprefixed) entry label. For example:
\begin{codeenv}
\atentry{dualindexentry}\marg{array,
  \field{description}=\marg{ordered list of values}
}
\end{codeenv}
This is effectively like:
\begin{codeenv}
\atentry{index}\marg{array}
\strut
\atentry{entry}\marg{dual.array,
  \field{name}=\marg{array},
  \field{description}=\marg{ordered list of values}
}
\end{codeenv}
The \glspl{primaryentry} are defined using
\gls!{bibglsnewdualindexentry},
which by default sets the \field{category} to \optfmt{index}
(although this may be overridden, for example, by the \csopt{category} option).
The \glspl{dualentry} are defined with
\gls!{bibglsnewdualindexentrysecondary}.

This is the most convenient way of having an entry that's also
automatically indexed. For example, suppose the file
\filefmt{terms.bib} contains:
\begin{codeenv}
\atentry{index}\marg{duck}
\atentry{index}\marg{zebra}
\atentry{index}\marg{aardvark}
\end{codeenv}
and suppose the file \filefmt{entries.bib} contains:
\begin{codeenv}
\atentry{dualindexentry}\marg{array,
  \field{description}=\marg{ordered list of values}
}
\strut
\atentry{dualindexentry}\marg{vector,
  \field{name}=\marg{vector},
  \field{description}=\marg{column or row of values}
}
\strut
\atentry{dualindexentry}\marg{set,
  \field{description}=\marg{collection of values}
}
\strut
\atentry{dualindexentry}\marg{matrix,
  \field{plural}=\marg{matrices},
  \field{description}=\marg{rectangular array of values}
}
\end{codeenv}
These entries can be used in an example document that has an 
index and a glossary:
\begin{codeenv}
\cmd{documentclass}\marg{article}
\strut
\cmd{usepackage}[colorlinks]\marg{hyperref}
\cmd{usepackage}[\styopt{record},\styopt{index},\styopt[mcols]{stylemods}]\marg{glossaries-extra}
\strut
\gls{GlsXtrLoadResources}\oarg{
 \csopt[terms,entries]{src},
 \csopt[index]{type},
 \csopt[idx.]{label-prefix},
 \csopt[gls.]{dual-prefix},
 \csopt[primary]{combine-dual-locations},
 \csopt[main]{dual-type}
}
\strut
\cmd{begin}\marg{document}
\cs{gls}\marg{gls.array}, \cs{gls}\marg{gls.vector}, \cs{gls}\marg{gls.set}, \cs{gls}\marg{gls.matrix}.
\strut
\cs{gls}\marg{idx.duck}, \cs{gls}\marg{idx.aardvark}, \cs{gls}\marg{idx.zebra}.
\strut
\cs{renewcommand}\marg{\cs{glstreenamefmt}}[1]\marg{\ics{textsc}\marg{\idx{param}1}}
\cs{printunsrtglossary}\oarg{\printglossopt[main]{type},\printglossopt[index]{style},\iprintglossopt{nogroupskip}}
\strut
\cs{renewcommand}\marg{\cs{glstreenamefmt}}[1]\marg{\idx{param}1}
\cs{renewcommand}\marg{\cs{glstreegroupheaderfmt}}[1]\marg{\cs{textbf}\marg{\idx{param}1}}
\cs{printunsrtglossary}\oarg{\printglossopt[index]{type},\printglossopt[mcolindexgroup]{style}}
\cmd{end}\marg{document}
\end{codeenv}
This uses \csopt{combine-dual-locations} to combine the \glspl{location}
for the \primary\ and \dual\ entries so that they only appear in the
index.

To avoid the inconvenience of remembering which prefix to use, you can
set up the prefixes with \ics{glsxtraddlabelprefix} and reference entries
with \ics{dgls}, \ics{dGls} etc instead of \cs{gls}, \cs{Gls} etc.

\entrysection{dualindexabbreviation}

The \atentry{dualindexabbreviation} \gls{entrytype} is similar to 
\atentry{dualindexentry} and again, by default, the
field mapping is:
\begin{itemize}
\item \field{name} $\mapsto$ \field{name}
\end{itemize}
If the prefix fields are defined, then the default mappings
additionally include:
\begin{itemize}
\item \field{prefix} $\mapsto$ \field{dualprefix}
\item \field{prefixplural} $\mapsto$ \field{dualprefixplural}
\item \field{prefixfirst} $\mapsto$ \field{dualprefixfirst}
\item \field{prefixfirstplural} $\mapsto$ \field{dualprefixfirstplural}
\item \field{dualprefix} $\mapsto$ \field{prefix}
\item \field{dualprefixplural} $\mapsto$ \field{prefixplural}
\item \field{dualprefixfirst} $\mapsto$ \field{prefixfirst}
\item \field{dualprefixfirstplural} $\mapsto$ \field{prefixfirstplural}
\end{itemize}
However in this case the required fields are \field{short} and
\field{long}. The \field{name} for the \gls{primaryentry} defaults to
\field{short} if omitted. (This may be changed with the
\csopt{abbreviation-name-fallback} option.) The fallback for the
\field{sort} field is given by \csopt{abbreviation-sort-fallback},
which defaults to the \field{short} field (see \sectionref{sec:fallbacks}).

For example:
\begin{codeenv}
\atentry{dualindexabbreviation}\marg{html,
  \field{short} = \marg{HTML},
  \field{long}  = \marg{hypertext markup language}
}
\end{codeenv}
is like:
\begin{codeenv}
\atentry{index}\marg{html,\field{name}=\marg{HTML}}
\strut
\atentry{abbreviation}\marg{dual.html,
  \field{short} = \marg{HTML},
  \field{long}  = \marg{hypertext markup language}
}
\end{codeenv}
The \primary\ term is defined using
\gls!{bibglsnewdualindexabbreviation}, which encapsulates the
\field{name} to match the font used by
the \dual\ abbreviation. The encapsulation command depends
on the \csopt{abbreviation-name-fallback} value. If it's
the \field{short} field then \gls{bibglsuseabbrvfont} is
used, otherwise \gls{bibglsuselongfont} is used.

The \primary\ definition also
by default sets the \field{category} to \optfmt{index} (although
this again may be overridden).
The \dual\ term is defined using
\gls!{bibglsnewdualindexabbreviationsecondary}.

\entrysection{dualindexsymbol}

The \atentry{dualindexsymbol} \gls{entrytype} is similar to 
\atentry{dualindexentry}, but by default the
field mappings are:
\begin{itemize}
\item \field{symbol} $\mapsto$ \field{name}
\item \field{name} $\mapsto$ \field{symbol}
\item \field{symbolplural} $\mapsto$ \field{plural}
\item \field{plural} $\mapsto$ \field{symbolplural}
\end{itemize}
If the prefix fields are defined, then the default mappings
additionally include:
\begin{itemize}
\item \field{prefix} $\mapsto$ \field{dualprefix}
\item \field{prefixplural} $\mapsto$ \field{dualprefixplural}
\item \field{prefixfirst} $\mapsto$ \field{dualprefixfirst}
\item \field{prefixfirstplural} $\mapsto$ \field{dualprefixfirstplural}
\item \field{dualprefix} $\mapsto$ \field{prefix}
\item \field{dualprefixplural} $\mapsto$ \field{prefixplural}
\item \field{dualprefixfirst} $\mapsto$ \field{prefixfirst}
\item \field{dualprefixfirstplural} $\mapsto$ \field{prefixfirstplural}
\end{itemize}
The required field is: \field{symbol}.
If the \field{name} field is omitted, the \dual\ entry is assigned a
symbol from the original (unprefixed) label.
The \glspl{primaryentry} are defined using
\gls!{bibglsnewdualindexsymbol},
which by default sets the \field{category} to \optfmt{index},
and the \glspl{dualentry} are defined using
\gls!{bibglsnewdualindexsymbolsecondary},
which by default sets the \field{category} to \optfmt{symbol}.
For example:
\begin{codeenv}
\atentry{dualindexsymbol}\marg{pi,
  \field{symbol}=\marg{\cs{ensuremath}\marg{\ics{pi}}},
  \field{description}=\marg{ratio of a circle's circumference to its diameter}
}
\end{codeenv}
is like:
\begin{codeenv}
\atentry{index}\marg{pi,\field{symbol}=\marg{\cs{ensuremath}\marg{\cs{pi}}}}
\strut
\atentry{symbol}\marg{dual.pi,
  \field{name}=\marg{\cs{ensuremath}\marg{\cs{pi}}},
  \field{symbol}=\marg{pi},
  \field{description}=\marg{ratio of a circle's circumference to its diameter}
}
\end{codeenv}

For example, suppose I have a file called \filefmt{symbols.bib} that
contains:
\begin{codeenv}
\atentry{dualindexsymbol}\marg{pi,
  \field{symbol}=\marg{\cs{ensuremath}\marg{\cs{pi}}},
  \field{description}=\marg{ratio of a circle's circumference to its diameter}
}
\strut
\atentry{dualindexsymbol}\marg{e,
  \field{name}=\marg{Euler's number},
  \field{symbol}=\marg{\cs{ensuremath}\marg{e}},
  \field{description}=\marg{base of the natural logarithm}
}
\end{codeenv}
Then the previous example document can be modified to have an index,
a glossary and a list of symbols:
\begin{codeenv}
\cmd{documentclass}\marg{report}
\strut
\cmd{usepackage}[colorlinks]\marg{hyperref}
\cmd{usepackage}[\styopt{record},\styopt{symbols},\styopt{index},\styopt[mcols]{stylemods}]\marg{glossaries-extra}
\strut
\cs{newcommand}\marg{\gls{bibglsnewdualindexsymbolsecondary}}[5]\marg{\comment{}
  \cs{longnewglossaryentry*}\marg{\idx{param}1}\marg{\field{name}=\marg{\idx{param}3},\field{category}=symbol,\comment{}
    \field{symbol}=\marg{\idx{param}4},\idx{param}2,\field{type}=\marg{symbols}}\marg{\idx{param}5}\comment{}
}
\strut
\cs{newcommand}\marg{\cmd{indexprimary}}[1]\marg{\ics{glsadd}\oarg{\glsaddopt[\encap{hyperbf}]{format}}\marg{idx.\idx{param}1}}
\strut
\cs{glsdefpostdesc}\marg{symbol}\marg{\cmd{indexprimary}\marg{\cs{glscurrententrylabel}}}
\cs{glsdefpostdesc}\marg{general}\marg{\cmd{indexprimary}\marg{\cs{glscurrententrylabel}}}
\strut
\gls{GlsXtrLoadResources}\oarg{
 \csopt[entries,terms,symbols]{src},
 \csopt[index]{type},
 \csopt{set-widest},
 \csopt[idx.]{label-prefix},
 \csopt[\empty]{dual-prefix},
 \csopt[primary]{combine-dual-locations},
 \csopt[letter-case]{dual-sort},
 \csopt[main]{dual-type}
}
\strut
\gls{glsxtrnewglslike}\oarg{\glsopt[false]{hyper}}\marg{idx.}\marg{\cmd{idx}}\marg{\cmd{idxpl}}\marg{\cmd{Idx}}\marg{\cmd{Idxpl}}
\strut
\cmd{begin}\marg{document}
\cs{gls}\marg{array}, \cs{gls}\marg{vector}, \cs{gls}\marg{set}, \cs{glspl}\marg{matrix}.
\cmd{idx}\marg{duck}, \cmd{idx}\marg{aardvark}, \cmd{idx}\marg{zebra}.
\cs{gls}\marg{e} and \cs{gls}\marg{pi}.
\strut
\cmd{newpage}
\cs{gls}\marg{array}, \cmd{idx}\marg{vector}, \cmd{idx}\marg{set}, \cs{gls}\marg{matrix}.
\strut
\cmd{newpage}
\cs{gls}\marg{array}, \cs{gls}\marg{vector}, \cs{gls}\marg{set}, \cs{gls}\marg{matrix}.
\strut
\cs{renewcommand}\marg{\cs{glstreenamefmt}}[1]\marg{\cs{textsc}\marg{\idx{param}1}}
\cs{printunsrtglossary}\oarg{\printglossopt[main]{type},\printglossopt{nogroupskip},\printglossopt[\glostyle{alttree}]{style}}
\strut
\cs{renewcommand}\marg{\cs{glstreenamefmt}}[1]\marg{\idx{param}1}
\cs{printunsrtglossary}\oarg{\printglossopt[symbols]{type},\printglossopt{nogroupskip},\printglossopt[index]{style}}
\strut
\cs{renewcommand}\marg{\cs{glstreenamefmt}}[1]\marg{\idx{param}1}
\cs{renewcommand}\marg{\cs{glstreegroupheaderfmt}}[1]\marg{\cs{textbf}\marg{\idx{param}1}}
\cs{printunsrtglossary}\oarg{\printglossopt[index]{type},\printglossopt[\glostyle{mcolindexgroup}]{style}}
\strut
\cmd{end}\marg{document}
\end{codeenv}
Here I've provided some convenient commands for referencing the
\primary\ (index) terms (\csfmt{idx}, \csfmt{idxpl}, \csfmt{Idx}
and \csfmt{Idxpl}). This means I don't need to worry about the
label prefix and it also switches off the hyperlinks (with 
\code{\glsopt[false]{hyper}}). These custom
commands are defined using:
\nosecformatdef{glsxtrnewglslike}
which, in this case, essentially does:
\begin{codeenv}
\cs{newcommand}\marg{\cmd{idx}}[2][]\marg{\cs{gls}\oarg{\glsopt[false]{hyper},\idx{param}1}\marg{idx.\idx{param}2}}
\cs{newcommand}\marg{\cmd{Idx}}[2][]\marg{\cs{Gls}\oarg{\glsopt[false]{hyper},\idx{param}1}\marg{idx.\idx{param}2}}
\cs{newcommand}\marg{\cmd{idxpl}}[2][]\marg{\cs{glspl}\oarg{\glsopt[false]{hyper},\idx{param}1}\marg{idx.\idx{param}2}}
\cs{newcommand}\marg{\cmd{Idxpl}}[2][]\marg{\cs{Glspl}\oarg{\glsopt[false]{hyper},\idx{param}1}\marg{idx.\idx{param}2}}
\end{codeenv}
but the new commands will also recognise the \ics{gls} modifiers, so
\verb|\idx+| will behave like \verb|\gls+| which wouldn't be
possible if \csfmt{idx} was defined using \ics{newcommand} in the
above manner. There's a similar command:
\nosecformatdef{glsxtrnewgls}
if no case-changing versions are required.

I've also redefined \gls!{bibglsnewdualindexsymbolsecondary} to put
the dual entries created with \atentry{dualindexsymbol} into
the \code{symbols} glossary (which is created with the
\styopt{symbols} package option), so it overrides the
\csopt[main]{dual-type} setting.

This command also sets the \field{category}
to \code{symbol}, so I can redefine the post-description hook
for symbols (\ics{glsxtrpostdescsymbol}) to automatically index 
the symbol definition. Similarly for the \code{general}
post-description hook \ics{glsxtrpostdescgeneral}. 

Since the post-description hook isn't done until the glossary has
been created, this requires a slightly longer build process. If the
document file is called \filefmt{myDoc.tex}, then the complete
document build is:
\begin{verbatim}
pdflatex myDoc
bib2gls -g myDoc
pdflatex myDoc
bib2gls -g myDoc
pdflatex myDoc
\end{verbatim}

As from \sty{glossaries-extra-bib2gls} version 1.37, an alternative
method is to identify possible label prefixes with
\ics{glsxtraddlabelprefix} or \ics{glsxtrprependlabelprefix} and use
\ics{dgls}, \ics{dglspl}, \ics{dGls} or \ics{dGlspl}. See the
\styfmt{glossaries-extra} user manual~\cite{glossaries-extra} for
further details.

\entrysection{dualindexnumber}

The \atentry{dualindexnumber} \gls{entrytype} is almost identical to
\atentry{dualindexsymbol}, but the \glspl{primaryentry} are defined using
\gls!{bibglsnewdualindexnumber},
which by default sets the \field{category} to \optfmt{index},
and the \glspl{dualentry} are defined using
\gls!{bibglsnewdualindexnumbersecondary},
which by default sets the \field{category} to \optfmt{number}.

\entrysection{dualabbreviationentry}

The \atentry{dualabbreviationentry} \gls{entrytype} is similar to 
\atentry{dualentry}, but by default the
field mappings are:
\begin{itemize}
\item \field{long} $\mapsto$ \field{name}
\item \field{longplural} $\mapsto$ \field{plural}
\item \field{short} $\mapsto$ \field{text}
\end{itemize}
If the prefix fields are defined, then the default mappings
additionally include:
\begin{itemize}
\item \field{prefix} $\mapsto$ \field{dualprefix}
\item \field{prefixplural} $\mapsto$ \field{dualprefixplural}
\item \field{prefixfirst} $\mapsto$ \field{dualprefixfirst}
\item \field{prefixfirstplural} $\mapsto$ \field{dualprefixfirstplural}
\item \field{dualprefix} $\mapsto$ \field{prefix}
\item \field{dualprefixplural} $\mapsto$ \field{prefixplural}
\item \field{dualprefixfirst} $\mapsto$ \field{prefixfirst}
\item \field{dualprefixfirstplural} $\mapsto$ \field{prefixfirstplural}
\end{itemize}
You may need to add a mapping from \field{shortplural} to
\field{plural} if the default is inappropriate.
(In \bibgls\ version 1.0 this \gls{entrytype} was originally called
\atentry{dualentryabbreviation}. In version 1.1, it was renamed
\atentry{dualabbreviationentry} which makes for a more consistent naming scheme
\code{@dual}\meta{primary}\meta{dual}.)

The required fields are: \field{short}, \field{long} and
\field{description}. This \gls{entrytype} is designed to emulate the
example \gls{newdualentry} command given in the \sty{glossaries}
user manual~\cite{glossaries}. The \gls{primaryentry} is an abbreviation with the given
\field{short} and \field{long} fields (but not the 
\field{description}) and the secondary entry is a regular entry with
the \field{name} copied from the \field{long} field.
The fallback for the \field{sort} is given by
\csopt{abbreviation-sort-fallback}, which defaults to the
\field{short} field (see \sectionref{sec:fallbacks}).

For example:
\begin{codeenv}
\atentry{dualabbreviationentry}\marg{svm,
   \field{long} = \marg{support vector machine},
   \field{short} = \marg{SVM},
   \field{description} = \marg{statistical pattern recognition technique}
}
\end{codeenv}
is rather like doing:
\begin{codeenv}
\atentry{abbreviation}\marg{svm,
   \field{long} = \marg{support vector machine},
   \field{short} = \marg{SVM}
}
\strut
\atentry{entry}\marg{dual.svm,
   \field{name} = \marg{support vector machine},
   \field{description} = \marg{statistical pattern recognition technique}
}
\end{codeenv}
but \code{dual.svm} will automatically be selected if \code{svm}
is indexed in the document. If \code{dual.svm} isn't explicitly
indexed, it won't have a \gls{locationlist}.

If the \field{sort} field is missing \bibgls\ by default falls back
on the \field{name} field. If this is missing, this sort value will
fallback on the \field{short} field. This means that if \field{name}
isn't explicitly given in \atentry{dualabbreviationentry}, then the 
\gls{primaryentry} will be sorted according to \field{short} but the
\dual\ will be sorted according its \field{name} (which has been copied
from the \primary\ \field{long}).

Entries provided using \atentry{dualabbreviationentry} will be defined
with:
\begin{codeenv}
\gls{bibglsnewdualabbreviationentry}
\end{codeenv}
(which uses \gls{newabbreviation}) for the \glspl{primaryentry} and with :
\begin{codeenv}
\gls{bibglsnewdualabbreviationentrysecondary}
\end{codeenv}
(which uses \gls{longnewglossaryentry}) for the secondary entries.
This means that if the \styopt{abbreviations} package option is
used, the \gls{primaryentry} will be put in the
\optfmt{abbreviations} glossary and the secondary entry in the
\mainglossary\.  Use the \csopt{type} and \csopt{dual-type} options
to override this.

\entrysection{dualentryabbreviation}

This \gls{entrytype} is deprecated as from \bibgls\ version 1.1. It's functionally
equivalent to \atentry{dualabbreviationentry} but its name
doesn't fit the general dual entry naming scheme.

\entrysection{dualsymbol}

This is like \atentry{dualentry} but the default mappings are:
\begin{itemize}
\item \field{name} $\mapsto$ \field{symbol}
\item \field{plural} $\mapsto$ \field{symbolplural}
\item \field{symbol} $\mapsto$ \field{name}
\item \field{symbolplural} $\mapsto$ \field{plural}
\end{itemize}
If the prefix fields are defined, then the default mappings
additionally include:
\begin{itemize}
\item \field{prefix} $\mapsto$ \field{dualprefix}
\item \field{prefixplural} $\mapsto$ \field{dualprefixplural}
\item \field{prefixfirst} $\mapsto$ \field{dualprefixfirst}
\item \field{prefixfirstplural} $\mapsto$ \field{dualprefixfirstplural}
\item \field{dualprefix} $\mapsto$ \field{prefix}
\item \field{dualprefixplural} $\mapsto$ \field{prefixplural}
\item \field{dualprefixfirst} $\mapsto$ \field{prefixfirst}
\item \field{dualprefixfirstplural} $\mapsto$ \field{prefixfirstplural}
\end{itemize}
The \field{name} and \field{symbol} fields are required.
For example:
\begin{codeenv}
\atentry{dualsymbol}\marg{pi,
   \field{name}=\marg{pi},
   \field{symbol}=\marg{\cs{ensuremath}\marg{\cs{pi}}},
   \field{description}=\marg{the ratio of the length of the circumference
    of a circle to its diameter}
}
\end{codeenv}

Entries are defined using \gls!{bibglsnewdualsymbol}, which by 
default sets the \field{category} to \optfmt{symbol}.

\entrysection{dualnumber}

This is almost identical to \atentry{dualsymbol} but entries are
defined using \gls!{bibglsnewdualnumber}, which by default sets
the \field{category} to \optfmt{number}.

The above example could be defined as a number since $\pi$ is a
constant:
\begin{codeenv}
\atentry{dualnumber}\marg{pi,
   \field{name}=\marg{pi},
   \field{symbol}=\marg{\cs{ensuremath}\marg{\cs{pi}}},
   \field{description}=\marg{the ratio of the length of the circumference
    of a circle to its diameter},
   \field{user1}=\marg{3.14159}
}
\end{codeenv}

This has stored the approximate value in the \field{user1} field.
The post-description hook could then be adapted to show this.
\begin{codeenv}
\cs{glsdefpostdesc}\marg{number}\marg{\comment{}
  \cs{ifglshasfield}\marg{useri}\marg{\cs{glscurrententrylabel}}
  \marg{ (approximate value: \cs{glscurrentfieldvalue})}\comment{}
  \marg{}\comment{}
}
\end{codeenv}

This use of the \field{user1} field means that the dual entries
could be sorted numerically according to the approximate value:
\begin{codeenv}
\cmd{usepackage}[\styopt{record},\styopt{postdot},\styopt{numbers},\styopt[index]{style}]\marg{glossaries-extra}
\strut
\gls{GlsXtrLoadResources}\oarg{
  \csopt[entries]{src},\comment{entries.bib}
  \csopt[numbers]{dual-type},
  \csopt[double]{dual-sort},\comment{decimal sort}
  \csopt[user1]{dual-sort-field}
}
\end{codeenv}

\entrysection{dualabbreviation}

The \atentry{dualabbreviation} \gls{entrytype} is similar to 
\atentry{dualentry}, but by default the
field mappings are:
\begin{itemize}
\item \field{short} $\mapsto$ \field{dualshort}
\item \field{shortplural} $\mapsto$ \field{dualshortplural}
\item \field{long} $\mapsto$ \field{duallong}
\item \field{longplural} $\mapsto$ \field{duallongplural}
\item \field{dualshort} $\mapsto$ \field{short}
\item \field{dualshortplural} $\mapsto$ \field{shortplural}
\item \field{duallong} $\mapsto$ \field{long}
\item \field{duallongplural} $\mapsto$ \field{longplural}
\end{itemize}
If the prefix fields are defined, then the default mappings
additionally include:
\begin{itemize}
\item \field{prefix} $\mapsto$ \field{dualprefix}
\item \field{prefixplural} $\mapsto$ \field{dualprefixplural}
\item \field{prefixfirst} $\mapsto$ \field{dualprefixfirst}
\item \field{prefixfirstplural} $\mapsto$ \field{dualprefixfirstplural}
\item \field{dualprefix} $\mapsto$ \field{prefix}
\item \field{dualprefixplural} $\mapsto$ \field{prefixplural}
\item \field{dualprefixfirst} $\mapsto$ \field{prefixfirst}
\item \field{dualprefixfirstplural} $\mapsto$ \field{prefixfirstplural}
\end{itemize}
The required fields are: \field{short}, \field{long}, 
\field{dualshort} and \field{duallong}.
This includes some new fields: \field{dualshort},
\field{dualshortplural}, \field{duallong} and
\field{duallongplural}. If these aren't already defined, they
will be provided in the \iext{glstex} file with
\begin{codeenv}
\ics{glsxtrprovidestoragekey}\margm{key}\marg{}\marg{}
\end{codeenv}
Note that this use with an empty third argument prevents
the creation of a field access command (analogous to
\ics{glsentrytext}). The value can be accessed with
\ics{glsxtrusefield} instead. Remember that the field won't be
available until the \ext{glstex} file has been created.

Note that \bibgls\ doesn't know what abbreviation styles
are in used, so if the \field{sort} field is missing
it will fallback on the \field{short} field. If the abbreviations
need to be sorted according to the \field{long} field instead,
use \csopt[long]{abbreviation-sort-fallback} (see \sectionref{sec:fallbacks}).

Terms that are defined using \atentry{dualabbreviation} will be
written to the output file using \gls!{bibglsnewdualabbreviation}.

If the \csopt{dual-abbrv-backlink} option is on, the default field
used for the backlinks is the \field{dualshort} field, so you'll need
to make sure you adapt the glossary style to show that field. The
simplest way to do this is through the category \idx{postdescriptionhook}.

For example, if the entries all have the \field{category} set
to \optfmt{abbreviation}, then this requires redefining
\ics{glsxtrpostdescabbreviation} (either with \cs{renewcommand}
or via \cs{glsdefpostdesc}).

Here's an example dual abbreviation for a document where English is
the primary language and German is the secondary language:
\begin{codeenv}
\atentry{dualabbreviation}\marg{rna,
  \field{short}=\marg{RNA},
  \field{dualshort}=\marg{RNS},
  \field{long}=\marg{ribonucleic acid},
  \field{duallong}=\marg{Ribonukleins\"aure}
}
\end{codeenv}
If the abbreviation is in the file called
\filefmt{entries-dual-abbrv.bib}, then here's an example document:
\begin{codeenv}
\cmd{documentclass}\marg{article}
\strut
\cmd{usepackage}[T1]\marg{fontenc}
\cmd{usepackage}[utf8]\marg{inputenc}
\strut
\cmd{usepackage}[ngerman,main=english]\marg{babel}
\cmd{usepackage}[colorlinks]\marg{hyperref}
\cmd{usepackage}[\styopt{record},\styopt{nomain}]\marg{glossaries-extra}
\strut
\cs{newglossary*}\marg{english}\marg{English}
\cs{newglossary*}\marg{german}\marg{German}
\strut
\cs{setabbreviationstyle}\marg{long-short}
\strut
\cs{glsdefpostdesc}\marg{abbreviation}\marg{\comment{}
  \cs{ifglshasfield}\marg{dualshort}\marg{\cs{glscurrententrylabel}}
  \marg{\comment{}
    \cmd{space}(\cs{glscurrentfieldvalue})\comment{}
  }\comment{}
  \marg{}\comment{}
}
\strut
\gls{GlsXtrLoadResources}\oarg{
 \csopt[entries-dual-abbrv]{src},\comment{entries-dual-abbrv.bib}
 \csopt[english]{type},\comment{put \primary\ entries in glossary 'english'}
 \csopt[german]{dual-type},\comment{put \dual\ entries in glossary 'german'}
 \csopt[en.]{label-prefix},\comment{\primary\ label prefix}
 \csopt[de.]{dual-prefix},\comment{\dual\ label prefix}
 \csopt[en]{sort},\comment{sort \glspl{primaryentry} according to language 'en'}
 \csopt[de-1996]{dual-sort},\comment{sort \glspl{dualentry} according to 'de-1996'}
                     \comment{(German new orthography)}
 \csopt{dual-abbrv-backlink}\comment{add links in the glossary to the opposite entry}
}
\strut
\cmd{begin}\marg{document}
\strut
English: \cs{gls}\marg{en.rna}; \cs{gls}\marg{en.rna}.
\strut
German: \cs{gls}\marg{de.rna}; \cs{gls}\marg{de.rna}.
\strut
\cs{printunsrtglossaries}
\cmd{end}\marg{document}
\end{codeenv}
If the \csopt{label-prefix} is omitted, then only the dual entries
will have a prefix:
\begin{codeenv}
English: \cs{gls}\marg{rna}; \cs{gls}\marg{rna}.
\strut
German: \cs{gls}\marg{de.rna}; \cs{gls}\marg{de.rna}.
\end{codeenv}
Another variation is to use the \abbrstyle{long-short-user}
abbreviation style and modify the associated \ics{glsxtruserfield} so that
the \field{duallong} field is selected for the parenthetical
material:
\begin{codeenv}
\cs{renewcommand}*\marg{\cs{glsxtruserfield}}\marg{duallong}
\end{codeenv}
This means that the first use of the \gls{primaryentry} is displayed as
\begin{quote}
ribonucleic acid (RNA, Ribonukleins\"aure)
\end{quote}
and the first use of the dual entry is displayed as:
\begin{quote}
Ribonukleins\"aure (RNS, ribonucleic acid)
\end{quote}

Here's an example to be used with the \abbrstyle{long-short-desc}
style:
\begin{codeenv}
\atentry{dualabbreviation}\marg{rna,
  \field{short}=\marg{RNA},
  \field{dualshort}=\marg{RNS},
  \field{long}=\marg{ribonucleic acid},
  \field{duallong}=\marg{Ribonukleins\"aure}
  \field{description}=\marg{a polymeric molecule},
  \field{user1}=\marg{Ein polymeres Molek\"ul}
}
\end{codeenv}
This stores the dual description in the \field{user1} field,
so this needs a mapping.
The new example document is much the same as the previous one, except
that the \csopt{dual-abbrv-map} option is needed to include the 
mapping between the \field{description} and \field{user1} fields:
\begin{codeenv}
\cmd{documentclass}\marg{article}
\strut
\cmd{usepackage}[T1]\marg{fontenc}
\cmd{usepackage}[utf8]\marg{inputenc}
\strut
\cmd{usepackage}[ngerman,main=english]\marg{babel}
\cmd{usepackage}[colorlinks]\marg{hyperref}
\cmd{usepackage}[\styopt{record},\styopt{nomain}]\marg{glossaries-extra}
\strut
\cs{newglossary*}\marg{english}\marg{English}
\cs{newglossary*}\marg{german}\marg{German}
\strut
\cs{setabbreviationstyle}\marg{long-short-desc}
\strut
\cs{glsdefpostdesc}\marg{abbreviation}\marg{\comment{}
  \cs{ifglshasfield}\marg{dualshort}\marg{\cs{glscurrententrylabel}}
  \marg{\comment{}
    \cmd{space}(\cs{glscurrentfieldvalue})\comment{}
  }\comment{}
  \marg{}\comment{}
}
\strut
\gls{GlsXtrLoadResources}\oarg{
 \csopt[entries-dual-abbrv-desc]{src},\comment{entries-dual-abbrv-desc.bib}
 \csopt[english]{type},\comment{put \glspl{primaryentry} in glossary 'english'}
 \csopt[german]{dual-type},\comment{put \glspl{dualentry} in glossary 'german'}
 \csopt[en.]{label-prefix},\comment{\primary\ label prefix}
 \csopt[de.]{dual-prefix},\comment{\dual\ label prefix}
 \csopt[en]{sort},\comment{sort \glspl{primaryentry} according to language 'en'}
 \csopt[long]{abbreviation-sort-fallback},\comment{fallback on 'long' field}
 \csopt[de-1996]{dual-sort},\comment{sort \glspl{dualentry} according to 'de-1996'}
                   \comment{(German new orthography)}
 \csopt{dual-abbrv-backlink},\comment{add links in the glossary to the opposite entry}
\comment{ dual key mappings:}
 \csopt[\comment{}
   \marg{short,shortplural,long,longplural,dualshort,dualshortplural,
     duallong,duallongplural,description,user1},
   \marg{dualshort,dualshortplural,duallong,duallongplural,short,shortplural,
     long,longplural,user1,description}
 ]{dual-abbrv-map}
}
\strut
\cmd{begin}\marg{document}
\strut
English: \cs{gls}\marg{en.rna}; \cs{gls}\marg{en.rna}.
\strut
German: \cs{gls}\marg{de.rna}; \cs{gls}\marg{de.rna}.
\strut
\cs{printunsrtglossaries}
\cmd{end}\marg{document}
\end{codeenv}
Note that since this document uses the \abbrstyle{long-short-desc}
abbreviation style, the \csopt{abbreviation-sort-fallback} needs to be changed
to \optfmt{long}.

If I change the order of the mapping to:
\begin{codeenv}
 \csopt[\comment{}
   \marg{long,longplural,short,shortplural,dualshort,dualshortplural,
     duallong,duallongplural,description,user1},
   \marg{duallong,duallongplural,dualshort,dualshortplural,short,shortplural,
     long,longplural,user1,description}
 ]{dual-abbrv-map}
\end{codeenv}
Then the back-link field will switch to \field{duallong}. The
\idx{postdescriptionhook} can be modified to allow for this:
\begin{codeenv}
\cs{glsdefpostdesc}\marg{abbreviation}\marg{\comment{}
  \cs{ifglshasfield}\marg{duallong}\marg{\cs{glscurrententrylabel}}
  \marg{\comment{}
    \cmd{space}(\cs{glscurrentfieldvalue})\comment{}
  }\comment{}
  \marg{}\comment{}
}
\end{codeenv}

An alternative is to use the \abbrstyle{long-short-user-desc} style
without the \idx!{postdescriptionhook}:
\begin{codeenv}
\cs{setabbreviationstyle}\marg{long-short-user-desc}
\cs{renewcommand}*\marg{\cs{glsxtruserfield}}\marg{duallong}
\end{codeenv}
However be careful with this approach as it can cause nested
hyperlinks. In this case it's better to use the
\abbrstyle{long-postshort-user-desc} style which defers the
parenthetical material until after the link-text:
\begin{codeenv}
\cs{setabbreviationstyle}\marg{long-postshort-user-desc}
\cs{renewcommand}*\marg{\cs{glsxtruserfield}}\marg{duallong}
\end{codeenv}
If the back-link field has been switched to \field{duallong} then
the \idx!{postdescriptionhook} is no longer required.

\entrysection{dualacronym}

As \atentry{dualabbreviation} but defines the entries with
\gls!{bibglsnewdualacronym}.

\section{Tertiary Entry Types}
\label{sec:tertiaryentry}

A \igls{tertiaryentry} type is essentially a \gls{dualentry} that creates three
separate (but related) \styfmt{glossaries-extra} entry definitions per 
\ext{bib} entry. As with \glspl{dualentry}, the first of these
is the \gls{primaryentry}. The second and third are referred to as the
\igls{secondaryentry} and \igls{tertiaryentry}.

The \gls{tertiaryentry} is effectively an appendage of the
\gls{secondaryentry}{secondary}, and is defined by the same associated
\csfmt{bibglsnew\ldots secondary} command that defines the 
\gls{secondaryentry}. Therefore the \glsdisp{secondaryentry}{secondary} 
and \glsdisp{tertiaryentry}{tertiary} are both considered the
\dual\ and are treated as a single entry for the purposes of sorting
and collating.

The \gls{tertiaryentry} will never have any \glspl{location}. Any \glspl{record}
found will be assigned to the \glsdisp{secondaryentry}{secondary}
(and may then be moved to the \primary\ with
\csopt[primary]{combine-dual-locations}). The
\glsdisp{tertiaryentry}{tertiary} will always have the same order as
the \glsdisp{secondaryentry}{secondary} and will have the same
\field{group} value. You can set the \field{type} for the
\glsdisp{tertiaryentry}{tertiary} with \csopt{tertiary-type} and the
\field{category} with \csopt{tertiary-category}. The label prefix
defaults to \idprefix{tertiary} and can be changed with
\csopt{tertiary-prefix}.

\entrysection{tertiaryindexabbreviationentry}

This \gls{entrytype} is very similar to
\atentry{dualindexabbreviation} but creates a \gls{tertiaryentry} as
well.  The required fields are: \field{short} and \field{long} (as for
\atentry{dualindexabbreviation}) and also \field{description}.  The
mappings are shared by both \glspl{entrytype}.  For example:
\begin{codeenv}
\atentry{tertiaryindexabbreviationentry}\marg{html,
  \field{short} = \marg{HTML},
  \field{long} = \marg{hypertext markup language},
  \field{description} = \marg{a markup language for creating web pages}
}
\end{codeenv}
is analogous to:
\begin{codeenv}
\gls{newglossaryentry}\marg{html,\field{name}=\marg{HTML},\field{description}=\marg{}}
\strut
\gls{newabbreviation}\marg{dual.html}\marg{HTML}\marg{hypertext markup language}
\strut
\gls{newglossaryentry}\marg{tertiary.html, 
  \field{name}=\marg{hypertext markup language},
  \field{description}=\marg{a markup language for creating web pages}
}
\end{codeenv}
The last two are actually defined using one command:
\begin{codeenv}
\gls{bibglsnewtertiaryindexabbreviationentrysecondary}
  \marg{dual.html}\comment{\glsdisp{secondaryentry}{secondary} label}
  \marg{tertiary.html}\comment{\glsdisp{tertiaryentry}{tertiary} label}
  \marg{\ldots}\comment{\glsdisp{secondaryentry}{secondary} fields}
  \marg{\ldots}\comment{\glsdisp{tertiaryentry}{tertiary} fields}
  \marg{HTML}\comment{\primary\ name}
  \marg{HTML}\comment{short}
  \marg{hypertext markup language}\comment{long}
  \marg{a markup language for creating web pages}\comment{description}
\end{codeenv}
The \gls!{bibglsnewtertiaryindexabbreviationentrysecondary}
command is provided in the \iext{glstex} file as:
\begin{codeenv}
\cs{providecommand}\marg{\gls{bibglsnewtertiaryindexabbreviationentrysecondary}}[8]\marg{\comment{}
  \gls{newabbreviation}\oarg{\idx{param}3}\marg{\idx{param}1}\marg{\idx{param}6}\marg{\idx{param}7}\comment{}
  \cs{longnewglossaryentry*}\marg{\idx{param}2}\comment{}
  \marg{\field{name}=\marg{\cmd{protect}\gls{bibglsuselongfont}\marg{\idx{param}7}\marg{\ics{glscategory}\marg{\idx{param}1}}},\idx{param}4}\comment{}
  \marg{\idx{param}8}\comment{}
}
\end{codeenv}
which defines the \glsdisp{secondaryentry}{secondary} as an
abbreviation using \csfmt{newabbreviation} and the
\glsdisp{tertiaryentry}{tertiary} as a regular entry using
\csfmt{longnewglossaryentry}. This means that the
\gls{tertiaryentry} is always defined immediately after the
corresponding \gls{secondaryentry}.  The \primary\ may be defined
earlier or later in the file depending on the way the entries are
sorted and on the \csopt{dual-sort} setting.

\section{Multi-Entry Types}
\label{sec:multientry}

A \gls{multientrytype} is an entry that may spawn multiple 
\glspl{primaryentry}. This means that both the \gls{mainentry} and the
\glspl{spawnedentry} are sorted together along with all the other 
\glspl{primaryentry}. In the case of \atentry{spawndualindexentry}, 
the \glsdisp{mainentry}{main} and \glspl{spawnedentry} are \primary. 
The \gls{mainentry}['s] \dual\ is created as per \atentry{dualindexentry}.

\entrysection{bibtexentry}

The \atentry{bibtexentry} type will typically need to be aliased
as it's designed for converting \BibTeX\ entries into \bibgls\
entries. For example, to make \bibgls\ treat \atentryfmt{article}
and \atentryfmt{book} as though they were both
\atentry{bibtexentry}:
\begin{codeenv}
\csopt[
  article=bibtexentry,
  book=bibtexentry
]{entry-type-aliases}
\end{codeenv}
For convenience, \isty{glossaries-extra-bib2gls} v1.29+
provides \ics{GlsXtrBibTeXEntryAliases} which covers all the
standard \BibTeX\ \glspl{entrytype}. Alternatively, you can use
\csopt[bibtexentry]{unknown-entry-alias} to alias all entries
that aren't recognised by \bibgls. If you use 
\csopt[same as original entry]{category}, the \field{category}
field will be set to the original \gls{entrytype} (for example,
\code{article} or \code{book}). Similarly you can use
\csopt[same as original entry]{type} to set the \field{type}
field (but remember that the glossary types will need to be defined
in the document).

There are no required fields. The fallback for the \field{sort}
field is given by \csopt{bibtexentry-sort-fallback} (see
\sectionref{sec:fallbacks}). If you want to access any of the
\BibTeX\ fields, you will need to alias or define them.  For
example:
\begin{codeenv}
\csopt[
  \fieldfmt{title}=\field{name}
]{field-aliases}
\end{codeenv}
Since \BibTeX's \fieldfmt{type} field conflicts with \bibgls's
\field{type} field, when \bibgls\ parses \atentry{bibtexentry}
if will convert \fieldfmt{type} to \field{bibtextype}, so you
must use \field{bibtextype} as the identifier when aliasing.

Alternatively, you can use \ics{GlsXtrProvideBibTeXFields} which
uses \ics{glsaddstoragekey} to provide all the standard \BibTeX\
fields. (Remember that new fields must be defined before the first 
\igls{resourceset}.)

The \atentry{bibtexentry} essentially creates an \atentry{index}
form of entry, but it additionally defines a \atentry{contributor}
entry for each listed author or editor
and updates the dependency lists: each \atentry{contributor} is
added to the \glsdisp{mainentry}{main}
\atentry{bibtexentry}'s list of dependencies (so if the
\atentry{bibtexentry} has a \gls{record} then all its satellite
\atentry{contributor}s are selected with the default
\csopt[recorded and deps]{selection}), and 
each \atentry{contributor} is treated as having a cross-reference to
the \glsdisp{mainentry}{main} \atentry{bibtexentry} (so if a \atentry{contributor}
has a \gls{record} then all the linked \atentry{bibtexentry} terms will
be selected if \csopt[recorded and deps and see]{selection}).
You can instruct \bibgls\ to treat \ics{citation} as an 
\igls{ignoredrecord} using \longarg{cite-as-record}.

Each contributor is effectively defined as:
\begin{codeenv}
\atentry{contributor}\marg{\meta{label},
  \field{name}=\marg{\gls{bibglscontributor}\margm{forenames}\margm{von}\margm{surname}\margm{suffix}}
}
\end{codeenv}
The label is obtained by converting the \field{name}
to a label, using the same function as \csopt{labelify} (which
means it's governed by \csopt{labelify-replace}).

The \fieldfmt{author} and \fieldfmt{editor} fields are always
checked, even if those fields aren't recognised by \bibgls, (which
they aren't by default). These checks are performed before field
aliases are applied. If neither field is present, no additional
entries are spawned. If the dependent \atentry{contributor} entry 
has already been defined, it won't be redefined, but will have the
new \atentry{bibtexentry} added to its internal \field{bibtexentry}
field.

The \glsdisp{mainentry}{main} \atentry{bibtexentry} is defined using
\gls{bibglsnewbibtexentry} and is followed by:
\begin{codeenv}
\gls{glsxtrfieldlistadd}\margm{id}\marg{bibtexcontributor}\margm{contributor-id}
\end{codeenv}
where \meta{id} is the label identifying the \glsdisp{mainentry}{main}
\atentry{bibtexentry} and \meta{contributor-id} is the
label identifying the contributor, for each contributor that has
been selected.

Each contributor is defined using \gls{bibglsnewcontributor}. The definition 
is followed by:
\begin{codeenv}
\gls{glsxtrfieldlistadd}\margm{contributor-id}\marg{bibtexentry}\margm{id}
\gls{glsxtrfieldlistadd}\margm{contributor-id}\marg{bibtexentry@\meta{entry-type}}\margm{id}
\end{codeenv}
for each selected \atentry{bibtexentry} associated with that contributor. The
second line provides the internal list field
\field{bibtexentry@entrytype}, where \meta{entry-type} is the
original \gls{entrytype} (before it was aliased to
\atentry{bibtexentry} and converted to \idx!{lowercase}).
For example \code{article} or \code{book}.

You can iterate over these internal list fields using
\glsadd{idx.loop}\ics{glsxtrfielddolistloop} or \ics{glsxtrfieldforlistloop}.
For example:
\begin{codeenv}
\cs{newcommand}\marg{\csfmt{contributorhandler}}[1]\marg{\csfmt{par}\cs{glsentryname}\marg{\#1}}

\cs{newcommand}\marg{\postdeschook{contributor}}\marg{\comment{}
  \cs{glsxtrifhasfield}\marg{bibtexentry}\marg{\cs{glscurrententrylabel}}\comment{}
  \marg{\comment{}
    \cs{glsxtrfieldforlistloop}
    \marg{\cs{glscurrententrylabel}}\marg{bibtexentry}\comment{}
    \marg{\csfmt{contributorhandler}}\comment{}
  }\comment{}
  \marg{\csfmt{par} No titles.}\comment{}
}
\end{codeenv}
(where the resource option \csopt[\fieldfmt{title}=\field{name}]{field-aliases}
has been used).

Here's an example that uses the test \file{xampl.bib} file that's
provided with \TeX\ distributions:
\begin{codeenv}
\cmd{documentclass}\marg{article}
\strut
\cmd{usepackage}[\styopt{record},\styopt{nomain}]\marg{glossaries-extra}
\strut
\cs{newglossary*}\marg{contributors}\marg{Authors/Editors}
\cs{newglossary*}\marg{titles}\marg{Titles}
\strut
\cs{newcommand}\marg{\gls{bibglsnewbibtexentry}}[4]\marg{\comment{}
 \cs{longnewglossaryentry*}\marg{\idx{param}1}\marg{\field{name}={\idx{param}3},\idx{param}2,\field{type}=\marg{titles}}\marg{\idx{param}4}\comment{}
}
\strut
\gls{GlsXtrLoadResources}\oarg{
  \csopt[xampl]{src},
  \csopt[false]{write-preamble},
  \csopt[
   \ics{GlsXtrBibTeXEntryAliases}
  ]{entry-type-aliases},
  \csopt[
    \fieldfmt{title}=\field{name}
  ]{field-aliases},
  \csopt[
    \fieldfmt{note}=\field{name}
  ]{replicate-fields},
  \csopt[
    \marg{[ \cs{cs.string}\csfmt{-}\cs{cs.string}\csfmt{.}]}\marg{}
  ]{labelify-replace},
  \csopt[contributors]{type},
  \csopt[same as original entry]{category},
  \csopt[\field{category}]{sort-field},
  \csopt[\field{name}]{sort-suffix}
}
\strut
\gls{glsxtrsetgrouptitle}\marg{article}\marg{Articles}
\gls{glsxtrsetgrouptitle}\marg{booklet}\marg{Booklets}
\gls{glsxtrsetgrouptitle}\marg{book}\marg{Books}
\gls{glsxtrsetgrouptitle}\marg{inbook}\marg{Book Chapters}
\gls{glsxtrsetgrouptitle}\marg{misc}\marg{Miscellaneous}
\strut
\cs{newcommand}\marg{\csfmt{contributorhandler}}[1]\marg{\csfmt{par}\ics{glsentryname}\marg{\idx{param}1} (\idx{param}1)}
\strut
\cs{newcommand}\marg{\postdeschook{contributor}}{\comment{}
  \cs{glsxtrifhasfield}\marg{bibtexentry}\marg{\cs{glscurrententrylabel}}\comment{}
  \marg{\comment{}
    \ics{glsxtrfieldforlistloop}
    \marg{\cs{glscurrententrylabel}}\marg{bibtexentry}\comment{}
    \marg{\csfmt{contributorhandler}}\comment{}
  }\comment{}
  \marg{\csfmt{par} No titles.}\comment{}
}
\strut
\csfmt{begin}\marg{document}
Sample\idx{nbspchar}\ics{cite}\marg{book-minimal,article-full,inbook-full,misc-minimal}.
Another sample\idx{nbspchar}\cs{cite}\marg{booklet-minimal,misc-full,article-minimal}.
\strut
\csfmt{bibliographystyle}\marg{plain}
\ics{bibliography}\marg{xampl}
\strut
\cs{printunsrtglossary}[\printglossopt[contributors]{type},\printglossopt[altlist]{style}]
\cs{printunsrtglossary*}[\printglossopt[titles]{type},\printglossopt[indexgroup]{style}]
\marg{\comment{}
 \cs{renewcommand}\marg{\ics{glsxtrgroupfield}}\marg{category}\comment{}
 \cs{renewcommand}\marg{\cs{glstreenamefmt}}[1]\marg{\cs{emph}\marg{\idx{param}1}}\comment{}
 \cs{renewcommand}\marg{\cs{glstreegroupheaderfmt}}[1]\marg{\cs{textbf}\marg{\idx{param}1}}\comment{}
}
\strut
\csfmt{end}\marg{document}
\end{codeenv}
If the file is called \filefmt{myDoc.tex} then the document build
is:
\begin{codeenv}
pdflatex myDoc
bib2gls \longarg{cite-as-record} myDoc
bibtex myDoc
pdflatex myDoc
pdflatex myDoc
\end{codeenv}

\entrysection{progenitor}

The \atentryfmt{progenitor} type of entries are the only place where
the \field{adoptparents} field is permitted. The value should be a
comma-separated list of labels. The \field{adoptparents} field
must be set and must contain a least one label. If the value
contains any of the characters \idx{backslashchar} (backslash), 
\idx{bgroupchar} (open brace) or \idx{egroupchar} (close brace) then
the field will be interpreted (if the default \longarg{interpret} settings 
is on).

Since entries are spawned before fields are processed, the
\field{adoptparents} field is parsed before any field aliases
(\csopt{field-aliases}) or replication (\csopt{replicate-fields})
takes place. However, if the \field{adoptparents} field isn't found,
\bibgls\ will check for a simple mapping in both the
\csopt{field-aliases} and \csopt{replicate-fields} settings.

This \gls{entrytype} creates a \glsdisp{mainentry}{main} \igls[textformat=emph]{progenitor}
term (with all the given fields except \field{adoptparents})
and $n$ spawned \igls[textformat=emph]{progeny} terms, where
$n$ is the number of elements in the \field{adoptparents} field,
that are dependent on the \glsdisp{mainentry}{main term}.

Each of the spawned \gls{progeny} entries have the field identified by
\csopt{adopted-parent-field} (\field{parent} by default) set to the
corresponding element in the \field{adoptparents} field.

All fields from the original definition are copied except for the
\field{adoptparents}, \field{alias} and \field{parent} fields. The
\field{parent} field is never copied, regardless of the value of
\csopt{adopted-parent-field}.  If the adopted parent field is
changed to one that's contained in the original entry, it's value
will be from \field{adoptparents} not the value from the original
entry.

The copied fields follow the same conditions as normal
entries. (For example, unknown fields are ignored, case-changes are
applied, if appropriate, and the \field{type} field must reference a
valid glossary, if set.) If \csopt{progenitor-type} is set, then
this assignment is made after the \igls{progeny} are created
and only applies to the \glsdisp{mainentry}{main} \igls{progenitor} entry. The
type for the \gls{progeny} can be set with \csopt{progeny-type}.
For example, \csopt[same as parent]{progeny-type} will ensure
that the \igls{progeny} are in the same glossary type as
their \gls{parententry}.

For example, an entry defined as:
\begin{codeenv}
\atentry{progenitor}\marg{\meta{id},
  \field{adoptparents} = \marg{\meta{parent-1 id},\ldots,\meta{parent-N id}},
  \meta{field-name-1} = \margm{text},
  \ldots
  \meta{field-name-n} = \margm{text}
}
\end{codeenv}
is essentially like:
\begin{codeenv}
\atentry{index}\marg{\meta{id},
  \field{progeny} = \marg{\meta{parent-1 id}.\meta{id},\ldots,\meta{parent-N id}.\meta{id}},
  \meta{field-name-1} = \margm{text},
  \ldots
  \meta{field-name-n} = \margm{text}
}
\strut
\atentry{index}\marg{\meta{parent-1 id}.\meta{id},
  \field{progenitor} = \margm{id},
  \field{parent} = \margm{parent-1 id},
  \meta{field-name-1} = \margm{text},
  \ldots
  \meta{field-name-n} = \margm{text}
}
\strut
\ldots
\strut
\atentry{index}\marg{\meta{parent-N id}.\meta{id},
  \field{progenitor} = \margm{id},
  \field{parent} = \margm{parent-N id},
  \meta{field-name-1} = \margm{text},
  \ldots
  \meta{field-name-n} = \margm{text}
}
\end{codeenv}
This creates the \glsdisp{mainentry}{main} (\gls{progenitor}) \meta{id} entry, which
contains all the fields (except for \field{adoptparents}) that were
in the original \atentry{progenitor} definition and has the new
field \field{progeny} set to the comma-separated list of 
\gls{spawnedentry} labels. The \glspl{mainentry} are defined in the \ext{glstex} file
with \gls{bibglsnewprogenitor}.

In addition to the \glsdisp{mainentry}{main} \meta{id} entry, the
above also creates the spawned \gls{progeny} entries
\code{\meta{parent-1 id}.\meta{id}}, \ldots, \code{\meta{parent-N
id}.\meta{id}} that are dependent on the \glsdisp{mainentry}{main} \meta{id} entry.

The \glspl{spawnedentry} have the \field{parent} field set to the
corresponding label obtained from the \field{adoptparents} list.
This \gls{parententry} must also be defined, as usual for the
\field{parent} field.  (This restriction obviously doesn't apply if
\csopt{adopted-parent-field} is changed from the default
\field{parent}.) The spawned entries are defined in the \ext{glstex}
file with \gls{bibglsnewspawnedindex}

If the \glsdisp{mainentry}{main} \gls{progenitor} entry is referenced in the document
then (assuming the default selection criteria) the
\glspl{spawnedentry} will also be automatically selected. You can
check for the existence of the \field{progenitor} field using
\cs{glsxtrifhasfield} and fetch the \field{location} field from the
\gls{mainentry}, if required.

Although the \glspl{spawnedentry} are considered dependents of the 
\gls{mainentry}, the reverse doesn't apply. If a \gls{spawnedentry} is referenced
in the document (with \code{\meta{parent-id}.\meta{id}}) then the
\gls{mainentry} and its other spawned entries aren't automatically
selected.

For example, suppose the file \filefmt{entries.bib} contains:
\begin{codeenv}
\atentry{indexplural}\marg{stylesheet, \field{text}=\marg{stylesheet language}}
\strut
\atentry{index}\marg{webdesign, \field{name}=\marg{web design}}
\strut
\atentry{indexplural}\marg{markup, \field{text}=\marg{markup language}}
\strut
\atentry{progenitor}\marg{xml,
  \field{name}=\marg{XML},
  \field{adoptparents}=\marg{markup}
}
\strut
\atentry{progenitor}\marg{css,
  \field{name}=\marg{CSS},
  \field{adoptparents}=\marg{stylesheet,webdesign}
}
\strut
\atentry{progenitor}\marg{html,
  \field{name}=\marg{HTML},
  \field{adoptparents}=\marg{markup,webdesign}
}
\strut
\atentry{progenitor}\marg{xsl,
  \field{name}=\marg{XSL},
  \field{adoptparents}=\marg{stylesheet}
}
\end{codeenv}
and if the document contains:
\begin{codeenv}
\cmd{documentclass}\marg{article}
\strut
\cmd{usepackage}\oarg{\styopt{record},\styopt[tree]{stylemods},\styopt[index]{style}}\marg{glossaries-extra}
\strut
\gls{GlsXtrLoadResources}\oarg{\csopt[entries]{src},\csopt[all]{selection}}
\strut
\cs{newcommand}*\marg{\cs{glstreenamefmt}}[1]\marg{\idx{param}1}
\cmd{begin}\marg{document}
\cs{printunsrtglossaries}
\cmd{end}\marg{document}
\end{codeenv}
Then the resulting list will be:
\begin{quote}
\setlength{\parindent}{0pt}%
\setlength{\parskip}{0pt plus 0.3pt}%
\glstreeitem CSS\par
\glstreeitem HTML\par
\glstreeitem markup language\par
\glstreesubitem HTML\par
\glstreesubitem XML\par
\glstreeitem stylesheet language\par
\glstreesubitem CSS\par
\glstreesubitem XSL\par
\glstreeitem web design\par
\glstreesubitem CSS\par
\glstreesubitem HTML\par
\glstreeitem XML\par
\glstreeitem XSL\par
\end{quote}
This allows the HTML and CSS entries to be listed under multiple
parents.

The following \atentryfmt{spawn\meta{single-type}} commands are all forms
of \atentry{progenitor} that create the given
\atentryfmt{\meta{single-type}} of entry. The \glspl{spawnedentry} are actually
created with the private \gls{entrytype} \atentryfmt{spawned\meta{type}}. In the
case of \atentry{progenitor}, the \glspl{spawnedentry} are defined as 
a \atentryfmt{spawnedindex} entry. These special
\atentryfmt{spawned\meta{type}} \glspl{entrytype} aren't intended for use
in the \ext{bib} file, but if you reference the \gls{entrytype} (for
example, with \csopt[same as entry]{category}) you will get 
\atentryfmt{spawned\meta{type}} as the \gls{entrytype}. The
original \gls{entrytype} for the \glspl{spawnedentry} is the same as
the original entry for the \glsdisp{mainentry}{main} \atentry{progenitor} entry.

There is currently only one form of \dual\ \atentry{progenitor} entry and that's
\atentry{spawndualindexentry}. Only the \glsdisp{mainentry}{main} \gls{progenitor} entry
is a \gls{dualentry}. The spawned \gls{progeny} are all \atentry{index} 
\glspl{primaryentry}.

\entrysection{spawnindex}

As \atentry{progenitor}, but the \glspl{mainentry} are defined in the
\ext{glstex} file with \gls{bibglsnewspawnindex} and the
spawned entries are defined with \gls{bibglsnewspawnedindex}.

\entrysection{spawnindexplural}

As \atentry{progenitor}, except that it creates
\atentry{indexplural} terms instead of \atentry{index}.
As with \atentry{indexplural}, if the \field{name} field isn't set,
it's assigned to the same value as the \field{plural} field (or the
\hyperref[sec:fallbacks]{fallback} for the \field{plural}, if not defined).

The \glspl{mainentry} are defined in the
\ext{glstex} file with \gls{bibglsnewspawnindexplural} and the
\glspl{spawnedentry} are defined with \gls{bibglsnewspawnedindexplural}.

\entrysection{spawnentry}

As \atentry{progenitor}, except that it creates
\atentry{entry} terms instead of \atentry{index}.
As with \atentry{entry}, the \field{description} field is required
and either \field{name} or \field{parent}.

The \glspl{mainentry} are defined in the
\ext{glstex} file with \gls{bibglsnewspawnentry} and the
\glspl{spawnedentry} are defined with \gls{bibglsnewspawnedentry}.

\entrysection{spawnabbreviation}

As \atentry{progenitor}, except that it creates
\atentry{abbreviation} terms instead of \atentry{index}.
As with \atentry{abbreviation}, the \field{short} and \field{long} 
fields are required.

The \glspl{mainentry} are defined in the
\ext{glstex} file with \gls{bibglsnewspawnabbreviation} and the
\glspl{spawnedentry} are defined with \gls{bibglsnewspawnedabbreviation}.

\entrysection{spawnacronym}

As \atentry{progenitor}, except that it creates
\atentry{acronym} terms instead of \atentry{index}.
As with \atentry{acronym}, the \field{short} and \field{long} 
fields are required.

The \glspl{mainentry} are defined in the
\ext{glstex} file with \gls{bibglsnewspawnacronym} and the
\glspl{spawnedentry} are defined with \gls{bibglsnewspawnedacronym}.

\entrysection{spawnsymbol}

As \atentry{progenitor}, except that it creates
\atentry{symbol} terms instead of \atentry{index}.
As with \atentry{symbol}, the required fields are \field{name} or
\field{parent}, and the \field{description} field is required if the
\field{name} field is missing.

The \glspl{mainentry} are defined in the
\ext{glstex} file with \gls{bibglsnewspawnsymbol} and the
\glspl{spawnedentry} are defined with \gls{bibglsnewspawnedsymbol}.

\entrysection{spawnnumber}

As \atentry{progenitor}, except that it creates
\atentry{number} terms instead of \atentry{index}.
As with \atentry{number}, the required fields are \field{name} or
\field{parent}, and the \field{description} field is required if the
\field{name} field is missing.

The \glspl{mainentry} are defined in the
\ext{glstex} file with \gls{bibglsnewspawnnumber} and the
\glspl{spawnedentry} are defined with \gls{bibglsnewspawnednumber}.

\entrysection{spawndualindexentry}

As \atentry{progenitor}, except that the \glsdisp{mainentry}{main} (\gls{progenitor})
entry behaves like \atentry{dualindexentry}. The spawned
\gls{progeny} behave like \atentry{index} are so are all considered 
\glspl{primaryentry}. The \field{adoptparents} field should therefore reference
\glspl{primaryentry} with the default \csopt[parent]{adopted-parent-field}.

The \glsdisp{mainentry}{main} \primary\ and
\glsdisp{secondaryentry}{secondary} (\dual) entries are defined in
the \ext{glstex} file with \gls{bibglsnewspawndualindexentry} and
\gls{bibglsnewspawndualindexentrysecondary}. The spawned
\gls{progeny} are defined with \gls{bibglsnewspawnedindex}.

\section{Compound Entry Sets}
\label{sec:compoundsetentry}

A \gls{compoundentry} isn't an entry in the same sense as the above
but corresponds to a multi-entry (compound or combined) set provided
by \sty{glossaries-extra} v1.48+, which is defined by the command
\ics{multiglossaryentry} (or \ics{providemultiglossaryentry}).
These are referred to as multi-entries in \sty{glossaries-extra} but
are referred to as \glspl{compoundentry} here to avoid confusion
with the \glspl{multientrytype}.

Essentially, a label is defined that refers to a set of labels
corresponding to entries that \emph{have already been defined}. One
element in the set is considered the \gls{compmainlabel}. Entry
labels may appear in multiple sets.

A \gls{compoundentry} provides a convenient way to apply commands
like \ics{gls} to multiple entries in one command (such as \ics{mgls}).
\Gls{compoundentry} labels may only be used in the \cs{mgls}-like
commands or in a \gls{cross-referencefield}.

For example, consider the following document:
\begin{codeenv}
\cmd{documentclass}\marg{article}
\cmd{usepackage}\marg{hyperref}
\cmd{usepackage}[\styopt{record},\styopt[tree]{style}]\marg{glossaries-extra}
\cs{setabbreviationstyle}\marg{long-only-short-only}
\cs{renewcommand}*\marg{\cs{glsxtronlyname}}\marg{\comment{}
  \cs{protect}\cs{glslongonlyfont}\marg{\cs{the}\cs{glslongtok}}\comment{}
}
\cs{newabbreviation}\marg{clostridium}\marg{C.}\marg{Clostridium}
\gls{newglossaryentry}\marg{botulinum}\marg{name=botulinum,
  description=\marg{},parent=clostridium}
\gls{newglossaryentry}\marg{perfringens}\marg{name=perfringens,
  description=\marg{},parent=clostridium}
\cmd{begin}\marg{document}
\cs{gls}\marg{clostridum} \cs{gls}\marg{botulinum},
\cs{gls}\marg{clostridum} \cs{gls}\marg{perfringens},
\cs{gls}\marg{clostridum} \cs{gls}\marg{botulinum}.
\cs{printunsrtglossary}
\cmd{end}\marg{document}
\end{codeenv}
This produces:
\begin{quote}
Clostridium botulinum,
C. perfringens,
C. botulinum.
\end{quote}
followed by the glossary. This is very cumbersome. Defining a
\gls{compoundentry} label simply provides a shortcut:
\begin{codeenv*}
\ics{multiglossaryentry}\marg{cbot}\marg{clostridium,botulinum}
\ics{multiglossaryentry}\marg{cperf}\marg{clostridium,perfringens}
\end{codeenv*}
(This has to be done after the entries have been defined.) Now the
entries can be more compactly referenced:
\begin{codeenv*}
\ics{mgls}\marg{cbot},
\ics{mgls}\marg{cperf},
\ics{mgls}\marg{cbot}.
\end{codeenv*}
Each \gls{compoundentry} set must contain at least two elements. The
\gls{compmainlabel} is the label of the element that is considered
the main entry of the set. If the \gls{compmainlabel} isn't
identified in \cs{multiglossaryentry} then it's assumed to be the
last element in the set.

In the above example, \code{botulinum} is the \gls{compmainlabel} of
the \code{cbot} set, and \code{perfringens} is the
\gls{compmainlabel} of the \code{cperf} set.  In both sets,
\code{clostridium} is the \qt{\gls{compotherlabel}}. If there are
more than two elements in the set then \qt{others} refers to all the
elements except for the \gls{compmainlabel}. An entry can be a
\gls{compmainlabel} of one set and an \gls{compotherlabel} of
another set.

The options, which can be applied to all sets with
\ics{multiglossaryentrysetup} or to a specific set using the
first optional argument of \ics{multiglossaryentry}, determine
if each element of the list has a separate hyperlink to their
own target, or if only the main element should have a hyperlink, or 
if the entire content of \ics{mgls} should be a single hyperlink to
the main entry's target.

With \bibgls, the entries that form the set should be in \ext{bib} files as usual. 
The \gls{compoundentry} set may either be defined in the document \ext{tex}
file using \ics{multiglossaryentry} (or \ics{providemultiglossaryentry})
or they can be defined in the \ext{bib} file using
\atentry{compoundset}. Remember that the set can only be defined
after the entries that make up the elements of the set have been
defined. If any \ext{bib} files in a \gls{resourceset} contain
\atentry{compoundset}, the definitions will be added at the end of
the \ext{glstex} file (using \gls{bibglsdefcompoundset}).

If you have multiple \glspl{resourceset} that reuse the same
\ext{bib} file containing \atentry{compoundset} then either redefine
\gls{bibglsdefcompoundset} to use \ics{providemultiglossaryentry} or
prevent duplicate definitions with \csopt[false]{compound-write-def}.

The elements of the set will still need to be indexed as usual to
ensure that they have \glspl{record} to enable selection.

The above example can be converted to \bibgls\ as follows
(\glspl{compoundentry} defined in the document \ext{tex} file):
\begin{codeenv}
\cmd{documentclass}\marg{article}
\cmd{usepackage}\marg{hyperref}
\cmd{usepackage}[\styopt{record},\styopt[tree]{style}]\marg{glossaries-extra}
\cs{setabbreviationstyle}\marg{long-only-short-only}
\cs{renewcommand}*\marg{\cs{glsxtronlyname}}\marg{\comment{}
  \cs{protect}\cs{glslongonlyfont}\marg{\cs{the}\cs{glslongtok}}\comment{}
}
\gls{GlsXtrLoadResources}\oarg{\csopt[bacteria]{src}}
\cs{multiglossaryentry}\marg{cbot}\marg{clostridium,botulinum}
\cs{multiglossaryentry}\marg{cperf}\marg{clostridium,perfringens}
\cmd{begin}\marg{document}
\cs{mgls}\marg{cbot}, \cs{mgls}\marg{cperf}, \cs{mgls}\marg{cbot}.
\cs{printunsrtglossary}
\cmd{end}\marg{document}
\end{codeenv}
Note that \cs{multiglossaryentry} must come after \gls{GlsXtrLoadResources}.

The \file{bacteria.bib} contains the definitions in the usual way:
\begin{codeenv}
\atentry{abbreviation}\marg{clostridium,
 \field{short}=\marg{C.},
 \field{long}=\marg{Clostridium}
}
\atentry{index}\marg{botulinum,
 \field{parent}=\marg{clostridium}
}
\atentry{index}\marg{perfringens,
 \field{parent}=\marg{clostridium}
}
\end{codeenv}

Alternatively, the \glspl{compoundentry} can be defined in the
\ext{bib} file instead:
\begin{codeenv}
\atentry{compoundset}\marg{cbot,
  \field{elements}=\marg{clostridium,botulinum}
}
\atentry{compoundset}\marg{cperf,
  \field{elements}=\marg{clostridium,perfringens}
}
\end{codeenv}
The \cs{multiglossaryentry} commands should now be removed from the
\ext{tex} file.

There's a difference between these two methods on the first \LaTeX\
build. In the first example, \code{cbot} is known, so
\code{\cs{mgls}\marg{cbot}} can perform
\code{\cs{gls}\marg{clostridum} \cs{gls}\marg{botulinum}}. These
commands aren't yet defined so they are both replaced by \qt{??}
(resulting in \qt{?? ??}). As usual, the \gls{locationlist} is
unreliable until entries are defined and the unknown markers \qt{??}
can be replaced with the correct content. If the document is in a
file called \filefmt{myDoc.tex} then the document build:
\begin{codeenv}
pdflatex myDoc
bib2gls myDoc
pdflatex myDoc
\end{codeenv}
will have \glspl{location} in the resulting PDF file, but they may
be incorrect if the associated temporary files were initially
missing.

In the second example, \code{cbot} is unknown, so
\code{\cs{mgls}\marg{cbot}} is simply displayed as \qt{??}.
In this case, the \ext{aux} file contains information that
\code{cbot} has been referenced, but there are no associated
\glspl{record}. The entries that belong to the \code{cbot} set will
be selected as they are considered dependent on the
\gls{compoundentry}. In this case, if you are starting from scratch
(no associated temporary files), you will need:
\begin{codeenv}
pdflatex myDoc
bib2gls myDoc
pdflatex myDoc
bib2gls myDoc
pdflatex myDoc
\end{codeenv}
At this point, the \glspl{locationlist} will appear.
After that, you can reduce the document build to:
\begin{codeenv}
pdflatex myDoc
bib2gls myDoc
pdflatex myDoc
\end{codeenv}
(Until you later add new entries.)

If you don't want \glspl{location} for the 
\glslink{compotherlabel}{other elements} then set the 
\idx{encap} to \encap{glsignore}:
\begin{codeenv*}
\ics{multiglossaryentrysetup}\marg{encapothers=\encap{glsignore}}
\end{codeenv*} 

\entrysection{compoundset}
The following fields are available:
\begin{definition}
\item[\field{elements}] The comma-separated list of element labels.
This corresponds to the final argument of \cs{multiglossaryentry}.
(Required.)
\item[\field{main}] The \gls{compmainlabel}. This field is optional.
If omitted, the main label is assumed to be the last element.
\item[\field{options}] A comma-separated list of options. This
corresponds to the first optional argument of
\cs{multiglossaryentry}. This field may be omitted.
\end{definition}
These fields can only be used in this \gls{entrytype}.

Most resource options don't apply to this \gls{entrytype}. Options
specific to \glspl{compoundentry} are listed in
\sectionref{sec:compoundentries}.

\chapter{Resource File Options}
\label{sec:resourceopts}

Make sure that you use \isty{glossaries-extra} with the
\styopt{record} package option. This ensures that \bibgls\ can pick
up the required information from the \iext{aux} file, and both 
\styopt[only]{record} and \styopt[nameref]{record} additionally
load the supplementary \isty{glossaries-extra-bib2gls} package. 
These two \styopt{record} option values also
switch on the \styopt[none]{sort} package option (if you have a new
enough version of the base \isty{glossaries} package), which means
that there's no attempt to assign or process the \field{sort} key if
it's omitted from \gls{newglossaryentry} (or similar commands). The
\field{sort} key will be provided by \bibgls\ for informational
purposes, but there's no need for \LaTeX\ to write it to any
external files (unless you use \styopt[hybrid]{record}, 
in which case you need to prevent \bibgls\ from sorting using the
\csopt[none]{sort} resource option).

The \iext{glstex} resource files created by \bibgls\ are loaded in
the document using
\nosecformatdef{glsxtrresourcefile}
where \meta{filename} is the name of the resource file without the
\ext{glstex} extension.
You can have multiple \gls{glsxtrresourcefile} commands within your
document, but each \meta{filename} must be unique, otherwise \LaTeX\
would attempt to input the same \ext{glstex} file multiple times
(\bibgls\ checks for non-unique file names). The associated data for each
resource file is called the \igls{resourceset} (see
\sectionref{sec:resourcesets}).

There's a shortcut command that uses
\ics{jobname} in the \meta{filename}:
\nosecformatdef{GlsXtrLoadResources}
The first instance of this command is equivalent to:
\begin{codeenv}
\gls{glsxtrresourcefile}\oargm{options}\marg{\ics{jobname}}
\end{codeenv}
Any additional use of \gls{GlsXtrLoadResources} is equivalent to:
\begin{codeenv}
\gls{glsxtrresourcefile}\oargm{options}\marg{\ics{jobname}-\meta{n}}
\end{codeenv}
where \meta{n} is number. For example:
\begin{codeenv}
\gls{GlsXtrLoadResources}\oarg{\csopt[entries-en]{src},\csopt[en]{sort}}
\gls{GlsXtrLoadResources}\oarg{\csopt[entries-fr]{src},\csopt[fr]{sort}}
\gls{GlsXtrLoadResources}\oarg{\csopt[entries-de]{src},\csopt[de-1996]{sort}}
\end{codeenv}
This is equivalent to:
\begin{codeenv}
\gls{glsxtrresourcefile}\oarg{\csopt[entries-en]{src},\csopt[en]{sort}}\marg{\cs{jobname}}
\gls{glsxtrresourcefile}\oarg{\csopt[entries-fr]{src},\csopt[fr]{sort}}\marg{\cs{jobname}-1}
\gls{glsxtrresourcefile}\oarg{\csopt[entries-de]{src},\csopt[de-1996]{sort}}\marg{\cs{jobname}-2}
\end{codeenv}
In general, it's simplest just to use \gls{GlsXtrLoadResources}.

The optional argument \meta{options} is a comma-separated
\keyvallist.  Allowed options are listed below.  The option list
applies only to that specific \meta{filename}\ext{glstex} and are
not carried over to the next instance of \gls{glsxtrresourcefile}.
Only the definitions provided in \atentry{preamble} (if the
interpreter is on and \csopt[true]{interpret-preamble}) are carried
over to the next resource set and, possibly,
\iglspl{crossresourceref} if permitted (see
\sectionref{sec:resourcesets}).  The \sty{glossaries-extra} package
doesn't parse the options, but just writes the information to the
\ext{aux} file. This means that any invalid options will be reported
by \bibgls\ not by \sty{glossaries-extra}.

As from \sty{glossaries-extra} v1.40 you can provide a default set
of options by redefining:
\nosecdef{GlsXtrDefaultResourceOptions}
This command will be inserted at the start of the options list for
all resource commands (and will expand as it's written to
the \ext{aux} file).  For example:
\begin{codeenv}
\cs{renewcommand}\marg{\gls{GlsXtrDefaultResourceOptions}}\marg{\comment{}
  \csopt[all]{selection},\csopt[entries]{src}}
\gls{GlsXtrLoadResources}\oarg{
  \csopt[symbols]{type},
  \csopt[entrytype=symbol]{match}}
\gls{GlsXtrLoadResources}\oarg{
  \csopt[abbreviations]{type},
  \csopt[entrytype=abbreviation]{match}}
\end{codeenv}
This acts like:
\begin{codeenv}
\gls{GlsXtrLoadResources}\oarg{
  \csopt[all]{selection},\csopt[entries]{src},
  \csopt[symbols]{type},
  \csopt[entrytype=symbol]{match}}
\gls{GlsXtrLoadResources}\oarg{
  \csopt[all]{selection},\csopt[entries]{src},
  \csopt[abbreviations]{type},
  \csopt[entrytype=abbreviation]{match}}
\end{codeenv}

If you have multiple \ext{bib} files you can either select them all
using \csopt[\meta{bib list}]{src} in a single
\gls{glsxtrresourcefile} call, if they all require the same settings,
or you can load them separately with different settings applied.

For example, if the files \filefmt{entries-terms.bib} and
\filefmt{entries-symbols.bib} have the same settings:
\begin{codeenv}
\gls{GlsXtrLoadResources}\oarg{\csopt[entries-terms,entries-symbols]{src}}
\end{codeenv}
Alternatively, if they have different settings:
\begin{codeenv}
\gls{GlsXtrLoadResources}\oarg{\csopt[entries-terms]{src},\csopt[main]{type}}
\gls{GlsXtrLoadResources}\oarg{\csopt[entries-symbols]{src},\csopt[use]{sort},\csopt[symbols]{type}}
\end{codeenv}

Note that the sorting is applied to each \igls{resourceset} independently
of other \iglspl{resourceset}. This means that if you have multiple instances
of \gls{glsxtrresourcefile} but only one glossary type, the glossary
will effectively contain blocks of sorted entries. For example, if
\filefmt{file1.bib} contains:
\begin{codeenv}
\atentry{index}\marg{duck}
\atentry{index}\marg{zebra}
\atentry{index}\marg{aardvark}
\end{codeenv}
and \filefmt{file2.bib} contains:
\begin{codeenv}
\atentry{index}\marg{caterpillar}
\atentry{index}\marg{bee}
\atentry{index}\marg{wombat}
\end{codeenv}
then
\begin{codeenv}
\gls{GlsXtrLoadResources}\oarg{\csopt[file1,file2]{src}}
\end{codeenv}
will result in the list: aardvark, bee, caterpillar, duck, wombat,
zebra. These six entries are all defined when
\csfmt{jobname}\filefmt{.glstex} is read.
Whereas
\begin{codeenv}
\gls{GlsXtrLoadResources}\oarg{\csopt[file1]{src}}
\gls{GlsXtrLoadResources}\oarg{\csopt[file2]{src}}
\end{codeenv}
will result in the list: aardvark, duck, zebra, bee, caterpillar,
wombat. The first three (aardvark, duck, zebra) are defined when 
\csfmt{jobname}\filefmt{.glstex} is read. The second three (bee,
caterpillar, wombat) are defined when \csfmt{jobname}\filefmt{-1.glstex}
is read. Since \ics{printunsrtglossary} simply iterates over all
defined entries, this is the ordering used.

Abbreviation styles must be set (using \ics{setabbreviationstyle})
before the resource command that selects the abbreviations from the
appropriate \ext{bib} file, since the entries are defined (through
\gls{newabbreviation} or \gls{newacronym}) when
\gls{glsxtrresourcefile} inputs the \ext{glstex} file. (Similarly for any
associated abbreviation style commands that must be set before
abbreviations are defined, such as \cs{glsxtrlongshortdescname}.)

Note \bibgls\ allows \iext{bib} files that don't provide any entries.
This can be used to provide commands in \atentry{preamble}.
For example, suppose I have \filefmt{defs.bib} that just contains:
\begin{codeenv}
\atentry{preamble}\marg{\qtdelim{\cs{providecommand}\marg{\cmd{strong}}[1]\marg{\cs{textbf}\marg{\ics{cs.color}\marg{red}\idx{param}1}}
\cs{providecommand}\marg{\cmd{parenswap}}[2]\marg{\idx{param}2 (\idx{param}1)}}}
\end{codeenv}
This provides two commands:
\nosecdef{strong}
(which sets the font weight and colour) and 
\nosecdef{parenswap}
(which just displays its second argument followed by the first in parentheses).

Suppose I also have \filefmt{entries.bib} that contains:
\begin{codeenv}
\atentry{index}\marg{example,
  \field{name}=\marg{\gls{strong}\marg{\gls{parenswap}\marg{stuff}\marg{example}}}
}
\atentry{index}\marg{sample}
\atentry{index}\marg{test}
\atentry{index}\marg{foo}
\atentry{index}\marg{bar}
\end{codeenv}
This contains an entry that requires the commands provided in
\filefmt{defs.bib}, so to ensure those commands are defined, I can
do:
\begin{codeenv}
\gls{GlsXtrLoadResources}\oarg{\csopt[defs,entries]{src}}
\end{codeenv}
Unfortunately this results in the sort value for \code{example} being
set to \code{redexample (stuff)} because the interpreter has
detected the provided commands and expanded:
\begin{codeenv}
\gls{strong}\marg{\gls{parenswap}\marg{stuff}\marg{example}}
\end{codeenv}
to:
\begin{codeenv}
\cs{textbf}\marg{\cs{cs.color}\marg{red}example (stuff)}
\end{codeenv}
It discards font changes, so \cs{textbf} is ignored, but it doesn't 
recognise \cs{cs.color} and so doesn't know that the first argument is
just the colour specifier and therefore doesn't discard it.
This means that \qt{\textbf{\color{red}example (stuff)}} is placed
between \qt{foo} and \qt{sample} instead of between \qt{bar} and
\qt{foo}.

I can prevent the interpreter from parsing \atentry{preamble}:
\begin{codeenv}
\gls{GlsXtrLoadResources}\oarg{\csopt[defs,entries]{src},\csopt[false]{interpret-preamble}}
\end{codeenv}
Now when the sort value for \code{example} is obtained from:
\begin{codeenv}
\gls{strong}\marg{\gls{parenswap}\marg{stuff}\marg{example}}
\end{codeenv}
no expansion occurs (since \gls{strong} and \gls{parenswap} are now
unrecognised) so the sort value ends up as:
\code{stuffexample}
which places \qt{\textbf{\color{red}example (stuff)}} between 
\qt{sample} and \qt{test}, which is again incorrect.

The best thing to do in this situation is to split the provided
commands into two \iext{bib} files: one that shouldn't be interpreted
and one that should.

For example, \filefmt{defs-nointerpret.bib}:
\begin{codeenv}
\atentry{preamble}\marg{\qtdelim{\cs{providecommand}\marg{\gls{strong}}[1]\marg{\cs{textbf}\marg{\cs{cs.color}\marg{red}\idx{param}1}}}}
\end{codeenv}
and \filefmt{defs-interpret.bib}:
\begin{codeenv}
\atentry{preamble}\marg{\qtdelim{\cs{providecommand}\marg{\gls{parenswap}}[2]\marg{\idx{param}2 (\idx{param}1)}}}
\end{codeenv}
Now the first one can be loaded with \csopt[false]{interpret-preamble}:
\begin{codeenv}
\gls{GlsXtrLoadResources}\oarg{\csopt[defs-nointerpret]{src},\csopt[false]{interpret-preamble}}
\end{codeenv}
This creates a \iext{glstex} file that provides \gls{strong} but
doesn't define any entries. The other file \filefmt{defs-interpret.bib} can
then be loaded with the default \csopt[true]{interpret-preamble}:
\begin{codeenv}
\gls{GlsXtrLoadResources}\oarg{\csopt[defs-interpret,entries]{src}}
\end{codeenv}
The provided commands are remembered by the interpreter, so you can
also do:
\begin{codeenv}
\gls{GlsXtrLoadResources}\oarg{\csopt[defs-interpret]{src}}
\gls{GlsXtrLoadResources}\oarg{\csopt[entries]{src}}
\end{codeenv}
The \emph{contents} of \atentry{preamble} are only written to the
associated \iext{glstex} file, but the definitions contained within
the \atentry{preamble} are retained by the interpreter for subsequent
\iglspl{resourceset}.

\section{String Concatenation}
\label{sec:optstringconcat}

Some resource options allow string \gls{concatenation} in their
syntax. This is where fragments or substrings can be joined together
to form a value. This is similar to the way \gls{concatenation}
occurs in \ext{bib} files, but a different operator is used. In
\ext{bib} files, the concatenation operator is \idx{stringconcat} (hash)
but, since this is a problematic character to use in the optional
argument of \gls{GlsXtrLoadResources}, the operator for string
\gls{concatenation} in resource options is \idx{concat-plus} (plus).

A string \gls{concatenation} \meta{element-list} in a resource option
has the following syntax:
\begin{codeenv}
\meta{element-list} ::= \meta{element-value} | \meta{element-value} \idx{concat-plus} \meta{element-list}
\meta{element-value} ::= \meta{string} | \meta{field-ref} | \meta{element-quark}\margm{element-list} | \meta{match-ref}
\meta{match-ref} ::= \gls{MGP}\margm{group-ref}
\meta{group-ref} ::= \meta{index} | \meta{name}
\meta{string} ::= \qtdelim{\meta{tokens}} | \margm{tokens}
\end{codeenv}
The \meta{field-ref} syntax is described below, and is used to
reference a field value.
The element quarks (\meta{element-quark}, described below) take an
\meta{element-list} argument. If the \meta{element-list} argument evaluates
to null, they will return null.

\begin{important}
Remember that the argument of \gls{GlsXtrLoadResources} is expanded
as it's written to the \ext{aux} file. This means that care must be
taken to prevent premature expansion of \idxpl{quark} or any
commands that need to be present in a string.
\end{important}

As from \styfmt{glossaries-extra} v1.51, the
\sty{glossaries-extra-bib2gls} package (which is automatically
loaded with the \styopt{record} option) provides the command
\ics{GlsXtrResourceInitEscSequences} which will locally redefine
these \idx{quark} commands to expand to their detokenized form.
So you can do:
\begin{codeenv}
\cs{renewcommand}*\marg{\cs{glsxtrresourceinit}}\marg{\comment{}
  \ics{GlsXtrResourceInitEscSequences}
}
\end{codeenv}
This means that you can simply write the \idx{quark} in the resource
option without needing to use \ics{protect} or \idx{cs.string}.
The remainder of this section assumes that \ics{glsxtrresourceinit}
has been redefined to use \ics{GlsXtrResourceInitEscSequences}, as
in the above example.

As with the \ext{bib} format, strings (\meta{string}) can be delimited by braces
\code{\margm{text}} or double-quotes \code{\qtdelim{\meta{text}}}. If you
need a literal double-quote (\idx{doublequotechar}) then either use
brace delimiters or use \gls{quark.doublequote}. If you need the
actual \LaTeX\ accent command \ics{umlaut} then use brace
delimiters. If you need braces that start and end in different
strings then use double-quote delimiters. For example:
\begin{codeenv}
\csopt[
 \field{first} = "\gls{cs}\marg{emph}\idx{bgroupchar}" \idx{concat-plus} \field{name} \idx{concat-plus} "\idx{egroupchar}"
]{assign-fields}
\end{codeenv}

The \meta{element-list} may just contain a single element, such as a
field reference or a constant string, but it must still conform to
the element syntax. For example, if you want to use
\csopt{copy-to-glossary} to copy all entries to a specific glossary,
such as \code{index}, then you will need to markup \code{index} as a
string. For example:
\begin{codeenv}
\csopt[\qtdelim{index}]{copy-to-glossary}
\end{codeenv}
or
\begin{codeenv}
\csopt[\marg{index}]{copy-to-glossary}
\end{codeenv}
Note that the outer braces are stripped by the resource option
parser before the content is parsed as an \meta{element-list}.
If only a single set of braces was used, those braces would be
stripped leaving a bare \code{index}, which would be parsed as a
field reference.

The element \idxpl{quark} are uppercase tokens that start with a leading
backslash. They have no meaning to \bibgls's interpreter nor are
they defined in the \LaTeX\ document outside of the scope of 
the resource command (unless they happen to coincidentally be
defined by another package or are a custom command). \Idxpl{quark}
occur outside of strings. Any escape sequences occurring within a
string are considered to be \LaTeX\ commands.

\nosecdef{CS}
Returns a control sequence with the control sequence
name obtained from concatenating \meta{element-list}. 
Note that this is different from
\gls{cs} which expands to the detokenized control sequence name as
the resource options are written to the \ext{aux} file.

For example, if the \LaTeX\ file has:
\begin{codeenv}
\gls{GlsXtrLoadResources}\oarg{
 \csopt[
   \field{name} = "\gls{cs}\marg{foo}\marg{" \idx{concat-plus} \field{user1} \idx{concat-plus} "}"
 ]{assign-fields}
}
\end{codeenv}
then this will expand the options to the \ext{aux} file as
\begin{codeenv}
\csopt[
\field{name} = "\csfmt{foo}\marg{" \idx{concat-plus} \field{user1} \idx{concat-plus} "}"
]{assign-fields}
\end{codeenv}
Compare this with:
\begin{codeenv}
\gls{GlsXtrLoadResources}\oarg{
 \csopt[
   \field{name} = \gls{CS} \marg{ \field{user1} }
 ]{assign-fields}
}
\end{codeenv}
which will set the \field{name} value to \csfmt{\meta{csname}} (no
arguments) where \meta{csname} is the value obtained from the
\field{user1} field for that entry. Note that \csfmt{\meta{csname}}
will need to be defined in the document to ensure that the document
compiles without error but will also need to be recognised by
\bibgls\ if the field value needs to be interpreted (such as when
obtaining the \field{sort} value).

\nosecdef{MGP}
The \meta{match-ref} element should only be used with a
\gls{regular-expression} from an associated conditional (see
\sectionref{sec:conditionals}).
For example, the \meta{condition} part of an assignment rule in
\csopt{assign-fields}.

If a match was found, \gls{MGP} can be used to reference a group within the match. 
The \meta{group-ref} argument may be either an integer (the group
index) or the group name. For example, suppose a custom field called
\fieldfmt{ordinal} may contain content such as \code{1st} or
\code{10th} and I want to encapsulate the suffix part without
altering the \ext{bib} file. This can be done as follows:
\begin{codeenv}
\csopt[
 \fieldfmt{ordinal} =\oarg{o} \gls{MGP}\marg{1} + " \gls{cs}\marg{ord}\marg{" + \gls{MGP}\marg{2} + "}"
   \oarg{ ordinal=/(\gls{cs}\marg{d}+)(st|nd|rd|th)/ }
]{assign-fields}
\end{codeenv}
Alternatively, using named groups:
\begin{codeenv*}
\csopt[
 ordinal =\oarg{o} \gls{MGP}\marg{num} + " \gls{cs}\marg{ord}\marg{" + \gls{MGP}\marg{suffix} + "}"
   \oarg{ ordinal=/(\idx{regex.question}\idx{regex.lt}num\idx{regex.gt}\gls{cs}\marg{d}+)(\idx{regex.question}\idx{regex.lt}suffix\idx{regex.gt}st|nd|rd|th)/ }
]{assign-fields}
\end{codeenv*}
Note that the group name shouldn't be delimited with double-quotes.

\begin{important}
The \gls{MGP} quark (which expands to the \gls{MGP} identifier for
\csopt{assign-fields}) isn't the same as \ics{glscapturedgroup}
(which expands to \cs{cs.string}\idx{dollarchar}, allowing a dollar
character to be written to the \ext{aux} file within
the replacement part of \csopt{labelify-replace}).
\end{important}

\nosecdef{TRIM}
Returns its argument with any leading and trailing spaces removed.

\nosecdef{INTERPRET}
Interprets the contents of \meta{element-list} using \bibgls's
interpreter and returns the result, which may be an empty string if
the content only contains unknown commands. Note that the result is 
a string with any special characters replaced by detokenized commands, such as
\ics{glsbackslash}. This ensures that the value is suitable to be written
in the entry definition in the \ext{glstex} file.

It's important to remember that the result of \gls{INTERPRET} is a detokenized string.
In general, it's therefore best to have \gls{INTERPRET} as the outermost quark.

Suppose I have a \ext{bib} file that contains:
\begin{codeenv}
\atentry{preamble}\marg{\qtdelim{\cs{providecommand}\marg{\cmd{csfmt}}[1]\marg{\cs{glsbackslash} \idx{param}1}}}
\atentry{index}\marg{relax,\field{name}=\marg{\cmd{csfmt}\marg{relax}}}
\atentry{index}\marg{comment,\field{name}=\marg{\cs{cs.percent} comment}}
\end{codeenv}
If the resource options include:
\begin{codeenv}
\csopt[
 \field{user1} = \gls{INTERPRET} \marg{ \field{name} } ,
]{assign-fields}
\end{codeenv}
Then the definitions in the \ext{glstex} file will be:
\begin{codeenv}
\gls{bibglsnewindex}\marg{comment}\comment{}
\marg{\field{user1}=\marg{\cs{glspercentchar} \cs{space} comment},
\field{name}=\marg{\cs{cs.percent} comment},
\field{sort}=\marg{comment|}}
\strut
\gls{bibglsnewindex}\marg{relax}\comment{}
\marg{\field{user1}=\marg{\cs{glsbackslash} relax},
\field{name}=\marg{\cmd{csfmt}\marg{relax}},
\field{sort}=\marg{relax|}}
\end{codeenv}
This ensures that the \LaTeX\ syntax in the \ext{glstex} definitions is valid and 
the \field{user1} field can expand to a literal string. (Remember that the document
definition of the custom \csfmt{csfmt} command may be different from the one provided
in the \atentry{preamble} content.)

The \gls{INTERPRET} quark converts \code{\cs{cs.percent} comment} to 
\qtt{\% comment} since the interpreter recognises \ics{cs.percent}. However, 
if this value is written to the \ext{glstex} file:
\begin{codeenv}
\field{user1}=\marg{\% comment},
\end{codeenv}
then this is invalid. When the \ext{glstex} file is input by \gls{GlsXtrLoadResources},
\LaTeX\ will interpret the percent symbol \code{\%} as a comment and the rest of the
line, including the closing brace, will be ignored. This leads to a missing closing brace error.

In the case of the \qt{relax} entry, since a definition of the custom \csfmt{csfmt}
is provided in \atentry{preamble}, the interpreter will convert \code{\csfmt{csfmt}\marg{relax}}
to the string \qtt{\csfmt{relax}}. If this value is written to the \ext{glstex} file:
\begin{codeenv}
\field{user1}=\marg{\csfmt{relax}},
\end{codeenv}
then this would be parsed as the command \csfmt{relax} when the file is input
by \gls{GlsXtrLoadResources}.

Therefore, \gls{INTERPRET} automatically replaces all the \TeX\ special characters
in the resulting string to ensure that they expand to their literal meanings 
in the \LaTeX\ document.

Now suppose instead that the resource options include:
\begin{codeenv}
\csopt[
 \field{user1} = \gls{FIRSTUC} \marg{ \gls{INTERPRET} \marg{ \field{name} } },
]{assign-fields}
\end{codeenv}
This first interprets the value of the \field{name} field and substitutes the special
characters (so \code{\csfmt{relax}} becomes the string \qtt{\csfmt{glsbackslash} relax}). 
The resulting string is then converted to \idx{sentencecase}, which results in
\qtt{\csfmt{Glsbackslash} relax} because the initial backslash is now a literal character,
and since it's a non-letter character it's skipped by the case-conversion.
So now the code written to the \ext{glstex} file is:
\begin{codeenv}
\gls{bibglsnewindex}\marg{comment}\comment{}
\marg{\field{user1}=\marg{\csfmt{Glspercentchar} \cs{space} comment},
\field{name}=\marg{\cs{cs.percent} comment},
\field{sort}=\marg{comment|}}
\strut
\gls{bibglsnewindex}\marg{relax}\comment{}
\marg{\field{user1}=\marg{\csfmt{Glsbackslash} relax},
\field{name}=\marg{\cmd{csfmt}\marg{relax}},
\field{sort}=\marg{relax|}}
\end{codeenv}
This is invalid because \csfmt{Glspercentchar} and \csfmt{Glsbackslash} are not defined.

Suppose the order of the quarks is swapped:
\begin{codeenv}
\csopt[
 \field{user1} = \gls{INTERPRET} \marg{ \gls{FIRSTUC} \marg{ \field{name} } },
]{assign-fields}
\end{codeenv}
This results in valid \LaTeX\ code:
\begin{codeenv}
\gls{bibglsnewindex}\marg{comment}\comment{}
\marg{\field{user1}=\marg{\cs{glspercentchar} \cs{space} comment},
\field{name}=\marg{\cs{cs.percent} comment},
\field{sort}=\marg{comment|}}
\strut
\gls{bibglsnewindex}\marg{relax}\comment{}
\marg{\field{user1}=\marg{\cs{glsbackslash} Relax},
\field{name}=\marg{\cmd{csfmt}\marg{relax}},
\field{sort}=\marg{relax|}}
\end{codeenv}
If you don't want the argument of a command to be affected by case-changing
commands, you can use \ics{MFUblocker} or \ics{MFUexcl}. The 
\sty{glossaries-extra} package writes this information in the \ext{aux}
file for the benefit of \bibgls. See the \sty{mfirstuc}
manual for further details of those commands.

\nosecdef{INTERPRETNOREPL}
As \gls{INTERPRET} but doesn't replace special characters.
Consider an adjustment to the previous example:
\begin{codeenv}
\csopt[
 \field{user1} = \gls{FIRSTUC} \marg{ \gls{INTERPRETNOREPL} \marg{ \field{name} } },
]{assign-fields}
\end{codeenv}
Now \code{\csfmt{csfmt}\marg{relax}} is converted into the literal string
\qtt{\csfmt{relax}} and then the case-conversion is performed,
which results in the literal string \qtt{\csfmt{Relax}} but this will become
an undefined command in the \ext{glstex} file:
\begin{codeenv}
\field{user1}=\marg{\csfmt{Relax}},
\end{codeenv}
In the case of the \qt{comment} entry, \code{\cs{cs.percent} comment} will 
be converted to the literal string \qtt{\% comment}, which will then have the case-conversion 
applied, but the \ext{glstex} file will be invalid:
\begin{codeenv}
\field{user1}=\marg{\% Comment},
\end{codeenv}

\nosecdef{REPLACESPCHARS}
Detokenises and replaces special characters with commands like \ics{glsbackslash}.
For example:
\begin{codeenv}
\csopt[
 \field{user1} = \gls{REPLACESPCHARS} \marg{
   \gls{FIRSTUC} \marg{ \gls{INTERPRETNOREPL} \marg{ \field{name} } } },
]{assign-fields}
\end{codeenv}
This rather long-winded assignment produces the same result as:
\begin{codeenv}
\csopt[
 \field{user1} = \gls{INTERPRET} \marg{ \gls{FIRSTUC} \marg{ \field{name} } },
]{assign-fields}
\end{codeenv}
A difference can be observed if the custom command is made a blocker or exclusion
using \ics{MFUblocker} or \ics{MFUexcl}.

\nosecdef{LABELIFY}
Converts the contents of \meta{element-list} into a label string,
according to the \csopt{labelify} criteria.

\nosecdef{LABELIFYLIST}
Converts the contents of \meta{element-list} into a label-list string,
according to the \csopt{labelify-list} criteria.

\nosecdef{LEN}
When used within an element list, \gls{LEN} returns the length of
its \meta{element-list} argument as a string or null if \meta{element-list}
evaluates to null. Note that this is different from using \gls{LEN}
in a numerical condition where the result is always an integer (see
\sectionref{sec:conditionals}).
This means that \code{\gls{LEN}\margm{list1} \idx{concat-plus}
\gls{LEN}\margm{list2}} performs string \gls{concatenation} not numerical
addition. Instead, use \code{\gls{LEN}\marg{\meta{list1}
\idx{concat-plus} \meta{list2}}} for the combined length.

The length is the detokenised length, for example,
if the \field{name} field has the value \code{\cs{emph}\marg{x}}
then \code{\gls{LEN}\marg{\field{name}}} will evaluate to the string
\qtdelim{8}. You can use
\begin{codeenv}
\gls{LEN}\marg{\gls{INTERPRET}\margm{element-list}}
\end{codeenv}
to find the length without \LaTeX\ commands.

The \idxpl{quark} below identify case-changing functions. The
\meta{element-list} argument will be converted using the appropriate
function and the result will be returned. If \meta{element-list}
evaluates to null then null will be returned.

The case-changing functions will use the \gls{resource-locale}, but
whether or not \bibgls\ recognises the correct rules for the locale
depends on whether or not the locale is correctly supported by the
Java locale provider. The \langxml\ may provide assistance with
case-conversion (see \sectionref{sec:lang.xml}).  Note that the
case-change is performed by \bibgls\ not by inserting \LaTeX\
case-changing commands into the code. 

\begin{itemize}
\item \inlinedef{LC} converts \meta{element-list} to \idx{lowercase};
\item \inlinedef{UC} converts \meta{element-list} to \idx{uppercase};
\item \inlinedef{FIRSTLC} converts the first letter of 
\meta{element-list} to \idx{lowercase};
\item \inlinedef{FIRSTUC} converts the first letter of
\meta{element-list} to \idx{uppercase} (\idx{sentencecase});
\item \inlinedef{TITLE} converts \meta{element-list} to \idx{titlecase}.
\end{itemize}

There is an additional token \inlinedef{NOCHANGE} which simply
evaluates \meta{element-list} and returns it
unchanged.\footnote{The \gls{NOCHANGE} support wasn't
intentional, but was simply a by-product of the original
implementation of the case-changing commands.} This isn't
like \cs{NoCaseChange} but is more like \cs{@firstofone}. There is
little need for it so it's not defined by
\ics{GlsXtrResourceInitEscSequences}. The only plausible use for it
is if you have a class or package that contains something like:
\begin{codeenv}
\csfmt{newcommand}\marg{\csfmt{mycase}}\marg{NOCHANGE}
\comment{later as the result of some condition:}
\csfmt{renewcommand}\marg{\csfmt{mycase}}\marg{FIRSTUC}
\comment{later on:}
\gls{GlsXtrLoadResources}\oarg{
 \csopt[
   \field{name}=\oarg{o} \gls{cs}\marg{\csfmt{mycase}}\marg{\field{name}},
\comment{other assignments \ldots}
 ]{assign-fields}
}
\end{codeenv}
In most cases, it should be possible to achieve the same result with a
conditional associated with the resource option or by adjusting the
content passed to the resource command. For example:
\begin{codeenv}
\csfmt{newcommand}\marg{\csfmt{nameassign}}\marg{}
\comment{later as the result of some condition:}
\csfmt{renewcommand}\marg{\csfmt{nameassign}}\marg{name=\oarg{o}\gls{FIRSTUC}\marg{name},}
\comment{later on:}
\gls{GlsXtrLoadResources}\oarg{
 \csopt[
   \csfmt{nameassign}
\comment{other assignments \ldots}
 ]{assign-fields}
}
\end{codeenv}

The field reference (\meta{field-ref}) syntax is more complicated:
\begin{codeenv}
\meta{field-ref} ::= \meta{value-ref} | \meta{entry-ref} \idx{follow} \meta{field-ref}
\meta{entry-ref} ::= self | parent | root
\meta{value-ref} ::= \meta{field-name} | \meta{label-ref}
\meta{label-ref} ::= \meta{label-type} \idx{follow} \meta{label-delineator}
\meta{label-type} ::= entrytype | entrylabel | entrybib
\meta{label-delineator} ::= original | actual
\end{codeenv}
where \meta{field-name} is the required field name. Note that field
names (which need to be used in a string \gls{concatenation}) can't include 
any of the \gls{concatenation} or conditional markup special characters:
\idx{concat-plus} \idx{startoptional} \idx{endoptional}
\idx{equalsassign} \idx{commasep} \idx{ltcmp} \idx{gtcmp} or 
\idx{doublequotechardelim}.

The \meta{entry-ref} part indicates which entry the referenced field
belongs to.  The keywords are: \code{self} (the entry itself),
\code{parent} (the entry's \parent), and \code{root} (the entry's
\glsdisp{hierarchicalglossary}{hierarchical root}, not including the
entry itself). Note that with options such as
\csopt{assign-fields} the entry's \glspl{ancestor} must be
defined before the entry in the \ext{bib} file because their fields
can only be referenced after they have been processed.  A
grandparent entry can be referenced with \code{parent \idx{follow}
parent \idx{follow}}.  Since \qt{parent} is also a field name, if
the keyword \code{parent} is followed by \idx{follow} then the
keyword refers to the \gls{parententry} otherwise it refers to the
\field{parent} field. For example, 
\code{self \idx{follow} parent} refers to the value of the entry's
\field{parent} field, which is the \gls{parententry}['s] label, whereas
\code{parent \idx{follow} name} refers to the value of the entry's parent's 
\field{name} field.


The special keywords identify values that aren't normally stored in
a field.  The keyword must be followed by the \meta{delineator},
which may be \code{original} or \code{actual}. Available keywords:

\begin{description}

\item[\code{entrytype}] the \gls{entrytype}, without the leading
\code{@}, where \code{original} refers to the original
\gls{entrytype} used in the \ext{bib} file and \code{actual} refers
to the actual \gls{entrytype}, which may have changed as a result of
\csopt{entry-type-aliases};

\item[\code{entrylabel}] the entry label, where \code{original}
refers to the original label used in the \ext{bib} file and
\code{actual} refers to the actual label, which may have been
altered by options such as \csopt{label-prefix};

\item[\code{entrybib}] the \ext{bib} file the entry was defined in,
where \code{original} refers to the basename (without the
\code{.bib} extension, regardless of whether or not it was included
in \csopt{src}) and \code{actual} refers to the file name
(including the extension and path).

\end{description}

If a syntax error occurs, the error message will show how \bibgls\
has scanned the information so far. For example, in the case of
\code{\csopt[parent \field{name}]{assign-fields}} the message will be:
\begin{quote}\ttfamily
Error: Invalid syntax for option '\csopt{assign-fields}': Expected
one of \idx{follow} \idx{concat-plus} \idx{startoptional} after
' self \idx{follow} parent', found 'n'
\end{quote}
This indicates that it has read \qt{parent} as meaning the
\field{parent} field of the current entry since it isn't followed by
\qt{\idx{follow}}.

\section{Complex Conditionals}
\label{sec:conditionals}

Some options may have a conditional in their value. In certain
cases, such as \csopt{match}, the condition is provided as a
\gls{regular-expression}, but other conditionals (such as in
\csopt{assign-fields}) are complex. This section describes that
complex conditional syntax.

The tokens \idx{ampand} and \idx{pipeor} indicate logical
\qt{AND} and \qt{OR}, respectively, and \idx{exclamnot} indicates
negation. Parentheses \idx{openparenchar} and \idx{closeparenchar}
may be used to control the order of precedence. For example,
\begin{codeenv}
\meta{boolean1} \idx{pipeor} (\meta{boolean2} \idx{ampand} \idx{exclamnot} \meta{boolean3})
\end{codeenv}

Available boolean functions are in the form:
\begin{codeenv}
\meta{value1} \meta{cmp} \meta{value2}
\end{codeenv}
where \meta{value1} is the left-hand value and \meta{value2} is the
right-hand value. The middle \meta{cmp} operator identifies the comparison
function.

The left-hand \meta{value1} may be a field reference \meta{field-ref} or
the integer \idx{quark} \code{\gls{LEN}\margm{element-list}} or
the concatenate \idx{quark} \code{\gls{CAT}\margm{element-list}},
where \meta{field-ref} references a field value and
\meta{element-list} is an element list, using the same
syntax described in \sectionref{sec:optstringconcat}.

The right-hand \meta{value2} may be a field reference
\meta{field-ref} or \code{\gls{CAT}\margm{element-list}} or
\gls{NULL} or a constant string (\qtdelim{\meta{string}} or
\margm{string}) or a number or a \gls{regular-expression}. You can't
use \gls{LEN} on the right-hand as a numeric value (but it may occur
inside the argument \gls{CAT}).  You can't use \gls{NULL} or a
\gls{regular-expression} on the left-hand side.

Where \meta{value1} is \code{\gls{LEN}\margm{element-list}}, the length
evaluates to an integer and may only be used in the numerical
comparisons. If \meta{element-list} is null, then the length will be~0.
The \gls{LEN} \idx{quark} can't be used in the right hand
\meta{value2} part of a numerical comparison. Note that if \gls{LEN}
occurs inside the argument of \gls{CAT} then it becomes a string not
a number.

\nosecdef{CAT}
Where \meta{value1} or \meta{value2} is \code{\gls{CAT}\margm{element-list}}, the
\meta{element-list} will be evaluated and treated as a string, which
will be null if \meta{element-list} evaluates to null.

\nosecdef{NULL}
The null \idx{quark} may only be used as \meta{value2} for the
equality and inequality comparisons. It can't be used in any other
context. Note that the numeric \gls{LEN} doesn't return null.

Where a field value is referenced (\meta{field-ref}), if the field
value is undefined (either the field isn't set or the referenced
ancestor entry hasn't been defined) then, if the designated action
is \qt{fallback} (for example,
\csopt[fallback]{assign-missing-field-action}), the fallback value
is obtained (see \sectionref{sec:fallbacks}). If the value is still
undefined it will be considered a null value for the purposes of the
comparison. Note that if the designated action is \qt{empty} (for
example, \csopt[empty]{assign-missing-field-action}) there will be
no null values.

\begin{codeenv*}
\meta{value1}\idx{equalscmp}\gls{NULL}
\end{codeenv*}
Evaluates to true if \meta{value1} is null.

\begin{codeenv*}
\meta{value1}\idx{notequalscmp}\gls{NULL}
\end{codeenv*}
Evaluates to true if \meta{value1} is not null.

For the remaining comparisons, null values will be treated as an
empty string. Once the \meta{field-ref} or \gls{CAT} references have
been evaluated, their returned value will be turned into a
detokenized string for the purposes of the comparison.

The detokenized values from a field reference may contain any TAB or
newline characters or additional spacing that are present in the
\ext{bib} file (unless they have already been stripped by other
resource options or field assignments).  However, redundant spacing
in any literal strings (\qtdelim{\meta{string}} or \code{\margm{string}})
are likely to be lost when the resource options are written to the
\ext{aux} file.

\begin{codeenv}
\meta{value1}=\idx{slashchar}\meta{regex}\idx{slashchar}
\meta{value1}=\idx{slashchar}\meta{regex}\idx{slashchar}i
\end{codeenv}
Evaluates to true if the value matches the given \gls{anchored} 
\gls{regular-expression} \meta{regex}. If \qtt{i} follows the terminating
\idx{slashchar} then the match is case-insensitive. No other modifiers
are recognised, but you can use embedded flag expressions, such as
\code{?s} for \qt{single-line} mode.

In the following string comparisons, the right-hand \meta{string} is
a constant string that must be delimited with double-quotes or braces. The
comparisons are according to the Unicode code points (not
locale-sensitive), but if the string is followed by \qtt{i}, a
case-insensitive comparison is used.

\begin{codeenv*}
\meta{value1}\idx{equalscmp}\meta{string}
\meta{value1}\idx{equalscmp}\meta{string}i
\end{codeenv*}
Evaluates to true if the value is equal to the string.
For example:
\begin{codeenv}
\field{category}\idx{equalscmp}\qtdelim{abbreviation}
\end{codeenv}

\begin{codeenv*}
\meta{value1}\idx{notequalscmp}\meta{string}
\meta{value1}\idx{notequalscmp}\meta{string}i
\end{codeenv*}
Evaluates to true if the value is not equal to the string.

\begin{codeenv*}
\meta{value1}\idx{ltcmp}\meta{string}
\meta{value1}\idx{ltcmp}\meta{string}i
\end{codeenv*}
Evaluates to true if the value is lexicographically less than the string.

\begin{codeenv*}
\meta{value1}\idx{lecmp}\meta{string}
\meta{value1}\idx{lecmp}\meta{string}i
\end{codeenv*}
Evaluates to true if the value is lexicographically less than or
equal to the string.

\begin{codeenv*}
\meta{value1}\idx{gtcmp}\meta{string}
\meta{value1}\idx{gtcmp}\meta{string}i
\end{codeenv*}
Evaluates to true if the value is lexicographically greater than the string.

\begin{codeenv*}
\meta{value1}\idx{gecmp}\meta{string}
\meta{value1}\idx{gecmp}\meta{string}i
\end{codeenv*}
Evaluates to true if the value is lexicographically greater than or
equal to the string.

In the following numerical comparisons, the given \meta{number}
should use \qtt{.} for the decimal point and no number group
separators. If the \meta{number} doesn't contain a decimal point or
if \meta{value1} is the \code{\gls{LEN}\margm{element-list}} quark
then an integer comparison is assumed.  If \meta{value1} is
empty or isn't numeric it will be treated as 0.  The number
shouldn't be delimited by quotes or braces.

\begin{codeenv*}
\meta{value1}\idx{equalscmp}\meta{number}
\end{codeenv*}
Evaluates to true if the value is equal to \meta{number}.
For example:
\begin{codeenv}
\gls{LEN}\marg{\field{user1}}\idx{equalscmp}0.9
\end{codeenv}
This will return true if the \field{user1} field length is 0 and
false otherwise. This is because \gls{LEN} enforces an integer
comparison which means that 0.9 is converted to 0.
Similarly:
\begin{codeenv}
\gls{CAT}\marg{\qtdelim{0.9}}\idx{equalscmp}0
\end{codeenv}
This will return true because the \meta{number} 0 is an integer
which enforces an integer comparison so the string \qtdelim{0.9}
will be converted to the number~0.
Compare this with:
\begin{codeenv}
\gls{CAT}\marg{\qtdelim{0.9}}\idx{equalscmp}0.0
\end{codeenv}
This will return false because the \meta{number} 0.0 is a decimal,
so a decimal comparison will be used.

\begin{codeenv*}
\meta{value1}\idx{notequalscmp}\meta{number}
\end{codeenv*}
Evaluates to true if the value is not equal to \meta{number}.

\begin{codeenv*}
\meta{value1}\idx{ltcmp}\meta{number}
\end{codeenv*}
Evaluates to true if the value is less than \meta{number}.

\begin{codeenv*}
\meta{value1}\idx{lecmp}\meta{number}
\end{codeenv*}
Evaluates to true if the value is less than or equal to \meta{number}.

\begin{codeenv*}
\meta{value1}\idx{gtcmp}\meta{number}
\end{codeenv*}
Evaluates to true if the value is greater than \meta{number}.

\begin{codeenv*}
\meta{value1}\idx{gecmp}\meta{number}
\end{codeenv*}
Evaluates to true if the value is greater than or equal to \meta{number}.

Finally, the following are string comparisons made after
evaluating and detokenizing both \meta{value1} and \meta{value2}. The
comparisons are case-sensitive and according to the Unicode code
points (not locale-sensitive).

\begin{codeenv*}
\meta{value1}\idx{equalscmp}\meta{value2}
\end{codeenv*}
Evaluates to true if \meta{value1} is equal to \meta{value2}.
For example:
\begin{codeenv}
\field{name} \idx{equalscmp} parent \idx{follow} \field{name}
\end{codeenv}

\begin{codeenv*}
\meta{value1}\idx{notequalscmp}\meta{value2}
\end{codeenv*}
Evaluates to true if \meta{value1} is not equal to \meta{value2}.

\begin{codeenv*}
\meta{value1}\idx{ltcmp}\meta{value2}
\end{codeenv*}
Evaluates to true if \meta{value1} is lexicographically less than
\meta{value2}.

\begin{codeenv*}
\meta{value1}\idx{lecmp}\meta{value2}
\end{codeenv*}
Evaluates to true if \meta{value1} is lexicographically less than or
equal to \meta{value2}.

\begin{codeenv*}
\meta{value1}\idx{gtcmp}\meta{value2}
\end{codeenv*}
Evaluates to true if \meta{value1} is lexicographically greater than
\meta{value2}.

\begin{codeenv*}
\meta{value1}\idx{gecmp}\meta{value2}
\end{codeenv*}
Evaluates to true if \meta{value} is lexicographically greater than or
equal to \meta{value2}.

\begin{codeenv*}
\meta{value1} \inlinedef{IN} \meta{value2}
\end{codeenv*}
Evaluates to true if \meta{value1} is a substring of \meta{value2}.
If \meta{value1} is empty or null it's considered
not a substring regardless of the value of \meta{value2}.

\begin{codeenv*}
\meta{value1} \inlinedef{NIN} \meta{value2}
\end{codeenv*}
The negation of the \gls{IN} test. Evaluates to true if \meta{value1} is
not a substring of \meta{value2}. This is equivalent to:
\begin{codeenv}
\idx{exclamnot} \meta{value1} \gls{IN} \meta{value2}
\end{codeenv}

\begin{codeenv*}
\meta{value1} \inlinedef{PREFIXOF} \meta{value2}
\end{codeenv*}
Evaluates to true if \meta{value1} is a prefix of \meta{value2}
(that is, \meta{value2} starts with \meta{value1}).
If \meta{value1} is empty or null it's considered
not a prefix regardless of \meta{value2}.

\begin{codeenv*}
\meta{value1} \inlinedef{NOTPREFIXOF} \meta{value2}
\end{codeenv*}
Evaluates to true if \meta{value1} is not a prefix of \meta{value2}.
This is equivalent to:
\begin{codeenv}
\idx{exclamnot} \meta{value1} \gls{PREFIXOF} \meta{value2}
\end{codeenv}

\begin{codeenv*}
\meta{value1} \inlinedef{SUFFIXOF} \meta{value2}
\end{codeenv*}
Evaluates to true if \meta{value1} is a suffix of \meta{value2}
(that is, \meta{value2} ends with \meta{value1}).
If \meta{value1} is empty or null it's considered
not a suffix regardless of \meta{value2}.

\begin{codeenv*}
\meta{value1} \inlinedef{NOTSUFFIXOF} \meta{value2}
\end{codeenv*}
Evaluates to true if \meta{value1} is not a suffix of \meta{value2}.
This is equivalent to:
\begin{codeenv}
\idx{exclamnot} \meta{value1} \gls{SUFFIXOF} \meta{value2}
\end{codeenv}


\section{General Options}
\label{sec:generalopts}

\optsection{charset}

If the character \igls{encoding} hasn't been supplied in the \iext{bib} file
with the \gls{encoding} comment
\begin{alltt}
\idx{commentchar} Encoding: \meta{encoding-name}
\end{alltt}
then you can supply the correct \gls{encoding} using
\csopt[encoding-name]{charset}. In general, it's better to include
the \gls{encoding} in the \ext{bib} file where it can also be read by
a \ext{bib} managing systems, such as JabRef.

See \longarg{tex-encoding} for the \gls{encoding} used to write the \ext{glstex}
file, and see \sectionref{sec:defencoding} for information about the
default encoding.

\optsection{locale}

Sets the default \glsdisp{resource-locale}{locale} for the current
\gls{resourceset}. In general, it's best to set this at the start of
the resource option list, if required. If not set, the default 
will be the \gls{document-locale}, if supplied, otherwise the 
\gls{Java-locale} will be used.

\optsection{wordify-math-greek}

Instructs the interpreter to replace known math Greek commands with
words instead of the applicable symbol. For example, if an entry has
been defined as:
\begin{codeenv}
\atentry{index}\marg{a-Fe,
 \field{name}=\marg{\cs{ensuremath}\marg{\ics{alpha}}-iron}
}
\end{codeenv}
Then with \csopt[true]{wordify-math-greek} the interpreter will obtain 
the sort value \qt{alpha-iron}. This only works for commands
recognised by the \texparserlib\ as Math Greek commands.

With the default \csopt[false]{wordify-math-greek}, the interpreter
will convert \cs{alpha} into the nearest appropriate Unicode
character.

The textual replacement depends on whether or not a corresponding
entry is available in the language file \metafilefmt{bib2gls-extra-}{lang}{.xml}
for the current resource locale. If no entry is found, the control
sequence name (or substring) will typically be used.

\optsection{wordify-math-symbol}

Similar to \csopt{wordify-math-greek}, this option will apply to
other known math symbol commands. Again, this only works for a
limited set of commands recognised by the \texparserlib. An
alternative is to provide alternative definitions for \bibgls\ that
aren't picked up by \LaTeX, or use \ics{IfNotBibGls}, or use a more
appropriate field for sorting.

\optsection{interpret-preamble}

This is a boolean option that determines whether or not the
interpreter should parse the contents of \atentry{preamble}.
The default is \optfmt{true}. If \optfmt{false}, the preamble
contents will still be written to the \iext{glstex} file, but any
commands provided in the preamble won't be recognised by the
interpreter (see \sectionref{sec:texparserlib}).

Related options are: \csopt{set-widest} (which uses the interpreter to determine
the widest name for the \glostyle{alttree} style or the
\isty{glossary-longextra} styles), \csopt{interpret-label-fields} (which governs
whether or not fields that must only contain a label should be interpreted),
\csopt{labelify} (which converts a field into a string suitable for use as a
label), and \csopt{labelify-list} (which converts a field into a string suitable
for use as a comma-separated list of labels).

\optsection{write-preamble}

This is a boolean option that determines whether or not the
preamble should be written to the \ext{glstex} file. The default
is \optfmt{true}. Note that the preamble will
still be parsed if \csopt[true]{interpret-preamble} even if
\csopt[false]{write-preamble}. This means it's possible to 
provide \bibgls\ command definitions in \atentry{preamble}
that don't get seen by \LaTeX.

\optsection{set-widest}

The \glostyle{alttree} glossary style needs to know the widest
\field{name} (for each level, if \hierarchical). This can be set
using \ics{glssetwidest} provided by the \isty{glossary-tree}
package (or similar commands like \ics{glsupdatewidest} provided by
\isty{glossaries-extra-stylemods}), but this requires knowing
which name is the widest. Alternatively, one of the iterative commands
such as \ics{glsFindWidestTopLevelName} can be used, which slows the
document build as it has to iterate over all defined entries.

The \isty{glossary-longextra} package, provided with
\sty{glossaries-extra} v1.37+, also needs to know the widest name,
but in this case only the \gls{top-levelentry}{top-level} is needed.
If this has already been found through the commands provided with
the \glostyle{alttree} style then that value will be used as the
default, but you can set another value that's only used for the
\sty{glossary-longextra} styles with \ics{glslongextraSetWidest}.

The \sty{glossaries-extra-bib2gls} package provides \ics{glsxtrSetWidest},
which sets the widest name for those styles that need it. As from version
1.8, \bibgls\ now checks for the existence of this command and will use it with
\csopt{set-widest} to allow for the new styles provided by the
\sty{glossary-longextra} package.

The boolean option \csopt[true]{set-widest} will try to calculate
the widest names for each \hierarchicallevel\ to help remove the need
to determine the correct value within the document. 
Since \bibgls\ doesn't know the fonts that will be used
in the document or if there are any non-standard commands that
aren't provided in the \ext{bib} files preamble, \emph{this option may not
work}. For example, if one entry has the \field{name} defined as:
\begin{codeenv}
\field{name}=\marg{some \marg{\cmd{Huge} huge} text}
\end{codeenv}
and another entry has the \field{name} defined as:
\begin{codeenv}
\field{name}=\marg{some \marg{\cmd{small} small} text}
\end{codeenv}
then \bibgls\ will determine that the second name is the widest
although the first will actually be wider when it's rendered in the
document. 

When using this option, the transcript file will include the message:
\begin{alltt}
Calculated width of '\meta{text}': \meta{number}
\end{alltt}
where \meta{text} is \bibgls's interpretation of the contents of the
\field{name} field and \meta{number} is a rough guide to the width
of \meta{text} assuming the operating system's default serif font.
The entry that has the largest \meta{number} is the one that will be
selected. This will then be implemented as follows:
\begin{itemize}
\item If the \field{type} is unknown then:
  \begin{itemize}
   \item if the interpreter resolves all \field{name} fields to the
    empty string (that is the \field{name} fields all consist of
    unknown commands) then
    \begin{itemize}
     \item if there are \glspl{childentry} \gls{bibglssetwidestfallback}
     is used,
     \item otherwise \gls{bibglssetwidesttoplevelfallback} is used;
    \end{itemize}
   \item otherwise \gls{bibglssetwidest} is used.
  \end{itemize}
\item If the \field{type} is known then:
  \begin{itemize}
   \item if the interpreter resolves all \field{name} fields for that
    type to the empty string (that is the \field{name} fields all consist of
     unknown commands) then
    \begin{itemize}
     \item if there are \glspl{childentry} \gls{bibglssetwidestfortypefallback}
     is used,
     \item otherwise 
      \gls{bibglssetwidesttoplevelfortypefallback} is used;
    \end{itemize}
   \item otherwise \gls{bibglssetwidestfortype} is used.
  \end{itemize}
\end{itemize}
This leaves \TeX\ to compute the width according to the document
fonts. If \bibgls\ can't correctly determine the widest entry then
you will need to use one of the commands provided by
\sty{glossary-tree}, \sty{glossary-longextra} or 
\sty{glossaries-extra-stylemods} to set it.

In general, if you have more than one glossary it's best to
set the \field{type} using options like \csopt{type} and
\csopt{dual-type} if you use \csopt{set-widest}.

\optsection{entry-type-aliases}

In the \ext{bib} file, the data is identified by 
\code{@\meta{entry-type}}, such as \atentry[noindex]{abbreviation}.
It may be that you want to replace all instances of
\code{@\meta{entry-type}} with a different 
type of entry. For example, suppose my \ext{bib} file
contains abbreviations defined in the form:
\begin{codeenv}
\atentry{abbreviation}\marg{html,
  \field{short} = \marg{html},
  \field{long}  = \marg{hypertext markup language},
  \field{description} = \marg{a markup language for creating web pages}
}
\end{codeenv}
but suppose in one of my documents I actually want all these
abbreviations defined with \atentry[noindex]{dualabbreviationentry}
instead of \atentry[noindex]{abbreviation}. Instead of editing
the \ext{bib} file I can just supply a mapping:
\begin{codeenv}
\gls{GlsXtrLoadResources}\oarg{
  \csopt[entries]{src},\comment{data in entries.bib}
  \csopt[abbreviation=dualabbreviationentry]{entry-type-aliases}
}
\end{codeenv}
This makes all instances of \atentry[noindex]{abbreviation}
behave as \atentry[noindex]{dualabbreviationentry}. You can have more than
one mapping. For example:
\begin{codeenv}
\gls{GlsXtrLoadResources}\oarg{
  \csopt[entries]{src},\comment{data in entries.bib}
  \csopt[
    \comment{\atentry{abbreviation} -> \atentry{dualabbreviationentry}:}
    abbreviation=dualabbreviationentry,
    \comment{\atentry{entry} -> \atentry{index}:}
    entry=index
  ]{entry-type-aliases}
}
\end{codeenv}
This option isn't cumulative.  Multiple instances of
\csopt{entry-type-aliases} override previous instances.
If \meta{\keyvallist} is empty there will be no mappings.
You can save the original \gls{entrytype} in the
\field{originalentrytype} field with
\csopt{save-original-entrytype}.

Here's another example entry in a \ext{bib} file:
\begin{codeenv}
@foo\marg{html,
   \field{name} = \marg{HTML},
   \field{short} = \marg{HTML},
   \field{long} = \marg{hypertext markup language},
   \field{description} = \marg{hypertext markup language}
}
\end{codeenv}
Ordinarily this entry would be ignored since \code{@foo}
isn't recognised, but it can be mapped like this:
\begin{codeenv}
\gls{GlsXtrLoadResources}\oarg{
  \csopt[entries]{src},\comment{data in entries.bib}
  \csopt[short,long]{ignore-fields},
  \csopt[foo=entry]{entry-type-aliases}
}
\end{codeenv}
This treats the entry as though it had been defined as:
\begin{codeenv}
\atentry{entry}\marg{html,
   \field{name} = \marg{HTML},
   \field{description} = \marg{hypertext markup language}
}
\end{codeenv}
whereas:
\begin{codeenv}
\gls{GlsXtrLoadResources}\oarg{
  \csopt[entries]{src},\comment{data in entries.bib}
  \csopt[name,description]{ignore-fields},
  \csopt[foo=abbreviation]{entry-type-aliases}
}
\end{codeenv}
treats the entry as though it had been defined as:
\begin{codeenv}
\atentry{abbreviation}\marg{html,
   \field{short} = \marg{HTML},
   \field{long} = \marg{hypertext markup language}
}
\end{codeenv}

\optsection{unknown-entry-alias}

If this option is set, the \meta{value} is used as the alias for any
unknown \glspl{entrytype} (after any aliases provided with
\csopt{entry-type-aliases} have been applied). If the value is
missing or empty, unknown \glspl{entrytype} will be ignored with a
warning.

\optsection{action}

This governs how the entries are written in the \iext{glstex} file.
The \meta{value} may be one of: 
\begin{itemize}
\item\code{define}: define the entries;
\item\code{provide}: provide the entries;
\item\code{copy}: copy the entries;
\item\code{define or copy}: copy existing entries and define
non-existing entries.
\end{itemize}
The default setting is \csopt[define]{action}, which writes the entry
definition to the \ext{glstex} file using one of the commands
described in \sectionref{sec:newentrydefs}. Since the \styopt{record}
package option automatically switches on the \styopt[warn]{undefaction} 
option, any attempt at defining an entry that's already been defined
will generate a warning rather than an error. The duplicate
definition will be ignored. (The warnings can be found in the
\ext{log} file since they are warnings produce by
\sty{glossaries-extra} not by \bibgls.)

If you have multiple \iglspl{resourceset} to help group different types of
entry for the same glossary, the \csopt[provide]{action} setting can
be used to suppress any warnings if the selection criteria is too
complex to filter out entries that were selected by a previous
\gls{resourceset}. If, however, you want duplicate entries so that
you can have the same entry listed in multiple glossary, you need a
different approach.

For example, if you try:
\begin{codeenv}
\cs{newglossary*}\marg{copies}\marg{Copies}
\gls{GlsXtrLoadResources}\oarg{\csopt[entries]{src}}
\gls{GlsXtrLoadResources}\oarg{\csopt[use]{sort},\csopt[copies]{type},\csopt[entries]{src}}
\end{codeenv}
you'll find that the \code{copies} glossary is empty and there will
be warnings in the \ext{log} file when the second resource file is
loaded.

There are various ways of having the same entries in multiple
glossaries. The simplest method is to use \csopt{secondary}, but
another method is to use \csopt[copy]{action} which simply writes
\nosecformatdef{glsxtrcopytoglossary}
instead of using one of the commands listed in
\sectionref{sec:newentrydefs}. This copies the entries rather than
defining them, which means the entries must already have been
defined. You can select entries that were selected in earlier
\iglspl{resourceset} with \csopt[selected before]{selection}.

The \meta{type} is determined as follows:
\begin{itemize}
\item if the entry has the \field{type} field set, that's used;
\item if the entry is a tertiary and \csopt{tertiary-type} is set, that's
used;
\item if the entry is a dual and \csopt{dual-type} is set, that's
used;
\item otherwise the value of the \csopt{type} option is used.
\end{itemize}
If you're not sure whether the entries may already be defined, you
can use \csopt[define or copy]{action} which will use
\ics{ifglsentryexists} in the resource file to determine whether to
define or copy the entry.

Options that set or modify fields, such as \csopt{category},
\csopt{group}, \csopt{save-locations}, \csopt{flatten} or
\csopt{name-case-change}, will be ignored if entries are copied.
However the \csopt{copy-action-group-field} may be used to copy
the \field{group} field (which may have been locally set by the 
\csopt{sort} method) to another field. This ensures that the
original \field{group} value from the entry definition in an earlier
resource set won't be overwritten (unless you set
\csopt[group]{copy-action-group-field}).

Remember that \gls{glsxtrcopytoglossary} simply copies the entry's
label to the glossary's internal list. The only checks that \bibgls\
performs if \csopt{action} is not \code{define} is to ensure that
the \csopt{master} or \csopt{secondary} options have not been used,
since they're incompatible, and that the \csopt{type} option is set,
since it's required as a fallback for any entries that don't have
the \field{type} field set. (There are too many options that alter
field values to check them all and some may be used to alter the
sorting.) The purpose of the copy action is simply to provide a
duplicate list in a different order.

Remember that if you are using \sty{hyperref}, you need to use
\code{\printglossopt[false]{target}} in the optional argument of
\ics{printunsrtglossary} for the glossary containing the copies to
prevent duplicate hypertargets. Commands like \cs{gls} will link to
the original entries. For example, in the preamble:
\begin{codeenv}
\cs{newignoredglossary}\marg{copies}
\strut
\gls{GlsXtrLoadResources}\oarg{\csopt[entries]{src}}
\strut
\gls{GlsXtrLoadResources}\oarg{
 \csopt[use]{sort},
 \csopt[copy]{action},
 \csopt[copies]{type},
 \csopt[entries]{src}
}
\end{codeenv}
and later in the document:
\begin{codeenv}
\cs{printunsrtglossary}\oarg{\printglossopt[Glossary (Alphabetical)]{title},\printglossopt[\glostyle{indexgroup}]{style}}
\cs{printunsrtglossary}\oarg{\printglossopt[copies]{type},\printglossopt[Glossary (Order of Use)]{title},
 \printglossopt[index]{style},\printglossopt{nogroupskip},\comment{no grouping}
 \printglossopt[false]{target}}
\end{codeenv}
Note also the need to use \code{nogroupskip} and a non-group style
for the duplicates since the \field{group} field will have been
assigned in the first resource set if \bibgls\ was invoked with 
\longarg{group}. The grouping is appropriate for alphabetical
ordering but not for order of use.

If you want different grouping for the duplicates, you can specify
the field name to use in which to store the group information using
\csopt{copy-action-group-field}. Unlike \csopt{secondary}, you
will need to redefine \ics{glsxtrgroupfield} to the relevant field
before you display the glossary. The simplest way to do this is with
the starred form of \cs{printunsrtglossary}.
For example, if \csopt[dupgroup]{copy-action-group-field} is added to
the options for the second resource set:
\begin{codeenv}
\ics{printunsrtglossary*}\oarg{\printglossopt[copies]{type},\printglossopt[Duplicates]{title},\printglossopt[\glostyle{indexgroup}]{style}}
 \marg{\cs{renewcommand}\marg{\cs{glsxtrgroupfield}}\marg{dupgroup}}
\end{codeenv}
This just does:
\begin{codeenv}
\cmd{begingroup}
 \cs{renewcommand}\marg{\ics{glsxtrgroupfield}}\marg{dupgroup}\comment{}
 \cs{printunsrtglossary}\oarg{\printglossopt[copies]{type},\printglossopt[Duplicates]{title},
  \printglossopt[indexgroup]{style}}
\cmd{endgroup}
\end{codeenv}

\optsection{copy-to-glossary}

This option can selectively copy an entry to a glossary after it has
been defined. If the supplied value \meta{list} is empty, no copying is
performed (except as a result of other options, such as
\csopt{action} or \csopt{secondary}). If set, the \meta{list}
argument is a list of string \glspl{concatenation} with optional
conditionals. Take care that constant strings are correctly
delimited, as described below, to ensure that they are not mistaken
for field labels.

The evaluation of the target glossary label for each entry is
performed while the \ext{glstex} file is being written (after
sorting) so all field values should be available in any field reference.
The \csopt{action} option is implemented first, so the
selected entry will first either be defined or copied according to
\csopt{action}. If the \csopt{copy-to-glossary} instruction is
successful, the entry will then be copied to the target glossary using
\gls{bibglscopytoglossary}. 

The \csopt{copy-to-glossary} value should be a comma-separated list,
where the syntax for each item in the list is in the form:
\begin{codeenv}
\meta{element-list} \oargm{condition}
\end{codeenv}
where \meta{element-list} is a string \gls{concatenation} (see
\sectionref{sec:optstringconcat}) and \meta{condition} is a complex
conditional (see \sectionref{sec:conditionals}).  For each 
\meta{element-list} \oargm{condition} specification, if the
condition evaluates to false or if the \meta{element-list} evaluates
to null then the copy instruction won't be added.

For example, the following first sets the \field{type} to
\qt{ignored} for any entries that only have \glspl{ignoredrecord}
and then copies all entries that don't have the \field{type} field
set to \qt{ignored} to the glossary labelled \qt{index}:
\begin{codeenv}
\gls{GlsXtrLoadResources}\oarg{
 \csopt[ignored]{ignored-type},
 \csopt[\qtdelim{index} \oarg{ \field{type} \idx{notequalscmp} \qtdelim{ignored}} ]{copy-to-glossary}
}
\end{codeenv}

The fallback action for a missing field value is governed by the
\csopt{copy-to-glossary-missing-field-action} setting. The result of
the string concatenation (if not null) is the label of the target
glossary.

You can have multiple copy instructions to copy an entry to multiple
glossaries. The definition of \gls{bibglscopytoglossary} will ensure
that an entry will only be copied to the designated glossary if it
isn't already in the glossary's internal list and will silently do
nothing if the glossary doesn't exist.

Remember that constant strings need to be marked with braces or
double-quote delimiters. For example, if you want to copy \emph{all}
entries to the \code{index} glossary then either do:
\begin{codeenv}
\csopt[\qtdelim{index}]{copy-to-glossary}
\end{codeenv}
or
\begin{codeenv}
\csopt[\marg{index}]{copy-to-glossary}
\end{codeenv}
Note that the outer braces are stripped by the resource option
parser, which first splits the \code{\meta{option}\dequals\margm{value}} list supplied via
\gls{GlsXtrLoadResources} into \meta{option} and \meta{value} pairs, and then
parses each \meta{option}. So by the time that the
\csopt{copy-to-glossary} option has its value parsed, the value has
become \code{\qtdelim{index}} or \code{\marg{index}}, respectively,
in the above two examples.

Remember that the \meta{value} itself may be a comma-separated list.
The outer grouping hides the inner list comma from the initial
\code{\meta{option}\dequals\margm{value}} split.  For example, to copy all entries
to the \code{index} and \code{symbols} glossaries:
\begin{codeenv}
\csopt[\qtdelim{index}, \qtdelim{symbols}]{copy-to-glossary}
\end{codeenv}
or
\begin{codeenv}
\csopt[\marg{index}, \marg{symbols}]{copy-to-glossary}
\end{codeenv}

The following example will only copy entries to the \code{index}
glossary if their actual entry type is \code{index}:
\begin{codeenv}
\csopt[\qtdelim{index} \oarg{ entrytype \idx{follow} actual \idx{equalscmp} \qtdelim{index} }]{copy-to-glossary}
\end{codeenv}
Alternatively, to copy aliased custom entry types
\atentryfmt{person} entries to a custom glossary \code{person} and
\atentryfmt{place} to a custom glossary \code{place}:
\begin{codeenv}
\csopt[
 entrytype \idx{follow} original
  \oarg{ entrytype \idx{follow} original =\idx{slashchar}person\idx{regexpor}place\idx{slashchar} }
]{copy-to-glossary}
\end{codeenv}

If the glossary types don't conveniently match the entry type,
the instructions can be split into a list.
For example:
\begin{codeenv}
\csopt[
  \qtdelim{abbreviations} \oarg{ entrytype \idx{follow} actual = \qtdelim{abbreviation} },
  \qtdelim{symbols} \oarg{ entrytype \idx{follow} actual = \qtdelim{symbol} },
  \qtdelim{numbers} \oarg{ entrytype \idx{follow} actual = \qtdelim{number} },
]{copy-to-glossary}
\end{codeenv}
Each instruction in the list will be tried and the copy instruction
will only be written if the condition evaluates to true and a
non-null value is successfully returned.

\optsection{copy-to-glossary-missing-field-action}

This option indicates what to do if a source field identified in 
\csopt{copy-to-glossary} is missing. The value may be one of:
\begin{itemize}
\item \optfmt{skip}: return null;
\item \optfmt{fallback}: use the fallback for the missing field 
(see \sectionref{sec:fallbacks}), if
one is available, otherwise return null (default);
\item \optfmt{empty}: treat the missing value as empty.
\end{itemize}
Returning null will result in the copy instruction being omitted.

\section{Selection Options}
\label{sec:selectionopts}

\optsection{src}

This identifies the \iext{bib} files containing the entry
definitions.  The value should be a comma-separated list of the
required \ext{bib} files. These may either be in the current working
directory or in the directory given by the \longarg{dir} switch or on
\TeX's path (in which case \idx{kpsewhich} will be used to find
them).  The \ext{bib} extension may be omitted.  Remember that if
\meta{list} contains multiple files it must be grouped to protect
the comma from the \meta{options} list.

For example:
\begin{codeenv}
\gls{GlsXtrLoadResources}\oarg{\csopt[entries-terms,entries-symbols]{src}}
\end{codeenv}
indicates that \bibgls\ must read the files
\filefmt{entries-terms.bib} and \filefmt{entries-symbols.bib} and
create the file given by \ics{jobname}\iext{glstex} on the first
instance or \ics{jobname}\code{-}\meta{n}\ext{glstex} on subsequent
use.

With \code{\gls!{glsxtrresourcefile}\oargm{options}\margm{filename}}, if 
the \csopt{src} option is omitted, 
the \ext{bib} file is assumed to be \meta{filename}\ext{bib}. For example:
\begin{codeenv}
\gls{glsxtrresourcefile}\marg{entries-symbols}
\end{codeenv}
indicates that \bibgls\ needs to read the file
\filefmt{entries-symbols.bib}, which contains the entry data, and create the file
\filefmt{entries-symbols.glstex}. If the \ext{bib} file is
different or if you have multiple \ext{bib} files, you need to use
the \csopt{src} option.

\gls{GlsXtrLoadResources} uses \ics{jobname} as the argument of 
\gls{glsxtrresourcefile} on the first instance, so:
\begin{codeenv}
\gls{GlsXtrLoadResources}\oarg{}
\end{codeenv}
will assume \csopt[\ics{jobname}]{src}. Remember that subsequent uses
of \gls{GlsXtrLoadResources} append a suffix, so in general it's
best to always supply \csopt{src}, except for small test cases with
a single \gls{resourcecommand}.

With old \LaTeX\ kernels, if you have non-ASCII characters in the
\ext{bib} filename but aren't using \XeLaTeX\ or \LuaLaTeX, then you
will need to use \ics{detokenize} to prevent expansion when the
information is written to the \ext{aux} file. Newer \LaTeX\ kernels
have better support for UTF-8. Similarly for any special characters
that need protecting (although it's better not to use special
characters in filenames). For example:
\begin{codeenv}
\cmd{documentclass}\marg{article}
\strut
\cmd{usepackage}\oarg{T2A}\marg{fontenc}
\cmd{usepackage}\oarg{utf8}\marg{inputenc}
\cmd{usepackage}\oarg{russian}\marg{babel}
\cmd{usepackage}\oarg{record}\marg{glossaries-extra}
\strut
\gls{GlsXtrLoadResources}\oarg{
  \csopt[\cs{detokenize}\marg{\textcyrillicmono{кириллица}}]{src},\comment{data in \textcyrillicmono{кириллица.bib}}
  \csopt[all]{selection}
}
\strut
\cmd{begin}\marg{document}
\cs{printunsrtglossary}
\cmd{end}\marg{document}
\end{codeenv}

\optsection{selection}

By default all entries that have records in the \iext{aux} file will
be selected as well as all their dependent entries. The dependent
entries that don't have corresponding records on the first \LaTeX\
run, may need an additional build to ensure their \glspl{locationlist}
are updated.

Remember that on the first \LaTeX\ run the \iext{glstex} files don't
exist. This means that the entries aren't defined at that point. The
\styopt{record} package option additionally switches on the
\styopt[warn]{undefaction} option, which means that you'll only get
warnings rather than errors when you reference entries in the
document. You can't use \ics{glsaddall} with
\bibgls\ because the glossary lists are empty on the first run, so
there's nothing for \ics{glsaddall} to iterate over.
Instead, if you want to add all defined entries, you need to
instruct \bibgls\ to do this with the \csopt{selection} option. The
following values are allowed:
\begin{itemize}
\item \optfmt{recorded and deps}: add all recorded entries and
their dependencies (default).

\item \optfmt{recorded and deps and see}: as above but will also
add unrecorded entries whose \field{see}, \field{seealso} or
\field{alias} field refers to a recorded entry.

\item \optfmt{recorded and deps and see not also}: as above but will 
add unrecorded entries whose \field{see} or \field{alias} (but not
\field{seealso}) field refers to a recorded entry.

\item \optfmt{recorded no deps}: add all recorded entries but not
their dependencies. The dependencies include those referenced in the
\field{see} or \field{seealso} field or fields identified by
\csopt{dependency-fields}, \field{parent} entries and those found
referenced with commands like \ics{gls} in the field values that are
parsed by \bibgls. With this setting, \glsdisp{parententry}{parents}
will be omitted unless they've been referenced in the document
through commands like \ics{gls}.  This setting won't add any
\field{see} or \field{seealso} lists to the \igls{locationlist}. The
given field will be set, so you can access the information, but
there's no guarantee that the cross-referenced entry will have been
selected. The \field{alias} cross-reference will be added to the
\igls{locationlist} but you will need to ensure that the target is
also selected (or use \csopt[omit]{alias} to suppress it).

\item \optfmt{recorded and ancestors}: this is like the previous
setting but \glsdisp{parententry}{parents} are added even if they
haven't been referenced in the document. The other dependent entries
are omitted if they haven't been referenced in the document. The
above notes regarding the cross-reference lists also applies.

\item \optfmt{deps but not recorded}: this first selects entries as
though \optfmt{recorded and deps} had been used, but after all
\glspl{ancestor} and dependencies have been added it then removes all
entries that have records. This means that you end up with only the
unrecorded dependencies. (Recorded entries will need to be selected
in a different resource set.)

\item \optfmt{ancestors but not recorded}: this first selects
entries as though \optfmt{recorded and ancestors} had been used, but
after all \glspl{ancestor} have been added it then removes all entries that
have records. This means that you end up with only the unrecorded
\glspl{ancestor}. (Recorded entries will need to be selected
in a different resource set.) See the \exfile{sample-nested.tex}
example document.

\item \optfmt{selected before}: select any entries that have been
selected in a previous \igls{resourceset}. This is intended
for use with \csopt[copy]{action} to copy entries to another
glossary as an alternative to (or in addition to) the
\csopt{secondary} option. Note that if you make any modifications to
the fields (such as case-changing) the modification won't be saved
to the \ext{glstex} file. This option can't be used in the first
\igls{resourceset}.

\item \optfmt{all}: add all entries found in the \ext{bib} files
supplied in the \csopt{src} option.
\end{itemize}
The \meta{value} must be supplied.

For example, suppose the file \filefmt{entries.bib} contains:
\begin{codeenv}
\atentry{index}\marg{run}
\atentry{index}\marg{sprint,\field{see}=\marg{run}}
\atentry{index}\marg{dash,\field{see}=\marg{sprint}}
\end{codeenv}
If the document only references the \qt{run} entry (for example,
using \code{\cs{gls}\marg{run}}) then:
\begin{itemize}
\item If \csopt[recorded and deps]{selection}, only the \qt{run}
entry is selected. The \qt{run} entry has a record, so it's
selected, but it has no dependencies. Neither \qt{sprint} nor 
\qt{dash} have records, so they're not selected.
\item If \csopt[recorded and deps and see]{selection}, the \qt{run}
and \qt{sprint} entries are selected, but not the \qt{dash} entry.
The \qt{run} entry is selected because it has a record. The
\qt{sprint} entry doesn't have a record but its \field{see} field
includes \qt{run}, which does have a record, so \qt{sprint} is also
selected. The \qt{dash} entry doesn't have a record. Its
\field{see} field references \qt{sprint}. Although \qt{sprint} has
been selected, it doesn't have any records, so \qt{dash} isn't
selected.
\end{itemize}
The above is just an example. The circuitous redirection of
\qt{dash} to \qt{sprint} to \qt{run} is unhelpful to the reader and
is best avoided (especially for an index where there are no accompanying 
descriptions and no \igls{locationlist} for the intermediate \qt{sprint}). 
A better method would be:
\begin{codeenv}
\atentry{index}\marg{run}
\atentry{index}\marg{sprint,\field{see}=\marg{run}}
\atentry{index}\marg{dash,\field{see}=\marg{run}}
\end{codeenv}
The \csopt[recorded and deps and see]{selection} in this case will
select all three entries, and the document won't send the reader on a
long-winded detour.

Now suppose that the file \filefmt{entries.bib} contains:
\begin{codeenv}
\atentry{entry}\marg{run,
 \field{name} = \marg{run},
 \field{description}=\marg{move fast using legs}
}
\strut
\atentry{entry}\marg{sprint,
 \field{name} = \marg{sprint},
 \field{description}=\marg{run at full speed over short distance},
 \field{seealso}=\marg{run}
}
\strut
\atentry{entry}\marg{dash,
 \field{name} = \marg{dash},
 \field{description}=\marg{run in a great hurry},
 \field{seealso}=\marg{sprint}
}
\end{codeenv}
and suppose the document only references \qt{dash} (for example, with
\code{\cs{gls}\marg{dash}}), then with the default \csopt[recorded and deps]{selection}
\qt{dash} will be selected because it has a record, and \qt{sprint}
will be selected because \qt{dash} requires it (for the cross-reference),
and \qt{run} will be selected because \qt{sprint} requires it
(for the cross-reference). In this case, neither \qt{sprint} nor \qt{run}
have a \igls{locationlist} but they do both provide additional information
for the reader in their descriptions.

A better method here would be for each entry to have a cross-reference
list that includes all related terms:
\begin{codeenv}
\atentry{entry}\marg{run,
 \field{name} = \marg{run},
 \field{description}=\marg{move fast using legs},
 \field{seealso}=\marg{sprint,dash}
}
\strut
\atentry{entry}\marg{sprint,
 \field{name} = \marg{sprint},
 \field{description}=\marg{run at full speed over short distance},
 \field{seealso}=\marg{run,dash}
}
\strut
\atentry{entry}\marg{dash,
 \field{name} = \marg{dash},
 \field{description}=\marg{run in a great hurry},
 \field{seealso}=\marg{sprint,run}
}
\end{codeenv}
Now, whichever one is indexed in the document, the other two will automatically
be selected.

\optsection{match}

It's possible to filter the selection by matching field values.
The value is required for this key but may be empty, which indicates 
that the setting is switched off, otherwise
\meta{\keyvallist} should be a \meta{key}=\meta{regexp} list, where
\meta{key} is the name of a field or \optfmt{id} for the entry's
label or \optfmt{entrytype} for the \bibgls\ \gls{entrytype} (as in
the part after \verb|@| identifying the entry not the \field{type}
field identifying the glossary label). If you've used
\csopt{entry-type-aliases}, this refers to the target \gls{entrytype} not
the original \gls{entrytype} specified in the \ext{bib} file.

The \meta{regexp} part should be a \gls{regular-expression} conforming
to
\href{http://docs.oracle.com/javase/8/docs/api/java/util/regex/Pattern.html}{Java's
Pattern class}~\cite{pattern}. The pattern is \gls{anchored} 
(\code{oo}\idx{matchanydot}\idx{zeroormore} matches
\code{oops} but not \code{loops}) and \meta{regexp} can't be
empty. Remember that \TeX\ will expand the option list as
it writes the information to the \ext{aux} file so take
care with special characters. For example, to match a literal
period use \cs{cs.string}\idx{cs.period} not \idx{cs.period} (backslash dot).

If the field is missing its value it is assumed to be empty for
the purposes of the pattern match even if it will be assigned a
non-empty default value when the entry is defined. If the field is
unrecognised by \bibgls\ any reference to it in \meta{\keyvallist} will
be ignored.

If a field is listed multiple times, the pattern for that
field is concatenated using:
\begin{alltt}
(?:\meta{pattern-1})|(?:\meta{pattern-2})
\end{alltt}
where \meta{pattern-1} is the current pattern for that field
and \meta{pattern-2} is the new pattern. This means it performs a
logical OR\@. For the non-duplicate
fields the logical operator is given by \csopt{match-op}.
For example:
\begin{codeenv}
\csopt[and]{match-op},
\csopt[
  \field{category}=animals,
  \fieldfmt{topic}=biology,
  \field{category}=vegetables
]{match}
\end{codeenv}
This will keep all the selected entries that satisfy:
\begin{itemize}
\item \field{category} matches \verb"(?:animals)|(?:vegetables)"

(the \field{category} is either \optfmt{animals} or
\optfmt{vegetables})

\item[AND]

\item \fieldfmt{topic} (custom key provided by user) is \code{biology}.
\end{itemize}
and will discard any entries that don't satisfy this condition.
A message will be written to the log file for each entry
that's discarded.

Patterns for unknown fields will be ignored. If the entire
list consists of patterns for unknown fields it will be
treated as \csopt[\empty]{match}. That is, no filtering will be
applied. In the above example, the custom \fieldfmt{topic} key must
be provided before the first \gls{GlsXtrLoadResources} with
\ics{glsaddkey} or \ics{glsaddstoragekey}.

\optsection{match-op}

If the value of \csopt{match} contains more than one 
\meta{key}=\meta{pattern} element, the \csopt{match-op}
determines whether to apply a logical AND or a logical OR.
The \meta{value} may be either \optfmt{and} or \optfmt{or}.
The default is \csopt[and]{match-op}.

\optsection{not-match}

If \csopt[\meta{\keyvallist}]{match} would cause an entry to be
selected then \csopt[\meta{\keyvallist}]{not-match} would cause that
entry to be ignored.  The value is required for this key but may be
empty, which indicates that the setting is switched off.  If you
have both \csopt{match} and \csopt{not-match} in the same resource
set, the last one listed takes precedence.

\optsection{match-action}

The default behaviour with \csopt{match} or \csopt{not-match}
is to filter the selection. This may be changed to append to the
selection instead. The \meta{value} may be one of:
\begin{itemize}
\item\optfmt{filter}: (default) filter selection;
\item\optfmt{add}: append any matches (with \csopt{match}) or
non-matches (with \csopt{not-match}) to the selection.
This setting can't be used with \csopt[use]{sort}.
\end{itemize}
For example, if I want to select all record entries and their
dependencies, but I also want to make sure that any entries with
the category set to \code{important} are always selected regardless
of whether or not they have any records:
\begin{codeenv}
\gls{GlsXtrLoadResources}\oarg{
  \csopt[entries]{src},\comment{data in entries.bib}
  \csopt[add]{match-action},
  \csopt[\field{category}=important]{match}
}
\end{codeenv}

\optsection{limit}

If \meta{number} is greater than 0 then this will truncate the
list of selected entries after sorting to \meta{number} (if the list
size is greater than that value). The transcript will show the
message:
\begin{alltt}
Truncating according to limit=\meta{number}
\end{alltt}
When used with \csopt{shuffle}, this provides a means of randomly
selecting at most \meta{number} entries. The default setting is
\csopt[0]{limit} (no truncation). A negative value of \meta{number} is not
permitted.

If you have any \glspl{dualentry}, then the truncation will be
applied to the combined list of \primary\ and
\glsdisp{dualentry}{duals} if \csopt[combine]{dual-sort} otherwise
each list will be truncated separately by \meta{number}, which
results in a maximum of $2 \times \meta{number}$. Remember that
\glspl{tertiaryentry} are created when \glspl{dualentry} are defined
in the \ext{glstex} file, so this will increase the total number of
entries.

\section{Hierarchical Options}
\label{sec:hierarchicalopts}

Hierarchy is established by setting the \field{parent} field to the
label of the parent entry. The parent and child entries are 
sorted together, but hierarchical comparators will place child
entries after their corresponding parent.

The \sty{glossaries} package provides \ics{ifglshasparent} to
determine whether or not an entry has the \field{parent} field set.
If also provides \ics{ifglshaschildren}, but this command is
inefficient as it has to iterate over all entries to find an entry
with the \field{parent} field set to the relevant label. It's also
non-trivial to determine which child entries have been included in
the glossary with \idx!{makeindex} or \idx!{xindy}. \bibgls\ can
provide this information with some of the options described in this
section.

It's also possible to \glsdisp{flatglossary}{flatten entries} (that is, remove the
hierarchical information) or just flatten \glspl{lonelychildentry}.

\optsection{save-child-count}

This is a boolean option. The default setting is
\csopt[false]{save-child-count}. 
If \csopt[true]{save-child-count}, each entry will be assigned a
field called \field{childcount} with the value equal to the number
of \glspl{childentry} that have been selected. As from version 1.5,
this option also creates the \field{childlist} field for entries
that have \glsdisp{childentry}{children} selected. This field is in
\sty{etoolbox}'s internal list format and can be iterated over using
\glsadd{idx.loop}\ics{glsxtrfieldforlistloop}.

The assignment is done using \ics{GlsXtrSetField} so there's
no associated key. You can test if the field is set and non-zero
using:
\nosecformatdef{GlsXtrIfHasNonZeroChildCount}
which is provided with \sty{glossaries-extra-bib2gls} v1.31+.
Within \meta{true}, you can access the actual value with
\ics{glscurrentfieldvalue}. If \csopt[false]{save-child-count},
this command will do \meta{false} as the \field{childcount} field
won't be set.

For example, suppose \filefmt{entries.bib} contains:
\begin{codeenv}
\atentry{index}\marg{birds}
\atentry{index}\marg{duck,\field{parent}=\marg{birds}}
\atentry{index}\marg{goose,\field{plural}=\marg{geese},\field{parent}=\marg{birds}}
\atentry{index}\marg{swan,\field{parent}=\marg{birds}}
\strut
\atentry{index}\marg{minerals}
\atentry{index}\marg{quartz,\field{parent}=\marg{minerals}}
\atentry{index}\marg{corundum,\field{parent}=\marg{minerals}}
\atentry{index}\marg{amethyst,\field{parent}=\marg{minerals}}
\atentry{index}\marg{gypsum,\field{parent}=\marg{minerals}}
\atentry{index}\marg{gold,\field{parent}=\marg{minerals}}
\end{codeenv}
and the document contains:
\begin{codeenv}
\cmd{documentclass}\marg{article}
\strut
\cmd{usepackage}[\styopt{record},\styopt[indexgroup]{style}]\marg{glossaries-extra}
\strut
\gls{GlsXtrLoadResources}\oarg{\csopt[entries]{src},\csopt{save-child-count}}
\strut
\cmd{begin}\marg{document}
\cs{gls}\marg{duck} and \cs{gls}\marg{goose}.
\cs{gls}\marg{quartz}, \cs{gls}\marg{corundum}, \cs{gls}\marg{amethyst}.
\strut
\cs{printunsrtglossaries}
\cmd{end}\marg{document}
\end{codeenv}
Then the \ext{glstex} file will contain:
\begin{codeenv}
\cs{GlsXtrSetField}\marg{birds}\marg{childcount}\marg{2}
\cs{GlsXtrSetField}\marg{duck}\marg{childcount}\marg{0}
\gls{glsxtrfieldlistadd}\marg{birds}\marg{childlist}\marg{duck}
\cs{GlsXtrSetField}\marg{goose}\marg{childcount}\marg{0}
\gls{glsxtrfieldlistadd}\marg{birds}\marg{childlist}\marg{goose}
\cs{GlsXtrSetField}\marg{minerals}\marg{childcount}\marg{3}
\cs{GlsXtrSetField}\marg{amethyst}\marg{childcount}\marg{0}
\gls{glsxtrfieldlistadd}\marg{minerals}\marg{childlist}\marg{amethyst}
\cs{GlsXtrSetField}\marg{corundum}\marg{childcount}\marg{0}
\gls{glsxtrfieldlistadd}\marg{minerals}\marg{childlist}\marg{corundum}
\cs{GlsXtrSetField}\marg{quartz}\marg{childcount}\marg{0}
\gls{glsxtrfieldlistadd}\marg{minerals}\marg{childlist}\marg{quartz}
\end{codeenv}
Note that although \code{birds} has three \children\ defined in the
\ext{bib} file, only two have been selected, so the \child\ count is
set to~2. Similarly the \code{minerals} entry has five \children\
defined in the \ext{bib} file, but only three have been selected, so
the \child\ count is~3.

The following uses the post-description hook to show the \child\ count
in parentheses:
\begin{codeenv}
\gls{GlsXtrLoadResources}\oarg{\csopt[entries]{src},\csopt[general]{category},\csopt{save-child-count}}
\strut
\cs{glsdefpostdesc}\marg{general}\marg{\comment{}
  \cs{glsxtrifhasfield}\marg{childcount}\marg{\cs{glscurrententrylabel}}\comment{}
  \marg{ (child count: \cs{glscurrentfieldvalue}.)}\comment{}
  \marg{}\comment{}
}
\end{codeenv}
\ics{glsxtrifhasfield} requires at least \sty{glossaries-extra} v1.19.
It's slightly more efficient than \ics{ifglshasfield} provided by
the base \styfmt{glossaries} package, and it doesn't complain if the
entry or field don't exist, but note that \ics{glsxtrifhasfield}
implicitly scopes its content. Use the starred version to omit the
grouping.  With \sty{glossaries-extra} v1.31+ you can perform a
numerical test with \ics{GlsXtrIfFieldNonZero} or
\ics{GlsXtrIfFieldEqNum}.

\optsection{save-sibling-count}

This is a boolean option. The default setting is
\csopt[false]{save-sibling-count}. This is like
\csopt{save-child-count} but saves the \gls{sibling} count in
\field{siblingcount} and the \gls{sibling} list in
\field{siblinglist}.  As with the \child list, the \gls{sibling}
list is in \sty{etoolbox}['s] internal list format. The
\gls{sibling} information is only saved for entries that have a
\parent.

The advantage with \field{siblinglist} over accessing the
\glsdisp{parententry}{parent's} \field{childlist} is that the entry
itself is excluded from the list.

\optsection{save-root-ancestor}

This is a boolean option. The default setting is
\csopt[false]{save-root-ancestor}. If true, the entry's top-most
\gls{ancestor} will be saved in the entry's \field{rootancestor} internal
field. If the entry doesn't have a \parent\ (that it, the entry itself
is the root) then the \field{rootancestor} field won't be set.

\optsection{flatten}

This is a boolean option. The default value is \csopt[false]{flatten}. 
If \csopt[true]{flatten}, the sorting will ignore \hierarchy\ and
the \field{parent} field will be omitted when writing
the definitions to the \ext{glstex} file, but the \glspl{parententry}
will still be considered a dependent \gls{ancestor} from the 
\csopt{selection} point of view.

Note the difference between this option and using
\csopt[parent]{ignore-fields} which will remove the dependency
(unless a dependency is established through another field).

\optsection{flatten-lonely}

This may take one of three values: \optfmt{false} (default),
\optfmt{presort} and \optfmt{postsort}. The value must be supplied. 

Unlike the \csopt{flatten} option, which completely
removes the \hierarchy, the \csopt{flatten-lonely} option can be used
to selectively alter the \hierarchy.  In this case only those entries
that have a \parent\ but have no \glspl{sibling} are considered. This option
is affected by the \csopt{flatten-lonely-rule} setting. The conditions for
moving a \child\ up one \hierarchicallevel\ are as follows:
\begin{itemize}
\item The \child\ must have a \parent, and
\item the \child\ can't have any selected \glspl{sibling}, and
\item if \csopt[only unrecorded parents]{flatten-lonely-rule}
      then the \parent\ can't have a \gls{locationlist}, where the 
      \gls{locationlist} includes \glspl{record} and \field{see} or \field{seealso}
      cross-references (for the other rules the \parent\ 
      may have a \gls{locationlist} as long as it only has the one \child\ selected).
\end{itemize}
If the \child\ is selected for \hierarchical\ adjustment, 
the \parent\ will be removed if:
\begin{itemize}
\item The \parent\ has no \gls{locationlist}, and
\item \csopt{flatten-lonely-rule} isn't set to \optfmt{no discard}.
\end{itemize}

The value of \csopt{flatten-lonely} determines whether the
adjustment should be made before sorting (\optfmt{presort})
or after sorting (\optfmt{postsort}). To disable this function
use \csopt[false]{flatten-lonely}.

For example, suppose the file \filefmt{entries.bib} contains:
\begin{codeenv}
\atentry{index}\marg{birds}
\atentry{index}\marg{duck,\field{parent}=\marg{birds}}
\atentry{index}\marg{goose,\field{plural}=\marg{geese},\field{parent}=\marg{birds}}
\atentry{index}\marg{swan,\field{parent}=\marg{birds}}
\atentry{index}\marg{chicken,\field{parent}=\marg{birds}}
\strut
\atentry{index}\marg{vegetable}
\atentry{index}\marg{cabbage,\field{parent}=\marg{vegetable}}
\strut
\atentry{index}\marg{minerals}
\atentry{index}\marg{quartz,\field{parent}=\marg{minerals}}
\atentry{index}\marg{corundum,\field{parent}=\marg{minerals}}
\atentry{index}\marg{amethyst,\field{parent}=\marg{minerals}}
\atentry{index}\marg{gypsum,\field{parent}=\marg{minerals}}
\strut
\atentry{index}\marg{aardvark}
\atentry{index}\marg{bard}
\atentry{index}\marg{buzz}
\strut
\atentry{index}\marg{item}
\atentry{index}\marg{subitem,\field{parent}=\marg{item}}
\atentry{index}\marg{subsubitem,\field{parent}=\marg{subitem}}
\end{codeenv}
and suppose the document contains:
\begin{codeenv}
\cmd{documentclass}\marg{article}
\strut
\cmd{usepackage}[\styopt{record},\styopt[indexgroup]{style}]\marg{glossaries-extra}
\strut
\gls{GlsXtrLoadResources}\oarg{\csopt[entries.bib]{src}}
\strut
\cmd{begin}\marg{document}
\cs{gls}\marg{duck}.
\cs{gls}\marg{quartz}, \cs{gls}\marg{corundum}, \cs{gls}\marg{amethyst}.
\cs{gls}\marg{aardvark}, \cs{gls}\marg{bard}, \cs{gls}\marg{buzz}.
\cs{gls}\marg{vegetable}, \cs{gls}\marg{cabbage}.
\cs{gls}\marg{subsubitem}.
\strut
\cs{printunsrtglossaries}
\cmd{end}\marg{document}
\end{codeenv}
Although the \code{duck} entry has \glspl{sibling} in the
\filefmt{entries.bib} file, none of them have been
\glsdisp{recordedentry}{recorded} in the document, nor has the
\parent\ \code{birds} entry.

This document hasn't used \csopt{flatten-lonely}, so the default
\csopt[false]{flatten-lonely} is assumed. This results in the
\hierarchical\ structure:
\begin{description}\setlength\itemsep{0pt}%
\item[A] \null
\item[aardvark] 1
\item[B] \null
\item[bard] 1
\item[birds] \mbox{}
  \begin{description}\setlength\itemsep{0pt}%
  \item[duck] 1
  \end{description}
\item[buzz] 1
\item[I] \null
\item[item] \mbox{}
  \begin{description}\setlength\itemsep{0pt}%
  \item[subitem] \mbox{}
    \begin{description}\setlength\itemsep{0pt}%
    \item[subsubitem] 1
    \end{description}
  \end{description}
\item[M] \null
\item[minerals] \mbox{}
  \begin{description}\setlength\itemsep{0pt}%
  \item[amethyst] 1
  \item[corundum] 1
  \item[quartz] 1
  \end{description}
\item[V] \null
\item[vegetable] 1
  \begin{description}\setlength\itemsep{0pt}%
  \item[cabbage] 1
  \end{description}
\end{description}
(The \qt{1} in the above indicates the page number.)
There are some entries here that look a little odd: \code{duck},
\code{cabbage} and \code{subsubitem}. In each case they are a
\glsdisp{lonelychildentry}{lone child entry}. It would look better
if they could be compressed, but I don't want to use the
\csopt{flatten} option, as I still want to keep the mineral
\hierarchy.

If I now add \csopt[postsort]{flatten-lonely}:
\begin{codeenv}
\gls{GlsXtrLoadResources}\oarg{\csopt[entries.bib]{src},\csopt[postsort]{flatten-lonely}}
\end{codeenv}
the \hierarchy\ becomes:
\begin{description}\setlength\itemsep{0pt}%
\item[A] \null
\item[aardvark] 1
\item[B] \null
\item[bard] 1
\item[birds, duck] 1
\item[buzz] 1
\item[I] \null
\item[item, subitem, subsubitem] 1
\item[M] \null
\item[minerals] \mbox{}
  \begin{description}\setlength\itemsep{0pt}%
  \item[amethyst] 1
  \item[corundum] 1
  \item[quartz] 1
  \end{description}
\item[V] \null
\item[vegetable] 1
  \begin{description}\setlength\itemsep{0pt}%
  \item[cabbage] 1
  \end{description}
\end{description}
The \field{name} field of the \code{duck} entry has been
set to:
\begin{codeenv}
\field{name}=\marg{\gls{bibglsflattenedchildpostsort}\marg{birds}\marg{duck}}
\end{codeenv}
the \field{text} field has been set to:
\begin{codeenv}
\field{text}=\marg{duck}
\end{codeenv}
the \field{group} field is copied over from the \gls{parententry} (\qt{B}),
and the \field{parent} field has been adjusted, moving \code{duck}
up one \hierarchicallevel.
Finally, the former \parent\ \code{birds} entry has been removed (the default
\csopt[only unrecorded parents]{flatten-lonely-rule} is in effect).

The default definition of \gls!{bibglsflattenedchildpostsort}
formats its arguments so that they are separated by a comma and
space (\qt{birds, duck}). If the \field{text} field had been set
in the original \atentry{index} definition of \text{duck}, it
wouldn't have been altered. This adjustment ensures that in the
document \code{\cs{gls}\marg{duck}} still produces \qt{duck} rather than
\qt{birds, duck}.
(If the \child\ and \parent\ \field{name} fields are identical,
the terms are considered \glspl{homograph}. See below for further details.)

The \code{subsubitem} entry has also been adjusted. This was done
in a multi-stage process, starting with sub-items and then moving down
the \hierarchical\ levels:
\begin{itemize}
\item The \code{subitem} entry was adjusted, moving it from a
\gls{sub-entry} to a \gls{top-levelentry}. The \field{name} field was then
modified to:
\begin{codeenv}
\field{name}=\marg{\gls{bibglsflattenedchildpostsort}\marg{item}\marg{subitem}}
\end{codeenv}
This now means that the \code{subsubitem} entry is now a \gls{sub-entry}
(rather than a sub-sub-entry). The \code{subitem} entry now has no
\parent, but at this stage the \code{subsubitem} entry still has
\code{subitem} as its \parent.
\item The \code{subsubitem} entry is then adjusted moving from a
\gls{sub-entry} to a \gls{top-levelentry}. The \field{name} field was then
modified to:
\begin{codeenv}
\field{name}=
\marg{\comment{}
 \gls{bibglsflattenedchildpostsort}
 \marg{\comment{}
   \comment{name from former parent}
   \gls{bibglsflattenedchildpostsort}\marg{item}\marg{subitem}\comment{}
 }\comment{}
 \marg{subsubitem}\comment{original name}
}
\end{codeenv}
The first argument of \gls!{bibglsflattenedchildpostsort} is
obtained from the \field{name} field of the entry's former \parent\
(which is removed from the \glsdisp{childentry}{child's} set of
\glspl{ancestor}). This field value was changed in the previous
step, and the change is reflected here.

This means that the name for \code{subitem} will be displayed as
\qt{item, subitem} and the name for \code{subsubitem} will be
displayed as \qt{item, subitem, subsubitem}.

\item The \glspl{parententry} \code{item} and \code{subitem}
are removed from the selection as they have no \glspl{locationlist}.
\end{itemize}

Note that the \code{cabbage} \gls{sub-entry} hasn't been adjusted. It
doesn't have any \glspl{sibling} but its \gls{parententry} (\code{vegetable})
has a \gls{locationlist} so it can't be discarded.
If I change the rule:
\begin{codeenv}
\gls{GlsXtrLoadResources}\oarg{\csopt[entries.bib]{src},
 \csopt[discard unrecorded]{flatten-lonely-rule},
 \csopt[postsort]{flatten-lonely}
}
\end{codeenv}
then this will move the \code{cabbage} entry up a level but the
original \gls{parententry} \code{vegetable} will remain:
\begin{description}\setlength\itemsep{0pt}%
\item[A] \null
\item[aardvark] 1
\item[B] \null
\item[bard] 1
\item[birds, duck] 1
\item[buzz] 1
\item[I] \null
\item[item, subitem, subsubitem] 1
\item[M] \null
\item[minerals] \mbox{}
  \begin{description}\setlength\itemsep{0pt}%
  \item[amethyst] 1
  \item[corundum] 1
  \item[quartz] 1
  \end{description}
\item[V] \null
\item[vegetable] 1
\item[vegetable, cabbage] 1
\end{description}

Remember that \csopt[postsort]{flatten-lonely} performs the
adjustment after sorting. This means that the entries are still in
the same relative location that they were in with the original
\csopt[false]{flatten-lonely} setting. For example, \code{duck}
remains in the B letter group before \qt{buzz}.

With \csopt[presort]{flatten-lonely} the adjustments are made before
the sorting is performed. For example, using:
\begin{codeenv}
\gls{GlsXtrLoadResources}\oarg{\csopt[entries.bib]{src},
 \csopt[discard unrecorded]{flatten-lonely-rule},
 \csopt[presort]{flatten-lonely}
}
\end{codeenv}
the \hierarchical\ order is now:
\begin{description}\setlength\itemsep{0pt}%
\item[A] \null
\item[aardvark] 1
\item[B] \null
\item[bard] 1
\item[buzz] 1
\item[C] \null
\item[cabbage] 1
\item[D] \null
\item[duck] 1
\item[M] \null
\item[minerals] \mbox{}
  \begin{description}\setlength\itemsep{0pt}%
  \item[amethyst] 1
  \item[corundum] 1
  \item[quartz] 1
  \end{description}
\item[S] \null
\item[subsubitem] 1
\item[V] \null
\item[vegetable] 1
\end{description}
This method uses a different format for the modified \field{name}
field. For example, the \code{duck} entry now has:
\begin{codeenv}
\field{name}=\marg{\gls{bibglsflattenedchildpresort}\marg{duck}\marg{birds}}
\end{codeenv}
The default definition of \gls{bibglsflattenedchildpresort} simply
does the first argument and ignores the second. The sorting is then
performed, but the interpreter recognises this command and can
deduce that the sort value for this entry should be \code{duck},
so \qt{duck} now ends up in the D letter group.

If you provide a definition of \gls{bibglsflattenedchildpresort} in
the \atentry{preamble}, it will be picked up by the interpreter.
For example:
\begin{codeenv}
\atentry{preamble}\marg{\qtdelim{\cs{providecommand}\marg{\gls{bibglsflattenedchildpresort}}[2]\marg{\idx{param}1 (\idx{param}2)}}}
\end{codeenv}

Note that the \field{text} field is only changed if not already set.
This option may have unpredictable results for abbreviations as the
\field{name} field (and sometimes the \field{text} field) is
typically set by the abbreviation style.  Remember that
if the \gls{parententry} doesn't have a \igls{locationlist} and the rule isn't
set to \optfmt{no discard} then the \gls{parententry} will be discarded
after all relevant entries and their dependencies have been
selected, so any cross-references within the \gls{parententry} (such as
\ics{gls} occurring in the description) may end up being selected
even if they wouldn't be selected if the \gls{parententry} didn't exist.

With both \optfmt{presort} and \optfmt{postsort}, if the \parent\ \field{name}
is the same as the \glsdisp{childentry}{child's} \field{name}
then the \child\ is considered a \igls{homograph} and
the \glsdisp{childentry}{child's} name is set to:
\begin{codeenv*}
\format{bibglsflattenedhomograph}
\end{codeenv*}
instead of the corresponding \csfmt{bibglsflattenedchild\ldots sort}. 
This defaults to just \meta{name}.


\optsection{flatten-lonely-rule}

This option governs the rule used by \csopt{flatten-lonely} to
determine which \glspl{sub-entry} (that have no \glspl{sibling}) to adjust and 
which \glsdisp{parententry}{parents} to remove. The value may be one
of the following, where \meta{condition} is the condition provided
by \csopt{flatten-lonely-condition}:
\begin{description}
\item[\optfmt{only unrecorded parents}] Only the \glspl{sub-entry}
that have a \parent\ without a \igls{locationlist} (and have
\meta{condition} evaluate to true) will be altered.
The \gls{parententry} will be removed from the selection 
if the child entry is adjusted.
This value is the default setting.

\item[\optfmt{discard unrecorded}] This setting will adjust all
\glspl{sub-entry} that have no \glspl{sibling} (and have
\meta{condition} evaluate to true) regardless of whether or not the
\parent\ has a \igls{locationlist}.
Only the \glspl{parententry} that don't have a \gls{locationlist} will be
removed from the selection if the child entry is adjusted.

\item[\optfmt{no discard}] This setting will adjust all
\glspl{sub-entry} that don't have \glspl{sibling} (and have
\meta{condition} evaluate to true) regardless of whether or not the
\parent\ has a \igls{locationlist}. No entries will be discarded, so 
\glspl{parententry} that don't have a \igls{locationlist} will still appear in the
glossary.
\end{description}
In the above, the \igls{locationlist} includes \glspl{record} and
cross-references obtained from the \field{see} or \field{seealso}
fields. See \csopt{flatten-lonely} for further details.

\optsection{flatten-lonely-condition}

The value may either be empty, to indicate true (the default), or a
complex condition using syntax described in
\sectionref{sec:conditionals}. After taking into account
\csopt{flatten-lonely} and \csopt{flatten-lonely-rule}, this option
determines whether or not the child entry will be adjusted.  If the
condition evaluates to false, the child entry won't be adjusted.

For example, if both the parent entry and the child entry have long
names, it may be better to keep their hierarchy. The following
will only flatten lonely entries where both the child name and the
parent name have less then 25 characters:
\begin{codeenv*}
 \csopt[postsort]{flatten-lonely},
 \csopt[\gls{LEN}\marg{parent \idx{follow} \field{name}} \idx{ltcmp} 25 \idx{ampand} \gls{LEN}\marg{\field{name}} \idx{ltcmp} 25]{flatten-lonely-condition}
\end{codeenv*}
Alternatively, for a combined length of less than 50 characters:
\begin{codeenv*}
 \csopt[postsort]{flatten-lonely},
 \csopt[\gls{LEN}\marg{parent \idx{follow} \field{name} \idx{concat-plus} \field{name}} \idx{ltcmp} 50]{flatten-lonely-condition}
\end{codeenv*}
This doesn't include the number of characters taken up by the
separator but the maximum value can be adjusted to allow for that,
given a constant string separator.

\optsection{flatten-lonely-missing-field-action}

This option indicates what to do if a source field identified in 
\csopt{flatten-lonely-condition} is missing. The value may be one of:
\begin{itemize}
\item \optfmt{skip}: return null;
\item \optfmt{fallback}: use the fallback for the missing field
(see \sectionref{sec:fallbacks}), if
one is available, otherwise return null (default);
\item \optfmt{empty}: treat the missing value as empty.
\end{itemize}
Returning null will result in the flatten lonely instruction being omitted.

\optsection{strip-missing-parents}

The \sty{glossaries} package requires that all \glspl{childentry} must be
defined after the \gls{parententry}. An error occurs otherwise, so
\bibgls\ will omit the \field{parent} field if it can't be found in
the given \igls{resourceset}. However, when the default
\csopt[false]{strip-missing-parents} is on, this omission only occurs
while writing the definitions in the \ext{glstex} file (after
selection and sorting).

Sorting is performed \hierarchically\ and the \field{group} field is
set accordingly for the \glspl{top-levelentry} (but not for
\glspl{childentry}), which means that an entry with a \field{parent}
field will be treated by the sort method as a \gls{childentry}. This can
lead to a strange result, which \bibgls\ warns about:
\begin{alltt}
Parent '\meta{parent id}' not found for entry \meta{child-id}
\end{alltt}

This is the default behaviour as it may simply be a result of a
typing mistake in the \field{parent} field. If you actually want
missing \glsdisp{parententry}{parents} to be stripped before sorting (but after the
selection process) then use \csopt[true]{strip-missing-parents}. If
you want all \glsdisp{parententry}{parents} stripped then use \csopt{flatten} or
\csopt[parent]{ignore-fields} instead. As from version 1.4, if you
want \bibgls\ to create the missing \glsdisp{parententry}{parents}, then you can use 
\csopt[create]{missing-parents}.

\optsection{missing-parents}

As an alternative to \csopt{strip-missing-parents}, as from version
1.4 you can now use \csopt[\meta{value}]{missing-parents} where
\meta{value} may be one of:
\begin{itemize}
\item \optfmt{strip}: this is equivalent to
\csopt[true]{strip-missing-parents};

\item \optfmt{warn}: this is equivalent to the default
\csopt[false]{strip-missing-parents};

\item \optfmt{create}: this will create a new \atentry{index} entry
with the missing \glsdisp{parententry}{parent's} label (after it's
been processed by options such as \csopt{labelify}) with the
\field{name} obtained from the \emph{original} value of the
\field{parent} field (before being processed by options like
\csopt{labelify}).  If the \gls{childentry} has the \field{type} field
set, then the new \gls{parententry} will be given the same value. The
\field{category} for the new \gls{parententry} can be assigned with
\csopt{missing-parent-category}.
\end{itemize}

For example, consider the \exfile!{books.bib} file which contains
entries like:
\begin{codeenv}
\atentry{entry}\marg{ubik,
  \field{name}=\marg{Ubik},
  \field{description}=\marg{novel by Philip K. Dick},
  \fieldfmt{identifier}=\marg{book},
  \fieldfmt{author}=\marg{\gls{sortmediacreator}\marg{Philip K.}\marg{Dick}},
  \fieldfmt{year}=\marg{1969}
}
\end{codeenv}
then the field alias:
\begin{codeenv}
\csopt[\fieldfmt{author}=\field{parent}]{field-aliases}
\end{codeenv}
will treat:
\begin{codeenv}
  \fieldfmt{author}=\marg{\gls{sortmediacreator}\marg{Philip K.}\marg{Dick}},
\end{codeenv}
as though it had been defined as:
\begin{codeenv}
  \field{parent}=\marg{\gls{sortmediacreator}\marg{Philip K.}\marg{Dick}},
\end{codeenv}
This can be converted into a label with the options:
\begin{codeenv}
 \csopt[parent]{labelify},
 \csopt[
   \marg{[ \cs{cs.string}\csfmt{.}]}\marg{}
 ]{labelify-replace}
\end{codeenv}
If the interpreter has been provided with the definition:
\begin{codeenv}
\cs{providecommand}*\marg{\gls{sortmediacreator}}[2]\marg{\idx{param}2 \idx{param}1}
\end{codeenv}
then the \field{parent} field for the \code{ubik} entry
will become \code{DickPhilipK} but the original value is stored
internally when \csopt[create]{missing-parents} is set so that it
can be used as the \field{name} if the \parent\ needs to be created.
Once all the entries have been processed, if \code{ubik} has been
selected but no entry can be found with the label \code{DickPhilipK}
then a new entry will be added as though it had been defined with:
\begin{codeenv}
\atentry{index}\marg{DickPhilipK,
  \field{name}=\marg{\gls{sortmediacreator}\marg{Philip K.}\marg{Dick}}
}
\end{codeenv}
This is an alternative approach to the \exfile{sample-authors.tex}
document from the \hyperref[sec:examples]{examples chapter}.

\optsection{missing-parent-category}

If a missing \gls{parententry} is created through the use of
\csopt[create]{missing-parents} then the \field{category} field can
be assigned to the new \gls{parententry} with this option. The
\meta{value} may be one of:
\begin{itemize}
 \item \code{same as child}: the \gls{parententry}['s] \field{category}
  field is set to the same value as the \glsdisp{childentry}{child's} (if set);
 \item \code{same as base}: the \gls{parententry}['s] \field{category} is
  set to the base name of the \ext{bib} file that provided the
  \gls{childentry}['s] definition;
 \item \code{no value} or \code{false}: don't set the \field{category} field;
 \item \meta{label}: the \gls{parententry}['s] \field{category} field is set to
  \meta{label} (which shouldn't contain any special characters). 
\end{itemize}
The default setting is \csopt[no value]{missing-parent-category}.

\optsection{group-level}

If letter \idx{group} formation is enabled (see \csopt{group},
\csopt{group-formation} and \longarg{group}) then the default
behaviour is to only assign the group label for \glspl{top-levelentry}.
This option allows the group label to be assigned to \glspl{sub-entry}
if \idxpl{sub-group} are required.
The value may be one of the following:
\begin{itemize}
\item \meta{n}: only assign the group for level~\meta{n} entries;
\item \code{>}\meta{n}: only assign the group for entries with a
level greater than \meta{n};
\item \code{>=}\meta{n}: only assign the group for entries with a
level greater than or equal to \meta{n};
\item \code{<}\meta{n}: only assign the group for entries with a
level less than \meta{n};
\item \code{<=}\meta{n}: only assign the group for entries with a
level less than or equal to \meta{n};
\item \code{all}: equivalent to \csopt[>=0]{group-level}.
\end{itemize}
The default setting is \csopt[0]{group-level}.
If no value is provided, \csopt[all]{group-level} is assumed. The
hierarchical levels start at 0 (\gls{top-levelentry}). For any value
other than \csopt[0]{group-level}, the parent entry label will be
included in the group label.

The \idx{hiergroup} titles are formatted according to
\gls{bibglshiersubgrouptitle}. If the group title would
usually be set with the command \csfmt{bibglsset\ldots group}
for \glspl{top-levelentry} then the \idx{hiergroup} title would
be set with the analogous \csfmt{bibglsset\ldots group} command.
For example, letter groups are normally set with
\gls{bibglssetlettergrouptitle} but hierarchical letter groups are set
with \gls{bibglssetlettergrouptitlehier}.

If the \longarg{no-group} setting is on then this option has no effect.

\begin{important}
Any value other than the default \csopt[0]{group-level} requires
\sty{glossaries-extra} v1.49+, which provides \cs{glssubgroupheading}.
\end{important}

\Idxpl{sub-group} are implemented by the glossary style command:
\nosecformatdef{glssubgroupheading}
The \sty{glossaries-extra} package automatically implements:
\begin{codeenv}
\cmd{renewcommand}*\marg{\cs{glssubgroupheading}}[4]\marg{\cs{glsgroupheading}\marg{\idx{param}4}}
\end{codeenv}
whenever a style is set, so that if the style doesn't provide a
definition for this command, it will behave like \cs{glsgroupheading}.

\optsection{merge-small-groups}

Merges consecutive \idxpl{smallgroup} that have less than \meta{n} entries. The
default is \csopt[0]{merge-small-groups}, which switches off this
action. If \meta{n} is omitted, \csopt[1]{merge-small-groups} is
assumed.

This setting only has an effect if group formation is enabled.
If hierarchical \idxpl{sub-group} are enabled (\csopt{group-level}) then
merging is only performed on consecutive small groups within the
same hierarchical level. Any \glspl{childentry} that aren't in their
own \idx{sub-group} are included in the higher level group count. 

For example, suppose you have a large number of entries in most of
the letter groups:
\begin{codeenv}
\atentry{index}\marg{aardvark}
\atentry{index}\marg{ant}
\atentry{index}\marg{alligator}
\atentry{index}\marg{ape}
\comment{etc}
\end{codeenv}
but you only have one entry in each of the \qt{X}, \qt{Y} and \qt{Z}
groups:
\begin{codeenv}
\atentry{index}\marg{xylem}
\atentry{index}\marg{yak}
\atentry{index}\marg{zebra}
\end{codeenv}
then you may prefer to merge these entries into a single group:
\begin{codeenv}
\gls{GlsXtrLoadResources}\oarg{merge-small-groups}
\end{codeenv}
The title of this merged group is obtained from
\gls{bibglsmergedgrouptitle} (or \gls{bibglsmergedgrouptitle} if
hierarchical groups have been enabled with \csopt{group-level}).
For the above example, the merged letter group would have the title
\qt{X, Y, Z}. If there are more than three groups then the middle
group titles are replace with an ellipsis. For example, if there is
also only one entry in the \qt{W} letter group, then the merged
title would be \qt{W,\ldots, Z}.

The small groups must be consecutive (there is no group between
them) and on the same hierarchical level in order to be merged.
In the above example, if the yak entry isn't selected so that there
is no \qt{Y} letter group, then the \qt{X} and \qt{Z} groups can be
merged (with the merged title \qt{X, Z}). If, on the other hand,
extra entries occur in the \qt{Y} letter group, so that it is larger
than the value of \csopt{merge-small-groups}, then \qt{X} and
\qt{Z} can no longer be merged.

\section{Master Documents}
\label{sec:master}

Suppose you have two documents \filefmt{mybook.tex} and
\filefmt{myarticle.tex} that share a common glossary that's shown
in \filefmt{mybook.pdf} but not in \filefmt{myarticle.pdf}.
Furthermore, you'd like to use \isty{hyperref} and be able to click
on a term in \filefmt{myarticle.pdf} and be taken to the relevant 
page in \filefmt{mybook.pdf} where the term is listed in the
glossary.

This can be achieved with the \catattr{targeturl} and
\catattr{targetname} category attributes. For example, without
\bibgls\ the file \filefmt{mybook.tex} might look like:
\begin{codeenv}
\cmd{documentclass}\marg{book}
\cmd{usepackage}[colorlinks]\marg{hyperref}
\cmd{usepackage}\marg{glossaries-extra}
\strut
\cs{cs.makeglossaries}
\strut
\gls{newglossaryentry}\marg{sample}\marg{\field{name}=\marg{sample},\field{description}=\marg{an example}}
\strut
\cmd{begin}\marg{document}
\cs{chapter}\marg{Example}
\cs{gls}\marg{sample}.
\strut
\cs{printglossaries}
\cmd{end}\marg{document}
\end{codeenv}
The other document \filefmt{myarticle.tex} might look like:
\begin{codeenv}
\cmd{documentclass}\marg{article}
\cmd{usepackage}[colorlinks]\marg{hyperref}
\cmd{usepackage}\marg{glossaries-extra}
\strut
\cs{newignoredglossary*}\marg{external}
\ics{glssetcategoryattribute}\marg{external}\marg{targeturl}\marg{mybook.pdf}
\cs{glssetcategoryattribute}\marg{external}\marg{targetname}\marg{\cs{glolinkprefix}\cs{glslabel}}
\strut
\gls{newglossaryentry}\marg{sample}\marg{\field{type}=external,\field{category}=external,
 \field{name}=\marg{sample},\field{description}=\marg{an example}}
\strut
\cmd{begin}\marg{document}
\cs{gls}\marg{sample}.
\cmd{end}\marg{document}
\end{codeenv}
In this case the \mainglossary\ isn't used, but the category 
attributes allow a mixture of internal and external references, so
the \mainglossary\ could be used for the internal
references. (In which case, \csfmt{makeglossaries} and
\csfmt{printglossaries} would need to be added back to
\filefmt{myarticle.tex}.)

Note that both documents had to define the common terms. The above
documents can be rewritten to work with \bibgls. First a \ext{bib}
file needs to be created:
\begin{codeenv}
\atentry{entry}\marg{sample,
  \field{name}=\marg{sample},
  \field{description}=\marg{an example}
}
\end{codeenv}
Assuming this file is called \filefmt{myentries.bib}, then
\filefmt{mybook.tex} can be changed to:
\begin{codeenv}
\cmd{documentclass}\marg{book}
\cmd{usepackage}[colorlinks]\marg{hyperref}
\cmd{usepackage}[record]\marg{glossaries-extra}
\strut
\gls{GlsXtrLoadResources}\oarg{\csopt[myentries]{src}}
\strut
\cmd{begin}\marg{document}
\cs{chapter}\marg{Example}
\cs{gls}\marg{sample}.
\strut
\cs{printunsrtglossaries}
\cmd{end}\marg{document}
\end{codeenv}
and \filefmt{myarticle.tex} can be changed to:
\begin{codeenv}
\cmd{documentclass}\marg{article}
\cmd{usepackage}[colorlinks]\marg{hyperref}
\cmd{usepackage}[record]\marg{glossaries-extra}
\strut
\cs{newignoredglossary*}\marg{external}
\cs{glssetcategoryattribute}\marg{external}\marg{targeturl}\marg{mybook.pdf}
\cs{glssetcategoryattribute}\marg{external}\marg{targetname}\marg{\cs{glolinkprefix}\cs{glslabel}}
\strut
\gls{GlsXtrLoadResources}\oarg{
 \csopt[myentries]{src},
 \csopt[none]{sort},
 \csopt[external]{type},
 \csopt[external]{category}
}
\strut
\cmd{begin}\marg{document}
\cs{gls}\marg{sample}.
\cmd{end}\marg{document}
\end{codeenv}
Most of the options related to sorting and the glossary format are
unneeded here since the glossary isn't being displayed. This may be
sufficient for your needs, but it may be that the book has changed
various settings that have been written to \filefmt{mybook.glstex}
but aren't present in the \iext{bib} file (such as
\csopt[uc]{short-case-change}). In this case, you could just remember to copy
over the settings from \filefmt{mybook.tex} to
\filefmt{myarticle.tex}, but
another possibility is to simply make \filefmt{myarticle.tex} input
\filefmt{mybook.glstex} instead of using \gls{GlsXtrLoadResources}. This
can work but it's not so convenient to set the label prefix, the
type and the category.  The \optfmt{master} option allows this, but
it has limitations (see below), so in complex cases (in particular
different label prefixes combined with \hierarchical\ entries or cross-references) you'll have to use
the method shown in the example code above.

\optsection{master}

This option will disable most of the options that relate to
parsing and processing data contained in \iext{bib} files
(since this option doesn't actually read any \ext{bib} files).
It also can't be used with \csopt[copy]{action} or \csopt[define
or copy]{action}. A value of \code{false} will switch off this
setting (the default).

The use of \optfmt{master} isn't always suitable. In particular
if any of the terms cross-reference each other, such as through
the \field{see} or \field{seealso} field or the \field{parent} field or
using commands like \ics{gls} in any of the other fields when
the labels have been assigned prefixes. In this
case you will need to use the method described in the example above.

The \meta{name} is the name of the \iext{aux} file for the 
\gls{masterdocument} without the extension (in this case, \filefmt{mybook}). It
needs to be relative to the document referencing it or an absolute
path using forward slashes as the directory divider. Remember that
if it's a relative path, the PDF files (\filefmt{mybook.pdf} and
\filefmt{myarticle.pdf}) will also need to be located in the same
relative position.

When \bibgls\ detects the \csopt{master} option, it won't search for
entries in any \ext{bib} files (for that particular resource set)
but will create a \iext{glstex} file that inputs the
\gls{masterdocument}['s] \ext{glstex} files, but it will
additionally temporarily adjust the internal commands used to define
entries so that the prefix given by \csopt{label-prefix}, the
glossary type and the category type are all automatically inserted.
If the \csopt{type} or \csopt{category} options haven't been used,
the corresponding value will default to \optfmt{master}. The
\catattr{targeturl} and \catattr{targetname} category attributes
will automatically be set, and the glossary type will be provided
using \code{\ics{provideignoredglossary*}\margm{type}} (even if
\longarg{no-provide-glossaries} is set).

The above \filefmt{myarticle.tex} can be changed to:
\begin{codeenv}
\cmd{documentclass}\marg{article}
\cmd{usepackage}[colorlinks]\marg{hyperref}
\cmd{usepackage}[record]\marg{glossaries-extra}
\strut
\gls{GlsXtrLoadResources}\oarg{\csopt[book.]{label-prefix},\csopt[mybook]{master}}
\strut
\cmd{begin}\marg{document}
\cs{gls}\marg{book.sample}.
\cmd{end}\marg{document}
\end{codeenv}

There are some settings from the \gls{masterdocument} that you 
still need to repeat in the other document. These include
the label prefixes set when the \gls{masterdocument} loaded
the resource files, and any settings in the \gls{masterdocument}
that relate to the \gls{masterdocument}['s] entries.

For example, if the \gls{masterdocument} loaded a resource file
with \csopt[term.]{label-prefix} then you also need this
prefix when you reference the entries in the dependent document
in addition to the \optfmt{label-prefix} for the dependent document.
Suppose \filefmt{mybook.tex} loads the resources using:
\begin{codeenv}
\gls{GlsXtrLoadResources}\oarg{\csopt[myentries]{src},\csopt[term.]{label-prefix}}
\end{codeenv}
and \filefmt{myarticle.tex} loads the resources using:
\begin{codeenv}
\gls{GlsXtrLoadResources}\oarg{\csopt[book.]{label-prefix},\csopt[mybook]{master}}
\end{codeenv}
Then the entries referenced in \filefmt{myarticle.tex} need
to use the prefix \optfmt{book.term.}\ as in:
\begin{codeenv}
This is a \cs{gls}\marg{book.term.sample} term.
\end{codeenv}

Remember that the category labels will need adjusting to reflect the
change in category label in the dependent document.

For example, if \filefmt{mybook.tex} included:
\begin{codeenv}
\cs{setabbreviationstyle}\marg{long-short-sc}
\end{codeenv}
then \filefmt{myarticle.tex} will need:
\begin{codeenv}
\cs{setabbreviationstyle}\oarg{master}\marg{long-short-sc}
\end{codeenv}
(change \optfmt{master} to \meta{value} if you have used
\csopt[\meta{value}]{category}). You can, of course, choose
a different abbreviation style for the dependent document, 
but the category in the optional argument needs to be correct.

\optsection{master-resources}

If the \gls{masterdocument} has multiple resource files
then by default all the \gls{masterdocument}['s]
\iext{glstex} files will be input. If you don't want them all
you can use \optfmt{master-resources} to specify
only those files that should be included. The value \meta{list} is 
a comma-separated list of names, where each name corresponds
to the final argument of \gls!{glsxtrresourcefile}.
Remember that \gls{GlsXtrLoadResources} is just a shortcut
for \gls{glsxtrresourcefile} that bases the name on \ics{jobname}.
(Note that, as with the argument of \gls{glsxtrresourcefile},
the \ext{glstex} extension should not be included in \meta{list}.) The file
\primaryresourcefmt\ is considered the primary
resource file and the files \suppresourcefmt{\meta{n}}
(starting with \meta{n} equal to 1) are considered the supplementary resource files.

For example, to just select the first and third of the
supplementary resource files (omitting the primary
\filefmt{mybook.glstex}):
\begin{codeenv}
\gls{GlsXtrLoadResources}\oarg{
  \csopt[mybook]{master},
  \csopt[mybook-1,mybook-3]{master-resources}
}
\end{codeenv}

\section{Field and Label Options}
\label{sec:fieldlabelopts}

The options in this section may be used to set or adjust field
values or labels.  Some field values are expected to be labels (such
as \field{group}).  These labels must not contain special
characters or commands, but it's possible to convert a field value
into a valid label using options such as \csopt{labelify}.

\subsection{Entry Labels}
\label{sec:labelopts}

\optsection[\subsubsection]{interpret-label-fields}

This is a boolean option that determines whether or not the fields
that may only contain labels should have their values interpreted
(\field{parent}, \field{category}, \field{type}, \field{group},
\field{seealso} and \field{alias}). Although this option interprets
commands within those fields, it doesn't strip any characters that
can't be used within a label. The \field{see} field isn't included
as it may optionally start with \code{\oargm{tag}} where \meta{tag}
may legitimately contain \LaTeX\ code that shouldn't be interpreted.

The default setting is \csopt[false]{interpret-label-fields}.
Note that if this setting is on, \iglspl{crossresourceref} aren't
permitted. This setting has no effect if the interpreter has been
disabled.

Related settings are \csopt{labelify} and
\csopt{labelify-list} which can be used to strip content that
can't be used in labels and may be used more generally for other
fields. The \csopt{labelify} and \csopt{labelify-list} options
are performed before \csopt{interpret-label-fields}.

\optsection[\subsubsection]{labelify}

This option should take a comma-separated list of recognised field names as the
value. (If a field is present in both \csopt{labelify} and
\csopt{labelify-list}, then \csopt{labelify-list} takes precedence.) 
Note that if this setting is on, \iglspl{crossresourceref} aren't
permitted. The value is required for this key but may be empty,
which indicates an empty set of fields (that is, the setting is 
switched off).

Each listed field will be converted into a string suitable
for use as a label. (Not necessarily a glossary entry label, but
any label that may be used in the construction of a control sequence
name.) 

The conversion is performed in the following order:
\begin{enumerate}
\item If the interpreter is on and the field value contains
any of the characters \idx{escchar}~(backslash), 
\idx{bgroupchar}~(begin group), \idx{egroupchar}~(end group), 
\idx{nbspchar}~(non-breakable space) or \idx{mshiftchar}~(maths shift), 
then the value is interpreted.

\item Any substitutions that have been specified with
\csopt{labelify-replace} are performed.

\item All characters that aren't alphanumeric or the space character
 or any of the following punctuation
characters \code{.}~(full stop), \code{-}~(hyphen), \code{+}~(plus),
\code{:}~(colon), \code{;}~(semi-colon), \code{\string|}~(pipe),
\code{/}~(forward slash), \code{!}~(exclamation mark), 
\code{?}~(question mark), \code{*}~(asterisk), \code{<}~(less than), 
\code{>}~(greater than), \code{\textasciigrave}~(backtick), 
\code{'}~(apostrophe) or \code{@}~(at-sign) are stripped. If you want to
retain commas, use \csopt{labelify-list} instead. If you want to
strip any of the allowed punctuation, use \csopt{labelify-replace} to
remove the unwanted characters. (Remember that \isty{babel} can make
some of these punctuation characters active, in which case they need
to be stripped.)

\item If \bibgls\ doesn't allow non-ASCII characters in labels, the
value is then decomposed and all non-ASCII characters are removed.
UTF-8 support is automatic if \bibgls\ detects \sty{fontspec} in the document's
transcript file, otherwise UTF-8 in labels will only be supported if
\bibgls\ detects that the versions of \sty{glossaries} and \sty{glossaries-extra}
are new enough to support it. 
To ensure better support for UTF-8 with \pdfLaTeX, make sure you
have a recent \TeX\ distribution and up-to-date versions of
\sty{glossaries} and \sty{glossaries-extra}.
\end{enumerate}

For example, suppose the \ext{bib} file contains:
\begin{codeenv}
\atentry{index}\marg{sample,
  \field{name}=\marg{\ics{AA} ngstr\ics{umlaut}om, \ics{O} stergaard, d'Arcy, and Fotheringay-Smythe}
}
\end{codeenv}
Then:
\begin{codeenv}
\gls{GlsXtrLoadResources}\oarg{
 \csopt[entries]{src},\comment{data in entries.bib}
 \csopt[name]{labelify}
}
\end{codeenv}
will convert the \field{name} field into:
\begin{verbatim}
Angstrom stergaard d'Arcy and Fotheringay-Smythe
\end{verbatim}
if \bibgls\ doesn't support non-ASCII characters in labels otherwise it will be:
\begin{verbatim}
Ångström Østergaard d'Arcy and Fotheringay-Smythe
\end{verbatim}
Note that \O\ is considered an unmodified letter and so can't
be decomposed into a basic Latin letter with a combining diacritic.
It's therefore removed completely from the ASCII label
version. Whereas \AA\ can be decomposed into \qt{A} followed by the
\qt{combining ring above} character and \"o can be decomposed into \qt{o}
followed by the \qt{combining diaresis} character. You can use
\csopt{labelify-replace} to replace non-ASCII characters into
the closest match. Alternatively, switch to using \XeLaTeX\ or
\LuaLaTeX.

You can use this option with \csopt{replicate-fields} if you need
to retain the original:
\begin{codeenv}
\gls{GlsXtrLoadResources}\oarg{
 \csopt[entries]{src},\comment{data in entries.bib}
 \csopt[\field{name}=\marg{\field{user1}}]{replicate-fields},
 \csopt[\field{user1}]{labelify}
}
\end{codeenv}

\optsection[\subsubsection]{labelify-list}

This option is like \csopt{labelify} but it retains commas, as it's
designed for fields that should be converted into a comma-separated
list of labels. Any empty elements are removed. For example, with
the \ext{bib} entry from above:
\begin{codeenv}
\gls{GlsXtrLoadResources}\oarg{
 \csopt[entries]{src},\comment{data in entries.bib}
 \csopt[\field{name}=\marg{\field{user1}}]{replicate-fields},
 \csopt[\field{user1}]{labelify-list}
}
\end{codeenv}
will convert the \field{user1} field into:
\begin{verbatim}
Angstrom, stergaard, d'Arcy, and Fotheringay-Smythe
\end{verbatim}
or:
\begin{verbatim}
Ångström, Østergaard, d'Arcy, and Fotheringay-Smythe
\end{verbatim}
depending on whether or not UTF-8 labels are supported.

\optsection[\subsubsection]{labelify-replace}

This option takes a comma-separated list as a value with each
element in the list in the form \code{\margm{regex}\margm{replacement}}
where \meta{regex} is a \gls{regular-expression} (that conforms to \href{http://docs.oracle.com/javase/8/docs/api/java/util/regex/Pattern.html}{Java's
Pattern class}~\cite{pattern}) and \meta{replacement}
is the replacement text. The value is required for this key but
may be empty, which indicates that the
setting is switched off.

Remember that the argument of
\gls{GlsXtrLoadResources} is expanded when written to the \ext{aux}
file so take care to protect any special characters. For example, to
match a literal \idx{full-stop} use \cs{cs.string}\idx{cs.period}
rather than just \idx{cs.period} (backslash dot).

In the \meta{replacement} part, you can use
\ics{glscapturedgroup}\meta{n} to reference a captured sub-sequence.
For example:
\begin{codeenv}
\csopt[\marg{([A-Z])\ics{cs.string}\idx{cs.period}}\marg{\cs{glscapturedgroup}1}]{labelify-replace}
\end{codeenv} 
This removes any \idx{full-stop} that follows any of the characters
A,\ldots,Z. Alternatively, you can just use \verb|\string\$|
instead of \cs{glscapturedgroup}. If you want a literal dollar
character, you need to use \cs{glshex}\code{24} (or
\verb|\string\u24|). This isn't recommended for labels (since special
characters are automatically stripped), but
\csopt{sort-replace} follows the same rules as
\csopt{labelify-replace}, and it may be needed
for that.

\begin{important}
You can't use the \gls{MGP} quark (which expands to the \gls{MGP} 
identifier in a string \gls{concatenation}) to identify the captured group in this
context, as the replacement text needs to use the correct
\gls{regular-expression} syntax.
\end{important}


Both \csopt{labelify} and \csopt{labelify-list} use the
\csopt{labelify-replace} setting to 
perform substitutions. For example, to replace the sub-string \qt{ and }
(including spaces) with a comma:
\begin{codeenv}
\gls{GlsXtrLoadResources}\oarg{
 \csopt[entries]{src},\comment{data in entries.bib}
 \csopt[\field{name}=\marg{\field{user1}}]{replicate-fields},
 \csopt[\marg{ and }\marg{,}]{labelify-replace},
 \csopt[\field{user1}]{labelify-list}
}
\end{codeenv}
The earlier example will now end up as:
\begin{verbatim}
Angstrom, stergaard, d'Arcy,Fotheringay-Smythe
\end{verbatim}
or:
\begin{verbatim}
Ångström, Østergaard, d'Arcy,Fotheringay-Smythe
\end{verbatim}
depending on whether or not UTF-8 labels are supported.

Note that this produces the same result regardless of whether or not
the Oxford comma is present as \verb*|, and | would first be
converted to \verb|,,| and then the empty element is removed
resulting in a single comma.

You can have more than one replacement:
\begin{codeenv}
\gls{GlsXtrLoadResources}\oarg{
 \csopt[entries]{src},\comment{data in entries.bib}
 \csopt[\field{name}=\marg{\field{user1}}]{replicate-fields},
 \csopt[
   \marg{ and }\marg{,},\comment{first substitution}
   \marg{[ '\cs{cs.string}\cmd{-}]}\marg{},\comment{second substitution}
   \marg{\ics{glshex}00D8}\marg{O}\comment{third substitution}
 ]{labelify-replace},
 \csopt[\field{user1}]{labelify-list}
}
\end{codeenv}
This additionally removes the space, apostrophe and hyphen
characters (second substitution) and replaces \qt{\O} (\hex{00D8})
with \qt{O} (third substitution) so the string now ends up as:
\begin{verbatim}
Angstrom,Ostergaard,dArcy,FotheringaySmythe
\end{verbatim}
or:
\begin{verbatim}
Ångström,Ostergaard,dArcy,FotheringaySmythe
\end{verbatim}
depending on whether or not UTF-8 labels are supported.

\optsection[\subsubsection]{label-prefix}

The \csopt{label-prefix} option prepends \meta{tag} to each entry's label. This
\meta{tag} will also be inserted in front of any cross-references, unless they
start with \idprefix{dual} or \idprefix{tertiary} or \idprefix{extn} (where
\meta{n} is an integer). Use \csopt{dual-prefix} to change the dual label
prefixes and \csopt{ext-prefixes} to change the external label prefixes.

If you set \csopt{label-prefix} and you define commands with
\gls{glsxtrnewglslike}, then any of those commands found in entry
fields won't have the \csopt{label-prefix} inserted if the prefix provided with
the command starts with the prefix given in \csopt{label-prefix}.
(This doesn't apply to other prefix options, such as
\csopt{dual-prefix}, so take care if you have a mixture of prefix
options and prefixes identified with \gls{glsxtrnewglslike}.)

As from version 1.8, the \primary\ label prefix is identified
in the \ext{glstex} file with:
\begin{codeenv}
\format{bibglsprimaryprefixlabel}
\end{codeenv}

For example, if the \ext{bib} file contains:
\begin{codeenv}
\atentry{entry}\marg{bird,
  \field{name}=\marg{bird},
  \field{description} = \marg{feathered animal, such as a \cs{gls}\marg{duck} or \cs{gls}\marg{goose}}
}
\strut
\atentry{entry}\marg{waterfowl,
  \field{name}=\marg{waterfowl},
  \field{description}=\marg{Any \cs{gls}\marg{bird} that lives in or about water},
  \field{see}=\marg{[see also]\marg{duck,goose}}
}
\strut
\atentry{index}\marg{duck}
\strut
\atentry{index}\marg{goose,\field{plural}=\qtdelim{geese}}
\end{codeenv}
Then if this \ext{bib} file is loaded with \csopt[gls.]{label-prefix}
it's as though the entries had been defined as:
\begin{codeenv}
\atentry{entry}\marg{gls.bird,
  \field{name}=\marg{bird},
  \field{description} = \marg{feathered animal, such as a \cs{gls}\marg{gls.duck} or \cs{gls}\marg{gls.goose}}
}
\strut
\atentry{entry}\marg{gls.waterfowl,
  \field{name}=\marg{waterfowl},
  \field{description}=\marg{Any \cs{gls}\marg{gls.bird} that lives in or about water},
  \field{see}=\marg{[see also]\marg{gls.duck,gls.goose}}
}
\strut
\atentry{index}\marg{gls.duck,\field{name}=\marg{duck}}
\strut
\atentry{index}\marg{gls.goose,\field{name}=\marg{goose},\field{plural}=\qtdelim{geese}}
\end{codeenv}

Remember to use this prefix when you reference the terms in the
document with commands like \ics{gls}.

\optsection[\subsubsection]{duplicate-label-suffix}

The \sty{glossaries} package doesn't permit entries with duplicate
labels (even if they're in different glossaries). If you
have multiple \iglspl{resourceset} and an entry that's selected
in one \igls{resourceset} is also selected in another, by
default, \bibgls\ will issue a warning, but it will still write the
entry definition to the \ext{glstex} file, which means you'll also
get a warning from \sty{glossaries-extra} and the duplicate
definition will be ignored, but associated internal fields
set with commands like \cs{GlsXtrSetField} may still be set.

If you actually want the duplicate, you need to specify a
suffix with \csopt{duplicate-label-suffix}. This suffix is only
set just before writing the entry definition to the \ext{glstex}
file, so it doesn't affect selection criteria nor can label
substitutions be performed in any cross-references. Options such as
\csopt{set-widest} that reference entry labels are incompatible
as they will use the unsuffixed label.

The actual suffix is formed from \meta{value}\meta{n} where \meta{n}
is an integer that's incremented in the event of multiple
duplicates.  For example, \csopt[.copy]{duplicate-label-suffix} will
change the label to \meta{id}\code{.copy1} for the first duplicate
of the entry whose label is \meta{id}, and \meta{id}\code{.copy2} for
the second duplicate, etc.

\optsection[\subsubsection]{record-label-prefix}

If set, this option will cause \bibgls\ to pretend that each record
label starts with \meta{tag}, if it doesn't already. For example, suppose
the records in the \ext{aux} file are:
\begin{codeenv}
\gls{glsxtr@record}\marg{bird}\marg{}\marg{\counter{page}}\marg{\encap{glsnumberformat}}\marg{1}
\gls{glsxtr@record}\marg{waterfowl}\marg{}\marg{\counter{page}}\marg{\encap{glsnumberformat}}\marg{1}
\gls{glsxtr@record}\marg{idx.duck}\marg{}\marg{\counter{page}}\marg{\encap{glsnumberformat}}\marg{1}
\gls{glsxtr@record}\marg{idx.goose}\marg{}\marg{\counter{page}}\marg{\encap{glsnumberformat}}\marg{1}
\end{codeenv}
The use of \csopt[idx.]{record-label-prefix} makes \bibgls\ act as 
though the records were given as:
\begin{codeenv}
\gls{glsxtr@record}\marg{idx.bird}\marg{}\marg{\counter{page}}\marg{\encap{glsnumberformat}}\marg{1}
\gls{glsxtr@record}\marg{idx.waterfowl}\marg{}\marg{\counter{page}}\marg{\encap{glsnumberformat}}\marg{1}
\gls{glsxtr@record}\marg{idx.duck}\marg{}\marg{\counter{page}}\marg{\encap{glsnumberformat}}\marg{1}
\gls{glsxtr@record}\marg{idx.goose}\marg{}\marg{\counter{page}}\marg{\encap{glsnumberformat}}\marg{1}
\end{codeenv}


\optsection[\subsubsection]{cs-label-prefix}

If you have commands such as \code{\ics{gls}\margm{label}} or
\code{\ics{glstext}\margm{label}} in field
values (in situations where nested link text won't cause a problem)
the \meta{label} will be converted as follows:
\begin{itemize}
\item if \meta{label} starts with \idprefix{dual} then
\idprefix{dual} will be replaced by the \csopt{dual-prefix} value;
\item if \meta{label} starts with \idprefix{tertiary} then
\idprefix{tertiary} will be replaced by the \csopt{tertiary-prefix} value;
\item if \meta{label} starts with \idprefix{extn} then
\idprefix{extn} will be replaced by the corresponding
\csopt{ext-prefixes} setting (if \igls{crossresourceref} mode is
enabled, see \sectionref{sec:resourcesets});
\item if \meta{label} doesn't start with one of the above recognised
prefixes then, if \csopt{cs-label-prefix} has been used the supplied
value will be inserted otherwise the \csopt{label-prefix} setting
will be inserted.
\end{itemize}
For example, given:
\begin{codeenv}
\atentry{entry}\marg{bird,
  \field{name}=\marg{bird},
  \field{description} = \marg{feathered animal, such as a \cs{gls}\marg{duck} or \cs{gls}\marg{goose}}
}
\end{codeenv}
then if \csopt[idx.]{label-prefix} is set but \csopt{cs-label-prefix}
isn't included in the resource option list this will convert the
\field{description} field to:
\begin{codeenv}
\field{description} = \marg{feathered animal, such as a \cs{gls}\marg{idx.duck} or
\cs{gls}\marg{idx.goose}}
\end{codeenv}
However with \csopt[gls.]{cs-label-prefix} the \field{description}
field will be converted to:
\begin{codeenv}
\field{description} = \marg{feathered animal, such as a \cs{gls}\marg{gls.duck} or
\cs{gls}\marg{gls.goose}}
\end{codeenv}
regardless of the \csopt{label-prefix} setting. Whereas if the
original entry definition is:
\begin{codeenv}
\atentry{entry}\marg{bird,
  \field{name}=\marg{bird},
  \field{description} = \marg{feathered animal, such as a \cs{gls}\marg{dual.duck} or
\cs{gls}\marg{dual.goose}}
}
\end{codeenv}
then \idprefix{dual} will be replaced by the value of the
\csopt{dual-prefix} option regardless of the \csopt{cs-label-prefix}
setting.

The \csopt{cs-label-prefix} setting doesn't affect labels in the
fields that have an entry label or label list as the value
(\field{parent}, \field{alias}, \field{see} and \field{seealso}).

\optsection[\subsubsection]{ext-prefixes}

Any cross-references in the \iext{bib} file that start with
\idprefix{extn} (where \meta{n} is a positive integer) will be 
substituted with the \meta{n}th tag listed in the comma-separated
\meta{list}. If there aren't that many items in the list, the
\idprefix{extn} will simply be removed. The default setting is
an empty list, which will strip all \idprefix{extn} prefixes.
Remember that \igls{crossresourceref} mode needs to be enabled for
this option to work (see \sectionref{sec:resourcesets}).

As from version 1.8, the external label prefixes are identified
in the \ext{glstex} file with:
\begin{codeenv}
\format{bibglsexternalprefixlabel}
\end{codeenv}

For example, suppose the file \filefmt{entries-terms.bib} contains:
\begin{codeenv}
\atentry{entry}\marg{set,
  \field{name}=\marg{set},
  \field{description}=\marg{collection of values, denoted \cs{gls}\marg{ext1.set}}
}
\end{codeenv}
and the file \filefmt{entries-symbols.bib} contains:
\begin{codeenv}
\atentry{symbol}\marg{set,
  \field{name}=\marg{\cs{ensuremath}\marg{\cs{mathcal}\marg{S}}},
  \field{description}=\marg{a \cs{gls}\marg{ext1.set}}
}
\end{codeenv}

These files both contain an entry with the label \code{set}
but the \field{description} field includes \code{\cs{gls}\marg{ext1.set}} 
which is referencing the entry from the other file. These
two files can be loaded without conflict using:
\begin{codeenv}
\cmd{usepackage}[\styopt{record},\styopt{symbols}]\marg{glossaries-extra}
\strut
\gls{GlsXtrLoadResources}\oarg{\csopt[entries-terms]{src},
 \csopt[gls.]{label-prefix},
 \csopt[sym.]{ext-prefixes}
}
\strut
\gls{GlsXtrLoadResources}\oarg{\csopt[entries-symbols]{src},
 \csopt[symbols]{type},
 \csopt[sym.]{label-prefix},
 \csopt[gls.]{ext-prefixes}
}
\end{codeenv}

Now the \code{set} entry from \filefmt{entries-terms.bib}
will be defined with the label \code{gls.set} and the
description will be:
\begin{codeenv}
collection of values, denoted \cs{gls}\marg{sym.set}
\end{codeenv}
The \code{set} entry
from \filefmt{entries-symbols.bib} will be defined with the label
\code{sym.set} and the description will be: 
\begin{codeenv}
a \cs{gls}\marg{gls.set}
\end{codeenv}

Note that in this case the \ext{bib} files have to be loaded 
as two separate resources. They can't be combined into a 
single \csopt{src} list as the labels aren't unique. 

If you want to allow the flexibility to choose between 
loading them together or separately, you'll have to give them
unique labels. For example, \filefmt{entries-terms.bib} could
contain:
\begin{codeenv}
\atentry{entry}\marg{set,
  \field{name}=\marg{set},
  \field{description}=\marg{collection of values, denoted \cs{gls}\marg{ext1.S}}
}
\end{codeenv}
and \filefmt{entries-symbols.bib} could contain:
\begin{codeenv}
\atentry{symbol}\marg{S,
  \field{name}=\marg{\cs{ensuremath}\marg{\cs{mathcal}\marg{S}}},
  \field{description}=\marg{a \cs{gls}\marg{ext1.set}}
}
\end{codeenv}
Now they can be combined with:
\begin{codeenv}
\gls{GlsXtrLoadResources}\oarg{\csopt[entries-terms,entries-symbols]{src}}
\end{codeenv}
which will simply strip the \glsdisp{idx.idprefix.extn}{\idprefixfmt{ext1}} 
prefix from the cross-references. Alternatively:
\begin{codeenv}
\gls{GlsXtrLoadResources}\oarg{\csopt[entries-terms,entries-symbols]{src},
 \csopt[gls.]{label-prefix},
 \csopt[gls.]{ext-prefixes}
}
\end{codeenv}
which will insert the supplied \csopt{label-prefix} at the
start of the labels in the entry definitions and will replace
the \idprefixfmt{ext1} prefix with \idprefixfmt{gls} in the
cross-references.

\optsection[\subsubsection]{prefix-only-existing}

This is a boolean option. It's possible that a prefix can end up being
inserted when there's no entry in the current \igls{resourceset} that
matches the prefixed label.  If this option is set then the prefix
won't be added if there's no matching entry.  The default setting is
\csopt[false]{prefix-only-existing}.

\optsection[\subsubsection]{dependency-fields}

The \meta{list} should be a comma-separated list of fields that
have values in the form \code{\oargm{tag}\meta{id-list}} where
\meta{id-list} is a comma-separated list of entry labels. The value
is required for this key but may be empty, which indicates an empty
set of fields (that is, the setting is switched off).
Note that the listed fields must be recognised as known fields, for
example, custom fields defined with \cs{glsaddstoragekey}.

This setting makes those fields act like the \field{see} field by
identifying the listed entries as dependencies, but the information
isn't added to the cross-reference part of the \gls{locationlist}.  This
action is performed after \csopt{labelify-list}, if that's also set.

For example, suppose the file \filefmt{entries-en.bib} contains:
\begin{codeenv}
\atentry{index}\marg{cat,
  \fieldfmt{translations-pt}=\marg{gato,gatinho},
  \field{seealso}=\marg{kitten}
}
\strut
\atentry{index}\marg{kitten,
  \fieldfmt{translations-pt}=\marg{gato,gatinho}
}
\strut
\atentry{index}\marg{staple}
\atentry{index}\marg{rivet}
\atentry{index}\marg{mat}
\atentry{index}\marg{carpet}
\atentry{index}\marg{rug}
\atentry{index}\marg{tapestry}
\atentry{index}\marg{doormat}
\atentry{index}\marg{matting}
\atentry{index}\marg{coconut-matting,
  \field{name}=\marg{coconut matting}
}
\atentry{index}\marg{track}
\atentry{index}\marg{furrow}
\end{codeenv}
and suppose the file \filefmt{entries-pt.bib} contains:
\begin{codeenv}
\atentry{index}\marg{gato,
 \field{prefix}=\marg{o},
 \fieldfmt{translations-en}=\marg{cat,staple,rivet},
 \field{seealso}=\marg{gatinho}
}
\strut
\atentry{index}\marg{gatinho,
 \fieldfmt{translations-en}=\marg{kitten}
}
\strut
\atentry{index}\marg{tapete,
  \fieldfmt{translations-en}=\marg{carpet,rug,mat,tapestry}
}
\strut
\atentry{index}\marg{esteira,
  \field{prefix}=\marg{a},
  \fieldfmt{translations-en}=\marg{mat,track,matting,furrow}
}
\strut
\atentry{index}\marg{capacho,
  \field{prefix}=\marg{o},
  \fieldfmt{translations-en}=\marg{doormat,matting,mat,coconut-matting}
}
\end{codeenv}
The aim here is to have a document containing an 
English-to-Portuguese and a Portuguese-to-English dictionary. The
custom \fieldfmt{translations-pt} and \fieldfmt{translations-pt}
fields contain comma-separated lists of possible translations. In
this case I don't want to use the \field{see} field (and, in fact,
can't for the entries that have the \field{seealso} field set), but
I can identify the values of those fields as dependent entries to
ensure that they are selected even if they're not referenced in the
document.

For convenience I've aliased the custom fields to \field{user1}:
\begin{codeenv}
\cmd{documentclass}\marg{article}
\strut
\cmd{usepackage}[T1]\marg{fontenc}
\cmd{usepackage}[utf8]\marg{inputenc}
\cmd{usepackage}[british,brazilian]\marg{babel}
\cmd{usepackage}[colorlinks]\marg{hyperref}
\cmd{usepackage}\oarg{\styopt{record},
 \styopt{nomain},
 \styopt{nostyles},
 \styopt[bookindex]{stylemods},
 \styopt[\glostyle{bookindex}]{style}
}\marg{glossaries-extra}
\cmd{usepackage}\marg{\isty{glossaries-prefix}}
\strut
\cs{newglossary*}\marg{en}\marg{English Terms}
\cs{newglossary*}\marg{pt}\marg{Portuguese Terms}
\strut
\gls{GlsXtrLoadResources}\oarg{
  \csopt[en]{type},
  \csopt[entries-en]{src},
  \csopt[en-GB]{sort},
  \csopt[en]{category},
  \csopt[\fieldfmt{translations-pt}=\field{user1}]{field-aliases},
  \csopt[\field{user1}]{dependency-fields},
  \csopt[\field{user1}:pt-BR:\encap{glsentryname}]{sort-label-list}
}
\gls{GlsXtrLoadResources}\oarg{
  \csopt[pt]{type},
  \csopt[entries-pt]{src},
  \csopt[pt-BR]{sort},
  \csopt[pt]{category},
  \csopt[\fieldfmt{translations-en}=\field{user1}]{field-aliases},
  \csopt[\field{user1}]{dependency-fields},
  \csopt[\field{user1}:en-GB:\encap{glsentryname}]{sort-label-list}
}
\strut
\cs{apptoglossarypreamble}\oarg{en}\marg{\cs{selectlanguage}\marg{british}}
\cs{apptoglossarypreamble}\oarg{pt}\marg{\cs{selectlanguage}\marg{brazilian}}
\strut
\cmd{begin}\marg{document}
\cs{selectlanguage}\marg{british}
The \cs{gls}\marg{cat} sat on the \cs{gls}\marg{mat}.
\strut
\cs{selectlanguage}\marg{brazilian}
O \cs{gls}\marg{gato} sentou-se no \cs{gls}\marg{tapete}.
\strut
\cs{renewcommand}*\marg{\cs{glsxtrbookindexname}}[1]\marg{\comment{}
 \cs{glsxtrifhasfield}\marg{\field{prefix}}\marg{\idx{param}1}\marg{\ics{xmakefirstuc}\cs{glscurrentfieldvalue}\cs{cs.space}}\marg{}\comment{}
 \cs{glossentryname}\marg{\idx{param}1}\comment{}
 \cs{glsxtrifhasfield}\marg{\field{useri}}\marg{\idx{param}1}
 \marg{; translations: \cs{glsxtrseelist}\cs{glscurrentfieldvalue}}\marg{}\comment{}
}
\cs{printunsrtglossaries}
\cmd{end}\marg{document}
\end{codeenv}

\subsection{Special Fields}
\label{sec:specialfieldopts}

\optsection[\subsubsection]{save-original-id}

The \meta{value} may be the keywords \code{false} or \code{true} or
the name of a field in which to store the entry's original
label (as given in the \ext{bib} file). The default setting is
\csopt[false]{save-original-id}. If \meta{value} is omitted or is the
keyword \code{true}, then \field{originalid} is assumed.

If \meta{value} has an associated key in \gls!{newglossaryentry}
(for example, one provided with \cs{glsaddstoragekey}) it will be
set after the field aliases, otherwise (for example,
\field{originalid}) it will simply be added to the \ext{glstex} file
using \ics{GlsXtrSetField} after the entry definition (which means
the field can't be referenced in other resource options). This
setting is governed by \csopt{save-original-id-action}.

\optsection[\subsubsection]{save-original-id-action}

This option determines whether or not \csopt{save-original-id}
should save the original entry label. No action is performed when
\csopt[false]{save-original-id} otherwise the action is determined
by \meta{value} which may be one of the following keywords:
\begin{itemize}
\item \code{always}: always save the original label (default);
\item \code{no override}: don't override a field that's already been
set;
\item \code{changed override} or \code{changed} or \code{diff}:
only save the original label if it's different from the final label;
\item \code{changed no override}:
only save the original label if it's different from the final label
and the specified field hasn't been set. 
\end{itemize}
The \qt{no override} options make no difference if the given field
has no associated key in \gls!{newglossaryentry} (such as
\field{originalid}). For known fields, bear in mind that the field
will be set after field aliasing but before other options, such as
\csopt{ignore-fields}.

\optsection[\subsubsection]{save-definition-index}

This is a boolean option. If the value is omitted \code{true} is
assumed. The default setting is \csopt[false]{save-definition-index}.

This setting will save the \gls{definitionindex} that's used by
\csopt[def]{identical-sort-action} to determine the order of
definition in the special internal field \field{definitionindex}.
This field is assigned when the entry is first created
and can be referenced with \gls{bibglsdefinitionindex}.
You can reference this field with certain resource options, such as
\csopt{format-integer-fields}, but you must place the
\csopt{save-definition-index} resource option first.

Note that (unless you need to maintain \hierarchy) if you want to
order all entries by definition, it's better to use
\csopt[none]{sort}, which doesn't perform any sorting, so the order
will be by definition.

\optsection[\subsubsection]{save-use-index}

This is a boolean option. If the value is omitted \code{true} is
assumed. The default setting is \csopt[false]{save-use-index}.

This setting will save the \gls{orderofuseindex} that's used by
\csopt[use]{identical-sort-action} in the special internal field
\field{useindex}.  This field is assigned when the entry picks up
its first record and can be referenced with \gls{bibglsuseindex}.
You can't reference this field in resource options such as
\csopt{format-integer-fields}.

Entries that don't have records won't have this field set. The order
of use corresponds to the first time the entry is recorded in the document.

Note that (unless you need to maintain \hierarchy) if you want to
order all entries by use, it's better to use \csopt[use]{sort},
which doesn't perform any sorting.

\optsection[\subsubsection]{save-from-see}

This option allows you to save a comma-separated list of entry
labels in a designated internal field of the target entry identified
by their \field{see} field. If the \meta{value} is omitted,
\csopt[from-see]{save-from-see} is assumed. The value may be the
keyword \code{false}, which switches off this setting, otherwise the
value should be the desired name of the internal field.
The default setting is \csopt[false]{save-from-see}.

For example, if the \ext{bib} file contains:
\begin{codeenv}
\atentry{index}\marg{gourd}
\atentry{index}\marg{cucumber,\field{see}=\marg{gourd}}
\atentry{index}\marg{pumpkin,\field{see}=\marg{gourd}}
\end{codeenv}
then the resource option \csopt[from-see]{save-from-see} will create
an internal field called \fieldfmt{from-see} for the \code{gourd} entry
that contains the comma-separated list \code{cucumber,pumpkin}.

Note that the given internal field isn't actually assigned within \bibgls, so it
can't be accessed via any resource options. Each item in this list is added using
\ics{glsxtrapptocsvfield} after the source entry (that is, the entry
containing the \field{see} field) is defined in the \ext{glstex}
file. This means that the list will be in the same order as the
entries. You can then pass the field value to \ics{glsseelist}.
For example:
\begin{codeenv}
\cs{glsdefpostdesc}\marg{\comment{}
  \cs{glsxtrifhasfield}\marg{from-see}\marg{\cs{glscurrententrylabel}}
  \marg{, related: \cs{glsseelist}\marg{\cs{glscurrentfieldvalue}}}\marg{}\%
}
\end{codeenv}

This option has no effect with the \qt{no dependency} selection
criteria (such as \csopt[recorded no deps]{selection}).

\optsection[\subsubsection]{save-from-seealso}

As \csopt{save-from-see} but for the \field{seealso} field. If the
value is omitted, \csopt[from-seealso]{save-from-seealso} is assumed.

\optsection[\subsubsection]{save-from-alias}

As \csopt{save-from-see} but for the \field{alias} field. If the
value is omitted, \csopt[from-alias]{save-from-alias} is assumed.

\optsection[\subsubsection]{save-crossref-tail}

If you have a cross-reference trail where one entry references
another entry using \field{see}, \field{seealso} or \field{alias}
and the referenced entry also references another, and so on, then
you can save the tail end of the trail with this option. Note that
the trail only follows single-label lists (in \field{see} or
\field{seealso}). The trail is terminated if an entry doesn't have
one of those three fields set or if it cross-references multiple
entries or if the trail loops back on itself.

If you have a loop, the tail for some entries may end prematurely
since the algorithm to obtain the tail saves the tail for each
sub-trail to avoid recalculating it.  It's best to avoid
this setting if you have cross-reference loops. (Aside from two-way
cross-references, it's best to avoid loops in general.)

The tail label is stored in the field identified by 
the \meta{value} of this option. If the value is omitted, 
\csopt[crossref-tail]{save-crossref-tail} is assumed.
The field won't be set if there's no tail. The tails are calculated
when writing the entry definitions to the \ext{glstex} file so the
value can't be referenced or otherwise accessed by \bibgls.

Example:
\begin{codeenv}
\atentry{index}\marg{sample1,\field{see}=\marg{sample2}}
\atentry{index}\marg{sample2,\field{see}=\marg{sample3}}
\atentry{index}\marg{sample3,\field{see}=\marg{sample4}}
\atentry{index}\marg{sample4}
\end{codeenv}
The tail for \code{sample1} is \code{sample4}. As a by-product of
the recursion used in calculating the tail for \code{sample1}, the
tail for each element in the trail (\code{sample2} and
\code{sample3}) is also calculated.  The tail is the same for each
entry in the trail. The final entry \code{sample4} doesn't have a
tail.

If \code{sample4} is modified to cross-reference \code{sample1}:
\begin{codeenv}
\atentry{index}\marg{sample4,\field{see}=\marg{sample1}}
\end{codeenv}
then when the tail for \code{sample4} is calculated the tail for its
cross-reference (\code{sample1}) is consulted. This has already been
set to \code{sample4}. An entry can't have itself as a tail so the
tail for \code{sample4} is set to \code{sample3}. All the other
entries still have \code{sample4} as their tail because their tail
was determined while traversing the trail for \code{sample1}, which
had to stop when it wrapped round to its starting point.

\optsection[\subsubsection]{save-original-entrytype}

The \meta{value} may be the keywords \code{false} or \code{true} or
the name of a field in which to store the original \gls{entrytype} (as
given in the \ext{bib} file but without the leading \idx{atchar} and
converted to \idx!{lowercase}). The setting is
\csopt[false]{save-original-entrytype}. If \meta{value} is omitted or the
keyword \code{true}, then \csopt[originalentrytype]{save-original-entrytype} 
If \meta{value} has an associated key in \gls!{newglossaryentry}
(for example, one provided with \cs{glsaddstoragekey}) it will be
set after the field aliases, otherwise (for example,
\field{originalentrytype}) it will simply be added to the \ext{glstex} file
using \ics{GlsXtrSetField} after the entry definition (which means
the field can't be referenced in other resource options). This
setting is governed by \csopt{save-original-entrytype-action}.

\optsection[\subsubsection]{save-original-entrytype-action}

This option determines whether or not \csopt{save-original-entrytype}
should save the original \gls{entrytype}. No action is performed when
\csopt[false]{save-original-entrytype} otherwise the action is determined
by \meta{value} which may be one of the following keywords:
\begin{itemize}
\item \code{always}: always save the original \gls{entrytype} (default);
\item \code{no override}: don't override a field that's already been
set;
\item \code{changed override} or \code{changed} or \code{diff}:
only save the original \gls{entrytype} if it's different from the final
\gls{entrytype};
\item \code{changed no override}:
only save the original \gls{entrytype} if it's different from the final
\gls{entrytype} and the specified field hasn't been set. 
\end{itemize}
The \qt{no override} options make no difference if the given field
is unknown (such as \field{originalentrytype}). For known fields, bear in
mind that the field will be set after field aliasing but before
other options, such as \csopt{ignore-fields}.

The \qt{changed} options ignore case. For example, if the \ext{bib}
file defined an entry with \atentryfmt{INDEX} then both the original
and final \gls{entrytype} will be \code{index}.

\optsection[\subsubsection]{gather-parsed-dependencies}

Dependencies that are found by parsing field values may be gathered into a 
comma-separated list saved in a field for later use.
The \meta{value} is the name of the desired field. If the \meta{value} is omitted
\csopt[seealso]{gather-parsed-dependencies} is assumed.
If the designated field is already set, the list will be appended to the existing value.

Note that other dependencies, such as those obtained by examining the 
cross-reference fields (\field{see}, \field{seealso} or \field{alias})
or ancestors or dual entries, are not automatically added.
The \meta{value} may be the keyword \code{false} to switch off this option (which is the default).

\subsection{Assignments}
\label{sec:fieldassignments}

\optsection[\subsubsection]{group}

The \csopt{group} option will set the \field{group} field to \meta{label}
unless \meta{label} is \optfmt{auto}. If \csopt[auto]{group} then if
the \longarg{group} switch is used the value of the \field{group}
field is set automatically during the sorting (see also
\csopt{group-formation}, \csopt{group-level}
and \sectionref{sec:logicaldivisions}). If the \longarg{no-group}
setting is on then \csopt[auto]{group} does nothing.

The corresponding group title can be set with
\gls{glsxtrsetgrouptitle} in the document if the title is different
from the label. The default behaviour is \csopt[auto]{group}.

For example:
\begin{codeenv}
\gls{GlsXtrLoadResources}\oarg{\csopt[integer]{sort},\csopt[Constants]{group},
 \csopt[entries-constants]{src}\comment{data in entries-constants.bib}
}
\gls{GlsXtrLoadResources}\oarg{\csopt[letter-case]{sort},\csopt[Variables]{group},
 \csopt[entries-variables]{src}\comment{data in entries-variables.bib}
}
\end{codeenv}
In this case, if the \field{type} field hasn't been set in the \ext{bib} files,
these entries will be added to the same glossary, but will
be grouped according to each instance of \gls{GlsXtrLoadResources},
with the provided group label.


\optsection[\subsubsection]{category}

The selected entries may all have their \field{category} field
changed before writing their definitions to the \ext{glstex} file.
The \meta{value} may be:
\begin{itemize}
\item \optfmt{false}: switch off this setting (default);
\item \optfmt{same as entry}: set the
\field{category} to the \ext{bib} \gls{entrytype} used to define it
(\idx!{lowercase} and without the initial \code{@}) after any aliasing,
if applicable;

\item \optfmt{same as original entry}: (new to v1.4) set the \field{category} 
to the original \gls{entrytype} (\idx!{lowercase} and without 
the initial \code{@}) before it was aliased (behaves like 
\optfmt{same as entry} if the \gls{entrytype} wasn't aliased);

\item \optfmt{same as base}: (new to v1.1) set the \field{category}
to the base name of the \ext{bib} file (without the extension) 
that provided the entry definition;

\item \optfmt{same as type}: set the \field{category} to the same
value as the \field{type} field (if that field has been provided
either in the \ext{bib} file or through the \csopt{type} option);

\item \meta{label}: the \field{category} is set to
\meta{label} (which mustn't contain any special characters).
\end{itemize}
This will override any
\field{category} fields supplied in the \ext{bib} file.

When used with \csopt{entry-type-aliases}, the option \csopt[same as
entry]{category} refers to the \emph{target} \gls{entrytype} whereas
\csopt[same as original entry]{category} refers to the
\emph{original} \gls{entrytype} given in the \ext{bib} file.  In both
cases, the value is converted to \idx!{lowercase} to ensure consistency.
An alternative is to use \csopt[category]{save-original-entrytype}.
When combined with \csopt[changed]{save-original-entrytype-action}
it's then possible to only set the \field{category} to the original
\gls{entrytype} for aliased entries and leave it unmodified for unaliased
entries.

For example, if the \ext{bib} file contains:
\begin{codeenv}
\atentry{entry}\marg{bird,
  \field{name}=\marg{bird},
  \field{description} = \marg{feathered animal}
}
\strut
\atentry{index}\marg{duck}
\strut
\atentry{index}\marg{goose,\field{plural}=\qtdelim{geese}}
\strut
\atentry{dualentry}\marg{dog,
  \field{name}=\marg{dog},
  \field{description}=\marg{chien}
}
\end{codeenv}
then if the document contains:
\begin{codeenv}
\gls{GlsXtrLoadResources}\oarg{\csopt[same as entry]{category},\csopt[entries]{src}}
\end{codeenv}
this will set the \field{category} of the \code{bird} term to \optfmt{entry}
(since it was defined with \atentry{entry}), the \field{category} of the
\code{duck} and \code{goose} terms to \optfmt{index} (since they were defined
with \atentry{index}), and the \field{category} of the \code{dog} term to
\optfmt{dualentry} (since it was defined with \atentry{dualentry}). Note that
the dual entry \code{dual.dog} doesn't have the category set, since that's
governed by \csopt{dual-category} instead.

If, instead, the document contains:
\begin{codeenv}
\gls{GlsXtrLoadResources}\oarg{\csopt[animals]{category},\csopt[entries]{src}}
\end{codeenv}
then the \field{category} of all the \primary\ selected entries will 
be set to \optfmt{animals}. Again the \gls{dualentry} \code{dual.dog}
doesn't have the \field{category} set.

Note that the categories may be overridden by the commands that are used to
actually define the entries (such as \gls!{bibglsnewindex}).

For example, if the document contains:
\begin{codeenv}
\cs{newcommand}\marg{\gls{bibglsnewdualentry}}[4]\marg{\comment{}
 \cs{longnewglossaryentry*}\marg{\idx{param}1}\marg{\field{name}=\marg{\idx{param}3},\idx{param}2,\field{category}=\marg{dual}}\marg{\idx{param}4}\comment{}
}
\strut
\gls{GlsXtrLoadResources}\oarg{\csopt[animals]{category},\csopt[entries]{src}}
\end{codeenv}
then both the \code{dog} and \code{dual.dog} entries will
have their \field{category} field set to \optfmt{dual} since the
new definition of \gls{bibglsnewdualentry} has overridden
the \csopt[animals]{category} option.

\optsection[\subsubsection]{type}

The \meta{value} may be one of:
\begin{itemize}
 \item \optfmt{false}: switches off this setting (default);

 \item \optfmt{same as entry}: set the \field{type} field
  to the \gls{entrytype} (\idx!{lowercase} and without the initial \code{@}); 

 \item \optfmt{same as original entry}: set the \field{type} 
  to the original \gls{entrytype} (\idx!{lowercase} and without 
  the initial \code{@}) before it was aliased (behaves like 
  \optfmt{same as entry} if the \gls{entrytype} wasn't aliased);

 \optfmt{same as base}: set the \field{type} field
  to the base name of the corresponding \ext{bib} file
  (without the extension);

 \item \optfmt{same as category}: set the \field{type} field
   to the same value as the \field{category} field
   (\field{type} unchanged if \field{category} not set);

\item \optfmt{same as parent}: sets the \field{type} to the same as
the entry's \parent\ (new to v1.9). If the entry doesn't have a
\parent\ or if the \parent\ doesn't have the \field{type} field set, then
no change is made. Entries should always have the same type
as their \parent, but it's possible for spawned entries to 
pick up the \field{type} field from their \igls{progenitor} entry
(if it was explicitly set in the \ext{bib} file),
which may be inappropriate.

 \item\meta{label}: sets the \field{type} field to the glossary 
   identified by \meta{label}.
\end{itemize}
When used with \csopt{entry-type-aliases}, the option \csopt[same as
entry]{type} refers to the \emph{target} \gls{entrytype} and \csopt[same
as original entry]{type} refers to the \emph{original} \gls{entrytype}
given in the \ext{bib} file.  
An alternative is to use \csopt[type]{save-original-entrytype}.
When combined with \csopt[changed]{save-original-entrytype-action}
it's then possible to only set the \field{type} to the original
\gls{entrytype} for aliased entries and leave it unmodified for unaliased
entries.

\begin{important}
It's not possible to have both
\csopt[same as type]{category} and \csopt[same as category]{type}.
\end{important}

Note that this setting only changes the \field{type} field for
\glspl{primaryentry}. Use \csopt{dual-type} for \glspl{dualentry}.

For example:
\begin{codeenv}
\cmd{usepackage}[\styopt{record},\styopt{symbols}]\marg{glossaries-extra}
\strut
\gls{GlsXtrLoadResources}\oarg{\csopt[entries-symbols]{src},\csopt[symbols]{type}}
\end{codeenv}

Make sure that the glossary type has already been defined
(see \sectionref{sec:newglossary}). In the above, the
\styopt{symbols} option defines the \code{symbols} glossary.
If you want to use a custom glossary, you need to provide it. For
example:
\begin{codeenv}
\cmd{usepackage}[\styopt{record},\styopt{nomain}]\marg{glossaries-extra}
\strut
\cs{newglossary*}\marg{dictionary}\marg{Dictionary}
\strut
\gls{GlsXtrLoadResources}\oarg{\csopt[entries-symbols]{src},\csopt[dictionary]{type}}
\end{codeenv}
(The \styopt{nomain} option was added to suppress the
creation of the default \mainglossary.)

\optsection[\subsubsection]{ignored-type}

Any entry that only has \iglspl{ignoredrecord} will still be
identified as having a record for selection purposes, which is
necessary for the entry to be defined in the document, but it may be
preferable to move such entries to a special \igls{ignoredglossary}.
This can be done with \csopt[\meta{type}]{ignored-type}, where
\meta{type} is the label of the \igls{ignoredglossary}. 

The glossary will be provided with 
\ics{provideignoredglossary} to ensure that it's
defined even if \longarg{no-provide-glossaries} is set
(see \sectionref{sec:newglossary}).
Note that it uses the unstarred \ics{provideignoredglossary}
since it's assumed that it won't be needed in a list and
therefore won't have a target. This is different to all the other
settings that provide an \gls{ignoredglossary}, which use the
starred version instead.

This option is not implemented for entries that have \emph{no} records
(those entries may have been selected because they are dependent on
another entry, such as a parent of a recorded child entry). The entry 
must have at least one \gls{ignoredrecord} and 
no other type of record and not be dependent on other entries or be
cross-referenced by other entries.

Note that the \csopt{trigger-type} option, if set, overrides the
\csopt{ignored-type} option for entries that have records with the
special format \ics{glstriggerrecordformat} (which is also
considered an \gls{ignoredrecord}).  However, \csopt{ignored-type}
will override the \csopt{type}, \csopt{dual-type},
\csopt{tertiary-type} and the type specification in
\csopt{secondary}.

Ignored entries may be copied to another glossary
with \csopt{copy-to-glossary}. If this is undesirable, a condition
may be applied to prevent it. For example, to copy all selected
entries to the glossary labelled \qt{index} except for ignored
entries:
\begin{codeenv}
\gls{GlsXtrLoadResources}\oarg{
 \csopt[\marg{terms,abbreviations}]{src}
 \csopt[ignored]{ignored-type},
 \csopt[\qtdelim{index} \oarg{ \field{type} \idx{notequalscmp} \qtdelim{ignored} }]{copy-to-glossary}
}
\end{codeenv}

\optsection[\subsubsection]{trigger-type}

The record counting commands, such as \gls{rgls}, use the special
format \ics{glstriggerrecordformat}, which \bibgls\ also treats
as an \igls{ignoredrecord}. This means the entry will still be
identified as having a record for selection purposes, which is
necessary for the entry to be defined for use in the document, but
in order to prevent it from appearing in the glossary you need to
transfer the entry with \csopt[\meta{type}]{trigger-type}.
This will override the \csopt{type}, \csopt{dual-type},
\csopt{tertiary-type} and the type specification in
\csopt{secondary}.

The provided value \meta{type} must be a glossary label (not one of
the keywords allowed by \csopt{type}) or \code{false} to switch
off this setting.
You can define the glossary before loading the resource, but
it's not required as \bibgls\ will write
\code{\ics{provideignoredglossary*}\margm{type}} to the \ext{glstex} file
even if \longarg{no-provide-glossaries} is set
(see \sectionref{sec:newglossary}).

\optsection[\subsubsection]{progenitor-type}

This sets the default \field{type} field for the 
\glsdisp{mainentry}{main term} defined by 
\atentry{progenitor}-like entries. The \meta{value} is as for
\csopt{type}. This doesn't change the \field{type} for the spawned
\gls{progeny}.

\optsection[\subsubsection]{progeny-type}

This sets the default \field{type} field for the \gls{progeny} term
spawned by \atentry{progenitor}-like entries. The \meta{value} is as
for \csopt{type}. This doesn't change the \field{type} for the
\glsdisp{mainentry}{main} \gls{progenitor}. Remember that with the
default \csopt[parent]{adopted-parent-field} setting, the given type
should match the type of the \gls{parententry}.

\optsection[\subsubsection]{adopted-parent-field}

This identifies the target field to be set to the corresponding
value of the \field{adoptparents} list by the \gls{progeny}
entries spawned by the \atentry{progenitor} type of entry.
The default is \field{parent}.

\optsection[\subsubsection]{ignore-fields}

\begin{important}
Take care not to confuse \csopt{ignore-fields} with \csopt{omit-fields}.
The argument of \csopt{ignore-fields} is a simple list of field names.
\end{important}

The \csopt{ignore-fields} key indicates that you want \bibgls\ to
skip the fields listed in the supplied comma-separated \meta{list} of field
labels. Remember that unrecognised fields will always be skipped.
However, an unrecognised field can still be referenced with some
options (such as \csopt{replicate-fields}) whereas any field
excluded with \csopt{ignore-fields} will be discarded and can't be
referenced.

This setting is always implemented after \csopt{field-aliases} (see
\sectionref{sec:resourcesets}). If a field has been aliased then the
original field name is no longer present and so ignoring it will
have no effect.

For example, suppose my \ext{bib} file contains:
\begin{codeenv}
\atentry{abbreviation}\marg{html,
  \field{short} =\qtdelim{html},
  \field{long}  = \marg{hypertext markup language},
  \field{description}=\marg{a markup language for creating web pages},
  \field{seealso}=\marg{xml}
}
\end{codeenv}
but I want to use the \abbrstyle{short-long} style and I don't want
the cross-referenced term, then I can use
\csopt[seealso,description]{ignore-fields}.

Note that \csopt[parent]{ignore-fields} removes the \field{parent}
before determining the dependency lists. This means that
\csopt[recorded and deps]{selection} and 
\csopt[recorded and ancestors]{selection} won't pick up the
label in the \field{parent} field.

If you want to maintain the dependency and \gls{ancestor} relationship but
omit the \field{parent} field when writing the entries to the
\ext{glstex} file, you need to use \csopt{flatten} instead.
Alternatively, you can use \csopt{omit-fields}, however the hierarchical structure
will continue to be maintained.

The \csopt{ignore-fields} instruction removes the unwanted fields 
early in the Stage~3 process (see \sectionref{sec:resourcesets}).
This means that those field values won't be available if they are referenced
nor can they be used to establish dependencies.
If fields should be omitted from the \ext{glstex} file but the field values should
still be available to \bibgls, then use \csopt{omit-fields} instead.

\optsection[\subsubsection]{omit-fields}

\begin{important}
Take care not to confuse \csopt{omit-fields} with \csopt{ignore-fields}.
The argument of \csopt{omit-fields} is a complex list of string \glspl{concatenation}.
\end{important}

The \csopt{omit-fields} action takes place in the final stage, at the point where 
each entry is written to the \ext{glstex} file. Unlike \csopt{ignore-fields},
the fields aren't removed from \bibgls's own internal data structure. This means that the fields
can be referenced in other options and will be parsed as usual for dependencies.

If set, the \meta{list} argument is a list of string \glspl{concatenation} with
optional conditionals. Take care that constant strings are correctly delimited
to ensure that they are not mistaken for references to field values.
Note that this is different from \csopt{ignore-fields}, which simply requires 
a list of field names.
For example, to omit the \field{description} field for all entries:
\begin{codeenv}
\csopt[\qtdelim{description}]{omit-fields}
\end{codeenv}

Each identified field is simply omitted when writing the field list in the \ext{glstex} file.
However, this may trigger a fallback if the field is required. This option only
applies to fields that have a corresponding key that may be used in commands such
as \gls{newglossaryentry}. The option does not apply to special internal fields.
Any fields identified in the given list that are not recognised will be ignored.

When each entry definition is being written to the \ext{glstex} file, the supplied
\meta{list} is evaluated for the current entry. Each non-empty non-null result
will be added to a temporary set of exclusion field names. Then each \keyval\ for the current
entry will be written to the \ext{glstex} for each known \meta{key} that has the
corresponding field set where the \meta{key} is not included in the exclusion set.

Suppose the file \filefmt{abbrv.bib} contains:
\begin{codeenv}
\atentry{abbreviation}\marg{ssi,
 \field{short}=\marg{SSI},
 \field{long}=\marg{server-side includes},
 \field{description}=\marg{a simple interpreted server-side
scripting language}
}
\atentry{abbreviation}\marg{html,
 \field{short}=\marg{HTML},
 \field{long}=\marg{hypertext markup language},
 \field{description}=\marg{a markup language for creating web
pages}
}
\atentry{abbreviation}\marg{shtml,
 \field{short}=\marg{SHTML},
 \field{long}=\marg{server-side includes enabled hypertext markup language},
 \field{description}=\marg{a combination of \cs{gls}\marg{html} and
\cs{gls}\marg{ssi}}
}
\end{codeenv}
With \csopt[description]{ignore-fields}, the \field{description} field will
be removed when the entry is first processed. This means that the
dependencies in the SHTML description field won't be detected.
Whereas with \csopt[\qtdelim{description}]{omit-fields}, the \field{description}
field won't be removed, so it will still be parsed and the dependent entries will 
be detected, however the \field{description} field won't be written in the
\ext{glstex} file. In this case, you may want to also use 
\csopt{gather-parsed-dependencies} to save a list of the dependent entries.

Remember that literal strings must be quoted with string \glspl{concatenation}.
For example, suppose an entry has been defined as:
\begin{codeenv}
\atentry{index}\marg{nom-fr,
 \field{name}=\marg{nom},
 \field{description}=\marg{name}
}
\end{codeenv}
In the case of \csopt[\qtdelim{description}]{omit-fields} then this will become:
\begin{codeenv}
\gls{newglossaryentry}\marg{name-fr}\marg{
 \field{name}=\marg{nom}
}
\end{codeenv} 
Whereas \csopt[description]{omit-fields} returns \emph{the value of} the 
\field{description} field. For the above example entry, the value of
the \field{description} field is \qt{name}, so \field{name} is added to the exclusion set
and so won't be written whilst \bibgls\ iterates over the \qt{name-fr} list of fields.

This means that the \field{name} field will be flagged
as not written to the \ext{glstex} file. In the case of \atentry{index}, this
means that the fallback for the \field{name} field will be used, which is
the entry label. The result will therefore end up as:
\begin{codeenv}
\gls{newglossaryentry}\marg{name-fr}\marg{
 \field{description}=\marg{name},
 \field{name}=\marg{name-fr}
}
\end{codeenv} 
In this case, the \field{description} field value just happens to be the name of
another recognised field. Mostly, this type of error would likely result
in a string that doesn't match any known field, which will trigger a warning.

Conditions may also be applied. For example, to omit the \field{description} 
field for any entries defined with \atentry{abbreviation}:
\begin{codeenv}
\csopt[
 \qtdelim{description} \oarg{entrytype \idx{follow} actual = \qtdelim{abbreviation}}
]{omit-fields}
\end{codeenv}

\optsection{omit-fields-missing-field-action}

This option indicates what to do if a source field identified in 
\csopt{omit-fields} is missing. The value may be one of:
\begin{itemize}
\item \optfmt{skip}: return null;
\item \optfmt{fallback}: use the fallback for the missing field 
(see \sectionref{sec:fallbacks}), if
one is available, otherwise return null (default);
\item \optfmt{empty}: treat the missing value as empty.
\end{itemize}
Returning null or empty skips the string \gls{concatenation} element from the
omision list.

\optsection[\subsubsection]{field-aliases}

You can instruct \bibgls\ to treat one field as though it
was another using this option. The value should be a comma-separated
list of \meta{field1}\dequals\meta{field2} pairs, where
\meta{field1} and \meta{field2} are field names. Identical
mappings and trails aren't permitted. (That is, \meta{field1}
and \meta{field2} can't be the same nor can you have both
\meta{field1}\dequals\meta{field2} and
\meta{field2}\dequals\meta{field3}.) If you want to swap
fields you need to use one of the dual \glspl{entrytype} instead.
Field aliases are performed before \csopt{ignore-fields},
so if \meta{field1} is listed in \csopt{ignore-fields} it won't
be ignored (unless \meta{field2} is in \csopt{ignore-fields}).

For example, suppose \filefmt{people.bib} contains:
\begin{codeenv}
\atentry{entry}\marg{alexander,
  \field{name}=\marg{Alexander III of Macedon},
  \field{description}=\marg{Ancient Greek king of Macedon},
  \fieldfmt{born}=\marg{20 July 356 BC},
  \fieldfmt{died}=\marg{10 June 323 BC},
  \fieldfmt{othername}=\marg{Alexander the Great}
}
\end{codeenv}
This contains three non-standard fields: \fieldfmt{born},
\fieldfmt{died} and \fieldfmt{othername}. I could define
these fields using \ics{glsaddkey}, but another possibility
is to map these onto the user keys \field{user1}, \field{user2}
and \field{user3}, which saves the overhead of providing new 
keys:
\begin{codeenv}
\gls{GlsXtrLoadResources}\oarg{
  \csopt[people]{src},\comment{data in people.bib}
  \csopt[\fieldfmt{born}=\field{user1},\fieldfmt{died}=\field{user2},\fieldfmt{othername}=\field{user3}]{field-aliases}
}
\end{codeenv}

\optsection[\subsubsection]{replicate-fields}

\begin{important}
Note the difference in syntax between \csopt{replicate-fields} and
\csopt{assign-fields}. Both have a \keyvallist\ as the option
argument, but the \keyval\ syntax is different. In the case of
\csopt{replicate-fields}, the left hand side (\meta{key}) is the
\emph{source} field. The right hand side (\meta{value}) is a
comma-separated list of \emph{destination} fields. The value of the
source field will be copied into each of the destination fields.
In the case of \csopt{assign-fields}, the left hand side
(\meta{key}) is the \emph{destination} field and the right hand side
\emph{value} is an assignment expression with an optional
conditional.
\end{important}

The value of one field can be copied to other fields using
this option where each \keyval\ pair is in the form
\code{\meta{field1}\dequals\marg{\meta{field2},\meta{field3},\ldots}}
where all values are field names. The value is required for this key
but may be empty, which indicates that the setting is switched off.

This option copies the contents
of \meta{field1} to \meta{field2}, \meta{field3}, \ldots\
(but only if the target field isn't already set with
\csopt[false]{replicate-override}). This action is
performed after \csopt{ignore-fields} (see
\sectionref{sec:resourcesets}). If the source field is missing, the
\csopt{replicate-missing-field-action} setting determines the
action.

If the target field doesn't have an associated key recognised by
\gls{newglossaryentry} then the value will be saved using
\cs{GlsXtrSetField}.  Special internal fields aren't permitted as
either source or target fields.

For example, suppose \filefmt{people.bib} contains:
\begin{codeenv}
\atentry{entry}\marg{alexander,
  \field{name}=\marg{Alexander III of Macedon (Alexander the Great)},
  \field{text}=\marg{Alexander},
  \field{description}=\marg{Ancient Greek king of Macedon}
}
\end{codeenv}
Since the \field{first} field hasn't been supplied, it
will default to the value of the \field{text} field, but
perhaps for one of my documents I'd like the \field{first}
field to be the same as the \field{name} field. Rather than
editing the \ext{bib} file, I can just do:
\begin{codeenv}
\gls{GlsXtrLoadResources}\oarg{
  \csopt[people]{src},\comment{data in people.bib}
  \csopt[\field{name}=\field{first}]{replicate-fields}
}
\end{codeenv}
This copies the contents of the \field{name} field into the
\field{first} field. If you have more than one field
in the list take care to brace the lists to avoid confusion.
For example, if for some reason I want to copy the value
of the \field{name} field to both \field{first} and
\field{firstplural} and copy the value of the \field{text}
field to the \field{plural} field, then this requires braces
for the inner list:
\begin{codeenv}
\gls{GlsXtrLoadResources}\oarg{
  \csopt[people]{src},\comment{data in people.bib}
  \csopt[\field{name}=\marg{\field{first},\field{firstplural}},\field{text}=\field{plural}]{replicate-fields}
}
\end{codeenv}

If my \filefmt{people.bib} file instead contained:
\begin{codeenv}
\atentry{entry}\marg{alexander,
  \field{name}=\marg{Alexander III of Macedon (Alexander the Great)},
  \field{first}=\marg{Alexander the Great},
  \field{text}=\marg{Alexander},
  \field{description}=\marg{Ancient Greek king of Macedon}
}
\end{codeenv}
then:
\begin{codeenv}
\gls{GlsXtrLoadResources}\oarg{
  \csopt[people]{src},\comment{data in people.bib}
  \csopt[\field{name}=\field{first}]{replicate-fields}
}
\end{codeenv}
won't alter the \field{first} field since \csopt{replicate-fields}
doesn't override existing values by default. You can use 
\csopt{replicate-override} to change this. Alternatively, since 
\csopt{replicate-fields} is always performed after \csopt{ignore-fields} 
it's possible to ignore the \field{first} field which means that the 
\field{name} value can then be copied into it:
\begin{codeenv}
\gls{GlsXtrLoadResources}\oarg{
  \csopt[people]{src},\comment{data in people.bib}
  \csopt[\field{first}]{ignore-fields},
  \csopt[\field{name}=\field{first}]{replicate-fields}
}
\end{codeenv}
Note that the ordering within the resource options doesn't
make a difference. The same result occurs with:
\begin{codeenv}
\gls{GlsXtrLoadResources}\oarg{
  \csopt[people]{src},\comment{data in people.bib}
  \csopt[\field{name}=\field{first}]{replicate-fields},
  \csopt[\field{first}]{ignore-fields}
}
\end{codeenv}

\optsection[\subsubsection]{replicate-override}

This is a boolean option. The default setting is
\csopt[false]{replicate-override}. If \optfmt{true},
\csopt{replicate-fields} will override the existing value if the
target field is already set.

\optsection[\subsubsection]{replicate-missing-field-action}

This option indicates what to do if a source field identified in 
\csopt{replicate-fields} is missing. The value may be one of:
\begin{itemize}
\item \optfmt{skip}: skip the replication of the missing field
(default);
\item \optfmt{fallback}: use the fallback for the missing field
(see \sectionref{sec:fallbacks}), if
one is available (otherwise skip);
\item \optfmt{empty}: make the target field empty.
\end{itemize}

\optsection[\subsubsection]{assign-fields}

\begin{important}
Note the difference in syntax between \csopt{replicate-fields} and
\csopt{assign-fields}, as highlighted in the
\csopt{replicate-fields} section, above. The \csopt{assign-fields}
option is implemented after the \csopt{replicate-fields} option (see
\sectionref{sec:resourcesets}).
\end{important}

The \csopt{assign-fields} option is a more complicated way of setting a
field than \csopt{replicate-fields}. Each \keyval\ element of the
\keyvallist\ argument has the syntax:
\begin{codeenv}
\meta{dest-field} \dequals\oargm{override} \meta{element-list} \oargm{condition}
\end{codeenv}
If the destination field (\meta{dest-field}) is already set, it will
only be overwritten if \csopt[true]{assign-override} or if
\oarg{override} is \qtt{o}. The \meta{dest-field} is simply the name
of the field for the entry under consideration and doesn't use the
more complex \meta{field-ref} syntax used in \meta{element-list},
which is described in \sectionref{sec:optstringconcat}.
You can, however, use the \cs{uhex} quark on either side of the
\keyval\ element to indicate a Unicode character.

The \oargm{override} following the equal sign is optional and may be
used to counteract the \csopt{assign-override} setting for the given
assignment. The \meta{override} value may be either \qtt{o}
(override) or \qtt{n} (no override). If not present, the
\csopt{assign-override} setting will be used.

The \meta{element-list} is a string \gls{concatenation}, as
described in \sectionref{sec:optstringconcat}.  If any individual
element in the list evaluates to null, the entire string is deemed
to be null, in which case the assignment won't be made.

The \oargm{condition} part is optional. If present, the assignment
is only made if the condition evaluates to true. The condition
should be placed in square brackets after the \meta{element-list}
part. This is a complex conditional, as described in
\sectionref{sec:conditionals}.

Note that, unlike most \keyval\ options, the value part
(\meta{element-list} \oargm{condition}) should not be grouped. The
\csopt{assign-fields} option is parsed in a different way to the
other \keyvallist\ options. However, it's best to group the \emph{entire}
\meta{value} argument of \csopt{assign-fields}. For example:
\begin{codeenv}
\csopt[
\field{name} \idx{equalsassign} \field{text} \idx{concat-plus} \qtdelim{, } \idx{concat-plus} \field{symbol}
]{assign-fields}
\end{codeenv}
Don't do \code{\field{name} \idx{equalsassign} \marg{\field{text}
\idx{concat-plus} \qtdelim{, } \idx{concat-plus} \field{symbol}}}.

\begin{important}
Remember that field values may be altered before or after
\csopt{assign-fields} by other resource options (see
\sectionref{sec:resourcesets}). The assignment will use the value
current at the time it is referenced during the processing of
\csopt{assign-fields}.  If you need to reference the destination
field in the assignment, make sure that the override setting is on
if the field needs to be updated.
\end{important}

For example, suppose I have defined the custom fields
\fieldfmt{surname} and \fieldfmt{forename}, and I have the following
in my \ext{bib} file:
\begin{codeenv}
\atentry{index}\marg{Smith}
\atentry{index}\marg{Jane-Smith,
 \fieldfmt{forename}=\marg{Jane},
 \field{parent}=\marg{Smith}
}
\end{codeenv}
Suppose that what I actually want is:
\begin{codeenv}
\atentry{index}\marg{Smith}
\atentry{index}\marg{Jane-Smith,
 \fieldfmt{forename}=\marg{Jane},
 \fieldfmt{surname}=\marg{Smith},
 \field{parent}=\marg{Smith},
 \field{name}=\marg{Jane},
 \field{text}=\marg{Jane Smith}
}
\end{codeenv}
This is quite repetitive to type out for every person you need to
index. The \csopt{replicate-fields} option can reduce some of this.
For example:
\begin{codeenv}
\csopt{replicate-fields}=\marg{
  \fieldfmt{forename}=\marg{name},
  \fieldfmt{surname}=\marg{parent}
}
\end{codeenv}
This doesn't deal with the \field{text} field and also has a problem
if the \field{parent} field (which should contain a label) doesn't
match the surname. For example, I might also have:
\begin{codeenv}
\atentry{index}\marg{de-la-Fontaine,
 \field{name}=\marg{de la Fontaine}
}
\atentry{index}\marg{Margaret-de-la-Fontaine,
 \fieldfmt{forename}=\marg{Margaret},
 \field{parent}=\marg{de-la-Fontaine}
}
\end{codeenv}
In this case the custom \fieldfmt{surname} field needs to match the
parent's \field{name} field, not the parent's label.

The desired result can instead be obtained with:
\begin{codeenv}
\csopt{assign-fields}=\marg{
  \fieldfmt{surname} = parent \idx{follow} \field{name},
  \field{name} = self \idx{follow} \fieldfmt{forename},
  \field{text} = self \idx{follow} \fieldfmt{forename}
    \idx{concat-plus} \qtdelim{ } \idx{concat-plus} self \idx{follow} \fieldfmt{surname}
}
\end{codeenv}
The \code{self \idx{follow}} part is optional so this can be written
more compactly as:
\begin{codeenv}
\csopt{assign-fields}=\marg{
  \fieldfmt{surname} = parent \idx{follow} \field{name},
  \field{name} = \fieldfmt{forename},
  \field{text} = \fieldfmt{forename} \idx{concat-plus} \qtdelim{ } \idx{concat-plus} \fieldfmt{surname}
}
\end{codeenv}
The last assignment in the above can also be written as:
\begin{codeenv}
\field{text} = \fieldfmt{forename} \idx{concat-plus} \marg{ } \idx{concat-plus} \fieldfmt{surname}
\end{codeenv}
Suppose, for some reason, I want the first use to show the surname
in bold. This means I need to add
\code{\cs{textbf}\idx{bgroupchar}} before the value of the
surname field and the closing \idx{egroupchar} needs to go after.
This can be achieved with:
\begin{codeenv}
\field{first} = \fieldfmt{forename} \idx{concat-plus} \qtdelim{ \cs{textbf}\idx{bgroupchar}} \idx{concat-plus} \fieldfmt{surname} \qtdelim{\idx{egroupchar}}
\end{codeenv}
Note that because there are unbalanced braces in the string
fragments, it's necessary to use quote delimiters. Since \cs{textbf}
is robust, there's no need to protect it from expansion.

Suppose, instead, I want the surname in \idx{uppercase} on first
use. I could simply replace \cs{textbf} with \cs{MakeUppercase}, but
I can get \bibgls\ to do the case-conversion instead:
\begin{codeenv}
\field{first} = \fieldfmt{forename} \idx{concat-plus} \qtdelim{ } \idx{concat-plus} \gls{UC}\marg{ \fieldfmt{surname} }
\end{codeenv}
This assumes that \ics{GlsXtrResourceInitEscSequences} has been
added to the definition of \cs{glsxtrresourceinit}, as described in
\sectionref{sec:quarks}. Otherwise you would need to protect \gls{UC}.

In the above example, the \fieldfmt{surname} field is obtained from
the value of the parent's \field{name} according to the assignment:
\begin{codeenv}
\fieldfmt{surname} = parent \idx{follow} \field{name},
\end{codeenv}
In the case of the Smith entry, the \field{name} field hasn't been
set.

If a referenced field hasn't been set then the behaviour depends on
the \csopt{assign-missing-field-action} setting. The default
behaviour is to use the fallback, if provided (see
\sectionref{sec:fallbacks}). In the case of \atentry{index}, the
fallback for the \field{name} field is the entry's label. This means
that the Jane-Smith entry will have the \fieldfmt{surname} field set
to \qt{Smith}.

If the fallback isn't set or if there is no fallback then the
assignment instruction will be skipped. Similarly, if an ancestor is
referenced but doesn't exist then assignment will again be skipped.

\begin{important}
The ancestor entries must be defined first to ensure that they have
been processed and have had their fields set before they can be
referenced in an assignment.
\end{important}

Since I haven't imposed any conditions on the assignments, the
instructions will be attempted on all entries. This includes the
parent entries.

The first instruction attempts to set the \fieldfmt{surname} field
to the parent's \field{name}. Neither the Smith nor the
de-la-Fontaine entries have a parent entry, so this instruction is
skipped for both of them.

The second instruction attempts to set the \field{name} field to the
entry's \fieldfmt{forename} field. The de-la-Fontaine entry already
has the \field{name} field set so the instruction is automatically
skipped (because of the default \csopt[false]{assign-override}).
The Smith entry doesn't have the \field{name} field set so the
assignment will be attempted but will fail because the
\fieldfmt{forename} field isn't set and doesn't have a fallback.

The \field{text} (and \field{first}) instruction similarly
references the \fieldfmt{forename} field, which isn't set, so the
instruction is skipped. The instruction also contains a reference to
the \fieldfmt{surname} field, but that part of the assignment isn't
reached as the attempt stops at the first unset reference.

This means that neither the Smith not the de-la-Fontaine entries
will be affected by the above \csopt{assign-fields} setting.

Each instruction will be attempted, in turn, unless the \csopt{assign-override}
setting prevents it. This means it's possible to have multiple
assignments for a particular field. The first one that succeeds will
be the one that has effect (with \csopt[false]{assign-override}).
For example:
\begin{codeenv}
\csopt{assign-fields}=\marg{
  \fieldfmt{surname} = parent \idx{follow} \field{name},
  \fieldfmt{surname} = \field{name},
  \field{name} = \fieldfmt{forename},
  \field{text} = \fieldfmt{forename} \idx{concat-plus} \qtdelim{ } \idx{concat-plus} \fieldfmt{surname}
}
\end{codeenv}
This has two instructions for the \fieldfmt{surname}. The first one
will be applied to entries that have a parent and the second one
will be applied to the other entries (that don't already have the
\fieldfmt{surname} set).

Suppose I also have some monarchs, who are more typically only
referred to by their forename (with no surname), possibly prefixed by their rank and
suffixed by their ordinal. Let's further suppose that in my document
I have also defined the custom fields \fieldfmt{rank} and
\fieldfmt{ordinal}, and in my \ext{bib} file I have:
\begin{codeenv}
\atentry{indexplural}\marg{king}
\atentry{indexplural}\marg{queen}
\atentry{indexplural}\marg{empress,\field{plural}=\marg{empresses}}
\strut
\atentry{index}\marg{King-Stephen,
  \field{parent}=\marg{king},
  \fieldfmt{forename}=\marg{Stephen}
}
\atentry{index}\marg{Empress-Matilda,
  \field{parent}=\marg{empress},
  \fieldfmt{forename}=\marg{Matilda},
}
\atentry{index}\marg{Elizabeth-I,
  \field{parent}=\marg{queen},
  \fieldfmt{forename}={Elizabeth},
  \fieldfmt{ordinal}={I}
}
\end{codeenv}
The earlier assignment rules won't be appropriate for these cases,
as they would result in the \field{text} fields: \qt{Stephen kings},
\qt{Matilda empresses} and \qt{Elizabeth queens}.

In this case, the assignment rules need to depend on the type of
entry. I could test if the parent label is \qt{king} or \qt{empress}
or \qt{queen}, but a more reliable approach is to have custom entry
types which can be aliased:

\begin{codeenv}
\atentry{index}\marg{Smith}
\atentryfmt{person}\marg{Jane-Smith,
 \fieldfmt{forename}=\marg{Jane},
 \field{parent}=\marg{Smith}
}
\atentry{index}\marg{de-la-Fontaine,
 \field{name}=\marg{de la Fontaine}
}
\atentryfmt{person}\marg{Margaret-de-la-Fontaine,
 \fieldfmt{forename}=\marg{Margaret},
 \field{parent}=\marg{de-la-Fontaine}
}
\strut
\atentry{indexplural}\marg{king}
\atentry{indexplural}\marg{queen}
\atentry{indexplural}\marg{empress,\field{plural}=\marg{empresses}}
\strut
\atentryfmt{monarch}\marg{King-Stephen,
  \field{parent}=\marg{king},
  \fieldfmt{forename}=\marg{Stephen}
}
\atentryfmt{monarch}\marg{Empress-Matilda,
  \field{parent}=\marg{empress},
  \fieldfmt{forename}=\marg{Matilda},
}
\atentryfmt{monarch}\marg{Elizabeth-I,
  \field{parent}=\marg{queen},
  \fieldfmt{forename}={Elizabeth},
  \fieldfmt{ordinal}={I}
}
\end{codeenv}
The resource options are now:
\begin{codeenv}
\csopt[person=index,monarch=index]{entry-type-aliases},
\csopt[
  \field{name} = \fieldfmt{forename} \idx{concat-plus} \qt{\idx{nbspchar}} \fieldfmt{ordinal},
  \field{name} = \fieldfmt{forename},
  \fieldfmt{surname} = parent \idx{follow} \field{name}
      \oarg{ entrytype \idx{follow} original \idx{equalscmp} \qtdelim{person} },
  \field{text} = \fieldfmt{forename} \idx{concat-plus} \qtdelim{ } \idx{concat-plus} \fieldfmt{surname}
      \oarg{ entrytype \idx{follow} original \idx{equalscmp} \qtdelim{person} },
  \field{first} = \gls{FIRSTUC} \marg{ parent \idx{follow} \field{text} } \idx{concat-plus} \qtdelim{ } \idx{concat-plus} \fieldfmt{name}
      \oarg{ entrytype \idx{follow} original \idx{equalscmp} \qtdelim{monarch} },
  \field{text} = \fieldfmt{name}
      \oarg{ entrytype \idx{follow} original \idx{equalscmp} \qtdelim{monarch} }
]{assign-fields}
\end{codeenv}

Additional fields can be added to accommodate nicknames or other
forms of address, which will add to the complexity of the assignment
specification.

\optsection[\subsubsection]{assign-override}

This is a boolean option. The default setting is
\csopt[false]{assign-override}. If \optfmt{true},
\csopt{assign-fields} will override the existing value if the
target field is already set.

\optsection[\subsubsection]{assign-missing-field-action}

This option indicates what to do if a source field identified in 
\csopt{assign-fields} is missing. The value may be one of:
\begin{itemize}
\item \optfmt{skip}: skip the assignment;
\item \optfmt{fallback}: use the fallback for the missing field
(see \sectionref{sec:fallbacks}), if
one is available, otherwise skip the assignment (default);
\item \optfmt{empty}: treat the missing value as empty.
\end{itemize}
Return null will result in the assignment instruction being omitted.

\optsection[\subsubsection]{counter}

The \csopt{counter} option assigns the default counter to use
for the selected entries. (This can be overridden with the
\glsopt{counter} key when using commands like \cs{gls}.)
The value must be the name of a counter. Since \bibgls\ doesn't know
which counters are defined within the document, there's no check to
determine if the value is valid (except for ensuring that
\meta{value} is non-empty).
A value of \code{false} will switch off this setting (the default).

Note that this will require an extra \LaTeX\ and \bibgls\ call since
the counter can't be used for the indexing until the entry has been
defined.

\optsection[\subsubsection]{copy-action-group-field}

This option may only be used when invoking \bibgls\ with the
\longarg{group} (or \shortargfmt{g}) switch. If an action
other than the default \csopt[define]{action} is set,
this option can be used to identify a field in which to save
the letter group information 
where \meta{value} is the name of the field. This just uses
\cs{GlsXtrSetField}. You will need to redefine
\ics{glsxtrgroupfield} to \meta{value} before displaying the glossary.
For example, if \csopt[dupgroup]{copy-action-group-field},
\csopt[copy]{action} and \csopt[copies]{type} are set in
the resource options and \code{copies} identifies a custom
glossary:
\begin{codeenv}
\ics{printunsrtglossary*}\oarg{\printglossopt[copies]{type},\printglossopt[\glostyle{indexgroup}]{style}}
 \marg{\cs{renewcommand}\marg{\cs{glsxtrgroupfield}}\marg{dupgroup}}
\end{codeenv}

This option is ignored when used with \csopt[define]{action}.
This option is not used by \csopt{secondary} which will
always save the group information in the \field{secondarygroup}
field. When used with \csopt[define or copy]{action}, entries
that are defined will have both \field{group} and
the field given by \csopt{copy-action-group-field} set.

Note that you may do \csopt[group]{copy-action-group-field} which
will override the \field{group} field from the original definition.
This may be useful if you don't use grouping in the 
\gls{primaryglossary}. That is, you use \code{nogroupskip} and a non-group
style. For example:
\begin{codeenv}
\cs{printunsrtglossary}\oarg{\printglossopt{nogroupskip},\printglossopt[\glostyle{index}]{style}}
\cs{printunsrtglossary}\oarg{\printglossopt[copies]{type},\printglossopt[indexgroup]{style}}
\end{codeenv}

\optsection[\subsubsection]{copy-alias-to-see}

If set, the value of the \field{alias} field is copied to
the \field{see} field. The default setting is \csopt[false]{copy-alias-to-see}.

\subsection{Field Adjustments}
\label{sec:fieldmods}

\optsection[\subsubsection]{post-description-dot}

The \styopt{postdot} package option (or \styopt[false]{nopostdot})
can be used to append a \idx{full-stop} (\idx{periodchar})\ to the
end of all the descriptions. This can be awkward if some of the
descriptions end with punctuation characters. This resource option
can be used instead. The \meta{value} may be one of:
\begin{itemize}
\item\optfmt{none}: don't append a \idx{full-stop} (default);
\item\optfmt{all}: append a \idx{full-stop} to all \field{description}
  fields in this resource set;
\item\optfmt{check}: selectively append a \idx{full-stop} (see below).
\end{itemize}
Note that if you have dual entries and you use this option to
append a \idx{full-stop}, then it will be copied over to the mapped field.
This is different to the \styopt{postdot} option which doesn't
add the dot to the field but incorporates it in the 
\idx{postdescriptionhook}. This means that a dot inserted with
\csopt{post-description-dot} will come before the
\idx{postdescriptionhook} whereas with \styopt{postdot} the
punctuation comes after any category-specific hook.

The \csopt[check]{post-description-dot} setting determines whether
to append the dot as follows:
\begin{itemize}
\item If the \field{description} field ends with
\ics{nopostdesc} or \ics{glsxtrnopostpunc}, then a dot isn't appended.
\item If the \field{description} field doesn't end with a regular
(ungrouped letter or other) character, then a dot is appended.
(For example, if the description ends with a control sequence
or an end group token.)
\item If the \field{description} field ends with a character 
that belongs to the Unicode category \idx{punctuationclose}
or \idx{punctuationfinalquote} then the token preceding
that character is checked.
\item If the \field{description} field doesn't end with
a character that belongs to the Unicode category
\idx{punctuationother} then the dot is added.
\end{itemize}
Note that the interpreter isn't used during the check.
If the \field{description} ends with a command then a dot will be
appended (unless it's \cs{glsxtrnopostpunc} or \cs{nopostdesc}) even if
that command expands in such a way that it ends with a terminating
punctuation character. This option only applies to the
\field{description} field.

\optsection[\subsubsection]{strip-trailing-nopost}

This option is always performed before \csopt{post-description-dot}. 
The default setting is \csopt[false]{strip-trailing-nopost}. If \optfmt{true}
any trailing ungrouped \ics{nopostdesc} or \ics{glsxtrnopostpunc} found in the
\field{description} field will be removed.  Note that the command (possibly
followed by ignored space) must be at the very end of the description for it to
be removed. A description should not contain both commands. This option only
applies to the \field{description} field.

For example, \cs{nopostdesc} will be stripped from:
\begin{codeenv}
\field{description}=\marg{sample\cs{nopostdesc}}
\end{codeenv}
since it's at the end. It will also be stripped from:
\begin{codeenv}
\field{description}=\marg{sample\cs{nopostdesc} }
\end{codeenv}
since the trailing space is ignored as it follows a control
word. It won't be stripped from:
\begin{codeenv}
\field{description}=\marg{sample\cs{nopostdesc}\marg{} }
\end{codeenv}
because the final space is now significant, but even without the
space it still won't be stripped as the field ends with
an empty group not with \cs{nopostdesc}. Similarly it won't be
stripped from:
\begin{codeenv}
\field{description}=\marg{sample\cs{nopostdesc}\cmd{relax}}
\end{codeenv}
because again it's not at the end.

\optsection[\subsubsection]{check-end-punctuation}

This options checks the end of all the fields given in \meta{list} for
end of sentence punctuation. This is determined as follows, for each
\meta{field} in the comma-separated \meta{list}:
\begin{itemize}
\item if the last character is of type \idx{punctuationclose}
or \idx{punctuationfinalquote}, check the character that comes
before it;
\item if the character is of type \idx{punctuationother}, then check
if it's listed in the entry given by \idx{sentence.terminators} 
in \bibgls's \langxml.
\end{itemize}
If a sentence terminator is found, an internal field is created
called \field{fieldendpunc} that contains the punctuation
character. Fields whose values must be labels (such as
\field{parent}, \field{category} and \field{type}) aren't checked,
even if they're included in \meta{list}.

The default \idx{sentence.terminators} is defined in \file{bib2gls-en.xml}
as:
\begin{verbatim}
<entry key="sentence.terminators">.?!</entry>
\end{verbatim}
Any character that isn't of type \idx{punctuationother} won't
match.

For example, the sample \exfile!{books.bib} file contains:
\begin{codeenv}
\atentry{entry}\marg{whydidnttheyaskevans,
  \field{name}=\marg{Why Didn't They Ask Evans?},
  \field{description}=\marg{novel by Agatha Christie},
  \fieldfmt{identifier}=\marg{book},
  \fieldfmt{author}=\marg{\gls{sortmediacreator}\marg{Agatha}\marg{Christie}},
  \fieldfmt{year}=\marg{1934}
}
\end{codeenv}
With \csopt[name]{check-end-punctuation}, this entry will be
assigned an internal field called 
\fielddisp{fieldendpunc}{nameendpunc} set to \code{?}\
as that's included in \idx{sentence.terminators} and is found
at the end of the \field{name} field:
\begin{codeenv}
\cs{GlsXtrSetField}\marg{whydidnttheyaskevans}\marg{nameendpunc}\marg{?}
\end{codeenv}
(Note that \csopt[first,text]{check-end-punctuation} won't match
as there's no \field{first} or \field{text} field supplied.)

If you have a field that ends with an abbreviation followed by a
\idx{full-stop}, this will be considered an end of sentence terminator, but
the main purpose of this option is to provide a way to deal with
cases like:
\begin{codeenv}
Agatha Christie wrote \cs{gls}\marg{whydidnttheyaskevans}.
\end{codeenv}
where the end of sentence punctuation following \cs{gls} needs to be
discarded. This is needed regardless of whether or not
the link text ends with an abbreviation or is a complete sentence.

It's then possible to hook into the \idx{postlinkhook} \qt{discard
period} check. By default this just checks the category attributes
that govern whether or not to discard a following period, but
(with \styfmt{glossaries-extra} v1.23+) it's possible to provide
an additional check by redefining:
\nosecdef{glsxtrifcustomdiscardperiod}
This should expand to \meta{true} if the check should be performed
otherwise it should expand to \meta{false}. You can reference the
label using \cs{glslabel}. For example:
\begin{codeenv}
\cs{renewcommand}*\marg{\gls{glsxtrifcustomdiscardperiod}}[2]\marg{\comment{}
 \cs{GlsXtrIfFieldUndef}\marg{nameendpunc}\marg{\cs{glslabel}}\marg{\idx{param}2}\marg{\idx{param}1}\comment{}
}
\end{codeenv}
This uses \ics{GlsXtrIfFieldUndef} rather than
\ics{glsxtrifhasfield*} since there's no need to access the field's
value. (The unstarred form \ics{glsxtrifhasfield} can't be used
as it introduces implicit scoping, which would interfere with the
punctuation lookahead.) The other difference between
\ics{GlsXtrIfFieldUndef} and the other \csfmt{\ldots hasfield} tests
is the case where the field is set to an empty value. In this case
the field is defined (so \ics{GlsXtrIfFieldUndef} does the
\meta{false} argument) but it's considered unset (so commands like
\ics{ifglshasfield} do the \meta{false} argument).

\optsection[\subsubsection]{sort-label-list}

This option takes a list as the value with each element in the list
in the form:
\begin{codeenv}
\code{\meta{field-list}:\meta{sort}:\meta{csname}}
\end{codeenv}
or:
\begin{codeenv}
\code{\meta{field-list}:\meta{sort}}
\end{codeenv}
where:
\begin{itemize}
\item \meta{field-list} is a comma-separated list of valid fields;
\item \meta{sort} is a valid sort method as per the \csopt{sort}
option, but not including \optfmt{none} or \optfmt{unsrt};
\item \meta{csname} is the name (without a leading backslash) of a
command that takes a label as its sole mandatory argument that's
recognised by \bibgls' interpreter (such as those listed in
\tableref{tab:bibglsdefs}).
\end{itemize}
The final \code{:\meta{csname}} part may be omitted if no command
need be applied. (That is, sort by label.) The \meta{list} value
is required for this option but may be empty, which indicates the setting is
switched off.

The sorting options are as those for the \gls{mainlist}.  For
example, for entries in the \glsdisp{mainlist}{primary list} the
break point is obtained from the \csopt{break-at} setting and for
entries in the \gls{duallist} the break point is obtained from
\csopt{dual-break-at}.  (Remember that if \csopt[combine]{dual-sort}
then there is only one list that contains both the \primary\ and
\dual\ entries, which is governed by the \primary\ options only.)

If the \meta{field-list} has more than one element
take care to use braces \code{\marg{}} to avoid confusion for the
list-parser. For example:
\begin{codeenv}
\gls{GlsXtrLoadResources}\oarg{
  \csopt[\marg{see,seealso}:en:glsentryname]{sort-label-list}
}
\end{codeenv}

Note that strange results may occur if this setting is used on any
fields that don't simply contain a list of entry labels or if any of
the referenced entries are processed in different
\iglspl{resourceset} (see \sectionref{sec:resourcesets}).

After the main sorting of each set of selected entries is performed
(as per \csopt{sort} or \csopt{dual-sort}), if this
option is set, then for each 
\code{\margm{field-list}:\meta{sort}:\meta{csname}} the following
steps are performed:
\begin{enumerate}
\item For each entry \meta{id}:
 \begin{enumerate}
  \item For each \meta{field} in \meta{field-list}, if the field is
    set for entry \meta{id} then:

  \begin{enumerate}
   \item The field value must be in the form
     \code{\oargm{tag}\meta{label-list}} where
     \code{\oargm{tag}} is optional and
     \meta{label-list} is a comma-separated list of entry labels
     \meta{label$_1$}, \ldots, \meta{label$_n$};
   \item A new list is constructed where the $i$th element is:
    \code{\marg{\csfmt{}\meta{csname}\margm{label$_i$}}}
    unless \meta{csname} hasn't been set, in which case the
    $i$th element is just \margm{label$_i$} (the optional
    \oargm{tag} part is omitted);
   \item This new list is sorted according to the interpreter's
    definition of the command given by \meta{csname} (if provided) 
    and the designated \meta{sort} method;
   \item The field value is reconstructed with the labels in the
    corresponding order (prefixed with \code{\oargm{tag}} if it was
    present in the original).
  \end{enumerate}
 \end{enumerate}
\end{enumerate}
Note that there is no \hierarchical\ structure in the sorting of the field list
even if any of the referenced entries has a \parent.

For example, suppose the file \filefmt{entries.bib} contains:
\begin{codeenv}
\atentry{index}\marg{bird}
\strut
\atentry{index}\marg{waterfowl, \field{parent}=\marg{bird} }
\strut
\atentry{index}\marg{duck,
  \field{parent}=\marg{waterfowl},
  \field{seealso}=\marg{swan,duckling,parrot,goose}
}
\strut
\atentry{index}\marg{swan,
  \field{parent}=\marg{waterfowl},
  \field{seealso}=\marg{goose,duck}
}
\strut
\atentry{index}\marg{goose,
 \field{parent}=\marg{waterfowl},
 \field{seealso}=\marg{duck}
}
\strut
\atentry{index}\marg{parrot, \field{parent}=\marg{bird} }
\strut
\atentry{index}\marg{duckling,
 \field{see}=\marg{[related terms]fluffy,velociraptor,duck,tardigrade}
}
\strut
\atentry{index}\marg{fluffy}
\strut
\atentry{index}\marg{tardigrade, \field{name}=\marg{water bear} }
\strut
\atentry{index}\marg{velociraptor}
\end{codeenv}
And suppose the document contains:
\begin{codeenv}
\cmd{documentclass}\marg{article}
\strut
\cmd{usepackage}[\styopt{record},\styopt[tree]{style}]\marg{glossaries-extra}
\strut
\gls{GlsXtrLoadResources}\oarg{ 
 \csopt[entries]{src},
 \csopt[en]{sort},
 \csopt[\marg{seealso,see}:en:glsentryname]{sort-label-list}
}
\strut
\cmd{begin}\marg{document}
\cs{Gls}\marg{parrot}, \cs{gls}\marg{tardigrade}, \cs{gls}\marg{swan}, \cs{gls}\marg{duck}, 
\cs{gls}\marg{goose}, \cs{gls}\marg{fluffy} \cs{gls}\marg{duckling}, \cs{gls}\marg{velociraptor}.
\strut
\cs{printunsrtglossaries}
\cmd{end}\marg{document}
\end{codeenv}
Then this reorders the \field{see} and \field{seealso} fields according to
the referenced entry's name (obtained with \ics{glsentryname}).

For example, the \field{see} field for the \code{duckling} entry was
originally:
\begin{codeenv}
\field{see}=\marg{[related terms]fluffy,velociraptor,duck,tardigrade}
\end{codeenv}
but in the \ext{glstex} file it's written as:
\begin{codeenv}
\field{see}=\marg{[related terms]duck,fluffy,velociraptor,tardigrade}
\end{codeenv}
The reason for \code{tardigrade} being placed after
\code{velociraptor} is because
\code{\ics{glsentryname}\marg{tardigrade}} is expanded to \qt{water
bear} (and \qt{W} comes after \qt{V}). If no encapsulating command 
was specified:
\begin{codeenv}
\csopt[\marg{seealso,see}:en]{sort-label-list}
\end{codeenv}
then the list would have been sorted according to the labels
instead (and so \code{tardigrade} would come before
\code{velociraptor}). Note that the optional tag is kept at the start of the 
list.

The \field{seealso} fields have also been changed. For example, 
the \code{duck} entry originally had:
\begin{codeenv}
\field{seealso}=\marg{swan,duckling,parrot,goose}
\end{codeenv}
but in the \ext{glstex} file it's written as:
\begin{codeenv}
\field{seealso}=\marg{duckling,goose,parrot,swan}
\end{codeenv}
Note that the \hierarchical\ structure hasn't been maintained. The glossary
lists \qt{duckling} (a \gls{top-levelentry}) after \qt{swan} (a level~2 entry)
but the \field{seealso} field has \code{duckling} first.

If you want to maintain the \hierarchy\ you can use \ics{glsxtrhiername} instead
of \ics{glsentryname}:
\begin{codeenv}
\gls{GlsXtrLoadResources}\oarg{ 
 \csopt[entries]{src},
 \csopt[en]{sort},
 \csopt[\marg{seealso,see}:en:glsxtrhiername]{sort-label-list}
}
\end{codeenv}
The separator between the levels is
given by \ics{glsxtrhiernamesep} which is defined by
\sty{glossaries-extra} to produce \qt{\glsxtrhiernamesep}.
The \bibgls\ interpreter's definition of this command is different
to assist sorting and simply expands to a \idx{full-stop} to prevent
it from being replaced by the default word break marker.

In this case \code{\ics{glsxtrhiername}\marg{swan}} would be
displayed as \qt{bird\glsxtrhiernamesep waterfowl\glsxtrhiernamesep
swan} if used in the document, but the interpreter converts it to
\qt{bird.waterfowl.swan}, so with the default \csopt{break-at}
setting the actual sort value becomes \code{bird.waterfowl.swan|}
(instead of \code{bird|waterfowl|swan|} which would be the result if
the interpreter used the same definition as \sty{glossaries-extra}).

Therefore the \field{seealso} field for the \code{duck} entry ends up as:
\begin{codeenv}
\field{seealso}=\marg{parrot,goose,swan,duckling}
\end{codeenv}
Now \code{swan} comes before \code{duckling} because the actual sort
value started with a \qt{B} not \qt{S}.

This \hierarchical\ information isn't shown in the cross-reference by
default, so the \code{duck} cross-reference list appears in the
document as: parrot, goose, swan \& duckling.

If you want the \hierarchical\ information to appear to help assist
the reader, you can redefine \ics{glsseeitemformat} in the document
to use \ics{glsxtrhiername}:
\begin{codeenv}
\cmd{renewcommand}*\marg{\cs{glsseeitemformat}}[1]\marg{\cs{glsxtrhiername}\marg{\idx{param}1}}
\end{codeenv}
This means that the \code{duck} cross-reference now appears in the
document as: bird\glsxtrhiernamesep parrot, bird\glsxtrhiernamesep
waterfowl\glsxtrhiernamesep goose, bird\glsxtrhiernamesep waterfowl
\glsxtrhiernamesep swan \& duckling.

This next example document has two languages, English and
Portuguese. The file \filefmt{entries-en.bib} contains the English
terms, such as:
\begin{codeenv}
\atentry{index}\marg{cat, \fieldfmt{translations}=\marg{gato,gatinho} }
\atentry{index}\marg{kitten, \fieldfmt{translations}=\marg{gatinho} }
\atentry{index}\marg{staple, \fieldfmt{translations}=\marg{grampo}}
\atentry{index}\marg{rivet, \fieldfmt{translations}=\marg{rebite}}
\end{codeenv}
The file \filefmt{entries-pt.bib} contains the Portuguese
terms, such as:
\begin{codeenv}
\atentry{index}\marg{gato, \fieldfmt{translations}=\marg{cat,staple,rivet} }
\atentry{index}\marg{gatinho, \fieldfmt{translations}=\marg{kitten} }
\end{codeenv}

Both files have a custom field called \fieldfmt{translations} that
will need to be either defined or aliased.  This field contains a
comma-separated list of labels for the corresponding entries in the
other language file that provide a possible translation.  Where a
word has multiple possible translations, I'd like the list sorted
alphabetically. (In practice, it would make more sense to sort them
according to how likely the translation is, but this is for
illustrative purposes.) For convenience, the
custom field is simply aliased to the \field{user1} field.

The document has two glossaries for each set of terms. The English
terms are sorted according to \csopt[en-GB]{sort} in one
\igls{resourceset} and the Portuguese terms are sorted according to
\csopt[pt-BR]{sort} in another \igls{resourceset}. This means that there
are cross-resource references, but since there are no instances of
\atentry{preamble} it should be possible to resolve the references.

The document code is:
\begin{codeenv}
\cmd{documentclass}\marg{article}
\strut
\cmd{usepackage}[T1]\marg{fontenc}
\cmd{usepackage}[utf8]\marg{inputenc}
\cmd{usepackage}[british,brazilian]\marg{babel}
\cmd{usepackage}\oarg{\styopt{record},
 \styopt{nomain},
 \styopt{nostyles},
 \styopt[bookindex]{stylemods},
 \styopt[bookindex]{style}
}\marg{glossaries-extra}
\cmd{usepackage}\marg{glossaries-prefix}
\strut
\cs{newglossary*}\marg{en}\marg{English Terms}
\cs{newglossary*}\marg{pt}\marg{Portuguese Terms}
\strut
\gls{GlsXtrLoadResources}\oarg{
  \csopt[all]{selection},
  \csopt[en]{type},
  \csopt[entries-en]{src},
  \csopt[en-GB]{sort},
  \csopt[\fieldfmt{translations}=\field{user1}]{field-aliases},
  \csopt[\field{user1}:pt-BR:\encap{glsentryname}]{sort-label-list}
}
\gls{GlsXtrLoadResources}\oarg{
  \csopt[all]{selection},
  \csopt[pt]{type},
  \csopt[entries-pt]{src},
  \csopt[pt-BR]{sort},
  \csopt[\fieldfmt{translations}=\field{user1}]{field-aliases},
  \csopt[\field{user1}:en-GB:\encap{glsentryname}]{sort-label-list}
}
\strut
\cs{apptoglossarypreamble}\oarg{en}\marg{\ics{selectlanguage}\marg{british}}
\cs{apptoglossarypreamble}\oarg{pt}\marg{\cs{selectlanguage}\marg{brazilian}}
\strut
\cmd{begin}\marg{document}
\cs{renewcommand}*\marg{\cs{glsxtrbookindexname}}[1]\marg{\comment{}
 \cs{glossentryname}\marg{\idx{param}1}\comment{}
 \cs{glsxtrifhasfield}\marg{useri}\marg{\idx{param}1}\marg{: \ics{glsxtrseelist}\cs{glscurrentfieldvalue}}\marg{}\comment{}
}
\cs{printunsrtglossaries}
\cmd{end}\marg{document}
\end{codeenv}
In verbose mode, the transcript file indicates that it's performing
the label list sorting. For example, when sorting according to
\csopt[user1:pt-BR:glsentryname]{sort-label-list}, the transcript
file contains:
\begin{verbatim}
Label list sort method 'pt-BR' on field: user1
\end{verbatim}
The \code{cat} entry has a list of two elements in this field:
\code{gato,gatinho}. This is converted into a new list where the
first element is:
\begin{codeenv}
\marg{\cs{glsentryname}\marg{gato}}
\end{codeenv}
and the second element is:
\begin{codeenv}
\marg{\cs{glsentryname}\marg{gatinho}}
\end{codeenv}
Regardless of the level of verbosity, the transcript file will
contain the conversions obtained by the interpreter:
\begin{verbatim}
texparserlib: {\glsentryname{gato}} -> gato
texparserlib: {\glsentryname{gatinho}} -> gatinho
\end{verbatim}
The \code{kitten} entry has the same list, and the same process is
repeated for that entry. The \commonlongarg{verbose} mode will provide additional
information. The \commonlongarg{debug} mode will indicate whether the
referenced label was found in the current \igls{resourceset} or if it had
to be fetched from another \igls{resourceset}. So if the resulting
order isn't what you expect, check the transcript file for messages.

\optsection[\subsubsection]{prune-xr}

If true, this is a shortcut for:
\begin{codeenv}
\csopt[entrytype=\marg{index(plural)?},see=\marg{},seealso=\marg{},alias=\marg{}]{prune-see-match},
\csopt[entrytype=\marg{index(plural)?},see=\marg{},seealso=\marg{},alias=\marg{}]{prune-seealso-match},
\end{codeenv}

This will remove any labels in an entry's \field{see} or
\field{seealso} field where the referenced label doesn't have any
records and hasn't been selected as another form of dependency and
whose \gls{entrytype} is either \atentry{index} or \atentry{indexplural}
and doesn't have the \field{see}, \field{seealso} or \field{alias}
fields set.

Both \csopt{prune-see-match} and \csopt{prune-seealso-match} can be
switched off at the same time with \csopt[false]{prune-xr}.

\optsection[\subsubsection]{prune-see-match}

The value has the same syntax as \csopt{match}. Omitting the value
switches off the setting. This option is not cumulative.

If a value is supplied, this setting will attempt to prune unnecessary
labels from \field{see} fields. Note that pruning may fail if there
are cross-reference trails.

A label will be stripped from a \field{see} field if the label
references an entry that has no records, isn't dependent on another
entry, hasn't previously been selected, and matches the given
criteria. If more that one pattern match is supplied,
\csopt{prune-see-op} determines whether to apply a logical AND or a
logical OR.

For example, suppose the file \filefmt{entries.bib} contains the
following:
\begin{codeenv}
\atentry{index}\marg{pumpkin}
\atentry{index}\marg{cucumber}
\atentry{index}\marg{melon}
\atentry{index}\marg{cucurbit,\field{see}=\marg{gourd}}
\atentry{index}\marg{gourd,\field{see}=\marg{pumpkin,cucumber,melon}}
\atentry{index}\marg{courgette}
\atentry{index}\marg{marrow,\field{seealso}=\marg{courgette}}
\atentry{index}\marg{broccoli}
\atentry{index}\marg{cauliflower,\field{seealso}=\marg{broccoli}}
\end{codeenv}
Suppose the document contains:
\begin{codeenv}
\gls{GlsXtrLoadResources}\oarg{\csopt[entries]{src}}
\cmd{begin}\marg{document}
\cs{gls}\marg{cucurbit}, \cs{gls}\marg{pumpkin}, \cs{gls}\marg{melon}, \cs{gls}\marg{broccoli},
\cs{gls}\marg{marrow}, \cs{gls}\marg{cauliflower}.
\cs{printunsrtglossary}\oarg{title=Index}
\cmd{end}\marg{document}
\end{codeenv}
This uses the default \csopt[recorded and deps]{selection} setting,
which selects recorded entries (cucurbit, pumpkin, melon, broccoli,
marrow and cauliflower) and their dependencies. In this case, the
dependencies are: courgette (because it's listed in the marrow's
\field{seealso} field), gourd (because it's listed in the cucurbit's
\field{see} field), and cucumber (because it's listed in the
gourd's \field{see} field).
The resulting list is:
\begin{quote}
\setlength{\parindent}{0pt}%
\setlength{\parskip}{0pt plus 0.3pt}%
\glstreeitem broccoli 1\par
\glstreeitem cauliflower 1, \emph{see also} broccoli\par
\glstreeitem courgette\par
\glstreeitem cucumber\par
\glstreeitem cucurbit 1, \emph{see} gourd\par
\glstreeitem gourd \emph{see} pumpkin, cucumber \& melon\par
\glstreeitem marrow 1, \emph{see also} courgette\par
\glstreeitem melon 1\par
\glstreeitem pumpkin 1\par
\end{quote}

This means that courgette and cucumber appear in the glossary
without a \igls{locationlist}. If this was an actual glossary with
descriptions, this may not be a problem, but it looks strange for an
index since the cross-reference essentially leads the reader to a dead end.

Switching to \csopt[recorded no deps]{selection} will remove
courgette, gourd and cucumber but the \field{see} and
\field{seealso} fields will be lost. Since gourd references both
pumpkin and melon (which are used in the document), it might be
useful to keep the gourd entry. The aim of pruning is to remove the
unwanted cucumber entry from the gourd's \field{see} list but retain
pumpkin and melon.

An appropriate filter is needed to switch on pruning. (This is in
addition to the criteria that the pruned entry has no records, isn't
dependent on another entry, and hasn't previously been selected.)
This type of pruning is usually only necessary for indexes so a
useful filter may be simply on the \gls{entrytype} (either
\atentry{index} or \atentry{indexplural}):
\begin{codeenv}
\gls{GlsXtrLoadResources}\oarg{\csopt[entries]{src},
 \csopt[entrytype=\marg{index(plural)?}]{prune-see-match}}
\end{codeenv}
Another possibility is to filter on an empty \field{description}:
\begin{codeenv}
\gls{GlsXtrLoadResources}\oarg{\csopt[entries]{src},\csopt[description=\marg{}]{prune-see-match}}
\end{codeenv}

The result is that the cauliflower and marrow entries keep their
\field{seealso} lists (since this option only applies to \field{see}
lists) and the courgette entry has been added (because it's
in the marrow entry's \field{seealso} list). The gourd entry is
removed from the cucurbit's \field{see} list (because it matches the
criteria) and is not selected (because it's no longer a dependency).

In this case, I'd like to include the gourd entry because it has the
\field{see} field set. This means adjusting the criteria so that
only entries without the \field{see} field can be pruned:
\begin{codeenv}
\gls{GlsXtrLoadResources}\oarg{\csopt[entries]{src},
 \csopt[entrytype=\marg{index(plural)?},see=\marg{}]{prune-see-match}}
\end{codeenv}
This means that gourd is now selected (and retained in the
cucurbit's \field{see} field) but cucumber is removed from the
gourd's \field{see} field.

A similar method can be applied for the \field{seealso} fields using
\csopt{prune-seealso-match}. There's no applicable setting for the
\field{alias} field (since it's expected that the alias be present
due to the nature of the way the \field{alias} field works).

For convenience, the \csopt{prune-xr} option is provided as a
shortcut. If the resource command in the above example is modified
to:
\begin{codeenv}
\gls{GlsXtrLoadResources}\oarg{\csopt[entries]{src},\csopt{prune-xr}}
\end{codeenv}
then the resulting list will be:
\begin{quote}
\setlength{\parindent}{0pt}%
\setlength{\parskip}{0pt plus 0.3pt}%
\glstreeitem broccoli 1\par
\glstreeitem cauliflower 1, \emph{see also} broccoli\par
\glstreeitem cucurbit 1, \emph{see} gourd\par
\glstreeitem gourd \emph{see} pumpkin \& melon\par
\glstreeitem marrow 1\par
\glstreeitem melon 1\par
\glstreeitem pumpkin 1\par
\end{quote}

Note that if the pumpkin and melon references are removed from the
document, then gourd will still be selected but will have no
cross-reference. This is because the cucurbit entry is checked for
pruning while the gourd entry still has a non-empty \field{see}
field so it's not removed from the cucurbit entry.

There are two ways around this problem: either switch the
definitions of cucurbit and gourd around in the \ext{bib} file
or use \csopt{prune-iterations} to reprune (in this case,
\csopt[2]{prune-iterations} is sufficient).

\begin{important}
This setting is only compatible with the \qt{recorded and dep}
selection criteria: \csopt[recorded and
deps]{selection}, \csopt[recorded and deps and see]{selection}
and \csopt[recorded and deps and see not also]{selection}.
\end{important}

\optsection[\subsubsection]{prune-see-op}

If the value of \csopt{prune-see-match} contains more than one 
\meta{key}=\meta{pattern} element, the \csopt{prune-see-op}
determines whether to apply a logical AND or a logical OR.
The \meta{value} may be either \optfmt{and} or \optfmt{or}.
The default is \csopt[and]{prune-see-op}.

\optsection[\subsubsection]{prune-seealso-match}

As \csopt{prune-see-match} but for \field{seealso} fields. If more
that one pattern match is supplied, \csopt{prune-seealso-op}
determines whether to apply a logical AND or a logical OR.

\begin{important}
This setting is only compatible with the \qt{recorded and dep}
selection criteria: \csopt[recorded and
deps]{selection}, \csopt[recorded and deps and see]{selection}
and \csopt[recorded and deps and see not also]{selection}.
\end{important}

\optsection[\subsubsection]{prune-seealso-op}

If the value of \csopt{prune-seealso-match} contains more than one 
\meta{key}=\meta{pattern} element, the \csopt{prune-seealso-op}
determines whether to apply a logical AND or a logical OR.
The \meta{value} may be either \optfmt{and} or \optfmt{or}.
The default is \csopt[and]{prune-seealso-op}.

\optsection[\subsubsection]{prune-iterations}

If you have cross-reference trails, you may need to reprune. The
value of this options indicates the number of pruning iterations.
The default is 1. The higher the number, the longer \bibgls\ will
take to complete. The value can't be less that 1.

The maximum number of iterations is capped at 20. A cross-reference
trail that long is excessive for an index.

\optsection[\subsubsection]{bibtex-contributor-fields}

This option indicates that the listed fields all use \BibTeX's name
syntax (as used in \BibTeX's \code{author} and \code{editor} fields). 
The \meta{list} value is required for this key but
may be empty, which indicates an empty set of fields (that is, the
setting is switched off).

The values of these fields will be converted into the form:
\begin{codeenv}
\gls{bibglscontributorlist}\margm{contributor list}\margm{n}
\end{codeenv}
where \meta{n} is the number of names in the list and
\meta{contributor-list} is a comma-separated list of names in the
form:
\begin{codeenv}
\gls{bibglscontributor}\margm{forenames}\margm{von-part}\margm{surname}\margm{suffix}
\end{codeenv}

The \gls{bibglscontributorlist} command is initially defined in 
\bibgls's interpreter to just do the first argument and ignore the
second. This means that if you're sorting on this field, the \qt{and}
part between the final names doesn't appear in the sort value.
The actual definition of \gls{bibglscontributorlist} provided in the 
\ext{glstex} file depends on whether or not \ics{DTLformatlist} is defined.
(Note that \styfmt{glossaries} automatically loads \sty{datatool-base}
so this command will be defined if you have at least v2.28 of
\sty{datatool-base}.)

For example, if the \field{name} field is specified as:
\begin{codeenv}
\field{name}=\marg{John Smith and Jane Doe and Dickie von Duck}
\end{codeenv}
then \csopt[name]{bibtex-contributor-fields} will convert the
\field{name} field value to:
\begin{codeenv}
\gls{bibglscontributorlist}\marg{\comment{}
 \gls{bibglscontributor}\marg{John}\marg{}\marg{Smith}\marg{},\comment{}
 \gls{bibglscontributor}\marg{Jane}\marg{}\marg{Doe}\marg{},\comment{}
 \gls{bibglscontributor}\marg{Dickie}\marg{von}\marg{Duck}\marg{}}\marg{3}
\end{codeenv}
With \csopt[von]{contributor-order} the sort value obtained from
this field will be:
\begin{verbatim}
Smith, John,Doe, Jane,von Duck, Dickie
\end{verbatim}
With one of the locale sort methods and with the default
\csopt[word]{break-at}, this will end up as:
\begin{verbatim}
Smith|John|Doe|Jane|von|Duck|Dickie
\end{verbatim}

\optsection[\subsubsection]{contributor-order}

The \gls{bibglscontributor} command is defined in
\bibgls's interpreter and its definition is dependent on this
setting. The \meta{value} may be one of (where the parts in square 
brackets are omitted if that argument is empty):
\begin{itemize}
\item \optfmt{surname}: \gls{bibglscontributor} expands to
\meta{surname}\oarg{, \meta{suffix}}\oarg{, \meta{forenames}}\oarg{,
\meta{von-part}};
\item \optfmt{von}: \gls{bibglscontributor} expands to
\oarg{\meta{von-part} }\meta{surname}\oarg{, \meta{suffix}}\oarg{, \meta{forenames}};
\item \optfmt{forenames}: \gls{bibglscontributor} expands to
\oarg{\meta{forenames} }\oarg{\meta{von-part} }\meta{surname}\oarg{,
\meta{suffix}}.
\end{itemize}
The default value is \optfmt{von}. Note that if you have multiple
resource sets, this option governs the way \bibgls's version of
\gls{bibglscontributor} behaves. The actual definition is written to
the \ext{glstex} using \cs{providecommand}, which means that \LaTeX\
will only pick up the first definition.

For example:
\begin{codeenv}
\cs{newcommand}*\marg{\gls{bibglscontributor}}[4]\marg{\comment{}
  \idx{param}1\ics{ifstrempty}\marg{\idx{param}2}\marg{}\marg{ \idx{param}2} \idx{param}3\cs{ifstrempty}\marg{\idx{param}4}\marg{}\marg{, \idx{param}4}\comment{}
}
\strut
\gls{GlsXtrLoadResources}\oarg{
  \csopt[entries]{src},\comment{data in entries.bib}
  \csopt[name]{bibtex-contributor-fields}
}
\end{codeenv}
This will display the names in the glossary with the forenames
first, but \bibgls\ will sort according to surname.

An alternative approach, if you need an initial resource set
such as with the \exfile{no-interpret-preamble.bib} file:
\begin{codeenv}
\gls{GlsXtrLoadResources}\oarg{
  \csopt[no-interpret-preamble]{src},
  \csopt[false]{interpret-preamble},
  \csopt[name]{bibtex-contributor-fields},
  \csopt[forenames]{contributor-order}
}
\strut
\gls{GlsXtrLoadResources}\oarg{
  \csopt[entries]{src},\comment{data in entries.bib}
  \csopt[name]{bibtex-contributor-fields}
}
\end{codeenv}
Note the need to use \csopt[name]{bibtex-contributor-fields}
in the first resource set even though there are no entries in 
the \ext{bib} file. This is because the definition of
\gls{bibglscontributor} is only written to the \ext{glstex} file if
\csopt{bibtex-contributor-fields} has been set to a non-empty list.
The second resource set will use the default 
\csopt[von]{bibtex-contributor-fields} setting when obtaining the
sort value.

\optsection[\subsubsection]{encapsulate-fields}

This option should take a comma-separated list of 
\optfmt{\meta{field}\dequals\meta{cs-name-1arg}} values, where
\meta{cs-name-1arg} is the name of a control sequence that takes
one argument. The value is required for this key but
may be empty, which indicates an empty set (that is, the
setting is switched off). 

During the processing stage, each field identified in
the list (if defined) will have its value replaced with:
\begin{codeenv}
\cmd{\meta{cs-name-1arg}}\margm{value}
\end{codeenv}
where \meta{value} was its previous value. An
empty list switches off encapsulation (the default).

This action overrides any previous use of \csopt{encapsulate-fields}
within the same \igls{resourceset} and is always performed before
\csopt{encapsulate-fields*}, regardless of the order in the 
\igls{resourceset}['s] list of options.

\optsection[\subsubsection]{encapsulate-fields*}

This option should take a comma-separated list of 
\optfmt{\meta{field}\dequals\meta{cs-name-2arg}} values, where 
\meta{cs-name-2arg} is the name of a control sequence that takes two
arguments. The value is required for this key but
may be empty, which indicates an empty set (that is, the
setting is switched off). 

During the processing stage, each field identified in the
list (if defined) will have its value replaced with:
\begin{codeenv}
\cmd{\meta{cs-name-2arg}}\margm{value}\margm{label}
\end{codeenv}
where \meta{value} was its previous value and \meta{label} is the
entry's label (including prefix, if appropriate). An
empty list switches off encapsulation (the default). 

This action overrides any previous use of
\csopt{encapsulate-fields*} within the same \igls{resourceset}, and
is always performed after \csopt{encapsulate-fields}, regardless of
the order in the \igls{resourceset}['s] list of options, so if the
same field is listed in both settings, its value will end up as:
\begin{codeenv}
\cmd{\meta{cs-name-2arg}}\marg{\cmd{\meta{cs-name-1arg}}\margm{value}}\margm{label}
\end{codeenv}
An alternative is to use the more complex \csopt{assign-fields} option.

\optsection[\subsubsection]{format-integer-fields}

This option should take a comma-separated list of
\optfmt{\meta{field}\dequals\meta{format}} values, where
\meta{format} is a
\href{https://docs.oracle.com/javase/8/docs/api/java/util/Formatter.html#syntax}{string
format pattern} that contains a single numeric specifier.
This will convert the value stored in the identified field to the
given format. If the field doesn't contain an integer value it won't
be changed. If the field contains a decimal value use
\csopt{format-decimal-fields} instead. This setting is performed
before field encapsulation.

Since format patterns uses \idx{percentchar} as a placeholder, which
can be problematic in the resource command, you will need to use
\ics{cs.percent} instead.  You may also use \ics{cs.hash},
\ics{cs.dollar}, \ics{cs.amp}, \ics{cs.openbrace},
\ics{cs.closebrace}, \ics{cs.underscore} and \ics{cs.backslash} to
indicate the corresponding literal character. You can use \cs{u}\meta{XXXX} to
indicate a character by its hexadecimal value, but remember that the
resource options will be expanded when they are written to the
resource file so use \cs{glshex} or \cs{cs.string}\cs{u}.

If you want to format the \field{definitionindex} field you must use
\csopt{save-definition-index} first. For example, to save this field
and then zero-pad it to four digits:
\begin{codeenv}
\csopt{save-definition-index},
\csopt[definitionindex=\ics{cs.percent}04d]{format-integer-fields}
\end{codeenv}
This option can't be used for the \field{useindex} field 
created with \csopt{save-use-index} as that field isn't set until
after the field modifications are made.

\optsection[\subsubsection]{format-decimal-fields}

As \csopt{format-integer-fields} but for decimal values. If a field
contains an integer then:
\begin{itemize}
\item if \csopt{format-integer-fields} has also been used to set a
format for the given field, the integer format will take precedence;
\item otherwise the integer value will be treated as a decimal
number.
\end{itemize}
If you get an error like:
\begin{verbatim}
Error: d != java.lang.Double
\end{verbatim}
then it means you have used an invalid specifier. (The above error
results from using \verb|%d| instead of \verb|%f| or \verb|%g|.)

\optsection[\subsubsection]{interpret-fields}

This option indicates that the listed fields should be replaced by
their interpreted values. The value is required for this key but
may be empty, which indicates an empty set of fields (that is, the
setting is switched off). Other fields not listed may still be
interpreted depending on other settings. As with the \field{sort}
field, any special characters are replaced with commands like
\ics{glsbackslash} and \ics{bibglsdollarchar}. This option is
applied after \csopt{field-case-change} (if set).

For example, suppose I have a file \filefmt{entries.bib} that
contains definitions like:
\begin{codeenv}
\atentry{symbol}\marg{pi,
  \field{name}=\marg{\cs{ensuremath}\marg{\ics{pi}}},
  \field{description}=\marg{the ratio of a circle's circumference to its diameter},
}
\strut
\atentry{symbol}\marg{sigma,
  \field{name} = \marg{\cs{ensuremath}\marg{\ics{sigma}}},
  \field{description} = \marg{standard deviation}
}
\end{codeenv}
Instead of having a list of terms (glossary), suppose I want to have
stand-alone definitions, where the term appears in a section heading.
I could define a command like this:
\begin{codeenv}
\cs{newcommand}\marg{\cmd{definition}}[1]\marg{\comment{}
  \cs{ifglsentryexists}\marg{\idx{param}1}\comment{}
  \marg{\comment{}
    \cs{section}\oarg{\cs{glsentryname}\marg{\idx{param}}}\marg{\cs{glsadd}\marg{\idx{param}1}\ics{glsxtrglossentry}\marg{\idx{param}1}}\comment{}
    \ics{Glossentrydesc}\marg{\idx{param}1}\cs{glspostdescription}
  }\comment{}
  \marg{\cs{section}\oarg{Missing `\idx{param}1'}\marg{\cs{glsadd}\marg{\idx{param}1}}}\comment{}
}
\end{codeenv}
which can be used in the document:
\begin{codeenv}
\ics{tableofcontents}
\cmd{definition}\marg{pi}
\cmd{definition}\marg{sigma}
\end{codeenv}
A problem with this definition of my custom command occurs if I add
\sty{hyperref} to the document, because this tries to write \cs{pi}
and \cs{sigma} to the PDF bookmarks, which doesn't work because
those commands can't be automatically converted to characters
permitted in a PDF string. This leads to a warning from
\sty{hyperref}:
\begin{alltt}
Token not allowed in a PDF string (Unicode)
\end{alltt}
Ideally I'd like to be able to convert these symbols to Unicode so
that they can appear in the bookmarks. Since \bibgls' interpreter
recognises these commands, I can get it to make the conversion
instead of trying to implement a method within \TeX:
\begin{codeenv}
\cs{glsaddstoragekey}\marg{pdfname}\marg{}\marg{\cmd{pdfname}}
\gls{GlsXtrLoadResources}\oarg{
  \csopt[entries]{src},
  \csopt[\field{name}=\fieldfmt{pdfname}]{replicate-fields},
  \csopt[fallback]{replicate-missing-field-action},
  \csopt[\fieldfmt{pdfname}]{interpret-fields}
}
\end{codeenv}
This first copies the \field{name} field to the custom
\fieldfmt{pdfname} and then interprets the copy. This leaves
the \field{name} field with the \LaTeX\ code to produce the symbol
in the document, but the \fieldfmt{pdfname} field ends up with all markup
stripped by the interpreter and the \cs{pi} and \cs{sigma} are
converted to the Unicode characters \hex{1D70B} (mathematical italic
small pi) and \hex{1D70E} (mathematical italic small sigma). With
\XeLaTeX\ or \LuaLaTeX\ these characters can be written to the PDF
bookmarks by adjusting the definition of the custom command:
\begin{codeenv}
\cs{newcommand}\marg{\cmd{definition}}[1]\marg{\comment{}
  \cs{ifglsentryexists}\marg{\idx{param}1}\comment{}
  \marg{\comment{}
    \cs{section}
    \oarg{\ics{texorpdfstring}\marg{\cs{glsentryname}\marg{\idx{param}1}}\marg{\cmd{pdfname}\marg{\idx{param}1}}}
    \marg{\cs{glsadd}\marg{\idx{param}1}\cs{glsxtrglossentry}\marg{\idx{param}1}}\comment{}
    \cs{Glossentrydesc}\marg{\idx{param}1}\cs{glspostdescription}
  }\comment{}
  \marg{\cs{section}\oarg{Missing `\idx{param}1'}\marg{\cs{glsadd}\marg{\idx{param}1}}}%
}
\end{codeenv}
With \pdfLaTeX\ and \sty{fontenc}, you will need 
\sty{hyperref}'s \styoptfmt{unicode} option:
\begin{codeenv}
\cs{usepackage}\oarg{\styoptfmt{unicode}}\marg{hyperref}
\end{codeenv}
If you still encounter problems with the Unicode characters not
appearing in the PDF bookmarks, then try the
\csopt{hex-unicode-fields} option. For example:
\begin{codeenv}
\csopt[pdfname]{hex-unicode-fields}
\end{codeenv}
This still requires \sty{hyperref}'s \styoptfmt{unicode} option.

\optsection[\subsubsection]{interpret-fields-action}

This option governs the behaviour of \csopt{interpret-fields}.
Available values are: 
\begin{itemize}
\item\code{replace}: replace the field content with its interpreted
value (default);
\item\code{replace non empty}: only replace the field content with
its interpreted value if the interpreted value isn't an empty
string.
\end{itemize}
If a field value consists solely of commands that are unknown to the
interpreter, then the resulting value will end up empty. In this
case, it may be more appropriate to leave the field unchanged.

\optsection[\subsubsection]{hex-unicode-fields}

This option will convert any Unicode characters (outside of the
Basic Latin set) that are found in the listed fields into
\code{\gls{bibglshexunicodechar}\margm{hex-code}} where
\meta{hex-code} is the hexadecimal character code.

The \meta{list} should be a comma-separated list of field names.
This action is performed after \csopt{interpret-fields}.

If the field contents need to be added to the PDF bookmarks (as in
the earlier example) then you need to make sure you use
\sty{hyperref}'s \styoptfmt{unicode} option
otherwise you'll get the warning:
\begin{verbatim}
Token not allowed in a PDF string (PDFDocEncoding):
removing `\char'
\end{verbatim}
and the bookmarks will show \idx{doublequotecharhex}\meta{hex-code}
instead of the Unicode character.

\optsection[\subsubsection]{date-time-fields}

This option indicates that the listed fields all contain 
date and time information. Primary entries will have these fields
parsed according to \csopt{date-time-field-format} and
\csopt{date-time-field-locale} and dual entries will have these
fields parsed according to \csopt{dual-date-time-field-format} and
\csopt{dual-date-time-field-locale}. If the field value is missing
or doesn't match the given pattern it remains unchanged, otherwise
it's converted into the form:
\formatdef{bibglsdatetime}
where \meta{original} is the value of the field before conversion.
If the interpreter is on, the value will be interpreted before
being parsed if it contains \idx{escchar}, \idx{mshiftchar},
\idx{bgroupchar}, \idx{egroupchar} or \idx{nbspchar}. (Remember that
\idx{nbspchar} is converted to the non-breaking space character
\hex{A0} unless \longarg{break-space} is used.)

\optsection[\subsubsection]{date-fields}

As \csopt{date-time-fields} but for fields that only contain date
(not time) information. If parsed correctly, the field is converted
to:
\formatdef{bibglsdate}
The fields are parsed according to
\csopt{date-field-format} and
\csopt{date-field-locale} for \glspl{primaryentry} and according to
\csopt{dual-date-field-format} and
\csopt{dual-date-field-locale} for \glspl{dualentry}.

\optsection[\subsubsection]{time-fields}

As \csopt{date-time-fields} but for fields that only contain time
(not date) information. If parsed correctly, the field is converted
to:
\formatdef{bibglstime}
The fields are parsed according to
\csopt{time-field-format} and
\csopt{time-field-locale} for \glspl{primaryentry} and according to
\csopt{dual-time-field-format} and
\csopt{dual-time-field-locale} for \glspl{dualentry}.

\optsection[\subsubsection]{date-time-field-format}

This option also sets
\csopt[\meta{value}]{dual-date-time-field-format}.
The value is the format pattern used when parsing fields identified
by \csopt{date-time-fields}. The \meta{value} is as for
\csopt{date-sort-format}.

\optsection[\subsubsection]{date-field-format}

This option also sets
\csopt[\meta{value}]{dual-date-field-format}.
The value is the format pattern used when parsing fields identified
by \csopt{date-fields}. The \meta{value} is as for
\csopt{date-sort-format}.

\optsection[\subsubsection]{time-field-format}

This option also sets
\csopt[\meta{value}]{dual-time-field-format}.
The value is the format pattern used when parsing fields identified
by \csopt{time-fields}. The \meta{value} is as for
\csopt{date-sort-format}.

\optsection[\subsubsection]{date-time-field-locale}

This option also sets
\csopt[\meta{value}]{dual-date-time-field-locale}.
The value is the locale used when parsing fields identified
by \csopt{date-time-fields}. The \meta{value} is as for
\csopt{date-sort-locale}.

\optsection[\subsubsection]{date-field-locale}

This option also sets
\csopt[\meta{value}]{dual-date-field-locale}.
The value is the locale used when parsing fields identified
by \csopt{date-fields}. The \meta{value} is as for
\csopt{date-sort-locale}.

\optsection[\subsubsection]{time-field-locale}

This option also sets
\csopt[\meta{value}]{dual-time-field-locale}.
The value is the locale used when parsing fields identified
by \csopt{time-fields}. The \meta{value} is as for
\csopt{date-sort-locale}.

\subsection{Prefix Fields}
\label{sec:prefixfields}

If you use the \isty{glossaries-prefix} package, the prefix
set of fields become available (\field{prefix},
\field{prefixplural}, \field{prefixfirst} and
\field{prefixfirstplural}). The default behaviour of \ics{pgls} is for no
separator between the prefix and the text produced with \cs{gls}.
This is because there are situations where there shouldn't be a
space, although a space is more commonly required.

This means that a space needs to be appended to the required prefix
fields, but an actual space character can't be used because
\sty{xkeyval} trims leading and trailing spaces. The \ics{space}
command needs to be used instead, but there are also situations
where an non-breakable space should be used (for example, where the
prefix is a single character). It's a bit tiresome having to
remember to put \ics{space} or \idx{nbspchar} at the end of the
field value. 

The \csopt{append-prefix-field} option allows the automatic insertion of a
space, but it may be used without the \sty{glossaries-prefix}
package. The fields that contain prefixes are identified by
\csopt{prefix-fields}.

If you have any dual entries, then \bibgls\ will also recognise the
special internal fields \field{dualprefix},
\field{dualprefixplural}, \field{dualprefixfirst} and
\field{dualprefixfirstplural}.

\optsection[\subsubsection]{prefix-fields}

Identifies the fields that are used to store prefixes. The default
set is: \field{prefix}, \field{prefixfirst}, \field{prefixplural},
\field{prefixfirstplural}, and their dual counterparts
\field{dualprefix}, \field{dualprefixfirst}, \field{dualprefixplural}
and \field{dualprefixfirstplural}.

\optsection[\subsubsection]{append-prefix-field}

Allowed values are:
\begin{itemize}
\item \optfmt{none}: don't append a space to the prefix fields (default);

\item \optfmt{space}: append the command identified by
\csopt{append-prefix-field-cs} (\ics{space} by default) to the
prefix field unless the field value ends with a character identified
by \csopt{append-prefix-field-exceptions} or a command identified by
\csopt{append-prefix-field-cs-exceptions}. Note that if the field
value ends with anything else (such as an empty group) then these
exceptions won't apply.

\item \optfmt{space or nbsp}: as above but uses \idx{nbspchar}
instead of \ics{space} if the field value matches the pattern given
by \csopt{append-prefix-field-nbsp-match}.

\end{itemize}

\optsection[\subsubsection]{append-prefix-field-cs}

Identifies the command \meta{cs} that should be used to append to
the prefix fields. The default value is \ics{space}. Remember to use
\cs{cs.string} or \cs{protect} to prevent the command from being
expanded as it's written to the \ext{aux} file.

\optsection[\subsubsection]{append-prefix-field-exceptions}

This setting identifies the set of characters that, if found at the
end of a prefix field, prevent \csopt{append-prefix-field} from
appending a space (either \ics{space} or \idx{nbspchar}).

The value should be a sequence of characters. You may use
\stringu\meta{hex} to identify a character by its hexadecimal code.
Spaces are ignored, so \csopt[ ' - ]{append-prefix-field-exceptions}
is equivalent to \csopt['-]{append-prefix-field-exceptions}.

The default set is the straight apostrophe character (\hex{0027}),
the hyphen-minus character (\hex{002D}), the tilde character
(\idx{nbspchar}), the hyphen character (\hex{2010}), the
non-breaking hyphen (\hex{2011}), and the right single quotation
mark (\hex{2019}).

\optsection[\subsubsection]{append-prefix-field-cs-exceptions}

This setting identifies the set of commands that, if found at the
end of a prefix field, prevent \csopt{append-prefix-field} from
appending a space (either \ics{space} or \idx{nbspchar}). Any spaces
found in \meta{sequence} are ignored. The default setting is the
set: \cs{space}, \ics{nobreakspace} and \cs{cs.space}.

Remember that you will need to use \cs{cs.string} or \cs{protect} to
prevent the command from being expanded while the resource options
are written to the \ext{aux} file.

\optsection[\subsubsection]{append-prefix-field-nbsp-match}

The value is the \gls{regular-expression} that identifies prefixes that should be
followed by \idx{nbspchar} instead of \ics{space}. The default
is \csopt[\idx{matchanydot}]{append-prefix-field-nbsp-match} which indicates a
single character.

\subsection{Case-Changing}
\label{sec:fieldcase}

The \sty{glossaries-extra} package comes with the category
attributes \catattr{glossdesc} and \catattr{glossname}, which may
take the values \optfmt{firstuc} or \optfmt{title}. These don't
change the actual \field{name} or \field{description} fields, but
instead \ics{glossentryname} and \ics{glossentrydesc} (which are
used by the default glossary styles) check for the corresponding
attribute and apply the appropriate \idx{case-change} to the field value.

So \ics{glossentryname} will use \ics{Glsentryname} if the
\catattr{glossname} attribute for the given entry is set to \optfmt{firstuc} 
and \ics{glossentrydesc} will use \ics{Glsentrydesc} if the
\catattr{glossdesc} attribute is set to \optfmt{firstuc}.
The \optfmt{title} setting will instead use \ics{capitalisewords}
applied to the field value.

The resource options described in this section provide an
alternative to those attributes that actually modify the relevant
field (rather than just adjusting the style code used to display it). 
There are two forms of modification: the field is adjusted so that
the original value is encapsulated by a command or \bibgls\ will
perform the actual \idx{case-change} according to its own algorithm.
The results can vary according to the field content.

Where \bibgls\ itself performs the case change, its case-changing
functions will use the \gls{resource-locale}, but whether or not
\bibgls\ recognises the correct rules for the locale depends on
whether or not the locale is correctly supported by the Java locale
provider. The \langxml\ may provide assistance with case-conversion.

\begin{important}
Note the difference between \ics{NoCaseChange} (which prevents case-changing
for both \bibgls\ and in the \LaTeX\ document) and \ics{BibGlsNoCaseChange}
(which only prevents case-changing in \bibgls). The options that defer
the case-change action to \LaTeX, such as \optfmt{uc-cs}, will treat 
\ics{NoCaseChange} as an exclusion but not \ics{BibGlsNoCaseChange}.
\end{important}

Each of the case-changing resource options may take one of the
following values:
\begin{itemize}
\item \optfmt{none}: don't apply any case-changing (default);

\item \optfmt{lc-cs}: make \bibgls\ behave as though the field
assignment:
\begin{codeenv}
\meta{field} = \margm{text}
\end{codeenv}
had actually been specified as:
\begin{codeenv*}
\meta{field} = \marg{\gls{bibglslowercase}\margm{text}}
\end{codeenv*}
which uses \TeX\ to convert the field to \idx{lowercase};

\item \optfmt{uc-cs}: make \bibgls\ behave as though the field
assignment:
\begin{codeenv}
\meta{field} = \margm{text}
\end{codeenv}
had actually been specified as:
\begin{codeenv*}
\meta{field} = \marg{\gls{bibglsuppercase}\margm{text}}
\end{codeenv*}
which uses \TeX\ to convert the field to \idx{uppercase};

\item \optfmt{firstuc-cs}: make \bibgls\ behave as though the field
assignment:
\begin{codeenv}
\meta{field} = \margm{text}
\end{codeenv}
had actually been specified as:
\begin{codeenv*}
\meta{field} = \marg{\gls{bibglsfirstuc}\margm{text}}
\end{codeenv*}
which uses \TeX\ to convert the field to first-letter \idx{uppercase};

\item \optfmt{title-cs}: make \bibgls\ behave as though the field
assignment:
\begin{codeenv}
\meta{field} = \margm{text}
\end{codeenv}
had actually been specified as:
\begin{codeenv*}
\meta{field} = \marg{\gls{bibglstitlecase}\margm{text}}
\end{codeenv*}
which uses \TeX\ to convert the field to \idx{titlecase};

\item \optfmt{lc}: convert to \idx{lowercase} by making the
appropriate modifications to tokens in the field value that have
a known \idx{lowercase} alternative (see below);

\item \optfmt{uc}: convert to \idx{uppercase} by making the
appropriate modifications to tokens in the field value that have
a known \idx{uppercase} alternative (see below);

\item \optfmt{firstuc}: convert to first letter \idx{uppercase} 
by making the appropriate modification, if it has 
a known \idx{uppercase} alternative (see below);

\item \optfmt{title}: convert to \idx{titlecase} by making the
appropriate modifications to the first letter of each identified
word in the field value that has a known \idx{uppercase}
alternative (see below).

A word-boundary is identified according to the
\csopt{word-boundaries} setting. Words to be excluded from the
case-changing (unless they occur at the start) can be identified
with \ics{MFUnocap} in the \atentry{preamble} or you can use
\code{\longarg{packages} mfirstuc-english} for the exclusion list
provided by the \isty{mfirstuc-english} package. Alternatively, you
can use \longarg{custom-packages} to load a simple package that
contains the required \cs{MFUnocap} commands (in a similar style to
\sty{mfirstuc-english}).

The \bibgls\ word-boundary implementation is slightly different with
this setting than with the \cs{capitalisewords} command (implemented
in \TeX\ or by the \texparserlib\ when interpreting field
values).  Only words in the exclusion list that start with an
alphabetical character can be matched.  Punctuation following a
word-boundary is not considered part of the next word.
If you want to identify that a particular character forms a word
break, you can use \code{\ics{MFUwordbreak}\margm{char}}. For
example:
\begin{codeenv}
\field{name}=\marg{some word\cs{MFUwordbreak}\marg{/}phrase}
\end{codeenv}

\end{itemize}

If you need to selectively change the case, based on some condition
(such as the \gls{entrytype}) then you can use the
\csopt{assign-fields} option instead, but remember that you will
need the override setting on. For example:
\begin{codeenv}
\csopt[
 \field{name} =\oarg{o} \gls{TITLE}\marg{ \field{name} }
   \oarg{ entrytype \idx{follow} original = \qtdelim{entry} }
]{assign-fields}
\end{codeenv}
This will convert the \field{name} field to \idx{titlecase} for
entries that were defined in the \ext{bib} file with
\atentry{entry}. Note that if you also use a case-changing option,
for example, \csopt{name-case-change}, then all entries will have
the change applied, according to the option's designated behaviour,
regardless of whether or not the applicable field has already been
altered by \csopt{assign-fields}.

\begin{important}
Major changes have been introduced to \sty{mfirstuc} v2.08. Some of
the information below refers to older versions and is not
applicable with \sty{mfirstuc} v2.08+. See the \sty{mfirstuc} manual
for further details.
\end{important}

The \optfmt{firstuc-cs} and \optfmt{firstuc} options are essentially
a \idx{sentencecase} change, but there's no check for
sentence-breaks within the value, so even if the value contains
multiple sentences, only the first is changed. If the text to be
changed starts with a punctuation character it should be
encapsulated with \ics{MFUskippunc} to apply the case-change to the
following object. For example:
\begin{codeenv}
\field{name}=\marg{\cs{MFUskippunc}\marg{'}tis}
\end{codeenv}
If the \optfmt{firstuc} option is applied to the \field{name} field
this will be converted to:
\begin{codeenv}
\field{name}=\marg{\cs{MFUskippunc}\marg{'}Tis}
\end{codeenv}
Using \ics{NoCaseChange} (provided by \sty{textcase}) instead will have the
same effect, but this isn't consistent with the behaviour of
\ics{makefirstuc} so it's best to use \cs{MFUskippunc} instead.

The \optfmt{\meta{option}-cs} settings defer the actual case-changing
to \TeX, which means that the case-changing has to be applied every
time the field is typeset (and it introduces non-expandable content
to the field value). Be aware of the limitations of using any of the
case-changing commands. See the \isty{textcase} and \isty{mfirstuc}
package documentation for further details~\cite{textcase,mfirstuc}.

For the settings where \bibgls\ itself performs the \idx{case-change}, then
\bibgls\ will iterate over each token of the field value and apply
the rules listed below.  Note that the case-change implemented by
\bibgls\ recognises the \gls{resource-locale}, but whether or not it
recognises the correct rules for the locale depends on whether or
not the locale is correctly supported by the Java locale provider.

\begin{enumerate}

\item If the token is a normal Unicode alphabetic character, it will
be replaced with the corresponding upper or lower case character, as
appropriate. In some cases, a single character, such as \ss, is
replaced by multiple characters, such as SS.

For \optfmt{title} and \optfmt{firstuc}, the \idx{titlecase} character is
used as the replacement, for \optfmt{uc} the \idx{uppercase} character is
used as the replacement, and for \optfmt{lc} the \idx{lowercase} character
is used as the replacement. Many characters have the same upper and
title case alternative (for example, \qt{a} will be converted to \qt{A}
for the \optfmt{title}, \optfmt{firstuc} and \optfmt{uc} settings), but
some characters have different title and upper versions (for example, 
the digraph \qt{\char"01F3} has the title version \qt{\char"01F2}
and \idx{uppercase} version \qt{\char"01F1}).

If the option is \optfmt{firstuc} then all the remaining tokens are
skipped. If the option is \optfmt{title} then the subsequent tokens
are skipped until a word-boundary is found.

\item If the token is a normal Unicode character that isn't alphabetical,
then this token will be skipped for all options.

\item If \code{\idx{mshiftchar}\meta{maths}\idx{mshiftchar}} is
encountered, it will be skipped. If the option is \optfmt{firstuc}
then all remaining tokens are skipped, so no \idx{case-change} will be
performed.

\item\label{case-change-group} If a group \code{\margm{content}} is
found, then the \idx{case-change} is applied to the entire \meta{content}
(which may be empty). This corresponds to the way \ics{makefirstuc}
and \ics{capitalisewords} work if a~word starts with a group. Note
that with \optfmt{firstuc} and \optfmt{title} the group content will
be converted according to \optfmt{uc}, so the normal \idx{uppercase}
character is used rather than the title case character (if they are
different).

If the option is \optfmt{firstuc} then all the remaining tokens are
skipped. If the option is \optfmt{title} then the subsequent tokens
are skipped until a word-boundary is found. A word-boundary can be
marked up with \ics{MFUwordbreak}.

\item If a control sequence \csfmt{}\meta{csname} is found, then:

  \begin{enumerate}
  \item If the control sequence is \cs{protect}, this token is skipped
  for all options.

  \item With \optfmt{firstuc} and \optfmt{title}, if 
  \code{\ics{MFUskippunc}\margm{text}} or \code{\ics{NoCaseChange}\margm{text}} 
  occurs at the start of a word, then \bibgls\ will act as though
  the word hasn't started yet (so the next token will be considered
  for a \idx{case-change}).

  \item If the control sequence is one of: \ics{o}, \ics{O}, \ics{l},
  \ics{L}, \ics{ae}, \ics{AE}, \ics{oe}, \ics{OE}, \ics{aa}, \ics{AA},
  \ics{ss}, \ics{SS}, \ics{ng}, \ics{NG}, \ics{th}, \ics{TH},
  \ics{dh}, \ics{DH}, \ics{dj} or \ics{DJ}, then
  it's replaced with its \idx{case-change} counterpart (if not already the
  correct case).

  If the option is \optfmt{firstuc} then all the remaining tokens are
  skipped. If the option is \optfmt{title} then the subsequent tokens
  are skipped until a word-boundary is found.

  \item If the control sequence is in the \csopt{no-case-change-cs}
  list or is \ics{ensuremath}, \ics{si} or if
  \meta{csname} ends with \qtt{ref} (for example, \ics{ref} or
  \ics{pageref}) then the control sequence and its argument is
  ignored.  In the case where \meta{csname} ends with \qtt{ref}, a
  following star (\code{*}) or optional argument before the mandatory
  argument will also be skipped. This allows for some common
  cross-referencing commands, such as \sty{hyperref}['s]
  \ics{autoref}, which may have a starred form, but does not allow for
  more complicated commands with multiple arguments.

  If the option is \optfmt{firstuc} then all the remaining tokens are
  skipped (so no \idx{case-change} will be performed). If the option is 
  \optfmt{title} then the subsequent tokens
  are skipped until a word-boundary is found (so no \idx{case-change} is
  performed for this word).

  \item If the control sequence is \ics{glsentrytitlecase} then:
   \begin{description}
   \item[\optfmt{lc}] the control sequence is converted to
    \ics{glsxtrusefield};
   \item[\optfmt{uc}] the control sequence is converted to
    \ics{GLSxtrusefield};
   \item[\optfmt{firstuc}] the control sequence is converted to
    \ics{Glsxtrusefield} and the remaining tokens are skipped;
   \item[\optfmt{title}] the control sequence is left unchanged
   and subsequent tokens are skipped until a word-boundary is found.
   \end{description}
  The field and entry label arguments are skipped.

  \item If the control sequence is \ics{glshyperlink} then the
  \idx{case-change} is applied to its optional argument. (If there was no
  optional argument in the original field value, one will be inserted.)
  The label argument is skipped.

  If the option is \optfmt{firstuc} then all the remaining tokens are
  skipped. If the option is \optfmt{title} then the subsequent tokens
  are skipped until a word-boundary is found.

  \item If the control sequence is \ics{glsdisp}, \ics{glslink},
  \ics{dglsdisp} or \ics{dglslink} then the \idx{case-change} will be
  applied to the appropriate argument. The optional argument (if
  present) and the label are skipped.

  If the option is \optfmt{firstuc} then all the remaining tokens are
  skipped. If the option is \optfmt{title} then the subsequent tokens
  are skipped until a word-boundary is found.

  \item If the control sequence has a known case variant, it will be
  substituted. For example, \ics{gls} will be changed to \ics{Gls}
  or \ics{GLS}. In some cases there isn't an appropriate variant.
  For example, \ics{glsentrytext} has a first-letter upper case 
  version \ics{Glsentrytext}, but not an all-caps version.

  If the option is \optfmt{firstuc} then all the remaining tokens are
  skipped. If the option is \optfmt{title} then the subsequent tokens
  are skipped until a word-boundary is found.

  \item If the control sequence is followed by a group, then the
   appropriate \idx{case-change} is applied to the group contents.
   Unlike step~\ref{case-change-group}, the \idx{case-change} isn't
   applied to the entire group content with \optfmt{firstuc} and
   \optfmt{title}. (Again, this follows the way that
   \ics{makefirstuc} and \ics{capitalisewords} work.)

   If there are subsequent groups, they won't be considered arguments, 
   but will be treated as groups, as per step~\ref{case-change-group}. 
   (This will only affect the \optfmt{title} setting as they will be 
   skipped by the \optfmt{firstuc} setting.) For complex cases,
   consider using a semantic command that hides non-textual context
   such as the \csfmt{strong} example described on page~\glsxtrpageref{strong}.

  \item Otherwise the control sequence is skipped.
  \end{enumerate}

\item Anything else is skipped.

\end{enumerate}

For example, if an entry is defined as:
\begin{codeenv}
\atentry{abbreviation}\marg{html,
 \field{short} = \marg{HTML},
 \field{long}  = \marg{hypertext markup language},
 \field{description}=\marg{a markup language for creating web pages}
}
\end{codeenv}
then:
\begin{codeenv}
\gls{GlsXtrLoadResources}\oarg{
  \csopt[lc]{short-case-change},
  \csopt[title]{long-case-change},
  \csopt[firstuc]{description-case-change}
}
\end{codeenv}
will make the entry behave as if it had been defined as:
\begin{codeenv}
\atentry{abbreviation}\marg{html,
 \field{short} = \marg{html},
 \field{long}  = \marg{Hypertext Markup Language},
 \field{description}=\marg{A markup language for creating web pages}
}
\end{codeenv}
whereas: 
\begin{codeenv}
\gls{GlsXtrLoadResources}\oarg{
  \csopt[lc-cs]{short-case-change},
  \csopt[title-cs]{long-case-change},
  \csopt[firstuc-cs]{description-case-change}
}
\end{codeenv}
will make the entry behave as if it had been defined as:
\begin{codeenv}
\atentry{abbreviation}\marg{html,
 \field{short} = \marg{\gls{bibglslowercase}\marg{HTML}},
 \field{long}  = \marg{\gls{bibglstitlecase}\marg{hypertext markup language}},
 \field{description}=\marg{\gls{bibglsfirstuc}\marg{a markup language for creating web pages}}
}
\end{codeenv}
If the given field is missing, no change is made, except under
certain circumstances (see the relevant resource option for details). 
For example, if an abbreviation is simply defined as:
\begin{codeenv}
\atentry{abbreviation}\marg{html,
 \field{short} = \marg{html},
 \field{long}  = \marg{hypertext markup language}
}
\end{codeenv}
then:
\begin{codeenv}
\gls{GlsXtrLoadResources}\oarg{
  \csopt[uc]{name-case-change},
  \csopt[title]{description-case-change}
}
\end{codeenv}
won't have an effect. Although the default \abbrstyle{long-short} abbreviation
style sets the \field{name} and \field{description} fields, \bibgls\ doesn't
have access to this information.

Remember that you can create missing fields by copying the value
from another field. So if the resource options are changed to:
\begin{codeenv}
\gls{GlsXtrLoadResources}\oarg{
  \csopt[uc]{name-case-change},
  \csopt[title]{description-case-change},
  \csopt[\field{short}=\field{name},\field{long}=\field{description}]{replicate-fields}
}
\end{codeenv}
then \bibgls\ will act as though the entry had been defined as:
\begin{codeenv}
\atentry{abbreviation}\marg{html,
 \field{short} = \marg{html},
 \field{long}  = \marg{hypertext markup language},
 \field{name} = \marg{HTML},
 \field{description}  = \marg{Hypertext Markup Language}
}
\end{codeenv}
If the \abbrstyle{long-short-sc} abbreviation style is set (before
\gls{GlsXtrLoadResources}) then this will override the default style
for the \field{name} and \field{description}, so
\code{\cs{gls}\marg{html}} will display the short form using
\code{\ics{textsc}\marg{html}} but the \field{name} in the glossary
will be displayed using just \code{HTML}.

Note that with \atentry{index} the \field{name} and \field{text}
fields will automatically be created if they are missing and
\csopt{name-case-change} is used. For example, if an entry is
defined as:
\begin{codeenv}
\atentry{index}\marg{duck}
\end{codeenv}
then \csopt[firstuc]{name-case-change} will make this entry behave
as though it was defined as:
\begin{codeenv}
\atentry{index}\marg{duck,
  \field{name} = \marg{Duck},
  \field{text} = \marg{duck}
}
\end{codeenv}

Suppose I have a slightly eccentric abbreviation definition:
\begin{codeenv}
\atentry{abbreviation}\marg{html,
  \field{short} = \qtdelim{ht\cs{emph}\marg{ml}},
  \field{long}  = \qtdelim{hypertext markup language}
}
\end{codeenv}
then \csopt[uc]{short-case-change} would convert the value of the
\field{short} field into:
\begin{codeenv}
HT\cs{emph}{ML}
\end{codeenv}
Note that \csfmt{emph} isn't modified as it's recognised as a
command. There's a difference between a group that follows a control
sequence and one that doesn't. For example:
\begin{codeenv}
\atentry{abbreviation}\marg{html,
  \field{short} = \qtdelim{\marg{ht}ml},
  \field{long}  = \qtdelim{hypertext markup language}
}
\end{codeenv}
In this case \csopt[firstuc]{short-case-change} will convert the
\field{short} field value to:
\begin{codeenv}
\marg{HT}ml
\end{codeenv}
(The entire contents of the group \code{\marg{ht}} has been
converted.) Whereas with:
\begin{codeenv}
\atentry{abbreviation}\marg{html,
  \field{short} = \qtdelim{\cs{emph}\marg{ht}ml},
  \field{long}  = \qtdelim{hypertext markup language}
}
\end{codeenv}
then \csopt[firstuc]{short-case-change} will convert the
\field{short} field value to:
\begin{codeenv}
\cs{emph}\marg{Ht}ml
\end{codeenv}
(Only the first letter of the argument \code{\marg{ht}} has been
converted.)

There's no attempt at interpreting the field contents at this point (but
the value may later be interpreted during sorting).
For example, suppose a \field{name} field is defined using:
\begin{codeenv}
\field{name} = \qtdelim{z\ics{ae}\ics{oe}},
\end{codeenv}
then with \csopt[uc]{name-case-change}, the value would be converted to
\begin{codeenv}
Z\ics{AE}\ics{OE}
\end{codeenv}
because \cs{ae} and \cs{oe} have known \idx{uppercase} versions.

With \csopt[uc-cs]{name-case-change}, the \field{name}
value would be converted to:
\begin{codeenv}
\gls{bibglsuppercase}\marg{z\cs{ae}\cs{oe}}
\end{codeenv}
If the interpreter is used during sorting, the sort value will be set to 
\code{Z\AE\OE} because the interpreter recognises all three commands.

You can use \code{\ics{NoCaseChange}\margm{text}} to prevent the given 
\meta{text} from having the case changed. For example, if the \field{short}
field is defined as:
\begin{codeenv}
\field{short} = \marg{a\cs{NoCaseChange}\marg{bc}d}
\end{codeenv}
then with \csopt[uc]{short-case-change}, this would be converted to
\begin{codeenv}
A\cs{NoCaseChange}\marg{bc}D
\end{codeenv}

Note that with \optfmt{firstuc} and \optfmt{title}, if
\code{\ics{MFUskippunc}\margm{text}} occurs at the start of a word
then it's skipped, and the case change is
applied to the material following its argument. For example,
suppose the \field{short} field is defined as:
\begin{codeenv}
\field{short}=\marg{\cs{MFUskippunc}\marg{h}tml}
\end{codeenv}
then the result is:
\begin{codeenv}
\cs{MFUskippunc}\marg{h}Tml
\end{codeenv}
whereas with:
\begin{codeenv}
\field{short}=\marg{\marg{}html}
\end{codeenv}
then the result is just \code{\marg{}html} (since the case change is
applied to the empty group, which has no effect).

If you have a command that takes a label or identifier as an argument then it's
best to hide the label in a custom command. For example, if the
\field{short} field in the \ext{bib} definition is defined as:
\begin{codeenv}
  \field{short} = \qtdelim{ht\ics{textcolor}\marg{red}\marg{ml}},
\end{codeenv}
then with \csopt[uc]{short-case-change} this would end up as:
\begin{codeenv}
HT\cs{textcolor}\marg{RED}\marg{ML}
\end{codeenv}
which is incorrect. Instead, provide a command that hides the label
(such as the \csfmt{strong} example described on page~\glsxtrpageref{strong}).

\optsection[\subsubsection]{no-case-change-cs}

Instructs the non-\TeX\ case-changing options (where \bibgls, not
\TeX, performs the modification) to treat the commands whose control
sequence names are given in the comma-separated \meta{list} in the
same way as it treats \cs{ensuremath} etc. That is, the case-change
is omitted for the argument that follows any of those commands.

For example, this manual defines some semantic commands such as
\csfmt{fieldfmt} (to format field names), \csfmt{abbrstylefmt}
(to format abbreviation style names) and \csfmt{glostylefmt} (to format
glossary style names). If any these occur in section
and subsection headings (which are converted to \idx{titlecase})
then the case-change would produce an inappropriate result. 
These formatting commands shouldn't have their argument changed so they 
are identified with:
\begin{codeenv}
\csopt[fieldfmt,abbrstylefmt,glostylefmt]{no-case-change-cs}
\end{codeenv}

\optsection[\subsubsection]{word-boundaries}

Governs how the \optfmt{title} \idx{case-change} option
determines word boundaries. The \meta{list} must contain one or
more of the following keywords:
\begin{description}
\item[\optfmt{white space}] any white space Unicode character
that is not a non-breakable space indicates a word-boundary;
\item[\optfmt{cs space}] the control sequences
\ics{space} or \ics{cs.space} indicate a word-boundary;
\item[\optfmt{dash}] a Unicode character that belongs to the
\idx{punctuationdash} block indicates a word-boundary;
\item[\optfmt{nbsp}] the \idx{nbspchar} active character or the Unicode
non-breakable characters \hex{00A0}, \hex{2007} and \hex{202F}
indicate a word-boundary.
\end{description}
Any keyword that is not listed indicates that particular setting is off.  This
option is not cumulative. Any subsequent use of
\csopt{word-boundaries} within the same set of resource options will
override previous settings.

The default setting is \csopt[white space,cs
space]{word-boundaries}, which excludes non-breakable
spaces and dashes.

Note that you can explicitly markup word-boundary punctuation using
\ics{MFUwordbreak}. For example:
\begin{codeenv*}
\field{name} = \marg{a book of rhyme\cs{MFUwordbreak}\marg{/}verse}
\end{codeenv*}

\optsection[\subsubsection]{short-case-change}

Applies a case-change to the \field{short} field (if present).
This option may take one of the values described above.

See \csopt{dual-short-case-change} to adjust the \field{dualshort}
field.

\optsection[\subsubsection]{long-case-change}

Applies a case-change to the \field{long} field (if present).
This option may take one of the values described above.

See \csopt{dual-long-case-change} to adjust the \field{duallong}
field.

\optsection[\subsubsection]{name-case-change}

Applies a case-change to the \field{name} field.
This option may take one of the values described above.

If the \field{text} field hasn't been set, the \field{name} value is
first copied to the \field{text} field. If the \field{name} field
hasn't been set (for example, with the \atentry{index}
\gls{entrytype}), it's copied from the fallback value (which depends
on the \gls{entrytype}, see \sectionref{sec:fallbacks}) unless the
\gls{entrytype} is \atentry{abbreviation} or \atentry{acronym}, in
which case if the \field{name} field is missing no action is
performed.

\optsection[\subsubsection]{description-case-change}

Applies a case-change to the \field{description} field (if present).
This option may take one of the values described above.

\optsection[\subsubsection]{field-case-change}

A general case-change instruction. The value should be a
comma-separated list of \code{\meta{field}\dequals\meta{setting}} for each
field that needs a case-change applied.  The value is required for this key but
may be empty, which indicates this option is switched off.

The \meta{setting} should be
the same as the permitted values for the above options. This
option is applied after all fields have been parsed but before
\csopt{interpret-fields}. If the
specified field is missing, the fallback for that field (if known,
see \sectionref{sec:fallbacks}) is copied
into the field. For example:
\begin{codeenv}
\csopt[\field{user1}=uc,\field{user2}={firstuc}]{field-case-change}
\end{codeenv}
This manual provides a custom storage key called
\fieldfmt{nametitle}:
\begin{codeenv}
\cs{glsxtrprovidestoragekey}\marg{\fieldfmt{nametitle}}\marg{}\marg{}
\end{codeenv}
The resource options copy the \field{name} value to this custom
field and convert \fieldfmt{nametitle} to \idx{titlecase}:
\begin{codeenv}
\csopt[\field{name}=\fieldfmt{nametitle}]{replicate-fields},
\csopt[\fieldfmt{nametitle}=title]{field-case-change},
\end{codeenv}
This means that it's possible to fetch the value of
\fieldfmt{nametitle} instead of \field{name}, which provides an
expandable \idx{titlecase} form that's suitable for the PDF
bookmarks. (Note that \LaTeX3 now provides some expandable
case-changing commands.)

This option isn't cumulative. If used multiple times in the same
\igls{resourceset}, the last instance will be the one used. If
the \keyvallist\ is missing, no general case-changing is applied
(the default).

\section{Field Fallbacks}
\label{sec:fallbacks}

The options in this section don't modify any field values but
provide instructions on what to do if \bibgls\ wants to know the
value of a field where the field hasn't been explicitly set. The most common
case is querying the \field{sort} field value with the default
\csopt[\field{sort}]{sort-field} setting. Being able to vary the
fallback used according to the \gls{entrytype} allows a more flexible
approach than explicitly setting the \field{sort} field in the
\ext{bib} file.

Note that if you specify a different field to use for the sort value 
with \csopt{sort-field} then the fallback for that field will be used
if that field is missing. The \field{sort} fallbacks will be
irrelevant if the \field{sort} field isn't being queried. If the
fallback system fails to provide a value for the field identified by
\csopt{sort-field} then \bibgls\ will follow the rule given by the
\csopt{missing-sort-fallback} setting.

If you require a complex sort value that can't be implemented by the
fallback system, you can use \csopt{assign-fields} to explicitly set
the \field{sort} field to a string expression
(\sectionref{sec:optstringconcat}). Bear in mind that if the
\field{sort} field is actually set to a value, either in the
\ext{bib} file or through resource options, then \emph{the
\field{sort} fallback won't be used} and the sort fallback options
describe in this section won't have any effect.

There are other fields that \bibgls\ may want to query that won't
necessarily be set in the \ext{bib} file but may be inferred from
another field. For example, if the \field{sort} field fallback
references the \field{name} field then the \field{name} field will
also need a fallback if it hasn't been set.

Another possibility is that the interpreter encounters content that
includes commands such as \ics{gls}.  Since the interpreter can't
tell at what point in the document the first use flag is changed,
\ics{gls} is treated as \ics{glstext} (and similarly \ics{glspl} is
treated as \ics{glsplural}) so the \field{text} (or \field{plural})
field will be queried by the interpreter.

The commands \ics{newglossaryentry} and \ics{longnewglossaryentry}
are the foundation for all commands that define glossary entries.
These commands both require that either the \field{name} or the
\field{parent} field are set. If the \field{name} is omitted, then
its value is obtained from the \gls{parententry}['s] name. The
\field{description} must also be provided but may be set to empty.
(Some entry types, such as \atentry{index}, will set 
\field{description} to empty if that field is missing, but for other
entry types, such as \atentry{entry}, the \field{description} is
required and will trigger a warning if omitted.)

All other entry definition commands, such as \ics{newabbreviation} and
\ics{glsxtrnewsymbol}, internally use one of those foundation
commands.\footnote{Or the internal command that both
\cs{newglossaryentry} and \cs{longnewglossaryentry} use.} In the
case of \ics{newabbreviation} (and \ics{newacronym}), the
\field{name} field is set by the style using values obtained from
the \field{short} and\slash or \field{long} fields. This is
information that \bibgls\ is unaware of and may guess incorrectly
when trying to determine an appropriate value for the \field{name}
field if it is omitted (which is typically the case) from
abbreviation entry types, such as \atentry{abbreviation} or
\atentry{acronym}.

The general \atentry{entry} entry type, uses the same rules as
\ics{newglossaryentry}:
\begin{description}
\item[\field{name}] If the \field{parent} field has been set, then
the parent's \field{name} field is used. If the parent's
\field{name} field isn't set, then the fallback for the parent's
\field{name} field is used (which will depend on the parent's
\gls{entrytype}). If neither the \field{name} nor the \field{parent}
field is set, then a warning is issued since at least one of those
fields must be set for \atentry{entry}.

\item[\field{text}] If the \field{text} field is missing, it's
obtained from the \field{name} field or the fallback for the
\field{name} field, if that hasn't been set.

\item[\field{plural}] If the \field{name} field has been set then
the \field{plural} value is obtained by appending \ics{glspluralsuffix}
to the value of the \field{text} field (or the fallback for the
\field{text} field, if that hasn't been set).

If the \field{name} field hasn't been set but the \field{parent}
field has been set, then the \field{plural} is obtained from the
parent's \field{plural} field. If the parent's \field{plural} field
hasn't been set then the fallback for that value will be used,
according to the parent's \gls{entrytype}.

\item[\field{first}] The fallback for the \field{first} field is
obtained from the \field{text} field (or the fallback for the
\field{text} field, if that hasn't been set).

\item[\field{firstplural}] The fallback for the \field{firstplural}
field is obtained by appending \ics{glspluralsuffix} to the value of
the \field{first} field, if that field has been set, otherwise it's
obtained from the \field{plural} field (or the fallback for the
\field{plural} field if that isn't set).

\end{description}

Note that although \bibgls\ follows the \ics{newglossaryentry} rules
in order to obtain the fallback, it doesn't explicitly set those
fields in the \ext{glstex} file if they weren't provided in the
\ext{bib} file or set using options such as \csopt{replicate-fields}
or \csopt{assign-fields}.

The exception to this is the \field{sort} field, which will be
obtained from the \field{name} field for most \glspl{entrytype} unless
overridden by one of the applicable sort fallback options, such as
\csopt{entry-sort-fallback}. If the designated fallback (such as
\field{name}) is missing, then the fallback value for that field
will be used.

The \atentry{index} and \atentry{indexplural} entry types are
slightly different. They have their own rules for obtaining the
value of the \field{name} field, and will explicitly set it in the
\ext{glstex} file via the helper commands \gls{bibglsnewindex}
and \gls{bibglsnewindexplural}.

In the case of \atentry{index}, if the \field{name} field is
missing, its value will be obtained from the entry's original label.
If the \field{sort} field is missing, its value is obtained from the
\field{name} field unless a different fallback is specified with
\csopt{custom-sort-fallbacks}.  The remaining fallbacks are as for
\atentry{entry}.

It's more complicated for \atentry{indexplural}, which has the following
fallback rules:
\begin{description}
\item[\field{name}] If the \field{name} field is missing, its value
is obtained from the entry's \field{plural} field (or the fallback
for the \field{plural} field, if that field is missing).

\item[\field{plural}] If the \field{plural} field is missing, its
value is obtained by appending \ics{glspluralsuffix} to the value
of the \field{text} field (or the fallback for the \field{text}
field, if that field is missing).

\item[\field{text}] If the \field{text} field is missing, its value
is obtained from the entry's original label.

\item[\field{sort}] If the \field{sort} field is missing, its value
is obtained from the \field{name} field unless a different fallback
is specified with \csopt{custom-sort-fallbacks}.
\end{description}
The remaining fallbacks are as for \atentry{entry}.

The most awkward of all the \glspl{entrytype} are, as indicated earlier,
the abbreviations where the field values such as \field{name} and
\field{text} are set by the abbreviation style. Therefore, there are resource
options specifically to identify the most appropriate fallback
values for abbreviations. The default is to use the value of the
\field{short} field as the fallback for the \field{name},
\field{sort} and \field{text} fields. If this is inappropriate for
your abbreviation style then you will need to use the options listed below to
provide more appropriate fallbacks. These options don't actually set
the \field{name} and \field{text} fields in the \ext{glstex} file
and don't include any style formatting (such as font changing
commands), which are irrelevant to \bibgls.\footnote{The \field{sort} field
will be set in the \ext{glstex} file as it's useful for debugging,
but it's typically irrelevant.}

For other entry types, see their description in \sectionref{sec:bib}.

\optsection{abbreviation-name-fallback}

The \glspl{entrytype} that define abbreviations (such as
\atentry{abbreviation} and \atentry{acronym}) will, by default,
fallback on the \field{short} field if the \field{name} field is
missing and it's required for some reason (for example, with
\csopt[name]{sort-field}).  If you prefer to
fallback on a different field, then you can use this option to
specify the field. For example,
\csopt[long]{abbreviation-name-fallback}.
The \meta{field} value must be a known field (not an internal field)
but can't be the \field{sort} field.

Note that the default fallback for the \field{sort} field for
abbreviations is given by \csopt{abbreviation-sort-fallback} which
is set to \field{short} not \field{name} by default. So changing the
fallback for the \field{name} field won't have an effect unless the
\field{sort} fallback is changed to \field{name} or
\csopt[\field{name}]{sort-field} is used or the \field{name} field
is referenced in an option such as \csopt{assign-fields}.

Field concatenation isn't available for this option.

\optsection{abbreviation-text-fallback}

Similar to \csopt{abbreviation-name-fallback} but for the
\field{text} field. The default fallback is the \field{short} field.
Field concatenation isn't available for this option.

Note that you can't have both
\csopt[\field{text}]{abbreviation-name-fallback} and
\csopt[\field{name}]{abbreviation-text-fallback} as it would cause
an infinite loop.

\optsection{abbreviation-sort-fallback}

The \glspl{entrytype} that define abbreviations (such as
\atentry{abbreviation} and \atentry{acronym}) will, by default,
fallback on the \field{short} field if the \field{sort} field is
missing (assuming \csopt[sort]{sort-field}). If you prefer to
fallback on a different field, then you can use this option to
specify the field. For example,
\csopt[long]{abbreviation-sort-fallback}. Note that if you use
\csopt[name]{sort-field}, then the fallback field will be given by
\csopt{abbreviation-name-fallback} if the \field{name} field is
omitted.

The \meta{field} may be a known field but not an internal
field. It can't be the \field{sort} field. It may also be
one of the keywords: \optfmt{id} (for the entry's label) or
\optfmt{original id} (for the entry's original label). The
\meta{field} may also be a composite in the form
\optfmt{\meta{field1}+\meta{field2}+\ldots} which indicates that the
sort value should be obtained by concatenating the values of given fields,
where the separator is given by \csopt{field-concat-sep}.

Note that \csopt{missing-sort-fallback} and
\csopt{custom-sort-fallbacks} override this setting.

\begin{important}
The \csopt{abbreviation-sort-fallback} setting is only used when \bibgls\
tries to access the \field{sort} field for an abbreviation and finds
that the field hasn't been set. This means that this setting has no effect 
if you explicitly set the \field{sort} field or if you change the field used 
for sorting (\csopt{sort-field}).
\end{important}

\optsection{entry-sort-fallback}

The regular \glspl{entrytype} (such as \atentry{entry} and
\atentry{dualentry}) will, by default, fallback on the \field{name}
field if the \field{sort} field is missing (assuming
\csopt[sort]{sort-field}). If you prefer to
fallback on a different field, then you can use this option to
specify the field.
Note that \csopt{missing-sort-fallback} and
\csopt{custom-sort-fallbacks} override this setting.

The \meta{field} may be a known field but not an internal
field. It can't be the \field{sort} field. It may also be
one of the keywords: \optfmt{id} (for the entry's label) or
\optfmt{original id} (for the entry's original label). The
\meta{field} may also be a composite in the form
\optfmt{\meta{field1}+\meta{field2}+\ldots} which indicates that the
sort value should be obtained by concatenating the values of given fields,
where the separator is given by \csopt{field-concat-sep}.

This setting doesn't affect the index type of entries, such as
\atentry{index} or \atentry{indexplural}. This is useful if your
glossary contains \iglspl{homograph} (terms with the same spelling) which
can't be distinguished by the sort comparators. For example, suppose
my file \filefmt{entries.bib} contains:
\begin{codeenv}
\atentry{index}\marg{pagelist,
  \field{name}=\marg{page list},
  \field{description}=\marg{a list of individual pages or page ranges}
}
\strut
\atentry{index}\marg{glossary}
\strut
\atentry{entry}\marg{glossarylist,
  \field{parent}=\marg{glossary},
  \field{description}=\marg{list of technical words}
}
\strut
\atentry{entry}\marg{glosscol,
  \field{parent}=\marg{glossary},
  \field{description}=\marg{collection of glosses}
}
\end{codeenv}
Now first consider a document that uses the default settings:
\begin{codeenv}
\cmd{documentclass}\marg{article}
\strut
\cmd{usepackage}\oarg{\styopt{record},\styopt{subentrycounter},\styopt[\glostyle{treenoname}]{style}}\marg{glossaries-extra}
\strut
\gls{GlsXtrLoadResources}\oarg{\csopt[entries]{src}}
\strut
\cmd{begin}\marg{document}
A test document describing \cs{glspl}\marg{pagelist} and
\cs{gls}\marg{glosscol} (collection) vs \cs{gls}\marg{glossarylist} (list).
\strut
\cs{printunsrtglossary}
\cmd{end}\marg{document}
\end{codeenv}
The default behaviour for \atentry{entry} if the \field{sort} field
is missing is to fallback on the \field{name} field. If the
\field{name} field is missing (as with \code{glossarylist} and
\code{glosscol}), then the value is obtained from the
\field{name} field from the \gls{parententry}. The \gls{parententry} for these
\glspl{homograph} is the \code{glossary} entry, which was defined with
\atentry{index} and doesn't have the \field{name} field. For the
\atentry{index} entries, if \field{name} is missing the value is
obtained from the label.

Therefore both \code{glossarylist} and \code{glosscol} end
up with the same sort value: \code{glossary}. This triggers a
message in verbose mode (\commonlongarg{verbose}) which can be found 
in the transcript file:
\begin{verbatim}
Identical sort values for 'glossarylist' and 'glosscol'
Falling back on ID
\end{verbatim}
So the actual sort values used are \qt{glossarylist} and
\qt{glosscol}. This puts the \code{glossarylist} entry
before the \code{glosscol} entry.

Now suppose a minor modification is made to the document:
\begin{codeenv}
\gls{GlsXtrLoadResources}
 \oarg{
   \csopt[entries]{src},
   \csopt[description]{entry-sort-fallback}
 }
\end{codeenv}
This means that when the sort function fails to find the
\field{sort} field for the terms defined with \atentry{entry}, it
will fallback on the \field{description} field. This doesn't affect
the terms defined with \atentry{index}, which still fallback on the
\field{name} field. This time there's no message in the transcript
file and the \code{glosscol} entry now comes before the
\code{glossarylist} entry.

\begin{important}
The \csopt{entry-sort-fallback} setting is only used when \bibgls\
tries to access the \field{sort} field for a term defined with
\atentry{entry} and finds that the field hasn't been set. This means
that this setting has no effect if you explicitly set the \field{sort}
field or if you change the field used for sorting
(\csopt{sort-field}).
\end{important}

\optsection{symbol-sort-fallback}

The \glspl{entrytype} that define symbols (such as \atentry{symbol} and
\atentry{number}) will, by default, fallback on the entry's original label if the
\field{sort} field is missing (assuming the default
\csopt[sort]{sort-field}). If you prefer to fallback on a different
field, then you can use this option to specify the field. For
example, \csopt[name]{symbol-sort-fallback}.

The \meta{field} may be a known field but not an internal
field. It can't be the \field{sort} field. It may also be
one of the keywords: \optfmt{id} (for the entry's label) or
\optfmt{original id} (for the entry's original label). The
\meta{field} may also be a composite in the form
\optfmt{\meta{field1}+\meta{field2}+\ldots} which indicates that the
sort value should be obtained by concatenating the values of given fields,
where the separator is given by \csopt{field-concat-sep}.

Note that \csopt{missing-sort-fallback} and
\csopt{custom-sort-fallbacks} override this setting.

\begin{important}
The \csopt{symbol-sort-fallback} setting is only used when \bibgls\
tries to access the \field{sort} field for a symbol and finds
that the field hasn't been set. This means that this setting has no effect 
if you explicitly set the \field{sort} field or if you change the field used 
for sorting (\csopt{sort-field}).
\end{important}

\optsection{bibtexentry-sort-fallback}

The \glsdisp{mainentry}{main} \atentry{bibtexentry}
\glspl{entrytype} will, by default,
fallback on the \field{name} if the
\field{sort} field is missing (assuming the default
\csopt[sort]{sort-field}). If you prefer to fallback on a different
field, then you can use this option to specify the field.

The \meta{field} may be a known field but not an internal
field. It can't be the \field{sort} field. It may also be
one of the keywords: \optfmt{id} (for the entry's label) or
\optfmt{original id} (for the entry's original label). The
\meta{field} may also be a composite in the form
\optfmt{\meta{field1}+\meta{field2}+\ldots} which indicates that the
sort value should be obtained by concatenating the values of given fields,
where the separator is given by \csopt{field-concat-sep}.

Note that \csopt{missing-sort-fallback} and
\csopt{custom-sort-fallbacks} override this setting.

\begin{important}
The \csopt{bibtexentry-sort-fallback} setting is only used when
\bibgls\ tries to access the \field{sort} field for a \gls{mainentry}
defined with \atentry{bibtexentry} and finds that the field hasn't
been set. This means that this setting has no effect if you
explicitly set the \field{sort} field or if you change the field used for
sorting (\csopt{sort-field}).
\end{important}

\optsection{custom-sort-fallbacks}

The value should be a \keyvallist\ in the form
\meta{entrytype}\dequals\meta{field} where \meta{entrytype} is the
\emph{original} \gls{entrytype} (before being aliased with
\csopt{entry-type-aliases}). This will override any of the sort fallback
options listed below for entries whose original \gls{entrytype} matches
\meta{entrytype}.

The \meta{field} may be a known field but not an internal field. For
obvious reasons, it can't be the \field{sort} field (since
\meta{field} is the fallback a missing \field{sort} field). It may also be one of the
keywords: \optfmt{id} (for the entry's label) or \optfmt{original id}
(for the entry's original label). The \meta{field} may also be a
composite in the form \optfmt{\meta{field1}+\meta{field2}+\ldots}
which indicates that the sort value should be obtained by
concatenating the values of the given fields, where the separator is
given by \csopt{field-concat-sep}.

For example, if the \ext{bib} file contains:
\begin{codeenv}
\atentryfmt{unit}\marg{ohm,
  \field{name}=\marg{\cs{si}\marg{\csfmt{ohm}}},
  \field{description}=\marg{electrical resistance}
}
\strut
\atentryfmt{constant}\marg{pi,
  \field{name}=\marg{\cs{ensuremath}\marg{\cs{pi}}},
  \field{description}=\marg{the ratio of the length of the circumference
    of a circle to its diameter},
  \field{user1}=\marg{3.14159}
}
\strut
\atentry{symbol}\marg{fx,
  \field{name}=\marg{\cs{ensuremath}\marg{f(x)}},
  \field{description}=\marg{a function of \idx{mshiftchar}x\idx{mshiftchar}}
}
\strut
\atentry{number}\marg{zero,
  \field{name}=\marg{0},
  \field{description}=\marg{nothing or no quantity}
}
\end{codeenv}
Then the resource options:
\begin{codeenv}
\csopt[\entryfmt{unit}=\entryref{symbol},\entryfmt{constant}=\entryref{number}]{entry-type-aliases},
\csopt[\entryfmt{unit}=\field{name},\entryfmt{constant}=\field{user1}]{custom-sort-fallbacks}
\end{codeenv}
will treat the custom \atentryfmt{unit} and \atentryfmt{constant}
entries as though they had been defined with \atentry{symbol}
and \atentry{number}, respectively, but the fallback for the
\field{sort} field is different: the \code{ohm} entry will use the
\field{name} field for the sort fallback (because its original 
\gls{entrytype} was \entryfmt{unit}), the \code{pi} entry will use the
\field{user1} field for the sort fallback (because its original
\gls{entrytype} was \entryfmt{constant}) and the \code{fx} and
\code{zero} entries will use the label for the sort fallback (since
neither \entryfmt{symbol} nor \entryfmt{number} were identified in
\csopt{custom-sort-fallbacks} so the \csopt{symbol-sort-fallback} is
used).

If an entry hasn't had its \gls{entrytype} aliased then \meta{entrytype}
is its actual \gls{entrytype}. For example, consider the following
definitions:
\begin{codeenv}
\atentry{abbreviation}\marg{svm,
   \field{short}=\marg{SVM},
   \field{long}=\marg{support vector machine}
}
\atentry{acronym}\marg{laser,
  \field{short}=\marg{laser},
  \field{long}=\marg{light amplification by stimulated emission of radiation}
}
\end{codeenv}
Then \csopt[short]{abbreviation-sort-fallback} will make both
entries fallback on the \field{short} field (since
\csopt{abbreviation-sort-fallback} applies to both \atentry{acronym}
and \atentry{abbreviation}), but the option:
\begin{codeenv}
\csopt[\entryref{abbreviation}=\field{long},\entryref{acronym}=\field{short}]{custom-sort-fallbacks} 
\end{codeenv}
will make the entry defined with \atentry{abbreviation} fallback on
the \field{long} field and the entry defined with \atentry{acronym}
will fallback on the \field{short} field.

Since the default setting is \csopt[short]{abbreviation-sort-fallback}
this only needs to be:
\begin{codeenv}
\csopt[\entryref{abbreviation}=\field{long}]{custom-sort-fallbacks} 
\end{codeenv}
In this case, the entry defined with \atentry{abbreviation} (\qt{SVM}) will use
the setting given in \csopt{custom-sort-fallbacks}, but the entry
defined with \atentry{acronym} (\qt{laser}) will use the setting given by
\csopt{abbreviation-sort-fallback} since \atentry{acronym} hasn't
been identified in \csopt{custom-sort-fallbacks}.

This option also covers \glspl{dualentry}. For example:
\begin{codeenv}
\csopt[
   \entryref{dualindexnumber}=\field{description},
   \entryfmt{dualindexnumbersecondary}=\field{user1}
]{custom-sort-fallbacks}
\end{codeenv}
Note that the \gls{entrytype} for the \dual\ is in the form 
\code{\meta{primary entry type}secondary}.

\begin{important}
The \csopt{custom-sort-fallbacks} setting is only used when \bibgls\
tries to access the \field{sort} field for an entry (whose original
\gls{entrytype} has been identified in this setting) and finds that the field
hasn't been set. This means that this setting has no effect if you
explicitly set the \field{sort} field or if you change the field
used for sorting (\csopt{sort-field}).
\end{important}

\optsection{field-concat-sep}

This option sets the field concatenation separator to \meta{value}
used by the \field{sort} fallback options.  The default is a space.
An empty value indicates no separator. You may use
\gls{uhex}\meta{hex} to indicate a character by its hexadecimal code
(see \sectionref{sec:quarks}). Note that the more complex field
concatenation specification described in
\sectionref{sec:optstringconcat} isn't available for this option.

For example, suppose the \ext{bib} file contains:
\begin{codeenv}
\atentry{abbreviation}\marg{ac,
  \field{short}=\marg{AC},
  \field{long}=\marg{alternating current}
}
\atentry{index}\marg{acacia}
\end{codeenv}
Then the resource option:
\begin{codeenv}
\gls{GlsXtrLoadResources}\oarg{
  \csopt[letter-nocase]{sort},
  \csopt[\field{short}+\field{long}]{abbreviation-sort-fallback}
}
\end{codeenv}
will set the \code{ac} sort value to \qt{AC alternating current}.
That is, the \field{short} value concatenated with the \field{long}
value using the default space separator. With the
\optfmt{letter-nocase} sort method, this will put the \code{ac}
entry before the \code{acacia} entry (because the space character
comes before \qt{a}).

If the resource options are changed to:
\begin{codeenv}
\gls{GlsXtrLoadResources}\oarg{
  \csopt[letter-nocase]{sort},
  \csopt[\empty]{field-concat-sep},
  \csopt[\field{short}+\field{long}]{abbreviation-sort-fallback}
}
\end{codeenv}
This will obtain the sort value for abbreviations from a
concatenation of the \meta{short} and \meta{long} values without a
separator. This means that the \code{ac} sort value will be
\qt{ACalternating current} and so the \code{ac} entry will
come after the \code{acacia} entry (since \qt{l} comes after \qt{c}).

This setting is only used for the sort fallback options that allow
field concatenation (such as \csopt{entry-sort-fallback} but not
\csopt{missing-sort-fallback}). 

Note that due to the way that the \keyvallist\ parser trims leading and
trailing spaces, you can't simply do \csopt[~]{field-concat-sep}
to indicate a space character as the value will end up as an empty
string. You can instead do \csopt[\stringu20]{field-concat-sep} but
since this is the default value there shouldn't be much need for it.

Remember that the separator may be replaced with a break point
marker depending on the sort method and \csopt{break-at} setting.

\section{Plurals}
\label{sec:plurals}

Some languages, such as English, have a general rule that plurals
are formed from the singular with a suffix appended. This isn't 
an absolute rule. There are plenty of exceptions (for example,
geese, children, churches, elves, fairies, sheep, mice), so a
simplistic approach of just doing \code{\ics{gls}\margm{label}[s]}
will sometimes produce inappropriate results, so the \styfmt{glossaries}
package provides a \field{plural} key with the corresponding command
\ics{glspl}.

In some cases a plural may not make any sense (for example, if the
term is a verb or symbol), so the \field{plural} key is optional, but to
make life easier for languages where the majority of plurals can
simply be formed by appending a suffix to the singular, the
\styfmt{glossaries} package lets the \field{plural} field default
to the value of the \field{text} field with \ics{glspluralsuffix}
appended. This command is defined to be just the letter \qt{s}.
This means that the majority of terms in such languages don't need to have the
\field{plural} supplied as well, and you only need to use it for the
exceptions.

For languages that don't have this general rule, the \field{plural}
field will always need to be supplied for nouns.

There are other plural fields, such as \field{firstplural},
\field{longplural} and \field{shortplural}. Again, if you are using
a language that doesn't have a simple suffix rule, you'll have to
supply the plural forms if you need them (and if a plural makes
sense in the context).

If these fields are omitted, the \styfmt{glossaries} package follows
these rules:
\begin{itemize}
\item If \field{firstplural} is missing, then \ics{glspluralsuffix}
is appended to the \field{first} field, if that field has been
supplied. If the \field{first} field hasn't been supplied but the
\field{plural} field has been supplied, then the \field{firstplural}
field defaults to the \field{plural} field. If the \field{plural}
field hasn't been supplied, then both the \field{plural} and
\field{firstplural} fields default to the \field{text} field (or
\field{name}, if no \field{text} field) with \ics{glspluralsuffix}
appended.

\item If the \field{longplural} field is missing, then 
\ics{glspluralsuffix} is appended to the \field{long} field, if the
\field{long} field has been supplied.

\item If the \field{shortplural} field is missing then, \emph{with
the base \styfmt{glossaries} acronym mechanism}, \ics{acrpluralsuffix}
is appended to the \field{short} field.

\end{itemize}

The last case is different with the \isty{glossaries-extra} extension 
package. The \field{shortplural} field defaults to the \field{short}
field with \ics{abbrvpluralsuffix} appended \emph{unless overridden
by category attributes}. This suffix command is set by the
abbreviation styles. This means that every time an abbreviation
style is implemented, \ics{abbrvpluralsuffix} is redefined. Most
styles simply define this command as:
\begin{codeenv}
\cs{renewcommand}*\marg{\cs{abbrvpluralsuffix}}\marg{\cs{glsxtrabbrvpluralsuffix}}
\end{codeenv}
where \ics{glsxtrabbrvpluralsuffix} expands to \ics{glspluralsuffix}.
The \qt{sc} styles (such as \abbrstyle{long-short-sc}) use a different
definition:
\begin{codeenv}
\cs{renewcommand}*\marg{\cs{abbrvpluralsuffix}}\marg{\cs{protect}\ics{glsxtrscsuffix}}
\end{codeenv}
This allows the suffix to be reverted back to the upright font,
counteracting the affect of the small-caps font.

This means that if you want to change or strip the suffix used for
the plural short form, it's usually not sufficient to redefine
\ics{abbrvpluralsuffix}, as the change will be undone the next time
the style is applied. Instead, for a document-wide solution, you
need to redefine \ics{glsxtrabbrvpluralsuffix}. Alternatively you can
use the category attributes.

There are two attributes that affect the short plural suffix
formation. The first is \catattr{aposplural} which uses the suffix
\begin{codeenv*}
\idx{aposchar}\cs{abbrvpluralsuffix}
\end{codeenv*}
That is, an \idx{apostrophe} followed by \ics{abbrvpluralsuffix} is
appended. The second attribute is \catattr{noshortplural} which
suppresses the suffix and simply sets \field{shortplural} to the
same as \field{short}.

With \bibgls, if you have some abbreviations where the plural should
have a suffix and some where the plural shouldn't have a suffix
(for example, the document has both English and French abbreviations)
then there are two approaches.

The first approach is to use the category attributes. For example:
\begin{codeenv}
\cs{glssetcategoryattribute}\marg{french}\marg{\catattr{noshortplural}}
\end{codeenv}
Now just make sure all the French abbreviations are have their
\field{category} field set to \optfmt{french}:
\begin{codeenv}
\gls{GlsXtrLoadResources}\oarg{\csopt[fr-abbrvs]{src},\csopt[french]{category}}
\end{codeenv} 

The other approach is to use the options listed below for the given
\igls{resourceset}. For example:
\begin{codeenv}
\gls{GlsXtrLoadResources}\oarg{\csopt[fr-abbrvs]{src},\csopt[\empty]{short-plural-suffix}}
\end{codeenv} 

\optsection{short-plural-suffix}

Sets the plural suffix for the default \field{shortplural} to
\meta{value}. The \meta{value} may be one of:
\begin{itemize}
\item \meta{suffix}: add the
\field{shortplural} field, if missing, with the given \meta{suffix}. 
\item \meta{empty}: add the
\field{shortplural} field, if missing, with no suffix.
\item \optfmt{use-default}: leave
it to \styfmt{glossaries-extra} to determine the appropriate default.
\end{itemize}
The default setting is \csopt[use-default]{short-plural-suffix}.
If the \code{=\meta{value}} part is omitted, then
\csopt[\empty]{short-plural-suffix} is assumed.

\optsection{dual-short-plural-suffix}

Sets the plural suffix for the default \field{dualshortplural} field to
\meta{value}. As with \csopt{short-plural-suffix}, the default
setting is \csopt[use-default]{dual-short-plural-suffix}. If the
\meta{value} is omitted or empty, the suffix is set to empty.

\section{Location List Options}
\label{sec:locationopts}

The \styopt{record} package option automatically adds two new keys: 
\field{loclist} and \field{location}. These two fields are set by
\bibgls\ from the information supplied in the \iext{aux} file (unless
the option \csopt[false]{save-locations} is used). The
\field{location} field contains the code to typeset the formatted
\igls{locationlist}.

Note that the cross-referencing information provided with the
\field{see}, \field{seealso} and \field{alias} fields is put in the
location list. If you only want the cross-reference and not any of
the locations, use \csopt[see]{save-locations} (or similar).

The \field{loclist} field has the syntax of an \isty{etoolbox}
internal list and includes every \gls{location} (except for the
\glsdisp{discardedrecord}{discarded duplicates} and
\iglspl{ignoredrecord}) with no \idx{range} formations.  Any \idx{explicit-range}
markup is stripped from the \glsopt{format} information to leave
just the \idx{encap} name, so you just get the start and end
\glspl{location} added as individual elements but they are still
encapsulated with the associated formatting command. Each item in
the list is provided in one of the following forms:
\begin{codeenv}
\ics{glsseeformat}\oargm{tag}\margm{label list}\marg{}
\end{codeenv}
for the cross-reference supplied by the \field{see} field,
\begin{codeenv}
\ics{glsxtruseseealsoformat}\margm{xr list}
\end{codeenv}
for the cross-reference supplied by the \field{seealso} field,
\nosecdef{glsnoidxdisplayloc}
for standard the internal \glspl{location}, 
\begin{codeenv}
\format{glsxtrdisplaysupploc}
\end{codeenv}
for \glsdisp{supplementalrecord}{supplemental (external) locations} and
\begin{codeenv}
\format{glsxtrdisplaylocnameref}
\end{codeenv}
for \code{nameref} \glspl{record}. (See \sectionref{sec:supplementalopts}
for more information about \glsdisp{supplementalrecord}{supplemental locations} and
\longarg{merge-nameref-on} for more information about \code{nameref}
records.)

You can iterate through the \field{loclist} value
using one of \sty{etoolbox}'s internal list \idxpl{loop} (either
by first fetching the list using \ics{glsfieldfetch}
or through \styfmt{glossaries-extra}'s \ics{glsxtrfielddolistloop}
or \ics{glsxtrfieldforlistloop} shortcuts).

The \meta{format} is that supplied by the \glsopt{format} key
when using commands like \ics{gls} or \ics{glsadd} (the encapsulator or 
\idx{encap} in \idx!{makeindex} parlance). If omitted, the default
\glsopt[glsnumberformat]{format} is assumed (unless this
default value is changed with \ics{GlsXtrSetDefaultNumberFormat}.
The value of the \glsopt{format} key must be the name of 
a text-block command without the leading backslash that takes 
a single argument (the \gls{location}). The \gls{location}
is encapsulated by that command. For example, 
\begin{codeenv}
\cs{gls}\oarg{\glsopt[\encap{textbf}]{format}}\marg{sample}
\end{codeenv}
will display the corresponding \gls{location} in bold, but note that this
will no longer have a hyperlink if you've used \sty{hyperref}.
If you want to retain the hyperlink you need the \gls{location}
encapsulated with \cs{hyperbf} instead of \cs{textbf}:
\begin{codeenv}
\cs{gls}\oarg{\glsopt[\encap{hyperbf}]{format}}\marg{sample}
\end{codeenv}
The \csfmt{hyper}\meta{xx} set of commands all internally use
\cs{glshypernumber} which adds the appropriate hyperlink to the
\gls{location}. See Table~6.1 in the \styfmt{glossaries}~\cite{glossaries}
user manual for a list of all the \csfmt{hyper}\meta{xx} commands.

\Idxpl{range} can be explicitly formed using the parenthetical 
syntax \glsopt[\idx{openrange}]{format} and
\glsopt[\idx{openrange}]{format} or \glsopt[\idx{openrange}\meta{csname}]{format} and
\glsopt[\idx{closerange}\meta{csname}]{format} (where \meta{csname} is again the name of 
a text-block command without the initial backslash) in the optional
argument of commands like \ics{gls} or \ics{glsadd}. With
\sty{glossaries-extra} v1.50+, you can also use \ics{glsstartrange}
and \ics{glsendrange} (which is useful if the unbalanced parentheses
upset syntax highlighting).

These \idxpl{explicit-range} will always form a \idx{range},
regardless of \csopt{min-loc-range}, unless the start and end
coincide and \longarg{collapse-same-location-range} is in effect.
The \idx{explicit-range} will be encapsulated with \gls!{bibglsrange}
(unless \csopt[true]{merge-ranges}). (This command is not used with
\idxpl{implicit-range} that are formed by collating consecutive \glspl{location}.)
The initial marker is stripped from the \meta{format} argument of
the \gls{location} formatting commands, such as
\gls{glsnoidxdisplayloc}, to allow for easy conversion to the
corresponding text-block command.

\Idxpl{explicit-range} don't merge with neighbouring
\glspl{location} (unless \csopt[true]{merge-ranges}), but will
absorb any individual \glspl{location} within the \idx{range} that doesn't
conflict.  (Conflicts, denoted \idxpl{interloper}, will be moved to the start of the
\idx{explicit-range}, regardless of \csopt{merge-ranges}.)
For example, if \code{\cs{gls}\marg{sample}} is used on page~1,
\code{\cs{gls}\oarg{\glsopt[\idx{openrange}]{format}}\marg{sample}}
is used on page~2, \code{\cs{gls}\marg{sample}} is used on page~3, and
\code{\cs{gls}\oarg{\glsopt[\idx{closerange}]{format}}\marg{sample}}
is used on page~4, then the \gls{locationlist} will be 1, 2--4. The
entry on page~3 is absorbed into the \idx{explicit-range}, but, with
the default \csopt[false]{merge-ranges}, the 
\idx{range} can't be expanded to include page~1.  If the entry on page~3 had a
different format to the \idx{explicit-range}, for example
\code{\cs{gls}\oarg{\glsopt[\encap{textbf}]{format}}\marg{sample}}
then this will cause a warning and the \idx{interloper} will be
moved before the start of the \idx{range} so that the
\gls{locationlist} would then be 1, \textbf{3}, 2--4. 

The \csopt[true]{merge-ranges} option will make \idxpl{explicit-range}
behave like \idxpl{implicit-range}, which allows them to merge with
neighbouring \idxpl{range}. The \gls{bibglsrange} command won't be used in
this case (regardless of whether or not the \idx{range} was merged with
neighbouring \glspl{location}). Options such as
\csopt{min-loc-range} won't have an effect on the merged \idx{range}, but
will still effect \idxpl{implicit-range} that haven't been merged with an
\idx{explicit-range}.

An \igls{ignoredrecord} identifies a term that needs to be treated as
though it has a \gls{record} for selection purposes, but the
\gls{record} should not be included in the \gls{locationlist}.
The special format \glsaddopt[\encap{glsignore}]{format} is provided
by the \styfmt{glossaries} package for cases where the \gls{location}
should be ignored. (The command \ics{glsignore} simply ignores its
argument.) This works reasonably well if an entry only has the one
\gls{location}, but if the entry happens to be indexed again, it can lead
to an odd empty gap in the \gls{locationlist} with a spurious comma. If
\bibgls\ encounters a \gls{record} with this special format, the entry
will be selected but the \gls{record} will be discarded.

This means that the \gls{locationlist} will be empty if the entry was
only indexed with the special ignored format, but if the entry was also
indexed with another format then the \gls{locationlist} won't include the
\iglspl{ignoredrecord}. (This format is used by \ics{glsaddallunused} but
remember that iterative commands like this don't work with \bibgls.
Instead, just use \csopt[all]{selection} to select all entries.
Those that don't have \glspl{record} won't have a \gls{locationlist}.)

For example, suppose you only want main matter \glspl{location} in the
number list, but you want entries that only appear in the back matter 
to still appear in the glossary (without a \gls{locationlist}), then you could do:
\begin{codeenv}
\ics{backmatter}
\ics{GlsXtrSetDefaultNumberFormat}\marg{\encap{glsignore}}
\end{codeenv}
If you also want to drop front matter \glspl{location} as well:
\begin{codeenv}
\ics{frontmatter}
\cs{GlsXtrSetDefaultNumberFormat}\marg{\encap{glsignore}}
\ldots
\ics{mainmatter}
\cs{GlsXtrSetDefaultNumberFormat}\marg{\encap{glsnumberformat}}
\ldots
\cs{backmatter}
\cs{GlsXtrSetDefaultNumberFormat}\marg{\encap{glsignore}}
\end{codeenv}

Note that \idx{explicit-range} formations aren't discarded, so if
\encap{glsignore} is used in a \idx{range}, such as:
\begin{codeenv}
\ics{glsadd}\oarg{\glsopt[\idx{openrange}\encap{glsignore}]{format}}\marg{sample}
\ldots 
\cs{glsadd}\oarg{\glsopt[\idx{closerange}\encap{glsignore}]{format}}\marg{sample} 
\end{codeenv}
then the \idx{range} will be included in the \gls{locationlist} (encapsulated
with \ics{glsignore}), but this case would be a rather odd use of
this special format and is not recommended.

The \gls{record} counting commands, such as \gls{rgls}, use the special
format \encap{glstriggerrecordformat}, which \bibgls\ also treats
as an \igls{ignoredrecord} and the same rules as for \encap{glsignore} apply.

The \glspl{location} are always listed in the order in which they were indexed,
(except for the cross-reference which may be placed at the start or
end of the list or omitted).
This is different to \idx!{xindy} and \idx!{makeindex} where you can
specify the ordering (such as \idx!{lowercase} Roman first, then digits,
etc), but unlike those applications, \bibgls\ allows any \gls{location},
although it may not be able to work out an integer representation.
(With \idx!{xindy}, you can define new \gls{location} formats, but you need
to remember to add the appropriate code to the custom module.)

It's possible to define a custom glossary style where 
\ics{glossentry} (and the \child\ form \ics{subglossentry}) ignore the
final argument (which will be the \field{location} field)
and instead parse the \field{loclist} field and re-order the
\glspl{location} or process them in some other way. Remember that you can
also use \ics{glsnoidxloclist}
provided by \styfmt{glossaries}. For example:
\begin{codeenv}
\cs{glsfieldfetch}\marg{gls.sample}\marg{loclist}\marg{\cmd{loclist}}\comment{fetch location list}
\cs{glsnoidxloclist}\marg{\cmd{loclist}}\comment{iterate over locations}
\end{codeenv}
This uses \ics{glsnoidxloclisthandler} as the list's \idx{handler}
macro, which simply displays each \gls{location} separated by \ics{delimN}.
(See also
\href{http://www.dickimaw-books.com/latex/admin/html/foreachtips.shtml}{Iteration
Tips and Tricks}~\cite{iterationtips}.)

Each regular \gls{location} is listed in the \iext{aux} file in the form:
\nosecdef{glsxtr@record}
(See \longarg{merge-nameref-on} for \code{nameref} records.)
Exact duplicates are discarded. For example, if \code{cat}
is indexed twice on page 1:
\begin{codeenv}
\gls{glsxtr@record}\marg{cat}\marg{}\marg{\counter{page}}\marg{\encap{glsnumberformat}}\marg{1}
\gls{glsxtr@record}\marg{cat}\marg{}\marg{\counter{page}}\marg{\encap{glsnumberformat}}\marg{1}
\end{codeenv}
then the second \gls{record} is discarded. Only the first
\gls{record} is added to the \gls{locationlist}.

Partial duplicates, where all arguments match except for
\meta{format}, may be discarded depending on the value
of \meta{format}. For example, if page~1 of the document
uses \code{\cs{gls}\marg{cat}} and
\code{\cs{gls}\oarg{\glsopt[\encap{hyperbf}]{format}}\marg{cat}}
then the \ext{aux} file will contain:
\begin{codeenv}
\gls{glsxtr@record}\marg{cat}\marg{}\marg{\counter{page}}\marg{\encap{glsnumberformat}}\marg{1}
\gls{glsxtr@record}\marg{cat}\marg{}\marg{\counter{page}}\marg{\encap{hyperbf}}\marg{1}
\end{codeenv}
This is a partial \gls{record} match. In this case, \bibgls\ 
makes the following tests:
\begin{itemize}
\item If one of the formats includes an \idx{explicit-range} formation marker, the
\idx{range} takes precedence.

\item If one of the formats is \encap{glsnumberformat} (as in the
above example) or an \igls{ignoredrecord} format such as
\encap{glsignore}, that format will be skipped.  So in the above
example, the second \gls{record} will be added to the \gls{locationlist}, but
not the first. (A message will only be written to the transcript if
the \commonlongarg{debug} switch is used.) The default
\encap{glsnumberformat} will take precedence over the
\igls{ignoredrecord} formats (\encap{glsignore} and
\encap{glstriggerrecordformat}).

\item If a mapping has been set with the \longarg{map-format}
switch that mapping will be checked.
\item Otherwise the duplicate \gls{record} will be discarded with a
warning.
\end{itemize}

The \field{location} field is used to store the formatted
\gls{locationlist}. The code for this list is generated by \bibgls\ based on the
information provided in the \iext{aux} file, the presence of the
\field{see} or \field{seealso} field and the various settings described in this
chapter. When you display the glossary using \ics{printunsrtglossary},
if the \field{location} field is present it will be displayed
according to the glossary style (and other factors, such as whether the
\styopt{nonumberlist} option has been used, either as a package
option or supplied in the optional argument of \ics{printunsrtglossary}).
For more information on adjusting the formatting see the
\styfmt{glossaries}~\cite{glossaries} and 
\styfmt{glossaries-extra}~\cite{glossaries-extra} user manuals.

\optsection{save-locations}

This was originally a boolean setting, but as from v3.0 there are
additional values.
\begin{itemize}
\item \optfmt{false}: don't save anything in the \field{location}
field;

\item \optfmt{true}: save cross-references and all non-ignored 
locations in the \field{location} field;

\item \optfmt{see}: only save cross-references (\field{see},
\field{seealso} and \field{alias}) in the \field{location} field;

\item \optfmt{see not also}: only save the \field{see} and
\field{alias} cross-references (not \field{seealso}) in the
\field{location} field;

\item \optfmt{alias only}: only save the \field{alias}
cross-references (not \field{see} or \field{seealso}) in the
\field{location} field.

\end{itemize}

By default, the \glspl{location} will be processed and stored in the
\field{location} and \field{loclist} fields. However, if you don't
want the \glspl{locationlist} (for example, you are using the
\styopt{nonumberlist} option or you are using \idx!{xindy} with a
custom \gls{location} rule), then there's no need for \bibgls\ to process
the \glspl{location}. To switch this function off, just use
\csopt[false]{save-locations}. Note that with this setting, if
you're not additionally using \idx!{makeindex} or \idx!{xindy}, then
the \glspl{location} won't be available even if you don't have the
\styopt{nonumberlist} option set.

The boolean \field{nonumberlist} key that may be used in
\gls{newglossaryentry} can also be used in a \ext{bib} file, but in
this case it can't have an empty value. The value must be either
\code{true} or \code{false}. If \code{true} then \bibgls\ won't
save the \field{location} or \field{loclist} fields, regardless of
the \csopt{save-locations} resource option.

The \field{nonumberlist} key provided by the base \sty{glossaries}
package doesn't represent a real field. The value isn't saved but,
if used, it will alter the indexing information that's written to
the \idx!{makeindex} or \idx!{xindy} file. It's a little hack to
ensure that the \gls{location} is hidden for a specific entry when used
with \idx!{makeindex} and \idx!{xindy}.

\bibgls\ will look for this key to determine if the \gls{location} should
be omitted for the given entry, but it won't write the key to the 
\ext{glstex} file.

\optsection{save-loclist}

If you want the \field{location} field but don't need
\field{loclist}, you can use \csopt[false]{save-loclist}.
This can help to save resources and build time.

\optsection{save-primary-locations}

A synonym for \csopt{save-principal-locations}.

\optsection{save-principal-locations}

It's sometimes useful to identify a \igls{principallocation} with a different
format, such as bold or italic. This helps the reader select which
\gls{location} to try first in the event of a long \gls{locationlist}. However,
you may prefer to store the \glspl{principallocation} in a different field to
give it a more prominent position. In order to do this you 
need to specify the format (or formats) used to identify 
\glspl{principallocation} with \csopt{principal-location-formats} and 
use \csopt{save-principal-locations} to determine how to deal with
these \glspl{location}.

This option may take one of the following values:
\begin{itemize}
\item \optfmt{false}: don't save \glspl{principallocation} (default);

\item \optfmt{retain}: save \glspl{principallocation} in the
\field{primarylocations} field but don't remove from the usual
\igls{locationlist};

\item \optfmt{default format}: similar to \optfmt{retain} but the
format for the \glsdisp{principallocation}{principal records} in the
\field{location} field is converted to the default
\encap{glsnumberformat} \idx{encap} (the \glspl{record} in the
\field{primarylocations} field retain their given format);

\item \optfmt{start}: save \glspl{principallocation} in the
\field{primarylocations} field and also move to the start of the
usual \igls{locationlist};

\item \optfmt{remove}: save \glspl{principallocation} in the
\field{primarylocations} field and remove from the usual
\igls{locationlist}. You may want to consider using the
\longarg{retain-formats} switch with this setting if you don't
want to lose a partial \gls{location} match (for example, if the 
\gls{principallocation} coincides with the start of an
\idx{explicit-range}).

\end{itemize}

The \glspl{principallocation} are copied to the \field{primarylocations}
field and are either encapsulated with \gls!{bibglsprimary} or can
be split into groups, according to \csopt{principal-loc-counters}.

If you use \csopt[remove]{save-principal-locations}, the
\field{location} field will end up empty if the \glspl{location} for the
associated entry were all identified as principal. If you use
\csopt[start]{save-principal-locations}, all
\glspl{principallocation} will be
moved to the start of the \igls{locationlist} stored in the
\field{location} field, but there will be no additional markup
(other than the given format) to identify them. If you need
additional markup, then use \csopt[remove]{save-principal-locations}
and adjust the \igls{locationlist} format to insert the 
\glspl{principallocation} at the start. This can be done by modifying the glossary
style.

For example, the \glostyle{bookindex} style inserts
\ics{glsxtrbookindexprelocation} before the \gls{location}, so you could
redefine this:
\begin{codeenv}
\cs{renewcommand}*\marg{\cs{glsxtrbookindexprelocation}}[1]\marg{\comment{}
  \cs{glsxtrifhasfield}\marg{primarylocations}\marg{\idx{param}1}\comment{}
  \marg{\comment{}
    \ics{glsxtrprelocation}
    \cs{glscurrentfieldvalue}
    \cs{glsxtrifhasfield}\marg{location}\marg{\idx{param}1}\marg{;}\marg{}\comment{}
  }\comment{}
  \marg{}\comment{}
  \cs{glsxtrprelocation}
}
\end{codeenv}
(Note that if \csopt{loc-prefix} is used, the prefix will be
in the \field{location} field and so will come after the 
\glspl{principallocation} in the above example. Similarly for cross-references
unless they've been omitted.)

You can switch from using the \field{location} field to the
\field{primarylocations} field by locally changing
\ics{GlsXtrLocationField}:
\begin{codeenv}
\cs{printunsrtglossary*}\marg{\comment{}
  \cs{renewcommand}\marg{\cs{GlsXtrLocationField}}\marg{primarylocations}\comment{}
}
\end{codeenv}
Remember that the \idx{handler} used by \cs{printunsrtglossary} will
fallback on the \field{loclist} field if the field identified
by \cs{GlsXtrLocationField} is missing or empty. You may want to
consider using \csopt[false]{save-loclist} to prevent this.

\optsection{primary-location-formats}

A synonym for \csopt{principal-location-formats}.

\optsection{principal-location-formats}

This option will automatically set
\csopt[retain]{save-principal-locations} unless it has already been
changed from the default \csopt[false]{save-principal-locations}
setting. The argument should be a comma-separated list of 
formats. If a record's format is contained in this list then it will
be considered a \igls{principallocation} and it will be included in the
associated entry's \field{primarylocations} field.

For example, suppose the file \filefmt{entries.bib} contains:
\begin{codeenv}
\atentry{entry}\marg{bird,
  \field{name}=\marg{bird},
  \field{description}=\marg{feathered animal}
}
\atentry{entry}\marg{waterfowl,
  \field{name}=\marg{waterfowl},
  \field{description}=\marg{any bird that lives in or about water}
}
\atentry{entry}\marg{zebra,
  \field{name}=\marg{zebra},
  \field{description}=\marg{striped African horse}
}
\atentry{entry}\marg{parrot,
  \field{name}=\marg{parrot},
  \field{description}=\marg{mainly tropical bird with bright plumage}
}
\end{codeenv}
and the document \filefmt{test.tex} contains:
\begin{codeenv}
\cmd{documentclass}\marg{report}
\strut
\cmd{usepackage}\oarg{colorlinks}\marg{hyperref}
\cmd{usepackage}\oarg{\styopt{record},
 \styopt[dot]{postpunc},
 \styopt{nostyles},
 \styopt[tree,bookindex]{stylemods},
 \styopt[bookindex]{style}}\marg{glossaries-extra}
\strut
\gls{GlsXtrLoadResources}\oarg{
  \csopt[entries]{src},
  \csopt[hyperbf,hyperemph]{principal-location-formats},
  \csopt[remove]{save-principal-locations}
}
\strut
\cmd{renewcommand}*\marg{\cs{glsxtrbookindexprelocation}}[1]\marg{\comment{}
  \cs{glsxtrifhasfield}\marg{\field{primarylocations}}\marg{\idx{param}1}\comment{}
  \marg{\comment{}
    \ics{glsxtrprelocation}
    \cs{glscurrentfieldvalue}
    \cs{glsxtrifhasfield}\marg{\field{location}}\marg{\idx{param}1}\marg{;}\marg{}\comment{}
  }\comment{}
  \marg{}\comment{}
  \cs{glsxtrprelocation}
}
\strut
\gls{glsxtrnewglslike}\oarg{\glsopt[\encap{hyperbf}]{format}}\marg{}\marg{\cmd{primary}}\marg{\cmd{primarypl}}\marg{\cmd{Primary}}\marg{\cmd{Primarypl}}
\strut
\cmd{begin}\marg{document}
\cmd{chapter}\marg{Sample}
\cmd{Primary}\marg{waterfowl}, \cs{gls}\marg{bird} and \cs{gls}\marg{zebra}.
\strut
\cmd{chapter}\marg{Another Sample}
\cs{Gls}\marg{waterfowl}, \cmd{primary}\marg{bird} and \cs{gls}\marg{zebra}.
\strut
\cmd{chapter}\marg{Yet Another Sample}
\cs{Gls}\marg{waterfowl}, \cs{gls}\marg{bird} and \cmd{primary}\marg{zebra}.
\strut
\cmd{chapter}\marg{Yet Another Sample Again}
\cs{Gls}\marg{waterfowl}, \cs{gls}\marg{bird}, \cmd{primarypl}\marg{parrot} and \cs{gls}\marg{zebra}.
\strut
\ics{printunsrtglossary*}\oarg{\printglossopt[\iglostyle{tree}]{style},\printglossopt{nonumberlist}}\marg{\comment{}
 \cmd{renewcommand}*\marg{\ics{glsextrapostnamehook}}[1]\marg{\ics{glsadd}\oarg{\glsopt[\encap{hyperemph}]{format}}\marg{\idx{param}1}}\comment{}
}
\strut
\cs{printunsrtglossary}\oarg{\printglossopt[Index]{title},\printglossopt[false]{target}}
\cmd{end}\marg{document}
\end{codeenv}

The \csopt[\encap{hyperbf},\encap{hyperemph}]{principal-location-formats}
setting in the above indicates that \glspl{location} encapsulated with
\ics{hyperbf} and \ics{hyperemph} are
\glsdisp{principallocation}{principal records}. In this case, the
bold format is used to indicate the \gls{principallocation} in the
main document text and the emphasized format is used to indicate the
\gls{location} in the \gls{mainglossary}.

The \glsdisp{principallocation}{principal records} are removed from
the \field{location} field due to the
\csopt[remove]{save-principal-locations} setting. This can lead to a
ragged \gls{locationlist}. The option \csopt[default
format]{save-principal-locations} can allow the
\gls{principallocation} to be absorbed into a \idx{range}.

The \gls{mainglossary} records are added through the category-independent post-name
hook with \cs{glsadd}. This won't be implemented until the entries are actually
defined as the page number can't be determined until the glossary can be
displayed. This means that the document build requires an extra \bibgls\ and
\LaTeX\ run:
\begin{verbatim}
pdflatex test
bib2gls --group test
pdflatex test
bib2gls --group test
pdflatex test
\end{verbatim}

For consistency, I've used \gls{glsxtrnewglslike} to provide commands
used to indicate a principal reference in the text. This means that if
I decide to change the optional arguments used for principal
references I only need to edit one line. For example, I might want to 
change the default counter:
\begin{codeenv}
\gls{glsxtrnewglslike}\oarg{\glsopt[\encap{hyperbf}]{format},\glsopt[\counter{chapter}]{counter}}\marg{}\marg{\cmd{primary}}\marg{\cmd{primarypl}}\marg{\cmd{Primary}}\marg{\cmd{Primarypl}}
\end{codeenv}

Here's another example that only has one principal format (\encap{hyperrm}) 
that's indexed through the use of \ics{GlsXtrAutoAddOnFormat}, which sets
up a hook that automatically inserts:
\begin{codeenv*}
\ics{glsadd}\oarg{\glsaddopt[\counter{chapter}]{counter},\glsaddopt[\encap{hyperrm}]{format}}\margm{label}
\end{codeenv*}
on each instance of \code{\cs{gls}\oarg{\glsopt[primaryfmt]{format}}\margm{label}} (or
similar).  This means that the entry is indexed twice when this particular
format is used: first with the \encap{hyperrm} format and \counter{chapter} 
counter (from the \cs{glsadd} command in the hook), and then with the
\code{primaryfmt} format and the default counter (as per normal behaviour):
\begin{codeenv}
\cmd{documentclass}\marg{report}
\strut
\cmd{usepackage}\oarg{colorlinks}\marg{hyperref}
\cmd{usepackage}\oarg{
 \styopt[nameref]{record},
 \styopt[dot]{postpunc},
 \styopt{nostyles},
 \styopt[tree,bookindex]{stylemods},
 \styopt[bookindex]{style}}\marg{glossaries-extra}
\strut
\gls{GlsXtrLoadResources}\oarg{
  \csopt[topics]{src},
  \csopt[hyperrm]{principal-location-formats},
  \csopt[remove]{save-principal-locations},
  \csopt[false]{save-loclist}
}
\strut
\cmd{newcommand}\marg{\cmd{primaryfmt}}\oarg{1}\marg{\cs{hyperbf}\marg{\idx{param}1}}
\strut
\ics{GlsXtrAutoAddOnFormat}\marg{primaryfmt}\marg{\glsaddopt[\counter{chapter}]{counter},\glsaddopt[\encap{hyperrm}]{format}}
\strut
\gls{glsxtrnewglslike}\oarg{\glsaddopt[primaryfmt]{format}}\marg{}\marg{\cmd{primary}}\marg{\cmd{primarypl}}\marg{\cmd{Primary}}\marg{\cmd{Primarypl}}
\strut
\cmd{begin}\marg{document}
\cmd{chapter}\marg{Sample}
\cmd{Primary}\marg{waterfowl}, \cs{gls}\marg{bird} and \cs{gls}\marg{zebra}.
\strut
\cmd{chapter}\marg{Another Sample}
\cs{Gls}\marg{waterfowl}, \cmd{primary}\marg{bird} and \cs{gls}\marg{zebra}.
\strut
\cmd{chapter}\marg{Yet Another Sample}
\cs{Gls}\marg{waterfowl}, \cs{gls}\marg{bird} and \cmd{primary}\marg{zebra}.
\strut
\cmd{chapter}\marg{Yet Another Sample Again}
\cs{Gls}\marg{waterfowl}, \cs{gls}\marg{bird}, \cmd{primarypl}\marg{parrot} and \cs{gls}\marg{zebra}.
\strut
\ics{printunsrtglossary*}\oarg{\printglossopt[\glostyle{tree}]{style},\printglossopt[Summary]{title}}\marg{\comment{}
 \cmd{renewcommand}*\marg{\ics{glsextrapostnamehook}}[1]\marg{\ics{glsadd}\oarg{\glsaddopt[\encap{hyperemph}]{format}}\marg{\idx{param}1}}\comment{}
 \cmd{renewcommand}\marg{\ics{GlsXtrLocationField}}\marg{\field{primarylocations}}\comment{}
}
\strut
\cs{printunsrtglossary}\oarg{\printglossopt[Index]{title},\printglossopt[false]{target}}
\cmd{end}\marg{document}
\end{codeenv}

Note that in this case, from \bibgls' point of view, the principal
format is \encap{hyperrm} not \code{primaryfmt}. This picks out the
records created with the automated \ics{glsadd}, which have the
counter set to \counter{chapter}.  The first glossary (with the
title \qt{Summary}) switches the location field to
\field{primarylocations} so that only the
\glsdisp{principallocation}{principal records} are listed. Since
\styopt[nameref]{record} has been used this means that the chapter
title is shown rather than the chapter number.

The second glossary (\qt{Index}) shows the \glspl{locationlist} that
only have the \counter{page} counter (because the automated
\cs{glsadd} records with the \counter{chapter} counter have been
removed because they were identified as
\glsdisp{principallocation}{principal records}). These just show the
page number as that's the default display with
\styopt[nameref]{record} for records with the \counter{page}
counter.

An alternative to \ics{GlsXtrAutoAddOnFormat} would be to simply
define the custom commands as follows:
\begin{codeenv}
\cmd{newcommand}\marg{\cmd{primary}}[2][]\marg{\comment{}
 \cs{glsadd}\oarg{\glsaddopt[\counter{chapter}]{counter},\glsaddopt[\encap{hyperrm}]{format}}\marg{\idx{param}2}\comment{}
 \cs{gls}\oarg{\glsopt[primaryfmt]{format},\idx{param}1}\marg{\idx{param}2}\comment{}
}
\cmd{newcommand}\marg{\cmd{primarypl}}[2][]\marg{\comment{}
 \cs{glsadd}\oarg{\glsaddopt[\counter{chapter}]{counter},\glsaddopt[\encap{hyperrm}]{format}}\marg{\idx{param}2}\comment{}
 \cs{glspl}\oarg{\glsopt[primaryfmt]{format},\idx{param}1}\marg{\idx{param}2}\comment{}
}
\cmd{newcommand}\marg{\cmd{Primary}}[2][]\marg{\comment{}
 \cs{glsadd}\oarg{\glsaddopt[\counter{chapter}]{counter},\glsaddopt[\encap{hyperrm}]{format}}\marg{\idx{param}2}\comment{}
 \cs{Gls}\oarg{\glsopt[primaryfmt]{format},\idx{param}1}\marg{\idx{param}2}\comment{}
}
\cmd{newcommand}\marg{\cmd{Primarypl}}[2][]\marg{\comment{}
 \cs{glsadd}\oarg{\glsaddopt[\counter{chapter}]{counter},\glsaddopt[\encap{hyperrm}]{format}}\marg{\idx{param}2}\comment{}
 \cs{Glspl}\oarg{\glsopt[primaryfmt]{format},\idx{param}1}\marg{\idx{param}2}\comment{}
}
\end{codeenv}
This is more useful if you want to simply omit the
\code{\glsopt[primaryfmt]{format}} option (just remove it from the above four 
definitions), which makes it easier to merge the \glspl{location} into
\idxpl{range} in the index.

\optsection{primary-loc-counters}

A synonym for \csopt{principal-loc-counters}.

\optsection{principal-loc-counters}

This option determines whether the \glspl{principallocation} should
be split into groups according to the location counter. The value
may be one or:
\begin{itemize}
\item \optfmt{combine}: don't split into groups (default);
\item \optfmt{match}: match the \csopt{loc-counters} setting;
\item \optfmt{split}: split into groups regardless of the
\csopt{loc-counters} setting.
\end{itemize}

With \csopt[combine]{principal-loc-counters} or with
\csopt[match]{principal-loc-counters} and the default \csopt[as
use]{loc-counters} settings, no groups will be formed and the
\glspl{principallocation} will be encapsulated with
\gls{bibglsprimary}. Otherwise, the locations will be split into
groups according to the counter and each group will be encapsulated
with \gls{bibglsprimarylocationgroup} and separated with
\gls{bibglsprimarylocationgroupsep}.

For example, suppose the file \filefmt{topics.bib} contains the
following entry:
\begin{codeenv}
\atentry{entry}\marg{zebra,
  \field{name}=\marg{zebra},
  \field{description}=\marg{striped African horse}
}
\end{codeenv}
The document sets up a \gls{principallocation} format identified by
the custom command \csfmt{primaryfmt}:
\begin{codeenv}
\cs{newcommand}\marg{\cmd{primaryfmt}}[1]\marg{\cs{hyperbf}\marg{\idx{param}1}}
\end{codeenv}
The \ics{GlsXtrAutoAddOnFormat} command is used to automatically
record an entry with the \counter{chapter} counter (using
\ics{glsadd}) every time the entry is
recorded with the \gls{principallocation} format:
\begin{codeenv}
\cs{GlsXtrAutoAddOnFormat}\marg{primaryfmt}\marg{\glsopt{counter}=\counter{chapter},\glsopt{format}=primaryfmt}
\end{codeenv}
This means that, for example,
\begin{codeenv}
\cs{gls}\oarg{\glsopt{format}=primaryfmt}\margm{zebra}
\end{codeenv}
will also first do:
\begin{codeenv}
\cs{glsadd}\oarg{\glsopt{counter}=\counter{chapter},\glsopt{format}=primaryfmt}\margm{zebra}
\end{codeenv}

If \idx{resourceset} is loaded with:
\begin{codeenv}
\gls{GlsXtrLoadResources}\oarg{\csopt[topics]{src},
 \csopt[primaryfmt]{primary-location-formats},
 \csopt[retain]{save-primary-locations}
}
\end{codeenv}
then both the \field{location} field and the
\field{primarylocations} field will include both the \counter{page}
and \counter{chapter} records mixed together. The
\field{primarylocations} field will have the locations encapsulated
with \gls{bibglsprimary}.

With 
\begin{codeenv}
\gls{GlsXtrLoadResources}\oarg{\csopt[topics]{src},
 \csopt[primaryfmt]{primary-location-formats},
 \csopt[retain]{save-primary-locations},
 \csopt[split]{primary-loc-counters}
}
\end{codeenv}
the \field{primarylocations} field will have the locations split into
two groups, each encapsulated with \gls{bibglsprimarylocationgroup}.
The \field{location} field will have the \counter{chapter} and
\counter{page} locations intermingled.

With 
\begin{codeenv}
\gls{GlsXtrLoadResources}\oarg{\csopt[topics]{src},
 \csopt[primaryfmt]{primary-location-formats},
 \csopt[retain]{save-primary-locations},
 \csopt[split]{primary-loc-counters},
 \csopt[page]{loc-counters}
}
\end{codeenv}
The \field{primarylocations} field will be the same as before, but
the \field{location} field will only have the \counter{page}
locations.

Whereas with 
\begin{codeenv}
\gls{GlsXtrLoadResources}\oarg{\csopt[topics]{src},
 \csopt[primaryfmt]{primary-location-formats},
 \csopt[retain]{save-primary-locations},
 \csopt[match]{primary-loc-counters},
 \csopt[page]{loc-counters}
}
\end{codeenv}
Both the \field{primarylocations} field and the \field{location}
field will only have the \counter{page} locations.

The order of the groups depends on whether \optfmt{split} or
\optfmt{match} is used. With \csopt[match]{primary-loc-counters} the
counter group order will match \csopt{loc-counters}. Whereas with
\csopt[split]{primary-loc-counters} the counter group order will be determined by
the order of records.

So in the case of the above document where the \counter{chapter}
record is automatically added before the \counter{page} record
(where the \glsopt{format} is \code{primaryfmt}) then with:
\begin{codeenv}
\gls{GlsXtrLoadResources}\oarg{\csopt[topics]{src},
 \csopt[primaryfmt]{primary-location-formats},
 \csopt[retain]{save-primary-locations},
 \csopt[match]{primary-loc-counters},
 \csopt[page,chapter]{loc-counters}
}
\end{codeenv}
then both the \field{location} field and the
\field{primarylocations} field will have the \counter{page} group
first, followed by the \counter{chapter} group. Whereas with:
\begin{codeenv}
\gls{GlsXtrLoadResources}\oarg{\csopt[topics]{src},
 \csopt[primaryfmt]{primary-location-formats},
 \csopt[retain]{save-primary-locations},
 \csopt[skip]{primary-loc-counters},
 \csopt[page,chapter]{loc-counters}
}
\end{codeenv}
then the \field{primarylocations} field will have the
\counter{chapter} group first, followed by the \counter{page} group.

\optsection{merge-ranges}

This boolean option determines whether or not \idxpl{explicit-range} should
merge with neighbouring \glspl{location} on either side of the
\idx{range}. The default setting is \csopt[false]{merge-ranges}.

Note that \gls{bibglsrange} won't be used with
\csopt[true]{merge-ranges}, regardless of whether or not the \idx{range}
was merged with neighbouring \glspl{location}. Options such as
\csopt{min-loc-range}, \csopt{suffixF} and \csopt{suffixFF} 
won't have an effect on the merged range, but
will still effect \idxpl{implicit-range} that haven't been merged with an
\idx{explicit-range}.

Regardless of the value of this option, \idxpl{interloper} will still be
moved to the start of the \idx{range} and encapsulated with
\gls{bibglsinterloper}.

\optsection{min-loc-range}

By default, three or more consecutive \glspl{location} \meta{loc-1},
\meta{loc-2}, \ldots, \meta{loc-n} are compressed into
the \idx{range} \code{\meta{loc-1}\ics{delimR} \meta{loc-n}}
(an \idx{implicit-range}). Otherwise
the \glspl{location} are separated by \gls!{bibglsdelimN} or \gls!{bibglslastDelimN}.
As mentioned above, these aren't merged with \idx{explicit-range}
formations unless \csopt[true]{merge-ranges}.

You can change how many consecutive \glspl{location} are need to
form an \idx{implicit-range} with the \csopt{min-loc-range} setting where
\meta{value} is either \optfmt{none} (don't form ranges) or an
integer greater than one indicating how many consecutive
\glspl{location} should be converted into a \idx{range}.

\bibgls\ determines if one \gls{location}
\code{\margm{prefix-2}\margm{counter-2}\margm{format-2}\margm{location-2}}
is one unit more than another \gls{location}
\code{\margm{prefix-1}\margm{counter-1}\margm{format-1}\margm{location-1} }
according to the following:
\begin{enumerate}
\item\label{itm:pre} If \meta{prefix-1} is not equal to \meta{prefix-2} or
\meta{counter-1} is not equal to \meta{counter-2} or \meta{format-1}
is not equal to \meta{format-2}, then the \glspl{location} aren't
considered consecutive.
\item\label{itm:emptyloc} If either \meta{location-1} or \meta{location-2} are empty,
then the \glspl{location} aren't considered consecutive.
\item\label{itm.csmatch} If both \meta{location-1} and \meta{location-2} match the
pattern (line break for clarity only)\footnote{The Java class \code{\csfmt{p}\marg{javaDigit}}
used in the \gls{regular-expression} will match any digits in the 
Unicode \idx{numberdecimaldigit} category not just the digits in the Basic Latin set. Similarly \code{\csfmt{p}\marg{javaAlphabetic}} will also match alphabetic
characters outside the Basic Latin set.}
\begin{verbatim}
(.*?)(?:\\protect\s*)?(\\[\p{javaAlphabetic}@]+)\s*\{([\p{javaDigit}
\p{javaAlphabetic}]+)\}
\end{verbatim}
then:
  \begin{itemize}
  \item if the control sequence matched by group 2 isn't the same for
  both \glspl{location}, the \glspl{location} aren't considered consecutive;
  \item if the argument of the control sequence (group 3) is the same for
  both \glspl{location}, then the test is retried with \meta{location-1}
set to group 1 of the first pattern match and \meta{location-2}
set to group 1 of the second pattern match;
  \item otherwise the test is retried with \meta{location-1} set to
group 3 of the first pattern match and \meta{location-2} set to
group 3 of the second pattern match.
  \end{itemize}
\item\label{itm:decmatch} If both \meta{location-1} and \meta{location-2} match the
pattern
\begin{verbatim}
(.*?)([^\p{javaDigit}]?)(\p{javaDigit}+)
\end{verbatim}
then:
  \begin{enumerate}
  \item\label{itm:decgrp3eq} if group 3 of both pattern matches are
equal then:
  \begin{enumerate}
   \item\label{itm:decgrp3nz} if group 3 isn't zero, the \glspl{location}
aren't considered consecutive;
   \item if the separators (group 2) are different the test is
retried with \meta{location-1} set to the concatenation of the first
two groups \meta{group-1}\meta{group-2} of the first pattern match 
and \meta{location-2} set to the concatenation of the first two groups
\meta{group-1}\meta{group-2} of the second pattern
match;
   \item\label{itm:decgrp3eqsepeq} if the separators (group 2) are the same the test is
retried with \meta{location-1} set to the first
group \meta{group-1} of the first pattern match 
and \meta{location-2} set to the first group
\meta{group-1} of the second pattern match.
  \end{enumerate}
  \item If \meta{group-1} of the first pattern match (of
\meta{location-1}) doesn't equal
\meta{group-1} of the second pattern match (of \meta{location-2})
or \meta{group-2} of the first pattern match (of
\meta{location-1}) doesn't equal
\meta{group-2} of the second pattern match (of \meta{location-2})
then the \glspl{location} aren't considered consecutive;
   \item\label{itm:decgrp3} If $0 < l_2 - l_1 \leq d $
where $l_2$ is \meta{group 3} of the second pattern match,
$l_1$ is \meta{group 3} of the first pattern match and
$d$ is the value of \optfmt{max-loc-diff} then the \glspl{location}
are consecutive otherwise they're not consecutive.
  \end{enumerate}
\item\label{itm:rommatch} The next pattern matches for \meta{prefix}\meta{sep}\meta{n}
where \meta{n} is a \idx!{lowercase} Roman numeral, which is converted to
a decimal value and the test is performed in the same way as the 
above \hyperref[itm:decmatch]{decimal test}.
\item\label{itm:Rommatch} The next pattern matches for \meta{prefix}\meta{sep}\meta{n}
where \meta{n} is an \idx!{uppercase} Roman numeral, which is converted to
a decimal value and the test is performed
in the same way as the above \hyperref[itm:decmatch]{decimal test}.
\item\label{itm:alphmatch} The next pattern matches for \meta{prefix}\meta{sep}\meta{c}
where \meta{c} is either a \idx!{lowercase} letter from \code{a} to
\code{z} or an \idx!{uppercase} letter from \code{A} to \code{Z}.
The character is converted to its code point and the test is
performed in the same way as the \hyperref[itm:decmatch]{decimal pattern} above.
\item\label{itm:nomatch} If none of the above, the \gls{location} aren't considered
consecutive.
\end{enumerate} 

Examples:
\begin{enumerate}
\item
\begin{codeenv}
\gls{glsxtr@record}\marg{gls.sample}\marg{}\marg{\counter{page}}\marg{\encap{glsnumberformat}}\marg{1}
\gls{glsxtr@record}\marg{gls.sample}\marg{}\marg{\counter{page}}\marg{\encap{glsnumberformat}}\marg{2}
\end{codeenv}
These records are consecutive. The prefix, counter and format are
identical (so the test passes step~\ref{itm:pre}), the \glspl{location} match
the \hyperref[itm:decmatch]{decimal pattern} and the test in
step~\ref{itm:decgrp3} passes.

\item
\begin{codeenv}
\gls{glsxtr@record}\marg{gls.sample}\marg{}\marg{\counter{page}}\marg{\encap{glsnumberformat}}\marg{1}
\gls{glsxtr@record}\marg{gls.sample}\marg{}\marg{\counter{page}}\marg{\encap{textbf}}\marg{2}
\end{codeenv}
These records aren't consecutive since the formats are different.

\item
\begin{codeenv}
\gls{glsxtr@record}\marg{gls.sample}\marg{}\marg{\counter{page}}\marg{\encap{glsnumberformat}}\marg{A.i}
\gls{glsxtr@record}\marg{gls.sample}\marg{}\marg{\counter{page}}\marg{\encap{glsnumberformat}}\marg{A.ii}
\end{codeenv}
These records are consecutive. The prefix, counter and format are
identical (so it passes step~\ref{itm:pre}). The \glspl{location} match
the \hyperref[itm:rommatch]{lower case Roman numeral pattern}, where
\code{A} is considered a prefix and the dot is consider a
separator. The Roman numerals i and ii are converted to decimal and
the test is retried with the \glspl{location} set to 1 and 2, respectively.
This now passes the decimal pattern test (step~\ref{itm:decgrp3}).

\item
\begin{codeenv}
\gls{glsxtr@record}\marg{gls.sample}\marg{}\marg{\counter{page}}\marg{\encap{glsnumberformat}}\marg{i.A}
\gls{glsxtr@record}\marg{gls.sample}\marg{}\marg{\counter{page}}\marg{\encap{glsnumberformat}}\marg{ii.A}
\end{codeenv}
These records aren't consecutive. They match the
\hyperref[itm:alphmatch]{alpha pattern}. The first \gls{location} is
considered to consist of the prefix \code{i}, the separator
\code{.}\ (dot) and the number given by the character code of A. 
The second \gls{location} is considered to consist of the prefix
\code{ii}, the separator \code{.}\ (dot) and the number 
given by the character code of A.

The test fails because the numbers are equal and the prefixes are
different.

\item
\begin{codeenv}
\gls{glsxtr@record}\marg{gls.sample}\marg{}\marg{\counter{page}}\marg{\encap{glsnumberformat}}\marg{1.0}
\gls{glsxtr@record}\marg{gls.sample}\marg{}\marg{\counter{page}}\marg{\encap{glsnumberformat}}\marg{2.0}
\end{codeenv}
These records are consecutive. They match the \hyperref[itm:decmatch]{decimal
pattern}, and then step~\ref{itm:decgrp3eq} followed by
step~\ref{itm:decgrp3eqsepeq}. The \code{.0} part is discarded and
the test is retried with the first \gls{location} set to 1 and the second
\gls{location} set to 2.

\item
\begin{codeenv}
\gls{glsxtr@record}\marg{gls.sample}\marg{}\marg{\counter{page}}\marg{\encap{glsnumberformat}}\marg{1.1}
\gls{glsxtr@record}\marg{gls.sample}\marg{}\marg{\counter{page}}\marg{\encap{glsnumberformat}}\marg{2.1}
\end{codeenv}
These records aren't consecutive as the test branches off into
step~\ref{itm:decgrp3nz}.

\item
\begin{codeenv}
\gls{glsxtr@record}\marg{gls.sample}\marg{}\marg{\counter{page}}\marg{\encap{glsnumberformat}}\marg{\cmd{@alph}\marg{1}}
\gls{glsxtr@record}\marg{gls.sample}\marg{}\marg{\counter{page}}\marg{\encap{glsnumberformat}}\marg{\cmd{@alph}\marg{2}}
\end{codeenv}
These records are consecutive. The \glspl{location} match the 
\hyperref[itm.csmatch]{control sequence pattern}. The control
sequences are the same, so the test is retried with the first
\gls{location} set to 1 and the second \gls{location} set to 2.

In this example, the \gls{location} has been written to the file as
\code{\csfmt{@alph}\margm{number}} instead of fully expanding according to the
normal behaviour of \code{\ics{alph}\margm{counter}}.  (Note that
\gls!{glsxtrresourcefile} changes the category code of \code{@} to allow for
internal commands in \glspl{location}.) This unusual case is for illustrative purposes.
\end{enumerate}

\optsection{max-loc-diff}

This setting is used to determine whether two \glspl{location} are
considered consecutive.
The value must be an integer greater than or equal to 1.
(The default is \optfmt{1}.)

For two \glspl{location}, \meta{location-1} and \meta{location-2},
that have numeric values $n_1$ and $n_2$ (and identical prefix,
counter and format), then the sequence \meta{location-1},
\meta{location-2} is considered consecutive if
\[
0 < n_2 - n_1 \leq \text{\meta{max-loc-diff}}
\]
The default value of 1 means that \meta{location-2} immediately
follows \meta{location-1} if $n_2 = n_1+1$.

For example, if \meta{location-1} is \qt{B} and \meta{location-2} 
is \qt{C}, then $n_1 = 66$ and $n_2 = 67$. Since $n_2 = 67 = 66+1=
n_1+1$ then \meta{location-2} immediately follows \meta{location-1}.

This is used in the \idx{implicit-range} formations within 
the \glspl{locationlist} (as described
in the above section).
So, for example, the list \qt{1, 2, 3, 5, 7, 8, 10, 11, 12, 58, 59,
61} becomes
\qt{1--3, 5, 7, 8, 10--12, 58, 59, 61}.

The automatically indexing of commands like \ics{gls} means that
the \glspl{locationlist} can become long and ragged. You could
deal with this by switching off the automatic indexing and
only explicitly index pertinent use or you can adjust
the value of \optfmt{max-loc-diff} so that a \idx{range} can be formed even
if there are one or two gaps in it.
By default, any \idxpl{range} that have skipped gaps in this
manner will be followed by \gls!{bibglspassim}. The default
definition of this command is obtained from the resource file.
For English, this is \verb*| passim| (space followed by \qt{passim}).

So with the above set of \glspl{location}, if \csopt[2]{max-loc-diff} then 
the list becomes \qt{1--12 passim, 58--61 passim} which now highlights that
there are two blocks within the document related to that
term.

\optsection{suffixF}

If set, an \idx{implicit-range} consisting of two consecutive
\glspl{location} \meta{loc-1} and \meta{loc-2} will be displayed in
the \gls{locationlist} as \meta{loc-1}\meta{value}.  This option
doesn't affect \idxpl{explicit-range}, even with
\csopt[true]{merge-ranges}.

Note that \csopt[\empty]{suffixF} sets the suffix to the
empty string. To remove the suffix formation use
\csopt[none]{suffixF}.

The default is \csopt[none]{suffixF}.

\optsection{suffixFF}

If set, an \idx{implicit-range} consisting of three or more
consecutive \glspl{location} \meta{loc-1} and \meta{loc-2} will be
displayed in the \gls{locationlist} as \meta{loc-1}\meta{value}.
This option doesn't affect \idxpl{explicit-range}, even with
\csopt[true]{merge-ranges}.

Note that \csopt[\empty]{suffixFF} sets the suffix to the
empty string. To remove the suffix formation use
\csopt[none]{suffixFF}.

The default is \csopt[none]{suffixFF}.

\optsection{compact-ranges}

The \meta{value} may be an integer \meta{n} or \optfmt{false} (equivalent to
\csopt[0]{compact-ranges}) or \optfmt{true} (equivalent to
\csopt[3]{compact-ranges}). If no \meta{value} is specified,
\optfmt{true} is assumed.

This setting allows location \idxpl{range} such as 184--189 to appear
more compactly as 184--9. The end \gls{location} is encapsulated
in the command \gls{bibglscompact}, so the \idx{range} would actually
become:
\begin{codeenv}
184\ics{delimR}\gls{bibglscompact}\marg{digit}\marg{18}\marg{9}
\end{codeenv}
If the \gls{location} is in the form \code{\meta{cs}\margm{loc}}
(where \meta{cs} is a command)
then \gls{bibglscompact} will be inside the argument.
For example, if the \idx{range} would normally be:
\begin{codeenv}
\cmd{custom}\marg{184}\ics{delimR}\cmd{custom}\marg{189}
\end{codeenv}
then it would become:
\begin{codeenv}
\cmd{custom}\marg{184}\ics{delimR}\cmd{custom}\marg{\gls{bibglscompact}\marg{digit}\marg{18}\marg{9}}
\end{codeenv}
The numerical value given in \csopt[\meta{n}]{compact-ranges}
indicates that compaction should only occur if the actual \gls{location}
consists of at least \meta{n} characters, for $\meta{n} \geq 2$. 
Any value of \meta{n} less than 2 will switch off compaction.

For example, \code{189} consists of 3 characters, so it will be
compacted with \csopt[3]{compact-ranges} but not with
\csopt[4]{compact-ranges}. Whereas \code{\csfmt{custom}\marg{89}}
would only be compacted with \csopt[2]{compact-ranges}
because \code{89} only consists of 2 characters.

The compaction isn't limited to decimal digits but it will only
occur if both the start and end \gls{location} have the same number of
characters. For example, xvi--xviii can't be compacted because
the start consists of three characters and the end consists
of five characters, whereas xxv--xxx can be compacted to xxv--x,
which may look a little strange. In this case, you may want to
consider changing the definition of \cs{bibglscompact} so that
it only performs the compaction for digits.

\optsection{see}

If an entry has a \field{see} field, this can be placed before or
after the \gls{locationlist}, or completely omitted (but the value will
still be available in the \field{see} field for use with
\ics{glsxtrusesee}). The required \meta{value} must be one of:
\begin{itemize}
\item \optfmt{omit}: omit the see reference from the
\gls{locationlist}.

\item \optfmt{before}: place the see reference before the
\gls{locationlist}.

\item \optfmt{after}: place the see reference after the
\gls{locationlist} (default).
\end{itemize}

The separator between the \gls{locationlist} and the cross-reference is
provided by \gls!{bibglsseesep}. This separator is omitted if the
\gls{locationlist} is empty. The cross-reference is written to the
\field{location} field using \code{\gls{bibglsusesee}\margm{label}}.

\optsection{seealso}

This is like \csopt{see} but governs the location of the cross-references
provided by the \field{seealso} field. You need at least v1.16
of \sty{glossaries-extra} for this option. The values are the
same as for \csopt{see} but the separator is given by
\gls!{bibglsseealsosep}. The cross-reference is written to the
\field{location} field using \code{\gls{bibglsuseseealso}\margm{label}}.

\optsection{alias}

This is like \csopt{see} but governs the location of the cross-references
provided by the \field{alias} field. The separator is given by
\gls!{bibglsaliassep}. The cross-reference is written to the
\field{location} field using \code{\gls{bibglsusealias}\margm{label}}.

\optsection{alias-loc}

If an entry has an \field{alias} field, the \gls{locationlist}
may be retained or omitted or transferred to the target entry.
The required \meta{value} must be one of:
\begin{itemize}
\item\optfmt{keep}: keep the \gls{locationlist};
\item\optfmt{transfer}: transfer the \gls{locationlist};
\item\optfmt{omit}: omit the \gls{locationlist}.
\end{itemize}
The default setting is \csopt[transfer]{alias-loc}.
In all cases, the target entry will be added to the \field{see}
field of the entry with the \field{alias} field, unless it
already has a \field{see} field (in which case the \field{see} value
is left unchanged).

Note that with \csopt[transfer]{alias-loc}, both the aliased
entry and the target entry must be in the same resource set.
(That is, both entries have been selected by the same instance of
\gls!{glsxtrresourcefile}.) If you have \sty{glossaries-extra} version~1.12, 
you may need to redefine \ics{glsxtrsetaliasnoindex} to do
nothing if the \glspl{locationlist} aren't showing correctly
with aliased entries. (This was corrected in version~1.13.)

\optsection{loc-prefix}

The \csopt{loc-prefix} setting indicates that the \glspl{locationlist}
should begin with \code{\gls!{bibglslocprefix}\margm{n}}. The \meta{value} may
be one of the following:
\begin{itemize}
\item \optfmt{false}: don't insert \code{\gls{bibglslocprefix}\margm{n}} at the
start of the \glspl{locationlist} (default).

\item \optfmt{\margm{prefix-1},\margm{prefix-2},\ldots,\margm{prefix-n}}:
insert \code{\gls{bibglslocprefix}\margm{n}} (where \meta{n} is the number of
\glspl{location} in the list) at the start of each \gls{locationlist} and the
definition of \gls{bibglslocprefix} will have an \ics{ifcase} condition:
\begin{codeenv}
\cmd{providecommand}\marg{\gls{bibglslocprefix}}[1]\marg{\comment{}
  \ics{ifcase}\idx{param}1
  \cmd{or} \meta{prefix-1}\gls{bibglspostlocprefix}
  \cmd{or} \meta{prefix-2}\gls{bibglspostlocprefix}
  \ldots
  \cmd{else} \meta{prefix-n}\gls{bibglspostlocprefix}
  \cmd{fi}
}
\end{codeenv}

\item \optfmt{comma}: equivalent to \csopt[\marg{,~}]{loc-prefix} 
but avoids confusion with the list syntax. That is, the prefix is a comma
followed by a space for non-empty \glspl{locationlist}.

\item \optfmt{list}: equivalent to \csopt[\ics{pagelistname} ]{loc-prefix}.

\item \optfmt{true}: equivalent to 
\csopt[\gls{bibglspagename},\gls{bibglspagesname}]{loc-prefix},
where the definitions of \gls{bibglspagename} and
\gls{bibglspagesname} are obtained from the \idx{tag.page} and
\idx{tag.pages} entries in \bibgls's \langxml.
This setting works best if the document's language matches the 
language file. However, you can redefine these commands within
the document's language hooks or in the glossary preamble.

\end{itemize}

If \meta{value} is omitted, \optfmt{true} is assumed. The definition
will be placed in the \ext{glstex} file according to
\csopt{loc-prefix-def}.

For example:
\begin{codeenv}
\gls{GlsXtrLoadResources}\oarg{\csopt[main]{type},\csopt[individual]{loc-prefix-def},\csopt[entries1]{src},\csopt[false]{loc-prefix}}
\gls{GlsXtrLoadResources}\oarg{\csopt[main]{type},\csopt[individual]{loc-prefix-def},\csopt[entries2]{src},\csopt{loc-prefix}}
\gls{GlsXtrLoadResources}\oarg{\csopt[symbols]{type},\csopt[entries3]{src},\csopt[p.,pp.]{loc-prefix}}
\end{codeenv}
This works since the conflicting \csopt[p.,pp.]{loc-prefix} and
\csopt[true]{loc-prefix} are in different glossaries (assigned through the
\csopt{type} key). The entries fetched from \filefmt{entries1.bib}
won't have a location prefix. The entries fetched from
\filefmt{entries2.bib} will have the location prefix obtained from
the language resource file. The entries fetched from
\filefmt{entries3.bib} will have the location prefix \qt{p.}\ or
\qt{pp.} (Note that using the \csopt{type} option isn't the same as
setting the \field{type} field for each entry in the \ext{bib}
file.)

If the \csopt{type} option isn't used:
\begin{codeenv}
\gls{GlsXtrLoadResources}\oarg{\csopt[entries1]{src},\csopt[false]{loc-prefix}}
\gls{GlsXtrLoadResources}\oarg{\csopt[entries2]{src},\csopt{loc-prefix}}
\gls{GlsXtrLoadResources}\oarg{\csopt[entries3]{src},\csopt[p.,pp.]{loc-prefix}}
\end{codeenv}
then \csopt[true]{loc-prefix} takes precedence over
\csopt[p.,pp.]{loc-prefix} (since it was used first). The entries fetched from
\filefmt{entries1.bib} still won't have a location prefix, but the entries
fetched from both \filefmt{entries2.bib} and \filefmt{entries3.bib}
have the location prefixes obtained from the language resource file.

Note that if you identify some glossaries but not others (for
example, you have dual entries in separate glossaries but only
use \csopt{type} and not \csopt{dual-type}), then you will need to
use \csopt[global]{loc-prefix-def} or \csopt[local]{loc-prefix-def}.

\optsection{loc-prefix-def}

This determines how the location prefix identified by
\csopt{loc-prefix} is written to the \ext{glstex} file. The value
may be one of:
\begin{itemize}
\item\optfmt{global} the definition is globally defined using
\cs{providecommand};
\item\optfmt{local} the definition is locally defined using
\cs{providecommand} in the general glossary preamble
(\cs{glossarypreamble});
\item\optfmt{individual} the definition is locally defined using
\cs{providecommand} in the glossary preamble of each type that has
been identified in the current \gls{resourceset}, using options like
\csopt{type} and \csopt{dual-type}
(\ics{apptoglossarypreamble}).
\end{itemize}
The default is \csopt[individual]{loc-prefix-def}. Note that this
can lead to an undefined control sequence error if locations appear
in a glossary that hasn't been detected by the \gls{resourceset}.

\optsection{loc-suffix}

This is similar to \csopt{loc-prefix} but there are some subtle
differences. In this case \meta{value} may either be the keyword
\optfmt{false} (in which case the location suffix is omitted) or a
comma-separated list
\optfmt{\meta{suffix-0}\dcomma\meta{suffix-1}\dcomma\ldots\dcomma\meta{suffix-n}}
where \meta{suffix-0} is the suffix to use when the
\gls{locationlist} only has a cross-reference with no
\glspl{location}, \meta{suffix-1} is the suffix to use when the
\gls{locationlist} has one \gls{location} (optionally with a
cross-reference), and so on. The final \meta{suffix-n} in the list
is the suffix when the \gls{locationlist} has \meta{n} or more
\glspl{location} (optionally with a cross-reference).

This option will append \code{\gls!{bibglslocsuffix}\margm{n}} to
\glspl{locationlist} that either have a cross-reference or have at
least one \gls{location}.  Unlike \gls{bibglslocprefix}, this
command isn't used when the \gls{locationlist} is completely empty.
Also, unlike \gls{bibglslocprefix}, this suffix command doesn't have
an equivalent to \gls{bibglspostlocprefix}.

If \meta{value} omitted, \csopt[\ics{at}\idx{periodchar}]{loc-suffix} 
is assumed.  The default is \csopt[false]{loc-suffix}.

The way the definition is written to the \ext{glstex} file is determined
by \csopt{loc-suffix-def}.
Note that if you identify some glossaries but not others (for
example, you have dual entries in separate glossaries but only
use \csopt{type} and not \csopt{dual-type}), then you will need to
use \csopt[global]{loc-suffix-def} or \csopt[local]{loc-suffix-def}.

\optsection{loc-suffix-def}

This determines how the location suffix identified by
\csopt{loc-suffix} is written to the \ext{glstex} file. The value
may be one of:
\begin{itemize}
\item\optfmt{global} the definition is globally defined using
\cs{providecommand};
\item\optfmt{local} the definition is locally defined using
\cs{providecommand} in the general glossary preamble
(\cs{glossarypreamble});
\item\optfmt{individual} the definition is locally defined using
\cs{providecommand} in the glossary preamble of each type that has
been identified in the current \gls{resourceset}, using options like
\csopt{type} and \csopt{dual-type}
(\ics{apptoglossarypreamble}).
\end{itemize}
The default is \csopt[individual]{loc-suffix-def}. Note that this
can lead to an undefined control sequence error if locations appear
in a glossary that hasn't been detected by the \gls{resourceset}.

\optsection{loc-counters}

Commands like \ics{gls} allow you to select a different 
counter to use for the \gls{location} for that specific instance
(overriding the default counter for the entry's glossary type).
This is done with the \glsopt{counter} option. For example,
consider the following document:
\begin{codeenv}
\cmd{documentclass}\marg{article}
\strut
\cmd{usepackage}\oarg{colorlinks}\marg{hyperref}
\cmd{usepackage}\oarg{\styopt{record},\styopt[tree]{style}}\marg{glossaries-extra}
\strut
\gls{GlsXtrLoadResources}\oarg{
  \csopt[entries]{src}\comment{data in entries.bib}
}
\strut
\cmd{begin}\marg{document}
\strut
\cs{gls}\marg{pi}.
\cmd{begin}\marg{equation}
\cs{gls}\oarg{\glsopt[\counter{equation}]{counter}}\marg{pi}
\cmd{end}\marg{equation}
\cmd{begin}\marg{equation}
\cs{gls}\oarg{\glsopt[\counter{equation}]{counter}}\marg{pi}
\cmd{end}\marg{equation}
\strut
\cmd{newpage}
\cmd{begin}\marg{equation}
\cs{gls}\oarg{\glsopt[\counter{equation}]{counter}}\marg{pi}
\cmd{end}\marg{equation}
\strut
\cmd{newpage}
\cs{gls}\marg{pi}.
\strut
\cmd{newpage}
\cs{gls}\marg{pi}.
\strut
\cmd{newpage}
\cs{gls}\marg{pi}.
\strut
\cmd{newpage}
\cs{printunsrtglossaries}
\cmd{end}\marg{document}
\end{codeenv}
This results in the \gls{locationlist} \qt{1, 1--3, 3--5}. This
looks a little odd and it may seem as though the \idx{implicit-range} formation
hasn't worked, but the \glspl{location} are actually: page~1, equation~1,
equation~2, equation~3, page~3, page~4 and page~5. \Idxpl{range} can't
be formed across different counters.

The \csopt[\meta{list}]{loc-counters} option instructs \bibgls\ 
to group the \glspl{location} according to the counters given in
the comma-separated \meta{list}. If a \gls{location} has a counter
that's not listed in \meta{list}, then the \gls{location} is discarded.

For example:
\begin{codeenv}
\gls{GlsXtrLoadResources}\oarg{
  \csopt[\counter{equation},\counter{page}]{loc-counters},\comment{group locations by counter}
  \csopt[entries]{src}\comment{data in entries.bib}
}
\end{codeenv}
This will first list the \glspl{location} for the \counter{equation}
counter and then the \glspl{location} for the \counter{page} counter.
Each group of \glspl{location} is encapsulated within the command
\gls!{bibglslocationgroup}\margm{n}\margm{counter}\margm{locations}.
The groups are separated by \gls!{bibglslocationgroupsep}.

The \meta{list} value must be non-empty.  Use
\csopt[as-use]{loc-counters} to restore the default behaviour, where
the \glspl{location} are listed in the document order of use, or
\csopt[false]{save-locations} to omit the \glspl{locationlist}.  Note that
you can't form counter groups from
\hyperref[sec:supplementalopts]{supplemental location lists}.

\optsection{save-index-counter}

\emph{This option requires at least version 1.29 of
\isty{glossaries-extra}.} The \meta{value} may be one of:
\begin{itemize}
\item \optfmt{false}: don't create the \field{indexcounter} field
(default);
\item \optfmt{true}: create the \field{indexcounter} field with the
value set to the first \counter{wrglossary} \gls{location};
\item \meta{encap}: create the \field{indexcounter} field with the
value set to the first \counter{wrglossary} \gls{location}
where the \glsopt{format} is \meta{encap}.
\end{itemize}
This setting will have no effect if the \styopt{indexcounter}
package option hasn't been used. In the case where the \meta{value}
is \meta{encap}, make sure that this format takes priority in the
\gls{location} precedence rules (\longarg{map-format}). If the
\gls{location} with that \meta{encap} format value is discarded then
it can't be saved.

The \styopt{indexcounter} package option
(\sty{glossaries-extra} v1.29+) creates a new counter called
\counter{wrglossary} that's incremented every time a term is
indexed (recorded), except for cross-references such as \cs{glssee}.
The increment is performed using \ics{refstepcounter} and is
followed by \code{\ics{label}\marg{wrglossary.\meta{n}}} where
\meta{n} is the value of the \counter{wrglossary} counter.
This option is intended for use with the \sty{hyperref} package to
allow \glspl{location} to link back to the particular part of the page where
the term was referenced rather than to the top of the page.

\begin{important}
Take care not to confuse this with the \field{indexed} special internal field
introduced in \sty{glossaries-extra} v1.49+. This is incremented on a per-entry basis 
and does not have an associated counter.
\end{important}

The \styopt{indexcounter} package option also automatically
implements the option \styopt[wrglossary]{counter}, which means that
each instance of \code{\ics{gls}\margm{id}} writes the label
information to the \iext{aux} file:
\begin{codeenv}
\cmd{newlabel}\marg{wrglossary.\meta{n}}\marg{\margm{n}\margm{page}\marg{}\marg{wrglossary.\meta{n}}\marg{}}
\end{codeenv}
(where \meta{page} is the page number) followed by the record:
\begin{codeenv}
\gls{glsxtr@record}\margm{id}\marg{}\marg{wrglossary}\marg{\encap{glsnumberformat}}\margm{n}
\end{codeenv}
The \gls{location} here is actually the value of the \counter{wrglossary}
counter not the page number, but \bibgls\ can pick up the
corresponding \meta{page} from the \csfmt{newlabel} command. It then
replaces the \gls{record}['s] location \meta{n} with:
\nosecdef{glsxtr@wrglossarylocation}
(but it only does this for \glspl{record} that have the \counter{wrglossary}
counter).

The \sty{glossaries-extra} package (v1.29+) adjusts the definition
of \ics{glshypernumber} (which is internally used by
\cs{glsnumberformat}, \cs{hyperbf} etc when \sty{hyperref} has been
loaded) so that if the counter is \counter{wrglossary} then
\ics{pageref} is used instead of \ics{hyperlink}. This means that the
page number is displayed in the \gls{locationlist} but it links back to
the place where the corresponding \cs{label} occurred.

This method works partially with \idx!{makeindex} and \idx!{xindy}
but from their point of view the \gls{location} is the value of the
\counter{wrglossary} counter, which interferes with their ability to
merge duplicate page numbers and form \idxpl{range}. Since \bibgls\ is
designed specifically to work with \sty{glossaries-extra}, it's
aware of this special counter and will merge and collate the
\glspl{location} according to the corresponding page number instead.

With the default \longarg{merge-wrglossary-records} switch, if a
term has multiple \counter{wrglossary} records for a given page they
will be merged. The reference link will be the dominant record for
that page.

The \csopt{save-index-counter} option allows you to save the first
of the \counter{wrglossary} \glspl{location} for a given entry or the first
instance of a specific format of the \counter{wrglossary} \glspl{location}
for a given entry. This \gls{location} is stored in the
\field{indexcounter} internal field using:
\begin{codeenv}
\ics{GlsXtrSetField}\margm{id}\marg{\field{indexcounter}}\marg{\gls{glsxtr@wrglossarylocation}\margm{n}\margm{page}}
\end{codeenv}
Since \gls{glsxtr@wrglossarylocation} simply expands to its first
argument, the corresponding label can be obtained with:
\begin{codeenv}
wrglossary.\gls{glsxtr@wrglossarylocation}\margm{n}\margm{page}
\end{codeenv}
For convenience, \sty{glossaries-extra-bib2gls} provides:
\nosecdef{GlsXtrIndexCounterLink}
which will do:
\begin{codeenv*}
\ics{cs.hyperref.int}\oarg{wrglossary.\meta{value}}\margm{text}
\end{codeenv*}
where \meta{value} is the value of the \field{indexcounter} field if
it has been set. If the \field{indexcounter} field hasn't been set
(or if \sty{hyperref} hasn't been loaded) then just \meta{text} is
done.

This provides a convenient way of encapsulating the \field{name} in
the glossary so that it links back to the first \counter{wrglossary}
entry or the first \glsopt[\meta{encap}]{format} \counter{wrglossary}
entry. This encapsulation can be done by providing a new glossary
style or more simply by redefining \ics{glsnamefont}:
\begin{codeenv}
\cmd{renewcommand}\marg{\cs{glsnamefont}}[1]\marg{\comment{}
 \gls{GlsXtrIndexCounterLink}\marg{\idx{param}1}\marg{\cs{glscurrententrylabel}}}
\end{codeenv}
Here's a complete example:
\begin{codeenv}
\cmd{documentclass}\marg{article}
\strut
\cmd{usepackage}\marg{lipsum}\comment{dummy filler text}
\cmd{usepackage}\oarg{colorlinks}\marg{hyperref}
\cmd{usepackage}\oarg{\styopt{record},\styopt{indexcounter}}\marg{glossaries-extra}
\strut
\cmd{newcommand}\marg{\cmd{primary}}[1]\marg{\cs{hyperbf}\marg{\idx{param}1}}
\strut
\gls{GlsXtrLoadResources}\oarg{
 \csopt[entries]{src},\comment{terms defined in entries.bib}
 \csopt[primary]{save-index-counter}
}
\strut
\cmd{renewcommand}\marg{\cs{glsnamefont}}[1]\marg{\comment{}
 \gls{GlsXtrIndexCounterLink}\marg{\idx{param}1}\marg{\cs{glscurrententrylabel}}}
\strut
\cmd{begin}\marg{document}
\strut
A \cs{gls}\marg{sample}. \cmd{lipsum}*[1] A \cs{gls}\marg{duck}.
\strut
An equation:
\cmd{begin}\marg{equation}
\cs{gls}\oarg{\glsopt[\counter{equation}]{counter}}\marg{pi}
\cmd{end}\marg{equation}
\strut
\cmd{lipsum}[2]
\strut
Another \cs{gls}\oarg{\glsopt[primary]{format}}\marg{sample}. \cmd{lipsum}*[3] Another
\cs{gls}\marg{duck}.
\strut
\cs{gls}\marg{pi}. \cmd{lipsum}[4]
\strut
A \cs{gls}\marg{sample}. \cmd{lipsum}*[5] A \cs{gls}\marg{duck} and
\cs{gls}\oarg{\glsopt[primary]{format}}\marg{pi}.
\strut
\cmd{lipsum}*[6] A \cs{gls}\oarg{\glsopt[primary]{format}}\marg{duck}.
\strut
\cs{printunsrtglossaries}
\cmd{end}\marg{document}
\end{codeenv}
Note that the \glsopt[\counter{equation}]{counter} entry will have its own
independent \gls{location}. In this example, it's difficult to tell the
difference between 1 (the equation reference) and 1 (the page
reference) in the \gls{locationlist} for the \code{pi} entry.

The \glsopt[primary]{format} instances indicate
\glsdisp{principallocation}{principal} references.
They're displayed in bold (since \csfmt{primary} is defined to use
\csfmt{hyperbf}) and these are the \glspl{location} saved in the
\field{indexcounter} field because that's the \meta{encap}
identified by the \csopt[primary]{save-index-counter} setting.

\section{Supplemental Locations}
\label{sec:supplementalopts}

\emph{These options require at least version 1.14 of
\isty{glossaries-extra}.} If you require \glspl{location} from multiple
external sources, then you need at least version 1.36 of
\isty{glossaries-extra} (or, more specifically,
\isty{glossaries-extra-bib2gls}, which is automatically loaded
by the \styopt[only]{record} package option).

The \sty{glossaries-extra} package (from v1.14) provides a way of
manually adding \glspl{location} in \glspl{supplementaldocument} through the use
of the \glsaddopt{thevalue} option in the optional argument of
\ics{glsadd}.  Setting values manually is inconvenient and can result
in errors, so \bibgls\ provides a way of doing this automatically.
Both the \gls{maindocument} and the \supplementarydocument\ need to use
the \styopt{record} option. The entries provided in the \csopt{src}
set must have the same labels as those used in the 
\supplementarydocument. (The simplest way to achieve this is to ensure that both
documents use the same \ext{bib} files and the same prefixes.)

For example, suppose the file \filefmt{entries.bib} contains:
\begin{codeenv}
\atentry{entry}\marg{sample,
  \field{name}=\marg{sample},
  \field{description}=\qtdelim{an example entry}
}
\strut
\atentry{abbreviation}\marg{html,
  \field{short}=\qtdelim{html},
  \field{long}=\marg{hypertext markup language}
}
\strut
\atentry{abbreviation}\marg{ssi,
  \field{short}=\qtdelim{ssi},
  \field{long}=\qtdelim{server-side includes}
}
\strut
\atentry{index}\marg{goose,\field{plural}=\qtdelim{geese}}
\end{codeenv}
Now suppose the supplementary document is contained in the file
\filefmt{suppl.tex}:
\begin{codeenv}
\cmd{documentclass}\marg{article}
\strut
\cmd{usepackage}\oarg{colorlinks}\marg{hyperref}
\cmd{usepackage}\oarg{\styopt{record},\styopt[section]{counter}}\marg{glossaries-extra}
\strut
\gls{GlsXtrLoadResources}\oarg{\csopt[entries]{src}}
\strut
\cmd{renewcommand}\marg{\cmd{thesection}}\marg{S\cmd{arabic}\marg{section}}
\cmd{renewcommand}\marg{\cmd{theHsection}}\marg{\cmd{thepart}.\cmd{thesection}}
\strut
\cmd{begin}\marg{document}
\cmd{part}\marg{Sample Part}
\cmd{section}\marg{Sample Section}
\cs{gls}\marg{goose}. \cs{gls}\marg{sample}.
\strut
\cmd{part}\marg{Another Part}
\cmd{section}\marg{Another Section}
\cs{gls}\marg{html}.
\cs{gls}\marg{ssi}.
\strut
\cs{printunsrtglossaries}
\cmd{end}\marg{document}
\end{codeenv}
This uses the \counter{section} counter for the \glspl{location} and has a
prefix (\code{\csfmt{thepart}.})\ for the section hyperlinks.

Now let's suppose I have another document called \filefmt{main.tex}
that uses the \code{sample} entry, but also needs to include the
\gls{location} (S1) from the \supplementarydocument. The manual approached
offered by \styfmt{glossaries-extra} is quite cumbersome and requires
setting the \catattr{externallocation} attribute and using
\ics{glsadd} with \glsaddopt[S1]{thevalue}, \glsaddopt[I.S1]{theHvalue}
and \glsaddopt[\encap{glsxtrsupphypernumber}]{format}.

This can be simplified with \bibgls\ by using the
\optfmt{supplemental-locations} option, described below. 

Version 1.36 of \isty{glossaries-extra-bib2gls} introduces some
special \gls{location} formatting commands that don't use the
\catattr{externallocation} attribute, but instead have an extra
argument that indicates the external reference. The additional
argument means that it can't be used by the \glsaddopt{format}
key, but with \bibgls\ you don't use \cs{glsadd} to record
the external \glspl{location}. Instead it obtains the \glspl{record} from
the corresponding \supplementary\ \ext{aux} file, and adjusts the
\gls{location} encapsulator as appropriate.

If \bibgls\ detects an older version of \sty{glossaries-extra},
it will only allow one external \supplemental\ source, and
will set the \catattr{externallocation} attribute and use
the \encap{glsxtrsupphypernumber} format. Otherwise \bibgls\
will allow multiple sources and use the newer method.

\optsection{supplemental-locations}

The value should be the base name (without the extension) of the
\supplementarydocument\ (\optfmt{suppl} in the above example).
If you have at least version 1.36 of \sty{glossaries-extra},
the value may be a comma-separated list of base names (without the
extensions) of the \supplementarydocuments. If an older version is
detected, \bibgls\ will issue a warning and only accept the first
element of the list.

For example:
\begin{codeenv}
\cmd{documentclass}\marg{article}
\strut
\cmd{usepackage}\oarg{colorlinks}\marg{hyperref}
\cmd{usepackage}\oarg{\styopt{record}}\marg{glossaries-extra}
\strut
\gls{GlsXtrLoadResources}\oarg{
 \csopt[suppl]{supplemental-locations},\comment{fetch records from suppl.aux}
 \csopt[entries]{src}}
\strut
\cmd{begin}\marg{document}
\cs{Gls}\marg{sample} document.
\strut
\cs{printunsrtglossaries}
\cmd{end}\marg{document}
\end{codeenv}
The \gls{locationlist} for \code{sample} will now be \qt{1, S1} (page~1
from the \gls{maindocument} and S1 from the \supplementarydocument).

With \sty{glossaries-extra} v1.36+, a regular \gls{location} from the
\supplementarydocument\ will be encapsulated with:
\nosecdef{glsxtrdisplaysupploc}
By default, this simply creates an external hyperlink to the
\supplementarydocument\ with the \gls{location} as the hyperlink text.
The hyperlink is created using \meta{src} as the target path
with the fragment part (anchor) formed from the prefix and 
\gls{location}. The \catattr{externallocation} attribute is not set in
this case. The actual formatting is done via:
\nosecdef{glsxtrmultisupplocation}
which ignores the \meta{format} argument by default. Its
definition is simply:
\begin{codeenv}
\cmd{newcommand}*\marg{\gls{glsxtrmultisupplocation}}[3]\marg{\comment{}
 \marg{\comment{scope required to localise changes}
   \cmd{def}\cs{glsxtrsupplocationurl}\marg{\idx{param}2}\comment{}
   \cs{glshypernumber}\marg{\idx{param}1}\comment{}
 }\comment{}
}
\end{codeenv}
This locally sets the command \ics{glsxtrsupplocationurl}, which is checked 
by \ics{glshypernumber} to establish an external rather than internal link. 
You can redefine the \supplementallocation\ command to retain the original 
\idx{encap} used in the target document:
\begin{codeenv}
\cmd{renewcommand}*\marg{\gls{glsxtrmultisupplocation}}[3]\marg{\comment{}
 \marg{\comment{scope required to localise changes}
   \cmd{def}\cs{glsxtrsupplocationurl}\marg{\idx{param}2}\comment{}
   \ics{csuse}\marg{\idx{param}3}\marg{\idx{param}1}\comment{}
 }\comment{}
}
\end{codeenv}
but remember that if a hyperlink is required, the identified
control sequence name must correspond to a command that
uses \cs{glshyperlink} (such as \cs{hyperbf}), otherwise you
will lose the hyperlink.

With older versions of \sty{glossaries-extra},
the original \gls{location} format from the \supplementarydocument\
will be replaced by \encap{glsxtrsupphypernumber}, which 
again produces an external hyperlink. The \catattr{externallocation}
attribute also needs to be set (this can be done automatically with
\csopt{supplemental-category}) to identify the external document.
The original format can't be accessed.

In both cases, if the document hasn't loaded the \isty{hyperref} package, the
\gls{location} will simply be displayed without a hyperlink.  Even if both the main
and the \supplementarydocuments\ have loaded \sty{hyperref}, note that not all
PDF viewers can handle external hyperlinks, and some that can open the external
PDF file may not recognise the destination within that file.

The special \code{nameref} \glspl{location} (see
\longarg{merge-nameref-on}) are still identified with
\gls!{glsxtrdisplaylocnameref} but the \meta{file} argument will now
be set.

As from \bibgls\ v1.7, any awkward characters in the file path are
replaced with \gls{bibglshrefchar} or (for \idx{non-ASCII}
characters, when supported) \gls{bibglshrefunicode}. Both commands
take two arguments: the hexadecimal character code and the actual
character. In the case of \gls{bibglshrefchar}, the second
argument is ignored, and the first is preceded by a literal percent
character, so \filefmt{file name.pdf} will be converted to:
\begin{codeenv}
file\gls{bibglshrefchar}\marg{20}\marg{ }name.pdf
\end{codeenv}
which will expand to \verb|file%20name.pdf|.
In the case of \gls{bibglshrefunicode}, the first argument is
ignored, so \filefmt{skr\'aarnafn.pdf} will be converted to:
\begin{codeenv}
skr\gls{bibglshrefunicode}\marg{E1}\marg{\'a}arnafn.pdf
\end{codeenv}
which will expand to \code{skr\'aarnafn.pdf}.


The \supplementarylocation\ lists are encapsulated within
\gls{bibglssupplemental}. With \sty{glossaries-extra} v1.36+,
this command will encapsulate the sub-lists with
\gls{bibglssupplementalsublist}.

So the above example with an old version of \sty{glossaries-extra}
(pre 1.36) will set the \supplementallocation\ list (which only consists
of one \gls{location}) to:
\begin{codeenv}
\gls{bibglssupplemental}
\marg{1}\marg{\ics{setentrycounter}\oarg{I}\marg{\counter{section}}\cs{glsxtrsupphypernumber}\marg{S1}}
\end{codeenv}
and the external target must be supplied through the
\catattr{externallocation} attribute, which can be set 
with the \csopt{supplemental-category} option.

Whereas with at least version 1.36, the list will be:
\begin{codeenv}
\gls{bibglssupplemental}\marg{1}\marg{\gls{bibglssupplementalsublist}\marg{1}\marg{suppl.pdf}
\marg{\gls{glsxtrdisplaysupploc}\marg{I}\marg{\counter{section}}\marg{\encap{glsnumberformat}}\marg{suppl.pdf}\marg{S1}}}
\end{codeenv}

If an entry has both a main \gls{locationlist} and a
\supplementarylocation\ list (such as the \code{sample} entry above), the lists
will be separated by \gls!{bibglssupplementalsep}. The sub-lists
(when supported) are separated by \gls{bibglssupplementalsubsep}.

\optsection{supplemental-selection}

In the above example, only the \code{sample} entry is listed in the
\gls{maindocument}, even though the \supplementarydocument\ also
references the \code{goose}, \code{html} and \code{ssi} entries. By
default, only those entries that are referenced in the \gls{maindocument}
will have \supplementarylocations\ added (if found in the
\glsdisp{supplementaldocument}{supplementary document's} \iext{aux}
file). You can additionally include other entries that are
referenced in the \supplementarydocument\ but not in the main
document using \optfmt{supplemental-selection}. The \meta{value} may
be one of the following:
\begin{itemize}
\item \optfmt{all}: add all the entries in the
\supplementarydocument\ that have been defined in the \ext{bib} files listed in
\csopt{src} for this resource set in the \gls{maindocument}.
\item \optfmt{selected}: only add \supplementallocations\ for entries
that have already been selected by this \gls{resourceset}.
\item \meta{label-1},\ldots,\meta{label-2}: in addition to all those
entries that have already been selected by this resource set, also
add the entries identified in the comma-separated list. If a label
in this list doesn't have a record in the
\glsdisp{supplementaldocument}{supplementary document's}
\ext{aux} file, it will be ignored.
\end{itemize}
Any records in the \supplementary\ \ext{aux} file that aren't defined
by the current resource set (through the \ext{bib} files listed in
\csopt{src}) will be ignored. Entry aliases aren't taken into
account when including \supplementarylocations.

For example:
\begin{codeenv}
\cmd{documentclass}\marg{article}
\strut
\cmd{usepackage}\oarg{colorlinks}\marg{hyperref}
\cmd{usepackage}\oarg{\styopt{record}}\marg{glossaries-extra}
\strut
\gls{GlsXtrLoadResources}\oarg{
 \csopt[suppl]{supplemental-locations},
 \csopt[html,ssi]{supplemental-selection},
 \csopt[entries]{src}}
\strut
\cmd{begin}\marg{document}
\cs{Gls}\marg{sample} document.
\strut
\cs{printunsrtglossaries}
\cmd{end}\marg{document}
\end{codeenv}
This will additionally add the \code{html} and \code{ssi} entries
even though they haven't been used in this document. The
\code{goose} entry used in the \supplementarydocument\ won't be
included.

\optsection{supplemental-category}

The \field{category} field for entries containing \supplementallocation\ 
lists may be set using this option. If unset,
\meta{value} defaults to the same as that given by the
\csopt{category} option. The \meta{value} may either be a known
identifier (as per \csopt{category}) or the category label. For example:
\begin{codeenv}
\cmd{documentclass}\marg{article}
\strut
\cmd{usepackage}\oarg{colorlinks}\marg{hyperref}
\cmd{usepackage}\oarg{\styopt{record}}\marg{glossaries-extra}
\strut
\gls{GlsXtrLoadResources}\oarg{
 \csopt[suppl]{supplemental-locations},
 \csopt[html,ssi]{supplemental-selection},
 \csopt[supplemental]{supplemental-category},
 \csopt[entries]{src}}
\strut
\cmd{begin}\marg{document}
\cs{Gls}\marg{sample} document.
\strut
\cs{printunsrtglossaries}
\cmd{end}\marg{document}
\end{codeenv}

A value of \code{false} will switch off this setting (the default).

\section{Sorting}
\label{sec:sortingopts}

Entries are typically displayed in an ordered list, but the
\styfmt{glossaries-extra} package is versatile enough to be used in
wider contexts than simple terms, symbols or abbreviations. For
example, entries could contain theorems or problems where the
\field{name} supplies the title and the \field{description} provides
a description of the theorem or problem. Another field might then
contain the proof or solution.  Therefore, somewhat unusually for an
indexing application, \bibgls\ also provides the option to shuffle
the entries instead of sorting them.

This section covers the resource options for sorting
\glspl{primaryentry}. See \sectionref{sec:dualoptssort} for sorting
\glspl{dualentry} and also \csopt{sort-label-list} for sorting field
values that contain a comma-separated list of entry labels (such as
the \field{see} or \field{seealso} fields).

The sort methods that use a comparison function (that is, all the
sort methods except those listed in \tableref{tab:sortoptionsnosort})
require a sort value for each entry. The function compares these
values to determine the order. By default, this sort value is
obtained from the \field{sort} field but for greater flexibility
it's best to not actually set this field. \bibgls\ has a set of fallbacks
that it uses if a field it needs to access is missing. These
fallbacks depend on the \gls{entrytype} and resource settings (see
\sectionref{sec:fallbacks}).

For example, if a term defined with \atentry{index} doesn't have the
\field{sort} field set then \bibgls\ will use the value given by the
\field{name} field because \field{name} is the fallback field for
\field{sort} for \atentry{index} entries. If the \field{name} field
isn't set either then \bibgls\ will use the fallback for that field.
In the case of \atentry{index} that's the entry's label. If the
\field{sort} field is explicitly set then there's no need to use the
fallback.

If, on the other hand, a term defined with \atentry{symbol} doesn't
have the \field{sort} field set then \bibgls\ will use the value
from the field identified by \csopt{symbol-sort-fallback}, which is
the entry's label by default (not the \field{name} field).

This means that if I don't explicitly set the \field{sort} field for
any entries then I can, for example, sort terms defined with
\atentry{index} by \field{name} and those defined with
\atentry{symbol} by \field{description} with the setting:
\begin{codeenv}
\csopt[description]{symbol-sort-fallback}
\end{codeenv}

If the field used to obtain the sort value is changed
(with \csopt{sort-field}) then the \field{sort} field won't be
queried. This reduces the flexibility of selecting the most
appropriate field for given \glspl{entrytype}. For example,
\csopt[name]{sort-field} will force all entries to be sorted by the
\field{name} field, which may not be appropriate for symbols.

\begin{important}
If you choose a field whose value must be a label (such as
\field{parent} or \field{group}) then the sort value will be that label.
\end{important}

You can have \atentry{preamble} definitions that
can be hidden from \bibgls's interpreter. For example,
\exfile{no-interpret-preamble.bib} might contain:
\begin{codeenv}
\atentry{preamble}\marg{\qtdelim{\cmd{providecommand}\marg{\gls{sortop}}[2]\marg{\idx{param}1 \idx{param}2}}}
\end{codeenv}
which is loaded using:
\begin{codeenv}
\gls{GlsXtrLoadResources}\oarg{\csopt[no-interpret-preamble]{src},
 \csopt[false]{interpret-preamble}}
\end{codeenv}
This provides a custom command:
\nosecdef{sortop}
for internal use in the document. (Remember it won't be defined on
the first \LaTeX\ run before the \iext{glstex} file has been
created and so is only used within entry fields.)

Another file, say, \filefmt{interpret-preamble.bib} may provide
a definition for \bibgls:
\begin{codeenv}
\atentry{preamble}\marg{\qtdelim{\cmd{providecommand}\marg{\gls{sortop}}[2]\marg{\idx{param}2, \idx{param}1}}}
\end{codeenv}
which can be processed with:
\begin{codeenv}
\gls{GlsXtrLoadResources}\oarg{\csopt[interpret-preamble]{src}}
\end{codeenv}
to provide \bibgls\ with this definition.
The \filefmt{entries.bib} file could contain:
\begin{codeenv}
\atentry{entry}\marg{caesar,
  \field{name}=\marg{\gls{sortop}\marg{Gaius Julius}\marg{Caesar}},
  \field{first}=\marg{Julius Caesar},
  \field{text}=\marg{Caesar},
  \field{description}=\marg{Roman politician and general}
}
\end{codeenv}
and then be processed with:
\begin{codeenv}
\gls{GlsXtrLoadResources}\oarg{\csopt[entries]{src}}
\end{codeenv}

The definition provided in \filefmt{interpret-preamble.bib}, which
swaps the two arguments around, is now picked up by \bibgls, so the
sort value becomes \code{Caesar, Gaius Julius}, but this new
definition doesn't affect the document since \LaTeX\ has already
defined \gls{sortop} from the first resource set, so the name will
appear as \qt{Gaius Julius Caesar} in the glossary. (If you have
\cs{renewcommand} rather than \cs{providecommand}, you can prevent
the redefinition occurring in the document with
\csopt[false]{write-preamble}.)

Alternatively both of these \ext{bib} files can be loaded in one
\igls{resourceset}:
\begin{codeenv}
\gls{GlsXtrLoadResources}\oarg{\csopt[interpret-preamble,entries]{src}}
\end{codeenv}
Another possibility is to provide a custom package that contains the
command definitions for the \bibgls\ interpreter and load it with
\longarg{custom-packages} instead of having the
\filefmt{interpret-preamble.bib} file.

\optsection{sort}

The \csopt{sort} key indicates how \glspl{primaryentry} should be sorted.
If the \meta{value} is omitted, \csopt[resource]{sort} is assumed. 
Note the differences between the keywords \optfmt{resource},
\optfmt{doc} and \optfmt{locale}:
\begin{description}
\item[\optfmt{resource}] The default \gls{resource-locale}, which
can be specified with the \csopt{locale} option. If that option
hasn't been set, then \optfmt{resource} will be equivalent to
\optfmt{doc}. This option is new to \bibgls\ v3.3. Previous versions
had \csopt[doc]{sort} as the default.

\item[\optfmt{doc}] The \gls{document-locale} if it has been detected by
\sty{tracklang}. If no document language has been detected (or
identified with \commonlongarg{locale}), then
\optfmt{doc} will be equivalent to \optfmt{locale}.

\item[\optfmt{locale}] The default \gls{Java-locale}.

\end{description}

The \optfmt{\meta{method}\dhyphen reverse} options reverse the
result returned by the corresponding \meta{method} comparator.
However \optfmt{\meta{method}\dhyphen reverse} may not produce a
list that's the exact reverse of the underlying non-reversed
\meta{method} as the \hierarchical\ structure or associated settings
can affect the order.

\begin{table}[p]
\caption{Summary of Available Sort Options: No Sort Field}
\label{tab:sortoptionsnosort}
\centering
\begin{tabular}{ll}
\optfmt{none} or \optfmt{unsrt} & don't sort\tabularnewline
\optfmt{random} & shuffle entries\tabularnewline
\optfmt{use} & order of use\tabularnewline
\optfmt{use-reverse} & reverse order of use\tabularnewline
\optfmt{recordcount}\textsuperscript\textdagger & order of record count\tabularnewline
\optfmt{recordcount-reverse}\textsuperscript\textdagger & reverse order of record count
\end{tabular}
\par
\medskip
\textsuperscript\textdagger Requires \longarg{record-count} switch.
\end{table}

\begin{table}[p]
\caption{Summary of Available Sort Options: Alphabet}
\label{tab:sortoptionsrule}
\centering
\begin{tabular}{ll}
\meta{lang tag} & sort according to this language tag\tabularnewline
\meta{lang tag}\optfmt{-reverse} & reverse sort according to this language tag\tabularnewline
\optfmt{resource} & sort according to the default
\gls{resource-locale}\tabularnewline
\optfmt{resource-reverse} & reverse sort according to the default
\gls{resource-locale}\tabularnewline
\optfmt{doc} & sort according to the \gls{document-locale}\tabularnewline
\optfmt{doc-reverse} & reverse sort according to the 
\gls{document-locale}\tabularnewline
\optfmt{locale} & sort according to the default
\gls{Java-locale}\tabularnewline
\optfmt{locale-reverse} & reverse sort according to the default
\gls{Java-locale}\tabularnewline
\optfmt{custom} & sort according to \csopt[\meta{custom
rule}]{sort-rule}\tabularnewline
\optfmt{custom-reverse} & reverse sort according to \csopt[\meta{custom rule}]{sort-rule}
\end{tabular}
\end{table}

\begin{table}[p]
\caption{Summary of Available Sort Options: Letter (Non-Locale)}
\label{tab:sortoptionsletter}
\centering
\begin{tabular}{ll}
\optfmt{letter-case} & case-sensitive letter sort\tabularnewline
\optfmt{letter-case-reverse} & reverse case-sensitive letter
sort\tabularnewline
\optfmt{letter-nocase} & case-insensitive letter sort\tabularnewline
\optfmt{letter-nocase-reverse} & reverse case-insensitive letter
sort\tabularnewline
\optfmt{letter-upperlower} & upper-lower letter sort\tabularnewline
\optfmt{letter-upperlower-reverse} & reverse upper-lower letter
sort\tabularnewline
\optfmt{letter-lowerupper} & lower-upper letter sort\tabularnewline
\optfmt{letter-lowerupper-reverse} & reverse lower-upper letter sort
\end{tabular}
\end{table}

\begin{table}[p]
\caption{Summary of Available Sort Options: Letter-Number}
\label{tab:sortoptionsletternumber}
\centering
\begin{tabular}{ll}
\optfmt{letternumber-case} & case-sensitive letter-number
sort\tabularnewline
\optfmt{letternumber-case-reverse} & reverse case-sensitive
letter-number sort\tabularnewline
\optfmt{letternumber-nocase} & case-insensitive letter-number
sort\tabularnewline
\optfmt{letternumber-nocase-reverse} & reverse case-insensitive
letter-number sort\tabularnewline
\optfmt{letternumber-upperlower} & upper-lower letter-number
sort\tabularnewline
\optfmt{letternumber-upperlower-reverse} & reverse upper-lower
letter-number sort\tabularnewline
\optfmt{letternumber-lowerupper} & lower-upper letter-number
sort\tabularnewline
\optfmt{letternumber-lowerupper-reverse} & reverse lower-upper letter-number sort
\end{tabular}
\end{table}

\begin{table}[p]
\caption{Summary of Available Sort Options: Numerical}
\label{tab:sortoptionsnumerical}
\centering
\begin{tabular}{ll}
\optfmt{integer} & integer sort\tabularnewline
\optfmt{integer-reverse} & reverse integer sort\tabularnewline
\optfmt{hex} & hexadecimal sort\tabularnewline
\optfmt{hex-reverse} & reverse hexadecimal sort\tabularnewline
\optfmt{octal} & octal sort\tabularnewline
\optfmt{octal-reverse} & reverse octal sort\tabularnewline
\optfmt{binary} & binary sort\tabularnewline
\optfmt{binary-reverse} & reverse binary sort\tabularnewline
\optfmt{float} & float sort\tabularnewline
\optfmt{float-reverse} & reverse float sort\tabularnewline
\optfmt{double} & double sort\tabularnewline
\optfmt{double-reverse} & reverse double sort\tabularnewline
\optfmt{numeric} & locale-sensitive numeric sort\tabularnewline
\optfmt{numeric-reverse} & reverse locale-sensitive numeric
sort\tabularnewline
\optfmt{currency} & locale-sensitive currency sort\tabularnewline
\optfmt{currency-reverse} & reverse locale-sensitive currency
sort\tabularnewline
\optfmt{percent} & locale-sensitive percent sort\tabularnewline
\optfmt{percent-reverse} & reverse locale-sensitive percent
sort\tabularnewline
\optfmt{numberformat} & locale-sensitive custom numeric
sort\tabularnewline
\optfmt{numberformat-reverse} & reverse locale-sensitive custom numeric sort
\end{tabular}
\end{table}

\begin{table}[p]
\caption{Summary of Available Sort Options: Date-Time}
\label{tab:sortoptionsdatetime}
\centering
\begin{tabular}{ll}
\optfmt{date} & locale-sensitive date sort\tabularnewline
\optfmt{date-reverse} & reverse locale-sensitive date
sort\tabularnewline
\optfmt{datetime} & locale-sensitive date-time sort\tabularnewline
\optfmt{datetime-reverse} & reverse locale-sensitive date-time
sort\tabularnewline
\optfmt{time} & locale-sensitive time sort\tabularnewline
\optfmt{time-reverse} & reverse locale-sensitive time sort
\end{tabular}
\end{table}

\subsubsection{No Sort Field}

Most of the sort methods listed in \tableref{tab:sortoptionsnosort} 
don't actually perform any sorting. This may cause a problem for
\hierarchical\ entries. In some cases this can lead to detached 
\glspl{childentry} or an attempt to define a \gls{childentry} before its \parent.
The methods listed in this section all ignore the \csopt{sort-field}
setting and all the various sort fallback settings, except where
noted below.
\begin{itemize}

\item \optfmt{none} (or \optfmt{unsrt}): don't sort the entries.
(The entries will be in the order they were processed when parsing
the data.)

If you need to order by definition but also maintain \hierarchy\ then use:
\begin{codeenv}
\csopt{save-definition-index},
\csopt[\field{definitionindex}]{sort-field},
\csopt[integer]{sort}
\end{codeenv}

\item \optfmt{random}: shuffles rather than sorts the entries.
This won't work if there are \hierarchical\ entries, so it's best
to use this option with \csopt{flatten}. The seed for the
random generator can be set using \csopt{shuffle} (which
also automatically sets \csopt[random]{sort} and \csopt{flatten}).

\item \optfmt{use}: order of use. This order is determined
by the \glspl{record} written to the \iext{aux} file by the \styopt{record}
package option. Dependencies and cross-references (including those
identified with \ics{glssee}) come after entries with \glspl{record}.

Note that this is different from using the analogous option with
\idx{makeindex} or \idx{xindy}, which does actually sort
numerically, where each entry has an associated number set on the
first use of that term that's used as the sort value.

If you need to order by use but also maintain \hierarchy\ then use:
\begin{codeenv}
\csopt{save-use-index},
\csopt[\field{useindex}]{sort-field},
\csopt[integer]{sort}
\end{codeenv}

\item \optfmt{use-reverse}: reverses the order that would be
obtained with \csopt[use]{sort} without reference to \hierarchy.

\item \optfmt{recordcount}: order of \gls{recordcount} (starting from 0). This order is
determined by the total number of records written to the \iext{aux}
file for each entry. Unlike the above methods, this performs a \hierarchical\ sort.
If letter groups are enabled with \longarg{group}, this method will assign the entries to
the \idx{numbergroup}.

This option requires the \longarg{record-count} switch. Although
that switch makes \bibgls\ write the total \gls{recordcount} to the
\ext{glstex} file in the \field{recordcount} internal field (so that it
can be accessed in the document), \bibgls\ doesn't actually have a
field itself that contains the information. So although this option
behaves much like \csopt[integer]{sort} it's not possible to select
a field containing the required value. In the event of two or more
entries having the same \gls{recordcount}, the
\csopt{identical-sort-action} option is used to determine the
relative ordering between them.

\item \optfmt{recordcount-reverse}: reverse order of \gls{recordcount}
(ending with 0). All the above notes applying to
\optfmt{recordcount} also apply here.

\end{itemize}

Suppose the file \filefmt{entries.bib} contains
definitions of a set of symbols that don't have any intuitive
ordering (for example, they are all pictographs) then there may be
no point in trying to order them, in which case you can do:
\begin{codeenv}
\gls{GlsXtrLoadResources}\oarg{\csopt[entries]{src},\csopt[none]{sort}}
\end{codeenv}
Alternatively, you could list them in order of use:
\begin{codeenv}
\gls{GlsXtrLoadResources}\oarg{\csopt[entries]{src},\csopt[use]{sort}}
\end{codeenv}
or by frequency of use. For example, starting with entries that
don't have any records followed by the least used entries (a rarely-used symbol 
may be harder to remember and most likely to be looked up in the glossary):
\begin{codeenv}
\gls{GlsXtrLoadResources}\oarg{\csopt[entries]{src},\csopt[recordcount]{sort}}
\end{codeenv}
Or starting with the most used entries:
\begin{codeenv}
\gls{GlsXtrLoadResources}\oarg{\csopt[entries]{src},\csopt[recordcount-reverse]{sort}}
\end{codeenv}
It all depends on what's likely to be most useful to the reader.

Consider the following:
\begin{codeenv}
\ics{newglossary*}\marg{frequent}\marg{Most Frequently Used Terms}
\gls{GlsXtrLoadResources}\oarg{\csopt[entries]{src},\csopt[use]{sort},
 \csopt[recordcount-reverse:frequent]{secondary}
}
\cs{newcommand}\marg{\csfmt{filterhook}}[1]\marg{\comment{}
  \ics{GlsXtrIfFieldCmpNum}*\marg{\field{recordcount}}\marg{\idx{hashchar}1}\marg{>}\marg{10}\comment{}
  \marg{}\comment{}
  \marg{\ics{printunsrtglossaryskipentry}}\comment{}
}
\csfmt{begin}\marg{document}
\cs{printunsrtglossary*}\oarg{\printglossopt[false]{target},\printglossopt[frequent]{type}}\marg{\comment{}
  \cs{let}\ics{printunsrtglossaryentryprocesshook}\csfmt{filterhook}
}
\comment{Main body of the document \ldots}
\cs{printunsrtglossary}
\csfmt{end}\marg{document}
\end{codeenv}
This has a summary at the start of the document that only contains
entries that have at least 10 records and is ordered according to
the total number of records (starting with the most frequently used
entry). The \gls{mainglossary} at the end of the document is ordered
according to use and contains all selected entries.

Compare this with the following:
\begin{codeenv}
\gls{GlsXtrLoadResources}\oarg{\csopt[entries]{src},\csopt[use]{sort}}
\cs{newcommand}\marg{\csfmt{filterhook}}[1]\marg{\comment{}
  \ics{GlsXtrIfFieldCmpNum}*\marg{\field{recordcount}}\marg{\idx{hashchar}1}\marg{>}\marg{10}\comment{}
  \marg{}\comment{}
  \marg{\ics{printunsrtglossaryskipentry}}\comment{}
}
\csfmt{begin}\marg{document}
\cs{printunsrtglossary*}\oarg{\printglossopt[false]{target},
 \printglossopt[Most Frequently Used Terms]{title}}\marg{\comment{}
  \cs{let}\ics{printunsrtglossaryentryprocesshook}\csfmt{filterhook}
}
\comment{Main body of the document \ldots}
\cs{printunsrtglossary}
\csfmt{end}\marg{document}
\end{codeenv}
This again has a summary at the start of the document that only
contains entries that have at least 10 records but is now ordered
according to use. 

Both examples assume there are no \glspl{childentry} as the filtering can
cause \glspl{parententry} to be omitted. Both examples require
\longarg{record-count} but only the first example sorts according to
the \gls{recordcount}.

\subsubsection{Alphabet}

The sort methods listed in \tableref{tab:sortoptionsrule} are for
alphabets that are defined by a rule. These usually ignore most
punctuation and may ignore modifiers (such as accents).  Use with
\csopt{break-at} to determine whether or not to split at word
boundaries. The collation rules (except for the custom options) are obtained
from the \idx{localeprovider} (see page~\pageref{locale.provider}).

\begin{itemize}
\item \meta{lang tag}: sort according to the rules of the locale
given by the \idx{IETF} language tag \meta{lang tag}.

\item \meta{lang tag}\optfmt{-reverse}: reverse sort according to the rules of the locale
given by the \idx{IETF} language tag \meta{lang tag}.

\item \optfmt{resource}: equivalent to 
\csopt[\meta{lang tag}]{sort} where \meta{lang tag} is obtained
from the default \gls{resource-locale}.

\item \optfmt{resource-reverse}: equivalent to 
\csopt[\meta{lang tag}-reverse]{sort} where \meta{lang tag} is obtained
from the default \gls{resource-locale}.

\item \optfmt{locale}: equivalent to 
\csopt[\meta{lang tag}]{sort} where \meta{lang tag} is obtained
from the \gls{Java-locale} (which usually matches the
operating system's locale).

\item \optfmt{locale-reverse}: equivalent to 
\csopt[\meta{lang tag}-reverse]{sort} where \meta{lang tag} is obtained
from the \gls{Java-locale}.

\item \optfmt{doc}: sort the entries according to the
\gls{document-locale}. This is equivalent to \csopt[\meta{lang tag}]{sort}
where \meta{lang tag} is the locale associated with the document
language. In the case of a multi-lingual document, \meta{lang tag}
is the locale of the last language resource file to be loaded through
\isty{tracklang}'s interface. It's best to explicitly set the locale
for multi-lingual documents to avoid confusion. If no document
language has been set, this option is equivalent to \csopt[locale]{sort}.

\item \optfmt{doc-reverse}: as \optfmt{doc} but in reverse order.

\item \optfmt{custom}: sort the entries according to the 
rule provided by \csopt{sort-rule}.

\item \optfmt{custom-reverse}: reverse sort the entries according to the 
rule provided by \csopt{sort-rule}.
\end{itemize}

Note that \csopt[\meta{lang tag}]{sort} can provide more detail about the
given locale than \csopt[doc]{sort}, depending on how the document
language has been specified.  For example, with:
\begin{codeenv}
\cmd{documentclass}\marg{article}
\cmd{usepackage}\oarg{ngerman}\marg{babel}
\cmd{usepackage}\oarg{\styopt{record}}\marg{glossaries}
\gls{GlsXtrLoadResources}\oarg{\csopt[german-terms]{src}}
\end{codeenv}
the language tag will be \code{de-1996}, which doesn't have an
associated region, so this is equivalent to using \csopt[de-1996]{sort}. 
Whereas with:
\begin{codeenv}
\cmd{documentclass}\oarg{de-DE-1996}\marg{article}
\cmd{usepackage}\oarg{ngerman}\marg{babel}
\cmd{usepackage}\oarg{\styopt{record}}\marg{glossaries}
\gls{GlsXtrLoadResources}\oarg{\csopt[german-terms]{src}}
\end{codeenv}
the language tag will be \code{de-DE-1996} because \isty{tracklang}
has picked up the locale from the document class options, so this is 
equivalent to using \csopt[de-DE-1996]{sort}. This is
only likely to cause a difference if a language has different
sorting rules according to the region or if the language may be
written in multiple scripts.

If no \gls{document-locale} has been set and the
\csopt{locale} resource option hasn't been used 
then the \csopt[resource]{sort} and \csopt[doc]{sort} 
will be equivalent to \csopt[locale]{sort}.
For example, with:
\begin{codeenv}
\cmd{documentclass}\marg{article}
\cmd{usepackage}\oarg{\styopt{record}}\marg{glossaries}
\gls{GlsXtrLoadResources}\oarg{\csopt[german-terms]{src}}
\end{codeenv}
the language tag will be whatever is the default locale for the \idx{JVM}.
For a user in Germany, this could be \code{de-DE-1996} and for a user
in Austria this could be \code{de-AT-1996}.

A multilingual document will need to have the \csopt{sort} specified
when loading the \idx{resourceset} to ensure the correct language is chosen. 
For example:
\begin{codeenv}
\gls{GlsXtrLoadResources}\oarg{\csopt[english-terms]{src},\csopt[en-GB]{sort}}
\gls{GlsXtrLoadResources}\oarg{\csopt[german-terms]{src},\csopt[de-DE-1996]{sort}}
\end{codeenv}

Alternatively (as from \bibgls\ v3.3), use \csopt{locale}:
\begin{codeenv}
\gls{GlsXtrLoadResources}\oarg{\csopt[en-GB]{locale},\csopt[english-terms]{src}}
\gls{GlsXtrLoadResources}\oarg{\csopt[de-DE-1996]{locale},\csopt[german-terms]{src}}
\end{codeenv}

\subsubsection{Letter (Non Locale)}

The sort methods listed in \tableref{tab:sortoptionsletter} 
use letter comparators. These simply compare the
character codes. The \optfmt{-nocase} options first convert the
\field{sort} field to \idx{lowercase} before performing the sort to
provide a case-insensitive comparison. 

Punctuation isn't ignored.
Use \csopt[\meta{lang tag}]{sort} with \csopt[none]{break-at} to
emulate \idx!{xindy}['s] locale letter ordering. The examples below
show the ordering of the list \code{antelope}, \code{bee}, \code{Africa}, 
\code{aardvark} and \code{Brazil}.
\begin{itemize}
\item \optfmt{letter-case}: case-sensitive letter sort.
\Idx{uppercase} and \idx{lowercase} are in separate letter groups.
Example: 

\code{Africa} (letter group \idx{uppercase} \qt{A}), \code{Brazil}
(letter group \idx{uppercase} \qt{B}), \code{aardvark} (letter group 
\idx{lowercase} \qt{a}), \code{antelope} (letter group
\idx{lowercase} \qt{a}), \code{bee} (letter group \idx{lowercase} \qt{b}).

\item \optfmt{letter-case-reverse}: reverse case-sensitive letter sort.
Example:

\code{bee} (letter group \idx{lowercase} \qt{b}), \code{antelope}
(letter group \idx{lowercase} \qt{a}), \code{aardvark} (letter group
\idx{lowercase} \qt{a}) \code{Brazil} (letter group \idx{uppercase}
\qt{B}), \code{Africa} (letter group \idx{uppercase} \qt{A}).

\item \optfmt{letter-nocase}: case-insensitive letter sort. (All
\idx{uppercase} characters will have first been converted to
\idx{lowercase} in the sort value.)
Example:

\code{aardvark} (letter group \qt{A}), \code{Africa}
(letter group \qt{A}), \code{antelope} (letter group \qt{A}), 
\code{bee} (letter group \qt{B}),
\code{Brazil} (letter group \qt{B}).

\item \optfmt{letter-nocase-reverse}: reverse case-insensitive letter sort.
Example:

\code{Brazil} (letter group \qt{B}), \code{bee} (letter group
\qt{B}), \code{antelope} (letter group \qt{A}), \code{Africa}
(letter group \qt{A}), \code{aardvark} (letter group \qt{A}).

\item \optfmt{letter-upperlower}: each character pair is first
compared according to their \idx{lowercase} values. If these are equal,
then they are compared according to case. This puts upper and lower
case in the same letter group but the \idx{uppercase} comes first.
Example:

\code{Africa} (letter group \qt{A}), \code{aardvark} (letter group
\qt{A}), \code{antelope} (letter group \qt{A}), \code{Brazil}
(letter group \qt{B}), \code{bee} (letter group \qt{B}).

\item \optfmt{letter-upperlower-reverse}: reverse upper-lower letter sort.
This now puts the \idx{lowercase} letters first within the letter group.
Example:

\code{bee} (letter group \qt{B}), \code{Brazil} (letter group
\qt{B}), \code{antelope} (letter group \qt{A}), \code{aardvark}
(letter group \qt{A}), \code{Africa} (letter group \qt{A}).

\item \optfmt{letter-lowerupper}: each character pair is first
compared according to their \idx{lowercase} values. If these are equal,
then they are compared according to case. This puts
\idxlink{uppercase}{upper} and \idx{lowercase} in the same letter group but the \idx{lowercase} comes first.
Example:

\code{aardvark} (letter group \qt{A}), \code{antelope} (letter group \qt{A}),
\code{Africa} (letter group \qt{A}), \code{bee} (letter group
\qt{B}), \code{Brazil} (letter group \qt{B}).

\item \optfmt{letter-lowerupper-reverse}: reverse lower-upper letter sort.
This now puts the \idx{uppercase} letters first within the letter group.
Example:

\code{Brazil} (letter group \qt{B}), \code{bee} (letter group
\qt{B}), \code{Africa} (letter group \qt{A}), \code{antelope}
(letter group \qt{A}), \code{aardvark} (letter group \qt{A}).

\end{itemize}

\subsubsection{Letter-Number}
\label{sec:letternumber}

The sort methods listed in \tableref{tab:sortoptionsletternumber} 
use a letter-integer hybrid. They behave in a similar way to the
above letter sort methods, but if an integer number pattern is detected in
the string then the sub-string containing the number will be compared.
This only detects base 10 integers (unlike the numeric methods
such as \csopt[hexadecimal]{sort} or \csopt[float]{sort}) but in
addition to recognising all the digits in the Unicode \idx{numberdecimaldigit} 
category it also recognises the subscript and
superscript digits, such as \textsuperscript{1} (\hex{00B9})
and \textsuperscript{2} (\hex{00B2}).

As with the letter sort methods, letters are compared using a
character code comparison not by a locale alphabet. The closest
locale-sensitive equivalent is to use \csopt{sort-number-pad} with a
locale sort method. Alternatively, use \ics{IfTeXParserLib} or
\ics{IfNotBibGls} and
\ics{bibglspaddigits} to pad the number for the interpreter but
not in the \TeX\ document.

\begin{figure}
\figcontents
{%
\begin{tikzpicture}[every node/.style={minimum size=9mm},
 ampersand replacement=\&]
\matrix[font={\Large\ttfamily}]
{
  \node[draw]{a\hexsb{61}}; \& \node{$=$}; \& \node[draw]{a\hexsb{61}}; \&[20mm]
  \node[draw]{a\hexsb{61}}; \& \node{$=$}; \& \node[draw]{a\hexsb{61}};
  \\
  \node[draw]{b\hexsb{62}}; \& \node{$=$}; \& \node[draw]{b\hexsb{62}}; \&
  \node[draw]{b\hexsb{62}}; \& \node{$=$}; \& \node[draw]{b\hexsb{62}};
  \\
  \node[draw]{c\hexsb{63}}; \& \node{$=$}; \& \node[draw]{c\hexsb{63}}; \&
  \node[draw]{c\hexsb{63}}; \& \node{$=$}; \& \node[draw]{c\hexsb{63}};
  \\
  \node[draw,ultra thick]{1\hexsb{31}}; \& \node{$<$}; \&
  \node[draw,ultra thick]{6\hexsb{36}}; \&
  \node (n1) {1}; \& \node{$>$}; \& \node[draw,ultra thick]{6};
  \\
  \node[draw]{2\hexsb{32}}; \&  \& \node[draw]{b\hexsb{62}}; \&
  \node (n2) {2}; \&  \& \node[draw]{b\hexsb{62}};
  \\
  \node[draw]{f\hexsb{66}}; \&  \& \node[draw]{a\hexsb{61}}; \&
  \node[draw]{f\hexsb{66}}; \&  \& \node[draw]{a\hexsb{61}};
  \\
  \node[draw]{o\hexsb{6f}}; \&  \& \node[draw]{r\hexsb{72}}; \&
  \node[draw]{o\hexsb{6f}}; \&  \& \node[draw]{r\hexsb{72}};
  \\
  \node[draw]{o\hexsb{6f}}; \&  \& \&
  \node[draw]{o\hexsb{6f}}; \&  \& 
  \\[10pt]
   \& \node[font=\rmfamily]{\subfigfmt{a}};  \& \&
   \& \node[font=\rmfamily]{\subfigfmt{b}};  \& 
  \\
};
\draw[ultra thick] (n1.north east) rectangle (n2.south west);
\end{tikzpicture}%
}
{%
 \caption[Regular letter comparison vs letter-number
 comparison]{Regular letter comparison vs letter-number comparison.
 Comparing the strings \code{abc12foo} and \code{abc6bar}:
 \subfigfmt{a} \optfmt{letter-case}; \subfigfmt{b}
 \optfmt{letternumber-case}.}%
}
{fig:letternumber}
\end{figure}

For example, suppose the first string is \code{abc12foo}
and the second string is \code{abc6bar}.
\Figureref{fig:letternumber}\subfigfmt{a} shows the regular letter
comparison using \csopt[letter-case]{sort}, where the subscript
indicates the hexadecimal character code. The first three characters
from each string are identical (\code{abc}). At this point there's
no difference detected, so the comparator moves on to the next
character, \code{1\hexsb{31}} for the first string and
\code{6\hexsb{36}} for the second string. Since \hex{31} is less than 
\hex{36}, the first string (\code{abc12foo}) is considered less than the second
(\code{abc6bar}).

With the letter-number comparison using
\csopt[letternumber-case]{sort}, the comparator starts in much the
same way. The first three characters from each string are still
identical, so the comparator moves on to the next character,
\code{1} for the first string and \code{6} for the second. 
These are now both recognised as digits, so the comparator
looks ahead and reads in any following digits (if present).  For the
first case, this is the sub-string \code{12} and, for the second
case, \code{6} (\figureref{fig:letternumber}\subfigfmt{b}).  These
are both compared according to their integer representation $12 >
6$, so \code{abc12bar} is considered greater than \code{abc6foo}
(that is, \code{abc12bar} comes after \code{abc6foo}). 

The same result occurs for other numbering systems, for example if
the Basic Latin digits 1, 2 and 6 are replaced with the
corresponding Devanagari digits \devone, \devtwo\ and
\devsix. (But note that the letter comparisons will still be
based on their Unicode values not according to a particular locale.
This type of sort method is intended primarily for symbolic values,
such as chemical formulae, rather than for words or phrases.)

Signed integers are also recognised, so \code{abc-12foo} is less
than \code{abc+6bar}, which is again different from the result
obtained with a straight letter comparator where the
character~\code{+} (\hex{2B})
comes before the character~\code{-} (\hex{2D}). The sign must be followed by at
least one digit for it to be recognised as a number otherwise it's
treated as a punctuation character.

If only one sub-string is numeric
then the \csopt{letter-number-rule} is used to determine the
result. Where both sub-strings are non-numeric, then the 
\csopt{letter-number-punc-rule} setting is used to determine the result
according to the category of the characters, which may be one of the
following:
\begin{itemize}
\item white space: belongs to the Unicode \idx{separatorspace}
category. If both characters are white space, then they are compared
according to their Unicode values otherwise they are ordered according to 
the \csopt{letter-number-punc-rule} setting.

\item letter: belongs to one of the Unicode categories \idx{letteruppercase}, 
\idx{letterlowercase}, \idx{lettertitlecase},
\idx{lettermodifier} or \idx{letterother}. If both characters
are letters then, for sort method \optfmt{letternumber-\meta{modifier}}, 
the characters are compared in the same way as the
corresponding \optfmt{letter-\meta{modifier}} sort method otherwise
they are ordered according to the 
\csopt{letter-number-punc-rule} setting.

\item punctuation: everything else. If both characters are
punctuation, then they are compared according to their Unicode value
otherwise they are ordered according to 
the \csopt{letter-number-punc-rule} setting.
\end{itemize}
For simplicity, the actual sort value used during sorting isn't a
simple string but is converted into a list of objects that represent
one of: letter, integer, space or other (punctuation). This reduces
the amount of parsing of substrings that needs to be performed.

The examples below show the ordering of the list:
\code{CH\textsubscript{2}O}, 
\code{C\textsubscript{10}H\textsubscript{10}O\textsubscript{4}}, 
\code{C\textsubscript{5}H\textsubscript{4}NCOOH}, 
\code{CO}, \code{Cl}, 
\code{Co}, \code{Co\textsubscript{2}O\textsubscript{3}}, 
\code{Co\textsubscript{2}}, \code{CO\textsubscript{2}},
\code{CoMoO\textsubscript{4}} and \code{CoCl\textsubscript{2}},
for the setting
\csopt[between]{letter-number-rule}, where the subscripts are
the Unicode subscript characters.

\begin{itemize}
\item \optfmt{letternumber-case}: case-sensitive
letter-number sort. Example:

\code{CH\textsubscript{2}O}, 
\code{CO},
\code{CO\textsubscript{2}},
\code{C\textsubscript{5}H\textsubscript{4}NCOOH}, 
\code{C\textsubscript{10}H\textsubscript{10}O\textsubscript{4}}, 
\code{Cl}, 
\code{Co}, 
\code{CoCl\textsubscript{2}},
\code{CoMoO\textsubscript{4}}, 
\code{Co\textsubscript{2}},
\code{Co\textsubscript{2}O\textsubscript{3}}.

(Order determined by: $\mathtt{H} < \mathtt{O} < 5 < 10 < \mathtt{l} < \mathtt{o}$.)

\item \optfmt{letternumber-case-reverse}: reverse case-sensitive
letter-number sort. Example:

\code{Co\textsubscript{2}O\textsubscript{3}},
\code{Co\textsubscript{2}},
\code{CoMoO\textsubscript{4}}, 
\code{CoCl\textsubscript{2}},
\code{Co}, 
\code{Cl}, 
\code{C\textsubscript{10}H\textsubscript{10}O\textsubscript{4}}, 
\code{C\textsubscript{5}H\textsubscript{4}NCOOH}, 
\code{CO\textsubscript{2}},
\code{CO},
\code{CH\textsubscript{2}O}.

\item \optfmt{letternumber-nocase}: case-insensitive
letter-number sort. The sort value is first converted to lower
case. Note that \csopt[between]{letter-number-rule} doesn't make
sense in this context as there won't be any \idx!{uppercase} characters in
the sort value, so numbers will always come before letters. Example:

\code{C\textsubscript{5}H\textsubscript{4}NCOOH}, 
\code{C\textsubscript{10}H\textsubscript{10}O\textsubscript{4}}, 
\code{CH\textsubscript{2}O}, 
\code{Cl}, 
\code{CO}, 
\code{Co}, 
\code{CO\textsubscript{2}},
\code{Co\textsubscript{2}},
\code{Co\textsubscript{2}O\textsubscript{3}},
\code{CoCl\textsubscript{2}},
\code{CoMoO\textsubscript{4}}. 

(Order determined by: $5 < 10 < \mathtt{h} < \mathtt{l} < \mathtt{o}$.)

\item \optfmt{letternumber-nocase-reverse}: reverse case-insensitive
letter-number sort, so numbers will now always come after letters. Example:

\code{CoMoO\textsubscript{4}}, 
\code{CoCl\textsubscript{2}},
\code{Co\textsubscript{2}O\textsubscript{3}},
\code{Co\textsubscript{2}},
\code{CO\textsubscript{2}},
\code{Co}, 
\code{CO}, 
\code{Cl}, 
\code{CH\textsubscript{2}O}, 
\code{C\textsubscript{10}H\textsubscript{10}O\textsubscript{4}}, 
\code{C\textsubscript{5}H\textsubscript{4}NCOOH}. 

\item \optfmt{letternumber-upperlower}: upper-lower
letter-number sort. This behaves slightly differently to
\optfmt{letter-upperlower} when used with \csopt[between]{letter-number-rule}
and has a more complicated rule that's determined by the character
following the number and implied numbers inserted between letters. 
(There was a bug in earlier versions that has been corrected in v1.8
so you may find a slightly different ordering when upgrading.)
Example:

\code{CH\textsubscript{2}O}, 
\code{C\textsubscript{5}H\textsubscript{4}NCOOH}, 
\code{C\textsubscript{10}H\textsubscript{10}O\textsubscript{4}}, 
\code{Cl}, 
\code{CO}, 
\code{CO\textsubscript{2}},
\code{Co}, 
\code{Co\textsubscript{2}},
\code{CoCl\textsubscript{2}},
\code{CoMoO\textsubscript{4}}, 
\code{Co\textsubscript{2}O\textsubscript{3}}.

(Order determined by: $\mathtt{H} < 5\mathtt{H} < 10\mathtt{H} <
\mathtt{l} < \mathtt{O} < \mathtt{o}$, and for the terms starting
with \code{CO} or \code{Co}: 2 comes after null and $\mathtt{C} <
\mathtt{M} < 2\mathtt{O}$.)

Compare this with \csopt[before letter]{letter-number-rule} which
results in the order:

\code{C\textsubscript{5}H\textsubscript{4}NCOOH}, 
\code{C\textsubscript{10}H\textsubscript{10}O\textsubscript{4}}, 
\code{CH\textsubscript{2}O}, 
\code{Cl}, 
\code{CO}, 
\code{CO\textsubscript{2}},
\code{Co}, 
\code{Co\textsubscript{2}},
\code{Co\textsubscript{2}O\textsubscript{3}},
\code{CoCl\textsubscript{2}},
\code{CoMoO\textsubscript{4}}.

\item \optfmt{letternumber-upperlower-reverse}: reverse upper-lower
letter-number sort. Example (with  \csopt[between]{letter-number-rule}):

\code{Co\textsubscript{2}O\textsubscript{3}},
\code{CoMoO\textsubscript{4}}, 
\code{CoCl\textsubscript{2}},
\code{Co\textsubscript{2}},
\code{Co}, 
\code{CO\textsubscript{2}},
\code{CO}, 
\code{Cl}, 
\code{C\textsubscript{10}H\textsubscript{10}O\textsubscript{4}}, 
\code{C\textsubscript{5}H\textsubscript{4}NCOOH}, 
\code{CH\textsubscript{2}O}. 

Compare this with \csopt[before letter]{letter-number-rule} which
results in the order:

\code{CoMoO\textsubscript{4}}, 
\code{CoCl\textsubscript{2}},
\code{Co\textsubscript{2}O\textsubscript{3}},
\code{Co\textsubscript{2}},
\code{Co}, 
\code{CO\textsubscript{2}},
\code{CO}, 
\code{Cl}, 
\code{CH\textsubscript{2}O}, 
\code{C\textsubscript{10}H\textsubscript{10}O\textsubscript{4}}, 
\code{C\textsubscript{5}H\textsubscript{4}NCOOH}. 

Remember that the associated settings are reversed as
well. So \csopt[before letter]{letter-number-rule} 
results in numbers \emph{after} letters.

\item \optfmt{letternumber-lowerupper}: lower-upper letter-number
sort. As with the upper-lower option, this behaves slightly differently to
\optfmt{letter-lowerupper} when used with \csopt[between]{letter-number-rule}
and has a more complicated rule.
Example:

\code{CH\textsubscript{2}O}, 
\code{C\textsubscript{5}H\textsubscript{4}NCOOH}, 
\code{C\textsubscript{10}H\textsubscript{10}O\textsubscript{4}}, 
\code{Cl}, 
\code{Co}, 
\code{Co\textsubscript{2}},
\code{CoCl\textsubscript{2}},
\code{CoMoO\textsubscript{4}}, 
\code{Co\textsubscript{2}O\textsubscript{3}},
\code{CO}, 
\code{CO\textsubscript{2}}.

Compare this with \csopt[before letter]{letter-number-rule} which
results in the order:

\code{C\textsubscript{5}H\textsubscript{4}NCOOH}, 
\code{C\textsubscript{10}H\textsubscript{10}O\textsubscript{4}}, 
\code{CH\textsubscript{2}O}, 
\code{Cl}, 
\code{Co}, 
\code{Co\textsubscript{2}},
\code{Co\textsubscript{2}O\textsubscript{3}},
\code{CoCl\textsubscript{2}},
\code{CoMoO\textsubscript{4}}, 
\code{CO}, 
\code{CO\textsubscript{2}}.

\item \optfmt{letternumber-lowerupper-reverse}: reverse lower-upper
letter-number sort. Example (with \csopt[between]{letter-number-rule}):

\code{CO\textsubscript{2}},
\code{CO}, 
\code{Co\textsubscript{2}O\textsubscript{3}},
\code{CoMoO\textsubscript{4}}, 
\code{CoCl\textsubscript{2}},
\code{Co\textsubscript{2}},
\code{Co}, 
\code{Cl}, 
\code{C\textsubscript{10}H\textsubscript{10}O\textsubscript{4}}, 
\code{C\textsubscript{5}H\textsubscript{4}NCOOH}, 
\code{CH\textsubscript{2}O}. 

\end{itemize}

\subsubsection{Numerical}
\label{sec:numerical}

The sort methods listed in \tableref{tab:sortoptionsnumerical} 
use numeric comparisons. The sort value is expected to
be a numeric value. If it can't be parsed then it's treated as 0
(and a warning will be written to the transcript).
These all recognise the digits in the Unicode \qt{Number,
Decimal Digit} category but, unlike the hybrid letter-number
comparators above, they don't recognise the superscript or subscript
digits. The \qt{non-locale} in some of the descriptions below
indicates that the method doesn't recognise locale-sensitive
formatting, such as group separators.
\begin{itemize}
\item \optfmt{integer}: integer sort. This is for non-locale integer sort
values.

\item \optfmt{integer-reverse}: as above but reverses the order.

\item \optfmt{hex}: hexadecimal integer sort. This is for non-locale
hexadecimal sort values.

\item \optfmt{hex-reverse}: as above but reverses the order.

\item \optfmt{octal}: octal integer sort. This is for non-locale
octal sort values.

\item \optfmt{octal-reverse}: as above but reverses the order.

\item \optfmt{binary}: binary integer sort. This is for non-locale binary sort
values.

\item \optfmt{binary-reverse}: as above but reverses the order.

\item \optfmt{float}: single-precision sort. This is for non-locale
decimal sort values.

\item \optfmt{float-reverse}: as above but reverses the order.

\item \optfmt{double}: double-precision sort. This is for non-locale
decimal sort values.

\item \optfmt{double-reverse}: as above but reverses the order.

\item \optfmt{numeric}: locale-sensitive numeric sort. Use
\csopt{numeric-locale} to set the locale.

\item \optfmt{numeric-reverse}: as above but reverses the order.

\item \optfmt{currency}: locale-sensitive currency sort. Use
\csopt{numeric-locale} to set the locale.

\item \optfmt{currency-reverse}: as above but reverses the order.

\item \optfmt{percent}: locale-sensitive percent sort. Use
\csopt{numeric-locale} to set the locale.

\item \optfmt{percent-reverse}: as above but reverses the order.

\item \optfmt{numberformat}: locale-sensitive custom numeric sort. Use
\csopt{numeric-locale} to set the locale and
\csopt{numeric-sort-pattern} to set the number pattern.

\item \optfmt{numberformat-reverse}: as above but reverses the order.

\end{itemize}

In general, it doesn't make much sense to have \hierarchical\ entries
that need to be sorted by a number, but it is possible as long as each
level uses the same type of numbering.

\subsubsection{Date-Time}

The sort methods listed in \tableref{tab:sortoptionsdatetime} 
are for dates and times.  Use \csopt{date-sort-format}
and \csopt{date-sort-locale} to specify the date format and locale.
\begin{itemize}
\item \optfmt{date}: sort dates.

\item \optfmt{date-reverse}: as above but reverses the order.

\item \optfmt{datetime}: sort date and time information.

\item \optfmt{datetime-reverse}: as above but reverses the order.

\item \optfmt{time}: sort times.

\item \optfmt{time-reverse}: as above but reverses the order.

\end{itemize}

If the field you want to sort by contains a date then the
simplest way to sort is to ensure the date is in ISO format
and then just use a letter sort. However it may be that
the date is in the format particular to your locale or you have a
mix of \era{AD} and \era{BC}. In which
case you can use one of the date/time sort options (such as
\csopt[date]{sort} or \csopt[date-reverse]{sort}). The locale
is assumed to be your default locale (as given by the \idx{JVM}) 
but if you are using a different locale this can
be set with \csopt{date-sort-locale}. The pattern is assumed
to be the default for that locale but you can change this with
\csopt{date-sort-format}. If you provide your
own custom pattern you must make sure that it matches the selected
\optfmt{sort} option.

Take care if you switch from using the \idx{JRE} to the \idx{CLDR}
 \idx{localeprovider} as you may find the default pattern changes.

The locale and pattern information is used by \bibgls\ to parse the
field.  If the field value can't be parsed then \bibgls\ will issue
a warning and assume the current date (or time).

The actual sort value that's used by the comparator is numeric.  In
the case of the time-based \csopt[datetime]{sort} and
\csopt[time]{sort} (or their \optfmt{-reverse} versions), this value
is the number of milliseconds since 1st~January, 1970. In the case
of \csopt[date]{sort} (or \csopt[date-reverse]{sort}), this value is
obtained from $a(y\times10000 + m\times100 + d)$ where $y$ is the
year, $m$ is the month number, $d$ is the day of month number, and
$a$ is an integer representation of the era ($-1$ for \era{BC} and
$+1$ for \era{AD}).

Unlike the numeric sort methods (such as \csopt[integer]{sort})
the date-time sort methods set the \field{sort} field to a value
that can be more easily parsed within the document and that should
mostly achieve the same ordering if a letter comparator were to be used
with it (except for \era{BC} dates, where the order needs to be
reversed). This has the by-product of providing a field that
you can access within the document that can be more easily parsed by
\LaTeX.

In general, it doesn't make much sense to have \hierarchical\ entries
that need to be sorted by date, but it is possible as long as each
level uses the same date format.

For example, suppose my \ext{bib} file contains:
\begin{codeenv}
\atentry{entry}\marg{journalentry,
  \field{name}=\marg{10 Jan 2017},
  \field{description}=\marg{an interesting journal entry}
}
\end{codeenv}
The \field{name} field uses an abbreviated UK date format.
If all my other entries also use this format in the \field{name}
then I can sort them chronologically:
\begin{codeenv}
\gls{GlsXtrLoadResources}\oarg{
  \csopt[entries]{src},\comment{data in entries.bib}
  \csopt[date]{sort},
  \csopt[en-GB]{date-sort-locale},
  \csopt[medium]{date-sort-format}
}
\end{codeenv}
(The medium format is actually the default for this locale,
and the locale matches my system locale, so I could omit
both \csopt{date-sort-locale} and \csopt{date-sort-format}.)

If \commonlongarg{verbose} mode is on, the transcript will show
the label, sort value and numeric value for each entry.
In this case, the information is:
\begin{verbatim}
journalentry -> '+1 2017-01-10' [20170110]
\end{verbatim}
The first value is the label (\code{journalentry}), the second
value is assigned to the \field{sort} field
(\code{+1 2017-01-10}) and the number in square brackets 
is the actual numeric value used by the comparator. The signed
number at the start of the sort field \code{+1} is the numeric
representation of the era as used for the $a$ variable in the
computation of the numeric value (as described earlier).

If I change the format to \csopt[short]{date-sort-format},
then the date can't be parsed correctly and \bibgls\ will 
issue the following warning:
\begin{verbatim}
Warning: Can't parse sort value '10 Jan 2017' for 'journalentry'
(pattern: 'dd/MM/yyyy')
\end{verbatim}
This shows the value that \bibgls\ is trying to
parse (\code{10 Jan 2017}) for the entry identified by
the given label (\code{journalentry}). The pattern \bibgls\ 
expects is also given (\code{dd/MM/yyyy}).

\optsection{shuffle}

Automatically sets \csopt[random]{sort} and \csopt{flatten}.
The value \meta{seed} may be omitted. If present, it should
be an integer used as a seed for the random number generator.

\optsection{sort-field}

The \csopt{sort-field} key indicates which field provides the sort
value. The \meta{field} must be a recognised field name or you
may use \csopt[id]{sort-field} to sort according to the label.
The default value is the \field{sort} field (which is typically
inferred rather than explicitly set).

Example:
\begin{codeenv}
\gls{GlsXtrLoadResources}\oarg{
 \csopt[entries-terms]{src},\comment{data in entries-terms.bib}
 \csopt[symbol]{sort-field},\comment{sort by 'symbol' field}
 \csopt[letter-case]{sort}\comment{case-sensitive letter sort}
}
\end{codeenv}
This sorts the entries according to the \field{symbol} field using
a case-sensitive letter comparison.

\begin{important}
In general it's better to use the default \csopt[sort]{sort-field} 
and adjust the fallbacks instead (see
\sectionref{sec:fallbacks}). The \csopt{sort-field} option is
provided if you want to use a specific field regardless of the 
\gls{entrytype}.
\end{important}

If an entry is missing a value for \meta{field}, then the value of
the fallback field will be used instead. If \csopt{missing-sort-fallback} is
set, then that's used as the fallback, otherwise it depends on the
\gls{entrytype}. 
If no fallback field can be found, the entry's label will be used.

For the specific case with the default \csopt[sort]{sort-field}
setting, the fallback for the \field{sort} field is governed not
only by the \gls{entrytype} but also by some associated settings:

\begin{itemize}
\item If the entry's original type (before being aliased with
\csopt{entry-type-aliases}) is identified in
\csopt{custom-sort-fallbacks}, then if the \field{sort} field is missing
the value is obtained from the supplied custom mapping.

\item If the entry is defined using \atentry{entry} (or a dual form
that acts like \atentry{entry}), then if the \field{sort} field is
missing the value is obtained from the field identified by
\csopt{entry-sort-fallback}. If that field is also missing then
that field's fallback is used.

\item For the index \glspl{entrytype} like \atentry{index} or
\atentry{indexplural}, then if the \field{sort} field is missing the
value is obtained from the \field{name} field. If that field is also
missing, then the value is obtained from the particular \gls{entrytype}['s]
fallback for the \field{name} field. (For example, the entry's label
for \atentry{index} or the \field{plural} field for
\atentry{indexplural}.)

\item If the entry is defined with an abbreviation type (for
example, \atentry{abbreviation} or \atentry{acronym}) then if the
\field{sort} field is missing, \bibgls\ will fallback on the field
given by \csopt{abbreviation-sort-fallback}.

\item The symbol-like \glspl{entrytype} fallback on the field given by
\csopt{symbol-sort-fallback} if the \field{sort} field is missing.

\item Entries defined using \atentry{bibtexentry} fallback on the field
given by \csopt{bibtexentry-sort-fallback}, which defaults to the
\field{name} field. Note that this only applies to the \gls{mainentry}.
The \glsdisp{spawnedentry}{spawned} \atentry{contributor} entries behave like
\atentry{index}.

\end{itemize}
Use \csopt{dual-sort-field} when sorting dual entries.

\optsection{missing-sort-fallback}

With \csopt[\meta{sort-field}]{sort-field}, if the value of the field
identified by \meta{sort-field} is missing, then \bibgls\ behaves as
follows:
\begin{enumerate}
\item If \csopt[\meta{fallback-field}]{missing-sort-fallback} is set, then
\bibgls\ will fallback on the value provided by the field
\meta{fallback-field}.  If \meta{fallback-field} is
missing, then \bibgls\ will query the \gls{entrytype}['s] fallback for
\meta{fallback-field} (not for \meta{sort-field}).

The \meta{fallback-field} must be a known field but not an internal
field. It can't be the \field{sort} field. (Take care not to cause
an infinite loop if \csopt{sort-field} has been changed.) Unlike the
other sort fallback options, such as \csopt{entry-sort-fallback},
the \meta{fallback-field} can't be a keyword (to identify the label)
and can't be a composite.

\item If the \gls{entrytype} has a fallback rule for
\meta{sort-field}, then that rule is used (see
\sectionref{sec:fallbacks}). When \csopt[sort]{sort-field} this
means:
\begin{itemize}
\item If the entry's original \gls{entrytype} has been identified in
\csopt{custom-sort-fallbacks}, then \bibgls\ will fallback on the
designated custom setting.

\item If the entry was defined using one of the index types (such as
\atentry{index}), then \bibgls\ will fallback on the \field{name}
field.

\item If the entry was defined using the \atentry{entry} type (or a
dual form that acts like \atentry{entry}), then \bibgls\ will
fallback on the field given by \csopt{entry-sort-fallback}.

\item If the entry was defined using
one of the symbol types (such as \atentry{symbol}), then \bibgls\
will fallback on the field given by \csopt{symbol-sort-fallback}.

\item If the entry was defined using
one of the abbreviation types (such as \atentry{abbreviation}), then \bibgls\
will fallback on the field given by
\csopt{abbreviation-sort-fallback}.

\item If the entry was defined using \atentry{bibtexentry} (but not
the spawned \atentry{contributor} entries), then \bibgls\ will
fallback on the field given by
\csopt{bibtexentry-sort-fallback}.
\end{itemize}
If \meta{sort-field} is not \field{sort}, then there may not be a
fallback, in which case the next condition applies:
\item Otherwise the sort value will be set to the entry label and \bibgls\
will issue a warning.
\end{enumerate}
The default setting is \csopt[\empty]{missing-sort-fallback}, which
means that step~1 above is omitted.

Use \csopt{dual-missing-sort-fallback} when sorting dual entries
separately from primaries, and use
\csopt{secondary-missing-sort-fallback} for \csopt{secondary}
sorting.

\optsection{trim-sort}

If the interpreter is used to determine the sort value, this setting
governs whether or not the interpreter should trim leading and
trailing spaces. The default setting is \csopt[true]{trim-sort}.

This option automatically sets \csopt[\meta{boolean}]{dual-trim-sort}
and \csopt[\meta{boolean}]{secondary-trim-sort}.

\optsection{sort-replace}

This option may be used to perform \gls{regular-expression} substitutions
on the sort value and has the same syntax as \csopt{labelify-replace}. 
The value is required for this key but may be empty, which indicates
that the setting is switched off.

This action is done after the interpreter parses the
sort value (if applicable) and before \csopt{sort-number-pad} (if
applicable). For example, suppose the sort value is:
\begin{codeenv}
\cs{ensuremath}\marg{\ics{approx} 3.14}
\end{codeenv}
then the interpreter will convert this to $\mathtt{\approx}$\texttt{3.14} but:
\begin{codeenv}
\csopt[\marg{\cs{glshex}2248}\marg{}]{sort-replace}
\end{codeenv}
can be used to strip the $\approx$ symbol (\hex{2248}) so that the value can
now be parsed as a number if \csopt[double]{sort} has been used.

Use \csopt{dual-sort-replace} for dual
and \csopt{secondary-sort-replace} for secondary sort methods.

\optsection{sort-rule}

If the \csopt[custom]{sort} option is used, the sort rule must be
provided with \optfmt{sort-rule}. If \csopt{sort} is not set to
\optfmt{custom}, the \optfmt{sort-rule} setting will be ignored.
This setting uses Java's
\href{http://docs.oracle.com/javase/8/docs/api/java/text/RuleBasedCollator.html}{RuleBasedCollator}
class~\cite{rulebasedcollator}, and the rule syntax needs to conform
to that format.

Remember that the options will be expanded as they are written to
the \iext{aux} file, so be careful of any special characters that
occur in the rule. For the special characters \idx{param}
\idx{commentchar} \idx{sbchar} \idx{colsep} \idx{bgroupchar} and
\idx{egroupchar} you can use \ics{cs.hash}, \ics{cs.percent},
\ics{cs.underscore}, \ics{cs.amp}, \ics{cs.openbrace} and
\ics{cs.closebrace}. These will be written to the \ext{aux} file
with the leading backslash, but \bibgls\ will remove it for this
resource option. Remember that the \styfmt{glossaries} package
provides \ics{glsbackslash} and \ics{glstildechar} which can be used
to produce a literal backslash (\idx{backslashchar}) and tilde
(\idx{tildechar}).

You can also
use \stringu\meta{hex} (where \meta{hex} is a hexadecimal
code) to represent a Unicode character. For example:
\begin{codeenv}
\gls{GlsXtrLoadResources}\oarg{
  \csopt[custom]{sort},
  \csopt[< a,A < b,B < c,C < ch,Ch,CH < d,D
   < dd,Dd,DD < e,E < f,F < ff,Ff,FF
   < g,G < ng,Ng,NG < h,H < ij,Ij,IJ
   < i,I < j,J < k,K < l,L < ll,Ll,LL < m,M
   < n,N < o,O < p,P < ph,Ph,PH < q,Q < r,R < rh,Rh,RH
   < s,S < t,T < th,Th,TH < u,U < v,V < w,W < x,X < y,Y < z,Z
   < \stringu00E6,\stringu00C6]{sort-rule}
}
\end{codeenv}
It's best to use \ics{cs.string} rather than \ics{protect} to avoid
unwanted spaces interfering with \meta{hex}. Note that
\sty{glossaries-extra} v1.21+ provides\footnote{The command
definition was moved to \isty{glossaries-extra-bib2gls} from version
1.27 since it's only needed with \bibgls.}\ \ics{glshex}
which just does \stringu\ so you can do
\code{\ics{glshex} 00E6} instead of \code{\stringu00E6}.
This is only one character different, but you can redefine
\ics{glsxtrresourceinit} to locally set \ics{u} to \ics{glshex} while
the protected write is performed. For example:
\begin{codeenv}
\cs{renewcommand}*\marg{\cs{glsxtrresourceinit}}\marg{\cmd{let}\cmd{u}\cs{glshex}}
\end{codeenv}
Then you can just do \code{\csfmt{u}00E6} instead of \code{\stringu00E6}.
Note that \ics{GlsXtrResourceInitEscSequences} performs a similar
assignment, so you can instead do:
\begin{codeenv}
\cs{renewcommand}*\marg{\cs{glsxtrresourceinit}}\marg{\comment{}
  \ics{GlsXtrResourceInitEscSequences}
}
\end{codeenv}

The \isty{glossaries-extra-bib2gls} package (which is automatically loaded by
the \styopt{record} option) provides some commands for common rule blocks 
that may be used in the construction of custom rules. For example:
\begin{codeenv}
\csopt[\ics{glsxtrcontrolrules}
 ;\ics{glsxtrspacerules}
 ;\ics{glsxtrnonprintablerules}
 ;\ics{glsxtrcombiningdiacriticrules}
 ,\ics{glsxtrhyphenrules}
 <\ics{glsxtrgeneralpuncrules}
 <\ics{glsxtrdigitrules}
 <\ics{glsxtrfractionrules}
 <\ics{glsxtrMathItalicGreekIrules}
 <\ics{glsxtrGeneralLatinIVrules}
 <\ics{glsxtrLatinAA}
 <\ics{glsxtrLatinOslash}
]{sort-rule}
\end{codeenv}
This places the Greek maths symbols (such as \ics{alpha}) before the
Latin block. See the \sty{glossaries-extra} documentation for
further details of these commands.

You might find it convenient to provide similar commands in a
package for rules you may often need. For example, suppose I have a
package called, say, \styfmt{mapsymbols} for providing map symbols:
\begin{codeenv}
\cmd{NeedsTeXFormat}\marg{LaTeX2e}
\cmd{ProvidesPackage}\marg{mapsymbols}
\comment{some package or font loading stuff here to provide}
\comment{the appropriate symbols}
\cmd{newcommand}\marg{\cmd{Stadium}}\marg{\ldots}
\cmd{newcommand}\marg{\cmd{Battlefield}}\marg{\ldots}
\cmd{newcommand}\marg{\cmd{Harbour}}\marg{\ldots}
\comment{etc}
\strut
\comment{Provide a rule block:}
\cmd{newcommand}\marg{\cmd{MapSymbolOrder}}\marg{\comment{}
 \cs{glshex} 2694 \comment{crossed-swords 0x2694}
 < \cs{glshex} 2693 \comment{anchor 0x2693}
 < \cs{glshex} 26BD \comment{football 0x26BD}
}
\end{codeenv}
In addition to \filefmt{mapsymbols.sty}, I also need to create
\filefmt{mapsymbols.bib} to provide the appropriate definitions for
\bibgls:
\begin{codeenv}
\atentry{preamble}\marg{\qtdelim{\cs{glsxtrprovidecommand}\marg{\cmd{Harbour}}\marg{\icharhex{2693}}
\cs{glsxtrprovidecommand}\marg{\cmd{Battlefield}}\marg{\charhex{2694}}
\cs{glsxtrprovidecommand}\marg{\cmd{Stadium}}\marg{\charhex{26BD}}}}
\end{codeenv}
The use of \ics{glsxtrprovidecommand} will override any previous
definitions of these commands in \bibgls's interpreter but will act
like \ics{providecommand} within the document, and so won't
interfere with the commands defined in \filefmt{mapsymbols.sty}.
Now I can just do:
\begin{codeenv}
\cmd{usepackage}\marg{mapsymbols}\comment{my custom package}
\cmd{usepackage}[\styopt{record}]\marg{glossaries-extra}
\strut
\gls{GlsXtrLoadResources}\oarg{
  \csopt[mapsymbols,\comment{<--- my custom mapsymbols.bib}
   entries\comment{data in entries.bib}
  ]{src},
  \csopt[custom]{sort},
  \csopt[\cs{glsxtrcontrolrules}
 ;\cs{glsxtrspacerules}
 ;\cs{glsxtrnonprintablerules}
 ;\cs{glsxtrcombiningdiacriticrules}
 ,\cs{glsxtrhyphenrules}
 <\cs{glsxtrgeneralpuncrules}
 <\cs{glsxtrdigitrules}
 <\cs{glsxtrfractionrules}
 <\cmd{MapSymbolOrder} \comment{<--- custom map symbols}
 <\cs{glsxtrMathItalicGreekIrules}
 <\ics{glsxtrGeneralLatinIrules}
 ]{sort-rule}
}
\end{codeenv}

An alternative to providing \filefmt{mapsymbols.bib} is to provide a
custom package just for \bibgls' use. For example,
\filefmt{mapsymbols-bib2gls.sty}:
\begin{codeenv}
\comment{Provided for \bibgls\ only.}
\comment{Use \cmd{usepackage}\marg{mapsymbols} in the document.}
\cmd{NeedsTeXFormat}\marg{LaTeX2e}
\cmd{ProvidesPackage}\marg{mapsymbols-bib2gls}
\cs{glsxtrprovidecommand}\marg{\cmd{Harbour}}\marg{\charhex{2693}}
\cs{glsxtrprovidecommand}\marg{\cmd{Battlefield}}\marg{\charhex{2694}}
\cs{glsxtrprovidecommand}\marg{\cmd{Stadium}}\marg{\charhex{26BD}}
\cmd{endinput}
\end{codeenv}
and instruct \bibgls\ to parse it with 
\code{\longarg{custom-packages} mapsymbols-bib2gls} (and use
\filefmt{mapsymbols.sty} in the document). Remember that \bibgls\
isn't a \TeX\ engine so make sure to only use simple commands in
this file.

\optsection{break-at}

This option automatically implements
\csopt[\meta{option}]{dual-break-at} and
\csopt[\meta{option}]{secondary-break-at}.

The alphabet sort options (\tableref{tab:sortoptionsrule}) typically
list non-letter characters before alphabetical characters and spaces
are quite often in the ignored set. This means that the alphabet
sort options are naturally in a letter order, similar to
\idx!{xindy}['s] \code{ord/letorder} module.  (This isn't the same as
\csopt[letter-nocase]{sort}, which just sorts according to the
Unicode value not according to a particular alphabet.)

In order to replicate \idx!{makeindex} and \idx!{xindy}['s] default word
order, \bibgls\ splits up the sort value at word boundaries and
inserts a marker (identified by \csopt{break-marker}).
For example, if the sort value is \qt{sea lion} then it's actually
converted to \verb"sea|lion|" whereas \qt{sea} becomes \verb"sea|"
and \qt{seal} becomes \verb"seal|". The default marker is \verb"|"
which is commonly placed in collation rules before digits but 
after the ignored characters, such as spaces and hyphens.

Note that this action removes non-letters, so for example,
if the sort value is \verb"# (parameter)" then it will be converted
to \verb"parameter|" (hash, space and parentheses removed).
If you only want to break at spaces (optionally following a comma)
use the following instead:
\begin{codeenv}
 \csopt[none]{break-at},
 \csopt[\marg{,? +}\marg{|}]{sort-replace}
\end{codeenv}

You can change the construction of the break points with
\csopt[\meta{option}]{break-at} where \meta{option} may be one of:
\begin{itemize}
\item \optfmt{word}: break at word boundaries (default).
Note that what constitutes a word varies
according to the locale but usually anything that's not alphanumeric 
will designate a word-boundary. The characters between words are
discarded.
For example, the sort value \qt{Tom, Dick, and Harry} becomes 
\verb"Tom|Dick|and|Harry", which has discarded the comma and space
characters.
\item \optfmt{character}: break after each character.
\item \optfmt{sentence}: break after each sentence.
\item \optfmt{upper-notlower}: break after any \idx{uppercase} character
that's not followed by a \idx{lowercase} character. For example, 
\qt{MathML} becomes \verb"MathM|L|" and \qt{W3C} becomes
\verb"W|3C|".
\item \optfmt{upper-upper}: break after any \idx{uppercase} character
that's followed by an \idx{uppercase} character.
\item \optfmt{upper-notlower-word}: first applies break-points
according to \optfmt{upper-notlower} and then according to
\optfmt{word}.
\item \optfmt{upper-upper-word}: first applies break-points
according to \optfmt{upper-upper} and then according to
\optfmt{word}.
\item \optfmt{none}: don't create break points. Use this option to
emulate \idx!{makeindex} or \idx!{xindy}'s letter ordering, or combine
with \csopt{sort-replace} to insert custom break points.
\end{itemize}

This option is ignored when used with the non-alphabetic
\csopt{sort} options. You can find the break points in the
\field{sort} field for the entry's definition in the \ext{glstex}
file (which is provided for information rather than for use in the
document).  Alternatively, use the \commonlongarg{debug} switch to show
the break points in the transcript. (This will also show the
collation rule.) 

If you want to selectively apply break points only to certain
entries, use \csopt{break-at-match} or \csopt{break-at-not-match}.

\optsection{break-marker}

This option automatically implements the dual and secondary settings
\csopt[\meta{marker}]{dual-break-marker} and
\csopt[\meta{marker}]{secondary-break-marker}.

The break marker can be changed using
\csopt[\meta{marker}]{break-marker}, where \meta{marker} is 
the character to use. For example, \csopt[-]{break-marker} will use a
hyphen. The marker may be empty, which effectively strips the
inter-word punctuation. For example, with
\csopt[\empty]{break-marker}, \qt{Tom, Dick, and Harry} becomes
\code{TomDickandHarry} and \qt{sea lion} simply becomes
\code{sealion}. If \meta{marker} is omitted,
\csopt[\empty]{break-marker} is assumed.

\optsection{break-at-match}

This option automatically implements
\csopt[\meta{option}]{dual-break-at-match} and
\csopt[\meta{option}]{secondary-break-at-match}.

If you have \csopt{break-at} set to create break points (for
example, with \csopt[word]{break-at}) then you can specify which
entries should have break points with this option. The value has the same
syntax as \csopt{match}. If an entry matches the criteria, then 
break points are added, otherwise no break points are added.
For example, to only have break points for entries defined with \atentry{index}
or \atentry{indexplural}:
\begin{codeenv}
\csopt[entrytype=index(plural)?]{break-at-match}
\end{codeenv}
This option has no effect with \csopt[none]{break-at}.

\optsection{break-at-match-op}
This option automatically implements
\csopt[\meta{option}]{dual-break-at-match-op} and
\csopt[\meta{option}]{secondary-break-at-match-op}.

If the value of \csopt{break-at-match} contains more than one 
\meta{key}=\meta{pattern} element, the \csopt{break-at-match-op}
determines whether to apply a logical AND or a logical OR.
The \meta{value} may be either \optfmt{and} or \optfmt{or}.
The default is \csopt[and]{break-at-match-op}.

\optsection{break-at-not-match}

This option automatically implements
\csopt[\meta{option}]{dual-break-at-not-match} and
\csopt[\meta{option}]{secondary-break-at-not-match}.

 For example, to prevent entries defined with
\atentry{symbol} from having break points:
\begin{codeenv}
\csopt[entrytype=symbol]{break-at-not-match}
\end{codeenv}
Like \csopt{break-at-match} but negates the match.
This option has no effect with \csopt[none]{break-at}.

\optsection{sort-number-pad}

This option automatically implements the dual and secondary settings
\csopt[\meta{number}]{dual-sort-number-pad},
\csopt[\meta{number}]{secondary-sort-number-pad}.

If \meta{number} is greater than 1, any integer sub-strings found
in the sort value will be zero-padded up to this value. Since the
\code{-} character is often ignored by rule-based sort methods, 
any signs found will be replaced with the markers given by
\csopt{sort-pad-plus} and \csopt{sort-pad-minus}, which should be
chosen to ensure that negative numbers are ordered before positive
numbers (if this is desired). An unsigned number will have the
\csopt{sort-pad-plus} marker inserted before it. The default value
is \csopt[0]{sort-number-pad}, which doesn't implement any padding.

If you use this with a locale sort method, it's best to also set
\csopt[none]{break-at}, as the default word boundary break points
will likely be confused by a mix of alphanumerics.

\optsection{sort-pad-plus}

This option automatically implements the dual and secondary settings
\csopt[\meta{marker}]{dual-sort-pad-plus},
\csopt[\meta{marker}]{secondary-sort-pad-plus}.

This option only has an effect when used with 
\csopt[\meta{number}]{sort-number-pad} where \meta{number} is
greater than 1. Positive numbers will have their sign replaced with 
\meta{marker}. The default setting is \csopt[>]{sort-pad-plus}.

\optsection{sort-pad-minus}

This option automatically implements the dual and secondary settings
\csopt[\meta{marker}]{dual-sort-pad-minus},
\csopt[\meta{marker}]{secondary-sort-pad-minus}.

This option only has an effect when used with 
\csopt[\meta{number}]{sort-number-pad} where \meta{number} is
greater than 1. Negative numbers will have their sign replaced with 
\meta{marker}. The default setting is \csopt[<]{sort-pad-plus}.

\optsection{identical-sort-action}

This option automatically implements the dual and secondary settings
\csopt[\meta{value}]{dual-identical-sort-action} and
\csopt[\meta{value}]{secondary-identical-sort-action}.

This option determines what the comparator should do if
two entries at the same \hierarchicallevel\ are considered
equal.  The \meta{value} may be one of:
\begin{itemize}
\item\optfmt{none}: don't take any further action if sort values are
identical;
\item\optfmt{def} if sort values are identical, order them according to definition;
\item\optfmt{use}: if sort values are identical, order them
according to use in the document (order determine by a normal record);
\item\optfmt{id}: if sort values are identical, compare
the entry labels;
\item\optfmt{original id}: if sort values are identical, compare the
original unprefixed entry labels (as given in the \ext{bib} file);
\item\meta{field}: if sort values are identical, compare
the values from the given \meta{field}.
\end{itemize}
For the last three cases, a simple case-sensitive string comparison is used. 
If \meta{value} isn't a recognised keyword or valid field an error will occur.  The
default setting is \csopt[id]{identical-sort-action}.  If you're
using one of the sort rules listed in \tableref{tab:sortoptionsrule}
and you also want a locale-sensitive sort used on the fallback, then
you need to use \csopt{sort-suffix} instead.

\bibgls\ allows duplicate sort values, but this can cause a problem
for \hierarchical\ entries where \glspl{parententry} with duplicate sort
fields are clumped together and their \children\ follow. To prevent
this from happening, the \csopt[id]{identical-sort-action}
setting will fallback on comparing the labels.  Since all labels
must be unique, this means comparisons between two different entries
are all either strictly higher or strictly lower.

This action occurs after any suffixes have been appended through
\csopt{sort-suffix}.

\optsection{sort-suffix}

This option automatically implements the dual and secondary settings
\csopt[\meta{value}]{dual-sort-suffix} and
\csopt[\meta{value}]{secondary-sort-suffix}.
The value may be one of:
\begin{itemize}
\item\optfmt{none}: don't append a suffix to any \field{sort} value;
\item\optfmt{non-unique}: append a numeric suffix to non-unique
  \field{sort} values;
\item\meta{field}: append the value of the given field (if set) to
  the \field{sort} field. The given field must be defined (has an
  associated key for use in \gls{newglossaryentry}) but may be unset.
  If the interpreter is on, the field contents will be interpreted.
  If the field is just a label (such as the \field{category} field)
  you may find it simpler to use
  \csopt[\meta{field}]{identical-sort-action} instead.
\end{itemize}
The default setting is \csopt[none]{sort-suffix}.

This option only affects the alphabetic
(\tableref{tab:sortoptionsrule}), letter
(\tableref{tab:sortoptionsletter}) and letter-number
(\tableref{tab:sortoptionsletternumber}) sort rules.  For the other
types of sort methods (not including the no-sort options listed in
\tableref{tab:sortoptionsnosort}) you'll need to use
\csopt{identical-sort-action} to prevent problems occurring with
duplicate sort values.

In the case of \csopt[non-unique]{sort-suffix}, this will only
append a suffix to the duplicate sort values (within the same
\hierarchicallevel). The first sort value to be encountered isn't
given a suffix.

The \csopt[\meta{field}]{sort-suffix} setting will only append a suffix
if that field is set, but (if set) it will apply the suffix to all
\field{sort} values, even those that are unique.

If you use \commonlongarg{verbose}, then \bibgls\ will write information
in the transcript when it appends a suffix to the sort value.  The
message:
\begin{alltt}
Sort value '\meta{sort}' (entry '\meta{id}') not unique for the entry's
\hierarchicallevel.
\end{alltt}
indicates that an entry with the given \meta{sort} value has already
been found within the same \hierarchicallevel\ as the currently
processed entry (whose label is given by \meta{id}). The same
\hierarchicallevel\ in this context means that either both entries
don't have a \parent\ or both entries have the same \parent. (That is,
the entries are considered \glspl{sibling}.)

This message will then be followed by:
\begin{alltt}
Appending suffix '\meta{suffix}' to the sort value '\meta{sort}' 
for entry '\meta{id}'.
\end{alltt}
which indicates that the entry (identified by the label \meta{id})
has been assigned the sort value given by \meta{sort}\meta{suffix}.
If any break markers are applied, this is done after the suffix has
been appended.

For example, suppose in my document I want to write about \appfmt{makeglossaries}
(the application) and \csfmt{makeglossaries} (the command). I might
decide to define semantic commands:
\begin{codeenv}
\cs{newcommand}*\marg{\cmd{application}}[1]\marg{\ics{texttt}\marg{\idx{param}1}}
\cs{newcommand}*\marg{\cmd{command}}[1]\marg{\cs{texttt}\marg{\cs{glsbackslash} \idx{param}1}}
\end{codeenv}
In my \ext{bib} file I might have:
\begin{codeenv}
\atentry{entry}\marg{cs.makeglossaries,
  \field{name}=\marg{\cmd{command}\marg{makeglossaries}},
  \field{category}=\marg{command},
  \field{description}=\marg{opens glossary files}
}
\strut
\atentry{entry}\marg{ap.makeglossaries,
  \field{name}=\marg{\cmd{application}\marg{makeglossaries}},
  \field{category}=\marg{application},
  \field{description}=\marg{Perl script}
}
\end{codeenv}
If \bibgls\ is provided with the definitions of \csfmt{application}
and \csfmt{command} (by interpreting the \atentry{preamble} or a
package provided with \longarg{custom-packages}) then it will
determine that the sort value for \code{cs.makeglossaries} is
\csfmt{makeglossaries} and the sort value for
\code{ap.makeglossaries} is just \code{makeglossaries}. These are 
two distinct sort values from \bibgls's point of view although the
sort rule may consider them identical if the rule ignores the
\idx{escchar} character (such as the locale sort methods), in which case,
\bibgls\ will then act according to \csopt{identical-sort-action}.

If \bibgls\ isn't provided with these custom definitions, then it
will ignore those semantic commands and both entries will end up with
the sort value \code{makeglossaries}. The second instance will be
recognised as a duplicate and the sort value will be converted to
\code{makeglossaries1} (where the automated suffix is \code{1} and
the suffix marker, see below, is the empty string). Whereas with,
say, \csopt[.]{sort-suffix-marker}\ then the sort value would become
\code{makeglossaries.1}.

For comparison, consider the following document:
\begin{codeenv}
\cmd{documentclass}\marg{article}
\strut
\cmd{usepackage}[\styopt[indexgroup]{style}]\marg{glossaries}
\strut
\ics{cs.makeglossaries}
\strut
\cs{newcommand}*\marg{\cmd{application}}[1]\marg{\cs{texttt}\marg{\idx{param}1}}
\cs{newcommand}*\marg{\cmd{command}}[1]\marg{\cs{texttt}\marg{\cs{glsbackslash} \idx{param}1}}
\strut
\gls{newglossaryentry}\marg{cs.makeglossaries}\marg{\comment{}
  \field{name}=\marg{\cmd{command}\marg{makeglossaries}},
  \field{description}=\marg{opens glossary files}}
\strut
\gls{newglossaryentry}\marg{ap.makeglossaries}\marg{\comment{}
  \field{name}=\marg{\cmd{application}\marg{makeglossaries}},
  \field{description}=\marg{Perl script}}
\strut
\cmd{begin}\marg{document}
\cs{gls}\marg{cs.makeglossaries} and \cs{gls}\marg{ap.makeglossaries}.
\ics{printglossaries}
\cmd{end}\marg{document}
\end{codeenv}

This uses \idx{makeindex}, which puts both entries in the
\qt{Symbols} group (since they both start with \idx{escchar} from the
start of \csfmt{command} and \csfmt{application}, respectively).
The ordering is \appfmt{makeglossaries}, \csfmt{makeglossaries}
because \qt{a} (second character of \csfmt{application}) comes
before \qt{c} (second character of \csfmt{command}).

The switch to \idx{xindy} just involves adding the \styopt{xindy}
package option:
\begin{codeenv}
\cs{usepackage}[\styopt{xindy},\styopt[indexgroup]{style}]\marg{glossaries}
\end{codeenv}
This results in a glossary that only contains one entry,
\csfmt{makeglossaries}, because \idx{xindy} merges entries with
duplicate sort values and the sort values end up as duplicates
because \idx{xindy} discards the \csfmt{application} and
\csfmt{command} control sequences. Although \bibgls\ also ignores
unknown control sequences, it doesn't perform this merger. 

If I add: 
\begin{codeenv}
\atentry{preamble}\marg{\qtdelim{\cs{providecommand}*\marg{\cmd{application}}[1]\marg{\cs{texttt}\marg{\idx{param}1}}
\cs{providecommand}\marg{\cmd{command}}[1]\marg{\cs{texttt}\marg{\cs{glsbackslash} \idx{param}1}}}}
\end{codeenv}
to the earlier \ext{bib} file (called, say, \filefmt{entries.bib})
then the document can be altered to use \bibgls:
\begin{codeenv}
\cmd{documentclass}\marg{article}
\strut
\cmd{usepackage}[\styopt{record},\styopt[indexgroup]{style}]\marg{glossaries-extra}
\strut
\gls{GlsXtrLoadResources}\oarg{\csopt[entries.bib]{src},
  \csopt[non-unique]{sort-suffix},
  \csopt[none]{identical-sort-action}
}
\strut
\cmd{begin}\marg{document}
\cs{gls}\marg{cs.makeglossaries} and \cs{gls}\marg{ap.makeglossaries}.
\cs{printunsrtglossaries}
\cmd{end}\marg{document}
\end{codeenv}
This uses the default \csopt[locale]{sort} which considers \idx{escchar}
an ignored (punctuation) character, so both \csfmt{makeglossaries} and
\appfmt{makeglossaries} are listed in the \qt{M} letter group, even
though the interpreter has determined that the sort value for
\code{cs.makeglossaries} is the literal string \csfmt{makeglossaries}. 
Note that in this case \bibgls\ doesn't detect duplicate sort values
since it only uses a simple string comparison to detect duplicates
rather than using the collator.

If I switch to using a letter-based sort rule instead, for example
\csopt[letter-nocase]{sort}, then \csfmt{makeglossaries} will be
listed in the \qt{Symbols} letter group since the leading \idx{escchar}
from the sort value \csfmt{makeglossaries} isn't ignored with this
rule.

Now let's suppose I use \csopt[false]{interpret-preamble} to prevent 
\bibgls\ from interpreting the preamble:
\begin{codeenv}
\gls{GlsXtrLoadResources}\oarg{\csopt[entries.bib]{src},\csopt[false]{interpret-preamble}}
\end{codeenv}
This means that the custom commands won't be recognised and will
therefore be ignored, so both entries will have their sort values reduced to 
\code{makeglosssaries}.

The first entry to be processed is \code{cs.makeglossaries} because
it's the first to be selected. This is assigned the sort value
\code{makeglossaries}. (Note that, unless you use
\csopt[unsrt]{sort}, the initial selection order is based on the record
order. In this example, \code{cs.makeglossaries} has the first
record in the \ext{aux} file.)

The next entry to be processed is \code{ap.makeglossaries}. This
also ends up with the sort value \code{makeglossaries} so \bibgls\
converts this to \code{makeglossaries1} and (with verbose mode on)
the following messages are written to the transcript:
\begin{verbatim}
Sort value 'makeglossaries' (entry 'ap.makeglossaries') not unique
for the entry's hierarchical level.
Appending suffix '1' to the sort value 'makeglossaries' for entry
'ap.makeglossaries'.
\end{verbatim}
Both entries are listed in the \qt{M} letter group in the order
\csfmt{makeglossaries}, \appfmt{makeglossaries}.

If the records are reversed:
\begin{codeenv}
\cs{gls}\marg{ap.makeglossaries} and \cs{gls}\marg{cs.makeglossaries}.
\end{codeenv}
then the sort value for \code{cs.makeglossaries} is now considered
the duplicate and the order is reversed: \appfmt{makeglossaries},
\csfmt{makeglossaries}.

Suppose now I modify the \ext{bib} file so that
\code{ap.makeglossaries} is defined as:
\begin{codeenv}
\atentry{entry}\marg{ap.makeglossaries,
  \field{name}=\marg{\cmd{application}\marg{makeglossaries}},
  \field{category}=\marg{application},
  \field{description}=\marg{Perl script (must be used with \cs{gls}\marg{cs.makeglossaries})}
}
\end{codeenv}
and suppose the document only contains an explicit reference to
\code{ap.makeglossaries}:
\begin{codeenv}
\cmd{begin}\marg{document}
\cs{gls}\marg{ap.makeglossaries}
\cs{printunsrtglossaries}
\cmd{end}\marg{document}
\end{codeenv}
Now \code{ap.makeglossaries} is the first entry to be selected
because entries with records are always selected before any
(unrecorded) dependencies. In this case \code{cs.makeglossaries} is
only selected because it's required by \code{ap.makeglossaries}. Now
\code{ap.makeglossaries} is the first to have its sort value
assigned, and it's \code{cs.makeglossaries} that has the duplicate.
This means that the ordering in the glossary is now:
\appfmt{makeglossaries}, \csfmt{makeglossaries}.

An oddity occurs if the glossary is moved to the start of the
document:
\begin{codeenv}
\cmd{begin}\marg{document}
\cs{printunsrtglossaries}
\cs{gls}\marg{ap.makeglossaries}
\cmd{end}\marg{document}
\end{codeenv}
In this case, the first document build:
\begin{verbatim}
pdflatex myDoc
bibgls --group --verbose myDoc
pdflatex myDoc
\end{verbatim}
leads to the ordering described above: 
\appfmt{makeglossaries}, \csfmt{makeglossaries}.
However, the next document build has a new record for
\code{cs.makeglossaries} occurring in the glossary (within the
description of \code{ap.makeglossaries}) which means it's now the
first entry to be selected so the ordering switches to:
\csfmt{makeglossaries}, \appfmt{makeglossaries}.
In this type of situation you might be better off with the
\csopt[id]{identical-sort-action} option instead.

Remember that you can temporarily switch off the indexing by locally
setting:
\begin{codeenv*}
\ics{GlsXtrSetDefaultGlsOpts}\marg{\glsopt{noindex}}
\end{codeenv*}
Since the glossary preamble is scoped, you can simply do
\begin{codeenv}
\ics{appto}\ics{glossarypreamble}\marg{\cs{GlsXtrSetDefaultGlsOpts}\marg{\glsopt{noindex}}}
\end{codeenv}
to switch off the indexing within the glossary (or use
\ics{apptoglossarypreamble}). Note that this is
different to using:
\begin{codeenv}
\ics{GlsXtrSetDefaultNumberFormat}\marg{\encap{glsignore}}
\end{codeenv}
which creates an \igls{ignoredrecord}. Even though the record is ignored
(and so won't show in the \gls{locationlist}) the record still influences
the selection order and the record count.

\optsection{sort-suffix-marker}

This automatically implements the dual and secondary settings
\csopt[\meta{value}]{dual-sort-suffix-marker} and
\csopt[\meta{value}]{secondary-sort-suffix-marker}.

If a suffix is appended to the sort value (see above) then it will
be separated by the suffix marker, which can be set with
\csopt[\meta{value}]{sort-suffix-marker} where \meta{value} is the
marker. By default the marker is empty. You can use \stringu\meta{hex}
or \cs{glshex}\meta{hex} to indicate Unicode characters outside the
\idx{ASCII} range. If, for some reason, you want to use a special
character, such as \idx{param}, you will need to precede it with
\ics{cs.string} (for example \code{\cs{cs.string}\idx{param}}) or
use the above hexadecimal markup. If you use \ics{cs.hash} it will
be treated as a literal string containing a backslash followed by a
hash character.

\optsection{encapsulate-sort}

This option will encapsulate the sort value (after modifications such 
as \csopt{sort-suffix} and \csopt{sort-number-pad}) with
\csfmt{\meta{csname}}\marg{value}\marg{entry-id} where \meta{value}
is the sort value that would otherwise have been used and
\meta{entry-id} is the entry's label. It will then be interpreted
(if enabled). Note that \meta{value} may already have been
interpreted in a previous step.

\optsection{strength}

This option automatically implements
\csopt[\meta{value}]{dual-strength} and
\csopt[\meta{value}]{secondary-strength}.

The collation strength used by the alphabet sort methods
(\tableref{tab:sortoptionsrule}) can be set to the following values:
\glsdisp{primarycollatorstrength}{\optfmt{primary}} (default),
\glsdisp{secondarycollatorstrength}{\optfmt{secondary}},
\glsdisp{tertiarycollatorstrength}{\optfmt{tertiary}} or
\glsdisp{identicalcollatorstrength}{\optfmt{identical}}. These
indicate the difference between two characters, but the exact
assignment is locale dependent. See the documentation for Java's
\href{http://docs.oracle.com/javase/8/docs/api/java/text/Collator.html}{\code{Collator}
class}~\cite{collator} for further details.

For example, suppose the file \filefmt{entries.bib} contains:
\begin{codeenv}
\atentry{index}\marg{resume}
\atentry{index}\marg{RESUME}
\atentry{index}\marg{resumee, \field{name}=\marg{r\ics{acute}esum\cs{acute}e}}
\atentry{index}\marg{rat}
\atentry{index}\marg{rot}
\atentry{index}\marg{aardvark}
\atentry{index}\marg{zoo}
\end{codeenv}
and the document contains:
\begin{codeenv}
\cmd{documentclass}\marg{article}
\strut
\cmd{usepackage}[\styopt{record}]\marg{glossaries-extra}
\strut
\gls{GlsXtrLoadResources}\oarg{\csopt[en]{sort},\csopt[entries]{src}}
\strut
\cmd{begin}\marg{document}
\cs{gls}\marg{resumee}, \cs{gls}\marg{resume}, \cs{gls}\marg{RESUME},
\cs{gls}\marg{aardvark}, \cs{gls}\marg{rat}, \cs{gls}\marg{rot}, \cs{gls}\marg{zoo}.
\strut
\cs{printunsrtglossaries}
\cmd{end}\marg{document}
\end{codeenv}
then this uses the default \csopt[primary]{strength}, so the entries
are listed as aardvark, rat, r\'esum\'e, resume, RESUME, rot, zoo.

If the strength is changed to \optfmt{secondary}:
\begin{codeenv}
\gls{GlsXtrLoadResources}\oarg{\csopt[en]{sort},\csopt[entries]{src},\csopt[secondary]{strength}}
\end{codeenv}
then the entries are listed as aardvark, rat, resume, RESUME,
r\'esum\'e, rot, zoo.

If the strength is changed to \optfmt{tertiary} or
\optfmt{identical}, there's no difference from
\csopt[secondary]{strength} for this particular example.

This option is ignored by non-alphabet sorts (such as letter or numeric).

\optsection{decomposition}

This option automatically implements the dual and secondary settings
\csopt[\meta{value}]{dual-decomposition} and
\csopt[\meta{value}]{secondary-decomposition}.

The collation decomposition used by alphabet sort methods
(\tableref{tab:sortoptionsrule}) can be set
to the following values: \optfmt{canonical} (default),
\optfmt{full} or \optfmt{none}. This determines how Unicode composed
characters are handled. The fastest mode is \optfmt{none} but is
only appropriate for languages without accents. The slowest mode is
\optfmt{full} but is the most complete for languages with \idx{non-ASCII}
characters. See the documentation for Java's
\href{http://docs.oracle.com/javase/8/docs/api/java/text/Collator.html}{\code{Collator}
class}~\cite{collator} for further details.
This option is ignored by non-alphabet sorts (such as letter or numeric).

\optsection{letter-number-rule}

This automatically implements the dual and secondary settings
\csopt[\meta{value}]{dual-letter-number-rule}
and
\csopt[\meta{value}]{secondary-letter-number-rule}.

If you use one of the letter-number sort methods
(\tableref{tab:sortoptionsletternumber}), then you can determine the
comparison between a number and letter. The \meta{value} may be
one of:
\begin{itemize}
\item\optfmt{before letter}: numbers are considered less than any
letter.

\item\optfmt{after letter}: numbers are considered greater than any
letter.

\item\optfmt{between}: (default) numbers come between letter cases.
With the \optfmt{letternumber\dhyphen case} sort option, this will put
numbers after \idx{uppercase} and before \idx{lowercase}.  This
setting doesn't make much sense with the
\optfmt{letternumber\dhyphen nocase} option but, if used, this will put
numbers before letters. The \optfmt{letternumber\dhyphen upperlower} and
\optfmt{letternumber\dhyphen lowerupper} options are more complicated. See 
\sectionref{sec:letternumber} for more detail.

\item\optfmt{first}: numbers are considered less than all
characters (including punctuation and spaces).

\item\optfmt{last}: numbers are considered greater than all
characters (including punctuation and spaces).
\end{itemize}
Note that the reverse sort methods will invert this setting.
Remember also that the case-insensitive letter-number sort methods always
first convert the \field{sort} field to \idx{lowercase}, which means that
if you use one of them then there won't be any \idx{uppercase}
characters. 

Use \csopt{letter-number-punc-rule} to determine the relative
position of white space and punctuation.

\optsection{letter-number-punc-rule}

This automatically implements the dual and secondary 
\csopt[\meta{value}]{dual-letter-number-punc-rule}
and
\csopt[\meta{value}]{secondary-letter-number-punc-rule}.

If you use one of the letter-number sort methods
(\tableref{tab:sortoptionsletternumber}), then you can determine the
order of white space and punctuation. In this context, punctuation
means any character that's not considered a letter, a number
or white space. This means that characters such as combining marks
are considered punctuation.

The \meta{value} may be one of the following:
\begin{itemize}
\item\optfmt{punc-space-first}: punctuation comes first, followed by
white space (then letters and optionally numbers according to the letter-number
rule);
\item\optfmt{punc-space-last}: punctuation followed by white space
come last (after letters and optionally numbers according to the letter-number
rule);
\item\optfmt{space-punc-first}: white space comes first, followed by
punctuation (then letters and optionally numbers according to the letter-number
rule);
\item\optfmt{space-punc-last}: white space followed by punctuation
come last (after letters and optionally numbers according to the letter-number
rule); 
\item\optfmt{space-first-punc-last}: white space comes first
(followed by letters and optionally numbers according to the letter-number
rule) and punctuation comes last; 
\item\optfmt{punc-first-space-last}: punctuation comes first
(followed by letters and optionally numbers according to the letter-number
rule) and white space comes last;
\item\optfmt{punc-first-space-zero}: punctuation comes first
(although numbers may come before)
and white space is replaced by the digit~\code{0} (\hex{30});
\item\optfmt{punc-last-space-zero}: punctuation comes last
(although numbers may come after)
and white space is replaced by the digit~\code{0} (\hex{30}).
\item\optfmt{punc-first-space-zero-match-next}: punctuation comes first
(although numbers may come before)
and white space is replaced by the appropriate zero character (see below);
\item\optfmt{punc-last-space-zero-match-next}: punctuation comes last
(although numbers may come after)
and white space is replaced by the appropriate zero character (see below).
\end{itemize}
Remember that the reverse sort methods will invert order governed by this setting.

For the \optfmt{space-zero-match-next} settings, the sort value will
have all spaces replaced with a digit that represents zero. If the
space isn't followed by a digit, the basic Latin \code{0} (\hex{30})
will be used, otherwise \bibgls\ will try to match the zero with the
following digit group. For example, if the space is followed by
\textsuperscript{1} (\hex{B9}) the space will be replaced by
\textsuperscript{0} (\hex{2070}),  resulting in the sub-string
\code{\textsuperscript{01}} (\hex{B9} \hex{2070}). 

If just the \optfmt{space-zero}
(without the \optfmt{-match-next}) is used then the space will just
be replaced with \code{0} resulting in the sub-string
\code{0\textsuperscript{1}} (\hex{30} \hex{2070}). In this case, the
\code{0} will be distinct from \textsuperscript{1} (rather than
being considered a leading zero). However, for 
other numbering systems the \code{0} will be treated as a leading
zero. For example, if the space is followed by the Devanagari digit
one (\hex{0967}) then the sub-string will be \hex{30} \hex{0967} but here the
mixture is allowed to form a number (with a leading zero) as both
characters belong to the Unicode category \idx{numberdecimaldigit}.

This means that the \optfmt{-match-next} settings are only really
needed if the sort string contains the superscript or subscript
digits that don't belong to the \qt{Number, Decimal Digit} category.
The plain \optfmt{space-zero} alternatives are more efficient as
they just perform a simple substitution.

The \texparserlib\ used by \bibgls\ recognises the standard
\LaTeX\ text-mode commands \ics{textsuperscript}\margm{text} and 
\ics{textsubscript}\margm{text}
and will use the Unicode superscript or subscript characters if they cover every
character in \meta{text}, otherwise HTML markup is used, but that's
then stripped by \bibgls. This means that:
\begin{codeenv}
C\cs{textsubscript}\marg{10}H\cs{textsubscript}\marg{10}O\cs{textsubscript}\marg{4}
\end{codeenv}
will be converted to:
\code{C\textsubscript{10}H\textsubscript{10}O\textsubscript{4}} but:
\begin{codeenv}
X\cs{textsubscript}\marg{1, 2}
\end{codeenv}
will be converted to:
\begin{verbatim}
X<sub>1, 2</sub>
\end{verbatim}
which ends up as \code{X1, 2}.


Note that \csopt[first]{letter-number-rule} and
\csopt[last]{letter-number-rule} overrides this option when
comparing a number with white space or punctuation.

\optsection{numeric-sort-pattern}

If you use the custom \csopt[numberformat]{sort} or
\csopt[numberformat-reverse]{sort}, you need to 
specify the format pattern with this option where
\meta{value} is a pattern recognised by Java's
\href{http://docs.oracle.com/javase/8/docs/api/java/text/DecimalFormat.html}{\code{java.text.DecimalFormat}}
class~\cite{decimalformat}. You can use \stringu\meta{hex} or
\cs{glshex}\meta{hex} to indicate Unicode characters by their
hexadecimal code. You can also use \ics{cs.hash}, \ics{cs.percent},
\ics{cs.underscore}, \ics{cs.amp}, \ics{cs.openbrace} and
\ics{cs.closebrace} to indicate \idx{hashchar}, \idx{percentchar},
\idx{underscorechar}, \idx{ampchar}, \idx{openbracechar} and
\idx{closebracechar}. 

Where the dual or secondary sort uses \optfmt{numberformat}
or \optfmt{numberformat-reverse}, use \csopt{dual-numeric-sort-pattern} for 
\csopt{dual-sort} and \csopt{secondary-numeric-sort-pattern} 
for \csopt{secondary}.

\optsection{numeric-locale}

If you use any of the locale-sensitive numeric sort methods
described in \sectionref{sec:numerical},
such as \csopt[numeric]{sort}, use this option to set
the locale if the default \gls{resource-locale} isn't appropriate. The value may be:
\begin{itemize}
\item\optfmt{resource}: use the default \gls{resource-locale}, if
set, otherwise assume \optfmt{doc};
\item\optfmt{doc}: use the \gls{document-locale} or, if not set, assume
\csopt[locale]{numeric-locale};
\item\optfmt{locale}: use the \gls{Java-locale} (which is usually
the operating system's locale);
\item\meta{lang-tag}: set to the locale identified by the given
a valid language tag \meta{lang-tag}.
\end{itemize}
Use \csopt{dual-numeric-locale} for 
\csopt{dual-sort} and \csopt{secondary-numeric-locale} 
for \csopt{secondary}.

\begin{important}
If you use the \csopt{locale} resource option with \csopt[resource]{numeric-locale},
then the \csopt{locale} option must be come before
\csopt{numeric-locale}.
\end{important}

\optsection{date-sort-locale}

If you use a date/time sort method (\tableref{tab:sortoptionsdatetime}), 
then you can set the locale used by Java's date-time parser.
The default setting is \csopt[resource]{date-sort-locale}.

The value may be \optfmt{resource} (use the \gls{resource-locale}),
\optfmt{doc} (use the \gls{document-locale}),
\optfmt{locale} (use the \gls{Java-locale}),
or a valid language tag \meta{lang-tag} identifying the locale.

Use \csopt{dual-date-sort-locale} and \csopt{secondary-date-sort-locale}
for the dual and secondary.

\begin{important}
If you use the \csopt{locale} resource option with \csopt[resource]{date-sort-locale},
then the \csopt{locale} option must be come before
\csopt{date-sort-locale}.
\end{important}

\optsection{date-sort-format}

If you use a date/time sort method (\tableref{tab:sortoptionsdatetime}), 
then you can set the format used by Java's date-time parser. 
If omitted, \csopt[default]{date-sort-format} is assumed.
The \meta{value} may be one of:
\begin{itemize}
\item\optfmt{default}: use the locale's default format.
\item\optfmt{short}: use the locale's short format.
\item\optfmt{medium}: use the locale's medium format.
\item\optfmt{long}: use the locale's long format.
\item\optfmt{full}: use the locale's full format.
\item\meta{pattern}: provide a custom pattern.
This should match the specifications for Java's
\href{http://docs.oracle.com/javase/8/docs/api/java/text/SimpleDateFormat.html}{\code{SimpleDateFormat}}
class~\cite{simpledateformat}. You may use \stringu\meta{hex} or
\code{\cs{glshex} \meta{hex}} to indicate Unicode characters
or \ics{cs.hash}, \ics{cs.percent}, \ics{cs.underscore}, \ics{cs.amp},
\ics{cs.openbrace} and \ics{cs.closebrace} to indicate
\idx{hashchar}, \idx{percentchar}, \idx{underscorechar},
\idx{ampchar}, \idx{openbracechar} and~\idx{closebracechar}.
\end{itemize}

With the custom setting, if the pattern only contains date (but not
time) information, then it must be used with \csopt[date]{sort}
or \csopt[date-reverse]{sort}. If the pattern only contains time
(but not date) information, then it must be used with
\csopt[time]{sort} or \csopt[time\dhyphen reverse]{sort}. If the pattern
contains date and time information, then it must be used with
\csopt[datetime]{sort} or \csopt[datetime-reverse]{sort}.

For example, suppose each entry provides information about a person
and the \field{user1} field is used to store their date of birth:
\begin{codeenv}
\atentry{entry}\marg{caesar,
  \field{name}=\marg{Gaius Julius Caesar},
  \field{first}=\marg{Julius Caesar},
  \field{text}=\marg{Caesar},
  \field{description}=\marg{Roman politician and general},
  \field{user1}=\marg{13 July 100 BC}
}
\strut
\atentry{entry}\marg{wellington,
  \field{name}=\marg{Arthur Wellesley, 1st Duke of Wellington},
  \field{first}=\marg{Arthur Wellesley (Duke of Wellington)},
  \field{text}=\marg{Wellington},
  \field{description}=\marg{Anglo-Irish soldier and statesman},
  \field{user1}=\marg{1 May 1769 AD}
}
\end{codeenv}
Then the entries can be sorted by date of birth using:
\begin{codeenv}
\gls{GlsXtrLoadResources}\oarg{
  \csopt[entries]{src},\comment{data in entries.bib}
  \csopt[user1]{sort-field},
  \csopt[date]{sort},
  \csopt[d MMM y G]{date-sort-format}
}
\end{codeenv}

The \code{G} (era) date pattern specifier expects a string, such as
\qt{AD}. It will match \idx{lowercase} forms, such as \qt{ad}, so if you
have \code{\cs{textsc}\marg{ad}} the interpreter will convert this to
\code{ad} (stripping the text-block command). However, in general
it's best to supply a semantic command that ensures that the
interpreted result matches the required format.

For example, if \csfmt{era} is provided with:
\begin{codeenv}
\atentry{preamble}\marg{\qtdelim{\cs{providecommand}\marg{\cmd{era}}[1]\marg{\cs{textsc}\marg{\cs{MakeLowercase}\marg{\idx{param}1}}}}}
\end{codeenv}
If the definition is hidden from the interpreter
(\csopt[false]{interpret-preamble}) and the field
value contains \verb|\era{AD}| then the custom command will simply be stripped
leaving \code{AD} which can be matched by \code{G}. 

If the definition is picked up by the interpreter then the field
value will contain \code{ad} (from \ics{MakeLowercase}) but this can
be matched by \code{G}, so it isn't a problem.
However, if the definition of \csfmt{era} is changed so that the era label
supplied in the argument is converted to something that doesn't
match \code{G} then the definition should be hidden from the
interpreter.

Here's a complete document that changes the \field{group} fields to use
the year and era:
\begin{codeenv}
\cmd{documentclass}\marg{article}
\strut
\cmd{usepackage}[\styopt{record},\styopt[indexgroup]{style}]\marg{glossaries-extra}
\strut
\cs{newcommand}\marg{\gls{bibglsdategroup}}[7]\marg{\idx{param}1\idx{param}4\idx{param}7}
\cs{newcommand}\marg{\gls{bibglsdategrouptitle}}[7]\marg{\ics{number}\idx{param}1\cs{cs.space}\idx{param}4}
\strut
\gls{GlsXtrLoadResources}\oarg{
  \csopt[entries]{src},
  \csopt[user1]{sort-field},
  \csopt[date]{sort},
  \csopt[d MMM y G]{date-sort-format},
  \csopt[all]{selection}
}
\strut
\cmd{begin}\marg{document}
\cs{printunsrtglossaries}
\cmd{end}\marg{document}
\end{codeenv}
(The use of \ics{number} strips the leading zero from the year.)

\optsection{group-formation}

If the \field{group} field hasn't been set in
the \ext{bib} file or through options like \csopt{group}, then it is 
assigned according to this option's setting during sorting if
\longarg{group} has been used. Permitted values:
\begin{itemize}
\item\optfmt{default}: the group is assigned according to 
the sort method's default group formation. This is the default
setting.

\item\optfmt{codepoint}: the group is set to
\format{bibglsunicodegroup}, where the first argument is the first
significant character (converted to \idx{lowercase} and decomposed,
if applicable) of the sort value.

\item\optfmt{unicode category}: the group is set to
\format{bibglsunicodegroup}, where the first argument is the label
identifying the Unicode category of the first significant character
of the sort value. For example, the label \code{Ll} signifies a
\idx!{lowercase} letter and \code{Lu} signifies an \idx!{uppercase}
letter.

\item\optfmt{unicode script}: the group is set to
\format{bibglsunicodegroup}, where the first argument is the
label identifying the Unicode script of the first significant
character of the sort value. For example, the label \code{LATIN}
indicates Latin, \code{GREEK} indicates Greek and \code{COMMON}
indicates common characters (such as mathematical Greek characters
that are often used with non-Greek scripts).

\item\optfmt{unicode category and script}: the group is set to
\format{bibglsunicodegroup}, where the first argument is the
label corresponding to the Unicode category and script of the first
significant character of the sort value. For example, the label
\code{Ll.LATIN} indicates a \idx!{lowercase} Latin letter.
\end{itemize}
This option has no effect with \longarg{no-group} or if no sorting
is applied. Use \csopt{secondary-group-formation} for secondary
sorting and \csopt{dual-group-formation} for dual entries.

\begin{important}
Settings other than the default can cause the groups to become
fragmented, so care is needed if you use this option.
See also \sectionref{sec:logicaldivisions}.
\end{important}


\section{Secondary Glossary}
\label{sec:secondaryopts}

The \gls{secondaryglossary} may only be used with \csopt[define]{action}
(within the same resource set) since it's incompatible with the copy actions.
You may use \csopt{secondary} in the first resource set and a copy
action in a subsequent resource set.

\optsection{secondary}

It may be that you want to display a glossary twice but with a
different order. For example, the first time alphabetically and the
second time by category.  One way to do this is to have two
\gls{GlsXtrLoadResources} that both load the same \ext{bib} file with
different \csopt{label-prefix} and \csopt{sort} settings, but this
is only possible with \csopt[all]{selection} or by ensuring you
reference each entry with both label prefixes. Another method is
to use \csopt[copy]{action} but this requires a second resource
command with the same selection criteria.

A simpler method is to use a single \gls{GlsXtrLoadResources}
with the \csopt{secondary} option. The value
(which must be supplied) should be in the format:
\begin{definition}
\meta{sort}:\meta{field}:\meta{type}
\end{definition}
or
\begin{definition}
\meta{sort}:\meta{type}
\end{definition}
If the \meta{field} is omitted, the value of \csopt{sort-field} is
used. Remember that when the \glspl{primaryentry} are sorted, the
\field{sort} field will be set, which means that the \field{sort} fallback field
(see \sectionref{sec:fallbacks}) won't be used in the
\glsdisp{secondaryglossary}{secondary} sort. In
general it's best to supply the field unless one type is sorted and the
other isn't. (The actual sort value obtained by the
\glsdisp{secondaryglossary}{secondary} sort
will be saved in the \field{secondarysort} field in case you require it.)

The value of \meta{sort} is as for \csopt{sort}, but note
that in this case the sort value \optfmt{unsrt} or \optfmt{none}
means to use the same ordering as the \glspl{primaryentry}. For
example, with \csopt[de-CH-1996]{sort},
\csopt[none:copies]{secondary} the \code{copies} list will be
ordered according to \code{de-CH-1996} and not according to the
order in which they were read when the \ext{bib} file or files were
parsed.  If \meta{sort} is \optfmt{custom}, then the rule should be
provided with \csopt{secondary-sort-rule}.

This option will copy all the selected entries into the glossary labelled
\meta{type} sorted according to \meta{sort} (using \meta{field} as
the sort value). Note that this \emph{just copies the entry's label}
to the \gls{secondaryglossary} list rather than creating a duplicate entry,
which saves resources but it means that all the fields will be
identical. If you want groups in your glossary, the group
information for the \glspl{secondaryglossary} will be stored in the
internal \field{secondarygroup} field. The \field{group} field will
contain the group for the \gls{primaryglossary}.

In order to switch fields in \ics{printunsrtglossary}, you need at
least v1.21 of \sty{glossaries-extra} which provides
\ics{glsxtrgroupfield} to keep track of the appropriate field label.
If this command is defined, the preamble for the \gls{secondaryglossary} 
will be adjusted to locally change the field to
\field{secondarygroup}. With older versions, the group information
in the \gls{secondaryglossary} will be the same as for the 
\gls{primaryglossary}.

If the glossary \meta{type} doesn't exist, it will be
defined with \ics{provideignoredglossary*}\margm{type} 
even if \longarg{no-provide-glossaries} is set.
Note that if the glossary already exists and contains entries,
the existing entries aren't re-ordered. The new entries are
simply appended to the list.

For example, suppose the \ext{bib} file contains entries like:
\begin{codeenv}
\atentry{entry}\marg{quartz,
  \field{name}=\marg{quartz},
  \field{description}=\marg{hard mineral consisting of silica},
  \field{category}=\marg{mineral}
}
\strut
\atentry{entry}\marg{cabbage,
  \field{name}=\marg{cabbage},
  \field{description}=\marg{vegetable with thick green or purple leaves},
  \field{category}=\marg{vegetable}
}
\strut
\atentry{entry}\marg{waterfowl,
  \field{name}=\marg{waterfowl},
  \field{description}=\marg{any bird that lives in or about water},
  \field{category}=\marg{animal}
}
\end{codeenv}
and the document preamble contains:
\begin{codeenv}
\gls{GlsXtrLoadResources}\oarg{\csopt[entries]{src},\csopt[en-GB]{sort},
  \csopt[en-GB:category:topic]{secondary}
}
\end{codeenv}
This sorts the \glspl{primaryentry} according to the default
\csopt{sort-field} and then sorts the entries according
to the \field{category} field and copies this list to
the \code{topic} glossary (which will be provided if not defined.)

The \glsdisp{secondaryglossary}{secondary list} can be displayed
with the hypertargets switched off to prevent duplicates. The
cross-references will link to the \glsdisp{primaryglossary}{original glossary}.

For example:
\begin{codeenv*}
\ics{printunsrtglossary}\oarg{\printglossopt[Summary (alphabetical)]{title}}
\ics{printunsrtglossary}\oarg{\printglossopt[Summary (by topic)]{title},\printglossopt[false]{target}}
\end{codeenv*}

The alternative (or if more than two lists are required) is to
reload the same \ext{bib} file with different label prefixes.
For example, if the entries are stored in \filefmt{entries.bib}:
\begin{codeenv}
\cs{newglossary*}\marg{nosort}\marg{Symbols (Unsorted)}
\cs{newglossary*}\marg{byname}\marg{Symbols (Letter Order)}
\cs{newglossary*}\marg{bydesc}\marg{Symbols (Ordered by Description)}
\cs{newglossary*}\marg{byid}\marg{Symbols (Ordered by Label)}
\strut
\gls{GlsXtrLoadResources}\oarg{
  \csopt[entries]{src},\comment{entries.bib}
  \csopt[unsrt]{sort},
  \csopt[nosort]{type}
}
\strut
\gls{GlsXtrLoadResources}\oarg{
  \csopt[entries]{src},\comment{entries.bib}
  \csopt[letter-case]{sort},
  \csopt[byname]{type},
  \csopt[byname.]{label-prefix}
}
\strut
\gls{GlsXtrLoadResources}\oarg{
  \csopt[entries]{src},\comment{entries.bib}
  \csopt[locale]{sort},
  \csopt[description]{sort-field},
  \csopt[bydesc]{type},
  \csopt[bydesc.]{label-prefix}
}
\strut
\gls{GlsXtrLoadResources}\oarg{
  \csopt[entries]{src},\comment{entries.bib}
  \csopt[letter]{sort},
  \csopt[id]{sort-field},
  \csopt[byid]{type},
  \csopt[byid.]{label-prefix}
}
\end{codeenv}

\optsection{secondary-match}

Similar to \csopt{match} but determines whether or not to include a
\gls{primaryentry} in the \glsdisp{secondaryglossary}{secondary
list}. The syntax is the same but this option is governed by
\csopt{secondary-match-op} and \csopt{secondary-match-action}. Note
that if an entry hasn't been selected for the
\glsdisp{primaryglossary}{primary list} then it won't be added to
the \glsdisp{secondaryglossary}{secondary list}, regardless of this
setting.

\optsection{secondary-not-match}

Similar to \csopt{not-match} but determines whether or not to
include a \gls{primaryentry} in the
\glsdisp{secondaryglossary}{secondary list}. The syntax is the same
but this option is governed by \csopt{secondary-match-op} and
\csopt{secondary-match-action}. Note that if an entry hasn't been
selected for the \glsdisp{primaryglossary}{primary list} then it
won't be added to the \glsdisp{secondaryglossary}{secondary list},
regardless of this setting.

\optsection{secondary-match-op}

As \csopt{match-op} but for the \glsdisp{secondaryglossary}{secondary list} selection.

\optsection{secondary-match-action}

As \csopt{match-action} but for the \glsdisp{secondaryglossary}{secondary list} selection.

\optsection{secondary-missing-sort-fallback}

As \csopt{missing-sort-fallback} but for
\glsdisp{secondaryglossary}{secondary} sorting.

\optsection{secondary-trim-sort}

As \csopt{trim-sort} but for \glsdisp{secondaryglossary}{secondary} sorting.

\optsection{secondary-sort-replace}

As \csopt{sort-replace} but for \glsdisp{secondaryglossary}{secondary} sorting.

\optsection{secondary-sort-rule}

As \csopt{sort-rule} but for \glsdisp{secondaryglossary}{secondary} custom sorting.

\optsection{secondary-break-at}

As \csopt{break-at} but for \glsdisp{secondaryglossary}{secondary} entries.

\optsection{secondary-break-marker}

As \csopt{break-marker} but for \glsdisp{secondaryglossary}{secondary} entries.

\optsection[\subsubsection]{secondary-break-at-match}

As \csopt{break-at-match} but for \glsdisp{secondaryglossary}{secondary} entries.

\optsection[\subsubsection]{secondary-break-at-match-op}

As \csopt{break-at-match-op} but for \glsdisp{secondaryglossary}{secondary} entries.

\optsection[\subsubsection]{secondary-break-at-not-match}

As \csopt{break-at-not-match} but for \glsdisp{secondaryglossary}{secondary} entries.

\optsection{secondary-sort-number-pad}

As \csopt{sort-number-pad} but for \glsdisp{secondaryglossary}{secondary} entries.

\optsection{secondary-sort-pad-plus}

As \csopt{sort-pad-plus} but for \glsdisp{secondaryglossary}{secondary} entries.

\optsection{secondary-sort-pad-minus}

As \csopt{sort-pad-minus} but for \glsdisp{secondaryglossary}{secondary} entries.

\optsection{secondary-identical-sort-action}

As \csopt{identical-sort-action} but for \glsdisp{secondaryglossary}{secondary} entries.

\optsection{secondary-sort-suffix}

As \csopt{sort-suffix} but for \glsdisp{secondaryglossary}{secondary} entries.

\optsection{secondary-sort-suffix-marker}

As \csopt{sort-suffix-marker} but for \glsdisp{secondaryglossary}{secondary} entries.

\optsection{secondary-strength}

As \csopt{strength} but for \glsdisp{secondaryglossary}{secondary} entries.

\optsection{secondary-decomposition}

As \csopt{decomposition} but for \glsdisp{secondaryglossary}{secondary} entries.

\optsection{secondary-letter-number-rule}

As \csopt{letter-number-rule} but for \glsdisp{secondaryglossary}{secondary} letter-number sorting.

\optsection{secondary-letter-number-punc-rule}

As \csopt{letter-number-punc-rule} but for \glsdisp{secondaryglossary}{secondary} letter-number sorting.

\optsection{secondary-numeric-sort-pattern}

As \csopt{numeric-sort-pattern} but for \glsdisp{secondaryglossary}{secondary}
locale-sensitive numeric sorting.

\optsection{secondary-numeric-locale}

As \csopt{numeric-locale} but for \glsdisp{secondaryglossary}{secondary}
locale-sensitive numeric sorting.

\optsection{secondary-date-sort-locale}

As \csopt{date-sort-locale} but for \glsdisp{secondaryglossary}{secondary} date-time sorting.

\optsection{secondary-date-sort-format}

As \csopt{date-sort-format} but for \glsdisp{secondaryglossary}{secondary} date-time sorting.

\optsection{secondary-group-formation}

As \csopt{group-formation} but for \glsdisp{secondaryglossary}{secondary} sorting.

\section{Dual Entries}
\label{sec:dualopts}

\subsection{General Dual Settings}
\label{sec:dualoptsgeneral}

\optsection[\subsubsection]{dual-prefix}

This option indicates the prefix to use for the \glspl{dualentry}. The
default value is \idprefix{dual} (including the terminating period). 
Any references to dual entries within the \ext{bib} file should use
the prefix \idprefix{dual} which will be replaced by \meta{value}
when the \ext{bib} file is parsed.

As from version 1.8, the dual label prefix is identified
in the \ext{glstex} file with:
\begin{codeenv}
\format{bibglsdualprefixlabel}
\end{codeenv}

\optsection[\subsubsection]{primary-dual-dependency}

This is a boolean setting that determines whether or not \primary\ and
\dual\ entries should be considered mutual dependencies. The default value is
\csopt[true]{primary-dual-dependency}, which means that if a
\primary\ has records then the \dual\ is added as a dependency and vice versa.
The setting \csopt[false]{primary-dual-dependency} can't be used
with \csopt[none]{dual-sort} or \csopt[use]{dual-sort} (but may be
used with \csopt[combine]{dual-sort} and \csopt[none]{sort} or
\csopt[use]{sort}).

\optsection[\subsubsection]{combine-dual-locations}

This setting allows the \glspl{locationlist} for each \gls{primaryentry}
to be merged with that of the corresponding \gls{dualentry}.
The \meta{value} may be one of:
\begin{itemize}
\item\optfmt{false} This is the default setting. The
\glspl{locationlist} aren't combined.

\item\optfmt{both} Both the \primary\ and \dual\ are given the combined
\gls{locationlist}.

\item\optfmt{dual} Only the \dual\ is given the combined
\gls{locationlist}. The \glsdisp{primaryentry}{primary's} location list is emptied.

\item\optfmt{primary} Only the \primary\ is given the combined
\gls{locationlist}. The \glsdisp{dualentry}{dual's}
\gls{locationlist} is emptied.

\item\optfmt{dual retain principal} Like \optfmt{dual} but any
\glspl{principallocation} for \glspl{primaryentry} will have a copy left in the
\gls{primaryentry}['s] \igls{locationlist}.
\item\optfmt{primary retain principal} Like \optfmt{primary} but any
\glspl{principallocation} for \glspl{dualentry} will have a copy left in the
\gls{dualentry}['s] \igls{locationlist}.
\end{itemize}

For example, suppose the file \filefmt{entries.bib} contains:
\begin{codeenv}
\atentry{dualindexentry}\marg{array,
  \field{description}=\marg{ordered list of values}
}
\strut
\atentry{dualindexentry}\marg{vector,
  \field{name}=\marg{vector},
  \field{description}=\marg{column or row of values}
}
\strut
\atentry{dualindexentry}\marg{set,
  \field{description}=\marg{collection of values}
}
\strut
\atentry{dualindexentry}\marg{matrix,
  \field{plural}=\marg{matrices},
  \field{description}=\marg{rectangular array of values}
}
\end{codeenv}
and the document contains:
\begin{codeenv}
\cmd{documentclass}\marg{article}
\strut
\cmd{usepackage}[colorlinks]\marg{hyperref}
\cmd{usepackage}[\styopt{record},\styopt{index},\styopt[indexgroup]{style}]\marg{glossaries-extra}
\strut
\gls{GlsXtrLoadResources}\oarg{
 \csopt[entries]{src},
 \csopt[index]{type},
 \csopt[idx.]{label-prefix},
 \csopt[gls.]{dual-prefix},
 \csopt[main]{dual-type}
}
\strut
\cmd{begin}\marg{document}
\cs{gls}\marg{gls.array}, \cs{gls}\marg{gls.vector}, \cs{gls}\marg{gls.set}, \cs{gls}\marg{gls.matrix}.
\strut
\cmd{newpage}
\cs{gls}\marg{gls.array}, \cs{gls}\marg{idx.vector}, \cs{gls}\marg{idx.set}, \cs{gls}\marg{gls.matrix}.
\strut
\cmd{newpage}
\cs{gls}\marg{gls.array}, \cs{gls}\marg{gls.vector}, \cs{gls}\marg{gls.set}, \cs{gls}\marg{gls.matrix}.
\strut
\cs{printunsrtglossaries}
\cmd{end}\marg{document}
\end{codeenv}
In this case, the \glspl{primaryentry} are placed in the \code{index}
glossary type and are assigned the prefix \idprefixfmt{idx} but only
two of the \glspl{primaryentry} have been used in the document (both on
page~2).

The \glspl{dualentry} are assigned the prefix \idprefixfmt{gls} and are
placed in the \mainglossary. The \code{gls.array} 
and \code{gls.matrix} entries have been indexed on pages~1, 2 and~3.
The \code{gls.vector} and \code{gls.set} entries have been indexed
on pages~1 and~3.

With the default setting, some of the \glspl{location} are in the
\mainglossary\ (corresponding to \code{\cs{gls}\marg{gls.array}},
\code{\cs{gls}\marg{gls.vector}}, \code{\cs{gls}\marg{gls.set}} and
\code{\cs{gls}\marg{gls.matrix}}) and some of the \glspl{location} are in the
\code{index} glossary (corresponding to \code{\cs{gls}\marg{idx.vector}}
and \code{\cs{gls}\marg{idx.set}}).

If the option \csopt[primary]{combine-dual-locations} is added to
the resource set, then all the \glspl{location} are moved to the
\code{index} glossary. The entries in the \mainglossary\ no longer
have \glspl{location}. This is actually preferable for this type of
document and it's best not to reference the \primary\ (index)
entries as the hyperlink created by \ics{gls} will point to the
index, but these entries don't have descriptions, so it's less
useful than referencing the \dual\ (\code{main}) entries as then the
hyperlink can point to the definition in the \mainglossary.


\subsection{Dual Fields}
\label{sec:dualoptsfields}

\optsection[\subsubsection]{dual-type}

This option sets the \field{type} field for all \glspl{dualentry}.
(The \glspl{primaryentry} obey the \csopt{type} option.) This
will override any value of \field{type} provided in the \ext{bib}
file (or created through a mapping). The \meta{value} is required and
should be one of:
\begin{itemize}
\item \optfmt{false}: switches off this setting (default);

\item \optfmt{same as entry}: sets the \field{type} to the 
\gls{entrytype} (\idx!{lowercase} and without 
the initial \code{@}). For example, if the entry was defined with
\atentry{dualentry}, the \field{type} will be set to
\optfmt{dualentry}.
If you've used \csopt{entry-type-aliases}, this refers to the target
\gls{entrytype} not the original \gls{entrytype} provided in the \ext{bib}
file.

\item \optfmt{same as original entry}: set the \field{type} field
to the original \gls{entrytype} (\idx!{lowercase} and without 
the initial \code{@}) before it was aliased (behaves like 
\optfmt{same as entry} if the \gls{entrytype} wasn't aliased).

\item \optfmt{same as base}: sets the \field{type} to the base name
of the \ext{bib} file (without the extension) that provided the
entry definition (new to v1.1);

\item \optfmt{same as primary}: sets the \field{type} to the same as
the corresponding \gls{primaryentry}['s] \field{type} (which may have been
set with \csopt{type}). If the \gls{primaryentry} doesn't have the
\field{type} field set, the \glsdisp{dualentry}{dual's} \field{type} will remain
unchanged.

\item \optfmt{same as parent}: sets the \field{type} to the same as
the entry's \parent\ (new to v1.9). If the entry doesn't have a
\parent\ or if the \parent\ doesn't have the \field{type} field set,
then no change is made.

\item \optfmt{same as category} set the \field{type} field
to the same value as the \field{category} field
(\field{type} unchanged if \field{category} not set);

\item \meta{label}: sets the \field{type} field to \meta{label}.
\end{itemize}

Remember that the glossary with that label must have already 
been defined (see \sectionref{sec:newglossary}).
For example:
\begin{codeenv}
\cs{newglossary*}\marg{english}\marg{English}
\cs{newglossary*}\marg{french}\marg{French}
\strut
\gls{GlsXtrLoadResources}\oarg{\csopt[entries]{src},\csopt[en]{sort},\csopt[fr]{dual-sort},
 \csopt[english]{type},
 \csopt[french]{dual-type}}
\end{codeenv}
Alternatively:
\begin{codeenv}
\cs{newglossary*}\marg{dictionary}\marg{Dictionary}
\strut
\gls{GlsXtrLoadResources}\oarg{\csopt[entries]{src},\csopt[en]{sort},\csopt[fr]{dual-sort},
 \csopt[dictionary]{type},
 \csopt[same as primary]{dual-type}}
\end{codeenv}

\optsection[\subsubsection]{dual-category}

This option sets the \field{category} field for all
\glspl{dualentry}. (The \glspl{primaryentry} obey the \csopt{category} option.) This
will override any value of \field{category} provided in the \ext{bib}
file (or created through a mapping). The \meta{value} may be empty or
one of:
\begin{itemize}
\item \optfmt{false}: switch off this setting (default);
\item \optfmt{same as entry}: sets the \field{category} to the 
\gls{entrytype} (\idx{lowercase} and without 
the initial \code{@}). For example, if the entry was defined with
\atentry{dualentry}, the \field{category} will be set to
\optfmt{dualentry}.
If you've used \csopt{entry-type-aliases}, this refers to the target
\gls{entrytype} not the original \gls{entrytype} provided in the \ext{bib}
file.

\item \optfmt{same as original entry}: set the \field{category} field
to the original \gls{entrytype} (\idx{lowercase} and without 
the initial \code{@}) before it was aliased (behaves like 
\optfmt{same as entry} if the \gls{entrytype} wasn't aliased).

\item \optfmt{same as base}: sets the \field{category} to the base
name of the \ext{bib} file (without the extension) that provided the
entry definition (new to v1.1);

\item \optfmt{same as primary}: sets the \field{category} to the
same as the corresponding \gls{primaryentry}['s] \field{category} (which
may have been set with \csopt{category}). If the \gls{primaryentry}
doesn't have the \field{category} field set, the \glsdisp{dualentry}{dual's}
\field{category} will remain unchanged.

\item \optfmt{same as type}: sets the \field{category} to the same
as the value of the entry's \field{type} field (which may have been
set with \csopt{dual-type}). If the entry doesn't have the
\field{type} field set, the \field{category} will remain unchanged.

\item \meta{label}: sets the \field{category} field to \meta{label}.
\end{itemize}

\optsection[\subsubsection]{dual-counter}

As \csopt{counter} but for the \glspl{dualentry}. In this case
\meta{value} may be the name of the counter or 
\optfmt{same as primary} which uses the counter for the
\gls{primaryentry} or \optfmt{false} to switch off this setting.

\optsection[\subsubsection]{dual-short-case-change}

As \csopt{short-case-change} but applies to the \field{dualshort}
field instead.

\optsection[\subsubsection]{dual-long-case-change}

As \csopt{long-case-change} but applies to the \field{duallong}
field instead.

\optsection[\subsubsection]{dual-field}

If this option is used, this will add \ics{glsxtrprovidestoragekey}
to the start of the \iext{glstex} file providing the key given by
\meta{value}.  Any entries defined using a dual \gls{entrytype}, such as
\atentry{dualentry}, will be written to the \ext{glstex} file with
an extra field called \meta{value} that is set to the mirror entry.
If \meta{value} is omitted \csopt[dual]{dual-field} is assumed. If
you use a different value, you will need to redefine
\gls{GlsXtrDualField} (either locally or globally).
A value of \code{false} will switch off this setting (the default).

For example, if the \ext{bib} file contains:
\begin{codeenv}
\atentry{dualentry}\marg{child,
  \field{name}=\marg{child},
  \field{plural}=\marg{children},
  \field{description}=\marg{enfant}
}
\end{codeenv}
Then with \csopt[dual]{dual-field} (or simply \csopt{dual-field}
without a value) this will first add the line:
\begin{codeenv}
\cs{glsxtrprovidestoragekey}\marg{\field{dual}}\marg{}\marg{}
\end{codeenv}
at the start of the file and will include the line:
\begin{codeenv*}
\field{dual}=\marg{dual.child},
\end{codeenv*}
for the \gls{primaryentry} (\code{child}) and the line:
\begin{codeenv}
\field{dual}=\marg{child},
\end{codeenv}
for the \gls{dualentry} (\code{dual.child}). It's then possible to
reference one entry from the other. For example, the post-description 
hook could contain:
\begin{codeenv}
 \cs{ifglshasfield}\marg{\field{dual}}\marg{\cs{glscurrententrylabel}}
 \marg{\comment{}
   \cs{space}
   (\cs{glshyperlink}\marg{\cs{glscurrentfieldvalue}})\comment{}
 }\comment{}
 \marg{}\comment{}
\end{codeenv}
Note that this new field won't be available for use within the
\iext{bib} file (unless it was previously defined in the document
before \gls!{glsxtrresourcefile}).

\optsection[\subsubsection]{dual-date-time-field-format}

As \csopt{date-time-field-format} but is used for \glspl{dualentry}.

\optsection[\subsubsection]{dual-date-field-format}

As \csopt{date-field-format} but is used for \glspl{dualentry}.

\optsection[\subsubsection]{dual-time-field-format}

As \csopt{time-field-format} but is used for \glspl{dualentry}.

\optsection[\subsubsection]{dual-date-time-field-locale}

As \csopt{date-time-field-locale} but is used for \glspl{dualentry}.

\optsection[\subsubsection]{dual-date-field-locale}

As \csopt{date-field-locale} but is used for \glspl{dualentry}.

\optsection[\subsubsection]{dual-time-field-locale}

As \csopt{time-field-locale} but is used for \glspl{dualentry}.

\subsection{Dual Sorting}
\label{sec:dualoptssort}

\optsection[\subsubsection]{dual-sort}

This option indicates how to sort the \glspl{dualentry}. 
The \glspl{primaryentry} are sorted with the normal entries according to
\csopt{sort}, and the \glspl{dualentry} are sorted according to
\csopt{dual-sort} unless \csopt[combine]{dual-sort} in which case the 
\glspl{dualentry} will be combined with the \glspl{primaryentry} and all the
entries will be sorted together according to the \csopt{sort} option.

If \meta{value} isn't set to \optfmt{combine} then the 
\glspl{dualentry} are sorted separately according to \meta{value} (as per
\csopt{sort}) and the \glspl{dualentry} will be appended at the end of
the \iext{glstex} file. The field used by the comparator is given by
\csopt{dual-sort-field}.
If \csopt[custom]{dual-sort}, then the \glspl{dualentry} are sorted according to the 
rule provided by \csopt{dual-sort-rule}.

For example:
\begin{codeenv}
\gls{GlsXtrLoadResources}\oarg{
 \csopt[entries-dual]{src},
 \csopt[en]{sort},
 \csopt[de-CH-1996]{dual-sort}
}
\end{codeenv}
This will sort the \glspl{primaryentry} according to \optfmt{en}
(English) and the \glspl{secondaryentry} according to \optfmt{de-CH-1996} 
(Swiss German new orthography) whereas:
\begin{codeenv}
\gls{GlsXtrLoadResources}\oarg{
 \csopt[entries-dual]{src},
 \csopt[en-GB]{sort},
 \csopt[combine]{dual-sort}
}
\end{codeenv}
will combine the dual entries with the \glspl{primaryentry} and sort them
all according to the \optfmt{en-GB} locale (British English).

If not set, \csopt{dual-sort} defaults to \optfmt{combine}. If
\meta{value} is omitted, \optfmt{resource} is assumed.

\optsection[\subsubsection]{dual-sort-field}

This option indicates the field to use when sorting dual entries
(when they haven't been combined with the \glspl{primaryentry}). The
default value is the same as the \csopt{sort-field} value.

\optsection[\subsubsection]{dual-missing-sort-fallback}

As \csopt{missing-sort-fallback} but for \dual\ sorting.

\optsection[\subsubsection]{dual-trim-sort}

As \csopt{trim-sort} but for \dual\ sorting.

\optsection[\subsubsection]{dual-sort-replace}

As \csopt{sort-replace} but for \dual\ sorting.

\optsection[\subsubsection]{dual-sort-rule}

As \csopt{sort-rule} but for \csopt[custom]{dual-sort}.

\optsection[\subsubsection]{dual-break-at}

As \csopt{break-at} but for \glspl{dualentry}.

\optsection[\subsubsection]{dual-break-marker}

As \csopt{break-marker} but for \glspl{dualentry}.

\optsection[\subsubsection]{dual-break-at-match}

As \csopt{break-at-match} but for \glspl{dualentry}.

\optsection[\subsubsection]{dual-break-at-match-op}

As \csopt{break-at-match-op} but for \glspl{dualentry}.

\optsection[\subsubsection]{dual-break-at-not-match}

As \csopt{break-at-not-match} but for \glspl{dualentry}.

\optsection[\subsubsection]{dual-sort-number-pad}

As \csopt{sort-number-pad} but for \glspl{dualentry}.

\optsection[\subsubsection]{dual-sort-pad-plus}

As \csopt{sort-pad-plus} but for \glspl{dualentry}.

\optsection[\subsubsection]{dual-sort-pad-minus}

As \csopt{sort-pad-minus} but for \glspl{dualentry}.

\optsection[\subsubsection]{dual-identical-sort-action}

As \csopt{identical-sort-action} but for \glspl{dualentry}.

\optsection[\subsubsection]{dual-sort-suffix}

As \csopt{sort-suffix} but for \glspl{dualentry}.

\optsection[\subsubsection]{dual-sort-suffix-marker}

As \csopt{sort-suffix-marker} but for \glspl{dualentry}.

\optsection[\subsubsection]{dual-strength}

As \csopt{strength} but for \glspl{dualentry}.

\optsection[\subsubsection]{dual-decomposition}

As \csopt{decomposition} but for \glspl{dualentry}.

\optsection[\subsubsection]{dual-letter-number-rule}

As \csopt{letter-number-rule} but for \glspl{dualentry} that use a
letter-number sort.

\optsection[\subsubsection]{dual-letter-number-punc-rule}

As \csopt{letter-number-punc-rule} but for \glspl{dualentry} that use a
letter-number sort.

\optsection[\subsubsection]{dual-numeric-sort-pattern}

As \csopt{numeric-sort-pattern} but for \glspl{dualentry} that use a
locale-sensitive numeric sort.

\optsection[\subsubsection]{dual-numeric-locale}

As \csopt{numeric-locale} but for \glspl{dualentry} that use a
locale-sensitive numeric sort.

\optsection[\subsubsection]{dual-date-sort-locale}

As \csopt{date-sort-locale} but for \glspl{dualentry} that
use a date/time sort.

\optsection[\subsubsection]{dual-date-sort-format}

As \csopt{date-sort-format} but for \glspl{dualentry} that
use a date/time sort.

\optsection[\subsubsection]{dual-group-formation}

As \csopt{group-formation} but for \dual\ sorting.


\subsection{Dual Mappings}
\label{sec:dualoptsmap}

\optsection[\subsubsection]{dual-entry-map}

This setting governs the behaviour of \atentry{dualentry}
definitions. The value consists of two comma-separated lists of
equal length identifying the field mapping used to create the 
\gls{dualentry} from the \primary\ one. Note that the \field{alias} field
can't be mapped.

The default setting is:
\begin{codeenv}
\csopt[
  \marg{\field{name},\field{plural},\field{description},\field{descriptionplural}},
  \marg{\field{description},\field{descriptionplural},\field{name},\field{plural}}
]{dual-entry-map}
\end{codeenv}
The \gls{dualentry} is created by copying the value of the field in the
first list \meta{list1} to the field in the corresponding place in the second
list \meta{list2}. Any additional fields are copied over to the same
field.
For example:
\begin{codeenv}
\atentry{dualentry}\marg{cat,
  \field{name}=\marg{cat},
  \field{description}=\marg{chat},
  \field{see}=\marg{dog}
}
\end{codeenv}
defines two entries. The \gls{primaryentry} is essentially like:
\begin{codeenv}
\atentry{entry}\marg{cat,
  \field{name}=\marg{cat},
  \field{plural}=\marg{cat\cs{glspluralsuffix} },
  \field{description}=\marg{chat},
  \field{descriptionplural}=\marg{chat\cs{glspluralsuffix} },
  \field{see}=\marg{dog}
}
\end{codeenv}
and the \gls{dualentry} is essentially like:
\begin{codeenv}
\atentry{entry}\marg{dual.cat,
  \field{description}=\marg{cat},
  \field{descriptionplural}=\marg{cat\cs{glspluralsuffix} },
  \field{name}=\marg{chat},
  \field{plural}=\marg{chat\cs{glspluralsuffix} },
  \field{see}=\marg{dog}
}
\end{codeenv}
(except they're defined using \gls{bibglsnewdualentry} instead of
\gls{bibglsnewentry}, and each is considered dependent on the other.)

The \field{see} field isn't listed in \csopt{dual-entry-map} so its
value is simply copied directly over to the \field{see} field in the
dual entry. Note that the missing \field{plural} and
\field{descriptionplural} fields have been filled in using their
fallback values (see \sectionref{sec:fallbacks}).

In general \bibgls\ doesn't try to supply missing fields, but in the
dual entry cases it needs to do this for the mapped fields. This is
because the shuffled fields might have different default values from
the \styfmt{glossaries-extra} package's point of view. For example, 
\gls{longnewglossaryentry} doesn't provide a default for
\field{descriptionplural} if it hasn't been set.

\optsection[\subsubsection]{dual-abbrv-map}

This is like \csopt{dual-entry-map} but applies to
\atentry{dualabbreviation} rather than \atentry{dualentry}. 
Note that the \field{alias} field can't be mapped. The
default setting is:
\begin{codeenv}
\csopt[
  \marg{\field{short},\field{shortplural},\field{long},\field{longplural},\field{dualshort},\field{dualshortplural},
   \field{duallong},\field{duallongplural}},
  \marg{\field{dualshort},\field{dualshortplural},\field{duallong},\field{duallongplural},\field{short},\field{shortplural},
   \field{long},\field{longplural}}
]{dual-abbrv-map}
\end{codeenv}

This essentially flips the \field{short} field with the
\field{dualshort} field and the \field{long} field with the
\field{duallong} field. See \atentry{dualabbreviation}
for further details.

\optsection[\subsubsection]{dual-abbrventry-map}

This is like \csopt{dual-entry-map} but applies to
\atentry{dualabbreviationentry} rather than \atentry{dualentry}. 
Note that the \field{alias} field can't be mapped. The
default setting is:
\begin{codeenv}
\csopt[
  \marg{\field{long},\field{short},\field{shortplural}},
  \marg{\field{name},\field{text},\field{plural}}
]{dual-abbrventry-map}
\end{codeenv}
See \atentry{dualabbreviationentry} for further details.


\optsection[\subsubsection]{dual-symbol-map}

This is like \csopt{dual-entry-map} but applies to
\atentry{dualsymbol} rather than \atentry{dualentry}. 
Note that the \field{alias} field can't be mapped. The
default setting is:
\begin{codeenv}
\csopt[
  \marg{\field{name},\field{plural},\field{symbol},\field{symbolplural}},
  \marg{\field{symbol},\field{symbolplural},\field{name},\field{plural}}
]{dual-symbol-map}
\end{codeenv}
This essentially flips the \field{name} field with the
\field{symbol} field.

\optsection[\subsubsection]{dual-indexentry-map}

This is like \csopt{dual-entry-map} but applies to
\atentry{dualindexentry} rather than \atentry{dualentry}. 
The default setting is:
\begin{codeenv}
\csopt[
  \marg{\field{name}},
  \marg{\field{name}}
]{dual-indexentry-map}
\end{codeenv}
Note that there must always be at least one pair, even if it's the
same field, since this identifies the field to use for the backlink,
if set.

\optsection[\subsubsection]{dual-indexsymbol-map}

This is like \csopt{dual-entry-map} but applies to both
\atentry{dualindexsymbol} and \atentry{dualindexnumber}. 
The default setting is:
\begin{codeenv}
\csopt[
  \marg{\field{symbol},\field{name},\field{symbolplural},\field{plural}},
  \marg{\field{name},\field{symbol},\field{plural},\field{symbolplural}}
]{dual-indexsymbol-map}
\end{codeenv}

\optsection[\subsubsection]{dual-indexabbrv-map}

This is like \csopt{dual-entry-map} but applies to both the dual
\atentry{dualindexabbreviation} and tertiary
\atentry{tertiaryindexabbreviationentry} \glspl{entrytype}. 
The default setting is:
\begin{codeenv}
\csopt[
  \marg{\field{name}},
  \marg{\field{name}}
]{dual-indexabbrv-map}
\end{codeenv}

\subsection{Dual Back-Links}
\label{sec:dualoptsbacklinks}

\optsection[\subsubsection]{dual-entry-backlink}

This is a boolean setting. If \meta{boolean} is missing \optfmt{true}
is assumed.

When used with \atentry{dualentry}, if \meta{boolean} is
\optfmt{true}, this will wrap the contents of the first mapped field
with \gls{bibglshyperlink}.  The field is obtained from the first
mapping listed in \csopt{dual-entry-map}.

For example, if the document contains:
\begin{codeenv}
\gls{GlsXtrLoadResources}\oarg{\csopt{dual-entry-backlink},
  \csopt[
    \marg{\field{name},\field{plural},\field{description},\field{descriptionplural}},
    \marg{\field{description},\field{descriptionplural},\field{name},\field{plural}}
  ]{dual-entry-map},
  \csopt[entries-dual]{src}}
\end{codeenv}
and if the \ext{bib} file contains:
\begin{codeenv}
\atentry{dualentry}\marg{child,
  \field{name}=\marg{child},
  \field{plural}=\marg{children},
  \field{description}=\marg{enfant}
}
\end{codeenv}
Then the definition of the \gls{primaryentry} (\code{child}) in the
\ext{glstex} file will set the \field{description} field to:
\begin{codeenv}
\gls{bibglshyperlink}\marg{enfant}\marg{dual.child}
\end{codeenv}
and the \gls{dualentry} (\code{dual.child}) will have the
\field{description} field set to:
\begin{codeenv}
\gls{bibglshyperlink}\marg{child}\marg{child}
\end{codeenv}
This use of the wrapper \gls{bibglshyperlink} (rather than explicitly
using \ics{glshyperlink}) and inserting the actual field value
(rather than using commands like \ics{glsentryname}) allows it 
to work with \ics{makefirstuc} if the field requires a case-change.

The reason the \field{description} field is chosen for the modification is
because the first field listed in \meta{list1} of \csopt{dual-entry-map} 
is the \field{name} field which maps to \field{description} (the
first field in the second list \meta{list2}). This means that the hyperlink for
the \gls{dualentry} should be put in the \field{description} field.

For the \gls{primaryentry}, the \field{name} field is looked up in the
second list from the \csopt{dual-entry-map} setting. This is the
third item in this second list, so the third item in the first list
is selected, which also happens to be the \field{description} field,
so the hyperlink for the \gls{primaryentry} is put in the
\field{description} field.

\optsection[\subsubsection]{dual-abbrv-backlink}

This is analogous to \csopt{dual-entry-backlink} but for entries
defined with \atentry{dualabbreviation} instead of
\atentry{dualentry}.

\optsection[\subsubsection]{dual-symbol-backlink}

This is analogous to \csopt{dual-entry-backlink} but for entries
defined with \atentry{dualsymbol} instead of \atentry{dualentry}.

\optsection[\subsubsection]{dual-abbrventry-backlink}

Analogous to \csopt{dual-entry-backlink} but for entries
defined with \atentry{dualabbreviationentry} instead of
\atentry{dualentry}. This setting can be problematic as the
backlinks rely on the relevant field being known to \bibgls.
Since the abbreviation style typically sets the \field{name}
field (and sometimes the \field{description} field as well), you may
find that no backlink appears. A simple workaround is to 
use \csopt{dual-field} (or \csopt[dual]{dual-field}) 
to store the dual label in the \field{dual} field, and then
use a style that checks for this field and adds the backlink.

With \sty{glossaries-extra} v1.30+ you can use:
\nosecdef{GlsXtrDualBackLink}
which encapsulates \meta{text} with a hyperlink to the \dual.
The \meta{label} identifies the entry that requires a backlink.
The \glsdisp{dualentry}{dual's} label is obtained from the field given by:
\nosecdef{GlsXtrDualField}
which defaults to \field{dual}. Note that if you assign a 
different field label with \csopt{dual-field}, then you will
need to redefine \gls{GlsXtrDualField} as appropriate.

For example:
\begin{codeenv}
\cs{renewcommand}*\marg{\ics{glsuserdescription}}[2]\marg{\comment{}
  \gls{GlsXtrDualBackLink}\marg{\ics{glslonguserfont}\marg{\idx{param}1}}\marg{\idx{param}2}\comment{}
}
\cs{setabbreviationstyle}\marg{\abbrstyle{long-short-user}}
\gls{GlsXtrLoadResources}\oarg{\csopt[entries]{src},\csopt{dual-field}}
\end{codeenv}

\optsection[\subsubsection]{dual-entryabbrv-backlink}

As \csopt{dual-abbrventry-backlink} but for entries
defined with \atentry{dualentryabbreviation} instead of
\atentry{dualabbreviationentry}.

\optsection[\subsubsection]{dual-indexentry-backlink}

This is analogous to \csopt{dual-entry-backlink} but for entries
defined with \atentry{dualindexentry} instead of
\atentry{dualentry}.

\optsection[\subsubsection]{dual-indexsymbol-backlink}

This is analogous to \csopt{dual-entry-backlink} but for entries
defined with \atentry{dualindexsymbol} and
\atentry{dualindexnumber}.

\optsection[\subsubsection]{dual-indexabbrv-backlink}

This is analogous to \csopt{dual-entry-backlink} but for entries
defined with \atentry{dualindexabbreviation} and
\atentry{tertiaryindexabbreviationentry}.

\optsection[\subsubsection]{dual-backlink}

Shortcut for:
\begin{flushleft}\obeylines
\csopt[\meta{boolean}]{dual-entry-backlink}, 
\csopt[\meta{boolean}]{dual-abbrventry-backlink}, 
\csopt[\meta{boolean}]{dual-abbrv-backlink},
\csopt[\meta{boolean}]{dual-symbol-backlink},
\csopt[\meta{boolean}]{dual-indexentry-backlink},
\csopt[\meta{boolean}]{dual-indexsymbol-backlink},
\csopt[\meta{boolean}]{dual-indexabbrv-backlink}
\end{flushleft}

\section{Tertiary Entries}

\optsection{tertiary-prefix}

This option indicates the prefix to use for the \glspl{tertiaryentry}. The
default value is \idprefix{tertiary} (including the terminating period). 

As from version 1.8, the \glsdisp{tertiaryentry}{tertiary} label prefix is identified
in the \ext{glstex} file with:
\begin{codeenv}
\format{bibglstertiaryprefixlabel}
\end{codeenv}

\optsection{tertiary-type}

This option indicates that the \glspl{tertiaryentry} should have
their \field{type} field set to \meta{value}. If \meta{value} is
empty the \field{type} is left unchanged. Unlike 
the \csopt{type} and \csopt{dual-type} options, there are no
recognised keywords.

\optsection{tertiary-category}

This option indicates that the \glspl{tertiaryentry} should have
their \field{category} field set to \meta{value}. If \meta{value} is
empty the \field{category} is left unchanged. Unlike 
the \csopt{category} and \csopt{dual-category} options, there are no
recognised keywords.

\section{Compound (Combined or Multi) Entries}
\label{sec:compoundentries}

These options refer to \glspl{compoundentry} which are either
defined in a \ext{bib} file with \atentry{compoundset} or are
defined in the document using \cs{multiglossaryentry} (or
\cs{provideglossaryentry}). See \sectionref{sec:compoundsetentry}
for further details.

\optsection{compound-options-global}

This is a boolean option. The default is
\csopt[true]{compound-options-global}.

If \optfmt{true}, the \gls{compoundentry} options described in this
section, except for \csopt{compound-write-def}, pick up all
\glspl{compoundentry} provided in the document (defined either with
\atentry{compoundset} or in the document with
\ics{multiglossaryentry}).

If \optfmt{false}, options only apply to \glspl{compoundentry} defined
with \atentry{compoundset} in the current \gls{resourceset}.

If \igls{crossresourceref} are disabled then any instances of
\atentry{compoundset} in subsequent \glspl{resourceset} can only be
picked up from the \ext{aux} file on the next build.

\optsection{compound-dependent}

This is a boolean option. The default is
\csopt[false]{compound-dependent}.

If you have chosen to switch off indexing for the \glspl{compotherlabel}
then they may not be selected (unless they have been indexed via
another method, such as explicitly using \cs{gls}). You can use this
option to make the \glspl{compotherlabel} dependencies of the
\gls{compmainlabel} (if the entry given by \gls{compmainlabel} is
present in the current \gls{resourceset}).

This means that if the \gls{compmainlabel} is selected then the
\glspl{compotherlabel} should also be selected (if dependencies are
part of the selection criteria).

\optsection{compound-add-hierarchy}

This is a boolean option. The default is
\csopt[false]{compound-add-hierarchy}.
If true, this will set the \field{parent} field for each element
$e_i$ to the previous element $e_{i-1}$ in the list, provided that:
\begin{itemize}
\item the element $e_i$ isn't the first element in the list;
\item the element $e_i$ and the previous element $e_{i-1}$ are 
present in the current \gls{resourceset};
\item the element $e_i$ doesn't already have the \field{parent} field set;
\item the element $e_i$ isn't an \gls{ancestor} of the previous element $e_{i-1}$.
\item the previous element $e_{i-1}$ isn't an \gls{ancestor} of the 
element $e_i$.
\end{itemize}
For example, if the \ext{bib} file contains:
\begin{codeenv}
\atentry{abbreviation}\marg{clostridium,
 \field{short}=\marg{C.},
 \field{long}=\marg{Clostridium}
}
\atentry{index}\marg{botulinum}
\atentry{compoundset}\marg{cbot,
 \field{elements}=\marg{clostridium,botulinum}}
\end{codeenv}
Then the \code{botulinum} entry will have its \field{parent} field
set to \code{clostridium}. The \code{clostridium} entry won't be
adjusted (since it's the first element in the list).

\optsection{compound-has-records}

This option may take one of the following values:
\begin{description}
\item[\optfmt{true}] 
Any \gls{compoundentry} referenced with commands like
\ics{mgls} is considered to have \glspl{record} for each element for
selection purposes, even if there are no \glspl{record} in the \ext{aux}
file. (This is useful on the first \LaTeX\ run where the
\glspl{compoundentry} are defined with \atentry{compoundset}.)

\item[\optfmt{false}]
The element \glspl{record} are created as they normally are with
commands like \cs{gls} that are internally used by \cs{mgls}.

\item[\optfmt{default}]
Behaves like \csopt[true]{compound-has-records} if the current 
\gls{resourceset} has any \ext{bib} files containing one or more
\atentry{compoundset} \glspl{entrytype}. Otherwise behaves like
\csopt[false]{compound-has-records}.
\end{description}

The default is \csopt[default]{compound-has-records}.
If the value is omitted, \optfmt{true} is assumed.

\optsection{compound-adjust-name}

If an entry has been identified as the \gls{compmainlabel} in any
\glspl{compoundentry} then the \field{name} field can be adjusted
with this option. Allowed values:
\begin{itemize}
\item \optfmt{false} don't adjust the \field{name} field (default);
\item \optfmt{unique} only adjust the \field{name} field if 
the entry is the \gls{compmainlabel} of exactly one set;
\item \optfmt{once} adjust the \field{name} field if 
the entry is the \gls{compmainlabel} of any set. Only one adjustment
is made. If the entry is the \gls{compmainlabel} of multiple
\glspl{compoundentry} there's no guarantee which set will be chosen
for the adjustment.
\end{itemize}

If the value isn't supplied, \optfmt{once} is assumed. As with
\csopt{name-case-change}, the pre-adjusted \field{name} value will
be copied to the \field{text} field provided the \field{text} field
hasn't already been set and provided that the entry isn't an
abbreviation.

The adjusted value will be in the form:
\begin{codeenv*}
\ics{glsxtrmultientryadjustedname}\margm{sublist1}\margm{name}\margm{sublist2}\marg{mlabel}
\end{codeenv*}
where \meta{label} is the \gls{compoundentry} label, \meta{name} was
the value of the \field{name} field before the adjustment,
\meta{sublist1} is the list of \glspl{compotherlabel} before the
\gls{compmainlabel} (which will be empty if the \gls{compmainlabel}
is the first element in the set) and
\meta{sublist2} is the list of \glspl{compotherlabel} after the
\gls{compmainlabel} (which will be empty if the \gls{compmainlabel}
is the last element in the set).

The adjustment is made before \csopt{name-case-change} (if set).
The control sequence case changes (such as
\csopt[firstuc-cs]{name-case-change}) will replace \cs{glsxtrmultientryadjustedname}
with the relevant command (\ics{Glsxtrmultientryadjustedname} for
\optfmt{firstuc-cs}, \ics{GlsXtrmultientryadjustedname} for
\optfmt{title-cs} and \cs{GLSxtrmultientryadjustedname} for
\optfmt{uc-cs}).

\optsection{compound-main-type}

Set the \field{type} field of the \glslink{compmainlabel}{main
entries}. The \meta{value} is required and should be one of:
\begin{itemize}
\item \optfmt{same as entry}: sets the \field{type} to the 
\gls{entrytype} (\idx!{lowercase} and without 
the initial \code{@}). For example, if the entry was defined with
\atentry{index}, the \field{type} will be set to \optfmt{index}.
If you've used \csopt{entry-type-aliases}, this refers to the target
\gls{entrytype} not the original \gls{entrytype} provided in the \ext{bib}
file.

\item \optfmt{same as original entry}: set the \field{type} field
to the original \gls{entrytype} (\idx!{lowercase} and without 
the initial \code{@}) before it was aliased (behaves like 
\optfmt{same as entry} if the \gls{entrytype} wasn't aliased).

\item \optfmt{same as base}: sets the \field{type} to the base name
of the \ext{bib} file (without the extension) that provided the
entry definition;

\item \optfmt{same as category}: sets the \field{type} to the same
as the \field{category} field;

\item \optfmt{same as parent}: sets the \field{type} to the same as
the entry's \parent. If the entry doesn't have a
\parent\ or if the \parent\ doesn't have the \field{type} field set,
then no change is made.

\item \meta{label}: sets the \field{type} field to \meta{label}.
\end{itemize}

This setting is governed by \csopt{compound-type-override}.

\optsection{compound-other-type}

Set the \field{type} field of the \glslink{compotherlabel}{other
entries}. The \meta{value} is required and should be one of:
\begin{itemize}
\item \optfmt{same as main}: sets the \field{type} to the same as
the \glslink{compmainlabel}{main entry}.

\item \optfmt{same as entry}: sets the \field{type} to the 
\gls{entrytype} (\idx!{lowercase} and without 
the initial \code{@}). For example, if the entry was defined with
\atentry{index}, the \field{type} will be set to \optfmt{index}.
If you've used \csopt{entry-type-aliases}, this refers to the target
\gls{entrytype} not the original \gls{entrytype} provided in the \ext{bib}
file.

\item \optfmt{same as original entry}: set the \field{type} field
to the original \gls{entrytype} (\idx!{lowercase} and without 
the initial \code{@}) before it was aliased (behaves like 
\optfmt{same as entry} if the \gls{entrytype} wasn't aliased).

\item \optfmt{same as base}: sets the \field{type} to the base name
of the \ext{bib} file (without the extension) that provided the
entry definition;

\item \optfmt{same as category}: sets the \field{type} to the same
as the \field{category} field;

\item \optfmt{same as parent}: sets the \field{type} to the same as
the entry's \parent. If the entry doesn't have a
\parent\ or if the \parent\ doesn't have the \field{type} field set,
then no change is made.

\item \meta{label}: sets the \field{type} field to \meta{label}.
\end{itemize}

This setting is governed by \csopt{compound-type-override}.

\optsection{compound-type-override}

This is a boolean option. The default is
\csopt[false]{compound-type-override}.

If \optfmt{true}, then the options \csopt{compound-main-type}
and \csopt{compound-other-type} will overwrite the \field{type}
field otherwise those options will only set the \field{type} field
if it hasn't already been set.

\optsection{compound-write-def}

If \glspl{compoundentry} are defined in the \ext{bib} files using
\atentry{compoundset}, this option governs whether or not to write
their definition to the \ext{glstex} file. The value may be one of:
\begin{definition}
\item[\optfmt{none}] Don't write the definitions to the \ext{glstex}
file. For example, if you are reloading a \ext{bib} file from
another \gls{resourceset}, you will need this option to prevent
duplicate definitions. (The alternative is to define
\gls{bibglsdefcompoundset} to use \ics{providemultiglossaryentry}
instead of \ics{multiglossaryentry}.)

\item[\optfmt{all}] Write all definitions to the \ext{glstex} file,
regardless of whether or not they have been referenced using
commands like \cs{mgls}.

\item[\optfmt{ref}] Only write the definitions for
\glspl{compoundentry} that have been referenced using commands like
\cs{mgls}. (Default.)
\end{definition}

The \glspl{compoundentry} are defined in the \ext{glstex} file with 
\gls{bibglsdefcompoundset}.

\chapter{Provided Commands}
\label{sec:bibglscs}

When \bibgls\ creates the \iext{glstex} file, it writes some
definitions for custom commands in the form \csfmt{bibgls\ldots} which
may be changed as required. The command definitions all use
\ics{providecommand} which means that you can define the command with
\ics{newcommand} before the resource file is loaded.

Note that if you try to redefine any of these commands after the
resource file has been loaded with \ics{renewcommand}, you will get
an error on the first \LaTeX\ run when the \ext{glstex} file doesn't
exist. You may prefer to use \ics{glsrenewcommand} instead, which
will generate a warning instead of an error.

Since many of the commands are actually used within the \ext{glstex}
file, it's best to use \cs{newcommand} before the first
\igls{resourceset} and \cs{renewcommand} between \iglspl{resourceset}
if adjustments are necessary.


\section{Entry Definitions}
\label{sec:newentrydefs}

This section lists the commands (\csfmt{bibglsnew\ldots}) used to
define entries.  Note that the entry definition commands are
actually used when \TeX\ inputs the resource file, so redefining
them after the resource file is loaded won't have an effect on the
entries defined in that resource file (but will affect entries defined
in subsequent resource files). Each provided command is defined in
the \ext{glstex} file immediately before the first entry that
requires it, so only the commands that are actually needed are
provided.

The \field{sort} key may be set within the \ext{glstex} entry
definition, but its value is usually not required in the document
unless you are using a hybrid method with \styopt[hybrid]{record}
(in which case, it's redundant to get \bibgls\ to sort).

After each entry is defined, if it has any associated \glspl{location} and
the default \csopt[true]{save-loclist} is set, then the
\glspl{location} are added using:
\nosecformatdef{glsxtrfieldlistadd}
Any additional fields that don't have associated keys are then set (if
required) with \ics{GlsXtrSetField}.

\cssection{bibglsnewentry}

\formatdef{bibglsnewentry}
This command is used to define terms identified with the
\atentry{entry} type. The definition provided in the \ext{glstex}
file is:
\begin{codeenv}
\cs{providecommand}\marg{\gls{bibglsnewentry}}[4]\marg{\comment{}
 \cs{longnewglossaryentry*}\marg{\idx{param}1}\marg{\field{name}=\marg{\idx{param}3},\idx{param}2}\marg{\idx{param}4}\comment{}
}
\end{codeenv}
This uses the starred form \ics{longnewglossaryentry*} that
doesn't automatically append \ics{nopostdesc} (which interferes with
the post-description hooks provided by category attributes).

\cssection{bibglsnewsymbol}

\formatdef{bibglsnewsymbol}
This command is used to define terms identified with the
\atentry{symbol} type. The definition provided in the \ext{glstex}
file is:
\begin{codeenv}
\cs{providecommand}\marg{\gls{bibglsnewsymbol}}[4]\marg{\comment{}
 \cs{longnewglossaryentry*}\marg{\idx{param}1}\marg{\field{name}=\marg{\idx{param}3},\field{sort}=\marg{\idx{param}1},\field{category}=\marg{symbol},\idx{param}2}\marg{\idx{param}4}\comment{}
}
\end{codeenv}
Note that this sets the \field{sort} field to the label, but this
may be overridden by the \meta{options} if the \field{sort} field
was supplied or if \bibgls\ has determined the value whilst sorting
the entries.

This also sets the \field{category} to \optfmt{symbol}, but again this may
be overridden by \meta{options} if the entry had the \field{category}
field set in the \ext{bib} file or if the \field{category} was
overridden with \csopt[\meta{value}]{category}.

\cssection{bibglsnewnumber}

\formatdef{bibglsnewnumber}
This command is used to define terms identified with the
\atentry{number} type. The definition provided in the \ext{glstex}
file is:
\begin{codeenv}
\cs{providecommand}\marg{\gls{bibglsnewnumber}}[4]\marg{\comment{}
 \cs{longnewglossaryentry*}\marg{\idx{param}1}\marg{\field{name}=\marg{\idx{param}3},\field{sort}=\marg{\idx{param}1},\field{category}=\marg{number},\idx{param}2}\marg{\idx{param}4}\comment{}
}
\end{codeenv}
This is much the same as \gls{bibglsnewsymbol} above but sets the 
\field{category} to \optfmt{number}. Again the \field{sort} and
\field{category} keys may be overridden by \meta{options}.

\cssection{bibglsnewindex}

\formatdef{bibglsnewindex}
This command is used to define terms identified with the
\atentry{index} type. The definition provided in the \ext{glstex}
file is:
\begin{codeenv}
\cs{providecommand}*\marg{\gls{bibglsnewindex}}[2]\marg{\comment{}
 \gls{newglossaryentry}\marg{\idx{param}1}\marg{\field{name}=\marg{\idx{param}1},\field{category}=\marg{index},\field{description}=\marg{},\idx{param}2}\comment{}
}
\end{codeenv}
This makes the \field{name} default to the \meta{label}, assigns 
the \field{category} to \code{index} and sets an empty
\field{description}. These settings may be overridden by
\meta{options}.

Note that the \field{description} doesn't include
\ics{nopostdesc} to allow for the \idx{postdescriptionhook} used by
category attributes.

\cssection{bibglsnewindexplural}

\formatdef{bibglsnewindexplural}
This command is used to define terms identified with the
\atentry{indexplural} type. The definition provided in the \ext{glstex}
file is:
\begin{codeenv}
\cs{providecommand}\marg{\gls{bibglsnewindexplural}}[3]\marg{\comment{}
 \gls{newglossaryentry}\marg{\idx{param}1}\marg{\field{name}=\marg{\idx{param}3},\field{category}=\marg{indexplural},\field{description}=\marg{},\idx{param}2}\comment{}
}
\end{codeenv}
This assigns the \field{category} to \code{indexplural} and sets an empty
\field{description}. These settings may be overridden by \meta{options}.

\cssection{bibglsnewabbreviation}

\formatdef{bibglsnewabbreviation}
This command is used to define terms identified with the
\atentry{abbreviation} type. The definition provided in the \ext{glstex}
file is:
\begin{codeenv}
\cs{providecommand}\marg{\gls{bibglsnewabbreviation}}[4]\marg{\comment{}
  \gls{newabbreviation}\oarg{\idx{param}2}\marg{\idx{param}1}\marg{\idx{param}3}\marg{\idx{param}4}\comment{}
}
\end{codeenv}
Since this uses \gls{newabbreviation}, it obeys the 
abbreviation style for its given \field{category} (which may have
been set in \meta{options}, either from the \field{category} field
in the \ext{bib} file or through the \csopt{category} option).
Similarly the \field{type} will obey \ics{glsxtrabbrvtype} unless 
the value is supplied in the \ext{bib} file or through the
\csopt{type} option.

\cssection{bibglsnewacronym}

\formatdef{bibglsnewacronym}
This command is used to define terms identified with the
\atentry{acronym} type. The definition provided in the \ext{glstex}
file is:
\begin{codeenv}
\cs{providecommand}\marg{\gls{bibglsnewacronym}}[4]\marg{\comment{}
  \gls{newacronym}\oarg{\idx{param}2}\marg{\idx{param}1}\marg{\idx{param}3}\marg{\idx{param}4}\comment{}
}
\end{codeenv}
This works in much the same way as \gls{bibglsnewabbreviation}.
Remember that with the \styfmt{glossaries-extra} package \gls{newacronym}
is redefined to just use \gls{newabbreviation} with the default \field{type}
set to \ics{acronymtype} and the default \field{category} set to
\code{acronym}.

\cssection{bibglsnewdualentry}

\formatdef{bibglsnewdualentry}
This command is used to define terms identified with the
\atentry{dualentry} type. The definition provided in the \ext{glstex}
file is:
\begin{codeenv}
\cs{providecommand}\marg{\gls{bibglsnewdualentry}}[4]\marg{\comment{}
 \cs{longnewglossaryentry*}\marg{\idx{param}1}\marg{\field{name}=\marg{\idx{param}3},\idx{param}2}\marg{\idx{param}4}\comment{}
}
\end{codeenv}

\cssection{bibglsnewdualindexentry}

\formatdef{bibglsnewdualindexentry}
This command is used to define \primary\ terms identified with the
\atentry{dualindexentry} type. The definition provided in the \ext{glstex}
file is:
\begin{codeenv}
\cs{providecommand}\marg{\gls{bibglsnewdualindexentry}}[4]\marg{\comment{}
 \cs{longnewglossaryentry*}\marg{\idx{param}1}\marg{\field{name}=\marg{\idx{param}3},\field{category}=\marg{index},\idx{param}2}\marg{}\comment{}
}
\end{codeenv}
Note that this definition ignores the \meta{description} argument.

\cssection{bibglsnewdualindexentrysecondary}

\formatdef{bibglsnewdualindexentrysecondary}
This command is used to define \glsdisp{dualentry}{secondary} terms identified with the
\atentry{dualindexentry} type. The definition provided in the \ext{glstex}
file is:
\begin{codeenv}
\cs{providecommand}\marg{\gls{bibglsnewdualindexentrysecondary}}[4]\marg{\comment{}
  \cs{longnewglossaryentry*}\marg{\idx{param}1}\marg{\field{name}=\marg{\idx{param}3},\idx{param}2}\marg{\idx{param}4}\comment{}
}
\end{codeenv}

\cssection{bibglsnewdualindexsymbol}

\formatdef{bibglsnewdualindexsymbol}
This command is used to define \primary\ terms identified with the
\atentry{dualindexsymbol} type. The definition provided in the \ext{glstex}
file is:
\begin{codeenv}
\cs{providecommand}\marg{\gls{bibglsnewdualindexsymbol}}[5]\marg{\comment{}
  \cs{longnewglossaryentry*}\marg{\idx{param}1}\marg{\field{name}=\marg{\idx{param}3},\field{category}=\marg{index},\field{symbol}=\marg{\idx{param}4},\idx{param}2}\marg{}\comment{}
}
\end{codeenv}
Note that this definition ignores the \meta{description} argument.

\cssection{bibglsnewdualindexsymbolsecondary}

\formatdef{bibglsnewdualindexsymbolsecondary}
This command is used to define \glsdisp{dualentry}{secondary} terms identified with the
\atentry{dualindexsymbol} type. The definition provided in the \ext{glstex}
file is:
\begin{codeenv}
\cs{providecommand}\marg{\gls{bibglsnewdualindexsymbolsecondary}}[5]\marg{\comment{}
  \cs{longnewglossaryentry*}\marg{\idx{param}1}\marg{\field{name}=\marg{\idx{param}3},\field{category}=\marg{symbol},\field{symbol}=\marg{\idx{param}4},\idx{param}2}\marg{\idx{param}5}\comment{}
}
\end{codeenv}

\cssection{bibglsnewdualindexnumber}

\formatdef{bibglsnewdualindexnumber}
This command is used to define \primary\ terms identified with the
\atentry{dualindexnumber} type. The definition provided in the \ext{glstex}
file is:
\begin{codeenv}
\cs{providecommand}\marg{\gls{bibglsnewdualindexnumber}}[5]\marg{\comment{}
 \cs{longnewglossaryentry*}\marg{\idx{param}1}\marg{\field{name}=\marg{\idx{param}3},\field{category}=\marg{index},\field{symbol}=\marg{\idx{param}4},\idx{param}2}\marg{}\comment{}
}
\end{codeenv}
Note that this definition ignores the \meta{description} argument.

\cssection{bibglsnewdualindexnumbersecondary}

\formatdef{bibglsnewdualindexnumbersecondary}
This command is used to define \glsdisp{dualentry}{secondary} terms identified with the
\atentry{dualindexnumber} type. The definition provided in the \ext{glstex}
file is:
\begin{codeenv}
\cs{providecommand}\marg{\cs{bibglsnewdualindexnumbersecondary}}[5]\marg{\comment{}
 \cs{longnewglossaryentry*}\marg{\idx{param}1}\marg{\field{name}=\marg{\idx{param}3},\field{category}=\marg{number},\field{symbol}=\marg{\idx{param}4},\idx{param}2}\marg{\idx{param}5}\comment{}
}
\end{codeenv}

\cssection{bibglsnewdualindexabbreviation}

\formatdef{bibglsnewdualindexabbreviation}
This command is used to define \primary\ terms identified with the
\atentry{dualindexabbreviation} type. The default definition provided in 
the \ext{glstex} file is:
\begin{codeenv}
\cs{providecommand}\marg{\gls{bibglsnewdualindexabbreviation}}[7]\marg{\comment{}
 \cs{longnewglossaryentry*}\marg{\idx{param}1}\marg{\comment{}
   \field{name}=\marg{\cs{protect}\gls{bibglsuseabbrvfont}\marg{\idx{param}4}\marg{\cs{glscategory}\marg{\idx{param}2}}},\comment{}
   \field{category}=\marg{index},\idx{param}3}\marg{}\comment{}
}
\end{codeenv}
In this case \meta{dual-label} is the \gls{dualentry}['s] label, which is used 
to fetch the category label in \gls{bibglsuseabbrvfont}. (The
\field{category} field for the dual isn't used since a custom
definition of
\gls!{bibglsnewdualindexabbreviationsecondary} may override the
value known to \bibgls.)

Note that (as shown above) with the default 
\csopt[short]{abbreviation-name-fallback} the \field{name} uses:
\nosecformatdef{bibglsuseabbrvfont}
to format the name, which ensures that it uses the same font as the
short form for the \dual\ abbreviation. This will use
\ics{glsuseabbrvfont} if it's defined otherwise it will be defined
to replicate that command. If \csopt{abbreviation-name-fallback} is
set to some other field then the \field{name} uses:
\nosecformatdef{bibglsuselongfont}
instead, which ensures that it uses the same font as the
long form for the \dual\ abbreviation.

\cssection{bibglsnewdualindexabbreviationsecondary}

\formatdef{bibglsnewdualindexabbreviationsecondary}
This command is used to define \glsdisp{dualentry}{secondary} terms identified with the
\atentry{dualindexabbreviation} \gls{entrytype}.
The definition provided in the \ext{glstex} file is:
\begin{codeenv}
\cs{providecommand}\marg{\gls{bibglsnewdualindexabbreviationsecondary}}[6]\marg{\comment{}
  \cs{ifstrempty}\marg{\idx{param}6}\comment{}
  \marg{\gls{newabbreviation}\oarg{\idx{param}2}\marg{\idx{param}1}\marg{\idx{param}4}\marg{\idx{param}5}}\comment{}
  \marg{\gls{newabbreviation}\oarg{\idx{param}2,\field{description}=\marg{\idx{param}6}}\marg{\idx{param}1}\marg{\idx{param}4}\marg{\idx{param}5}}\comment{}
}
\end{codeenv}
This ensures that a missing or empty \field{description} doesn't
interfere with the abbreviation style.

\cssection{bibglsnewdualabbreviationentry}

\formatdef{bibglsnewdualabbreviationentry}
This command is used to define \primary\ terms identified with the
\atentry{dualabbreviationentry} type. The definition provided in the \ext{glstex}
file is:
\begin{codeenv}
\cs{providecommand}\marg{\gls{bibglsnewdualabbreviationentry}}[5]\marg{\comment{}
  \gls{newabbreviation}\oarg{\idx{param}2}\marg{\idx{param}1}\marg{\idx{param}3}\marg{\idx{param}4}\comment{}
}
\end{codeenv}
Note that this definition ignores the \meta{description} argument.

\cssection{bibglsnewdualabbreviationentrysecondary}

\formatdef{bibglsnewdualabbreviationentrysecondary}
This command is used to define \glsdisp{dualentry}{secondary} terms identified with the
\atentry{dualabbreviationentry} type. The definition provided in the \ext{glstex}
file is:
\begin{codeenv}
\cs{providecommand}\marg{\gls{bibglsnewdualabbreviationentrysecondary}}[5]\marg{\comment{}
  \cs{longnewglossaryentry*}\marg{\idx{param}1}\marg{\idx{param}2}\marg{\idx{param}5}\comment{}
}
\end{codeenv}
Note that this definition ignores the \meta{short} and \meta{long}
arguments (which will typically be empty unless the default mappings
are changed).

\cssection{bibglsnewdualentryabbreviation}

\formatdef{bibglsnewdualentryabbreviation}
This command is used to define \primary\ terms identified with the
(now deprecated) \gls{entrytype}
\atentry{dualentryabbreviation}. The definition provided in the \ext{glstex}
file is:
\begin{codeenv}
\cs{providecommand}\marg{\gls{bibglsnewdualentryabbreviation}}[5]\marg{\comment{}
  \gls{newabbreviation}\oarg{\idx{param}2}\marg{\idx{param}1}\marg{\idx{param}3}\marg{\idx{param}4}\comment{}
}
\end{codeenv}
Note that this definition ignores the \meta{description} argument.

\cssection{bibglsnewdualentryabbreviationsecondary}

\formatdef{bibglsnewdualentryabbreviationsecondary}
This command is used to define \glsdisp{dualentry}{secondary} terms identified with the
(now deprecated) \gls{entrytype}
\atentry{dualentryabbreviation}. The definition provided in the \ext{glstex}
file is:
\begin{codeenv}
\cs{providecommand}\marg{\gls{bibglsnewdualentryabbreviationsecondary}}[5]\marg{\comment{}
  \cs{longnewglossaryentry*}\marg{\idx{param}1}\marg{\idx{param}2}\marg{\idx{param}5}\comment{}
}
\end{codeenv}
Note that this definition ignores the \meta{short} and \meta{long}
arguments (which will typically be empty unless the default mappings
are changed).

\cssection{bibglsnewdualsymbol}

\formatdef{bibglsnewdualsymbol}
This command is used to define terms identified with the
\atentry{dualsymbol} type. The definition provided in the \ext{glstex}
file is:
\begin{codeenv}
\cs{providecommand}\marg{\gls{bibglsnewdualsymbol}}[4]\marg{\comment{}
 \cs{longnewglossaryentry*}\marg{\idx{param}1}\marg{\field{name}=\marg{\idx{param}3},\field{sort}=\marg{\idx{param}1},\field{category}=\marg{symbol},\idx{param}2}\marg{\idx{param}4}}
\end{codeenv}

\cssection{bibglsnewdualnumber}

\formatdef{bibglsnewdualnumber}
This command is used to define terms identified with the
\atentry{dualnumber} type. The definition provided in the \ext{glstex}
file is:
\begin{codeenv}
\cs{providecommand}\marg{\gls{bibglsnewdualnumber}}[4]\marg{\comment{}
 \cs{longnewglossaryentry*}\marg{\idx{param}1}\marg{\field{name}=\marg{\idx{param}3},\field{sort}=\marg{\idx{param}1},\field{category}=\marg{symbol},\idx{param}2}\marg{\idx{param}4}}
\end{codeenv}

\cssection{bibglsnewdualabbreviation}

\formatdef{bibglsnewdualabbreviation}
This command is used to define terms identified with the
\atentry{dualabbreviation} type where the \field{duallong} field
is swapped with the \field{long} field and the \field{dualshort}
field is swapped with the \field{short} field. The definition provided in the
\ext{glstex} file is:
\begin{codeenv}
\cs{providecommand}\marg{\gls{bibglsnewdualabbreviation}}[4]\marg{\comment{}
  \gls{newabbreviation}\oarg{\idx{param}2}\marg{\idx{param}1}\marg{\idx{param}3}\marg{\idx{param}4}\comment{}
}
\end{codeenv}

\cssection{bibglsnewdualacronym}

\formatdef{bibglsnewdualacronym}
This command is used to define terms identified with the
\atentry{dualacronym} type. The definition provided in the \ext{glstex}
file is:
\begin{codeenv}
\cs{providecommand}\marg{\gls{bibglsnewdualacronym}}[4]\marg{\comment{}
  \gls{newacronym}\oarg{\idx{param}2}\marg{\idx{param}1}\marg{\idx{param}3}\marg{\idx{param}4}\comment{}
}
\end{codeenv}
This works in much the same way as \gls{bibglsnewdualabbreviation}.
Remember that with the \styfmt{glossaries-extra} package \gls{newacronym}
is redefined to just use \gls{newabbreviation} with the default \field{type}
set to \ics{acronymtype} and the default \field{category} set to
\code{acronym}.

\cssection{bibglsnewtertiaryindexabbreviationentry}

\formatdef{bibglsnewtertiaryindexabbreviationentry}
This is used to define \primary\ terms identified with the
\atentry{tertiaryindexabbreviationentry} type. It's essentially
the same as \gls!{bibglsnewdualindexabbreviation}.
The definition provided in the \ext{glstex} file is:
\begin{codeenv}
\cs{providecommand}\marg{\gls{bibglsnewtertiaryindexabbreviationentry}}[7]\marg{\comment{}
  \cs{longnewglossaryentry*}\marg{\idx{param}1}\marg{\comment{}
      \field{name}=\marg{\cs{protect}\gls{bibglsuseabbrvfont}\marg{\idx{param}4}\marg{\cs{glscategory}\marg{\idx{param}2}}},\comment{}
      \field{category}=\marg{index},\idx{param}3}{}\comment{}
}
\end{codeenv}

\cssection{bibglsnewtertiaryindexabbreviationentrysecondary}

\formatdef{bibglsnewtertiaryindexabbreviationentrysecondary}
This command is used to define both the
\glsdisp{secondaryentry}{secondary} and
\glsdisp{tertiaryentry}{tertiary} terms identified with the
\atentry{tertiaryindexabbreviationentry} type. The
\glsdisp{secondaryentry}{secondary} term is
an abbreviation and the \glsdisp{tertiaryentry}{tertiary} term is a regular entry.  The
definition written to the \ext{glstex} file is:
\begin{codeenv}
\cs{providecommand}\marg{\gls{bibglsnewtertiaryindexabbreviationentrysecondary}}[8]\marg{\comment{}
  \gls{newabbreviation}\oarg{\idx{param}3}\marg{\idx{param}1}\marg{\idx{param}6}\marg{\idx{param}7}\comment{}
  \cs{longnewglossaryentry*}\marg{\idx{param}2}\comment{}
  \marg{\field{name}=\marg{\cs{protect}\gls{bibglsuselongfont}\marg{\idx{param}7}\marg{\cs{glscategory}\marg{\idx{param}1}}},\idx{param}4}\comment{}
  \marg{\idx{param}8}\comment{}
}
\end{codeenv}
The \meta{label} is the label for the
\glsdisp{secondaryentry}{secondary} (abbreviation) entry and
\meta{tertiary-label} is the label for the
\glsdisp{tertiaryentry}{tertiary} (regular) entry. The fifth
argument (\meta{primary name}) isn't used but is provided if
required for a custom redefinition.  The \field{name} field for the
\glsdisp{tertiaryentry}{tertiary} is obtained from the \meta{long}
argument encapsulated by \gls{bibglsuselongfont} to format the name,
which ensures that it uses the same font as the long form for the
dual abbreviation. This will use \ics{glsuselongfont} if it's
defined otherwise it will be defined to replicate that command.

\cssection{bibglsnewbibtexentry}

\formatdef{bibglsnewbibtexentry}
This command is used to define the \glsdisp{mainentry}{main term} identified with
\atentry{bibtexentry}.
The definition written to the \ext{glstex} file is:
\begin{codeenv}
\cs{providecommand}\marg{\gls{bibglsnewbibtexentry}}[4]\marg{\comment{}
 \cs{longnewglossaryentry*}\marg{\idx{param}1}\marg{\field{name}=\marg{\idx{param}3},\idx{param}2}\marg{\idx{param}4}\comment{}
}
\end{codeenv}

\cssection{bibglsnewcontributor}

\formatdef{bibglsnewcontributor}
This command is used to define terms identified with
\atentry{contributor} (typically implicitly created through
\atentry{bibtexentry}).
The definition written to the \ext{glstex} file is:
\begin{codeenv}
\cs{providecommand}\marg{\gls{bibglsnewcontributor}}[4]\marg{\comment{}
 \cs{longnewglossaryentry*}\marg{\idx{param}1}\marg{\field{name}=\marg{\idx{param}3},\idx{param}2}\marg{\idx{param}4}\comment{}
}
\end{codeenv}

\cssection{bibglsnewprogenitor}

\formatdef{bibglsnewprogenitor}
This command is used to define the \glsdisp{mainentry}{main terms} created by
\atentry{progenitor}. The definition is written to the \ext{glstex}
file as:
\begin{codeenv}
\cs{providecommand}\marg{\gls{bibglsnewprogenitor}}[4]\marg{\comment{}
 \cs{longnewglossaryentry*}\marg{\idx{param}1}\marg{\field{name}=\marg{\idx{param}3},\idx{param}2}\marg{\idx{param}4}\comment{}
}
\end{codeenv}

\cssection{bibglsnewspawnindex}

\formatdef{bibglsnewspawnindex}
This command is used to define the \glsdisp{mainentry}{main terms} created by
\atentry{spawnindex}. The definition is written to the \ext{glstex}
file as:
\begin{codeenv}
\cs{providecommand}\marg{\gls{bibglsnewspawnindex}}[4]\marg{\comment{}
 \cs{longnewglossaryentry*}\marg{\idx{param}1}\marg{\field{name}=\marg{\idx{param}3},\idx{param}2}\marg{\idx{param}4}\comment{}
}
\end{codeenv}

\cssection{bibglsnewspawnedindex}

\formatdef{bibglsnewspawnedindex}
This command is used to define the \glsdisp{spawnedentry}{terms spawned} by
\atentry{progenitor} or \atentry{spawnindex}. The definition is written 
to the \ext{glstex} file as:
\begin{codeenv}
\cs{providecommand}\marg{\gls{bibglsnewspawnedindex}}[2]\marg{\comment{}
 \gls{newglossaryentry}\marg{\idx{param}1}\marg{\field{name}=\marg{\idx{param}1},\field{category}={index},\field{description}=\marg{},\idx{param}2}\comment{}
}
\end{codeenv}

\cssection{bibglsnewspawnindexplural}

\formatdef{bibglsnewspawnindexplural}
This command is used to define the \glsdisp{mainentry}{main terms} created by
\atentry{spawnindexplural}. The definition is written to the \ext{glstex}
file as:
\begin{codeenv}
\cs{providecommand}\marg{\gls{bibglsnewspawnindexplural}}[4]\marg{\comment{}
 \cs{longnewglossaryentry*}\marg{\idx{param}1}\marg{\field{name}=\marg{\idx{param}3},\idx{param}2}\marg{\idx{param}4}\comment{}
}
\end{codeenv}

\cssection{bibglsnewspawnedindexplural}

\formatdef{bibglsnewspawnedindexplural}
This command is used to define the \glsdisp{spawnedentry}{terms spawned} by
\atentry{spawnindexplural}. 
The definition is written to the \ext{glstex} file as:
\begin{codeenv}
\cs{providecommand}\marg{\gls{bibglsnewspawnedindexplural}}[3]\marg{\comment{}
 \gls{newglossaryentry}\marg{\idx{param}1}\marg{\field{name}=\marg{\idx{param}3},\field{category}=\marg{indexplural},\field{description}=\marg{},\idx{param}2}\comment{}
}
\end{codeenv}

\cssection{bibglsnewspawnentry}

\formatdef{bibglsnewspawnentry}
This command is used to define the \glsdisp{mainentry}{main terms} created by
\atentry{spawnentry}. The definition is written to the \ext{glstex}
file as:
\begin{codeenv}
\cs{providecommand}\marg{\gls{bibglsnewspawnentry}}[4]\marg{\comment{}
 \cs{longnewglossaryentry*}\marg{\idx{param}1}\marg{\field{name}=\marg{\idx{param}3},\idx{param}2}\marg{\idx{param}4}\comment{}
}
\end{codeenv}

\cssection{bibglsnewspawnedentry}

\formatdef{bibglsnewspawnedentry}
This command is used to define the \glsdisp{spawnedentry}{terms spawned} by \atentry{spawnentry}. 
The definition is written to the \ext{glstex} file as:
\begin{codeenv}
\cs{providecommand}\marg{\gls{bibglsnewspawnedentry}}[4]\marg{\comment{}
 \cs{longnewglossaryentry*}\marg{\idx{param}1}\marg{\field{name}=\marg{\idx{param}3},\idx{param}2}\marg{\idx{param}4}\comment{}
}
\end{codeenv}

\cssection{bibglsnewspawnabbreviation}

\formatdef{bibglsnewspawnabbreviation}
This command is used to define the \glsdisp{mainentry}{main terms} created by
\atentry{spawnabbreviation}. The definition is written to the \ext{glstex}
file as:
\begin{codeenv}
\cs{providecommand}\marg{\gls{bibglsnewspawnabbreviation}}[4]\marg{\comment{}
  \gls{newabbreviation}\oarg{\idx{param}2}\marg{\idx{param}1}\marg{\idx{param}3}\marg{\idx{param}4}\comment{}
}
\end{codeenv}

\cssection{bibglsnewspawnedabbreviation}

\formatdef{bibglsnewspawnedabbreviation}
This command is used to define the \glsdisp{spawnedentry}{terms spawned} by \atentry{spawnabbreviation}. 
The definition is written to the \ext{glstex} file as:
\begin{codeenv}
\cs{providecommand}\marg{\gls{bibglsnewspawnedabbreviation}}[4]\marg{\comment{}
  \gls{newabbreviation}\oarg{\idx{param}2}\marg{\idx{param}1}\marg{\idx{param}3}\marg{\idx{param}4}\comment{}
}
\end{codeenv}

\cssection{bibglsnewspawnacronym}

\formatdef{bibglsnewspawnacronym}
This command is used to define the \glsdisp{mainentry}{main terms} created by
\atentry{spawnacronym}. The definition is written to the \ext{glstex}
file as:
\begin{codeenv}
\cs{providecommand}\marg{\gls{bibglsnewspawnacronym}}[4]\marg{\comment{}
  \gls{newacronym}\oarg{\idx{param}2}\marg{\idx{param}1}\marg{\idx{param}3}\marg{\idx{param}4}\comment{}
}
\end{codeenv}

\cssection{bibglsnewspawnedacronym}

\formatdef{bibglsnewspawnedacronym}
This command is used to define the \glsdisp{spawnedentry}{terms spawned} by \atentry{spawnacronym}. 
The definition is written to the \ext{glstex} file as:
\begin{codeenv}
\cs{providecommand}\marg{\gls{bibglsnewspawnedacronym}}[4]\marg{\comment{}
  \gls{newacronym}\oarg{\idx{param}2}\marg{\idx{param}1}\marg{\idx{param}3}\marg{\idx{param}4}\comment{}
}
\end{codeenv}

\cssection{bibglsnewspawnsymbol}

\formatdef{bibglsnewspawnsymbol}
This command is used to define the \glsdisp{mainentry}{main terms} created by
\atentry{spawnsymbol}. The definition is written to the \ext{glstex}
file as:
\begin{codeenv}
\cs{providecommand}\marg{\gls{bibglsnewspawnsymbol}}[4]\marg{\comment{}
 \cs{longnewglossaryentry*}\marg{\idx{param}1}\marg{\field{name}=\marg{\idx{param}3},\idx{param}2}\marg{\idx{param}4}\comment{}
}
\end{codeenv}

\cssection{bibglsnewspawnedsymbol}

\formatdef{bibglsnewspawnedsymbol}
This command is used to define the \glsdisp{spawnedentry}{terms spawned} by \atentry{spawnsymbol}. 
The definition is written to the \ext{glstex} file as:
\begin{codeenv}
\cs{providecommand}\marg{\gls{bibglsnewspawnedsymbol}}[4]\marg{\comment{}
 \cs{longnewglossaryentry*}\marg{\idx{param}1}\marg{\field{name}=\marg{\idx{param}3},\field{sort}=\marg{\idx{param}1},\field{category}=\marg{spawnedsymbol},\idx{param}2}\marg{\idx{param}4}}
\end{codeenv}

\cssection{bibglsnewspawnnumber}

\formatdef{bibglsnewspawnnumber}
This command is used to define the \glsdisp{mainentry}{main terms} created by
\atentry{spawnnumber}. The definition is written to the \ext{glstex}
file as:
\begin{codeenv}
\cs{providecommand}\marg{\gls{bibglsnewspawnnumber}}[4]\marg{\comment{}
 \cs{longnewglossaryentry*}\marg{\idx{param}1}\marg{\field{name}=\marg{\idx{param}3},\idx{param}2}\marg{\idx{param}4}\comment{}
}
\end{codeenv}

\cssection{bibglsnewspawnednumber}

\formatdef{bibglsnewspawnednumber}
This command is used to define the \glsdisp{spawnedentry}{terms spawned} by \atentry{spawnnumber}. 
The definition is written to the \ext{glstex} file as:
\begin{codeenv}
\cs{providecommand}\marg{\gls{bibglsnewspawnednumber}}[4]\marg{\comment{}
 \cs{longnewglossaryentry*}\marg{\idx{param}1}\marg{\field{name}=\marg{\idx{param}3},\field{sort}=\marg{\idx{param}1},\field{category}=\marg{spawnednumber},\idx{param}2}\marg{\idx{param}4}}
\end{codeenv}

\cssection{bibglsnewspawndualindexentry}

\formatdef{bibglsnewspawndualindexentry}
This command is used to define the \gls{progenitor}['s] \primary\ term created by 
\atentry{spawndualindexentry}. 
The definition is written to the \ext{glstex} file as:
\begin{codeenv}
\cs{providecommand}\marg{\gls{bibglsnewspawndualindexentry}}[4]\marg{\comment{}
 \cs{longnewglossaryentry*}\marg{\idx{param}1}\marg{\field{name}=\marg{\idx{param}3},\field{category}=\marg{index},\idx{param}2}{}\comment{}
}
\end{codeenv}
The \meta{description} argument is ignored.

\cssection{bibglsnewspawndualindexentrysecondary}

\formatdef{bibglsnewspawndualindexentrysecondary}
This command is used to define the \gls{progenitor}['s]
\glsdisp{dualentry}{secondary} (dual) term created by 
\atentry{spawndualindexentry}. 
The definition is written to the \ext{glstex} file as:
\begin{codeenv}
\cs{providecommand}\marg{\gls{bibglsnewspawndualindexentrysecondary}}[4]\marg{\comment{}
 \cs{longnewglossaryentry*}\marg{\idx{param}1}\marg{\field{name}=\marg{\idx{param}3},\idx{param}2}\marg{\idx{param}4}\comment{}
}
\end{codeenv}

\section{Compound Entry Sets}
\label{sec:compounddefs}

\cssection{bibglsdefcompoundset}

\formatdef{bibglsdefcompoundset}

The default definition uses \ics{multiglossaryentry}.

\section{Location Lists and Cross-References}
\label{sec:loclistdefs}

These commands deal with the way the \iglspl{locationlist} and 
cross references are formatted. The commands typically aren't used until
the entry information is displayed in the glossary, so you may
redefine these commands after the resource file has been loaded.

\cssection{bibglsseesep}

\formatdef{bibglsseesep}
Any entries that provide a \field{see} field (and that field hasn't
be omitted from the \gls{locationlist} with \csopt[omit]{see}) will
have \gls{bibglsseesep} inserted between the \field{see} part and the
\gls{locationlist} (unless there are no \glspl{location}, in which case just
the \field{see} part is displayed without \gls{bibglsseesep}).

This command is provided with:
\begin{codeenv}
\cs{providecommand}\marg{\gls{bibglsseesep}}\marg{, }
\end{codeenv}
You can define this before you load the \ext{bib} file:
\begin{codeenv}
\cs{newcommand}\marg{\gls{bibglsseesep}}\marg{; }
\gls{GlsXtrLoadResources}\oarg{\csopt[entries]{src}}
\end{codeenv}
Or you can redefine it afterwards:
\begin{codeenv}
\gls{GlsXtrLoadResources}\oarg{\csopt[entries]{src}}
\cs{glsrenewcommand}\marg{\gls{bibglsseesep}}\marg{; }
\end{codeenv}

\cssection{bibglsseealsosep}

\formatdef{bibglsseealsosep}
This is like \gls{bibglsseesep} but is used with cross-reference
lists provided with the \field{seealso} field, if supported.

\cssection{bibglsaliassep}

\formatdef{bibglsaliassep}
This is like \gls{bibglsseesep} but is used with cross-reference
lists provided with the \field{alias} field.

\cssection{bibglsusesee}

\formatdef{bibglsusesee}
Displays the formatted cross-reference list stored in the
\field{see} field for the given entry. This just defaults to 
\ics{glsxtrusesee}\margm{label}.

\cssection{bibglsuseseealso}

\formatdef{bibglsuseseealso}
Displays the formatted cross-reference list stored in the
\field{seealso} field for the given entry. This just defaults to 
\ics{glsxtruseseealso}\margm{label}.

\cssection{bibglsusealias}

\formatdef{bibglsusealias}
Displays the formatted cross-reference stored in the
\field{alias} field for the given entry. This is defined to use
\ics{glsseeformat}.

\cssection{bibglsdelimN}

\formatdef{bibglsdelimN}
Separator between individual \glspl{location}, except for the last.
This defaults to \ics{delimN}.

\cssection{bibglslastDelimN}

\formatdef{bibglslastDelimN}
Separator between penultimate and final individual \glspl{location}.
This defaults to \code{,\idx{nbspchar}} to discourage lonely \glspl{location}.

\cssection{bibglscompact}

\formatdef{bibglscompact}
This command is used with \csopt{compact-ranges} when the end
\gls{location} in a \idx{range} is compacted.
The first argument \meta{pattern} indicates the \gls{location} pattern:
\code{digit} for digits, \code{roman} for \idx!{lowercase}
Roman numerals, \code{ROMAN} for \idx!{uppercase} Roman
numerals and \code{alpha} for alphabetical \glspl{location}. The actual
\gls{location} is split into two parts, \meta{part1} and \meta{part2}.
The string concatenation \meta{part1}\meta{part2} forms the
actual \gls{location}.

This just does \meta{part2} by default.

\cssection{bibglspassim}

\formatdef{bibglspassim}
If \csopt{max-loc-diff} is greater than~1, then any
\idxpl{implicit-range} that have skipped over gaps will be followed
by \gls{bibglspassim}, which is defined as:
\begin{codeenv}
\cs{providecommand}\marg{\gls{bibglspassim}}\marg{ \gls{bibglspassimname}}
\end{codeenv}
You can define this before you load the \ext{bib} file:
\begin{codeenv}
\cs{newcommand}\marg{\gls{bibglspassim}}\marg{}
\gls{GlsXtrLoadResources}\oarg{\csopt[entries]{src}}
\end{codeenv}
Or you can redefine it afterwards:
\begin{codeenv}
\gls{GlsXtrLoadResources}\oarg{\csopt[entries]{src}}
\cs{glsrenewcommand}\marg{\gls{bibglspassim}}\marg{}
\end{codeenv}

\cssection{bibglspassimname}

\formatdef{bibglspassimname}
The default definition is obtained from the \langxml. For example, with 
\file{bib2gls-en.xml} the provided definition is:
\begin{codeenv}
\cs{providecommand}\marg{\gls{bibglspassimname}}\marg{passim}
\end{codeenv}

\cssection{bibglsrange}

\formatdef{bibglsrange}
\Idxpl{explicit-range} formed using \glsopt[\idx{openrange}]{format} and
\glsopt[\idx{closerange}]{format} or \glsopt[\idx{openrange}\meta{csname}]{format} and
\glsopt[\idx{closerange}\meta{csname}]{format} (where \meta{csname} matches and is a
text-block command without the initial backslash) in the optional
argument of commands like \ics{gls} or \ics{glsadd} are encapsulated within
the argument of \gls{bibglsrange}. By default, this simply does its
argument. This command is not used with \idxpl{implicit-range} that
are formed by collating consecutive \glspl{location} or when
\csopt[true]{merge-ranges} is used.

\cssection{bibglsinterloper}

\formatdef{bibglsinterloper}
If an \idxpl{explicit-range} conflicts with a record, a warning will be
issued and the conflicting record (the \idx{interloper}) will be shifted to the front
of the \idx{range} inside the argument of \gls{bibglsinterloper}.
The default definition just does \meta{location}\gls!{bibglsdelimN}
so that it fits neatly into the list.

For example, suppose on page~4 of my document I start a \idx{range} with:
\begin{codeenv*}
\cs{glsadd}\oarg{\glsaddopt[\idx{openrange}]{format}}\marg{sample}
\end{codeenv*}
and end it on page~9 with:
\begin{codeenv*}
\cs{glsadd}\oarg{\glsaddopt[\idx{closerange}]{format}}\marg{sample}
\end{codeenv*}
This forms an \idx{explicit-range}, but let's suppose on page~6 I
have:
\begin{codeenv}
\cs{gls}\oarg{\glsopt[\encap{hyperbf}]{format}}\marg{sample}
\end{codeenv}
This record conflicts with the \idx{explicit-range} (which doesn't include
\encap{hyperbf} in the format). This causes a warning and
the conflicting entry will be moved before the start of the
\idx{explicit-range} resulting in \textbf{6}, 4--9.

Note that \idxpl{implicit-range} can't be formed from \idxpl{interloper} (nor can
\idxpl{implicit-range} be merged with explicit ones with the default
\csopt[false]{merge-ranges}),
so if \code{\cs{gls}\oarg{\glsopt[\encap{hyperbf}]{format}}\marg{sample}} 
also occurs on pages~7 and~8 then the result will be \textbf{6}, \textbf{7},
\textbf{8}, 4--9. Either remove the \idx{explicit-range} or
remove the conflicting entries. (Alternatively, redefine
\gls{bibglsinterloper} to ignore its argument, which will
discard the conflicting entries.)

\cssection{bibglspostlocprefix}

\formatdef{bibglspostlocprefix}
If the \csopt{loc-prefix} option is on, \gls!{bibglslocprefix} will
be inserted at the start of \glspl{locationlist}, and its default
definition includes \gls{bibglspostlocprefix}
placed after the prefix text. This command is provided with:
\begin{codeenv}
\cs{providecommand}\marg{\gls{bibglspostlocprefix}}\marg{\cs{cs.space}}
\end{codeenv}
which puts a space between the prefix text and the \gls{locationlist}.
You can define this before you load the \ext{bib} file:
\begin{codeenv}
\cs{newcommand}\marg{\gls{bibglspostlocprefix}}\marg{: }
\gls{GlsXtrLoadResources}\oarg{\csopt[entries]{src},\csopt{loc-prefix}}
\end{codeenv}
Or you can redefine it afterwards:
\begin{codeenv}
\gls{GlsXtrLoadResources}\oarg{\csopt[entries]{src},\csopt{loc-prefix}}
\cs{glsrenewcommand}\marg{\gls{bibglspostlocprefix}}\marg{: }
\end{codeenv}

\cssection{bibglslocprefix}

\formatdef{bibglslocprefix}
If the \csopt{loc-prefix} option is on, this command will be
provided. The location of the definition is determined by the
\csopt{loc-prefix-def} option.  For example, if the document has:
\begin{codeenv}
\gls{GlsXtrLoadResources}\oarg{\csopt[main]{type},\csopt[individual]{loc-prefix-def},\csopt[p.,pp.]{loc-prefix},\csopt[entries]{src}}
\end{codeenv}
then the following will be added to the \ext{glstex} file:
\begin{codeenv}
\cs{apptoglossarypreamble}\oarg{main}\marg{\comment{}
 \cs{providecommand}\marg{\gls{bibglslocprefix}}[1]\marg{\comment{}
  \cs{ifcase}\idx{param}\idx{param}1
  \cmd{or} p.\gls{bibglspostlocprefix}
  \cmd{else} pp.\gls{bibglspostlocprefix}
  \cmd{fi}
 }\comment{}
}
\end{codeenv}
However, if the \csopt{type} key is missing or if the option
\csopt[local]{loc-prefix-def} is used, then the following will
be added instead:
\begin{codeenv}
\cs{appto}\cs{glossarypreamble}\marg{\comment{}
 \cs{providecommand}\marg{\gls{bibglslocprefix}}[1]\marg{\comment{}
  \cs{ifcase}\idx{param}1
  \cmd{or} p.\gls{bibglspostlocprefix}
  \cmd{else} pp.\gls{bibglspostlocprefix}
  \cmd{fi}
 }\comment{}
}
\end{codeenv}
If \csopt[global]{loc-prefix-def} is used then the definition is
global (outside of the glossary preamble).

\cssection{bibglspagename}

\formatdef{bibglspagename}
If \csopt[true]{loc-prefix} is used, then this command is provided
using the value of \idx{tag.page} from the \langxml. For example with \file{bib2gls-en.xml} the definition is:
\begin{codeenv}
\cs{providecommand}\marg{\gls{bibglspagename}}\marg{Page}
\end{codeenv}

\cssection{bibglspagesname}

\formatdef{bibglspagesname}
If \csopt[true]{loc-prefix} is used, then this command is provided
using the value of \idx{tag.pages} from the \langxml. For example with \file{bib2gls-en.xml} the definition is:
\begin{codeenv}
\cs{providecommand}\marg{\gls{bibglspagesname}}\marg{Pages}
\end{codeenv}

\cssection{bibglslocsuffix}

\formatdef{bibglslocsuffix}
If the \csopt{loc-suffix} option is on, this command will be
provided. The location of the definition depends on the
\csopt{loc-suffix-def} option.

This command's definition depends on the value provided by
\csopt{loc-suffix}. For example, with 
\csopt[\ics{at}{\idx{periodchar}}]{loc-suffix}
the command is defined as:
\begin{codeenv}
\cs{providecommand}\marg{\gls{bibglslocsuffix}}[1]\marg{\ics{at}.}
\end{codeenv}
(which ignores the argument).

Whereas with \csopt[\meta{A}\dcomma\meta{B}\dcomma\meta{C}]{loc-suffix}
the command is defined as:
\begin{codeenv}
\cs{providecommand}\marg{\gls{bibglslocsuffix}}[1]\marg{\cs{ifcase}\idx{param}1 \meta{A}\cmd{or} \meta{B}\cmd{else} \meta{C}\cmd{fi}}
\end{codeenv}

Note that this is slightly different from \gls{bibglslocprefix} as
it includes the 0 case, which in this instance means that there were
no \glspl{location} but there was a cross-reference. This command isn't
added when the \gls{locationlist} is empty.

\cssection{bibglslocationgroup}

\formatdef{bibglslocationgroup}
When the \csopt{loc-counters} option is used, the \glspl{location}
for each entry are grouped together according to the counter
(in the order specified in the value of \csopt{loc-counters}).
Each group of \glspl{location} is encapsulated within
\gls{bibglslocationgroup}, where \meta{n} is the number
of \glspl{location} within the group, \meta{counter} is the
counter name and \meta{list} is the formatted \gls{location} sub-list.
By default, this simply does \meta{list}, but may be
defined (before the resources are loaded) or redefined 
(after the resources are loaded) as required.

For example:
\begin{codeenv}
\cs{newcommand}*\marg{\gls{bibglslocationgroup}}[3]\marg{\comment{}
  \cs{ifnum}\idx{param}1=1
   \idx{param}2:
  \cmd{else}
   \idx{param}2s:
  \cmd{fi}
  \idx{param}3\comment{}
}
\strut
\gls{GlsXtrLoadResources}\oarg{
  \csopt[equation,page]{loc-counters},\comment{group locations by counter}
  \csopt[entries]{src}\comment{data in entries.bib}
}
\end{codeenv}
This will prefix each group with the counter name, if there's
only one \gls{location}, or the counter name followed by \qt{s},
if there are multiple \glspl{location} within the group.

There are various ways to adapt this to translate the counter
name to a different textual label, such as:
\begin{codeenv}
\cs{providecommand}\marg{\cmd{pagename}}\marg{Page}
\cs{providecommand}\marg{\cmd{pagesname}}\marg{Pages}
\cs{providecommand}\marg{\cmd{equationname}}\marg{Equation}
\cs{providecommand}\marg{\cmd{equationsname}}\marg{Equations}
\strut
\cs{newcommand}*\marg{\gls{bibglslocationgroup}}[3]\marg{\comment{}
  \ics{ifnum}\idx{param}1=1
   \ics{ifcsdef}\marg{\idx{param}2name}\marg{\cs{csuse}\marg{\idx{param}2name}}\marg{\idx{param}2}:
  \cmd{else}
   \cs{ifcsdef}\marg{\idx{param}2sname}\marg{\cs{csuse}\marg{\idx{param}2sname}}\marg{\idx{param}2s}:
  \cmd{fi}
  \idx{param}3\comment{}
}
\end{codeenv}

\cssection{bibglslocationgroupsep}

\formatdef{bibglslocationgroupsep}
When the \csopt{loc-counters} option is set, this command
is used to separate each \gls{location} sub-group. It may be defined
before the resources are loaded:
\begin{codeenv}
\cs{newcommand}*\marg{\gls{bibglslocationgroupsep}}\marg{; }
\strut
\gls{GlsXtrLoadResources}\oarg{
  \csopt[equation,page]{loc-counters},\comment{group locations by counter}
  \csopt[entries]{src}\comment{data in entries.bib}
}
\end{codeenv}
or redefined after the resources are loaded:
\begin{codeenv}
\gls{GlsXtrLoadResources}\oarg{
  \csopt[equation,page]{loc-counters},\comment{group locations by counter}
  \csopt[entries]{src}\comment{data in entries.bib}
}
\strut
\cs{glsrenewcommand}*\marg{\gls{bibglslocationgroupsep}}\marg{; }
\end{codeenv}

\cssection{bibglsprimary}

\formatdef{bibglsprimary}
When the \csopt{save-principal-locations} option is used, the 
\glspl{principallocation} are stored in the \field{primarylocations} field
encapsulated with this command, unless the locations are split into
groups according to the location counter.

The first argument is the number of \glspl{location} in the list.
The second argument is the list of \glspl{location} formatted in the
usual way. The default definition is to ignore the first argument
and simply do the second.

If the locations are split into groups, then \gls{bibglsprimarylocationgroup}
is used for each group instead (separated with
\gls{bibglsprimarylocationgroupsep}).

\cssection{bibglsprimarylocationgroup}

\formatdef{bibglsprimarylocationgroup}
As \gls{bibglslocationgroup} but for primary group
locations.

\cssection{bibglsprimarylocationgroupsep}

\formatdef{bibglsprimarylocationgroupsep}
As \gls{bibglslocationgroupsep} but for primary group locations.

\cssection{bibglssupplemental}

\formatdef{bibglssupplemental}
When the \csopt{supplemental-locations} option is used, the
\glspl{location} from a \supplementarydocument\ are encapsulated
within the \meta{list} part of \gls{bibglssupplemental}. The first
argument \meta{n} (ignored by default) is the number of
\supplementarylocations.

If multiple \supplemental\ sources are permitted (that is, 
\bibgls\ has detected that the document is using at least
version 1.36 of \sty{glossaries-extra}), then the \meta{list}
part will consist of sub-lists for each external source. In this
case, \meta{n} will be the total number of elements across
all the sub-lists.

\cssection{bibglssupplementalsublist}

\formatdef{bibglssupplementalsublist}
If multiple \supplemental\ sources are permitted, this will be used
to format each sub-list, where \meta{n} (ignored by default)
is the number of elements in the sub-list, \meta{external document}
(ignored by default) is the external source and \meta{list}
is the list of \supplementarylocations\ in \meta{external document}.

\cssection{bibglssupplementalsep}

\formatdef{bibglssupplementalsep}
The separator between the main \gls{locationlist} and the supplementary
\gls{locationlist}. By default this is just \gls!{bibglsdelimN}. This may be
defined before the resources are loaded:
\begin{codeenv}
\cs{newcommand}\marg{\gls{bibglssupplementalsep}}\marg{; }
\strut
\gls{GlsXtrLoadResources}\oarg{
 \csopt[supplDoc]{supplemental-locations},
 \csopt[entries]{src}}
\end{codeenv}
or redefined after the resources are loaded:
\begin{codeenv}
\gls{GlsXtrLoadResources}\oarg{
 \csopt[supplDoc]{supplemental-locations},
 \csopt[entries]{src}}
\strut
\cs{glsrenewcommand}\marg{\gls{bibglssupplementalsep}}\marg{; }
\end{codeenv}

\cssection{bibglssupplementalsubsep}

\formatdef{bibglssupplementalsubsep}
The separator between the supplementary
\gls{location} sub-lists. By default this is just \gls!{bibglsdelimN}.

\cssection{bibglshrefchar}

\formatdef{bibglshrefchar}
Expands to a literal percent character followed by \meta{hex}. The
second argument is ignored.

\cssection{bibglshrefunicode}

\formatdef{bibglshrefunicode}
Expands to the second argument. The first argument is ignored.

\cssection{bibglshexunicodechar}

\formatdef{bibglshexunicodechar}
This command is used by the \csopt{hex-unicode-fields} option when
replacing any Unicode characters. The argument \meta{hex} is the hexadecimal
character code. Note that the argument isn't preceded by the double-quote
character~\idx{doublequotecharhex} (which is normally used to
identify hexadecimal numbers in \LaTeX). Instead, the definition
needs to insert this character, if appropriate.

If \bibgls\ has detected that the \sty{hyperref} package has been
loaded, it will provide a definition that may be used in PDF
bookmarks provided that \sty{hyperref}'s \styoptfmt{unicode} option is
set. Otherwise the command will simply do
\code{\ics{cs.symbol}\marg{\idx{doublequotecharhex}\meta{hex}}}
(which will require an appropriate font in order to render the
symbol correctly).

\section{Letter Groups}
\label{sec:lettergroupdefs}

The commands listed in this section are provided for use with the
\longarg{group} switch and glossary styles that display the letter
group title. If these need their definitions altered, they should
be defined before the resource file is loaded if field expansion
is on (\longarg{expand-fields}) otherwise they may be redefined
afterwards.

The base \isty{glossaries} package determines group titles through a fairly
simplistic rule. Both \idx!{makeindex} and \idx!{xindy} write the
line:
\nosecformatdef{glsgroupheading}
to the associated glossary file at the start of each new letter group.
For example, the \qt{A} letter group will be written as:
\begin{codeenv}
\gls{glsgroupheading}\marg{A}
\end{codeenv}
This is quite straightforward and the heading title can just be
\qt{A}. The \qt{Symbols} group is written as:
\begin{codeenv}
\gls{glsgroupheading}\marg{glssymbols}
\end{codeenv}
To allow for easy translation, the base \sty{glossaries} package has
the simple rule: 
\begin{itemize}
\item if \csfmt{\meta{heading}groupname} exists use that;
\item otherwise just use \meta{heading}.
\end{itemize}
There's no \csfmt{Agroupname} provided, but \ics{glssymbolsgroupname}
is provided and is supported by the associated language modules,
such as \styfmt{glossaries-french}. (Similarly for the \qt{Numbers}
group.)

The glossary styles that provide hyperlinks to the groups (such as 
\glostyle{indexhypergroup}) use \meta{heading} to form the target
name. A problem arises when active characters occur in
\meta{heading}, which happens with extended characters and
\isty{inputenc}.

The \sty{glossaries-extra} package (as from version 1.14) provides:
\nosecformatdef{glsxtrsetgrouptitle}
to set the title for a group with the given label. The internal
workings of \gls{glsgroupheading} are modified to use a slightly
altered rule:
\begin{itemize}
\item if a title has been set using
\gls{glsxtrsetgrouptitle}\margm{heading}\margm{title} for the given
\meta{heading}, use that;
\item if \csfmt{\meta{heading}groupname} exists, use that;
\item just use \meta{heading} for the title.
\end{itemize}
So if \gls{glsxtrsetgrouptitle} hasn't been used, it falls back on
the original rule.

The problem is now how to make the indexing application use the
desired label in the argument of \gls{glsgroupheading} instead of selecting
the heading based on the first character of each sort value for each
\gls{top-levelentry} in that group. This can't be done with
\idx!{makeindex}, and with \idx!{xindy} it requires a custom language
module, which isn't a trivial task.

With \bibgls, a different approach is used. The \iext{glstex} file
created isn't comparable to the \iext{gls} file created by
\idx!{makeindex} or \idx!{xindy}. There's nowhere for \bibgls\ to
write the \gls{glsgroupheading} line as it isn't creating the code
that typesets the glossary list. Instead it's creating the code that
defines the entries. The actual group heading is inserted by 
\ics{printunsrtglossary} and it's only able to do this by checking if
the entry has a \field{group} field and comparing it to the previous
entry's \field{group} field. 

The behaviour of the group formation implemented by the sort
methods may be changed with \csopt{group-formation}. With any
setting other than \csopt[default]{group-formation}, the group label
is set to \format{bibglsunicodegroup} and the title is set to
\format{bibglsunicodegrouptitle} (see below) otherwise the label and
title are determined by the sort method.

The collators used by the locale and letter-based rules save the
following information for each entry based on the first significant letter of
the \field{sort} field (if the letter is recognised as alphabetical,
according to the rule):
\begin{itemize}
\item \meta{title} The group's title. This is typically title-cased.
For example, if the rule recognises the digraph \qt{dz}, then the
title is \qt{Dz}. Exceptions to this are included in the \langxml.
If the key \idx{grouptitle.case.lc} exists, where
\meta{lc} is the \idx!{lowercase} version of \meta{title}, then the value
of that key is used instead. For example, the Dutch digraph \qt{ij}
should be converted to \qt{IJ}, so \file{bib2gls-en.xml} includes:
\begin{verbatim}
<entry key="grouptitle.case.ij">IJ</entry>
\end{verbatim}
(See the \longarg{group} switch for more details.)
\item \meta{letter} This is the actual letter at the start of the
given entry's \field{sort} field, which may be \idx!{lowercase} or may
contain diacritics that don't appear in \meta{title}.
\item \meta{id} A numeric identifier. This may be the collation key
or the code point for the given letter, depending on the sort
method.
\item \meta{type} The entry's glossary type. If not known, this will
be empty. (\bibgls\ won't know if you've modified the associated 
\csfmt{bibglsnew\ldots} command to set the \field{type}. It can only
know the type if it's in the original \ext{bib} definition or is set
using resource options such as \csopt{type}.)
\end{itemize}

The \field{group} field is then set using:
\begin{codeenv}
\field{group}=\marg{\gls{bibglslettergroup}\margm{title}\margm{letter}\margm{id}\margm{type}}
\end{codeenv}
This field needs to expand to a simple label, which \gls!{bibglslettergroup}
is designed to do.  Note that non-letter groups are dealt with
separately (see below).

\cssection{bibglssetlastgrouptitle}

In the last resource (\ext{glstex}) file, after all the relevant
group titles have been set with the commands listed below, there's a
final title setting:
\formatdef{bibglssetlastgrouptitle}
This does nothing by default, but the arguments are set to
correspond to the group with the maximum id for that resource file.
It's provided as a convenient way of overriding the final group
title without the inconvenience of looking up the group label in the
\ext{glstex} file. If you have multiple glossaries or if you want to
override a different group, then you need to inspect the
\ext{glstex} file to work out the corresponding label (by finding
the \field{group} assignment for one of the entries in that group).

The \meta{cs} argument is the control sequence used in the
\field{group} field to obtain the label from \meta{specs}. For
example, if the highest \meta{id} is 2147418112 from:
\begin{codeenv}
\field{group}=\marg{\gls{bibglslettergroup}\marg{Ø}\marg{Ø}\marg{2147418112}\marg{}}
\end{codeenv}
then the last group is identified with:
\begin{codeenv}
\gls{bibglssetlastgrouptitle}\marg{\gls{bibglslettergroup}}\marg{\marg{Ø}\marg{Ø}\marg{2147418112}\marg{}}
\end{codeenv}
In this case \meta{cs} is \gls{bibglslettergroup} and \meta{specs}
are the arguments for that command. If you want \gls{bibglssetlastgrouptitle}
to change the group title then you need to define it before the
\igls{resourceset}. For example:
\begin{codeenv}
\cs{newcommand}*\marg{\gls{bibglssetlastgrouptitle}}[2]\marg{\comment{}
 \cs{glsxtrsetgrouptitle}\marg{\idx{param}1\idx{param}2}\marg{Foreign Words}}
\gls{GlsXtrLoadResources}\oarg{\csopt[entries]{src}}
\end{codeenv}
If you need to change a particular group title, then it has to be
done after the \idx{resourceset}:
\begin{codeenv}
\gls{GlsXtrLoadResources}\oarg{\csopt[entries]{src}}
\cs{glsxtrsetgrouptitle}
 \marg{\gls{bibglslettergroup}\marg{\marg{Ø}\marg{Ø}\marg{2147418112}\marg{}}}\comment{label}
 \marg{Foreign Words}\comment{title}
\end{codeenv}

\cssection{bibglshypergroup}

\formatdef{bibglshypergroup}

If the \iext{log} file indicates that \isty{hyperref} has been loaded
and the \longarg{group} switch is used, then this command will be
used to create the navigation information for glossary styles such
as \glostyle{indexhypergroup}. Note that this requires at least
\isty{glossaries} v4.53 and \isty{glossaries-extra} v1.53. If older
versions are detected, this command will simply ignore its
arguments.

\subsection{Top-Level Groups Only}
\label{sec:toplevelgroups}
\renewcommand{\cssectioncmd}{\subsubsection}

The default \csopt{group-level} setting will only create groups for
the \glspl{top-levelentry}. Any \glspl{sub-entry} are considered to
be part of the \gls{top-levelentry}['s] group. If hierarchical
groups are enabled, the commands defined in \sectionref{sec:hiergroups}
are provided.

\cssection{bibglssetlettergrouptitle}

For each \idx{lettergroup} that's detected, \bibgls\ will write the line:
\formatdef{bibglssetlettergrouptitle}
in the \ext{glstex} file, which sets the group's title using:
\begin{codeenv}
\format{glsxtrsetgrouptitle}
\end{codeenv}
where the \meta{group label} part matches the corresponding \field{group}
value.

Note that \gls{bibglssetlettergrouptitle} only has a single
argument, but that argument contains the four arguments needed by
\gls!{bibglslettergroup} and \gls!{bibglslettergrouptitle}.
These arguments are as described above.

If \gls{glsxtrsetgrouptitle} has been defined (\sty{glossaries-extra}
version 1.14 onwards), then \gls{bibglssetlettergrouptitle} will be
defined as:
\begin{codeenv}
\cs{providecommand}\marg{\gls{bibglssetlettergrouptitle}}[1]\marg{\comment{}
  \cs{glsxtrsetgrouptitle}\marg{\gls{bibglslettergroup}\idx{param}1}\marg{\gls{bibglslettergrouptitle}\idx{param}1}}
\end{codeenv}
If an earlier version of \sty{glossaries-extra} is used, then this
function can't be supported and the command will be defined to
simply ignore its argument. This will fall back on the original
method of just using \meta{title} as the label.

Since \gls{bibglssetlettergrouptitle} is used in the \ext{glstex} file to
set the group titles, the associated commands need to be defined
before the resource file is loaded if their definitions require
modification. After the resource file has been loaded, you can
adjust the title of a specific group, but you'll need to check the
\ext{glstex} file for the appropriate arguments. For example, if the
\ext{glstex} file contains:
\begin{codeenv}
\gls{bibglssetlettergrouptitle}\marg{\marg{\AE}\marg{\ae}\marg{7274496}\marg{}}
\end{codeenv}
but you actually want the group title to appear as \qt{\AE\ (AE)}
instead of just \qt{\AE}, then after the resource file has been
loaded you can do:
\begin{codeenv}
\gls{glsxtrsetgrouptitle}
 \marg{\gls{bibglslettergroup}\marg{\AE}\marg{\ae}\marg{7274496}\marg{}}\comment{label}
 \marg{\AE\ (AE)}\comment{title}
\end{codeenv}

\cssection{bibglslettergroup}

\formatdef{bibglslettergroup}
This command is used to determine the \idx{lettergroup} label. The
default definition is \meta{type}\meta{id}, which ensures that no
problematic characters occur in the label since \meta{type} can't
contains special characters and \meta{id} is numeric. The
\meta{type} is included in case there are multiple glossaries, since
the hyperlink name must be unique.

\cssection{bibglslettergrouptitle}

\formatdef{bibglslettergrouptitle}
This command is used to determine the \idx{lettergroup} title.
The default definition is \ics{unexpanded}\margm{title}, which guards
against any expansion issues that may arise with characters outside
the basic Latin set.

For example:
\begin{codeenv}
\atentry{entry}\marg{angstrom,
  \field{name}=\marg{\cs{AA} ngstr\cs{umlaut}om}
  \field{description}=\marg{a unit of length equal to one hundred-millionth 
of a centimetre}
}
\end{codeenv}
The \field{sort} value is \qtt{\AA ngstr\"om}. With \csopt[en]{sort}
the \meta{title} part will be \code{A} but with \csopt[sv]{sort}
the \meta{title} part will be \code{\AA}. In both cases the
\meta{letter} argument will be \code{\AA}.

Take care if you are using a script that needs encapsulating. For
example, with the \isty{CJKutf8} package the \idx{CJK} characters need to
be placed within the \env{CJK} environment, so any letter group
titles that contain \idx{CJK} characters will need special attention.

For example, suppose the \ext{bib} file contains entries in the
form:
\begin{codeenv}
\atentry{dualentry}\marg{\meta{label},
  \field{name} = \marg{\gls{cjkname}\margm{\idx{CJK} characters}},
  \field{description} = \margm{English translation}
}
\end{codeenv}
and the document contains:
\begin{codeenv}
\cmd{usepackage}\marg{CJKutf8}
\cmd{usepackage}[\styopt{record},\styopt[indexgroup]{style},\styopt{nomain}]\marg{glossaries-extra}
\strut
\cs{newglossary*}\marg{japanese}\marg{Japanese to English}
\cs{newglossary*}\marg{english}\marg{English to Japanese}
\strut
\ics{newrobustcmd}\marg{\gls{cjkname}}[1]\marg{\cmd{begin}\marg{CJK}\marg{UTF8}\marg{min}\idx{param}1\cmd{end}\marg{CJK}}
\strut
\gls{GlsXtrLoadResources}\oarg{
  \csopt[testcjk]{src},\comment{bib file}
  \csopt[ja-JP]{sort},\comment{locale used to sort \glspl{primaryentry}}
  \csopt[en-GB]{dual-sort},\comment{locale used to sort \glspl{dualentry}}
  \csopt[japanese]{type},\comment{put the \gls{primaryentry} in the 'japanese' glossary}
  \csopt[english]{dual-type},\comment{put the \gls{dualentry} in the 'english' glossary}
  \csopt[en.]{dual-prefix}
}
\end{codeenv}
then \idx{CJK} characters will appear in the \meta{title} argument of 
\gls{bibglslettergrouptitle} which causes a problem because they need
to be encapsulated within the \env{CJK} environment. This can be more
conveniently done with the user supplied \inlinedef{cjkname}, but the \idx{CJK}
characters need to be protected from expansion so \ics{unexpanded} is
also needed. The new definition of \gls{bibglslettergrouptitle} needs
to be defined before \gls{GlsXtrLoadResources}. For example:
\begin{codeenv}
\cs{newcommand}\marg{\gls{bibglslettergrouptitle}}[4]\marg{\cs{unexpanded}\marg{\gls{cjkname}\marg{\idx{param}1}}}
\end{codeenv}
There's a slight problem here in that the English letter group titles 
also end up encapsulated. An alternative approach is to use the
\meta{type} part to provide different forms. For example:
\begin{codeenv}
\cs{newcommand}*\marg{\cmd{englishlettergroup}}[1]\marg{\idx{param}1}
\cs{newcommand}*\marg{\cmd{japaneselettergroup}}[1]\marg{\gls{cjkname}\marg{\idx{param}1}}
\cs{newcommand}\marg{\gls{bibglslettergrouptitle}}[4]\marg{\comment{}
 \cs{unexpanded}\marg{\cs{csuse}\marg{\idx{param}4lettergroup}\marg{\idx{param}1}}}
\end{codeenv}

\cssection{bibglssetothergrouptitle}

The label and title for \idxpl{symbolgroup}
are dealt with in a similar way to the \idxpl!{lettergroup}, but in this
case the title is set using:
\formatdef{bibglssetothergrouptitle}
This is defined in an analogous manner:
\begin{codeenv}
\cs{providecommand}\marg{\gls{bibglssetothergrouptitle}}[1]\marg{\comment{}
 \cs{glsxtrsetgrouptitle}\marg{\gls{bibglsothergroup}\idx{param}1}\marg{\gls{bibglsothergrouptitle}\idx{param}1}}
\end{codeenv}
where the group label is obtained using \gls!{bibglsothergroup} and
the group title is obtained from \gls!{bibglsothergrouptitle}.
Note that since non-alphabetic characters don't have upper or lower
case versions, there are only three arguments. The other difference
between this and the letter group version is that the \meta{id} is
given in hexadecimal format (corresponding to the character code).

For example, suppose my \ext{bib} file contains:
\begin{codeenv}
\atentry{entry}\marg{sauthor,
  \field{name}=\marg{/Author},
  \field{description} = \marg{author string}
}
\end{codeenv}
If a locale sort is used, the leading slash \code{/} will be
ignored and this entry will belong to the \qt{A} letter group using
the letter commands described above. If, instead, one of the
character code sort methods are used, such as
\csopt[letter-case]{sort}, then this entry will be identified as
belonging to a symbol (or \qt{other}) group and the title will be
set using:
\begin{codeenv}
\gls{bibglssetothergrouptitle}\marg{\marg{/}\marg{2F}\marg{}}
\end{codeenv}

\cssection{bibglsothergroup}

\formatdef{bibglsothergroup}
This expands to the label for \idxpl{symbolgroup}. This just
defaults to \code{glssymbols} (ignoring all arguments), which
replicates the label used when \idx!{makeindex} or \idx!{xindy}
generate the glossary files.

\cssection{bibglsothergrouptitle}

\formatdef{bibglsothergrouptitle}
This expands to the title for \idxpl{symbolgroup}. This just
expands to \ics{glssymbolsgroupname} by default.

\cssection{bibglssetemptygrouptitle}

Used when the sort value degenerates to an empty string. This command
sets the label and title.
\formatdef{bibglssetemptygrouptitle}
(Note the inner group, as with the other similar 
\csfmt{bibglsset\ldots grouptitle} commands.)

\cssection{bibglsemptygroup}

\formatdef{bibglsemptygroup}
This expands to the label for \idxpl{emptygroup}. This
defaults to \code{glssymbols} to make it consistent with 
\idxpl{nonlettergroup} (since the sort value likely contained
unknown symbol commands).

\cssection{bibglsemptygrouptitle}

\formatdef{bibglsemptygrouptitle}
This expands to the group title for \idx{emptygroup}. This just
expands to \ics{glssymbolsgroupname} by default.

\cssection{bibglssetnumbergrouptitle}

The numeric sort methods (\tableref{tab:sortoptionsnumerical}) 
all create \idxpl{numbergroup} instead of letter
or symbol groups. These behave in an analogous way to the above.
\formatdef{bibglssetnumbergrouptitle}
In this case \meta{value} is the actual numeric sort value, and
\meta{id} is a decimal number obtained from converting \meta{value}
to an integer. This command is defined as:
\begin{codeenv}
\cs{providecommand}\marg{\gls{bibglssetnumbergrouptitle}}[1]\marg{\comment{}
  \cs{glsxtrsetgrouptitle}\marg{\gls{bibglsnumbergroup}\idx{param}1}\marg{\gls{bibglsnumbergrouptitle}\idx{param}1}}
\end{codeenv}

\cssection{bibglsnumbergroup}

The \idx{numbergroup} label is obtained from:
\formatdef{bibglsnumbergroup}
This just defaults to \code{glsnumbers}.

\cssection{bibglsnumbergrouptitle}

The \idx{numbergroup} title is obtained from:
\formatdef{bibglsnumbergrouptitle}
This just defaults to \ics{glsnumbersgroupname}.

\cssection{bibglssetdatetimegrouptitle}

The date-time sort methods (\tableref{tab:sortoptionsdatetime}) 
create \idxpl{datetimegroup}. These behave in an analogous way to the above.
\formatdef{bibglssetdatetimegrouptitle}
This command is defined as:
\begin{codeenv}
\cs{providecommand}\marg{\gls{bibglssetdatetimegrouptitle}}[1]\marg{\comment{}
 \cs{glsxtrsetgrouptitle}
  \marg{\gls{bibglsdatetimegroup}\idx{param}1}\comment{}
  \marg{\gls{bibglsdatetimegrouptitle}\idx{param}1}\comment{}
}
\end{codeenv}

\cssection{bibglsdatetimegroup}

\formatdef{bibglsdatetimegroup}

This command is used for \idx{datetimegroup} labels with \optfmt{datetime} sorting
(\tableref{tab:sortoptionsdatetime}). This has ten arguments, which means a
little trickery is needed to deal with the tenth argument.
The default definition is:
\begin{codeenv}
\cs{providecommand}\marg{\gls{bibglsdatetimegroup}}[9]\marg{\idx{param}1\idx{param}2\idx{param}3\ics{@firstofone}}
\end{codeenv}
This forms the group label from the year \meta{YYYY}, month \meta{MM}, 
day \meta{DD} and \meta{type}.

\cssection{bibglsdatetimegrouptitle}

\formatdef{bibglsdatetimegrouptitle}

This command is used for \idx{datetimegroup} titles with \optfmt{datetime} sorting
(\tableref{tab:sortoptionsdatetime}).
The default definition is:
\begin{codeenv}
\cs{providecommand}\marg{\gls{bibglsdatetimegrouptitle}}[9]\marg{\idx{param}1-\idx{param}2-\idx{param}3\ics{@gobble}}
\end{codeenv}
This sets the title to the numeric
\code{\meta{YYYY}-\meta{MM}-\meta{DD}}
but may be redefined as appropriate.

\cssection{bibglssetdategrouptitle}

The date sort methods (\tableref{tab:sortoptionsdatetime}) 
create \idxpl{dategroup} (the time isn't included). These behave in an 
analogous way to the above.
\formatdef{bibglssetdategrouptitle}
This command is defined as:
\begin{codeenv}
\cs{providecommand}\marg{\gls{bibglssetdategrouptitle}}[1]\marg{\comment{}
 \cs{glsxtrsetgrouptitle}\marg{\gls{bibglsdategroup}\idx{param}1}\marg{\gls{bibglsdategrouptitle}\idx{param}1}}
\end{codeenv}

\cssection{bibglsdategroup}

\formatdef{bibglsdategroup}

This command is used for \idx{dategroup} labels with \optfmt{date} (no time) sorting 
(\tableref{tab:sortoptionsdatetime}). The default definition is:
\begin{codeenv}
\cs{providecommand}\marg{\gls{bibglsdategroup}}[7]\marg{\idx{param}1\idx{param}2\idx{param}4\idx{param}7}
\end{codeenv}
This forms the group label from the year, month, era and type.
In this case, the era is a textual representation not the numeric
value used in calculating the sort value.

\cssection{bibglsdategrouptitle}

\formatdef{bibglsdategrouptitle}
This command is used for \idx{dategroup} titles with \optfmt{date} (no time) sorting
(\tableref{tab:sortoptionsdatetime}).  The default definition is:
\begin{codeenv}
\cs{providecommand}\marg{\gls{bibglsdategrouptitle}}[7]\marg{\idx{param}1-\idx{param}2}
\end{codeenv}
This just sets the title to the numeric year-month form 
\code{\meta{YYYY}-\meta{MM}}.

\cssection{bibglssettimegrouptitle}

The time sort methods (\tableref{tab:sortoptionsdatetime}) 
create \idxpl{timegroup} (the date isn't included). These behave in an 
analogous way to the above.
\formatdef{bibglssettimegrouptitle}
This command is defined as:
\begin{codeenv}
\cs{providecommand}\marg{\gls{bibglssettimegrouptitle}}[1]\marg{\comment{}
 \cs{glsxtrsetgrouptitle}\marg{\gls{bibglstimegroup}\idx{param}1}\marg{\gls{bibglstimegrouptitle}\idx{param}1}}
\end{codeenv}

\cssection{bibglstimegroup}

\formatdef{bibglstimegroup}
This command is used for \idx{timegroup} labels with \optfmt{time} (no date) sorting
(\tableref{tab:sortoptionsdatetime}).
This command is defined as:
\begin{codeenv}
\cs{providecommand}\marg{\gls{bibglstimegroup}}[7]\marg{\idx{param}1\idx{param}2\idx{param}7}
\end{codeenv}

\cssection{bibglstimegrouptitle}

\formatdef{bibglstimegrouptitle}

This command is used for \idx{timegroup} titles with \optfmt{time} (no date) sorting
(\tableref{tab:sortoptionsdatetime}).
This command is defined as:
\begin{codeenv}
\cs{providecommand}\marg{\gls{bibglstimegrouptitle}}[7]\marg{\idx{param}1}
\end{codeenv}

\cssection{bibglssetunicodegrouptitle}

\formatdef{bibglssetunicodegrouptitle}
This command is used to assign the group titles when the group
formation is set to any value other than the default. For example,
this command will be used with \csopt[codepoint]{group-formation}.
The label is obtained from \gls{bibglsunicodegroup} and the title is
obtained from \gls{bibglsunicodegrouptitle}.

\cssection{bibglsunicodegroup}

\formatdef{bibglsunicodegroup}
The \meta{label} depends on the \csopt{group-formation} setting:
\begin{itemize}
\item\csopt[codepoint]{group-formation}: the \meta{label} is the
Unicode value of \meta{character} (converted to \idx!{lowercase} and
decomposed, if applicable);
\item\csopt[unicode category]{group-formation}: the \meta{label} is the
Unicode category of \meta{character} (for example, \code{Lu} means
an \idx!{uppercase} letter);
\item\csopt[unicode script]{group-formation}: the \meta{label} is the
Unicode script associated with \meta{character} (for example,
\code{LATIN});
\item\csopt[unicode category and script]{group-formation}: the \meta{label}
identifies both the Unicode category and script associated with 
\meta{character} (for example, \code{Lu.LATIN}).
\end{itemize}
(Similarly for \csopt{secondary-group-formation} and
\csopt{dual-group-formation}.) By default this command expands to
\meta{type}\meta{label}.

The \meta{character} is the first significant character of the sort
value. The \meta{id} is the hexadecimal code of (possibly decomposed) 
\meta{character}. The case of codepoint \meta{id} may or may not correspond 
to the case of \meta{character}.

For example, with \csopt[codepoint]{group-formation}, an unset
\field{type} and a sort value of \qt{\AA ngstr\"om} with \qt{\AA} as a
significant character distinct from \qt{A} then the \field{group} 
field will be assigned using:
\begin{codeenv}
\field{group}=\marg{\gls{bibglsunicodegroup}\marg{å}\marg{Å}\marg{C5}\marg{}}
\end{codeenv}
whereas with \csopt[unicode category and script]{group-formation} it will be:
\begin{codeenv}
\field{group}=\marg{\gls{bibglsunicodegroup}\marg{Lu.LATIN}\marg{Å}\marg{C5}\marg{}}
\end{codeenv}
(\idx!{uppercase} Latin letter). 

If instead \qt{\AA} is considered equivalent to \qt{A} according to the 
collator, then with \csopt[codepoint]{group-formation}, the value will be:
\begin{codeenv}
\field{group}=\marg{\gls{bibglsunicodegroup}\marg{a}\marg{Å}\marg{61}\marg{}}
\end{codeenv}
Note that the \meta{id} is now \hex{61} (the decomposed \qt{A}
converted to \idx{lowercase}) not \hex{C5}.

\cssection{bibglsunicodegrouptitle}

\formatdef{bibglsunicodegrouptitle}
The title for Unicode group formations is simply defined as 
\code{\cs{unexpanded}\margm{label}} so you will need to change it to
something more appropriate. For example (before the \igls{resourceset}):
\begin{codeenv}
\cs{newcommand}\marg{\gls{bibglsunicodegrouptitle}}[4]\marg{\comment{}
  \ifnumhex{\idx{param}3}>64 
   \ifnumhex{\idx{param}3} < 91 
     A-{}-Z\comment{}
   \cmd{else} 
     \ifnumhex{\idx{param}3} > 96
       \ifnumhex{\idx{param}3} < 123
         A-{}-Z\comment{}
       \cmd{fi}
     \cmd{fi}
   \cmd{fi}
  \cmd{fi}
}
\end{codeenv}
This will make the title \qt{A--Z} if \meta{id} is greater
than 64 and less than 91 or greater than 96 and less than 123 
(and will be empty otherwise).

Note that this setting can create an odd effect if the sorting
causes the groups to be split up. For example, if some of the sort values
start with extended or non-Latin characters this can break up the
groups. First check how the group labels are assigned using:
\begin{codeenv}
\cs{newcommand}\marg{\gls{bibglsunicodegrouptitle}}\marg{\gls{bibglsunicodegroup}}
\end{codeenv}
then adjust the definition of \gls{bibglsunicodegroup} until the
grouping is correct, and then change the definition of 
\gls{bibglsunicodegrouptitle} so that the title is correct.

\cssection{bibglssetmergedgrouptitle}

Used when groups are merged by \csopt{merge-small-groups}. This command
sets the label and title.
\formatdef{bibglssetmergedgrouptitle}
(Note the inner group, as with the other similar 
\csfmt{bibglsset\ldots grouptitle} commands.)

\cssection{bibglsmergedgroup}

\formatdef{bibglsmergedgroup}
This expands to the label for merged groups.

\cssection{bibglsmergedgrouptitle}

\formatdef{bibglsmergedgrouptitle}
This expands to the group title for merged groups.

\cssection{bibglsmergedgroupfmt}

\formatdef{bibglsmergedgroupfmt}
Used by \gls{bibglsmergedgrouptitle} and
\gls{bibglsmergedgrouphierfmt} to format merged group titles. The
first argument \meta{n} is the total number of groups that have been
merged, the second argument, \meta{g$_1$} is the first group title,
\marg{\meta{g$_2$}\ldots\meta{g$_n$}} are the middle group titles
(empty if $\meta{n}=2$), and \meta{g$_n$} is the last group title.

The default definition is:
\begin{codeenv}
\cs{providecommand}\marg{\gls{bibglsmergedgroupfmt}}[4]\marg{\idx{param}2,
\cs{ifcase}\idx{param}1\cmd{or}\cmd{or}\cmd{or} \idx{param}3,
\cmd{else}\ldots, \cmd{fi} \idx{param}4}
\end{codeenv}


\subsection{Hierarchical Groups}
\label{sec:hiergroups}

Hierarchical letter groups are set with analogous commands that have
\code{hier} appended to the command name. There are also two extra
arguments, \meta{parent} (the \gls{parententry}['s] label) and
\meta{level} (the hierarchical level).

\cssection{bibglsgrouplevel}

\formatdef{bibglsgrouplevel}
This command is used when \csopt{group-level} is used to apply group
formations for different hierarchical levels. The \meta{label}
argument is the label that would normally be applied to level~0.
The \meta{n} argument is the hierarchical level (0 for
\glspl{top-levelentry}). By default, this command simply expands to
\meta{label}.

\cssection{bibglshiersubgrouptitle}

Hierarchical group titles are formatted using:
\formatdef{bibglshiersubgrouptitle}
This is defined as:
\begin{codeenv}
\cs{providecommand}\marg{\gls{bibglshiersubgrouptitle}}[3]\marg{\cs{ifnum}\idx{param}1>0 \cs{Glsxtrhiername}\marg{\idx{param}2} / \cmd{fi} \idx{param}3}
\end{codeenv}
The first argument \meta{level} is the hierarchical level, the
second argument is the \gls{parententry} label (empty for
$\meta{level}=0$), and the third argument \meta{title} is the 
normal title that would apply to the group with the default
\csopt[0]{group-level} setting. Note that this uses
\cs{Glsxtrhiername}\margm{parent} if $\meta{level}>0$.
If this isn't required then redefine \gls{bibglshiersubgrouptitle}
to simply expand to \meta{title} (\code{\idx{param}3}).

\cssection{bibglssetlettergrouptitlehier}
\formatdef{bibglssetlettergrouptitlehier}
As \gls{bibglssetlettergrouptitle} but used for hierarchical groups.

\cssection{bibglslettergrouphier}
\formatdef{bibglslettergrouphier}
As \gls{bibglslettergroup} but used for hierarchical groups. It
expands to the hierarchical letter group label.
This is defined as:
\begin{codeenv}
\cs{providecommand}\marg{\gls{bibglslettergrouphier}}[6]\marg{\idx{param}4\idx{param}5\idx{param}3}
\end{codeenv}

\cssection{bibglslettergrouptitlehier}
\formatdef{bibglslettergrouptitlehier}
As \gls{bibglslettergrouptitle} but used for hierarchical groups. It
expands to the hierarchical letter group title. This is defined as:
\begin{codeenv}
\cs{providecommand}\marg{\gls{bibglslettergrouptitlehier}}[6]\marg{\cs{protect}\gls{bibglshiersubgrouptitle}\marg{\idx{param}6}\marg{\idx{param}5}\marg{\cs{unexpanded}\marg{\idx{param}1}}}
\end{codeenv}

\cssection{bibglssetothergrouptitlehier}
\formatdef{bibglssetothergrouptitlehier}
As \gls{bibglssetothergrouptitle} but used for hierarchical groups.

\cssection{bibglsothergrouphier}
\formatdef{bibglsothergrouphier}
As \gls{bibglsothergroup} but used for hierarchical groups. It
expands to the hierarchical other group label.
This is defined as:
\begin{codeenv}
\cs{providecommand}\marg{\gls{bibglsothergrouphier}}[5]\marg{\idx{param}3\idx{param}4glssymbols}
\end{codeenv}

\cssection{bibglsothergrouptitlehier}
\formatdef{bibglsothergrouptitlehier}
As \gls{bibglsothergrouptitle} but used for hierarchical groups. It
expands to the hierarchical other group title. This is defined as:
\begin{codeenv}
\cs{providecommand}\marg{\gls{bibglsothergrouptitlehier}}[5]\marg{\cs{protect}\gls{bibglshiersubgrouptitle}\marg{\idx{param}5}\marg{\idx{param}4}\marg{\cs{protect}\cs{glssymbolsgroupname}}}
\end{codeenv}

\cssection{bibglssetemptygrouptitlehier}
\formatdef{bibglssetemptygrouptitlehier}
As \gls{bibglssetemptygrouptitle} but used for hierarchical groups.

\cssection{bibglsemptygrouphier}
\formatdef{bibglsemptygrouphier}
As \gls{bibglsemptygroup} but used for hierarchical groups. It
expands to the hierarchical empty group label.
This is defined as:
\begin{codeenv}
\cs{providecommand}\marg{\gls{bibglsemptygrouphier}}[3]\marg{\idx{param}1\idx{param}2glssymbols}
\end{codeenv}

\cssection{bibglsemptygrouptitlehier}
\formatdef{bibglsemptygrouptitlehier}
As \gls{bibglsemptygrouptitle} but used for hierarchical groups. It
expands to the hierarchical empty group title. This is defined as:
\begin{codeenv}
\cs{providecommand}\marg{\gls{bibglsemptygrouptitlehier}}[3]\marg{\cs{protect}\gls{bibglshiersubgrouptitle}\marg{\idx{param}3}\marg{\idx{param}2}\marg{\cs{protect}\cs{glssymbolsgroupname}}}
\end{codeenv}

\cssection{bibglssetnumbergrouptitlehier}
\formatdef{bibglssetnumbergrouptitlehier}
As \gls{bibglssetnumbergrouptitle} but used for hierarchical groups.

\cssection{bibglsnumbergrouphier}
\formatdef{bibglsnumbergrouphier}
As \gls{bibglsnumbergroup} but used for hierarchical groups. It
expands to the hierarchical number group label.
This is defined as:
\begin{codeenv}
\cs{providecommand}\marg{\gls{bibglsnumbergrouphier}}[5]\marg{\idx{param}3\idx{param}4glsnumbers}
\end{codeenv}

\cssection{bibglsnumbergrouptitlehier}
\formatdef{bibglsnumbergrouptitlehier}
As \gls{bibglsnumbergrouptitle} but used for hierarchical groups. It
expands to the hierarchical number group title. This is defined as:
\begin{codeenv}
\cs{providecommand}\marg{\gls{bibglsnumbergrouptitlehier}}[5]\marg{\cs{protect}\gls{bibglshiersubgrouptitle}\marg{\idx{param}5}\marg{\idx{param}4}\marg{\cs{protect}\cs{glsnumbersgroupname}}}
\end{codeenv}

\cssection{bibglssetdatetimegrouptitlehier}
\formatdef{bibglssetdatetimegrouptitlehier}
As \gls{bibglssetdatetimegrouptitle} but used for hierarchical groups.

\cssection{bibglsdatetimegrouphier}
\formatdef{bibglsdatetimegrouphier}
As \gls{bibglsdatetimegroup} but used for hierarchical groups. It
expands to the hierarchical \idx{datetimegroup} label.
Since this has more than 9 arguments, it is defined in two parts:
\begin{codeenv}
\cs{providecommand}\marg{\gls{bibglsdatetimegrouphier}}[9]\marg{\idx{param}1\idx{param}2\idx{param}3\gls{bibglsdatetimegrouphierfinalargs}}
\end{codeenv}
The final arguments are obtained with \gls{bibglsdatetimegrouphierfinalargs}.

\cssection{bibglsdatetimegrouphierfinalargs}
\formatdef{bibglsdatetimegrouphierfinalargs}
This picks up the final three arguments for \gls{bibglsdatetimegrouphier}:
\begin{codeenv}
\cs{providecommand}*\marg{\gls{bibglsdatetimegrouphierfinalargs}}[3]\marg{\idx{param}1\idx{param}2}
\end{codeenv}

\cssection{bibglsdatetimegrouptitlehier}
\formatdef{bibglsdatetimegrouptitlehier}
As \gls{bibglsdatetimegrouptitle} but used for hierarchical groups. It
expands to the hierarchical \idx{datetimegroup} title.
Since this has more than 9 arguments, it is defined in two parts:
\begin{codeenv}
\cs{providecommand}\marg{\gls{bibglsdatetimegrouptitlehier}}[9]\marg{\gls{bibglsdatetimegrouptitlehierfinalargs}\marg{\idx{param}1-\idx{param}2-\idx{param}3}}
\end{codeenv}
The final arguments are obtained with \gls{bibglsdatetimegrouptitlehierfinalargs}.

\cssection{bibglsdatetimegrouptitlehierfinalargs}
\formatdef{bibglsdatetimegrouptitlehierfinalargs}
This picks up the final three arguments for \gls{bibglsdatetimegrouptitlehier}:
\begin{codeenv}
\cs{providecommand}*\marg{\gls{bibglsdatetimegrouptitlehierfinalargs}}[4]\marg{\cs{protect}\gls{bibglshiersubgrouptitle}\marg{\idx{param}4}\marg{\idx{param}3}\marg{\idx{param}1}}
\end{codeenv}
In this case, the first argument is set by
\gls{bibglsdatetimegrouptitlehier} to the date
(\code{\meta{YYYY}-\meta{MM}-\meta{DD}}). The remaining arguments
are the \meta{type}, \meta{parent} and \meta{level} information.

\cssection{bibglssetdategrouptitlehier}
\formatdef{bibglssetdategrouptitlehier}
As \gls{bibglssetdategrouptitle} but used for hierarchical groups.

\cssection{bibglsdategrouphier}
\formatdef{bibglsdategrouphier}
As \gls{bibglsdategroup} but used for hierarchical groups. It
expands to the hierarchical date group label.
This is defined as:
\begin{codeenv}
\cs{providecommand}\marg{\gls{bibglsdategrouphier}}[9]\marg{\idx{param}1\idx{param}2\idx{param}4\idx{param}7\idx{param}8}
\end{codeenv}

\cssection{bibglsdategrouptitlehier}
\formatdef{bibglsdategrouptitlehier}
As \gls{bibglsdategrouptitle} but used for hierarchical groups. It
expands to the hierarchical date group title. This is defined as:
\begin{codeenv}
\cs{providecommand}\marg{\gls{bibglsdategrouptitlehier}}[9]\marg{\cs{protect}\gls{bibglshiersubgrouptitle}\marg{\idx{param}9}\marg{\idx{param}8}\marg{\idx{param}1-\idx{param}2}}
\end{codeenv}

\cssection{bibglssettimegrouptitlehier}
\formatdef{bibglssettimegrouptitlehier}
As \gls{bibglssettimegrouptitle} but used for hierarchical groups.

\cssection{bibglstimegrouphier}
\formatdef{bibglstimegrouphier}
As \gls{bibglstimegroup} but used for hierarchical groups. It
expands to the hierarchical time group label. This is defined as:
\begin{codeenv}
\cs{providecommand}\marg{\gls{bibglstimegrouphier}}[9]\marg{\idx{param}1\idx{param}2\idx{param}7\idx{param}8}
\end{codeenv}

\cssection{bibglstimegrouptitlehier}
\formatdef{bibglstimegrouptitlehier}
As \gls{bibglstimegrouptitle} but used for hierarchical groups. It
expands to the hierarchical time group title. This is defined as:
\begin{codeenv}
\cs{providecommand}\marg{\gls{bibglstimegrouptitlehier}}[9]\marg{\cs{protect}\gls{bibglshiersubgrouptitle}\marg{\idx{param}9}\marg{\idx{param}8}{\idx{param}1}}
\end{codeenv}

\cssection{bibglssetunicodegrouptitlehier}
\formatdef{bibglssetunicodegrouptitlehier}
As \gls{bibglssetunicodegrouptitle} but used for hierarchical groups.

\cssection{bibglsunicodegrouphier}
\formatdef{bibglsunicodegrouphier}
As \gls{bibglsunicodegroup} but used for hierarchical groups. This
expands to the hierarchical group label:
\begin{codeenv}
\cs{providecommand}\marg{\gls{bibglsunicodegrouphier}}[6]\marg{\idx{param}4\idx{param}5\idx{param}3}
\end{codeenv}

\cssection{bibglsunicodegrouptitlehier}
\formatdef{bibglsunicodegrouptitlehier}
As \gls{bibglsunicodegrouptitle} but used for hierarchical groups. This
expands to the hierarchical group title:
\begin{codeenv}
\cs{providecommand}\marg{\gls{bibglsunicodegrouptitlehier}}[6]\marg{\cs{protect}\gls{bibglshiersubgrouptitle}\marg{\idx{param}6}\marg{\idx{param}5}\marg{\cs{unexpanded}\marg{\idx{param}1}}}
\end{codeenv}

\cssection{bibglssetmergedgrouptitlehier}
\formatdef{bibglssetmergedgrouptitlehier}
As \gls{bibglssetmergedgrouptitle} but used for hierarchical groups.

\cssection{bibglsmergedgrouphier}
\formatdef{bibglsmergedgrouphier}
As \gls{bibglsmergedgroup} but used for hierarchical groups. This
expands to the hierarchical group label:
\begin{codeenv}
\cs{providecommand}\marg{\gls{bibglsmergedgrouphier}}[8]\marg{merged.\idx{param}1}
\end{codeenv}

\cssection{bibglsmergedgrouptitlehier}
\formatdef{bibglsmergedgrouptitlehier}
As \gls{bibglsmergedgrouptitle} but used for hierarchical groups. This
expands to the hierarchical group title:
\begin{codeenv}
\cs{providecommand}\marg{\gls{bibglsmergedgrouptitlehier}}[8]\marg{\comment{}
 \cs{unexpanded}\marg{\cs{ifnum}\idx{param}8=0\gls{bibglsmergedgroupfmt}\marg{\idx{param}3}\marg{\idx{param}4}\marg{\idx{param}5}\marg{\idx{param}6}\cmd{else}\gls{bibglsmergedgrouphierfmt}\marg{\idx{param}3}\marg{\idx{param}4}\marg{\idx{param}5}\marg{\idx{param}6}\cmd{fi}}\comment{}
}
\end{codeenv}
This uses \gls{bibglsmergedgroupfmt} ($\meta{level}=0$) or 
\gls{bibglsmergedgrouphierfmt} ($\meta{level}>0$) to format the title.

\cssection{bibglsmergedgrouphierfmt}

\formatdef{bibglsmergedgrouphierfmt}
Used by \gls{bibglsmergedgrouphierfmt} to format merged hierarchical group titles.
The first argument \meta{n} is the total number of groups that have been
merged, the second argument, \meta{g$_1$} is the first group title,
\marg{\meta{g$_2$}\ldots\meta{g$_n$}} are the middle group titles
(empty if $\meta{n}=2$), and \meta{g$_n$} is the last group title.

The default definition depends on whether or not \bibgls\ has
detected \sty{hyperref} in the document's \ext{log} file. If it has,
then the definition is:
\begin{codeenv}
\cs{providecommand}\marg{\gls{bibglsmergedgrouphierfmt}}[4]\marg{\idx{param}2, 
\ics{texorpdfstring}\marg{\marg{\cs{def}\gls{bibglshiersubgrouptitle}\idx{param}\idx{param}1\idx{param}\idx{param}2\idx{param}\idx{param}3{\idx{param}\idx{param}3}\cs{ifcase}\idx{param}1\cmd{or}\cmd{or}\cmd{or}
\idx{param}3, \cmd{else}\cmd{ldots}, \cmd{fi}
\idx{param}4}}\marg{\cs{ifcase}\idx{param}1\cmd{or}\cmd{or}\cmd{or}
\idx{param}3, \cmd{else}\cmd{ldots}, \cmd{fi} \idx{param}4}}
\end{codeenv}
Otherwise the definition is:
\begin{codeenv}
\cs{providecommand}\marg{\gls{bibglsmergedgrouphierfmt}}[4]\marg{\idx{param}2, 
\marg{\cs{def}\gls{bibglshiersubgrouptitle}\idx{param}\idx{param}1\idx{param}\idx{param}2\idx{param}\idx{param}3{\idx{param}\idx{param}3}\cs{ifcase}\idx{param}1\cmd{or}\cmd{or}\cmd{or}
\idx{param}3, \cmd{else}\cmd{ldots}, \cmd{fi}
\idx{param}4}}
\end{codeenv}
This locally redefines \gls{bibglshiersubgrouptitle} to just do its
final argument to allow for a more compact title.

\renewcommand{\cssectioncmd}{\subsection}
\section{Flattened Entries}
\label{sec:flattendefs}

These commands relate to the way the \field{name} field is altered
when flattening \glspl{lonelychildentry} with the \csopt{flatten-lonely}
option.

\cssection{bibglsflattenedhomograph}

\formatdef{bibglsflattenedhomograph}
The default definition simply does \meta{name}.

This command is used if the \child\ and \parent\ names are identical.
For example, suppose the \ext{bib} file contains:
\begin{codeenv}
\atentry{index}\marg{super.glossary, \field{name}=\marg{glossary}}
\strut
\atentry{entry}\marg{glossarycol,
  \field{parent}=\marg{super.glossary},
  \field{description}=\marg{collection of glosses}
}
\strut
\atentry{entry}\marg{glossarylist,
  \field{parent}=\marg{super.glossary},
  \field{description}=\marg{list of technical words}
}
\end{codeenv}
The \glspl{childentry} don't have a \field{name} field, so the value is assumed
to be the same as the \glsdisp{parententry}{parent's} \field{name} field. 
Here's an example document where both \glspl{childentry} are used:
\begin{codeenv}
\cmd{documentclass}\marg{article}
\strut
\cmd{usepackage}[\styopt{record},\istyopt{subentrycounter},\styopt[treenoname]{style}]\marg{glossaries-extra}
\strut
\gls{GlsXtrLoadResources}\oarg{\csopt[entries]{src}}
\strut
\cmd{begin}\marg{document}
\cs{gls}\marg{glossarycol} (collection) vs \cs{gls}\marg{glossarylist} (list).
\strut
\cs{printunsrtglossary}
\cmd{end}\marg{document}
\end{codeenv}
This uses one of the glossary styles designed for \iglspl{homograph} and the
glossary has the structure:
\begin{flushleft}
\textbf{glossary}\par
\quad 1) collection of glosses 1\par
\quad 2) list of technical words 1
\end{flushleft}
If \glsdisp{lonelychildentry}{only one child entry} is selected, then the result looks a little
odd. For example:
\begin{flushleft}
\textbf{glossary}\par
\quad 1) collection of glosses 1
\end{flushleft}
With the \csopt{flatten-lonely} option, the \parent\ is removed and
the \child\ is moved up a \hierarchicallevel. With 
\csopt[postsort]{flatten-lonely} this would normally adjust the
name so that it appears as \meta{parent name}, \meta{child name}
but in this case it would look a little odd for the name to 
appear as \qt{glossary, glossary} so instead the name is
set to:
\begin{codeenv}
\gls{bibglsflattenedhomograph}\marg{glossary}\marg{super.glossary}
\end{codeenv}
(where the first argument is the original name and the second argument is the
label of the \gls{parententry}).

This means that the name simply appears as \qt{glossary}, even if
the \csopt[postsort]{flatten-lonely} option is used. Note that if
the \gls{parententry} is removed, the \parent\ label won't be of
much use.  You can test for existence using \ics{ifglsentryexists}
in case you want to vary the way the name is displayed according to
whether or not the \parent\ is still present.

\cssection{bibglsflattenedchildpresort}

\formatdef{bibglsflattenedchildpresort}

Used by the \csopt[presort]{flatten-lonely} option. This defaults
to just \meta{child name}. If you want to change this, remember that
you can let the interpreter know by adding the definition to
\atentry{preamble}. For example:
\begin{codeenv}
\atentry{preamble}\marg{\qtdelim{\cs{providecommand}\marg{\gls{bibglsflattenedchildpresort}}[2]\marg{\idx{param}1 (\idx{param}2)}}}
\end{codeenv}

\cssection{bibglsflattenedchildpostsort}

\formatdef{bibglsflattenedchildpostsort}

Used by the \csopt[postsort]{flatten-lonely} option. This defaults
to \code{\meta{parent name}, \meta{child name}}.

Note that the arguments are in the reverse order to those of the
previous command. This is done to assist the automated first letter
upper-casing. If either command is redefined to alter the ordering,
then this can confuse the case-changing mechanism, in which case you
may want to consider switching on the expansion of the \field{name}
field using:
\begin{codeenv}
\gls{glssetexpandfield}\marg{name}
\end{codeenv}
(before \gls{GlsXtrLoadResources}).

\section{Other}

\cssection{bibglscopytoglossary}

\formatdef{bibglscopytoglossary}
This command is provided if the \csopt{copy-to-glossary} option is
set and is used to copy an entry to another glossary. The definition
is:
\begin{codeenv}
\cmd{providecommand}\marg{\gls{bibglscopytoglossary}}[2]{\comment{}
 \ics{ifglossaryexists*}\marg{}\comment{}
 \marg{\ics{GlsXtrIfInGlossary}\marg{\idx{param}1}\marg{\idx{param}2}\marg{}\marg{\ics{glsxtrcopytoglossary}\marg{\idx{param}1}\marg{\idx{param}2}}}\comment{}
 \marg{}\comment{}
}
\end{codeenv}
This ensures that the entry is only copied if the glossary exists
and if the entry hasn't already been copied to it.

This command isn't used by the \csopt[copy]{action} or 
\csopt[copy or define]{action} settings, which use \ics{glsxtrcopytoglossary}
directly.

\cssection{bibglssettotalrecordcount}

\formatdef{bibglssettotalrecordcount}
This command is provided if \longarg{record-count} is used. It's
used to set the \field{recordcount} field to the total number of
records for the given entry. This is defined as:
\begin{codeenv}
\cmd{providecommand}*\marg{\gls{bibglssettotalrecordcount}}[2]\marg{\comment{}
 \cs{GlsXtrSetField}\marg{\idx{param}1}\marg{\field{recordcount}}\marg{\idx{param}2}\comment{}
}
\end{codeenv}

\cssection{bibglssetrecordcount}

\formatdef{bibglssetrecordcount}
This command is provided if \longarg{record-count} is used. It's
used to set the \field{recordcount.counter} field to the total number of
records associated with the given counter for the given entry. This
is defined as:
\begin{codeenv}
\cmd{providecommand}*\marg{\gls{bibglssetrecordcount}}[3]\marg{\comment{}
 \cs{GlsXtrSetField}\marg{\idx{param}1}\marg{recordcount.\idx{param}2}\marg{\idx{param}3}\comment{}
}
\end{codeenv}

\cssection{bibglssetlocationrecordcount}

\formatdef{bibglssetlocationrecordcount}
This command is provided if \longarg{record-count-unit} is used.
It's used to set the \field{recordcount.counter.location} field to
the total number of records associated with the given location for
the given entry. This is defined as:
\begin{codeenv}
\cmd{providecommand}*\marg{\gls{bibglssetlocationrecordcount}}[4]\marg{\comment{}
 \cs{GlsXtrSetField}\marg{\idx{param}1}\marg{recordcount.\idx{param}2.\ics{glsxtrdetoklocation}\idx{param}3}\marg{\idx{param}4}\comment{}
}
\end{codeenv}

\cssection{bibglshyperlink}

\formatdef{bibglshyperlink}
Used by the \hyperref[sec:dualoptsbacklinks]{back link options},
this just defaults to: 
\begin{codeenv}
\ics{glshyperlink}\oargm{text}\margm{label}
\end{codeenv}

\cssection{bibglssetwidest}

\formatdef{bibglssetwidest}
This is used by \csopt{set-widest} to set the widest name for the
given \hierarchicallevel\ where the glossary type can't be determined.
This is defined as:
\begin{codeenv}
\cs{providecommand}*\marg{\gls{bibglssetwidest}}[2]\marg{\cs{glsxtrSetWidest}\marg{}\marg{\idx{param}1}\marg{\idx{param}2}}
\end{codeenv}
if \ics{glsxtrSetWidest} has been defined, or:
\begin{codeenv}
\cs{providecommand}*\marg{\gls{bibglssetwidest}}[2]\marg{\cs{glsupdatewidest}\oarg{\idx{param}1}\marg{\idx{param}2}}
\end{codeenv}
if \ics{glsupdatewidest} is defined, otherwise it will be
defined to use \ics{glssetwidest}:
\begin{codeenv}
\cs{providecommand}*\marg{\gls{bibglssetwidest}}[2]\marg{\cs{glssetwidest}\oarg{\idx{param}1}\marg{\idx{param}2}}
\end{codeenv}
Since this isn't scoped, this will affect other glossaries.
In general, if you have more than one glossary it's best to
set the \field{type} using options like \csopt{type}.

\cssection{bibglssetwidestfortype}

\formatdef{bibglssetwidestfortype}
This is used by \csopt{set-widest} to set the widest name for the
given \hierarchicallevel\ where the glossary type is known. This
is defined as:
\begin{codeenv}
\cs{providecommand}*\marg{\gls{bibglssetwidestfortype}}[3]\marg{\comment{}
  \cs{glsxtrSetWidest}\marg{\idx{param}1}\marg{\idx{param}2}\marg{\idx{param}3}\comment{}
}
\end{codeenv}
if \ics{glsxtrSetWidest} has been defined, or:
\begin{codeenv}
\cs{providecommand}*\marg{\gls{bibglssetwidestfortype}}[3]\marg{\comment{}
  \cs{apptoglossarypreamble}\oarg{\idx{param}1}\marg{\cs{glsupdatewidest}\oarg{\idx{param}2}\marg{\idx{param}3}}\comment{}
}
\end{codeenv}
if \ics{glsupdatewidest} is defined, otherwise it will be
defined to use \ics{glssetwidest}:
\begin{codeenv}
\cs{providecommand}*\marg{\gls{bibglssetwidestfortype}}[3]\marg{\comment{}
  \cs{apptoglossarypreamble}\oarg{\idx{param}1}\marg{\cs{glssetwidest}\oarg{\idx{param}2}\marg{\idx{param}3}}\comment{}
}
\end{codeenv}
Since the glossary preamble is scoped, this won't affect other
glossaries.

\cssection{bibglssetwidestfallback}

\formatdef{bibglssetwidestfallback}
This is used by \csopt{set-widest} instead of \gls{bibglssetwidest}
when all \field{name} fields end up as an empty string when
interpreted by \bibgls. This typically means that all the
\field{name} fields contain unknown commands. This fallback command
will use:
\begin{codeenv}
\ics{glsxtrSetWidestFallback}\marg{2}\margm{glossary list}
\end{codeenv}
if defined otherwise it will use \ics{glsFindWidestLevelTwo}, which
sets the widest name for the \glsdisp{top-levelentry}{top-level} and
first two sub-levels across all the listed glossaries.

\cssection{bibglssetwidestfortypefallback}

\formatdef{bibglssetwidestfortypefallback}
This is used by \csopt{set-widest} instead of
\gls{bibglssetwidestfortype}
when all \field{name} fields end up as an empty string when
interpreted by \bibgls. This typically means that all the
\field{name} fields contain unknown commands. This fallback command
will append \gls{bibglssetwidestfallback} to the glossary preamble
for the given type.

\cssection{bibglssetwidesttoplevelfallback}

\formatdef{bibglssetwidesttoplevelfallback}
This is used by \csopt{set-widest} instead of \gls{bibglssetwidest}
when all \field{name} fields end up as an empty string when
interpreted by \bibgls. This typically means that all the
\field{name} fields contain unknown commands. This fallback command
will use:
\begin{codeenv}
\ics{glsxtrSetWidestFallback}\marg{0}\margm{glossary list}
\end{codeenv}
if defined otherwise it will use \ics{glsFindWidestTopLevelName},
which sets the widest name for the \glsdisp{top-levelentry}{top-level}.

\cssection{bibglssetwidesttoplevelfortypefallback}

\formatdef{bibglssetwidesttoplevelfortypefallback}
This is used by \csopt{set-widest} instead of
\gls{bibglssetwidestfortype}
when all \field{name} fields end up as an empty string when
interpreted by \bibgls. This typically means that all the
\field{name} fields contain unknown commands. This fallback command
will append \gls{bibglssetwidesttoplevelfallback} to the glossary
preamble of the given type.

\cssection{bibglscontributorlist}

\formatdef{bibglscontributorlist}
This is used when \csopt{bibtex-contributor-fields} is set. The
definition depends on whether or not \ics{DTLformatlist} has been
defined:
\begin{codeenv}
\ics{ifdef}\cs{DTLformatlist}
\marg{\comment{datatool v2.28+}
 \cs{providecommand}*\marg{\gls{bibglscontributorlist}}[2]\marg{\cs{DTLformatlist}\marg{\idx{param}1}}
}
\marg{\comment{datatool v2.27 or earlier}
 \cs{providecommand}*\marg{\gls{bibglscontributorlist}}[2]\marg{\comment{}
  \ics{def}\cmd{bibgls@sep}\marg{}\comment{}
  \ics{@for}\cmd{bibgls@item}:=\idx{param}1\cmd{do}\marg{\cmd{bibgls@sep}\cmd{bibgls@item}\cs{def}\cmd{bibgls@sep}\marg{, }}\comment{}
 }
}
\end{codeenv}
The second argument allows you to provide definitions like:
\begin{codeenv}
\cs{newcommand}*\marg{\gls{bibglscontributorlist}}[2]\marg{\comment{}
 \cs{ifcase}\idx{param}2
 \cmd{or}
   name:
 \cmd{else}
   names:
 \cmd{fi}
 \cs{DTLformatlist}\marg{\idx{param}1}\comment{}
}
\end{codeenv}

\cssection{bibglscontributor}

\formatdef{bibglscontributor}
This is used when \csopt{bibtex-contributor-fields} is set. The
definition depends on the value of \csopt{contributor-order}. Note that if you
have multiple resource sets, that option governs the way \bibgls's version of
\gls{bibglscontributor} behaves. The definition is written to the \ext{glstex}
using \cs{providecommand}, so \LaTeX\ will only pick up the first definition.

\cssection{bibglsdatetime}

\formatdef{bibglsdatetime}
Used to encapsulate date-time fields identified with
\csopt{date-time-fields}.
Since \gls{bibglsdatetime} requires more than nine arguments, the
remaining four arguments are picked up with:
\nosecdef{bibglsdatetimeremainder}
The default definitions are:
\begin{codeenv}
\cs{providecommand}\marg{\gls{bibglsdatetime}}[9]\marg{\gls{bibglsdatetimeremainder}}
\cs{providecommand}\marg{\gls{bibglsdatetimeremainder}}[4]\marg{\idx{param}4}
\end{codeenv}

\cssection{bibglsdate}

\formatdef{bibglsdate}
Used to encapsulate date fields identified with
\csopt{date-fields}.
The default definition is:
\begin{codeenv}
\cs{providecommand}\marg{\gls{bibglsdate}}[7]\marg{\idx{param}7}
\end{codeenv}

\cssection{bibglstime}

\formatdef{bibglstime}
Used to encapsulate date fields identified with
\csopt{time-fields}.
The default definition is:
\begin{codeenv}
\cs{providecommand}\marg{\gls{bibglstime}}[7]\marg{\idx{param}7}
\end{codeenv}

\cssection{bibglsprimaryprefixlabel}

\formatdef{bibglsprimaryprefixlabel}
A hook to pick up the \primary\ prefix label (identified with
\csopt{label-prefix}) if required. This does nothing by default. If
required, this command should be defined before the
\igls{resourceset} is loaded.

\cssection{bibglsdualprefixlabel}

\formatdef{bibglsdualprefixlabel}
A hook to pick up the \dual\ prefix label (identified with
\csopt{dual-prefix}) if required. This does nothing by default. If
required, this command should be defined before the
\igls{resourceset} is loaded.

\cssection{bibglstertiaryprefixlabel}

\formatdef{bibglstertiaryprefixlabel}
A hook to pick up the \glsdisp{tertiaryentry}{tertiary} prefix label (identified with
\csopt{tertiary-prefix}) if required. This does nothing by default. If
required, this command should be defined before the
\igls{resourceset} is loaded.

\cssection{bibglsexternalprefixlabel}

\formatdef{bibglsexternalprefixlabel}
A hook to pick up the \meta{n}th external prefix label (identified with
\csopt{ext-prefixes}) if required. This does nothing by default and
won't be used if the list of external prefixes is empty. If required,
this command should be defined before the \igls{resourceset} is
loaded.

\cssection{bibglshashchar}

\formatdef{bibglshashchar}
Expands to a literal hash character (\,\idx{hashchar}\,).

\cssection{bibglsunderscorechar}

\formatdef{bibglsunderscorechar}
Expands to a literal underscore character (\,\idx{underscorechar}\,).

\cssection{bibglsdollarchar}

\formatdef{bibglsdollarchar}
Expands to a literal dollar character (\,\idx{dollarchar}\,).

\cssection{bibglsampersandchar}

\formatdef{bibglsampersandchar}
Expands to a literal ampersand character (\,\idx{ampchar}\,).

\cssection{bibglscircumchar}

\formatdef{bibglscircumchar}
Expands to a literal circumflex character (\,\idx{circumchar}\,).

\cssection{bibglsaposchar}

\formatdef{bibglsaposchar}
Expands to a literal apostrophe character (\,\idx{aposchar}\,).
This command is only provided if \longarg{replace-quotes} is used.

\cssection{bibglsdoublequotechar}

\formatdef{bibglsdoublequotechar}
Expands to a literal double-quote character (\,\idx{doublequotechar}\,).
This command is only provided if \longarg{replace-quotes} is used.

\cssection{bibglsuppercase}

\formatdef{bibglsuppercase}
Converts \meta{text} to \idx{uppercase}. This just uses
\ics{glsuppercase} (if \sty{glossaries} v4.50+,
\sty{glossaries-extra} v1.49+ and \sty{mfirstuc} v2.08+)
or \ics{MakeTextUppercase} by default.

\cssection{bibglslowercase}

\formatdef{bibglslowercase}
Converts \meta{text} to \idx{lowercase}. This just uses
\ics{glslowercase} (if \sty{glossaries} v4.50+,
\sty{glossaries-extra} v1.49+ and \sty{mfirstuc} v2.08+)
\ics{MakeTextLowercase} by default.

\cssection{bibglstitlecase}

\formatdef{bibglstitlecase}
Converts \meta{text} to \idx{titlecase}. This just uses
\ics{capitalisewords} by default.

\cssection{bibglsfirstuc}

\formatdef{bibglsfirstuc}
Converts the first letter of \meta{text} to \idx{uppercase}. This just uses
\ics{makefirstuc} by default.

\cssection{BibGlsNoCaseChange}

Behaves as \cs{NoCaseChange} within \bibgls, so it will prevent its argument
from being altered by options such as \csopt{short-case-change} or if the
interpreter encounters this command within a case-changing command such as
\cs{glsuppercase}. However, the definition provided in the \ext{glstex} file
simply expands to \meta{text} in the document without adding the command to the
case-changing exclusion list so it won't prevent any case-change within
the \LaTeX\ document.

\cssection{bibglsdefinitionindex}

\formatdef{bibglsdefinitionindex}
If \csopt{save-definition-index} has been set this command expands to the
\gls{definitionindex} of the entry identified by \meta{label}. This
command will only be provided in the \ext{glstex} file if 
\csopt{save-definition-index} has been set. However, the command is
always defined by the \texparserlib\
but will expand to empty if the associated resource option hasn't
been set.

\cssection{bibglsuseindex}

\formatdef{bibglsuseindex}
If \csopt{save-use-index} has been set this command expands to the
\gls{orderofuseindex} of the entry identified by \meta{label}. This
command will only be provided in the \ext{glstex} file if 
\csopt{save-use-index} has been set. However, the command is
always defined by the \texparserlib\
but will expand to empty if the associated resource option hasn't
been set.

\chapter{Converting Existing \iext{tex} to \iext{bib}}
\label{sec:tex2bib}
\setsecdepth{1}
\renewcommand{\cssectioncmd}{\subsubsection}

There are some supplementary command line tools provided with
\bibgls\ that can convert certain types of data from existing
documents to \iext{bib} files:
\begin{itemize}
  \item \idx{convertgls2bib} converts
data from commands like \gls{newglossaryentry} to \ext{bib} entry
data (see \sectionref{sec:gls2bib}).

  \item \idx{datatool2bib} converts \sty{datatool} databases to
\ext{bib} entry data (see \sectionref{sec:gls2bib}).
\end{itemize}

These command line tools share the same common switches listed in
\sectionref{sec:commonswitches}. Additionally, they share the switches
listed in \sectionref{sec:shared2bibswitches}, but they may also
have their specific switches, which are described in the relevant
section.

\begin{important}
It's likely that you will need to post-process the \ext{bib} files
or at least double-check that the conversion has been performed
adequately.
\end{important}

The \ext{bib} format has case-insensitive field names. However, the
source files input by the converter tool typically have
case-sensitive labels that correspond to the destination \ext{bib}
field. The converter tools will convert the fields to
\idx{lowercase} after any mappings have been applied.
Use \converterlongarg{field-case} to change this.
Additionally, the \ext{bib} format prohibits certain characters in
field names and entry labels (such as spaces), so these will be
stripped. The resulting field names must be non-empty.

The \texparserlib\ is used to parse the input files. The conversion
tool may automatically implement the relevant packages (such as
\sty{datatool} and \sty{datagidx} for \idx{datatool2bib}). 
However, the behaviour of some commands may be
modified to better suit the conversion process. 

Note that \ics{IfTeXParserLib} will expand to its first argument,
but \ics{IfNotBibGls} will also expand to its first argument, since
it's only defined to expand to its second argument by the \bibgls\
interpreter not by the \texparserlib\ core code.

Avoid any overly complicated code within the \ext{tex} file. The
\texparserlib\ isn't a \TeX\ engine! The input \ext{tex} file
doesn't need to be a complete document, but if you want certain
commands recognised from packages that the \texparserlib\ supports,
you'll need to include \ics{usepackage} in the \ext{tex} file. If
you want to quit parsing the \ext{tex} file at the start of the
document, use the \converterlongarg{preamble-only} switch.

\begin{important}
The \texparserlib\ recognises
\ics{input} and \ics{include} and so will also parse the referenced
file if either command is encountered.
If you have multiple files, it's better to pass just the relevant
file to the conversion tool rather than the main document file.
\end{important}

\section{Shared Conversion Tool Switches}
\label{sec:shared2bibswitches}

\converterarg{texenc}

The character \igls{encoding} of the source \ext{tex} file. If omitted, the
\idx{JVM}['s] default \gls{encoding} is assumed (which may or
may not be the same as the operating system's default).

\converterarg{bibenc}

The character \igls{encoding} of the \ext{bib} file. If omitted, the same
\gls{encoding} as the \ext{tex} file is assumed.

\converterarg{space-sub}

The \ext{bib} format doesn't allow spaces in labels. If your original
definitions in your \ext{tex} file have spaces, use this option to
replace spaces in labels. Each space will be substituted with
\meta{replacement}. Where applicable, the cross-referencing fields, \field{see},
\field{seealso} and \field{alias}, will also be adjusted, but any
references using \ics{gls} etc will have to be substituted manually 
(or use a global search and replace action in your text editor).
If you want to strip the spaces, use an empty string for
\meta{replacement}. You'll need to delimit this according to your
operating system. For example:
\begin{verbatim}
convertgls2bib --space-sub '' entries.tex entries.bib
\end{verbatim}
Alternatively, use \converterlongarg{field-map} to supply an
alternative field name.

\begin{important}
Note that all forbidden punctuation characters
(such as commas, equal signs and braces) will be automatically stripped.
\end{important}

\converterarg{preamble-only}

Stop parsing if the start of the \env{document} environment is found.

\converterarg{no-preamble-only}

Parse the entire file (default). Be prepared for a lot of
\qt{unknown command} warnings if you make the conversion tool parse an
entire document.

\converterarg{overwrite}

Allow existing \ext{bib} files to be overwritten. (Default unless
otherwise specified.)

\converterarg{no-overwrite}

Don't allow existing \ext{bib} files to be overwritten.

\converterarg{ignore-fields}

Omit all the fields listed in \meta{list} from the \ext{bib} file.
This option is cumulative. 
Note that in earlier versions of \idx{convertgls2bib} an empty list 
cancelled the list of ignored fields. As from version 4.0, you will
need to use \converterlongarg{no-ignore-fields} to clear the list.

Each item in the list should identify
the field by its original case-sensitive key.

\converterarg{no-ignore-fields}

Cancels the effect of \converterlongarg{ignore-fields}.

\converterarg{field-map}

Provides a mapping from source labels to the field names used in
the destination \ext{bib} file. The argument should be a
comma-separated list of \code{\meta{src-label}=\meta{dest-field}}
pairs.
This option is cumulative.

For \idx{convertgls2bib}, the source labels are the keys used in
commands such as \gls{newglossaryentry}. These should ordinarily not
need any mappings. However, you may, for example, want to convert
the user keys to something more appropriate.
For \idx{datatool2bib}, the source labels are the column keys.

\converterarg{no-field-map}

Cancels all field mappings.

\converterarg{field-case}

Specifies whether or not to change the case of the field name when
writing to the \ext{bib} file. Valid \meta{setting}:
\optfmt{none} (no case-change), \optfmt{lc} (to \idx{lowercase}, default),
\optfmt{uc} (to \idx{uppercase}), \optfmt{title} (to \idx{titlecase}),
or \optfmt{sentence} (to \idx{sentencecase}). Note that since spaces
aren't permitted in field names, the last two are equivalent.
The case-change is applied after mapping.

\converterarg{index-conversion}

Use \atentry{index} instead of \atentry{entry} if the
\field{description} is empty or, for \idx{convertgls2bib}, if the
\field{description} is simply \ics{nopostdesc} or
\ics{glsxtrnopostpunc}. (Only applies to terms that would otherwise
be converted to \atentry{entry}, such as those defined with
\gls{newglossaryentry} in the case of \idx{convertgls2bib}.)

\converterarg{no-index-conversion}

Don't convert \atentry{entry} to \atentry{index} (default).

\converterarg{log-file}

The log file is optional for the converter applications. It's
provided for use with the \texparserlib\ debugging modes 
(\commonlongarg{debug-mode}) which writes
logging information to this file (which can be sizeable).
A log file will automatically be created if debugging is enabled,
otherwise it will need to be explicitly created with
\converterlongarg{log-file} if required. The default name, if
automatically created, is the name of the converter application with
the \ext{log} file extension.

\begin{important}
Unlike \bibgls's \longarg{log-file} switch, the conversion
tools don't have a corresponding short switch.
\end{important}

\section{\glsfmttext{idx.convertgls2bib}: Conversion from \styfmt{glossaries} or
\styfmt{glossaries-extra}}
\label{sec:gls2bib}
\setsecdepth{2}

If you have already been using the \styfmt{glossaries} or
\styfmt{glossaries-extra} package with a large file containing all your
definitions using commands like \gls{newglossaryentry}, then you can
use the supplementary tool \idx{convertgls2bib} to convert the definitions
to the \ext{bib} format required by \bibgls. The syntax is:
\begin{alltt}
convertgls2bib \oargm{options} \meta{tex file} \meta{bib file}
\end{alltt}
where \meta{tex file} is the \ext{tex} file and \meta{bib file} is
the \ext{bib} file. This application is less secure than \bibgls\ as
it doesn't use \idx{kpsewhich} to check \texmfcnf{openinany} and 
\texmfcnf{openoutany}. Take care not to accidentally overwrite
existing \ext{bib} files as there's no check to determine if
\meta{bib file} already exists with the default \converterlongarg{overwrite}. 

If the \ext{bib} extension is missing from \meta{bib file}, it will
be added. The extension is required for \meta{tex file}.

\subsection{Command Line Arguments}

The \meta{options} specific to \idx{convertgls2bib} are described below.
This is in addition to those described in \sectionref{sec:commonswitches}
and \sectionref{sec:shared2bibswitches}. Note that
\convertglsbiblongarg{split-on-type} overrides the default
\converterlongarg{overwrite} setting.

\convertglsbibarg{ignore-sort}

Omit the \field{sort} field.  This is the default since \bibgls\ can
work out a more intuitive sort value than either \idx!{makeindex} or
\idx!{xindy}.
This option is automatically set if the \field{sort} field is
included in \converterlongarg{ignore-fields}.

\convertglsbibarg{no-ignore-sort}

Don't ignore the \field{sort} field.

\convertglsbibarg{ignore-type}

Omit the \field{type} field in the \ext{bib} file.  You may find it
more flexible not to be locked into a specific glossary type if you
have a large database of entries.
This option is automatically set if the \field{type} field is
included in \converterlongarg{ignore-fields}.

\convertglsbibarg{no-ignore-type}

Don't omit the \field{type} field (default unless
\convertglsbiblongarg{split-on-type}).

\convertglsbibarg{ignore-category}

Omit the \field{category} field in the \ext{bib} file.
This option is automatically set if the \field{category} field is
included in \converterlongarg{ignore-fields}.

\convertglsbibarg{no-ignore-category}

Don't omit the \field{category} field (default unless
\convertglsbiblongarg{split-on-category}).

\convertglsbibarg{split-on-type}

Split the entries into separate files according to the \field{type}
field. Any entries that have the \field{type} field set to \ics{glsdefaulttype}
or that don't have the \field{type} field set and there's no default provided
by the command used to define the entry (see below) then the
\atentryfmt{\meta{entry}} data will be written to the main \meta{bib
file}. Otherwise entries will be written to the split file
(in the same directory as \meta{bib file}). 

The split file name depends on whether or not the
\convertglsbiblongarg{split-on-category} switch has also been used.  With both
and if the category and field values are different then
the file name is \metametafilefmt{}{type}{-}{category}{.bib} otherwise
it's \metafilefmt{}{type}{.bib}.

Commands that have a default type are as follows:
\begin{itemize}
\item \gls{newabbreviation}, \gls{newacronym}, \ics{oldacronym},
\gls{newdualentry}:
the default type is assumed to be \code{abbreviations} (regardless
of the definition of \ics{acronymtype} or \ics{glsxtrabbrvtype});
\item \gls{glsxtrnewsymbol}: the default type is assumed to be \code{symbols}; 
\item \gls{glsxtrnewnumber}: the default type is assumed to be \code{numbers}; 
\item \gls{newterm}: the default type is assumed to be \code{index}. 
\end{itemize}

This option automatically implements \convertglsbiblongarg{ignore-type} and
\converterlongarg{no-overwrite}.

\convertglsbibarg{no-split-on-type}

Don't split the entries into separate files according to their type
(default).

\convertglsbibarg{split-on-category}

Split the entries into separate files according to the \field{category}.
If the \field{category} field isn't present and there's no default provided
by the command used to define the entry (see below) then the
\atentryfmt{\meta{entry}} data will be written to the main \meta{bib
file}. Otherwise entries will be written to the split file
(in the same directory as \meta{bib file}). 

The split file name depends on whether or not the
\convertglsbiblongarg{split-on-type} switch has also been used.  With both and
if the category and field values are different then
the file name is \metametafilefmt{}{type}{}{category}{.bib} otherwise
it's \metafilefmt{}{category}{.bib}.

Commands that have a default category are as follows:
\begin{itemize}
\item \gls{newabbreviation}, \gls{newacronym}, \ics{oldacronym},
\gls{newdualentry}:
the default category is assumed to be \code{abbreviation};
\item \gls{glsxtrnewsymbol}: the default category is assumed to be \code{symbol}; 
\item \gls{glsxtrnewnumber}: the default category is assumed to be \code{number}; 
\item \gls{newterm}: the default category is assumed to be \code{index}. 
\end{itemize}

For example, if you have both \convertglsbiblongarg{split-on-type} and
\convertglsbiblongarg{split-on-category}, then the default file name for
\gls{newabbreviation} will be
\filefmt{abbreviations-abbreviation.bib} but the default file name
for \gls{newterm} will be \filefmt{index.bib}. Whereas if you only
have \convertglsbiblongarg{split-on-category} and not
\convertglsbiblongarg{split-on-type}, then then default file name for
\gls{newabbreviation} will be \filefmt{abbreviation.bib}.

This option automatically implements \convertglsbiblongarg{ignore-category} and
\converterlongarg{no-overwrite}.

\convertglsbibarg{no-split-on-category}

Don't split the entries into separate files according to their
category (default).


\convertglsbibarg{absorb-see}

Absorb any cross-referencing information identified with 
\ics{glssee} or \ics{glsxtrindexseealso} commands into
the corresponding entry (default).

\convertglsbibarg{no-absorb-see}

Don't absorb any cross-referencing information identified with \ics{glssee} 
or \ics{glsxtrindexseealso} commands.

\convertglsbibarg{internal-field-map}

Appends the given internal field to key mappings to the predefined
list. This action is cumulative. The mappings are needed for
\gls{glssetexpandfield} and \gls{glssetnoexpandfield}, which use
\idx{internalfields} for their arguments.
Note that this is different from \converterlongarg{field-map} which
maps the key used in commands like \gls{newglossaryentry} to
the corresponding field label written the \ext{bib} file.

\subsection{Recognised Commands}

The \idx{convertgls2bib} application recognises the commands listed below as well as
some standard commands such as \ics{newcommand}. The \texparserlib\
is instructed to skip its usual implementation of the
\sty{glossaries} and \sty{glossaries-extra} packages as only a
limited set of commands need to be defined.

In all cases below, if \meta{\keyvallist} contains:
\begin{codeenv*}
\field{see}=\marg{[\ics{seealsoname}]\meta{label(s)}}
\end{codeenv*}
or
\begin{codeenv*}
\field{see}=\marg{[\ics{alsoname}]\meta{label(s)}}
\end{codeenv*}
this will be substituted with:
\begin{codeenv*}
\field{seealso}=\margm{label(s)}
\end{codeenv*}
For example:
\begin{codeenv}
\gls{newterm}\oarg{\field{see}=\marg{[\cs{seealsoname}]goose}}\marg{duck}
\end{codeenv}
will be written as:
\begin{codeenv}
\atentry{index}\marg{duck,
  \field{seealso} = \marg{goose}
}
\end{codeenv}
Note that it won't convert \code{\field{see}=\marg{[see also]\meta{labels}}}.
If you have used explicit text instead of \ics{seealsoname} or
\ics{alsoname} then consider performing a global search and replace
on your file using your text editor.

Additionally, if \meta{\keyvallist} contains:
\begin{codeenv}
\field{type}=\marg{\ics{glsdefaulttype}}
\end{codeenv}
then this field will be ignored. (This \field{type} value is
recommended in \meta{\keyvallist} when loading files with
\gls{loadglsentries}\oargm{type}\margm{file} to allow the optional
argument to set the \field{type}. With \bibgls\ you can use the
\csopt{type} option instead.)

\cssection{glsexpandfields}
The base \styfmt{glossaries} package provides:
\formatdef{glsexpandfields}
If present, this instructs \idx{convertgls2bib} to expand all fields
except for those explicitly identified by \gls{glssetnoexpandfield}.
Remember that there are many commands that aren't recognised by
\idx{convertgls2bib} so it may not be possible to correctly expand
field values. Conversely, there are some commands that will be
expanded by \idx{convertgls2bib} that aren't expandable in \TeX\
(such as \ics{MakeUppercase} and \ics{char}).

\cssection{glsnoexpandfields}
The base \styfmt{glossaries} package provides:
\formatdef{glsnoexpandfields}
If present, this instructs \idx{convertgls2bib} to not expand fields
unless explicitly identified by \gls{glssetexpandfield}.

\cssection{glssetexpandfield}
The base \styfmt{glossaries} package provides:
\formatdef{glssetexpandfield}
If present, this instructs \idx{convertgls2bib} to expand the
given field, even if \gls{glsnoexpandfields} has been used.
Note that the argument should be the internal field label.
The default set of mappings are recognised. Any additional mappings
will need to be added with
\convertglsbiblongarg{internal-field-map}.

\cssection{glssetnoexpandfield}
The base \styfmt{glossaries} package provides:
\formatdef{glssetnoexpandfield}
If present, this instructs \idx{convertgls2bib} to not expand the
given field, even if \gls{glsexpandfields} has been used.
Unlike the default behaviour with the \sty{glossaries} package, there are no
fields explicitly switched off by default with \idx{convertgls2bib}.

\cssection{loadglsentries}
The base \styfmt{glossaries} package provides:
\formatdef{loadglsentries}
This locally redefines \ics{glsdefaulttype} to \meta{type} (if the optional argument is supplied)
and inputs \meta{file}. If the \meta{file} doesn't have an extension, \ext{tex} is assumed.
(The \texparserlib\ also recognises \ics{input}.)

In general, it's better to simply use \meta{file} as the source file
when running \idx{convertgls2bib} instead of using the main document
file as the source.

\cssection{newglossaryentry}
The base \styfmt{glossaries} package provides:
\formatdef{newglossaryentry}
This is converted to:
\begin{codeenv}
\atentry{entry}\marg{\meta{label},
  \meta{\keyvallist}
}
\end{codeenv}
\ics{newentry} is recognised as a synonym of \gls{newglossaryentry}.

\cssection{provideglossaryentry}
The base \styfmt{glossaries} package provides:
\formatdef{provideglossaryentry}
This is converted to:
\begin{codeenv}
\atentry{entry}\marg{\meta{label},
  \meta{\keyvallist}
}
\end{codeenv}
but only if \meta{label} hasn't already been defined.

\cssection{longnewglossaryentry}
The base \styfmt{glossaries} package provides:
\formatdef{longnewglossaryentry}
This is converted to:
\begin{codeenv}
\atentry{entry}\marg{\meta{label},
 \meta{\keyvallist},
 \field{description} = \margm{description}
}
\end{codeenv}
The starred version provided by the \styfmt{glossaries-extra} package
is also recognised. The unstarred version strips trailing spaces
from \meta{description}. (This doesn't add \ics{nopostdesc}, but
\styfmt{glossaries-extra} defaults to \styopt{nopostdot}.)

\cssection{longprovideglossaryentry}

The base \styfmt{glossaries} package provides:
\formatdef{longprovideglossaryentry}
As above, but only if \meta{label} hasn't already been defined.

\cssection{newterm}

The base \styfmt{glossaries} package provides:
\formatdef{newterm}
(when the \styopt{index} option is used).
This is converted to:
\begin{codeenv}
\atentry{index}\marg{\meta{label},
 \meta{\keyvallist}
}
\end{codeenv}
if the optional argument is present, otherwise it's just converted
to:
\begin{codeenv}
\atentry{index}\marg{\meta{label}}
\end{codeenv}

If \converterlongarg{space-sub} is used and \meta{label} contains one or
more spaces, then \field{name} will be set if not included in
\meta{\keyvallist}. For example, if \filefmt{entries.bib}
contains:
\begin{codeenv}
\gls{newterm}\marg{sea lion}
\gls{newterm}\oarg{\field{seealso}=\marg{sea lion}}\marg{seal}
\end{codeenv}
then:
\begin{verbatim}
convertgls2bib --space-sub '-' entries.bib entries.tex
\end{verbatim}
will write the terms to \filefmt{entries.tex} as:
\begin{codeenv}
\atentry{index}\marg{sea-lion,
  \field{name} = \marg{sea lion}
}
\strut
\atentry{index}\marg{seal,
  \field{seealso} = \marg{sea-lion}
}
\end{codeenv}
whereas just:
\begin{verbatim}
convertgls2bib entries.bib entries.tex
\end{verbatim}
will write the terms to \filefmt{entries.tex} as:
\begin{codeenv}
\atentry{index}\marg{sea lion}
\strut
\atentry{index}\marg{seal,
  \field{seealso} = \marg{sea lion}
}
\end{codeenv}
which will cause a problem when the \ext{bib} file is parsed
by \bibgls\ (and will probably also cause a problem for 
bibliographic management systems).

\cssection{newabbreviation}
The \styfmt{glossaries-extra} package provides:
\formatdef{newabbreviation}
This is converted to:
\begin{codeenv}
\atentry{abbreviation}\marg{\meta{label},
 \field{short} = \margm{short},
 \field{long} = \margm{long},
 \meta{\keyvallist}
}
\end{codeenv}
if the optional argument is present, otherwise it's converted to:
\begin{codeenv}
\atentry{abbreviation}\marg{\meta{label},
 \field{short} = \margm{short},
 \field{long} = \margm{long}
}
\end{codeenv}

\cssection{newacronym}
The base \styfmt{glossaries} package provides:
\formatdef{newacronym}
(which is redefined by \styfmt{glossaries-extra} to use
\gls{newabbreviation}).

As above but uses \atentry{acronym} instead. The base package also
provides \ics{oldacronym}, which emulates the way abbreviations were
defined with the precursor \styfmt{glossary} package. This has
different syntax to \gls{newacronym} but is also recognised by \idx{convertgls2bib}
and is converted to \atentry{acronym}.

\cssection{glsxtrnewsymbol}
The \styfmt{glossaries-extra} package provides:
\formatdef{glsxtrnewsymbol}
(when the \styopt{symbols} option is used).
This is converted to:
\begin{codeenv}
\atentry{symbol}\marg{\meta{label},
 \field{name} = \margm{symbol}
}
\end{codeenv}
if the optional argument is missing, otherwise it's converted to:
\begin{codeenv}
\atentry{symbol}\marg{\meta{label},
 \field{name} = \margm{symbol},
 \meta{\keyvallist}
}
\end{codeenv}
unless \meta{\keyvallist} contains the \field{name} field,
in which case it's converted to:
\begin{codeenv}
\atentry{symbol}\marg{\meta{label},
 \meta{\keyvallist}
}
\end{codeenv}
\ics{newsym} is recognised as a synonym for \gls{glsxtrnewsymbol}.

\cssection{glsxtrnewnumber}

The \styfmt{glossaries-extra} package provides:
\formatdef{glsxtrnewnumber}
(when the \styopt{numbers} option is used).
This is converted to:
\begin{codeenv}
\atentry{number}\marg{\meta{label},
 \field{name} = \margm{label}
}
\end{codeenv}
if the optional argument is missing, otherwise it's converted to:
\begin{codeenv}
\atentry{number}\marg{\meta{label},
 \field{name} = \margm{label},
 \meta{\keyvallist}
}
\end{codeenv}
if \field{name} isn't listed in \meta{\keyvallist},
otherwise it's converted to:
\begin{codeenv}
\atentry{number}\marg{\meta{label},
 \meta{\keyvallist}
}
\end{codeenv}
\ics{newnum} is recognised as a synonym for \gls{glsxtrnewnumber}.

\cssection{newdualentry}

\formatdef{newdualentry}
This command isn't provided by either \styfmt{glossaries} or
\styfmt{glossaries-extra} but is used as an example in the
\sty{glossaries} user manual~\cite{glossaries} and in the sample file
\file{sample-dual.tex} that accompanies the \sty{glossaries}
package. Since this command seems to be used quite a bit (given the
number of times it crops up on sites like
\href{https://tex.stackexchange.com/}{\TeX\ on StackExchange}), 
\idx{convertgls2bib} also supports it unless this command is
defined using \ics{newcommand} or \ics{renewcommand} in the 
input file. In which case the default definition will be overridden.

If the command definition isn't overridden, then it's converted to:
\begin{codeenv}
\atentry{dualabbreviationentry}\marg{\meta{label},
  \field{short} = \margm{short},
  \field{long} = \margm{long},
  \field{description} = \margm{description},
  \meta{\keyvallist}
}
\end{codeenv}
if \meta{\keyvallist} is supplied, otherwise it's converted to:
\begin{codeenv}
\atentry{dualabbreviationentry}\marg{\meta{label},
  \field{short} = \margm{short},
  \field{long} = \margm{long},
  \field{description} = \margm{description}
}
\end{codeenv}

For example, if the original \ext{tex} file contains:
\begin{codeenv}
\cs{newcommand}*\marg{\gls{newdualentry}}[5][]\marg{\comment{}
  \gls{newglossaryentry}\marg{main-\idx{param}2}\marg{\field{name}=\marg{\idx{param}4},\comment{}
    \field{text}=\marg{\idx{param}3\cs{glsadd}\marg{\idx{param}2}},\comment{}
    \field{description}=\marg{\idx{param}5},\comment{}
    \idx{param}1
  }\comment{}
  \gls{newacronym}\marg{\idx{param}2}\marg{\idx{param}3\cs{glsadd}\marg{main-\idx{param}2}}\marg{\idx{param}4}\comment{}
}
\strut
\gls{newdualentry}\marg{svm}\comment{label}
  \marg{SVM}\comment{abbreviation}
  \marg{support vector machine}\comment{long form}
  \marg{Statistical pattern recognition technique}\comment{description}
\end{codeenv}
then the \ext{bib} file will contain:
\begin{codeenv}
\atentry{entry}\marg{main-svm,
  \field{name} = \marg{support vector machine},
  \field{description} = \marg{Statistical pattern recognition technique},
  \field{text} = \marg{SVM\cs{glsadd}\marg{svm}}
}
\strut
\atentry{acronym}\marg{svm,
  \field{short} = \marg{SVM\cs{glsadd}\marg{main-svm}},
  \field{long} = \marg{support vector machine}
}
\end{codeenv}
since \gls{newdualentry} was defined with \ics{newcommand}. However,
if the original file uses \ics{providecommand} or omits the
definition of \gls{newdualentry}, then the \ext{bib} file will
contain:
\begin{codeenv}
\atentry{dualabbreviationentry}\marg{svm,
  \field{short} = \marg{SVM},
  \field{description} = \marg{Statistical pattern recognition technique},
  \field{long} = \marg{support vector machine}
}
\end{codeenv}

\section{\glsfmttext{idx.datatool2bib}: Conversion from
\styfmt{datatool}}
\label{sec:datatool2bib}

A \isty{datatool} database can be converted to a \ext{bib} file
with the supplementary tool \idx{datatool2bib}. The syntax is:
\begin{alltt}
datatool2bib \oargm{options} \meta{tex file} \meta{bib file}
\end{alltt}
where \meta{tex file} is the \ext{tex} file and \meta{bib file} is
the \ext{bib} file. This application is less secure than \bibgls\ as
it doesn't use \idx{kpsewhich} to check \texmfcnf{openinany} and 
\texmfcnf{openoutany}. Take care not to accidentally overwrite
existing \ext{bib} files as there's no check to determine if
\meta{bib file} already exists with the default \converterlongarg{overwrite}. 

This tool is primarily intended to convert \sty{datagidx} databases,
so it will automatically assume that \sty{datagidx} is loaded.
It can, however, be used for general databases, but you will need to
choose an appropriate column with unique values for the entry label (with
\datatoolbiblongarg{label} or use \datatoolbiblongarg{auto-label}).

The \meta{tex file} may be a complete document (in which case, you
may want to use \converterlongarg{preamble-only} if the database
commands are all in the preamble) or may be a file that just
contains the database construction commands. The conversion tool
uses the \texparserlib\ which has some limited support for
\sty{datatool}, including database construction commands and file
input commands \ics{DTLloaddb}, \ics{DTLloaddbtex} and the newer
\gls{DTLread}.

For example, suppose I have a document file called
\filefmt{myDoc.tex} that contains:
\begin{codeenv}
\csfmt{documentclass}\marg{article}
\csfmt{usepackage}\marg{datatool}
\gls{DTLnewdb}\marg{mydata}
\gls{DTLnewrow}\marg{mydata}
\gls{DTLnewdbentry}\marg{mydata}\marg{Forename}\marg{John}
\gls{DTLnewdbentry}\marg{mydata}\marg{Surname}\marg{Smith}
\comment{...}
\cmd{begin}\marg{document}
\comment{...}
\cmd{end}\marg{document}
\end{codeenv}
Then I can extract this data with:
\begin{codeenv}
datatool2bib \converterlongarg{preamble-only} myDoc.tex mydata.bib
\end{codeenv}
If the database commands are instead in a separate file called, say,
\filefmt{mydata.tex} which is simply input in \filefmt{myDoc.tex}
then \filefmt{mydata.tex} can then be used for the \meta{tex file}
input file:
\begin{verbatim}
datatool2bib mydata.tex mydata.bib
\end{verbatim}
Note that the \texparserlib\ also recognises commands such as
\gls{dtlexpandnewvalue}. If this is in \filefmt{myDoc.tex} rather
than in \filefmt{mydata.tex} and only \filefmt{mydata.tex} is parsed
then the \texparserlib\ won't encounter \gls{dtlexpandnewvalue} and
so won't enable new value expansion. You can, however, use
\code{\datatoolbiblongarg{setup} expand-new-value} to switch on new
value expansion.

If the data source is in a CSV file, then the \texparserlib\ can
parse it but only if the file is referenced in \ics{DTLloaddb} (pre
\sty{datatool} v3.0) or \gls{DTLread} (\sty{datatool} v3.0+). 
For the newer \gls{DTLread} command, the \code{format=\iext{csv}} setting is
required to parse CSV files or \code{format=\iext{tsv}} for TSV files.
If the file contains \LaTeX\ commands, \code{content=tex} is
required otherwise \code{content=literal} is required.

So in order to convert a CSV file called \filefmt{mydata.csv} to 
a \ext{bib} file, the \texparserlib\ needs to parse:
\begin{codeenv}
\gls{DTLread}\oarg{format=csv,content=literal}\marg{mydata.csv}
\end{codeenv}
Rather than having to parse the entire document source file or
create a stub file that simply contains this command, you can use
the \datatoolbiblongarg{read} switch instead:
\begin{codeenv}
datatool2bib \datatoolbiblongarg{read} format=csv,content=literal mydata.csv mydata.bib
\end{codeenv}
This is equivalent to creating a stub file with just the
\gls{DTLread} line.

In general, better results are obtained if \idx{datatool2bib} parses
\gls{datagidx.newterm} and \gls{datagidx.newacro} instead of the
actual \sty{datagidx} databases. The custom \sty{datagidx} databases
have a number of columns for internal use that are not applicable to
\bibgls.

When you run \idx{datatool2bib} from the command line, there will be
two lines reporting the number of databases found by parsing the
input. For example:
\begin{verbatim}
1 database (datatool) found
0 databases (datagidx) found
\end{verbatim}
The first line indicates the number of regular \sty{datatool}
databases that were found by the \texparserlib. The second line
indicates the number of \sty{datagidx} databases that were detected
by scanning for \gls{datagidx.newterm} and \gls{datagidx.newacro}.

In the first case, the source field labels will correspond to the
column keys (such as \optfmt{Label}, \optfmt{Name},
\optfmt{Description} and \optfmt{Symbol}), whereas in the second
case the source field labels will correspond to the option keys for
\gls{datagidx.newterm} and \gls{datagidx.newacro} (such as \optfmt{description} and
\optfmt{symbol}, but not the \optfmt{label} key, which will be
dropped from the list after its value is saved elsewhere).

\subsection{Command Line Arguments}

The \meta{options} specific to \idx{datatool2bib} are described below.
This is in addition to those described in \sectionref{sec:commonswitches}
and \sectionref{sec:shared2bibswitches}.

\datatoolbibarg{label}

Obtain the \ext{bib} entry label from the column identified by the
\meta{column-key}. Note that the column key is the original
case-sensitive column label not the mapped field.
The label will be obtained by copying the value from the given
column and processing according to \converterlongarg{space-sub}
(and stripping forbidden characters). This option is not applicable
to \gls{datagidx.newterm} and \gls{datagidx.newacro}.

\begin{important}
The chosen column should have unique values (and should still be
unique after processing).
\end{important}

The default setting is \code{\datatoolbiblongarg{label} Label}
since this is the column used for labels with \styfmt{datagidx} (but
is only applicable if you save the database with \cs{DTLwrite}
rather than having \idx{datatool2bib} parse \gls{datagidx.newterm} 
and \gls{datagidx.newacro}).

If you are converting a general \sty{datatool} database, either
choose a different column or switch on the auto label.

\datatoolbibarg{auto-label}

If there is no appropriate column for a label, this option may be
used to automatically generate the label in the form
\meta{prefix}\meta{n}. If
\datatoolbiblongarg{auto-label-prefix} is set, that will provide the
\meta{prefix}, otherwise the prefix will match the database label.

\datatoolbibarg{no-auto-label}

Switches off automatic labelling. The labels will be created from
the column identified by \datatoolbiblongarg{label}.

\datatoolbibarg{auto-label-prefix}

Sets the prefix for use with \datatoolbiblongarg{auto-label}.
This option automatically implements
\datatoolbiblongarg{auto-label}.

\datatoolbibarg{read}

Instead of parsing the provided \ext{tex} file in the usual manner, 
the \datatoolbiblongarg{read} option will act as though the input
is:
\begin{codeenv}
\gls+{DTLread}\oargm{options}\marg{filename}
\end{codeenv}
where \meta{filename} is the input filename supplied to
\idx{datatool2bib}. This can be use where the source file is a
CSV file rather than a \LaTeX\ file.

\datatoolbibarg{setup}
Implements \code{\gls{DTLsetup}\margm{options}} before parsing.
Note that this can be counteracted by any changes to the settings
within the file being parsed.

\datatoolbibarg{save-datum}
Equivalent to:
\begin{codeenv}
\datatoolbiblongarg{save-value} "-value" \datatoolbiblongarg{save-currency} "-currency"
\end{codeenv}
For example, if the CSV file
\filefmt{products.csv} contains:
\begin{alltt}
ID,Product,Quantity,Price
1,Sample,3,"\$1,234.50"
\end{alltt}
then with:
\begin{codeenv}
datatool2bib \datatoolbiblongarg{label} Product \datatoolbiblongarg{read} "format=csv,content=literal" products.csv products.bib
\end{codeenv}
The \filefmt{products.bib} file will contain:
\begin{codeenv}
\atentry{entry}\marg{Sample,
  \fieldfmt{id} = \marg{1},
  \fieldfmt{product} = \marg{Sample},
  \fieldfmt{quantity} = \marg{3},
  \fieldfmt{price} = \marg{\cs{cs.dollar}1,234.5}
}
\end{codeenv}
Whereas with:
\begin{codeenv}
datatool2bib \datatoolbiblongarg{save-datum} \datatoolbiblongarg{label} Product \datatoolbiblongarg{read} "format=csv,content=literal" products.csv products.bib
\end{codeenv}
\begin{codeenv}
\atentry{entry}\marg{Sample,
  \fieldfmt{id} = \marg{1},
  \fieldfmt{product} = \marg{Sample},
  \fieldfmt{quantity} = \marg{3},
  \fieldfmt{price} = \marg{\cs{cs.dollar}1,234.5},
  \fieldfmt{price-value} = \marg{1234.5},
  \fieldfmt{price-currency} = \marg{\cs{cs.dollar}}
}
\end{codeenv}
Note that the original number formatting is retained in the
\fieldfmt{price} field but the plain numeric value is saved in the
\fieldfmt{price-value} field and the currency symbol is saved in the
\fieldfmt{price-currency} field. These custom fields will all need
to be defined or aliased when the \ext{bib} file is read by \bibgls.

For further information on the \sty{datatool} datum format see the
\sty{datatool} v3.0+ user manual. You need to set the
decimal and number group characters with \ics{DTLsetnumberchars} and
the currency symbol with \ics{DTLnewcurrencysymbol} for the values
to be parsed correctly.

\datatoolbibarg{no-save-datum}
Equivalent to:
\begin{codeenv}
\datatoolbiblongarg{no-save-value} \datatoolbiblongarg{no-save-currency}
\end{codeenv}

\datatoolbibarg{save-value}

If a field contains a formatted numeric value or currency (according
to the current \sty{datatool} settings), the numeric value will be
saved in the field \meta{field}\meta{prefix} if it's different from
the value in \meta{field}.

\datatoolbibarg{no-save-value}

Don't save numeric values in a separate field (default).

\datatoolbibarg{save-currency}

If a field contains a formatted currency (according
to the current \sty{datatool} settings), the currency symbol will be
saved in the field \meta{field}\meta{prefix}.

\datatoolbibarg{no-save-currency}

Don't save the currency symbol in a separate field (default).

\datatoolbibarg{split}

If multiple databases are found in the input file, create a separate
output file for each database. If the output \meta{bib file} provided to
\idx{datatool2bib} is given as \filefmt{\meta{base}.bib} then each 
database \meta{dbname} will be saved in the file
\filefmt{\meta{base}-\meta{dbname}.bib}.

\datatoolbibarg{no-split}

All data found from parsing the input file will be saved in the
single output \meta{bib file} (default).

\datatoolbibarg{detect-symbols}

Attempt to detect entries that should be \atentry{symbol} or 
\atentry{number} based on the value of the \field{name} field.
This tries to expand the value of the name field with
\ics{DTLgidxParen} and \ics{DTLgidxIgnore} locally defined to
discard their argument. If the result is empty or if the first
character of the result is not alphabetic then,
if the result parses as an integer, \atentry{number}
will be chosen otherwise \atentry{symbol} will be chosen.
(An empty result can occur if the \field{name} field solely consists
of commands that are unknown to the \texparserlib, which suggests
symbols rather than words.)

This setting is only applicable for entries defined with 
\sty{datagidx}['s] \cs{newterm} command, not with general
\sty{datatool} databases. Note that if this
results in a change to the entry type from its default
\atentry{entry} then \converterlongarg{index-conversion} won't
convert the type to \atentry{index}, regardless of whether or
not the \field{description} field is set.

\datatoolbibarg{no-detect-symbols}

Don't attempt to detect symbols (default).

\datatoolbibarg{numeric-locale}

For use with \datatoolbiblongarg{detect-symbols}, this switch
identifies the locale used for numeric parsing to determine whether
or not the \atentry{number} entry type should be used.
If omitted, the \idx{JVM}['s] default locale is assumed.

\datatoolbibarg{adjust-gls}

This setting is the default.
Any instance of commands like \ics{datagidx.gls} found in field
values will have their labels adjusted, if they have been modified
due to stripping spaces etc. Also, the syntax of \sty{datagidx}['s]
\ics{datagidx.gls} is slightly different from \sty{glossaries}['s]
\ics{gls} as the format is incorporated at the start of the label.
If this is detected, the format will be added as an optional
argument, matching the \sty{glossaries} syntax.

Additionally, commands that have different names from those provided
by \sty{glossaries} will be replaced with the closest match. For
example, \ics{datagidx.glsnl} will be replaced with \code{\ics{gls}*}
(no hyperlink).

\datatoolbibarg{no-adjust-gls}

Don't adjust commands like \ics{datagidx.gls} in field values.

\datatoolbibarg{dependency-field}

Sets the name of the dependency field. Has no effect with
\datatoolbiblongarg{no-strip-glsadd}. With
\datatoolbiblongarg{strip-glsadd}, the label from the
stripped \ics{datagidx.glsadd} argument will be added to the field identified
by \meta{field}. The default is \fieldfmt{dependency} which will be ignored
by \bibgls\ unless instructed otherwise (via resource options such
as \csopt{field-aliases} or by defining the field with commands such
as \cs{glsaddstoragekey}).

For example, \meta{field} could be set to \field{see} (but be aware
that this won't be merged with any existing \field{see} value for
the same entry) or the new document to use the created \ext{bib}
file could define \meta{field} and reference it in
\csopt{dependency-fields}.

The \meta{field} will have any invalid content stripped and will be
adjusted to match the \converterlongarg{field-case} setting.
If the \meta{field} ends up as an empty string, it will behave as 
\datatoolbiblongarg{no-dependency-field}.

\datatoolbibarg{no-dependency-field}

Don't save any labels stripped with
\datatoolbiblongarg{strip-glsadd}.

\datatoolbibarg{strip}

Switch on all strip options.

\datatoolbibarg{no-strip}

Switch off all strip options.

\datatoolbibarg{strip-glsadd}

Any instance of \ics{datagidx.glsadd} found in field values will be
stripped, along with the argument. Note that this command is
provided by \sty{datagidx} and, unlike the analogous
\sty{glossaries} command, it doesn't have an optional argument but
instead allows the format option to be included in the mandatory
argument. 

This setting is the default as \cs{glsadd} is rarely needed in field
values with \bibgls, as dependencies and the \csopt{selection}
criteria are usually sufficient.

\datatoolbibarg{no-strip-glsadd}

No search is made for \ics{datagidx.glsadd} in field values. Note
that this means that no adjustments for the different syntax will be
made.

\datatoolbibarg{strip-acronym-font}

Any instance of \ics{acronymfont} found in field values will be
strip and the braces around the argument will be removed.

This setting is the default since \cs{acronymfont} should not be
used with \sty{glossaries-extra}.

\datatoolbibarg{no-strip-acronym-font}

No search is made for \cs{acronymfont} in the field values. Note
that the use of \cs{acronymfont} is deprecated with
\sty{glossaries-extra} as it has a different abbreviation mechanism
to the base \sty{glossaries} package.

\datatoolbibarg{strip-case-change}

The following \idxpl{casecommands} will be stripped from field
values and the braces around the argument will be removed:
\csfmt{uppercase}, \csfmt{lowercase},
\ics{MakeTextUppercase}, \ics{MakeTextLowercase},
\ics{MakeUppercase}, \ics{MakeLowercase}, and the \sty{mfirstuc}
commands.

\datatoolbibarg{no-strip-case-change}

Don't strip the \idxpl{casecommands} from field values (default).

\subsection{Recognised Commands}

This section lists \isty{datatool} and \isty{datagidx} commands that
are recognised by \idx{datatool2bib}. (There may be other commands
that are also recognised but aren't relevant to the conversion
process.) See the \sty{datatool} user manual for further
information.

\cssection{DTLsetup}
Some but not all \sty{datatool} settings are
recognised by the \texparserlib. Of those that are recognised, the
settings pertinent to \idx{datatool2bib} are: \optfmt{default-name},
\optfmt{store-datum}, \optfmt{new-value-trim},
\optfmt{new-value-expand}, and \optfmt{io}.

The \optfmt{io} setting
takes a comma-separated list of default I/O options applicable to
\gls{DTLread} (and \cs{DTLwrite}). Recognised I/O options:
\optfmt{name}, \optfmt{keys}, \optfmt{headers}, \optfmt{expand}, 
\optfmt{format}, \optfmt{add-delimiter}, \optfmt{csv-escape-chars},
\optfmt{csv-content}, \optfmt{csv-blank}, \optfmt{csv-skip-lines}
(and synonym \optfmt{omitlines}), \optfmt{no-header} (and synonym
\optfmt{noheader}), \optfmt{auto-keys} (and synonym
\optfmt{autokeys}), \optfmt{overwrite}, \optfmt{load-action},
\optfmt{delimiter}, and \optfmt{separator}.

\cssection{DTLread}
Loads a database from the given file, where the format should be
identified in \meta{options} with the \optfmt{format} key. The
default is \code{format=csv}, which indicates that the file is a CSV
file. For tab-separated files, use \code{format=tsv}. If the CSV/TSV
file contains \TeX\ markup within the cells, use
\code{csv-content=tex} otherwise use \code{csv-content=literal}.

Note that:
\begin{codeenv}
datatool2bib \datatoolbiblongarg{read} \meta{io-options} \meta{filename} \meta{bib file}
\end{codeenv}
is equivalent to creating a file \filefmt{otherfile.tex} with a single line:
\begin{codeenv}
\gls{DTLread}\oargm{io-options}\margm{filename}
\end{codeenv}
and then running:
\begin{alltt}
datatool2bib otherfile.tex \meta{bib file}
\end{alltt}

\begin{important}
The \texparserlib\ also recognises \ics{DTLwrite} but
\idx{datatool2bib} redefines the command to do nothing.
\end{important}

\cssection{dtlexpandnewvalue}
New values will be expanded before being added to the database.
Note that the \texparserlib\ may not expand commands in the same way
as \TeX, and it won't be able to expand unknown commands.

\cssection{dtlnoexpandnewvalue}
New values will not be expanded before being added to the database
(default).

\cssection{DTLnewdb}
Defines a new database with the given label.

\cssection{DTLnewrow}
Creates a new row in the database identified the given label.

\cssection{DTLnewdbentry}
Adds a new entry to the final row of the given database.

For example:
\begin{codeenv}
\gls+{DTLnewdb}\marg{mydata}
\gls+{DTLnewrow}\marg{mydata}
\gls+{DTLnewdbentry}\marg{mydata}\marg{Forename}\marg{John}
\gls{DTLnewdbentry}\marg{mydata}\marg{Surname}\marg{Smith}
\gls{DTLnewdbentry}\marg{mydata}\marg{Email}\marg{j.smith@example.com}
\gls{DTLnewrow}\marg{mydata}
\gls{DTLnewdbentry}\marg{mydata}\marg{Forename}\marg{Jane}
\gls{DTLnewdbentry}\marg{mydata}\marg{Surname}\marg{Doe}
\gls{DTLnewdbentry}\marg{mydata}\marg{Email}\marg{j.doe@example.com}
\end{codeenv}
This can be converted using, for example (assuming the above is in
the file \filefmt{mydata.tex}):
\begin{codeenv}
datatool2bib \datatoolbiblongarg{auto-label} mydata.tex mydata.bib
\end{codeenv}
This will create the file \filefmt{mydata.bib}, which will contain:
\begin{codeenv}
\atentry{entry}\marg{mydata1,
  \fieldfmt{forename} = \marg{John},
  \fieldfmt{surname} = \marg{Smith},
  \fieldfmt{email} = \marg{j.smith@example.com}
}
\strut
\atentry{entry}\marg{mydata2,
  \fieldfmt{forename} = \marg{Jane},
  \fieldfmt{surname} = \marg{Doe},
  \fieldfmt{email} = \marg{j.doe@example.com}
}
\end{codeenv}
(Use \converterlongarg{index-conversion} to use \atentry{index} instead
of \atentry{entry} where there's no \field{description}.)
The \csopt{assign-fields} resource option can be used to set
the \field{name} field to a concatenation of the forename and
surname fields (which would need to be aliased or defined if
required).

\cssection{DTLaction}
This general purpose action command can be used to create and append
to databases. The applicable actions that are recognised by the
\texparserlib\ (and therefore by \idx{datatool2bib}) are:
\optfmt{new}, \optfmt{new row}, \optfmt{new entry}, and
\optfmt{add column}.

For example:
\begin{codeenv}
\gls+{DTLsetup}\marg{default-name=mydata}
\gls+{DTLaction}\marg{new}
\gls{DTLaction}\oarg{
  assign =
   \marg{
      Name = \marg{José Arara},
      Score = 68,
      Award = \marg{\cs{cs.dollar}2,453.99}
   }
}\marg{new row}
\gls{DTLaction}\oarg{
  assign =
   \marg{
      Name = \marg{Zoë Zebra},
      Score = 73,
      Award = \marg{\cs{cs.dollar}2,542}
   }
}\marg{new row}
\end{codeenv}
This can be converted using, for example (assuming the above is in
the file \filefmt{mydata.tex}):
\begin{codeenv}
datatool2bib \datatoolbiblongarg{label} Name mydata.tex mydata.bib
\end{codeenv}
This will create the file \filefmt{mydata.bib}, which will contain:
\begin{codeenv}
\atentry{entry}\marg{JoséArara,
  \field{name} = \marg{José Arara},
  \fieldfmt{score} = \marg{68},
  \fieldfmt{award} = \marg{\cs{cs.dollar}2,453.99}
}
\strut
\atentry{entry}{ZoëZebra,
  \field{name} = \marg{Zoë Zebra},
  \fieldfmt{score} = \marg{73},
  \fieldfmt{award} = \marg{\cs{cs.dollar}2,542}
}
\end{codeenv}
Again, use \converterlongarg{index-conversion} to use
\atentry{index} instead of \atentry{entry} if the
\field{description} field is missing.

The \iext{dtltex} and \ext{dbtex} file formats are also recognised
by the \texparserlib. The commands used in those files are not
intended for document use, but the files may be passed to
\idx{datatool2bib} as the input file. See the \sty{datatool} v3.0+
user manual for further details of those formats.

\prefixedcssection{datagidx}{newgidx}
Defines a new \sty{datagidx} database, which may be referenced in
\gls{datagidx.newterm} and \gls{datagidx.newacro} by the
\optfmt{database} key in their optional argument (which be parsed
and then dropped from the field list).

\prefixedcssection{datagidx}{newterm}
This is similar to the \gls{newterm} command provided by
\sty{glossaries} (via the \styopt{index} package option), but has
slightly different behaviour. With \sty{datagidx}, the mandatory
argument is the name. The label may be supplied in the optional
argument. If omitted, the label is obtained from the name. The
\idx{datatool2bib} application will try to mimic the process used by
\sty{datagidx} to create the label from the name, but may not
produce an exact match. It's therefore important to double-check the
resulting \ext{bib} file afterwards.

Each \gls{datagidx.newterm} will be converted to \atentry{entry}
unless it has both the \field{long} and \field{short} fields set, in
which case it will be converted to \atentry{abbreviation}. If there
is no \field{description} and \converterlongarg{index-conversion} is
used, then \atentry{index} will be used instead of \atentry{entry}
(but not if the term has been identified as an abbreviation).

\prefixedcssection{datagidx}{newacro}
This is simply a wrapper command that internally uses
\gls{datagidx.newterm}. However, \idx{datatool2bib} has a slightly
different behaviour that doesn't add the formatting or case-changing
commands (since these can be dealt with by the abbreviation style or
\bibgls\ resource options).

\chapter{Examples}
\label{sec:examples}
\setsecdepth*{0}
\renewcommand{\cssectioncmd}{\subsection}

The example files described here can be found in the
\filefmt{examples} sub-directory. The \ext{bib} files are listed
first and then sample files that use the \ext{bib} data.
Make sure you have the latest versions of \styfmt{glossaries},
\styfmt{mfirstuc}, \styfmt{glossaries-extra} and \bibgls\ if you
want to try these out.  (The \exfile{sample-media.tex} file requires
at least \styfmt{datatool} v2.28.) If you get any undefined control
sequence or undefined style errors then you need to update your
\TeX\ distribution.  Use the \longarg{group} switch when invoking
\bibgls\ for all these examples if you want the glossaries divided
into groups.  The set of system calls for the document build in the
examples below may require an extra \LaTeX\ run to ensure the PDF
bookmarks are up-to-date when \sty{hyperref} is used.

These files are just examples of how to use \bibgls. There are other
ways of defining similar entries and sometimes alternatives are
suggested. Use the code here as a starting point if you need data
like this and adapt it to a format appropriate for your
requirements. There are also some example documents in the 
\href{https://www.dickimaw-books.com/gallery}{Dickimaw Books gallery}.

\filesection{no-interpret-preamble.bib}

The \exfile{no-interpret-preamble.bib} file contains command
definitions used in some of the \field{name} fields. Although these
commands aren't used explicitly in the document, they need to be
defined when the names are displayed in the document (typically in
the glossary). These commands are much like the \gls{sortop} command 
described on \glsxtrpageref{sortop} and need to be hidden 
from \bibgls's interpreter. This file doesn't contain any entry
definitions and must be loaded first with
\csopt[false]{interpret-preamble}. The
\exfile{interpret-preamble.bib} or \exfile{interpret-preamble2.bib} 
file can then be loaded to provide alternative definitions for
\bibgls's interpreter.

The first command is:
\nosecdef{sortname}
This is used in the \field{name} fields for entries containing
information about a person. The aim here is for \bibgls\ to sort
according to \meta{surname}, \meta{first name(s)} but for the
glossary to display \meta{first name(s)} \meta{surname}. For names
with a \qt{von} part, there's another command:
\nosecdef{sortvonname}
which has a similar purpose.
The third command is:
\nosecdef{sortart}
This is the same as \gls{sortname} but is designed for titles,
phrases or sentences that start with an article (such as \qt{a} or
\qt{the}). Although it has the same definition as \gls{sortname}
in this file, in the interpreted files the article part is omitted
to completely ignore them in the sorting.
The fourth command is:
\nosecdef{sortmediacreator}
which again is functionally the same as \gls{sortname}.

The names could be specified using \BibTeX's
syntax instead with \csopt{bibtex-contributor-fields} to convert it,
but the aim here is to show a variety of ways to use \bibgls. For an
example of \csopt{bibtex-contributor-fields}, see the way the 
\fieldfmt{cast} field in \exfile{films.bib} is dealt with.

Although the file only contains ASCII characters, it starts with 
an \gls{encoding} line to prevent \bibgls\ from searching the entire file
for it. (That's not so much of an issue with a short file, but may
cause an unnecessary delay for much longer files.)

The contents of \filefmt{no-interpret-preamble.bib} are as follows:
\lstinputlisting[firstline=5]{../examples/no-interpret-preamble.bib}

\filesection{interpret-preamble.bib}

This provides definitions of \gls{sortname}, \gls{sortvonname},
\gls{sortart} and \gls{sortmediacreator} in \atentry{preamble} 
that can be picked up by the interpreter and used during sorting.
Note that in this case \gls{sortart} is defined to ignore the
article to completely ignore it from sorting. If you happen to have 
\qt{a \meta{something}} and \qt{the \meta{something}} where the 
\meta{something}s are identical, you may want to append the article
to disambiguate them.

The contents of \filefmt{interpret-preamble.bib} are as follows:
\lstinputlisting[firstline=5]{../examples/interpret-preamble.bib}

\filesection{interpret-preamble2.bib}

An alternative to \exfile{interpret-preamble.bib} with a different
definition of \gls{sortmediacreator}. This uses \cs{renewcommand}
instead of \cs{providecommand} so \csopt[false]{write-preamble}
is required to prevent \LaTeX\ from picking up the definitions.

The contents of \filefmt{interpret-preamble2.bib} are as follows:
\lstinputlisting[firstline=5]{../examples/interpret-preamble2.bib}

\filesection{constants.bib}

The \exfile{constants.bib} file contains mathematical constants.
These all use a custom \gls{entrytype} \atentryfmt{constant}, which must
be aliased otherwise the entries will all be ignored. The entries
all have custom fields, which also need to be aliased.
For example:
\begin{codeenv}
\csopt[constant=entry]{entry-type-aliases},
\csopt[
  \fieldfmt{constantname}=\field{name},
  \fieldfmt{constantsymbol}=\field{symbol},
  \fieldfmt{definition}=\field{description},
  \fieldfmt{identifier}=\field{category},
  \fieldfmt{value}=\field{user1}
]{field-aliases}
\end{codeenv}
This setting means that, for example,
\begin{codeenv}
\atentryfmt{constant}\marg{root2,
  \fieldfmt{constantname}=\marg{Pythagoras' constant},
  \fieldfmt{constantsymbol}=\marg{\cs{ensuremath}\marg{\ics{surd}2}},
  \fieldfmt{definition}=\marg{the square root of 2},
  \fieldfmt{value}=\marg{1.41421},
  \fieldfmt{identifier}=\marg{constant}
}
\end{codeenv}
is treated as though it was defined as:
\begin{codeenv}
\atentry{entry}\marg{root2,
  \field{name}=\marg{Pythagoras' constant},
  \field{symbol}=\marg{\cs{ensuremath}\marg{\cs{surd}2}},
  \field{description}=\marg{the square root of 2},
  \field{user1}=\marg{1.41421},
  \field{category}=\marg{constant}
}
\end{codeenv}
This use of custom fields and \glspl{entrytype} allows more flexibility.
For example, I may have another document that uses the same
\ext{bib} file but requires a different definition:
\begin{codeenv}
\atentry{number}\marg{root2,
  \field{description}=\marg{Pythagoras' constant},
  \field{name}=\marg{\cs{ensuremath}\marg{\cs{surd}2}}
}
\end{codeenv}
which can be obtained with:
\begin{codeenv}
\csopt[constant=number]{entry-type-aliases},
\csopt[
  \fieldfmt{constantname}=\field{description},
  \fieldfmt{constantsymbol}=\field{name}
]{field-aliases}
\end{codeenv}
Since the other custom fields haven't be aliased, they're ignored.

The custom fields are: \fieldfmt{identifier} (set to \code{constant}
for all the entries), \fieldfmt{constantname} (the constant's name),
\fieldfmt{definition} (a definition of the constant),
\fieldfmt{value} (the approximate numeric value of the constant),
\fieldfmt{constantsymbol} (the symbolic representation of the
constant) and \fieldfmt{alternative} (alternative symbol). 
There are three entries that don't have the custom
\fieldfmt{value} field: \code{zero} and \code{one} (the exact value
is in the \fieldfmt{constantsymbol} field in both cases) and \code{imaginary}
(where there's no real number value).

I've provided some commands in the \atentry{preamble} for
constants that are represented by Latin and Greek letters.
These can be defined in the document before the \igls{resourceset}
if different notation is required. The upright Greek commands require
the \isty{upgreek} package.

If it's likely that there may be a need to sort according to
\fieldfmt{definition}, then it would be better to use \gls!{sortart}
describe above:
\begin{codeenv}
\atentryfmt{constant}\marg{root2,
  \fieldfmt{constantname}=\marg{Pythagoras' constant},
  \fieldfmt{constantsymbol}=\marg{\cs{ensuremath}\marg{\cs{surd}2}},
  \fieldfmt{definition}=\marg{\gls{sortart}\marg{the}\marg{square root of 2}},
  \fieldfmt{value}=\marg{1.41421},
  \fieldfmt{identifier}=\marg{constant}
}
\end{codeenv}
Remember that this would need \exfile{no-interpret-preamble.bib} to
ensure the command is recognised in the document.

The contents of \filefmt{constants.bib} are as follows:
\lstinputlisting[firstline=5]{../examples/constants.bib}

\filesection{chemicalformula.bib}

The \exfile{chemicalformula.bib} file contains chemical formulae.
Each entry has a field that uses \ics{ce} provided by \sty{mhchem}
so the document will need to load that package. Since all resource
files must be loaded in the preamble, it's possible to ensure that
the package is loaded using:
\begin{codeenv}
\atentry{preamble}\marg{\qtdelim{\cs{usepackage}\marg{mhchem}}}
\end{codeenv}
However, it's best just to load it in the document otherwise it
won't be available before the \ext{glstex} file has been loaded.
Also, \sty{glossaries} (and therefore \sty{glossaries-extra}) must be
loaded after \sty{hyperref}, which usually needs to be loaded last
so most packages should be loaded before \sty{glossaries-extra}.
Instead, I've just put a comment in the \ext{bib} file as a
reminder.

All entries are defined using a custom \gls{entrytype}
\atentryfmt{chemical}. This must be aliased using
\csopt{entry-type-aliases} or the entries will be ignored. For
example, to make \atentryfmt{chemical} behave like \atentry{symbol}:
\begin{codeenv}
\csopt[\fieldfmt{chemical}=\field{symbol}]{entry-type-aliases}
\end{codeenv}
Remember that with the \atentry{symbol} type, if the \field{sort}
field is omitted \bibgls\ will fallback on the label by default. It
can be changed to fallback on the \field{name} field instead using
\csopt[name]{symbol-sort-fallback}.  This will require the use of
the interpreter if the name contains a command but \bibgls\
recognises the \sty{mhchem} package and has a limited ability to
interpret \cs{ce}.  If \atentryfmt{chemical} is changed to
\atentry{entry} instead then the fallback for the \field{sort} will be the
entry's \field{name}.

All entries only contain custom fields, which will all be ignored by
\bibgls\ unless defined or aliased: \fieldfmt{identifier}, which is set to
\code{chemical} for all entries, \fieldfmt{formula}, which is set to the
chemical formula, and \fieldfmt{chemicalname}, which is set to the chemical
name. This allows the flexibility of determining whether the
\field{name} or \field{symbol} field should contain the chemical
formula on a per-resource basis. For example:
\begin{codeenv}
\csopt[\fieldfmt{formula}=\field{name},\fieldfmt{chemicalname}=\field{description}]{field-aliases}
\end{codeenv}
or
\begin{codeenv}
\csopt[\fieldfmt{chemicalname}=\field{name},\fieldfmt{formula}=\field{symbol}]{field-aliases}
\end{codeenv}

The contents of \filefmt{chemicalformula.bib} are as follows:
\lstinputlisting[firstline=5]{../examples/chemicalformula.bib}

\filesection{bacteria.bib}

The \exfile{bacteria.bib} file contains bacteria abbreviations.
These all use the \atentry{abbreviation} \gls{entrytype} with a
\field{short} and \field{long} field.

The entries all have a custom field \fieldfmt{identifier} set to 
\code{bacteria}. This will be ignored by \bibgls\ unless it's
defined using \cs{glsaddkey} or \cs{glsaddstoragekey} or if it's
aliased with \csopt{field-aliases}.

The contents of \filefmt{bacteria.bib} are as follows:
\lstinputlisting[firstline=5]{../examples/bacteria.bib}

\filesection{baseunits.bib}

The \exfile{baseunits.bib} file contains base \idxpl{SIunit}. The entries
are all defined using the custom \atentryfmt{unit} 
\gls{entrytype}. This must be aliased with \csopt{entry-type-aliases} otherwise
\bibgls\ will ignore all the entries. For example:
\begin{codeenv}
\csopt[\fieldfmt{unit}=\field{symbol}]{entry-type-aliases}
\end{codeenv}
will make \bibgls\ treat the entries as though they were defined
using \atentry{symbol}. (Remember that \atentry{symbol} \glspl{entrytype}
use the label as the fallback field for \field{sort}.)

The entries all have custom fields \fieldfmt{unitname},
\fieldfmt{unitsymbol} and \fieldfmt{measurement}, one of which must
be aliased or copied to \field{name} if \atentryfmt{unit} is aliased
to an \gls{entrytype} that requires it. The other custom fields may be aliased or
copied to \field{symbol} and \field{description} as required. The
\fieldfmt{unitsymbol} fields all use \ics{si} provided by the
\isty{siunitx} package, so that package must be loaded in the
document. This is one of the small number of packages recognised by
\bibgls, so it's possible to sort according to the symbol if
required.

The entries also all have a custom field \fieldfmt{identifier} set to 
\code{baseunit}. This will be ignored by \bibgls\ unless it's
defined using \cs{glsaddkey} or \cs{glsaddstoragekey} or if it's
aliased with \csopt{field-aliases}.

The contents of \filefmt{baseunits.bib} are as follows:
\lstinputlisting[firstline=5]{../examples/baseunits.bib}

\filesection{derivedunits.bib}

The \exfile{derivedunits.bib} file is much like \exfile{baseunits.bib}
but contains derived units and in this case the custom \gls{entrytype}
is \atentryfmt{measurement}, which must be aliased
otherwise the entries will all be ignored.
The entries all have a custom field \fieldfmt{identifier} set to 
\code{derivedunit}. This will be ignored by \bibgls\ unless it's
defined using \cs{glsaddkey} or \cs{glsaddstoragekey} or if it's
aliased with \csopt{field-aliases}.

The contents of \filefmt{derivedunits.bib} are as follows:
\lstinputlisting[firstline=5]{../examples/derivedunits.bib}

\filesection{people.bib}

The \exfile{people.bib} file contains details about people. 
The \field{name} fields contain custom commands provided in
\exfile{no-interpret-preamble.bib} and
\exfile{interpret-preamble.bib}.
Remember that if \exfile{no-interpret-preamble.bib} is
loaded first, the definitions provided in that file will be the one
in use in the document. The \exfile{interpret-preamble.bib} file
then needs to be loaded to provide the definitions for \bibgls's
interpreter.

The information for each person is supplied in an \atentry{entry}
type. There are some non-standard fields: \fieldfmt{born},
\fieldfmt{died} and \fieldfmt{othername}. These fields will be ignored
unless keys are provided (using \cs{glsaddkey} or
\cs{glsaddstoragekey}) or the fields are aliased (using
\csopt{field-aliases}). The \fieldfmt{born} and \fieldfmt{died}
fields have dates that are \emph{almost} in the default \code{en-GB}
locale format with the \idx{JRE} \idx{localeprovider}, but they include a
tilde \idx{nbspchar} to prevent awkward line breaks. By default \bibgls's
interpreter converts \idx{nbspchar} to the non-breaking space character
\hex{A0} which isn't recognised by the date format.  This can easily
be fixed with the \longarg{break-space} switch which will interpret
\idx{nbspchar} as a normal breakable space (\hex{20}), so with that switch
\csopt[date]{sort} or \csopt[date-reverse]{sort} can be used on
either of those fields. However, the \idx{CLDR} has a slightly
different default format than the \idx{JRE} for dates with
\code{en-GB}, so it's probably simplest to actually specify the
required format.

An alternative approach would be to provide a command that can be
modified in the document to adjust the date style. For example, the
\fieldfmt{born} field could be specified as:
\begin{codeenv}
\fieldfmt{born}=\marg{\cmd{formatdate}\marg{13}\marg{7}\marg{100}\marg{BC}}
\end{codeenv}
The definition provided for the document could then be, for example:
\begin{codeenv}
\cs{providecommand}\marg{\cmd{formatdate}}[4]\marg{\ics{DTMdisplaydate}\marg{\idx{param}3}\marg{\idx{param}2}\marg{\idx{param}1}\marg{-1} \idx{param}4}
\end{codeenv}
(where \ics{DTMdisplaydate} is provided by the \isty{datetime2}
package) and a definition could be provided for \bibgls's
interpreter, for example:
\begin{codeenv}
\cs{providecommand}\marg{\cmd{formatdate}}[4]\marg{\idx{param}1/\idx{param}2/\idx{param}3 \idx{param}4}
\end{codeenv}
This would need the date format set.  For example,
\csopt[d/M/y G]{date-sort-format}.

Some of the entries, such as \code{caesar}, have a \field{first}
field. In those cases the \field{first} field is slightly different
from the \field{name} field (for example, \qt{Gaius} is omitted in
\code{caesar}'s \field{first} field). The other entries don't have a
\field{first} field. They can simply have the \field{name} copied to
\field{first} with the \csopt{replicate-fields} option (so that the
full name is shown on first use) or the \field{first} field can be
ignored with \csopt{ignore-fields} (so all entries will use the
\field{text} field on first use). The \csopt{replicate-override}
option can be used to force the \field{name} field to be copied to
the \field{first} field, even if the \field{first} field is already
set. Alternatively, with \csopt[true]{replicate-override} and
\csopt[\field{first}=\field{name}]{replicate-fields}, the \field{first} field be
copied to the \field{name} field. For consistency, the \field{first}
fields use the same custom commands as used in the \field{name}
field.

There's one name with a \qt{von} part. In this case the \field{name}
field is set to:
\begin{codeenv}
\gls{sortvonname}\marg{Manfred}\marg{von}\marg{Richthofen}
\end{codeenv}
which will come under the \qt{V} letter group since \gls{sortvonname}
is defined as \meta{von} \meta{surname}, \meta{first name(s)}

If you prefer that this name should come under \qt{R} instead, then
you need to adjust the definition of \gls{sortvonname}:
\begin{codeenv}
\atentry{preamble}\marg{\qtdelim{\cs{providecommand}\marg{\cmd{sortname}}[2]\marg{\idx{param}2, \idx{param}1}
\cs{providecommand}\marg{\cmd{sortvonname}}[3]\marg{\idx{param}3, \idx{param}1 \idx{param}2}}}
\end{codeenv}
An alternative approach would be to format the names using
\BibTeX's contributor syntax and use
\csopt[name]{bibtex-contributor-fields} to convert them.

There are also some synonyms provided with \atentry{index} 
\glspl{entrytype} that have the \field{alias} field to redirect to the 
\gls{mainentry}. These don't include a \field{description} or any of the other
fields as that would be redundant. All the information can be found
in the \gls{mainentry}.

Except for the aliases, the entries have a custom field
\fieldfmt{identifier} set to \code{person}. This will be ignored by
\bibgls\ unless it's defined using \cs{glsaddkey} or
\cs{glsaddstoragekey} or if it's aliased with \csopt{field-aliases}.

The contents of \filefmt{people.bib} are as follows:
\lstinputlisting[firstline=5]{../examples/people.bib}

\filesection{books.bib}

The \exfile{books.bib} file contains details about books.
As above, the entries use custom commands provided in
\exfile{no-interpret-preamble.bib} and
\exfile{interpret-preamble.bib} or
\exfile{interpret-preamble2.bib}.
The entries all have a custom field \fieldfmt{identifier} set to 
\code{book} and other custom fields \fieldfmt{author} and
\fieldfmt{year}.
These will be ignored by \bibgls\ unless they're
defined using \cs{glsaddkey} or \cs{glsaddstoragekey} or if they're
aliased with \csopt{field-aliases}.

There are other ways in which this data could be specified. For
example, the \field{description} field could contain a brief summary
(or \qt{log line}). The \fieldfmt{author} field could use \BibTeX's
syntax instead with \csopt{bibtex-contributor-fields} to convert it.
Alternatively, the entries could be defined using standard \BibTeX\
\glspl{entrytype} that are all aliased to \atentry{bibtexentry}.

The contents of \filefmt{books.bib} are as follows:
\lstinputlisting[firstline=5]{../examples/books.bib}

\filesection{films.bib}

The \exfile{films.bib} file contains details about films.
As above, the entries use custom commands provided in
\exfile{no-interpret-preamble.bib} and
\exfile{interpret-preamble.bib}.
The entries all have a custom field \fieldfmt{identifier} set to 
\code{film} and other custom fields \fieldfmt{cast}, \fieldfmt{director} and
\fieldfmt{year}. These will be ignored by \bibgls\ unless they're
defined using \cs{glsaddkey} or \cs{glsaddstoragekey} or if they're
aliased with \csopt{field-aliases}.

This example file references entries defined in \exfile{books.bib}
through the use of the special 
\glsdisp{idx.idprefix.extn}{\idprefixfmt{ext1}} prefix.
To avoid a label conflict \exfile{films.bib} prefixes all labels
with \idprefixfmt{film} rather than relying on \csopt{label-prefix}.
This ensures that both \exfile{books.bib} and \exfile{films.bib}
can be loaded in the same resource set (otherwise they'd have to be
loaded in separate resource sets with different prefixes). Remember
that you can use \gls{glsxtrnewgls}. For example:
\begin{codeenv}
\gls{glsxtrnewgls}\marg{film.}\marg{\cmd{film}}
\end{codeenv}
This means you can do, for example, just \verb|\film{bladerunner}|
if you want to reference a film without worrying about the prefix.

As with all the example files, there are other ways in which
to specify the data, depending on your requirements. For example,
the \fieldfmt{director} field could use \BibTeX's contributor syntax
(as the \fieldfmt{cast} field does). Some of
the films actually had more than one director but only one is listed
per film in this sample file for simplicity. Similarly, the
\fieldfmt{cast} field only contains the principal actors rather than the
complete list. The book on which the film is based
could be contained in a cross-reference field or a custom
\fieldfmt{basedon} field.

The book \qt{Do Androids Dream of Electric Sheep?}\ referenced at the end 
of the \qt{Blade Runner} film's \field{description} ends with 
a question mark. (Similarly for \qt{Why Didn't They Ask Evans?}) 
If the \field{description} field is simply set as:
\begin{codeenv}
\field{description}=\marg{a film starring Harrison Ford, Rutger Hauer
 and Sean Young loosely based on the novel
 \cs{gls}\marg{ext1.doandroidsdreamofelectricsheep}},
\end{codeenv}
then the \styopt{postdot} package option will produce an odd
result as the inserted \idx{full-stop} immediately follows the question
mark. This is an awkward situation. One possibility is to explicitly 
put the \idx{full-stop} at the end of the \field{description} field for all the 
other entries and omit it for the problematic entries, but this
interferes with the possibility of a category-dependent
\idx{postdescriptionhook}. 

Another option is to put \ics{nopostdesc} in the 
problematic entries. For example:
\begin{codeenv}
\field{description}=\marg{a film starring Harrison Ford, Rutger Hauer
 and Sean Young loosely based on the novel
 \cs{gls}\marg{ext1.doandroidsdreamofelectricsheep}\cs{nopostdesc}},
\end{codeenv}
Be careful with this as it will completely suppress the
\idx{postdescriptionhook}.
A third possibility is to use \ics{glsxtrnopostpunc} instead:
\begin{codeenv}
\field{description}=\marg{a film starring Harrison Ford, Rutger Hauer
 and Sean Young loosely based on the novel
 \cs{gls}\marg{ext1.doandroidsdreamofelectricsheep}\cs{glsxtrnopostpunc}},
\end{codeenv}
This doesn't interfere with the \idx{postdescriptionhook} but if a hook
is provided the post-punctuation may then be required. In both of
the above two cases, \csopt{strip-trailing-nopost} could be used
to remove the suppression commands from the \field{description}
fields if a hook is defined. However this doesn't deal with hooks
that only conditionally append text.

The best solution is with \styfmt{glossaries-extra} v1.23+ which
provides \ics{glsxtrrestorepostpunc} for use in the category
\idxpl{postdescriptionhook} that counteracts \cs{glsxtrnopostpunc}.
This can be placed inside a \idx{conditional}, as used in
\exfile{sample-media.tex}, and does nothing if \cs{glsxtrnopostpunc}
doesn't occur in the \field{description} field. (Note that 
\cs{glsxtrrestorepostpunc} can't be used to counteract
\cs{nopostdesc}, since that completely suppresses the hook.)

The contents of \filefmt{films.bib} are as follows:
\lstinputlisting[firstline=5]{../examples/films.bib}

\filesection{citations.bib}

The \exfile{citations.bib} file is actually a \BibTeX\ file, but it
can be parsed by \bibgls\ if the \BibTeX\ \glspl{entrytype} are converted
to \atentry{bibtexentry}, which can easily be done with:
\begin{codeenv}
\csopt[\ics{GlsXtrBibTeXEntryAliases}]{entry-type-aliases}
\end{codeenv}
The field names will also need to be defined or aliased. For
example:
\begin{codeenv}
\csopt[\fieldfmt{title}=\field{name}]{field-aliases}
\end{codeenv}
If \bibgls\ is then run with \longarg{cite-as-record} any
\ics{citation} commands found in the \ext{aux} file will be treated
as ignored records. The \atentry{preamble} provides a formatting
command that's used by both \BibTeX\ and \bibgls, so \cs{providecommand}
is required rather than \cs{newcommand} as it will appear in both
the \iext{bbl} and the \iext{glstex} files. (In general it's best to
use \cs{providecommand} rather than \cs{newcommand} in the
\atentry{preamble} but in this case it's essential.)
The contents of \filefmt{citations.bib} are as follows:
\lstinputlisting[firstline=5]{../examples/citations.bib}

\filesection{mathgreek.bib}

The \exfile{mathgreek.bib} file contains Greek letters for use in
maths mode. These are all defined with \atentry{symbol}, which means
that by default the \field{sort} field will be obtained from the
label not from the \field{name} field. However, if you want to sort
by the \field{name} field (for example, with \csopt[name]{sort-field})
the \texparserlib\ recognises all the mathematical Greek letter commands 
provided in the \LaTeX\ kernel. Additionally it recognises
\ics{omicron} which isn't provided by \LaTeX\ (the symbol can be
reproduced with a \idx!{lowercase} Latin \qt{o}). Note that
\isty{glossaries-extra-bib2gls} (\sty{glossaries-extra} v1.27+)
provides all the missing Greek letters (such as \cs{omicron}).

The \ext{bib} file could just use \code{o}:
\begin{codeenv}
\atentry{symbol}\marg{omicron,
  \field{name}=\marg{\cs{ensuremath}\marg{o}},
  \field{description}=\marg{omicron},
  \fieldfmt{identifier}=\marg{mathgreek}
}
\end{codeenv}
but this means that if \bibgls\ sorts according to the \field{name}
field using a letter sort, this entry will come before all the other
Greek letters since the character \qt{o} has Unicode value \hex{6F}
whereas, for example, mathematical italic small alpha ($\alpha$)
has Unicode value \hex{1D6FC}. This means that for sorting purposes
it's better to use \ics{omicron}:
\begin{codeenv}
\atentry{symbol}\marg{omicron,
  \field{name}=\marg{\cs{ensuremath}\marg{\cs{omicron}}},
  \field{description}=\marg{omicron},
  \fieldfmt{identifier}=\marg{mathgreek}
}
\end{codeenv}
but \LaTeX\ needs a definition for this, so it's provided in the
\atentry{preamble}:
\begin{codeenv}
\atentry{preamble}\marg{\qtdelim{\cs{providecommand}\marg{\cs{omicron}}\marg{o}}}
\end{codeenv}
(With \sty{glossaries-extra} v1.27+, this is no longer needed.)
The \texparserlib\ and \sty{glossaries-extra-bib2gls} 
similarly provide the missing \idx!{uppercase}
Greek letters, and these can be dealt with in the same way.

The contents of \filefmt{mathgreek.bib} are as follows:
\lstinputlisting[firstline=5]{../examples/mathgreek.bib}

\filesection{bigmathsymbols.bib}

The \exfile{bigmathsymbols.bib} file contains mathematical symbols
that have a large version in display mode. As with
\exfile{mathgreek.bib} the entries are defined using
\atentry{symbol}. This example file requires the \isty{stix} package
as not all of the commands are provided by the \LaTeX\ kernel. 
This file also has a preamble:
\begin{codeenv}
\atentry{preamble}\marg{\qtdelim{\cmd{providecommand}\marg{\cmd{bigoperatornamefmt}}[1]\marg{\comment{}
 \idx{mshiftchar}\ics{displaystyle}\idx{param}1\ics{textstyle}\idx{param}1\idx{mshiftchar}}
\cmd{providecommand}\marg{\cmd{nary}}[1]\marg{\idx{mshiftchar}\idx{param}1\idx{mshiftchar}-ary}}}
\end{codeenv}
The first command \inlinedef{bigoperatornamefmt}
is used in the \field{name} field to display both the in-line and display
versions of the symbol. The \texparserlib\ only has a limited ability to
interpret this as not all the symbols have Unicode in-line and large versions.
In some cases, such as the integral symbol $\int$, there is only a small 
version. (A large version would require construction from \hex{2320}, 
\hex{23AE} and \hex{2321}, which is too complicated in this context.) 
However, the interpreter works well enough to guess at the widest
name if \csopt{set-widest} is used.  There's no advantage in sorting
according to the \field{name} field here, unless a custom rule is
provided, as the Unicode symbols are scattered about different
blocks. Better approaches are to sort according to document use
(\csopt[use]{sort}) or to sort according to the \field{description}
field.

The other custom command is \inlinedef{nary} to provide semantic markup for
\qt{$n$-ary}. This could be defined without an argument:
\begin{codeenv}
\cmd{providecommand}\marg{\cmd{nary}}\marg{\idx{mshiftchar}n\idx{mshiftchar}-ary}
\end{codeenv}
but providing an argument will allow \code{\csfmt{nary}\marg{n}} to work
with first letter upper-casing in the event that the
\field{description} field has a case-change applied (otherwise
it would end up as \qt{$N$-ARY}). Of course, it may be that no case-change 
should be applied, but this example is just for illustrative purposes.

As with the other sample \ext{bib} files, each entry is given a
custom \fieldfmt{identifier} field, which by default will be
ignored. In this case, \fieldfmt{identifier} is either set to 
\code{naryoperator} (for $n$-ary operators) or \code{integral}
for integrals.

The contents of \filefmt{bigmathsymbols.bib} are as follows:
\lstinputlisting[firstline=5]{../examples/bigmathsymbols.bib}

\filesection{mathsrelations.bib}

The \exfile{mathsrelations.bib} file contains mathematical
relational symbols. These use the maths shift character
\idx{mshiftchar} in the \field{name} field and just the symbol in
the \field{text} field. This just illustrates an alternative way of
defining symbols.  Since \ics{ensuremath} isn't used, commands like
\cs{gls} must be explicitly placed in maths mode. For example,
\code{\idx{mshiftchar}\cs{gls}\marg{leq}\idx{mshiftchar}} rather than simply 
\code{\cs{gls}\marg{leq}}. The custom \fieldfmt{identifier} field is
set to \code{relation}.

The contents of \filefmt{mathsrelations.bib} are as follows:
\lstinputlisting[firstline=5]{../examples/mathsrelations.bib}

\filesection{binaryoperators.bib}

The \exfile{binaryoperators.bib} file contains mathematical binary
operators. The format is much like the above
\exfile{mathsrelations.bib} file. The custom \fieldfmt{identifier}
field is set to \code{binaryoperator}.

The contents of \filefmt{binaryoperators.bib} are as follows:
\lstinputlisting[firstline=5]{../examples/binaryoperators.bib}

\filesection{unaryoperators.bib}

The \exfile{unaryoperators.bib} file contains mathematical unary
operators. As above, this again uses \atentry{symbol} to define the symbols,
but in this case \ics{ensuremath} is used in the \field{name}
field and there's no \field{text} field. I've also used \ics{mathord}
to ensure the symbol is treated as a unary (rather than binary)
operator, except for the \ics{forall} entry which is already defined as an
ordinary maths symbol.

The contents of \filefmt{unaryoperators.bib} are as follows:
\lstinputlisting[firstline=5]{../examples/unaryoperators.bib}

\filesection{mathsobjects.bib}

The \exfile{mathsobjects.bib} file contains entries related to 
mathematical objects (sets, spaces, vectors and matrices). This
provides some custom formatting commands in the preamble:
\nosecdef{setfmt}
which is used to format \meta{symbol} as a set,
\nosecdef{setcontentsfmt}
which is used to format the set contents,
\nosecdef{setmembershipfmt}
which is used to format the set membership criteria,
\nosecdef{setcardfmt}
which is used to format the cardinality of a set,
(Note this uses \ics{vert} not \textbar\ as in some of the earlier
examples.)
\nosecdef{numspacefmt}
which is used to format \meta{symbol} as a number space,
\nosecdef{transposefmt}
which is used to format matrix and vector transposes,
\nosecdef{invfmt}
which is used to format inverses,
\nosecdef{vecfmt}
which is used to format \meta{symbol} as a vector, and
\nosecdef{mtxfmt}
which is used to format \meta{symbol} as a matrix.
These commands are intended for use with \gls{glsxtrfmt},
but \gls!{setmembershipfmt} causes a problem as it has two
arguments and \gls{glsxtrfmt} requires the control sequence to 
have exactly one argument. This means employing a little trick.
A command with just one argument is provided:
\nosecdef{setmembershiponeargfmt}
that requires the actual two arguments to be supplied inside
\code{\idx!{param}1}. The outer grouping is removed and the two-argument
\gls!{setmembershipfmt} command is applied:
\begin{codeenv}
\cs{providecommand}\marg{\gls{setmembershiponeargfmt}}[1]\marg{\gls{setmembershipfmt}\idx{param}1}
\end{codeenv}
This means that the entry needs to be referenced in the document
using:
\begin{codeenv}
\gls{glsxtrfmt}\marg{setmembership}\marg{\margm{variable(s)}\margm{condition}}
\end{codeenv}
The simplest thing to do here is to provide a wrapper command in the
document, for example:
\begin{codeenv}
\cs{newcommand}*\marg{\cmd{setmembership}}[2]\marg{\gls{glsxtrfmt}\marg{setmembership}\marg{\marg{\idx{param}1}\marg{\idx{param}2}}}
\end{codeenv}
Now this can be used as:
\begin{codeenv}
\cmd{setmembership}\margm{variable(s)}\margm{condition}
\end{codeenv}

There are essentially two types of entry defined in this file: 
entries that demonstrate the formatting for the objects
and entries that represent specific objects. In the first case
there's a custom \fieldfmt{format} field
that's set to the control sequence name of the relevant semantic
command. If this field is defined or aliased then it can be used
with \gls!{glsxtrfmt} (as in the example above).

In both cases there's a custom \fieldfmt{identifier} field that
reflects the type of object: \code{set} for sets, 
\code{numberspace} for number spaces,
\code{matrix} for matrices or vectors.

Be careful with the set cardinality example. Remember that nested
links cause problems and the \sty{glossaries-extra} manual advises
against using commands like \cs{gls} or \gls!{glsxtrfmt} within
link text and that includes within the \meta{text} argument of 
\gls!{glsxtrfmt}. See \exfile{sample-maths.tex} for suggested
usage.

Some of the \field{description} fields use \gls!{sortart}, so
\exfile{no-interpret-preamble.bib} and
\exfile{interpret-preamble.bib} are also needed.

The contents of \filefmt{mathsobjects.bib} are as follows:
\lstinputlisting[firstline=5]{../examples/mathsobjects.bib}

\filesection{miscsymbols.bib}

The \exfile{miscsymbols.bib} file contains text symbols provided
by the \isty{marvosym} and \isty{ifsym} packages. The \sty{ifsym}
package needs to be loaded with the \styoptfmt{weather} option
to provide the weather commands. Unfortunately both packages define
\csfmt{Sun} and \csfmt{Lightning}, which causes a conflict. See
\exfile{sample-textsymbols.tex} for a workaround. Alternatively,
you can load \sty{ifsym} without the \styoptfmt{weather} option
and use the internal definition of \sty{ifsym}'s \csfmt{Sun}
and \csfmt{Lightning} commands:
\begin{codeenv}
\atentryfmt{icon}\marg{sun,
  \fieldfmt{icon}=\marg{\ics{textweathersymbol}\marg{16}},
  \fieldfmt{description}=\marg{sunny},
  \fieldfmt{identifier}=\marg{weather}
}
\strut
\atentryfmt{icon}\marg{lightning,
  \fieldfmt{icon}=\marg{\cs{textweathersymbol}\marg{26}},
  \field{description}=\marg{thunderstorm},
  \fieldfmt{identifier}=\marg{weather}
}
\end{codeenv}
This removes the conflict, and \csfmt{Sun} and \csfmt{Lightning}
are as defined by \sty{marvosym}.

This file uses a custom \gls{entrytype} \atentryfmt{icon}, which must be
aliased to a recognised entry identifier otherwise the entries will
all be ignored. For example:
\begin{codeenv}
\csopt[\fieldfmt{icon}=\field{symbol}]{entry-type-aliases}
\end{codeenv}

There are three types of symbols defined: media controls, information
and weather. They have the custom \fieldfmt{identifier}
field set to \code{mediacontrol}, \code{information} and
\code{weather}, respectively. There are two other custom fields: 
\fieldfmt{icon} and \fieldfmt{icondescription}. These will need to
be aliased to \field{name} and \field{description}.

Neither of these packages are recognised by \bibgls, which means
that \csopt{set-widest} won't be able to determine the widest name
nor is this data suitable for sorting according to the
\fieldfmt{icon} field (or its alias). Instead, either sort by label
(which is the default for \atentry{symbol}) or by the
\field{description}. If you want to use one of the
\glostyle{alttree} styles you can still use \csopt{set-widest}, but
it will have to use the fallback command.  Alternatively, you can
omit \csopt{set-widest} and explicitly use
\cs{glsFindWidestTopLevelName}.

The contents of \filefmt{miscsymbols.bib} are as follows:
\lstinputlisting[firstline=5]{../examples/miscsymbols.bib}

\filesection{markuplanguages.bib}

The \exfile{markuplanguages.bib} file includes a mixture of 
\atentry{entry} and \atentry{abbreviation} definitions. A~custom
command is provided in \atentry{preamble} to tag the letters
in the \field{long} field that are used to form the abbreviation.
This simply does its argument and is provided in case it's not set
up in the document. If you do want to enable tagging using
\ics{GlsXtrEnableInitialTagging}, remember that this command must be
used before the abbreviations are defined, which means before
the resource file is input with \gls{GlsXtrLoadResources}.
Similarly, the abbreviation style must be set before the
abbreviations are defined.

For convenience \atentry{string} is also used to define a \ext{bib}
variable, which may be appended to fields using the \ext{bib}
concatenation character \idx{stringconcat}. As with the other sample
\ext{bib} files, there's a custom field \fieldfmt{identifier}
which will be ignored unless defined or aliased.

The empty braces at the start some of the fields are there to
protect against first letter uppercasing within \TeX, where it might cause a
problem. (For example, with the \catattr{glossname} attribute.)

The contents of \filefmt{markuplanguages.bib} are as follows:
\lstinputlisting[firstline=5]{../examples/markuplanguages.bib}

\filesection{usergroups.bib}

The \exfile{usergroups.bib} file requires either \XeLaTeX\ or
\LuaLaTeX\ as some of the entry labels use \idx{non-ASCII} characters.
This file has a mixture of \atentry{abbreviation} and
\atentry{index} entries. It also uses \atentry{string} for
convenience and provides a custom command \csfmt{dash} in 
\atentry{preamble}. Each entry is the name of a \TeX\ user group:
the international \idx{TUG} and all the local groups.
Most of them have an abbreviated name, so they're defined with
\atentry{abbreviation}. There are a few without an abbreviation, so
they're defined with \atentry{index} instead. There's one alias.
(The information was obtained from \idx{TUG}['s] 
\href{http://tug.org/usergroups.html}{user groups
page}~\cite{tugusergroups}.)

As with the other examples, there are some custom fields which will
be ignored if they aren't defined or aliased: \fieldfmt{identifier}
(set to \code{texusergroup}), \fieldfmt{language} (a comma-separated
list of language tags) and \fieldfmt{translation} (provides a
translation if the user group name isn't in English). 

Not all entries have a \fieldfmt{translation} field. It it's
omitted, then the user group name is in English, otherwise it's
in the first language listed in the \fieldfmt{language} field.
Most of the language tags are just the ISO 639-1 language code, but
a few of them include the ISO 3166-1 region code as well.

The contents of \filefmt{usergroups.bib} are as follows:
\begin{lstlisting}[escapechar=|]
% Encoding: UTF-8

% Requires XeLaTeX/LuaLaTeX for non-ASCII labels

@string{tug={\TeX\ Users Group}}

@preamble{"\providecommand{\dash}{\,---\,}"}

@abbreviation{TUG,
  short={TUG},
  long=tug,
  language={en},
  identifier={texusergroup}
}

@abbreviation{bgTeX,
  short={bgTeX},
  long={Bulgarian \LaTeX\ Users Group},
  language={bg},
  identifier={texusergroup}
}

@abbreviation{latex-br,
  short={latex-br},
  long={Grupo de Usuários},
  language={pt-BR},
  identifier={texusergroup},
  translation={Brazilian }#tug
}

@abbreviation{CTeX,
  short={CTeX},
  long={Chinese \TeX\ Society},
  identifier={texusergroup},
  language={zh}
}

@abbreviation{CSTUG,
  short={CSTUG},
  long={|Československé sdružení uživatelů TeXu|, z.~s.},
  language={cs},
  identifier={texusergroup},
  translation={Czech Republic }#tug
}

@abbreviation{DANTE,
  short={DANTE e.V.},
  long={Deutschsprachige Anwendervereinigung \TeX\ e.V.},
  language={de},
  identifier={texusergroup},
  translation={German Speaking }#tug
}

@abbreviation{DKTUG,
  short={DK-TUG},
  long={Danish }#tug,
  language={da},
  identifier={texusergroup}
}

@index{EUG,
  name={Estonian User Group},
  language={et},
  identifier={texusergroup}
}

@abbreviation{CervanTeX,
  short={CervanTeX},
  long={Grupo de Usuarios de \TeX\ Hispanohablantes},
  language={es},
  identifier={texusergroup},
  translation={Spanish Speaking }#tug
}

@abbreviation{TirantloTeX,
  short={Tirant lo \TeX},
  long={Catalan }#tug,
  language={ca},
  identifier={texusergroup}
}

@abbreviation{GUTenberg,
  short={GUTenberg},
  long={Groupe francophone des utilisateurs de \TeX},
  language={fr},
  identifier={texusergroup},
  translation={French Speaking }#tug
}

@abbreviation{UKTUG,
  short={UK-TUG},
  long={UK }#tug,
  language={en-GB},
  identifier={texusergroup}
}

@abbreviation{|\greekmono ɛϕτ|,
  short={|\greekmono ɛϕτ|},
  long={|\greekmono Σύλλογος Ελλήνων Φίλων του| \TeX},
  language={el},
  identifier={texusergroup},
  translation={Greek \TeX\ Friends}
}

@abbreviation{MaTeX,
  short={MaTeX},
  long={Magyar \TeX\ Egyesület},
  language={hu},
  identifier={texusergroup},
  translation={Hungarian }#tug
}

@abbreviation{ITALIC,
  short={ITALIC},
  long={Irish \TeX\ and \LaTeX\ In-print Community},
  language={en-IE,en-GB},
  identifier={texusergroup}
}

@abbreviation{ÍsTeX,
  short={ÍsTeX},
  long={Vefur íslenskra \TeX\ notenda},
  language={is},
  identifier={texusergroup},
  translation={Icelandic }#tug
}

@abbreviation{GuIT,
  short={GuIT},
  long={Gruppo Utilizzatori Italiani di \TeX},
  language={it},
  identifier={texusergroup},
  translation={Italian }#tug
}

@abbreviation{KTS,
  short={KTS},
  identifier={texusergroup},
  long={Korean \TeX\ Society},
  language={ko}
}

@index{KTUG,
  alias={KTS},
  identifier={texusergroup}
}

@index{LTVG,
  name={Lietuvos \TeX'o |Vartotojų Grupė|},
  language={lt},
  identifier={texusergroup},
  translation={Lithuanian }#tug
}

@index{mxTeX,
  name={\TeX\ México},
  language={es-MX},
  identifier={texusergroup},
  translation={Mexican }#tug
}

@abbreviation{NTG,
  short={NTG},
  long={Nederlandstalige \TeX\ Gebruikersgroep},
  language={nl},
  identifier={texusergroup},
  translation={Netherlands }#tug
}

@index{NTUG,
  name={Nordic \TeX\ Users Group},
  language={da,et,fi,fo,is,nb,nn,sv},
  identifier={texusergroup}
}

@abbreviation{GUST,
  short={GUST},
  long={Polska Grupa |Użytkowników| Systemu \TeX},
  language={pl},
  identifier={texusergroup},
  translation={Polish }#tug
}

@abbreviation{GUTpt,
  short={GUTpt},
  long={Grupo de Utilizadores de \TeX},
  language={pt},
  identifier={texusergroup},
  translation={Portuguese }#tug
}

@abbreviation{VietTUG,
  short={VietTUG},
  long={Vietnamese }#tug,
  language={vi},
  identifier={texusergroup}
}

@abbreviation{LUGSA,
  short={LUGSA},
  long={\LaTeX\ User Group\dash South Africa},
  language={en-ZA},
  identifier={texusergroup}
}


\end{lstlisting}

\filesection{animals.bib}

The \exfile{animals.bib} file contains entries defined using
\atentry{entry}. As with the above example \ext{bib} files, there's
a custom \fieldfmt{identifier} field that will be ignored unless
defined or aliased.

The contents of \filefmt{animals.bib} are as follows:
\lstinputlisting[firstline=5]{../examples/animals.bib}

\filesection{minerals.bib}

The \exfile{minerals.bib} file contains entries defined using
\atentry{entry}. As with the above example \ext{bib} files, there's
a custom \fieldfmt{identifier} field that will be ignored unless
defined or aliased.

The contents of \filefmt{minerals.bib} are as follows:
\lstinputlisting[firstline=5]{../examples/minerals.bib}

\filesection{vegetables.bib}

The \exfile{vegetables.bib} file contains entries defined using
\atentry{entry} and an entry defined with \atentry{index} with just
the \field{alias} field. As with the above example \ext{bib} files, there's
a custom \fieldfmt{identifier} field that will be ignored unless
defined or aliased.

The contents of \filefmt{vegetables.bib} are as follows:
\lstinputlisting[firstline=5]{../examples/vegetables.bib}

\filesection{terms.bib}

The \exfile{terms.bib} file contains entries defined using
\atentry{index}. Unlike the above sample \ext{bib} files, there
are no custom fields here.

The contents of \filefmt{terms.bib} are as follows:
\lstinputlisting[firstline=5]{../examples/terms.bib}

\filesection{topics.bib}

The \exfile{topics.bib} file contains entries defined using
\atentry{index}. Again there are no custom fields here.

The contents of \filefmt{topics.bib} are as follows:
\lstinputlisting[firstline=5]{../examples/topics.bib}

\filesection{sample-hierarchical.tex}

This example uses the \exfile{terms.bib}, \exfile{animals.bib},
\exfile{minerals.bib} and \exfile{vegetables.bib} files to create a
\gls{hierarchicalglossary}. These are specified with the resource option:
\begin{codeenv}
\csopt[{terms,animals,minerals,vegetables}]{src}
\end{codeenv}
The custom \fieldfmt{identifier} field is
aliased to the \field{parent} field since it conveniently matches
the labels of the animal, mineral and vegetable entries in the
\exfile{terms.bib} file:
\begin{codeenv}
\csopt[{identifier=parent}]{field-aliases}
\end{codeenv}
The default \csopt{selection} setting means that only those terms
referenced in the document and their dependencies are selected. The
referenced entries simply have \qt{1} in the \igls{locationlist} as
it's only a trivial single-paged example.

The dependencies that haven't actually been referenced in the
document don't have a \igls{locationlist}. (The \qt{seal} entry is a
dependency, but it's also been referenced in the document, so it has a
\igls{locationlist}.) The \qt{quartz}, \qt{beryl} and \qt{marrow}
entries are dependencies because they occur in the description of
some of the referenced entries. Normally this would mean that they
have no \igls{locationlist} after the first \LaTeX+\bibgls+\LaTeX\
build but once the glossary has been created the references to those
dependent entries in the descriptions will create records and so on
the next \bibgls+\LaTeX\ they will also have \iglspl{locationlist}.
This would make the complete document build:
\begin{verbatim}
pdflatex sample-hierarchical
bib2gls --group sample-hierarchical
pdflatex sample-hierarchical
bib2gls --group sample-hierarchical
pdflatex sample-hierarchical
\end{verbatim}
However, in this example I've decided to \glsdisp{ignoredrecord}{ignore} any records 
created in the glossary:
\begin{codeenv}
\ics{GlsXtrSetDefaultNumberFormat}\marg{\encap{glsignore}}
\end{codeenv}
This means that the document build is the usual \LaTeX+\bibgls+\LaTeX.

I've used the \glostyle{treegroup} style so I need to invoke
\bibgls\ with the \longarg{group} switch. This creates letter groups
for the \glspl{top-levelentry}. Note that \glspl{sub-entry} never have letter groups.

The complete code is listed below.  The document build is:
\begin{verbatim}
pdflatex sample-hierarchical
bib2gls --group sample-hierarchical
pdflatex sample-hierarchical
\end{verbatim}
The complete document is shown in \figureref{fig:sample-hierarchical.pdf}.

\lstinputlisting[firstline=5]{../examples/sample-hierarchical.tex}

\begin{figure}
\figcontents
{%
 \frame{\includegraphics[height=.9\textheight]{../examples/sample-hierarchical.pdf}}%
}
{\caption{\filefmt{sample-hierarchical.pdf}}}
{fig:sample-hierarchical.pdf}
\end{figure}

\filesection{sample-nested.tex}

As discussed in \sectionref{sec:logicaldivisions} there are three
ways of creating logical divisions when displaying the entries
through the use of the \field{type}, \field{group} and \field{parent} fields.
In general, \glspl{hierarchicalglossary} are created with
the \field{parent} field and an appropriate glossary style (as in
the previous \exfile{sample-hierarchical.tex} example).

This example creates a \hierarchical\ effect but the entries don't
actually have a \hierarchical\ structure as none of them have the
\field{parent} field set. Instead, what were the \glspl{childentry} in
\exfile{sample-hierarchical.tex} now have the \field{type} field
set. The \hierarchical\ effect is achieved with
\ics{printunsrtinnerglossary} (which requires at least
\sty{glossaries-extra} v1.44).

\begin{important}
The \ics{printunsrtinnerglossary} command is unsuitable for use
with tabular-like styles, such as \glostyle{long}, and can be
problematic with \glostyle{list} styles. However, those styles
aren't suitable for \glspl{hierarchicalglossary} anyway.
\end{important}

Normally, \hierarchy\ is achieve through definitions like:
\begin{codeenv}
\atentry{index}\marg{animal}
\atentry{entry}\marg{duck,\field{name}=\marg{duck},
  \field{description}=\marg{a waterbird with webbed feet},
  \field{parent}=\marg{animal}
}
\end{codeenv}
The previous example did this by loading both the \exfile{terms.bib}
and \exfile{animals.bib} files and aliasing the custom
\fieldfmt{identifier} field to \field{parent}. In this example, the
custom field is aliased to \field{type}, which effectively makes the
definitions behave like:
\begin{codeenv}
\atentry{index}\marg{animal}
\atentry{entry}\marg{duck,\field{name}=\marg{duck},
  \field{description}=\marg{a waterbird with webbed feet},
  \field{type}=\marg{animal}
}
\end{codeenv}
The aim here is for the \code{animal} entry to be placed in the 
\gls{mainglossary} so that it's listed with \cs{printunsrtglossary}. The
\code{duck} entry is placed in a glossary that has a label
(\code{animal}) that matches the label of the \qt{parent} entry
(even though it's technically not a \parent). This new glossary
(\code{animal}) can be automatically defined by invoking \bibgls\
with the \longarg{provide-glossaries} switch.

This example document defines a custom handler function that
will do the current entry as normal (with \cs{glsxtrunsrtdo}) but
will then check for the existence of a glossary that has the same
label as the current entry. This requires the starred version of
\ics{ifglossaryexists} to included ignored glossaries in the
existence check. If the glossary exists, it's then displayed using
\ics{printunsrtinnerglossary}:
\begin{codeenv}
\cs{newcommand}\marg{\csfmt{nestedhandler}}[1]\marg{\comment{}
  \cs{glsxtrunsrtdo}\marg{\idx{hashchar}1}\comment{}
  \ics{ifglossaryexists*}\marg{\idx{hashchar}1}\comment{}
  \marg{\comment{}
    \cs{printunsrtinnerglossary}\oarg{\printglossopt[\idx{hashchar}1]{type},\printglossopt[++1]{leveloffset},\printglossopt[false]{groups}}
    \marg{}\marg{}\comment{}
  }\comment{}
  \marg{}\comment{}
}
\end{codeenv}
The \printglossopt{leveloffset} option is required to achieve a
\hierarchical\ effect (provided the glossary style supports it) and
the \printglossopt[false]{groups} option is needed to prevent letter
groups showing for the nested glossary, which would otherwise create
a strange effect. (This example uses the \glostyle{treegroup} style,
which provides a \gls{hierarchicalglossary} with letter groups.)

The \cs{printunsrtglossary} handler macro then needs to be set to
this custom macro when the \gls{mainglossary} is displayed:
\begin{codeenv}
\cs{printunsrtglossary*}\marg{\cs{let}\cs{printunsrtglossaryhandler}\csfmt{nestedhandler}}
\end{codeenv}
The main difficulty comes with ensuring that all the necessary
entries are selected. Now that the custom \fieldfmt{identifier}
field has been aliased to \field{type} rather than \field{parent},
the \code{animal} entry is no longer considered a dependent. The
\code{duck} entry has been referenced in the document with \cs{gls}
but the \code{animal} entry hasn't. The previous example ensured
that the \code{animal} entry was selected because it was a \parent\ of
a selected entry. If the same resource options are used in this
example, the \gls{mainglossary} will be empty, which means that the
nested glossaries won't be displayed either.

One way to ensure that the \code{animal}, \code{mineral} and
\code{vegetable} entries are selected is to identify the \field{type}
field as a dependency field:
\begin{codeenv}
\gls{GlsXtrLoadResources}\oarg{\csopt[{terms,animals,minerals,vegetables}]{src},
  \csopt[{identifier=type}]{field-aliases},
  \csopt[{type}]{dependency-fields}
}
\end{codeenv}
This will achieve the same effect as the
\exfile{sample-hierarchical.tex} document, but it's a far more
convoluted method. The reason this example document is listed here
is to demonstrate a sightly modified \hierarchical\ effect that can't
be achieved through the normal method.

Suppose that, for some strange reason, I want the \qt{animal},
\qt{mineral} and \qt{vegetable} entries to be listed in a different
order (say, reverse alphabetical). The other entries (\qt{duck} etc)
need to be sorted in normal alphabetical order.

The \csopt{sort} option applies the same sort method to all
\hierarchical\ levels. The sort \emph{value} chosen for particular entries
can be altered through the use of fallbacks (such as the
\csopt{entry-sort-fallback} or \csopt{symbol-sort-fallback} options)
and a letter comparator may be used to resolve identical sort
values (\csopt{identical-sort-action}), but the same sort algorithm
is applied to all entries in the same set (primary, secondary or dual
within the same resource set). The only way to apply different sort
methods is to separate the entries into different resource sets (or
use dual or secondary sorting).

This can be achieved by having one resource set for the \glspl{mainentry}
with one sort method and another resource set for all the other
entries with a different sort method:
\begin{codeenv}
\gls{GlsXtrLoadResources}\oarg{\csopt[terms]{src},\csopt[en-reverse]{sort}}

\gls{GlsXtrLoadResources}\oarg{\csopt[animals,minerals,vegetables]{src},\csopt[en]{sort},
  \csopt[{identifier=type}]{field-aliases},\csopt[{type}]{dependency-fields}
}
\end{codeenv}
This works when cross-resource dependencies are permitted (see
\sectionref{sec:resourcesets}). In the event that cross-resource
dependencies aren't permitted, the selection criteria is more
complicated:
\begin{codeenv}
\gls{GlsXtrLoadResources}\oarg{\csopt[terms,animals,minerals,vegetables]{src},
 \csopt[en-reverse]{sort},
 \csopt[{identifier=parent}]{field-aliases},
 \csopt[ancestors but not recorded]{selection}
}

\gls{GlsXtrLoadResources}\oarg{
  \csopt[animals,minerals,vegetables]{src},
  \csopt[{identifier=type}]{field-aliases},
  \csopt[{type}]{dependency-fields}
  \csopt[en]{sort}
}
\end{codeenv}
Fortunately in this example, cross-resource dependencies are
permitted so the simpler alternative works. (If they're not
permitted, the \bibgls\ transcript file will contain
\qt{Cross-resource references can't be supported for resource set
\meta{filename}}.)

The complete code is listed below.  The document build is:
\begin{verbatim}
pdflatex sample-nested
bib2gls --group --provide-glossaries sample-nested
pdflatex sample-nested
\end{verbatim}
The complete document is shown in \figureref{fig:sample-nested.pdf}.

\lstinputlisting[firstline=5]{../examples/sample-nested.tex}

\begin{figure}
\figcontents
{%
 \frame{\includegraphics[height=.9\textheight]{../examples/sample-nested.pdf}}%
}
{\caption{\filefmt{sample-nested.pdf}}}
{fig:sample-nested.pdf}
\end{figure}

\filesection{sample-constants.tex}

This example uses the \exfile{constants.bib} file. The aim here
is to just have a list of all the constants defined in the \ext{bib}
file. (There are no references in the document.) This means I need
to use:
\begin{codeenv}
\csopt[all]{selection}
\end{codeenv}
in order to select all entries. I also need to alias the custom
\atentryfmt{constant} \gls{entrytype} otherwise all the entries will be
ignored. I decided to make \atentryfmt{constant} behave like
\atentry{number} for semantic reasons:
\begin{codeenv}
\csopt[constant=number]{entry-type-aliases}
\end{codeenv}
The custom fields also need aliasing:
\begin{codeenv}
\csopt[
  \fieldfmt{identifier}=\field{category},
  \fieldfmt{constantsymbol}=\field{name},
  \fieldfmt{constantname}=\field{description},
  \fieldfmt{value}=\field{user1},
  \fieldfmt{definition}=\field{user2},
  \fieldfmt{alternative}=\field{user3},
]{field-aliases}
\end{codeenv}
I decided to use the \glostyle{altlist} style, so I've instructed
\bibgls\ to determine the widest name:
\begin{codeenv}
\csopt{set-widest}
\end{codeenv}
It's always a good idea to specify the glossary type when using
\csopt{set-widest}, although in this example there's only one
glossary so it doesn't make much difference.
\begin{codeenv}
\csopt[main]{type}
\end{codeenv}
I decided to order the constants according to their (approximate)
numerical value. I've aliased the custom \fieldfmt{value} field to
\field{user1}, so I can sort by that field using a numerical
comparison:
\begin{codeenv}
\csopt[\field{user1}]{sort-field},
\csopt[double]{sort}
\end{codeenv}
There are three entries without the \fieldfmt{user1} field (as the
custom \fieldfmt{value} field is missing in the \ext{bib} file):
\code{zero}, \code{one} and \code{imaginary}. In the case of
\code{zero} and \code{one} the exact value can be obtained from the
\field{name} field. Since I've change the default \csopt{sort-field}, I
can't use \csopt{symbol-sort-fallback}. Instead I need to use:
\begin{codeenv}
\csopt[\field{name}]{missing-sort-fallback}
\end{codeenv}

What happens with the \code{imaginary} entry? It has no real
representation.  The transcript (\ext{glg}) file shows the message:
\begin{verbatim}
Warning: Can't parse sort value 'i' for: imaginary
\end{verbatim}
With the numerical sort methods, if the field can't be parsed the
value defaults to 0. This means that both \code{zero} and
\code{imaginary} have 0 as the sort value, so the
\csopt{identical-sort-action} is implemented. The default setting
means that \bibgls\ will fallback on comparing the entry labels, so
\code{imaginary} comes before \code{zero}.

Since I'm just using the \glostyle{alttree} style, I only need
\sty{glossary-tree}. I can improve efficiency in the document build
by preventing the other glossary style packages from being loaded
using the \styopt{nostyles} package option. This also prevents
\sty{glossary-tree} from being loaded, but I can both load it and
patch the styles with \sty{glossaries-extra-stylemods} through the
option \styopt[tree]{stylemods}. Since the default \glostyle{list}
style is no longer available, I need to set a new default with
\styopt[alttree]{style}. I also want to automatically insert a
\idx{full-stop} after the description, which can be done with
\styopt{postdot}. Don't forget that the \styopt{record} option is
always needed when using \bibgls. This means that the
\sty{glossaries-extra} package needs to be loaded as follows:
\begin{codeenv}
\cs{usepackage}\oarg{\styopt{record},\styopt{nostyles},\styopt{postdot},\styopt[tree]{stylemods},\styopt[alttree]{style}}
 \marg{glossaries-extra}
\end{codeenv}
I've assigned the custom \fieldfmt{constantname} field to the
\field{description} field and the custom \fieldfmt{constantsymbol} field to the
\field{name} field. This means that by default the glossary list
will just show the symbolic representation and the constant's name.
I'd like to append the value and definition after the description.
With the base \sty{glossaries} package this would require defining a
new glossary style but with \sty{glossaries-extra} it can easily be
achieved through the \idx{postdescriptionhook}.

I've aliased the custom \fieldfmt{identifier} field to
\field{category}, which means that all the entries will have the
\field{category} set to \code{constant}. The
\idx{postdescriptionhook} is obtained from
\idx{glsxtrpostdesccategory}, so I need to define
the command \csfmt{glsxtrpostdescconstant}. A simple definition is:
\begin{codeenv}
\cs{newcommand}\marg{\cmd{glsxtrpostdescconstant}}\marg{\comment{}
 \cs{space} (approximately \cs{glsentryuseri}\marg{\cs{glscurrententrylabel}})\comment{}
 : \cs{glsentryuserii}\marg{\cs{glscurrententrylabel}}\comment{}
}
\end{codeenv}
This is fine if all entries have the \field{user1} and \field{user2}
fields set. A more generic approach tests for the existence of these
fields. This can either be done with \ics{ifglshasfield}:
\begin{codeenv}
\cs{newcommand}\marg{\cmd{glsxtrpostdescconstant}}\marg{\comment{}
  \cs{ifglshasfield}\marg{\field{user1}}\marg{\cs{glscurrententrylabel}}\comment{}
  \marg{ (approximately \cs{glscurrentfieldvalue})}\comment{}
  \marg{}\comment{}
  \cs{ifglshasfield}\marg{\field{user2}}\marg{\cs{glscurrententrylabel}}\comment{}
  \marg{: \cs{glscurrentfieldvalue}}\comment{}
  \marg{}\comment{}
}
\end{codeenv}
or with \ics{glsxtrifhasfield}:
\begin{codeenv}
\cs{newcommand}\marg{\cmd{glsxtrpostdescconstant}}{\comment{}
  \cs{glsxtrifhasfield}\marg{\field{useri}}\marg{\cs{glscurrententrylabel}}\comment{}
  \marg{ (approximately \cs{glscurrentfieldvalue})}\comment{}
  \marg{}\comment{}
  \cs{glsxtrifhasfield}\marg{\field{userii}}\marg{\cs{glscurrententrylabel}}\comment{}
  \marg{: \cs{glscurrentfieldvalue}}\comment{}
  \marg{}\comment{}
}
\end{codeenv}
(Note the need to use the internal field label \field{useri} and
\field{userii} with \ics{glsxtrifhasfield}.)

A modification can be made to also show the alternative
representation (obtained from the custom \fieldfmt{alternative}
field which has been aliased to \field{user3}):
\begin{codeenv}
\cs{newcommand}\marg{\cmd{glsxtrpostdescconstant}}{\comment{}
  \cs{glsxtrifhasfield}\marg{\field{useriii}}\marg{\cs{glscurrententrylabel}}\comment{}
  \marg{ (also denoted \cs{glscurrentfieldvalue}
      \cs{glsxtrifhasfield}\marg{\field{useri}}\marg{\cs{glscurrententrylabel}}\comment{}
      \marg{, approximately \cs{glscurrentfieldvalue}}\comment{}
      \marg{}\comment{}
   )\comment{}
  }\comment{}
  \marg{\comment{}
    \cs{glsxtrifhasfield}\marg{\field{useri}}\marg{\cs{glscurrententrylabel}}\comment{}
    \marg{ (approximately \cs{glscurrentfieldvalue})}\comment{}
    \marg{}\comment{}
  }\comment{}
  \cs{glsxtrifhasfield}\marg{\field{userii}}\marg{\cs{glscurrententrylabel}}\comment{}
  \marg{: \cs{glscurrentfieldvalue}}\comment{}
  \marg{}\comment{}
}
\end{codeenv}
If you have at least \styfmt{glossaries-extra} v1.31, it's better to use:
\begin{codeenv}
\ics{glsdefpostdesc}\marg{constant}
\end{codeenv}
instead of:
\begin{codeenv}
\cs{newcommand}\marg{\cmd{glsxtrpostdescconstant}}
\end{codeenv}
as it can guard against accidental misspelling of the \code{glsxtrpostdesc}
part of the command name.

The complete code is listed below.  The document build is:
\begin{verbatim}
pdflatex sample-constants
bib2gls sample-constants
pdflatex sample-constants
\end{verbatim}
The complete document is shown in \figureref{fig:sample-constants.pdf}.

\lstinputlisting[firstline=5]{../examples/sample-constants.tex}

\begin{figure}
\figcontents
{%
 \frame{\includegraphics[height=.9\textheight]{../examples/sample-constants.pdf}}%
}
{\caption{\filefmt{sample-constants.pdf}}}
{fig:sample-constants.pdf}
\end{figure}

\filesection{sample-chemical.tex}

This example just uses the \exfile{chemicalformula.bib} file. The aim here
is to have a list of chemical formulae referenced in the document
but not have a number list. I could use the \styopt{nonumberlist}
package option to suppress the number list display, but it's more
efficient to instruct \bibgls\ to not save the number list with:
\begin{codeenv}
\csopt[false]{save-locations}
\end{codeenv}
All entries are defined in \exfile{chemicalformula.bib} using
a custom \gls{entrytype} \atentryfmt{chemical} which needs to be aliased
in order for the entries to be recognised:
\begin{codeenv}
\csopt[chemical=symbol]{entry-type-aliases}
\end{codeenv}
Additionally, the entries only have custom fields, so these also
need to be aliased. In this case I want the formula in the \field{name} field
and the chemical name in the \field{description} field:
\begin{codeenv}
\csopt[\fieldfmt{formula}=\field{name},\fieldfmt{chemicalname}=\field{description}]{field-aliases}
\end{codeenv}
The \atentry{symbol} \gls{entrytype} falls back on the label for the
\field{sort} value by default, but I've decided
to fallback on the \field{name} field for sorting:
\begin{codeenv}
\csopt[\field{name}]{symbol-sort-fallback}
\end{codeenv}
An alternative approach would simply be to alias
\atentryfmt{chemical} to \atentry{entry} instead.

Since the \field{name} field contains chemical formulae rather than
words, it makes more sense to use one of the letter sort methods
rather than a locale collator. In this case the names contain
mixtures of letters and numbers, so one of the letter-number sort
methods (listed in \tableref{tab:sortoptionsletternumber}) would be
appropriate.

I want to use the \glostyle{alttreegroup} style (provided by
\sty{glossary-tree}). Since I don't require the other style
packages, I've used \styopt{nostyles} to suppress the automatic
loading and \styopt[tree]{stylemods} to both load
\sty{glossary-tree} and patch it. The \glostyle{alttreegroup} style
needs to know the widest name, so I've use \csopt{set-widest} for
convenience. The default behaviour of the tree styles is to 
format the name in bold. This is done through the command
\ics{glstreenamefmt} which is defined as:
\begin{codeenv}
\cs{newcommand}*\marg{\cs{glstreenamefmt}}[1]\marg{\cs{textbf}\marg{\idx{param}1}}
\end{codeenv}
The group headings use \ics{glstreegroupheaderfmt} which defaults to
\cs{glstreenamefmt}. Since I want to keep bold headings, I need to
redefine this as well:
\begin{codeenv}
\cs{renewcommand}*\marg{\cs{glstreenamefmt}}[1]\marg{\idx{param}1}
\cs{renewcommand}*\marg{\cs{glstreegroupheaderfmt}}[1]\marg{\cs{textbf}\marg{\idx{param}1}}
\end{codeenv}
(For a more compact layout, you could use \glostyle{mcolalttreegroup}
instead.)
I also need the \longarg{group} switch to make the sort method
automatically assign letter groups.

The complete code is listed below.  The document build is:
\begin{verbatim}
pdflatex sample-chemical
bib2gls --group sample-chemical
pdflatex sample-chemical
\end{verbatim}
The complete document is shown in \figureref{fig:sample-chemical.pdf}.

\lstinputlisting[firstline=5]{../examples/sample-chemical.tex}

\begin{figure}
\figcontents
{%
 \frame{\includegraphics[height=.9\textheight]{../examples/sample-chemical.pdf}}%
}
{\caption{\filefmt{sample-chemical.pdf}}}
{fig:sample-chemical.pdf}
\end{figure}

\filesection{sample-bacteria.tex}

This example just uses the \exfile{bacteria.bib} file. The aim here
is to have a simple list of the bacteria referenced in the document.
Bacteria names are often shown in the long form on first use
(without the short form) and then the short form on subsequent use.
This can easily be done with the \abbrstyle{long-only-short-only}
style. Bacteria are usually typeset in italic. It's best to create a
semantic command for this:
\begin{codeenv}
\cs{newcommand}\marg{\cmd{bacteriafont}}[1]\marg{\cs{emph}\marg{\idx{param}1}}
\end{codeenv}
There are two methods to apply this to the bacteria entries. The
first is to redefine the formatting commands used by the
\abbrstyle{long-only-short-only} style:
\begin{codeenv}
\cs{renewcommand}*\marg{\ics{glsabbrvonlyfont}}[1]\marg{\cmd{bacteriafont}\marg{\idx{param}1}}
\cs{renewcommand}*\marg{\ics{glslongonlyfont}}[1]\marg{\cmd{bacteriafont}\marg{\idx{param}1}}
\end{codeenv}
This is fine if I don't intend to use this style for other types of
abbreviations. However, I may decide to extend the document at a
later date to include other abbreviations that need
\abbrstyle{long-only-short-only} but shouldn't be emphasized.
This can be done through the use of \idxpl{categoryattribute}.
The font used for the \field{name} in the glossary is governed by
the \catattr{glossnamefont} attribute, the font used for the
\field{description} in the glossary is governed by the
\catattr{glossdescfont} attribute and the font used by commands like
\cs{gls} in the document is governed by the \catattr{textformat}
attribute (\styfmt{glossaries-extra} v1.21+). So if I set
the \field{category} to \code{bacteria} then I can do:
\begin{codeenv}
\cs{setabbreviationstyle}\oarg{bacteria}\marg{\abbrstyle{long-only-short-only}}
\cs{glssetcategoryattribute}\marg{bacteria}\marg{\catattr{textformat}}\marg{bacteriafont}
\cs{glssetcategoryattribute}\marg{bacteria}\marg{\catattr{glossnamefont}}\marg{bacteriafont}
\end{codeenv}
and (if the \field{description} field is displayed in the glossary):
\begin{codeenv}
\cs{glssetcategoryattribute}\marg{bacteria}\marg{\catattr{glossdescfont}}\marg{bacteriafont}
\end{codeenv}
(Note that the attribute value is the control sequence name without
the initial backslash.)

I'd like to use the \glostyle{bookindex} style, which is
provided by the \isty{glossary-bookindex} 
package.\footnote{\styfmt{glossary-bookindex}
is distributed with \styfmt{glossaries-extra} v1.21+.} This isn't loaded
automatically, but it can be loaded through the
\styopt{stylemods} package option:
\begin{codeenv}
\cmd{usepackage}[\styopt{record},\comment{use bib2gls}
 \styopt{nostyles},\comment{don't load default style packages}
 \styopt[bookindex]{stylemods},\comment{load glossary-bookindex.sty and patch styles}
 \styopt[bookindex]{style}]\marg{glossaries-extra}
\end{codeenv}
I've used the \styopt{nostyles} package option to suppress loading
the default style packages, since I'm not using them. If you inspect
the \ext{log} file, you may notice that \sty{glossary-tree} is still
loaded. This is because it's required by \sty{glossary-bookindex}
as the \glostyle{bookindex} style is based on the \glostyle{index}
style provided by \sty{glossary-tree}.
With this style I need to use the \longarg{group} switch to instruct
the sort method to automatically create the letter groups.

The \glostyle{bookindex} style doesn't show the \field{description}
field (which means I don't need the \catattr{glossdescfont}
attribute) and, since the \abbrstyle{long-only-short-only} style sets
the \field{name} to the short form by default, only the short form
will show in the glossary. I'd rather it was just the long form.
This could simply be done using \csopt{replicate-fields} to copy the
\field{long} field to the \field{name} field:
\begin{codeenv}
\csopt[\field{long}=\field{name}]{replicate-fields}
\end{codeenv}
Again, I want to consider the possibility of adding other types of
abbreviations and this might not be appropriate for them (for
example, I might want some abbreviations with the long form followed
by the short form in parentheses). Another
approach is to redefine \ics{glsxtrbookindexname} which is used by
the \glostyle{bookindex} style to display the name. This takes the
entry's label as the argument. The default definition is:
\begin{codeenv}
\cs{newcommand}*\marg{\cs{glsxtrbookindexname}}[1]\marg{\cs{glossentryname}\marg{\idx{param}1}}
\end{codeenv}
This can be changed to test for the entry's category:
\begin{codeenv}
\cs{renewcommand}*\marg{\cs{glsxtrbookindexname}}[1]\marg{\comment{}
 \ics{glsifcategory}\marg{\idx{param}1}\marg{bacteria}
 \marg{\cs{glossentrynameother}\marg{\idx{param}1}\marg{\field{long}}}\comment{}
 \marg{\cs{glossentryname}\marg{\idx{param}1}}\comment{}
}
\end{codeenv}
Note that I've used \ics{glossentrynameother} here rather than
\ics{glsentrylong}. This ensures that it follows the same formatting
as \ics{glossentryname} (so it will use \ics{glsnamefont} or the
\catattr{glossnamefont} attribute, the \catattr{glossname}
attribute, and the \idx{postnamehook}, if set). In this case it
picks up the \catattr{glossnamefont} attribute, which is used
instead of \cs{glsnamefont}.

If the \field{sort} field is missing for abbreviation styles, the
fallback value is the \field{short} field (not the \field{name}
field). In this case it would be better to fallback on the
\field{long} field instead, which can be done with the
\csopt{abbreviation-sort-fallback} option:
\begin{codeenv}
\csopt[\field{long}]{abbreviation-sort-fallback}
\end{codeenv}
If I do add other types of abbreviations, they will all be sorted
according to the \field{long} form, but at least this way I can have
some \meta{long} (\meta{short}) names as well.

The complete code is listed below.  The document build is:
\begin{verbatim}
pdflatex sample-bacteria
bib2gls --group sample-bacteria
pdflatex sample-bacteria
\end{verbatim}
This simple example only references entries on the first page so all
entries just have~1 in the number list. The complete document is
shown in \figureref{fig:sample-bacteria.pdf}.

\lstinputlisting[firstline=5]{../examples/sample-bacteria.tex}

\begin{figure}
\figcontents
{%
 \frame{\includegraphics[height=.9\textheight]{../examples/sample-bacteria.pdf}}%
}
{\caption{\filefmt{sample-bacteria.pdf}}}
{fig:sample-bacteria.pdf}
\end{figure}

\filesection{sample-units1.tex}

This example uses the \exfile{baseunits.bib} and
\exfile{derivedunits.bib} files. The aim here is to have a glossary
in two blocks: base units and derived units. This can be achieved by
first loading \exfile{baseunits.bib} with \csopt{group} set to the
desired group title (\qt{Base Units} in this case) and then load
\exfile{derivedunits.bib} with the \csopt{group} set to the desired
title (\qt{Derived Units} in this case). Remember that the
\field{group} field needs to be used as a label. If the group title
contains any problematic characters or commands, then it's better to
use labels:
\begin{codeenv}
\csopt[baseunits]{group}
\end{codeenv}
for the first \igls{resourceset} and
\begin{codeenv}
\csopt[derivedunits]{group}
\end{codeenv}
for the second, and then set the group titles:
\begin{codeenv}
\gls{glsxtrsetgrouptitle}\marg{baseunits}\marg{Base Units}
\gls{glsxtrsetgrouptitle}\marg{derivedunits}\marg{Derived Units}
\end{codeenv}
I've used this method to make it easier to adapt to other languages
that may need extended characters in the group titles.
The \csopt{group} option requires the \longarg{group} switch to
ensure that the \field{group} field is correctly assigned.

The \exfile{baseunits.bib} file use a custom \gls{entrytype}
\atentryfmt{unit}, which must be aliased otherwise \bibgls\
will ignore the entries. I decided to use \atentry{symbol} for
semantic reasons:
\begin{codeenv}
\csopt[unit=symbol]{entry-type-aliases}
\end{codeenv}
Similarly for the custom \atentryfmt{measurement} \gls{entrytype} in
\exfile{derivedunits.bib}:
\begin{codeenv}
\csopt[measurement=symbol]{entry-type-aliases}
\end{codeenv}
Remember that \atentry{symbol} uses the label as the default sort fallback,
so I've changed it to use \field{name} instead:
\begin{codeenv}
\csopt[\field{name}]{symbol-sort-fallback}
\end{codeenv}
An alternative approach would be to alias \atentryfmt{unit}
and \atentryfmt{measurement} to \atentry{entry} instead.

Since there's no \csopt{type} set, all entries end up in the 
\gls{mainglossary}, but since there are two resource commands the glossary
ends up with sorted blocks.

The document doesn't include any commands like \cs{gls}, so
I've use \csopt[all]{selection} to select all entries in the \ext{bib}
files. There won't be any number lists since there are no records.
I need a glossary style that shows the \field{symbol} field so I've
used \glostyle{mcolindexgroup}. Again I've suppressed the automatic
loading of the default styles with \styopt{nostyles} and used
\styopt[mcols]{stylemods} to load \isty{glossary-mcols} and patch the
styles. Note that although I've used \styopt{nostyles}, the
\sty{glossary-tree} style is loaded as it's required by
\sty{glossary-mcols}.

As with the previous example, the custom fields need to be aliased:
\begin{codeenv}
\csopt[
  \fieldfmt{unitname}=\field{name},
  \fieldfmt{unitsymbol}=\field{symbol},
  \fieldfmt{measurement}=\field{description}
]{field-aliases}
\end{codeenv}

The complete document code is listed below. The document build is:
\begin{verbatim}
pdflatex sample-units1
bib2gls --group sample-units1
pdflatex sample-units1
\end{verbatim}
The complete document is shown in \figureref{fig:sample-units1.pdf}.

\lstinputlisting[firstline=5]{../examples/sample-units1.tex}

\begin{figure}
\figcontents
{%
 \frame{\includegraphics[height=.9\textheight]{../examples/sample-units1.pdf}}%
}
{\caption{\filefmt{sample-units1.pdf}}}
{fig:sample-units1.pdf}
\end{figure}

\filesection{sample-units2.tex}

This example is provided for comparison with
\exfile{sample-units1.tex}. Instead of having a single glossary with
sorted blocks this example has two glossaries:
\begin{codeenv}
\ics{newglossary*}\marg{baseunits}\marg{Base Units}
\cs{newglossary*}\marg{derivedunits}\marg{Derived Units}
\end{codeenv}
I've used the \styopt{section} package option to use \ics{section*} for the
glossary titles. This overrides the default \ics{chapter*} which is
used with book or report type of classes. I've also used the
\styopt{nomain} option to suppress the creation of the \gls{mainglossary}
as I want to define my own glossary types instead.

As before the custom \glspl{entrytype} need to be aliased:
\begin{codeenv}
\csopt[unit=symbol]{entry-type-aliases}
\end{codeenv}
for the first \igls{resourceset} and
\begin{codeenv}
\csopt[measurement=symbol]{entry-type-aliases}
\end{codeenv}
for the second. Similarly for the custom entry fields:
\begin{codeenv}
\csopt[
  \fieldfmt{unitname}=\field{name},
  \fieldfmt{unitsymbol}=\field{symbol},
  \fieldfmt{measurement}=\field{description}
]{field-aliases}
\end{codeenv}
The \longarg{group} switch is needed to ensure that the
\field{group} field is automatically assigned by the sort method.

The complete document code is listed below. The document build is:
\begin{verbatim}
pdflatex sample-units2
bib2gls --group sample-units2
pdflatex sample-units2
\end{verbatim}
The complete document is shown in \figureref{fig:sample-units2.pdf}.

\lstinputlisting[firstline=5]{../examples/sample-units2.tex}

\begin{figure}
\figcontents
{%
 \frame{\includegraphics[height=.9\textheight]{../examples/sample-units2.pdf}}%
}
{\caption{\filefmt{sample-units2.pdf}}}
{fig:sample-units2.pdf}
\end{figure}

\filesection{sample-units3.tex}

This is another example that uses the \exfile{baseunits.bib} and
\exfile{derivedunits.bib} files. As before the custom fields need to be aliased:
\begin{codeenv}
\csopt[
  \fieldfmt{unitname}=\field{name},
  \fieldfmt{unitsymbol}=\field{symbol},
  \fieldfmt{measurement}=\field{description}
]{field-aliases}
\end{codeenv}
This time I want two glossaries containing all the units (base and
derived) where the first glossary is ordered by name and the second
is ordered by symbol. This can be done with a single resource
command that instructs \bibgls\ to make the custom \atentryfmt{unit}
and \atentryfmt{measurement} \glspl{entrytype} behave like
\atentry{dualsymbol}:
\begin{codeenv}
\csopt[
  unit=dualsymbol,
  measurement=dualsymbol
]{entry-type-aliases}
\end{codeenv}
This causes the \field{name} and \field{symbol} fields to be swapped in the
dual list. Remember that the fallback for the \field{sort} field is the label
for the symbol \glspl{entrytype} so I need \csopt[\field{name}]{symbol-sort-fallback}
to fallback on \field{name} field instead. (Alternative, I could just sort by
the \field{name} field instead using \csopt[\field{name}]{sort-field}.)

The primary entries can still be sorted according to the default
locale collator, but the dual entries need a sort method that's
better suited to symbols. Fortunately, \bibgls\ has some (very
limited) support for \isty{siunitx} and is able to interpret
the \ics{si} commands in the sample \ext{bib} files. Since 
\idxpl{SIunit} are a mix of letters and numbers I've used one of the
letter-number methods listed in
\tableref{tab:sortoptionsletternumber}.

I've decided to define a custom style for the first glossary. Since
it's based on the \glostyle{long3col-booktabs} style I need to 
load \isty{glossary-longbooktabs}, which can conveniently be done with
the \styopt{stylemods} option. This uses \env{longtable} (provided
by \isty{longtable}, which is automatically loaded) which means
an extra \LaTeX\ call is required in the build process to ensure the
column widths are correct. Again I'm using \styopt{nostyles} to
suppress the automatic loading of the default styles, however
\sty{glossary-tree} will be loaded as it's listed in the value of
\styopt{stylemods} and \isty{glossary-long} will be loaded as it's
required by \sty{glossary-longbooktabs}. I can't use my custom style
in the \styopt{style} package option as it hasn't been defined at
that point. The default \glostyle{list} style is now unavailable
since \styopt{nostyles} has prevented it from being defined, so 
I've used \styopt[alttree]{style} to ensure there's a valid default
style.

Since my custom style is based on one of the long styles, I need to
set the length register \ics{glsdescwidth} to adjust the width of
the description column:
\begin{codeenv}
\cmd{setlength}\marg{\cs{glsdescwidth}}\marg{.4\cmd{hsize}}
\end{codeenv}
The \glostyle{long3col-booktabs} style sets up a three column
\env{longtable} so I just need to adjust the table header (to rename
the column headers) and the way each row is formatted:
\begin{codeenv}
\ics{newglossarystyle}\marg{units}\comment{style name}
\marg{\comment{base it on \glostyle{long3col-booktabs}}
  \ics{setglossarystyle}\marg{long3col-booktabs}\comment{}
  \cs{renewcommand}*\marg{\ics{glossaryheader}}\marg{\comment{}
    \ics{toprule}
    \ics{bfseries} Name \idx{colsep}
    \cs{bfseries} Measurement \idx{colsep}
    \cs{bfseries} Symbol
    \ics{tabularnewline}\ics{midrule}\ics{endhead}
    \ics{bottomrule}\ics{endfoot}}\comment{}
\comment{main entries:}
  \cs{renewcommand}\marg{\ics{glossentry}}[2]\marg{\comment{}
    \ics{glsentryitem}\marg{\idx{param}\idx{param}1}\ics{glstarget}\marg{\idx{param}\idx{param}1}\marg{\cs{glossentryname}\marg{\idx{param}\idx{param}1}} \idx{colsep}
    \cs{glossentrydesc}\marg{\idx{param}\idx{param}1}\ics{glspostdescription} \idx{colsep}
    \cs{glossentrysymbol}\marg{\idx{param}\idx{param}1}\cs{tabularnewline}
  }\comment{}
}
\end{codeenv}
There are no \glspl{sub-entry} in this document so I haven't bothered to
redefine \ics{subglossentry}. (The tabular styles aren't appropriate
for \glspl{hierarchicalglossary}.) This puts the symbol into the third
column (rather than the \gls{locationlist}, which is ignored).
This style supports the letter group separator (although it doesn't
title the groups), so if I want this I need to use the
\longarg{group} switch.

I also need to make sure I've defined a glossary for the dual
entries:
\begin{codeenv}
\cs{newglossary*}\marg{units}\marg{Units of Measurement (by SI unit)}
\end{codeenv}
and specify the glossary types for the primary and dual entries:
\begin{codeenv}
\csopt[main]{type},
\csopt[units]{dual-type}
\end{codeenv}
The complete document code is listed below. The document build is:
\begin{verbatim}
pdflatex sample-units3
bib2gls --group sample-units3
pdflatex sample-units3
pdflatex sample-units3
\end{verbatim}
The two pages of the document are shown in
\figureref{fig:sample-units3.pdf}.

\lstinputlisting[firstline=5]{../examples/sample-units3.tex}

\begin{figure}
\figcontents
{\pdftwocol{../examples/sample-units3.pdf}{2}}
{\caption{\filefmt{sample-units3.pdf}}}
{fig:sample-units3.pdf}
\end{figure}

\filesection{sample-media.tex}

This example uses the sample files \exfile{books.bib},
\exfile{films.bib}, \exfile{no-interpret-preamble.bib} and
\exfile{interpret-preamble.bib}.  The aim is to produce a combined
list of books and films in a single glossary. The films are based on
some of the books so some of the entries have the same name. The
default setting for identical sort values is
\csopt[id]{identical-sort-action}, which means that the ordering for
the duplicate names is based on the entry labels. This can lead to
the odd effect of sometimes having the film listed first
(\code{film.thebigsleep} comes before \code{thebigsleep}) and
sometimes having the book listed first (\code{brightonrock} comes
before \code{film.brightonrock}).

One possible solution would be to also assign prefixes for the book
labels, but \csopt{label-prefix} is applied to all primary entries
for the given resource set and can't be applied selectively, so this
would require editing the \exfile{books.bib} file.

A more consistent approach would be to fallback on the category.
This means that the \field{category} field needs to be set. There
are two simple ways to achieve this: use \csopt[same as base]{category}
(which sets the \field{category} to \code{books} for entries in 
\exfile{books.bib} and to \code{films} for entries in
\exfile{films.bib}) or alias the custom \fieldfmt{identifier} field to
\field{category}. I've chosen the latter method and also provided
aliases for the custom \fieldfmt{year} and \fieldfmt{cast} fields:
\begin{codeenv}
\csopt[\fieldfmt{identifier}=\field{category},\fieldfmt{year}=\field{user1},\fieldfmt{cast}=\field{user2}]{field-aliases},
\csopt[\field{category}]{identical-sort-action}
\end{codeenv}
This ensures that books always come before films with the same
title. An oddity is the film \qt{Whisky Galore!}\ which is one
character different from the book \qt{Whisky Galore} but the default
locale collator ignores punctuation so the two titles are considered
identical by the collator (but not by \csopt[non-unique]{sort-suffix}).
If a letter comparison was used instead, they would no longer be 
considered identical, but in this case the film would
still be placed after the book since the film title is longer.

Since I've set the \field{category} I can provide semantic
formatting commands (as for \exfile{sample-bacteria.tex}):
\begin{codeenv}
\cs{newcommand}*\marg{\cmd{bookfont}}[1]\marg{\cs{emph}\marg{\idx{param}1}}
\cs{newcommand}*\marg{\cmd{filmfont}}[1]\marg{\ics{textsf}\marg{\ics{em} \idx{param}1}}
\cs{glssetcategoryattribute}\marg{book}\marg{\catattr{textformat}}\marg{bookfont}
\cs{glssetcategoryattribute}\marg{book}\marg{\catattr{glossnamefont}}\marg{bookfont}
\cs{glssetcategoryattribute}\marg{film}\marg{\catattr{textformat}}\marg{filmfont}
\cs{glssetcategoryattribute}\marg{film}\marg{\catattr{glossnamefont}}\marg{filmfont}
\end{codeenv}
I've given films a slightly different format to make them easier to
distinguish from books of the same name.

Both \exfile{books.bib} and \exfile{films.bib} had the custom
\fieldfmt{year} field, indicating the year of first publication
or release, which I've assigned to the \field{user1} field. 
I can define \idxpl{postnamehook} for each category
to append the year in brackets after the name is displayed in the
glossary:
\begin{codeenv}
\cs{newcommand}*\marg{\postnamehook{book}}\marg{\comment{}
 \cs{ifglshasfield}\marg{\field{user1}}\marg{\cs{glscurrententrylabel}}\comment{}
 \marg{\cs{space}(published \cs{glscurrentfieldvalue})}\comment{}
 \marg{}\comment{}
}
\strut
\cs{newcommand}*\marg{\postnamehook{film}}\marg{\comment{}
 \cs{ifglshasfield}\marg{\field{user1}}\marg{\cs{glscurrententrylabel}}\comment{}
 \marg{\cs{space}(released \cs{glscurrentfieldvalue})}\comment{}
 \marg{}\comment{}
}
\end{codeenv}
As with the post-description hook, if you have at least 
\styfmt{glossaries-extra} v1.31, it's better to use:
\begin{codeenv}
\ics{glsdefpostname}\margm{category}
\end{codeenv}
instead of:
\begin{codeenv}
\cs{newcommand}\marg{\cs{glsxtrpostnamecategory}}
\end{codeenv}
as it can guard against accidental misspelling of the \code{glsxtrpostname}
part of the command name.

I've assigned the \fieldfmt{cast} field to the \field{user2} field,
and since this field uses \BibTeX's contributor markup I need to 
convert this to a form that's easier to customize:
\begin{codeenv}
\csopt[\field{user2}]{bibtex-contributor-fields}
\end{codeenv}
I'm not sorting by this field and it would look better in the
document to list the forenames before the surname so I've also done:
\begin{codeenv}
\csopt[forenames]{contributor-order}
\end{codeenv}
Since I have at least version 2.28 of \sty{datatool-base} installed, 
the list will be formatted using \ics{DTLformatlist}. If I want an 
Oxford comma, I need to redefine \ics{DTLlistformatoxford} in the document:
\begin{codeenv}
\cs{renewcommand}*\marg{\cs{DTLlistformatoxford}}\marg{,}
\end{codeenv}
If I want to change \qt{\&} to \qt{and} I also need to redefine
\ics{DTLandname}:
\begin{codeenv}
\cs{renewcommand}*\marg{\cs{DTLandname}}\marg{and}
\end{codeenv}
If \ics{DTLformatlist} isn't defined (\sty{datatool-base} v2.27 or
earlier), the cast list will look a little odd as it uses a comma
separator between all elements of this list, including the final
pair (so there's no final \& or \qt{and}).

I've provided a \idx{postdescriptionhook}
\ics{glsxtrpostdesccategory} to append the cast list:
\begin{codeenv}
\cs{newcommand}*\marg{\postdeschook{film}}\marg{\comment{}
 \cs{ifglshasfield}\marg{\field{user2}}\marg{\cs{glscurrententrylabel}}\comment{}
 \marg{\comment{}
   \cs{glsxtrrestorepostpunc} \comment{requires glossaries-extra v1.23+}
   \cs{cs.space}featuring \cs{glscurrentfieldvalue}
 }\comment{}
 \marg{}\comment{}
}
\end{codeenv}
This uses \ics{glsxtrrestorepostpunc} to restore the
post-description punctuation if it was suppressed with
\ics{glsxtrnopostpunc}. This means that if I decide not to include
the \field{user2} field then the post-description punctuation will
be revert back to being suppressed for entries containing
\cs{glsxtrnopostpunc} in the \field{description} field.

I haven't referenced any of the entries in the main body of the
document, so I've used \csopt[all]{selection} to select all entries.
This means that there are no number lists on the first document
build (\LaTeX+\bibgls+\LaTeX) but the next build would show \glspl{location} for 
the books that have been referenced by the film entries. Since this
looks a bit odd, I've added \csopt[false]{save-locations} to
prevent \bibgls\ from saving the \glspl{location}.

I've used a style that shows letter group headings so I need to use
the \longarg{group} switch.
The complete document code is listed below. The document build is:
\begin{verbatim}
pdflatex sample-media
bib2gls --group sample-media
pdflatex sample-media
\end{verbatim}
The four pages of the document are shown in
\figureref{fig:sample-media.pdf}.

\lstinputlisting[firstline=5]{../examples/sample-media.tex}

\begin{figure}
\figcontents
{\pdftwocol{../examples/sample-media.pdf}{4}}
{\caption{\filefmt{sample-media.pdf}}}
{fig:sample-media.pdf}
\end{figure}

\filesection{sample-people.tex}

This example uses the files \exfile{people.bib},
 \exfile{no-interpret-preamble.bib} and \exfile{interpret-preamble.bib}. The
aim here is to have a list of people ordered alphabetically by
surname with a brief description, the same list ordered by date of
birth and an index of all the people without their details but with
a number list indicating where that person was mentioned in the
document. The first two lists shouldn't include aliases but the
index should. Not all the entries defined in \exfile{people.bib}
are included in the document. Those that aren't either explicitly
referenced or aliased are filtered by the \csopt{selection} criteria.
I've used a style that shows letter group headings so I need to use
the \longarg{group} switch.

Since this is just an example document all the \cs{gls} commands
only occur on page~1, which means that each number list is just
\qt{1}. A real document would have the references scattered about.
The aliases haven't actually been referenced anywhere in the
document.

The \fieldfmt{born}, \fieldfmt{died} and \fieldfmt{othername} fields will be
ignored by default since they don't correspond to recognised keys,
so the keys either need to be defined or the fields need to be
mapped to existing keys. In this case I've decided to map them to
the \field{user1}, \field{user2} and \field{user3} fields using
\csopt{field-aliases}:
\begin{codeenv}
\csopt[\fieldfmt{born}=\field{user1},\fieldfmt{died}=\field{user2},\fieldfmt{othername}=\field{user3}]{field-aliases}
\end{codeenv}
Although the aliases haven't been referenced in the document, I've
taken into account the possibility that they might later be added.
To prevent them from showing in the first two lists I've filtered
them out. This is easy to do since the aliases are all defined using
\atentry{index} whereas the remaining (non-aliased) entries are
defined using \atentry{entry} so \csopt{match} can be used to only
select entries defined with \atentry{entry}: 
\begin{codeenv}
\csopt[entrytype=entry]{match}
\end{codeenv}

I'd like the first use of \ics{gls} to display the full name, except
for the entry that has the \field{first} field set. The remaining
entries only have \field{text} set to a shortened version of the
name so they need to have the \field{name} field copied to the
\field{first} field using \csopt{replicate-fields}:
\begin{codeenv}
\csopt[\field{name}=\marg{\field{first}}]{replicate-fields}
\end{codeenv}

I'd like the first use to show the other name in parentheses where
provided. The simplest way to achieve this is by defining the
\idx{postlinkhook} \ics{glsxtrpostlinkcategory}. If the
\field{category} field isn't specified it will default to
\code{general} (for entries defined with \atentry{entry}), so I
could just define \csfmt{glsxtrpostlinkgeneral} but to allow for
the possibility of extending the document to incorporate other types
of entries I decided to set the \field{category} to \code{people}
through the use of the \csopt{category} option:
\begin{codeenv}
\csopt[people]{category}
\end{codeenv}

This means that I now need to define a command called
\csfmt{glsxtrpostlinkpeople} that will be used after instances of
\cs{gls} etc where the entry has the \field{category} set to
\code{people}. This first tests if that was the first use of the
entry with \ics{glsxtrifwasfirstuse} and then tests if the
\field{user3} field is set. If so, it does a space followed by that
field's value in parentheses. The entry's label can be obtained from
\ics{glslabel}:
\begin{codeenv}
\cs{newcommand}*\marg{\postlinkhook{people}}\marg{\comment{}
 \cs{glsxtrifwasfirstuse}
 \marg{\comment{}
   \cs{ifglshasfield}\marg{\field{user3}}\marg{\cs{glslabel}}\comment{}
   \marg{\cs{space}(\cs{glscurrentfieldvalue})}\comment{}
   \marg{}\comment{}
 }\comment{}
 \marg{}\comment{}
}
\end{codeenv}

I'd also like to do something similar after the name when the entry
is displayed in the glossary. This means defining the \idx{postnamehook}
\ics{glsxtrpostnamecategory}, in this case
\csfmt{glsxtrpostnamepeople}. The entry's label is referenced
with \ics{glscurrententrylabel}:
\begin{codeenv}
\cs{newcommand}*\marg{\postnamehook{people}}\marg{\comment{}
 \cs{ifglshasfield}\marg{\field{user3}}\marg{\cs{glscurrententrylabel}}\comment{}
 \marg{\cs{space}(\cs{glscurrentfieldvalue})}\comment{}
 \marg{}\comment{}
}
\end{codeenv}
(A different command is used since \cs{gls} may occur in the
description, which would interfere with the current entry label if
they shared the same command to reference the label.)

The \idx{postdescriptionhook} can be used to append the birth and
death dates. Although all the entries that have been selected from \exfile{people.bib} have a 
\fieldfmt{died} field, I've added a check for the corresponding
\field{user3} field in case new references are added for people who are
still alive:
\begin{codeenv}
\cs{newcommand}*\marg{\postdeschook{people}}\marg{\comment{}
 \cs{ifglshasfield}\marg{\field{user1}}\marg{\cs{glscurrententrylabel}}
 \marg{\comment{born}
   \cs{space}(\cs{glscurrentfieldvalue}\ics{comma}-{}-\cs{comma}\comment{}
     \cs{ifglshasfield}\marg{user2}\marg{\cs{glscurrententrylabel}}
     \marg{\comment{died}
        \cs{glscurrentfieldvalue}
     }\comment{}
     \marg{}\comment{}
   )\comment{}
 }\comment{}
 \marg{}\comment{}
}
\end{codeenv}

The first list is quite straight-forward and can be created with:
\begin{codeenv}
\gls!{GlsXtrLoadResources}\oarg{
  \csopt[people]{src},
  \csopt[entrytype=entry]{match},
  \csopt[people]{category},
  \csopt[\field{name}=\marg{\field{first}}]{replicate-fields},
  \csopt[\fieldfmt{born}=\field{user1},\fieldfmt{died}=\field{user2},\fieldfmt{othername}=\field{user3}]{field-aliases}
}
\end{codeenv}
I have used the \csopt{sort} option and there's no document language, 
so \bibgls\ will sort according to my locale. The custom
commands \gls{sortname} and \gls{sortvonname} ensure that the
entries are all sorted alphabetically according to the surnames.

The second list can easily be created by adding the \csopt{secondary}
option:
\begin{codeenv}
\csopt[date:\field{user1}:bybirth]{secondary}
\end{codeenv}
This sorts according to the \field{user1} field (which was
originally the \fieldfmt{birth} field).
Note that different locales have different default date formats.
There may also be a difference in the default date format depending
on the Java \idx{localeprovider}. For example, if you switch from
using the \idx{JRE} to using the \idx{CLDR} you may find a change in
the default format. In case the format provided in the \ext{bib}
file isn't recognised, the required format can be set with:
\begin{codeenv}
  \csopt[d MMM YYYY G]{secondary-date-sort-format}
\end{codeenv}

I've changed the date group headings by redefining \gls{bibglsdategroup} and
\gls{bibglsdategrouptitle}, which means that the grouping in the
\code{bybirth} glossary will be in the form \meta{year} \meta{era}:
\begin{codeenv}
\cs{newcommand}\marg{\gls{bibglsdategroup}}[7]\marg{\idx{param}1\idx{param}4\idx{param}7}
\cs{newcommand}\marg{\gls{bibglsdategrouptitle}}[7]\marg{\cs{number}\idx{param}1\ \idx{param}4}
\end{codeenv}
I've also defined the \code{bybirth} glossary and supplied a title:
\begin{codeenv}
\cs{newglossary*}\marg{bybirth}\marg{People (Ordered by Birth)}
\end{codeenv}

The first two glossaries have entries with fairly long names (especially those
with the \idx{postnamehook}), so the best style is the
\glostyle{altlistgroup}. The \sty{glossaries-extra-stylemods}
package patches this style to discourage page breaks occurring after
group headings, so I've also used the \styopt{stylemods} 
option to automatically load that package. I'd like to use the
\glostyle{bookindex} style for the index, which is provided by 
\sty{glossary-bookindex}, so I need:
\begin{codeenv}
\styopt[list,bookindex]{stylemods}
\end{codeenv}
This ensures that \isty{glossary-list} and \sty{glossary-bookindex}
are loaded and patches the list styles.

The first two glossaries would look better with a terminating 
\idx{full-stop}, so I've used the \styopt{postdot} package option. 
(The \glostyle{bookindex} style doesn't use the \field{description}
field and therefore doesn't use the \idx{postdescriptionhook}.)
The index glossary type can be defined with the \styopt{index}
package option. I've set the default style to \glostyle{altlistgroup}
but this can locally be changed to \glostyle{bookindex} when I
display the index. The \styopt{record} option is needed to use
\bibgls, so the \styfmt{glossaries-extra} package is loaded with:
\begin{codeenv}
\cmd{usepackage}[\styopt{record},\comment{using bib2gls}
 \styopt{index},\comment{create index glossary}
 \styopt{postdot},\comment{dot after descriptions}
\comment{load glossary-list.sty and glossary-bookindex.sty and patch:}
 \styopt[list,bookindex]{stylemods},
 \styopt[altlistgroup]{style}]\marg{glossaries-extra}
\end{codeenv}

The index needs to include all the entries that have already been
defined but also needs to include the aliased entries. This means
that existing entries simply need their label copied to the
\code{index} glossary but the other entries need to be defined
so this requires setting the \csopt{action} option:
\begin{codeenv}
\csopt[define or copy]{action}
\end{codeenv}
I would also like to have groups in the index (which the
\glostyle{bookindex} style supports) so I need to specify a field
in which to save the group information using
\csopt{copy-action-group-field}:
\begin{codeenv}
\csopt[indexgroup]{copy-action-group-field}
\end{codeenv}
I need to remember to redefine \ics{glsxtrgroupfield} to this value before
displaying the index:
\begin{codeenv}
\cs{renewcommand}\marg{\cs{glsxtrgroupfield}}\marg{indexgroup}
\end{codeenv}

The aliased entries won't be selected by default since they haven't
been used in the document, so I need to change the selection
criteria with \csopt{selection}:
\begin{codeenv}
\csopt[recorded and deps and see]{selection}
\end{codeenv}

In the index, I'd like the surnames first. This can be done by
redefining the custom commands used in the \field{name} fields.
There's a slight complication here. These commands aren't defined on
the first \LaTeX\ run as their definitions are written to the
\ext{glstex} file by \bibgls, so I can't use \cs{renewcommand} (although 
I could use \cs{glsrenewcommand}).
Instead I've provided some custom commands:
\begin{codeenv}
\cs{newcommand}*\marg{\cmd{swaptwo}}[2]\marg{\idx{param}2, \idx{param}1}
\cs{newcommand}*\marg{\cmd{swapthree}}[3]\marg{\idx{param}2 \idx{param}3, \idx{param}1}
\end{codeenv}
Now I just need to make an assignment using \ics{let}:
\begin{codeenv}
\cs{let}\gls{sortname}\cmd{swaptwo}
\cs{let}\gls{sortart}\cmd{swaptwo}
\cs{let}\gls{sortvonname}\cmd{swapthree}
\end{codeenv}
This doesn't perform any check to determine if the commands are
already defined so there won't be a problem on the first run.

The first two glossaries shouldn't have number lists:
\begin{codeenv*}
\cs{printunsrtglossary}\oarg{\printglossopt[People (Alphabetical)]{title},\printglossopt{nonumberlist}}
\cs{printunsrtglossary}\oarg{\printglossopt[bybirth]{type},\printglossopt[false]{target},\printglossopt{nonumberlist}}
\end{codeenv*}
I'd like to use \sty{hyperref} but I have to switch off the
hypertargets for the second glossary otherwise I'll end up with
duplicate targets. This is done with \code{\printglossopt[false]{target}}. 
All references using \cs{gls} etc will link to the first glossary.

I could also do this for the index but the cross-references in the
aliased entries will link to the first glossary rather than the
relevant entry in the index. The simplest way to fix this is to
redefine \ics{glolinkprefix} to provide a different target:
\begin{codeenv}
\cs{renewcommand}*\marg{\cs{glolinkprefix}}\marg{idx:}
\end{codeenv}
These redefinitions need to be done before the index. I've decided
to use the starred \ics{printunsrtglossary*} to localise these
changes, although that's not needed for this document since the
index comes right at the end:
\begin{codeenv*}
\cs{printunsrtglossary*}
 \oarg{\printglossopt[index]{type},\printglossopt[\glostyle{bookindex}]{style}}
 \marg{\comment{}
   \cs{let}\gls!{sortname}\cmd{swaptwo}
   \cs{let}\gls!{sortart}\cmd{swaptwo}
   \cs{let}\gls!{sortvonname}\cmd{swapthree}
   \cs{renewcommand}\marg{\cs{glsxtrgroupfield}}\marg{indexgroup}\comment{}
   \cs{renewcommand}*\marg{\cs{glolinkprefix}}\marg{idx:}\comment{}
 }
\end{codeenv*}
The complete document code is listed below. The document build is:
\begin{verbatim}
pdflatex sample-people
bib2gls --group --break-space sample-people
pdflatex sample-people
\end{verbatim}
The four pages of the document are shown in \figureref{fig:sample-people.pdf}.

\lstinputlisting[firstline=5]{../examples/sample-people.tex}

\begin{figure}
\figcontents
{\pdftwocol{../examples/sample-people.pdf}{4}}
{\caption{\filefmt{sample-people.pdf}}}
{fig:sample-people.pdf}
\end{figure}

\filesection{sample-authors.tex}

This example uses the files \exfile{people.bib},
\exfile{books.bib}, \exfile{no-interpret-preamble.bib} and 
\exfile{interpret-preamble2.bib}. The aim is to reference the books
in \exfile{books.bib} and have them listed by author. This means
finding a way of assigning each book entry a \field{parent} field that
contains the label identifying the relevant author in \exfile{people.bib}.
I've used a style that shows letter group headings so I need to use
the \longarg{group} switch.

To recap, each author is defined in \exfile{people.bib} in the form:
\begin{codeenv}
\atentry{entry}\marg{dickens,
  \field{name}=\marg{\gls{sortname}\marg{Charles}\marg{Dickens}},
  \field{text}=\marg{Dickens},
  \field{description}=\marg{English writer and social critic},
  \fieldfmt{born}=\marg{7\idx{nbspchar}February 1812 AD},
  \fieldfmt{died}=\marg{9\idx{nbspchar}June 1870 AD},
  \fieldfmt{identifier}=\marg{person}
}
\end{codeenv}
and each book is defined in \exfile{books.bib} in the form:
\begin{codeenv}
\atentry{entry}\marg{bleakhouse,
  \field{name}=\marg{Bleak House},
  \field{description}=\marg{novel by Charles Dickens},
  \fieldfmt{identifier}=\marg{book},
  \fieldfmt{author}=\marg{\gls{sortmediacreator}\marg{Charles}\marg{Dickens}},
  \fieldfmt{year}=\marg{1852}
}
\end{codeenv}
There's a field here (the custom \fieldfmt{author} field) that contains
the author's name, and this can be aliased to the \field{parent}
field with \csopt{field-aliases}:
\begin{codeenv}
\csopt[\fieldfmt{author}=\field{parent}]{field-aliases}
\end{codeenv}
but the author's label in the \exfile{people.bib}
file is just the \idx!{lowercase} surname.

Remember from \sectionref{sec:texparserlib} that the interpreter
will be used on the \field{parent} field if the value contains
\idx{escchar} or \idx{bgroupchar} or \idx{egroupchar} and
\csopt[true]{interpret-label-fields}.  This means that with this
field alias and the interpreter on, \bibgls\ will attempt to
interpret the field contents. So all that's needed is to ensure that
\bibgls\ is given a definition of \gls{sortmediacreator} that
ignores the first argument and converts the second argument to lower
case. This definition is available in
\exfile{interpret-preamble2.bib} but, since this file uses
\cs{renewcommand} rather than \cs{providecommand},
\csopt[false]{write-preamble} is required to prevent \LaTeX\ from
picking up this definition.

As with the \exfile{sample-people.tex} example, I need to copy
the \field{name} field to the \field{first} field if that field is
missing using \csopt{replicate-fields}:
\begin{codeenv}
\csopt[\field{name}=\marg{\field{first}}]{replicate-fields}
\end{codeenv}
and I also want to provide a semantic command to format the book
title, so the field aliases also need to convert the custom
\fieldfmt{identifier} field to \field{category}:
\begin{codeenv}
\csopt[\fieldfmt{identifier}=\field{category},\fieldfmt{author}=\field{parent}]{field-aliases}
\end{codeenv}
so that the document can set the \catattr{textformat} and
\catattr{glossnamefont} attributes:
\begin{codeenv}
\cs{newcommand}*\marg{\cmd{bookfont}}[1]\marg{\cs{emph}\marg{\idx{param}1}}
\cs{glssetcategoryattribute}\marg{book}\marg{\catattr{textformat}}\marg{bookfont}
\cs{glssetcategoryattribute}\marg{book}\marg{\catattr{glossnamefont}}\marg{bookfont}
\end{codeenv}
As with \exfile{sample-media.tex}, the terminating question mark at
the end of some of the \field{name} fields can cause an awkward
situation if \cs{gls} is used at the end of a sentence. This can be
dealt with by getting \bibgls\ to make a note of the fields that
end with sentence-terminating punctuation through the use of
the \csopt{check-end-punctuation} option. In this example, the
\field{name}, \field{text} and \field{first} fields are the same for
all the books, so it's sufficient just to check the \field{name}
field:
\begin{codeenv}
\csopt[\field{name}]{check-end-punctuation}
\end{codeenv}
With \sty{glossaries-extra} v1.23+ it's easy to hook into the
\idx{postlinkhook} to check if 
\fielddisp{fieldendpunc}{nameendpunc} exists:
\begin{codeenv}
\cs{renewcommand}*\marg{\cs{glsxtrifcustomdiscardperiod}}[2]\marg{\comment{}
 \cs{GlsXtrIfFieldUndef}\marg{\fielddisp{fieldendpunc}{nameendpunc}}\marg{\cs{glslabel}}\marg{\idx{param}2}\marg{\idx{param}1}\comment{}
}
\end{codeenv}
This will now cause the full stops following:
\begin{codeenv}
\cs{gls}\marg{whydidnttheyaskevans}.
\end{codeenv}
and
\begin{codeenv}
\cs{gls}\marg{doandroidsdreamofelectricsheep}.
\end{codeenv}
to be discarded.

The complete document code is listed below. The document build is:
\begin{verbatim}
pdflatex sample-authors
bib2gls --group sample-authors
pdflatex sample-authors
\end{verbatim}
The resulting document is shown in \figureref{fig:sample-authors.pdf}.

\lstinputlisting[firstline=5]{../examples/sample-authors.tex}

\begin{figure}
\figcontents
{%
 \frame{\includegraphics[height=.9\textheight]{../examples/sample-authors.pdf}}%
}
{\caption{\filefmt{sample-authors.pdf}}}
{fig:sample-authors.pdf}
\end{figure}

\filesection{sample-citations.tex}

This example uses the \BibTeX\ file \exfile{citations.bib} to create
a document that has both a bibliography created by \BibTeX\ and
glossaries created by \bibgls\ listing the authors and the titles.
There are no glossary reference commands, such as \cs{gls}, but
\bibgls\ can be run with \longarg{cite-as-record} to treat the
\ics{citation} commands (written to the \ext{aux} file by \ics{cite})
as ignored records. Since \cs{cite} doesn't record the page number,
there are no associated \glspl{location}.

The \mainglossary\ isn't required, so I've used
\styopt{nomain} to suppress its creation. I want to use both the
\glostyle{altlist} and \glostyle{indexgroup} styles but none of the
other styles, so I've used \styopt{nostyles} to prevent the
automatic loading of the default style packages and 
\styopt{stylemods} to load the \sty{glossary-tree} and
\sty{glossary-list} packages and patch the styles. A \idx{full-stop}
is automatically placed after the descriptions with
\styopt{postdot}.
\begin{codeenv}
\cs{usepackage}\oarg{\styopt{record},\comment{using bib2gls}
\styopt{nomain},\comment{don't define main glossary}
\styopt{postdot},\comment{full stop after descriptions}
\styopt{nostyles},\comment{don't load default styles}
\comment{load glossary-tree and glossary-list and patch styles:}
\styopt[tree,list]{stylemods}
}\marg{glossaries-extra}
\end{codeenv}
Next I need to create the glossaries for the list of authors and
list of titles:
\begin{codeenv}
\cs{newglossary*}\marg{contributors}\marg{Authors}
\cs{newglossary*}\marg{titles}\marg{Titles}
\end{codeenv}
The simplest way of assigning the authors to the \code{contributors}
glossary and the titles to the \code{titles} glossary is to use:
\begin{codeenv}
\csopt[contributors]{type}
\end{codeenv}
in the \idx{resourceset} and provide a
modified version of \gls{bibglsnewbibtexentry} that assigns
\field{type} after the options:
\begin{codeenv}
\cs{newcommand}\marg{\gls{bibglsnewbibtexentry}}[4]\marg{\comment{}
 \gls{longnewglossaryentry}*\marg{\idx{param}1}\marg{\field{name}=\marg{\idx{param}3},\idx{param}2,\field{type}=\marg{titles}}\marg{\idx{param}4}\comment{}
}
\end{codeenv}
The standard \BibTeX\ \glspl{entrytype} need aliasing to
\atentry{bibtexentry}:
\begin{codeenv}
\csopt[\ics{GlsXtrBibTeXEntryAliases}]{entry-type-aliases}
\end{codeenv}
and the \fieldfmt{title} field is aliased to \field{name}:
\begin{codeenv}
\csopt[\fieldfmt{title}=\field{name}]{field-aliases}
\end{codeenv}
(The other fields aren't required for the glossary lists.)
The \field{category} is set to the original \gls{entrytype}:
\begin{codeenv}
\csopt[same as original entry]{category}
\end{codeenv}
So, for example, an entry that's provided in the \ext{bib} file with
\atentryfmt{article} has the \field{category} field set to
\code{article}. (Compare this with \csopt[same as entry]{category}
which would set the \field{category} to \code{bibtexentry}.)
The spawned entries are all defined using \atentry{contributor} and
aren't aliased so both the \gls{entrytype} and the original \gls{entrytype}
are \code{contributor}.

In order to list the titles according to category, I've use this as
the sort field:
\begin{codeenv}
\csopt[\field{category}]{sort-field}
\end{codeenv}
and setting the sort suffix to the \field{name} field sub-sorts
the \atentry{bibtexentry} types according to the title 
(which was aliased to the \field{name}) and the
\atentry{contributor} types according to the author:
\begin{codeenv}
\csopt[\field{name}]{sort-suffix}
\end{codeenv}

Next the groups identified by the labels \code{article} and \code{book} 
are assigned titles.
\begin{codeenv}
\gls{glsxtrsetgrouptitle}\marg{article}\marg{Articles}
\gls{glsxtrsetgrouptitle}\marg{book}\marg{Books}
\end{codeenv}
The \field{group} field is actually set to the associated letter by the default
\csopt{sort} method. The desired labels are stored in the \field{category} 
field. Since the entries are sorted by category, then they are naturally 
in those sub-blocks, which means that the group titles can be set by locally
redefining \ics{glsxtrgroupfield} to \code{category}:
\begin{codeenv}
\cs{printunsrtglossary*}[\printglossopt[titles]{type},\printglossopt[indexgroup]{style}]
\marg{\comment{}
 \cs{renewcommand}\marg{\cs{glsxtrgroupfield}}\marg{\field{category}}\comment{}
 \cs{renewcommand}\marg{\cs{glstreenamefmt}}[1]\marg{\cs{emph}\marg{\idx{param}1}}\comment{}
 \cs{renewcommand}\marg{\cs{glstreegroupheaderfmt}}[1]\marg{\ics{textbf}\marg{\idx{param}1}}\comment{}
}
\end{codeenv}
This again contradicts the advice given in
\sectionref{sec:logicaldivisions} as I'm sorting by the
\field{group} label. (Technically it's sorting by the
\field{category} label but this is being used as the group.) In this
case it's not a problem as the labels closely match the titles and
the sorting options ensure that the groups aren't broken up.

There's no \field{description} field set for these entries, but the
\idx{postdescriptionhook} can still be used to append information.
In this case, I've appended a cross-reference to the bibliography.
Since the bibliography entry and the glossary term both have the
same label, the citation can easily be obtained with
\code{\ics{cite}\marg{\cs{glscurrententrylabel}}}:
\begin{codeenv}
\cs{newcommand}\marg{\postdeschook{article}}\marg{\cs{cite}\marg{\cs{glscurrententrylabel}}}
\cs{newcommand}\marg{\postdeschook{book}}\marg{\cs{cite}\marg{\cs{glscurrententrylabel}}}
\end{codeenv}
Note that this needs to be done for each \BibTeX\ \gls{entrytype}, but in
this case the \ext{bib} file only contains \atentryfmt{article} and
\atentryfmt{book} entries. (Similarly for the group titles above.)

The list of contributors can simply be displayed with:
\begin{codeenv}
\cs{printunsrtglossary}[\printglossopt[contributors]{type},\printglossopt[altlist]{style}]
\end{codeenv}
This will only list the names as there's no description, but again
the \idx{postdescriptionhook} can be used, in this case for the \code{contributor}
category. The hook iterates
over the internal list provided by the \field{bibtexentry} field.
This allows the titles to be listed as well:
\begin{codeenv}
\cs{newcommand}\marg{\postdeschook{contributor}}\marg{\comment{}
  \ics{glsxtrifhasfield}\marg{\field{bibtexentry}}\marg{\cs{glscurrententrylabel}}\comment{}
  \marg{\comment{}
    \cs{glsxtrfieldforlistloop}
    \marg{\cs{glscurrententrylabel}}\marg{\field{bibtexentry}}\comment{}
    \marg{\csfmt{contributorhandler}}\comment{}
  }\comment{}
  \marg{\ics{par} No titles.}\comment{}
}
\end{codeenv}
The \idx{handler} macro displays the name of the associated
\atentry{bibtexentry} term and the citation:
\begin{codeenv}
\cs{newcommand}\marg{\csfmt{contributorhandler}}[1]\marg{\cs{par}\ics{glsentryname}\marg{\idx{param}1} \cs{cite}\marg{\idx{param}1}}
\end{codeenv}

The complete document code is listed below. The document build is:
\begin{verbatim}
pdflatex sample-citations
bib2gls --cite-as-record sample-citations
bibtex sample-citations
pdflatex sample-citations
pdflatex sample-citations
\end{verbatim}
The resulting document is shown in \figureref{fig:sample-citations.pdf}.

\lstinputlisting[firstline=7]{../examples/sample-citations.tex}

\begin{figure}
\figcontents
{\pdftwocol{../examples/sample-citations.pdf}{2}}
{\caption{\filefmt{sample-citations.pdf}}}
{fig:sample-citations.pdf}
\end{figure}

\filesection{sample-msymbols.tex}

This example uses \exfile{bigmathsymbols.bib},
\exfile{mathsrelations.bib}, \exfile{binaryoperators.bib},
\exfile{unaryoperators.bib} and \exfile{mathgreek.bib}. The \isty{stix}
package is required for some of the commands used in
\exfile{bigmathsymbols.bib}, so that must be loaded in the document.

I'm using the \glostyle{mcolalttree} style for this document,
which means that the \sty{glossary-mcols} package is required and
the styles need patching, which can be done with 
the \styopt{stylemods} package option:
\begin{codeenv}
\cs{usepackage}\oarg{\styopt{record},\comment{using bib2gls}
 \styopt{nostyles},\comment{don't load default styles}
 \styopt{postdot},\comment{append a dot after descriptions}
 \styopt[mcols]{stylemods},\comment{load glossary-mcols.sty and patch}
 \styopt[mcolalttree]{style}}\marg{glossaries-extra}
\end{codeenv}

I'm not referencing any of the entries in the document as I'm just
generating a complete list of all the defined symbols. This means I
need to tell \bibgls\ to select all entries and don't bother saving
the \field{location} field:
\begin{codeenv}
\csopt[false]{save-locations},
\csopt[all]{selection}
\end{codeenv}
Since I'm using a style that's based on \glostyle{alttree} I need to
find the widest \field{name}, which can be done with \csopt{set-widest}.

The simplest way of dividing the glossary into logical blocks is to
sort according to the category, but first I need to use \csopt{field-aliases} 
to convert the custom \fieldfmt{identifier} field to \field{category}:
\begin{codeenv}
\csopt[\fieldfmt{identifier}=\field{category}]{field-aliases}
\end{codeenv}
and sort by the \field{category} field:
\begin{codeenv}
\csopt[\field{category}]{sort-field}
\end{codeenv}
Since this will cause identical sort values, I need to provide a
a way of ordering these identical values. Here I've decided 
to fallback on the \field{description} field:
\begin{codeenv}
\csopt[\field{description}]{identical-sort-action}
\end{codeenv}
This means that entries will be ordered by \field{category} and 
then \field{description}, which naturally creates blocks of symbol
types in the glossary. This only uses a simple case-sensitive string
comparison which is fine for English, but for another language it
would be better to use \csopt{sort-suffix} as in the
\exfile{sample-textsymbols.tex} file.

Remember that I want a small vertical gap between each logical
block. These need the \field{group} field which, with the default
locale sort, is obtained from the first letter of the sort value.
In this case the sort value is obtained from the \field{category}
field, and as each category happens to start with a different
letter, this means I get the desired effect. However, in the event
that I add more entries with a new category that happens to start
with the same letter as an existing category, it's better to provide
a more future-proof method, so I've set the \field{group} field to 
fetch its value from the \field{category} field:
\begin{codeenv}
\csopt[\field{category}=\field{group}]{replicate-fields}
\end{codeenv}
(Since the \csopt{field-aliases} option is always performed before
\csopt{replicate-fields}, the \field{category} field will already
have been set and is available for replicating.)

This means that I'm essentially sorting by the \field{group} labels,
which this manual has warned against doing. In this case, it's an
acceptable break from that rule as I've used options that ensure the
groups aren't broken up during sorting and I'm not concerned with the group
titles. A method such as that used in \exfile{sample-textsymbols2.tex}
would end up with titled blocks, which I don't want here. By using
resource options such as \csopt{field-aliases} and
\csopt{replicate-fields} I can avoid the warning that's triggered
with the default \longarg{warn-non-bib-fields}.

The complete document code is listed below. The document build is:
\begin{verbatim}
pdflatex sample-msymbols
bib2gls sample-msymbols
pdflatex sample-msymbols
\end{verbatim}
The resulting document is shown in \figureref{fig:sample-msymbols.pdf}.

\lstinputlisting[firstline=5]{../examples/sample-msymbols.tex}

\begin{figure}
\figcontents
{%
 \frame{\includegraphics[height=.9\textheight]{../examples/sample-msymbols.pdf}}%
}
{\caption{\filefmt{sample-msymbols.pdf}}}
{fig:sample-msymbols.pdf}
\end{figure}

\filesection{sample-maths.tex}

This example uses \exfile{bigmathsymbols.bib} and \exfile{mathsobjects.bib}.
It has a fairly similar preamble to \exfile{sample-msymbols.tex},
but \exfile{no-interpret-preamble.bib} and
\exfile{interpret-preamble.bib} are now needed to provide
the \gls!{sortart} command:
\begin{codeenv}
\gls{GlsXtrLoadResources}\oarg{
  \csopt[no-interpret-preamble]{src},
  \csopt[false]{interpret-preamble}
}
\end{codeenv}
There's also an extra custom field to alias:
\begin{codeenv}
\csopt[\fieldfmt{identifier}=\field{category},\fieldfmt{format}=\field{user1}]{field-aliases}
\end{codeenv}
I've aliased \fieldfmt{format} to \field{user1} since \gls!{glsxtrfmt}
defaults to that field. If I decided to use a different field I also
need to remember to redefine \ics{GlsXtrFmtField} to match.

As with \exfile{sample-msymbols.tex} I'm sorting by the
\field{category} label and this value is copied to the \field{group}
field, but again I don't have a \gls{hierarchicalglossary} as the logical
blocks don't have titles.

In this document I only want to select entries that have been
indexed, so I've omitted the \csopt{selection} option I used in the
\exfile{sample-msymbols.tex} example, however
I still don't want any number lists so I still have
\csopt[false]{save-locations}.

I want \gls!{glsxtrfmt} to index the term (which it doesn't by default) so
that means I need to redefine \ics{GlsXtrFmtDefaultOptions} 
to prevent it from using \code{noindex}:
\begin{codeenv}
\cs{renewcommand}\marg{\cs{GlsXtrFmtDefaultOptions}}\marg{}
\end{codeenv}
I've provided some convenient wrapper commands that use
\gls!{glsxtrfmt*} or the non-linking \gls!{glsxtrentryfmt} that
are in the form:
\begin{codeenv}
\cs{newcommand}\marg{\cmd{set}}[2][]\marg{\gls{glsxtrfmt*}\oarg{\idx{param}1}\marg{set}\marg{\idx{param}2}}
\cs{newcommand}\marg{\cmd{nlset}}[1]\marg{\gls{glsxtrentryfmt}\marg{set}\marg{\idx{param}1}}
\end{codeenv}
The use of the starred form allows:
\begin{codeenv}
\cmd{[}\cmd{set}\marg{A} = \cs{gls}\marg{bigcup}[\idx{sbchar}\marg{i=1}\idx{spchar}n] \cmd{set}\marg{B}[\idx{sbchar}i] \cmd{]}
\end{codeenv}
which produces:
\[ \mathcal{A} = \bigcup_{i=1}^n \mathcal{B}_i \]
Note the difference if the optional arguments aren't used:
\begin{codeenv}
\cmd{[}\cmd{set}\marg{A} = \cs{gls}\marg{bigcup}\idx{sbchar}\marg{i=1}\idx{spchar}n \cmd{set}\marg{B}\idx{sbchar}i \cmd{]}
\end{codeenv}
This produces:
\[ \mathcal{A} = \bigcup{}_{i=1}^n \mathcal{B}{}_i \]

Be careful with the set cardinality example. You might be tempted to
nest \csfmt{set} within the argument of \csfmt{setcard} but this
results in nested hyperlinks. These are unpredictable and there's no
consistent handling of them between different PDF viewers. It can
also be confusing to the reader. If
$\lvert\mathcal{B}_1\cup\mathcal{B}_2\rvert$
shows up as what appears to be a single hyperlink, where would the 
reader expect the target? This is the reason for providing the
non-linking commands like \csfmt{nlset} and \csfmt{nlsetcard}.

The complete document code is listed below. The document build is:
\begin{verbatim}
pdflatex sample-maths
bib2gls sample-maths
pdflatex sample-maths
\end{verbatim}
The resulting document is shown in \figureref{fig:sample-maths.pdf}.

\lstinputlisting[firstline=5]{../examples/sample-maths.tex}

\begin{figure}
\figcontents
{%
 \frame{\includegraphics[height=.9\textheight]{../examples/sample-maths.pdf}}%
}
{\caption{\filefmt{sample-maths.pdf}}}
{fig:sample-maths.pdf}
\end{figure}

\filesection{sample-textsymbols.tex}

This example uses \exfile{miscsymbols.bib}. 
This requires both \isty{marvosym} and (with the \styoptfmt{weather} option)
\isty{ifsym}. Unfortunately both define the commands \csfmt{Sun}
and \csfmt{Lightning}, so these commands need to be undefined after
the first package is loaded and before the second.
Since I want the definitions provide by \sty{ifsym} I have to first
load \sty{marvosym}, then undefine the conflicting commands and then
load \sty{ifsym}:
\begin{codeenv}
\cs{usepackage}\marg{etoolbox}
\cs{usepackage}\marg{marvosym}
\cs{undef}\cmd{Sun}
\cs{undef}\cmd{Lightning}
\cs{usepackage}\oarg{\styoptfmt{weather}}\marg{ifsym}
\end{codeenv}
The \sty{etoolbox} package is also loaded as it provides
\ics{undef}. (An alternative is to modify the
\exfile{miscsymbols.bib} file so that it uses \sty{ifsym}'s more
generic \cs{textweathersymbol} command and omit the \styoptfmt{weather}
option when loading the package, but the method used here
demonstrates how to deal with such conflicts.)

The custom \gls{entrytype} \atentryfmt{icon} must be aliased for the
entries to be recognised:
\begin{codeenv}
\csopt[icon=symbol]{entry-type-aliases}
\end{codeenv}

Since none of the entries have a \field{name} or \field{description}
field, the custom fields \fieldfmt{icon} and
\fieldfmt{icondescription} need to be aliased to them.
The document uses the \glostyle{alttreegroup} style where the groups
are obtained from the \field{category}, which again I obtain from
the custom \fieldfmt{identifier} field using:
\begin{codeenv}
\csopt[
 \fieldfmt{identifier}=\field{category},
 \fieldfmt{icon}=\field{name},
 \fieldfmt{icondescription}=\field{description}]{field-aliases},
\csopt[\field{category}=\field{group}]{replicate-fields}
\end{codeenv}
The \field{group} field is just a label and an appropriate title needs to be
supplied for each group label:
\begin{codeenv}
\cs{glsxtrsetgrouptitle}\marg{information}\marg{Information}
\cs{glsxtrsetgrouptitle}\marg{mediacontrol}\marg{Media Controls}
\cs{glsxtrsetgrouptitle}\marg{weather}\marg{Weather Symbols}
\end{codeenv}
This also requires sorting first by \field{category} and then
fallback on another field. The most appropriate here is the
\field{description} field, but instead of using
\csopt{identical-sort-action}, I'm using \csopt{sort-suffix},
which works better with the default locale sort when the fallback
field consists of words or phrases.
\begin{codeenv}
\csopt[\field{category}]{sort-field},
\csopt[\field{description}]{sort-suffix},
\csopt[|]{sort-suffix-marker}
\end{codeenv}
Since I'm using one of the \glostyle{alttree} styles, I need to set
the widest name:
\begin{codeenv}
\csopt{set-widest}
\end{codeenv}
In this case, \bibgls\ won't be able to determine the widest name
since it doesn't recognise any of the commands, so it will have to
use the fallback command, which will use one of the commands
provided by the \sty{glossaries-extra-stylemods} package.

This is actually not the best method as \bibgls\ can't see the group
titles as they're in the document, so it's only able to sort by the
label. While this might work for English, it can become a problem
for other languages that use extended Latin or non-Latin characters
in their alphabet. A much better method is to treat this as a
\gls{hierarchicalglossary} with topic titles as the \glspl{top-levelentry}.
This is covered in the next example \exfile{sample-textsymbols2.tex}.

The complete document code is listed below. The document build is:
\begin{verbatim}
pdflatex sample-textsymbols
bib2gls sample-textsymbols
pdflatex sample-textsymbols
\end{verbatim}
The resulting document is shown in \figureref{fig:sample-textsymbols.pdf}.

\lstinputlisting[firstline=6]{../examples/sample-textsymbols.tex}

\begin{figure}
\figcontents
{%
 \frame{\includegraphics[height=.9\textheight]{../examples/sample-textsymbols.pdf}}%
}
{\caption{\filefmt{sample-textsymbols.pdf}}}
{fig:sample-textsymbols.pdf}
\end{figure}

\filesection{sample-textsymbols2.tex}

This example is a better approach than the 
\exfile{sample-textsymbols.tex} example above. As with the previous
example, this requires both \isty{marvosym} and \isty{ifsym} so the
same patch is applied to avoid conflict.

As before, the custom \gls{entrytype} \atentryfmt{icon} must be aliased for the
entries to be recognised:
\begin{codeenv}
\csopt[icon=symbol]{entry-type-aliases}
\end{codeenv}

The \exfile{topics.bib} file contains terms with
labels that match the custom \fieldfmt{identifier} fields used in
the \exfile{miscsymbols.bib} file. So both files are loaded and the
\fieldfmt{identifier} field is now aliased to \field{parent}. These
\glspl{parententry} represent the topics and unlike the previous example
it's now possible to sort by the topic title (obtained from the
\field{name} field) instead of by the label.
\begin{codeenv}
\csopt[topics,miscsymbols]{src},
\csopt[
 \fieldfmt{identifier}=\field{parent},
 \fieldfmt{icon}=\field{name},
 \fieldfmt{icondescription}=\field{description}]{field-aliases},
\end{codeenv}
There's no \csopt{sort-field} option in this example. The default
\field{sort} field is used. Since it's not set for any of the
entries, the fallback value will be used. In the case of the topic
titles (\atentry{index} and \atentry{indexplural}), I want to sort
by the \field{name}, which is the default fallback if the
\field{sort} field is missing for the index \glspl{entrytype}.

The default fallback for the \field{sort} field for \atentry{symbol}
entries is the label, but in this case I want to use the
\field{description} field:
\begin{codeenv}
  \csopt[description]{symbol-sort-fallback}
\end{codeenv}
The best styles for this kind of glossary are the topic styles
provided by \isty{glossary-topic}. This package was only added to
\sty{glossaries-extra} v1.40, so you need to make sure you have at
least that version installed.

In this case I've decided to use the \glostyle{topic} style. I can
use it with or without the \csopt{set-widest} option. 
As with the previous example, \bibgls\ won't be able to determine
the widest name since it doesn't recognise any of the commands
contained in the \field{name} fields, so it will have to use the
fallback method, which will use one of the commands provided by the
\sty{glossaries-extra-stylemods} package. The \code{tree} option is
needed to enable the appropriate commands:
\begin{codeenv}
\cmd{usepackage}\oarg{\styopt{record},
 \styopt{nostyles},
 \styopt{postdot},
 \styopt[tree,topic]{stylemods},
 \styopt[topic]{style}}\marg{glossaries-extra}
\end{codeenv}

The complete document code is listed below. The document build is:
\begin{verbatim}
pdflatex sample-textsymbols2
bib2gls --group sample-textsymbols2
pdflatex sample-textsymbols2
\end{verbatim}
The resulting document is shown in \figureref{fig:sample-textsymbols2.pdf}.

\lstinputlisting[firstline=5]{../examples/sample-textsymbols2.tex}

\begin{figure}
\figcontents
{%
 \frame{\includegraphics[height=.9\textheight]{../examples/sample-textsymbols2.pdf}}%
}
{\caption{\filefmt{sample-textsymbols2.pdf}}}
{fig:sample-textsymbols2.pdf}
\end{figure}

\filesection{sample-markuplanguages.tex}

This example uses \exfile{markuplanguages.bib}. Since the file
includes abbreviations, any commands that must be used before
abbreviations are defined need to go before
\gls!{GlsXtrLoadResources}. This includes the abbreviation style, 
which I've set to \abbrstyle{long-short-desc}:
\begin{codeenv}
\cs{setabbreviationstyle}\oarg{markuplanguage}\marg{\abbrstyle{long-short-desc}}
\end{codeenv}
This style sets the \field{name} field using
\ics{glsxtrlongshortdescname}, which defaults to the long form
followed by the short form in parentheses. I decided to switch this
round so that the short form is shown first, which conveniently
matches the default \csopt{abbreviation-sort-fallback}.
\begin{codeenv}
\cs{renewcommand}*\marg{\cs{glsxtrlongshortdescname}}\marg{\comment{}
  \cs{protect}\ics{glsabbrvfont}\marg{\ics{the}\ics{glsshorttok}}\cs{space}
  \ics{glsxtrparen}\marg{\ics{glslongfont}\marg{\cs{the}\ics{glslongtok}}}\comment{}
}
\end{codeenv}
(The long form is still shown before the short form on the first use
of \cs{gls} in the document. The switch in the above code only
affects how the term is displayed in the glossary.)

This redefinition must be done before the abbreviations are defined
as it's expanded when the \field{name} field is set. (Note the need
to protect commands that shouldn't be expanded.) If I decide not to change
the \field{name} format in this way, I would then need to use
\csopt[long]{abbreviation-sort-fallback}.

I also decided to make use of the custom command \csfmt{abbrvtag}
that marks up the letters in the \field{long} field used to obtain
the abbreviation. As with the abbreviation style, this must be
done before the abbreviations are defined:
\begin{codeenv}
\cs{GlsXtrEnableInitialTagging}\marg{markuplanguage}\marg{\cmd{abbrvtag}}
\end{codeenv}
If you accidentally place it after \gls{GlsXtrLoadResources}, you'll
encounter an error on the second \LaTeX\ run (but not the first).
This is because \ics{GlsXtrEnableInitialTagging} requires that the
supplied command (\csfmt{abbrvtag} in this case) be undefined. On
the first \LaTeX\ it's undefined, but on the second it picks up
the \atentry{preamble} definition, which is now in the resource file.

The tagging format is governed by \ics{glsxtrtagfont} which
underlines its argument by default. I've redefined it to also
convert the letter to \idx{uppercase}:
\begin{codeenv}
\cs{renewcommand}*\marg{\cs{glsxtrtagfont}}[1]\marg{\ics{underline}\marg{\cs{glsuppercase}\marg{\idx{param}1}}}
\end{codeenv}
Note that in the \code{mathml} case, the first tag consists of more
than one letter:
\begin{codeenv}
\field{long}=\marg{\cmd{abbrvtag}\marg{m\cs{NoCaseChange}\marg{ath}}ematical }\idx{stringconcat}markuplang
\end{codeenv}
Here \ics{NoCaseChange} prevents \ics{glsuppercase}
from applying the case change.

The default \csopt{selection} criteria includes entries that have
been indexed and any cross-references. Some of the
\field{description} fields include \ics{glsxtrshort}, which \bibgls\ picks
up and the referenced entry is included in the dependency list.
However, I don't want any indexing
performed by commands occurring in the glossary. This can be dealt
with in one of two ways: either switch the format to \encap{glsignore}
or suppress the indexing by changing the default options with
\ics{GlsXtrSetDefaultGlsOpts}. In this case I decided to turn the
records into ignored records:
\begin{codeenv}
\ics{GlsXtrSetDefaultNumberFormat}\marg{\encap{glsignore}}
\end{codeenv}
This means that some of the entries won't have \glspl{locationlist}, so
I've defined a \idx{postdescriptionhook} that inserts a \idx{full-stop} after the
\field{description} if there's no \gls{location} otherwise it inserts a
comma:
\begin{codeenv}
\cs{newcommand}\marg{\postdeschook{markuplanguage}}\marg{\comment{}
  \cs{glsxtrifhasfield}\marg{\field{location}}\marg{\cs{glscurrententrylabel}}\comment{}
  \marg{,}\comment{}
  \marg{.}\comment{}
}
\end{codeenv}
I've used \csopt{loc-suffix} to append a \idx{full-stop} after the
\glspl{locationlist}. This doesn't affect the entries that haven't been
indexed.

I decided to convert the first letter of the \field{name} field to
\idx{uppercase}. Since the \field{name} is implicitly set for
abbreviations based on the style, I've decided to implement this
through the \catattr{glossname} attribute rather than using
\csopt{name-case-change}:
\begin{codeenv}
\cs{glssetcategoryattribute}\marg{markuplanguage}\marg{\catattr{glossname}}\marg{firstuc}
\end{codeenv}
If this line causes an error when the glossary is displayed that goes 
away if it's commented out, make sure you have at least version 2.06
of \sty{mfirstuc}. For most of the entries, this doesn't make a
difference as they already start with a capital. It's only the 
\code{markdown} entry that's actually affected.

The description case change is dealt with by \bibgls\ instead:
\begin{codeenv}
\csopt[firstuc]{description-case-change}
\end{codeenv}
This works better than the \catattr{glossdesc} attribute as \bibgls\ can
convert commands like \ics{glstext} into \ics{Glstext} which
\ics{makefirstuc} can't do. (Although in this particular example,
there's no difference as both instances of \cs{glstext} already
produce \idx{uppercase} text.)

The complete document code is listed below. The document build is:
\begin{verbatim}
pdflatex sample-markuplanguages
bib2gls --group sample-markuplanguages
pdflatex sample-markuplanguages
\end{verbatim}
The resulting document is shown in \figureref{fig:sample-markuplanguages.pdf}.

\lstinputlisting[firstline=5]{../examples/sample-markuplanguages.tex}

\begin{figure}
\figcontents
{%
 \frame{\includegraphics[height=.9\textheight]{../examples/sample-markuplanguages.pdf}}%
}
{\caption{\filefmt{sample-markuplanguages.pdf}}}
{fig:sample-markuplanguages.pdf}
\end{figure}


\filesection{sample-usergroups.tex}

This example uses \exfile{usergroups.bib}. This requires \XeLaTeX\
or \LuaLaTeX\ as the \ext{bib} file includes \idx{non-ASCII} labels.
The entries include fields in different languages, the main one
being English. If an entry has a non-English \field{name} or
\field{long} field, it also includes the custom field
\fieldfmt{translation} that provides an (approximate) translation.
If this field is present, the language is given by the first element
of the custom \fieldfmt{language} field.

In this case, I'm providing keys for the custom
\fieldfmt{language} and \fieldfmt{translation} fields, and, for a bit
of variety from the other examples, I'm ignoring the custom
\fieldfmt{identifier} field. The custom keys are provided with
\ics{glsaddstoragekey}:
\begin{codeenv}
\cs{glsaddstoragekey}\marg{language}\marg{}\marg{\cmd{glsentrylanguage}}
\cs{glsaddstoragekey}\marg{translation}\marg{}\marg{\cmd{glsentrytranslation}}
\end{codeenv}
The \ext{bib} file includes abbreviations. Remember that the
abbreviation style must be set before the resource file is loaded:
\begin{codeenv}
\cs{setabbreviationstyle}\oarg{tug}\marg{\abbrstyle{long-short-user}}
\end{codeenv}
For this example, I'm explicitly setting the \field{category} field
to \code{tug}:
\begin{codeenv}
\csopt[tug]{category}
\end{codeenv}
Some of the fields end with a \idx{full-stop}. This isn't a problem with
the \field{long} field as the first use follows the long form with
the short form in parentheses, but it will be a problem on
subsequent use if the \field{short} field ends with a \idx{full-stop}.
This means I need to check for end-of-sentence punctuation
for the \field{short} field. It's also a good idea to do this for
the \field{name} field for the non-abbreviations.
\begin{codeenv}
\csopt[\field{name},\field{short}]{check-end-punctuation}
\end{codeenv}
It's now possible to discard a \idx{full-stop} that follows
\ics{gls}:
\begin{codeenv}
\cs{renewcommand}*\marg{\cs{glsxtrifcustomdiscardperiod}}[2]\marg{\comment{}
 \ics{ifglshasshort}\marg{\cs{glslabel}}\comment{}
 \marg{\comment{}
   \cs{glsxtrifwasfirstuse}\marg{}\comment{}
   \marg{\comment{}
     \cs{GlsXtrIfFieldUndef}\marg{\fielddisp{fieldendpunc}{shortendpunc}}\marg{\cs{glslabel}}\marg{\idx{param}2}\marg{\idx{param}1}\comment{}
   }\comment{}
 }\comment{}
 \marg{\comment{}
   \cs{GlsXtrIfFieldUndef}\marg{\fielddisp{fieldendpunc}{nameendpunc}}\marg{\cs{glslabel}}\marg{\idx{param}2}\marg{\idx{param}1}\comment{}
 }\comment{}
}
\end{codeenv}
This first tests if the entry that's just been referenced has a
\field{short} field. If it has, then the next test is to check if
that was the first use for that entry. If it was, nothing is done.
If it wasn't, then \cs{GlsXtrIfFieldUndef} is used to determine
if \fielddisp{fieldendpunc}{shortendpunc} has been set. If it has
been set then the period discard function is performed. If the
entry doesn't have a \field{short} field, then the
\fielddisp{fieldendpunc}{nameendpunc} field needs checking instead.

Since the document requires \XeLaTeX\ or \LuaLaTeX\ and has some
\idx{non-ASCII} characters, it needs \sty{fontspec} and an
appropriate font. In this case I've chosen \qt{Linux Libertine O}.
If you don't have it installed, you'll need to change it.
\begin{verbatim}
\usepackage{fontspec}
\setmainfont{Linux Libertine O}
\end{verbatim}
Since it's a multilingual document I also need \isty{polyglossia}
with the main language set to \code{english}:
\begin{codeenv}
\cmd{usepackage}\marg{polyglossia}
\ics{setmainlanguage}\oarg{variant=uk}\marg{english}
\end{codeenv}
Now comes the difficult bit. The document needs to determine what
other languages need to be loaded. The \sty{tracklang} package
provides a convenient interface when dealing with language tags.
This is automatically loaded by \styfmt{glossaries} but I've loaded
it here explicitly as a reminder:
\begin{verbatim}
\usepackage{tracklang}
\end{verbatim}
Once the \hyperref[sec:resourcesets]{resource file} has been loaded, I need to
iterate over all the defined entries and check if the \fieldfmt{translation}
field has been set. If it has, then the first language tag in the
\fieldfmt{language} field will supply the language, but this needs to be
converted from the \idx{IETF} language tag to a language name recognised by
\isty{polyglossia}.

Iterating over all entries can be done with \ics{forglsentries}
but remember that no entries will be defined before \bibgls\ has
been run, so this does nothing on the first \LaTeX\ run.
\begin{codeenv}
\cs{forglsentries}\marg{\cmd{thislabel}}\marg{\comment{}
  \cs{glsxtrifhasfield}\marg{translation}\marg{\cmd{thislabel}}\comment{}
  \marg{\comment{}
    \comment{requires glossaries-extra v1.24}
    \cs{glsxtrforcsvfield}\marg{\cmd{thislabel}}\marg{language}\marg{\cmd{addfirstlang}}\comment{}
  }\comment{}
  \marg{}\comment{}
}
\end{codeenv}
Within the outer (\cs{forglsentries}) \idx{loop}, there's a check
for the \fieldfmt{translation} field using \ics{glsxtrifhasfield}.
If it's present, then the first element of the \fieldfmt{language}
field is required. The simplest way to get this is to use
\ics{glsxtrforcsvfield} which iterates over all elements of the
given field (\fieldfmt{language} in this case) and break out of the
\idx{loop} (with \ics{glsxtrendfor}) once the language has been found.

The \idx{handler} function (\csfmt{addfirstlang}) is defined so that it
adds the given language tag as a tracked language using
\ics{TrackLocale}. This command sets
\ics{TrackLangLastTrackedDialect} to the associated
(\sty{tracklang}) dialect label for convenience. This dialect label
can then be converted to the root language label using
\ics{TrackedLanguageFromDialect}. If this language is supported
by \isty{polyglossia}, then there should be a file called
\metafilefmt{gloss-}{language}{.ldf}.

Some of the entries use the same language, so it's necessary to
check if the language has already been defined before loading it.
There's also a problem in that the language file should not be
loaded in a scoped context, but both \ics{glsxtrforcsvfield}
and the unstarred \ics{glsxtrifhasfield} add implicit grouping.
To solve both problems, an internal \sty{etoolbox} list is 
defined:
\begin{verbatim}
\newcommand{\langlist}{}%
\end{verbatim}
and \ics{xifinlist} is used to first check if the language label is
already in the list before adding it. Since this part of the code is
scoped, the global \ics{listxadd} is used to add the language label
to the list.

Next the \field{useri} field is set to 
\encapdisp{textlanguage}{text\meta{language}}
which is the name of the control sequence used with
\sty{polyglossia} to switch language for a short block of text. This means that
\gls{glsxtrentryfmt}\margm{text} can be used to format \meta{text}
in the relevant language. Finally, \ics{glsxtrendfor} is used to
break out of the \idx{loop}.
\begin{codeenv}
\cs{newcommand}*\marg{\cmd{addfirstlang}}[1]\marg{\comment{}
  \cs{TrackLocale}\marg{\idx{param}1}\comment{}
  \ics{edef}\cmd{thislanguage}\marg{\comment{}
    \cs{TrackedLanguageFromDialect}\cs{TrackLangLastTrackedDialect}}\comment{}
  \ics{IfFileExists}\marg{gloss-\cmd{thislanguage}.ldf}\comment{}
  \marg{\comment{}
    \ics{xifinlist}\marg{\cmd{thislanguage}}\marg{\cmd{langlist}}\marg{}\comment{}
    \marg{\ics{listxadd}\marg{\cmd{langlist}}\marg{\cmd{thislanguage}}}\comment{}
    \ics{xGlsXtrSetField}\marg{\cmd{thislabel}}\marg{\field{useri}}\marg{\encapdisp{textlanguage}{text\cmd{thislanguage}}}\comment{}
    \ics{glsxtrendfor}
  }\comment{}
  \marg{}\comment{}
}
\end{codeenv}
Once the \ics{forglsentries} \idx{loop} has found the appropriate
languages, it's now necessary to iterate over the internal list
\csfmt{langlist} and set the language:
\begin{codeenv}
\ics{forlistloop}\marg{\ics{setotherlanguage}}\marg{\cmd{langlist}}
\end{codeenv}
The \abbrstyle{long-short-user} style now needs to be adjusted
to ensure that it picks up the appropriate language change.
By default this style checks the \field{useri} field, so this needs
to be changed to \fieldfmt{translation} by redefining
\ics{glsxtruserfield}:
\begin{codeenv}
\cs{renewcommand}*\marg{\cs{glsxtruserfield}}\marg{translation}
\end{codeenv}
The command that governs the format of the parenthetical material
(\ics{glsxtruserparen}) also needs adjusting. I've changed the
space before the parenthesis to \ics{cs.space} because some of the
long fields end with a \idx{full-stop} and this corrects the
spacing. The \fieldfmt{translation} field is in English, 
so this needs to be encapsulated with \icsdisp{textlanguage}{textenglish}
in case the surrounding text is in a different language.
\begin{codeenv}
\cs{renewcommand}*\marg{\cs{glsxtruserparen}}[2]\marg{\comment{}
  \cs{cs.space}
  \cs{glsxtrparen}\marg{\idx{param}1\comment{}
  \cs{ifglshasfield}\marg{\cs{glsxtruserfield}}\marg{\idx{param}2}\marg{,
   \csdisp{textlanguage}{textenglish}\marg{\cs{glscurrentfieldvalue}}}\marg{}}\comment{}
}
\end{codeenv}
Next I've defined a convenient command for use in the \catattr{textformat}
attributes for the custom \code{tug} category:
\begin{codeenv}
\cs{newcommand}*\marg{\cmd{tugtextformat}}[1]\marg{\comment{}
  \cs{glsxtrentryfmt}\marg{\cs{glslabel}}\marg{\idx{param}1}\comment{}
}
\end{codeenv}
This uses \gls{glsxtrentryfmt} to encapsulate the given text
in the appropriate language command (if provided). When this is set
as the \catattr{textformat} attribute, it will be used instead of
\ics{glstextformat}, which means that the entry label can be
referenced with \cs{glslabel}.

There's a similar command for use in the \catattr{glossnamefont}
attribute. This is used in the glossary, so the label is referenced
with \cs{glscurrententrylabel}:
\begin{codeenv}
\cs{newcommand}*\marg{\cmd{tugnameformat}}[1]\marg{\comment{}
  \cs{glsxtrentryfmt}\marg{\cs{glscurrententrylabel}}\marg{\idx{param}1}\comment{}
}
\end{codeenv}
The attributes can now be set to the relevant control sequence name:
\begin{codeenv}
\cs{glssetcategoryattribute}\marg{tug}\marg{\catattr{textformat}}\marg{tugtextformat}
\cs{glssetcategoryattribute}\marg{tug}\marg{\catattr{glossnamefont}}\marg{tugnameformat}
\end{codeenv}

The document uses the \glostyle{bookindex} style, which is set in
the package options:
\begin{codeenv}
\cmd{usepackage}\oarg{\styopt{record},
 \styopt{nostyles},
 \styopt[bookindex]{stylemods},
 \styopt[bookindex]{style}
}\marg{glossaries-extra}
\end{codeenv}
The \glostyle{bookindex} style ignores the \field{description} field, so I've 
provided a \idx{postnamehook} to append it in parentheses (with the
translation, if provided):
\begin{codeenv}
\cs{newcommand}\marg{\postnamehook{tug}}\marg{\comment{}
  \ics{ifglshasdesc}\marg{\cs{glscurrententrylabel}}\comment{}
  \marg{\cs{cs.space}(\cs{glossentrydesc}\marg{\cs{glscurrententrylabel}}\comment{}
     \cs{glsxtrifhasfield}\marg{translation}\marg{\cs{glscurrententrylabel}}\comment{}
     \marg{, \csdisp{textlanguage}{textenglish}\marg{\cs{glscurrentfieldvalue}}}\comment{}
     \marg{}\comment{}
  )}\comment{}
  \marg{\comment{}
   \cs{glsxtrifhasfield}\marg{translation}\marg{\cs{glscurrententrylabel}}\comment{}
   \marg{\cs{cs.space}(\csdisp{textlanguage}{textenglish}\marg{\cs{glscurrentfieldvalue}})}\comment{}
   \marg{}\comment{}
  }\comment{}
}
\end{codeenv}
Remember that this hook is included within the \field{name} font
(provided by the \catattr{glossnamefont} attribute in this case)
so \csdisp{textlanguage}{textenglish} is again used to switch the language to
English for the translation.

The complete document code is listed below. The document build is:
\begin{verbatim}
xelatex sample-usergroups
bib2gls --group sample-usergroups
xelatex sample-usergroups
xelatex sample-usergroups
\end{verbatim}
The two pages of the document are shown in
\figureref{fig:sample-usergroups.pdf}. Since the entries have all
been referenced on page~1, the \glspl{locationlist} are all simply~\qt{1}.

%\lstinputlisting[firstline=5]{../examples/sample-usergroups.tex}
\begin{lstlisting}[escapechar=|]
\documentclass{scrreprt}

\usepackage{fontspec}
\setmainfont{Linux Libertine O}

\usepackage{polyglossia}
\setmainlanguage[variant=uk]{english}
\usepackage{tracklang}
\usepackage{etoolbox}

\usepackage[record,% use bib2gls
 nostyles,% don't load default styles
 stylemods={bookindex},
 style={bookindex}
]{glossaries-extra}

\glsaddstoragekey{language}{}{\glsentrylanguage}
\glsaddstoragekey{translation}{}{\glsentrytranslation}

\setabbreviationstyle[tug]{long-short-user}

\GlsXtrLoadResources[
  src={usergroups}, % data in usergroups.bib
  check-end-punctuation={name,short},
  category=tug
]

\renewcommand*{\glsxtrifcustomdiscardperiod}[2]{%
 \ifglshasshort{\glslabel}%
 {%
   \glsxtrifwasfirstuse{}%
   {%
     \GlsXtrIfFieldUndef{shortendpunc}{\glslabel}{#2}{#1}%
   }%
 }%
 {%
   \GlsXtrIfFieldUndef{nameendpunc}{\glslabel}{#2}{#1}%
 }%
}

\newcommand{\langlist}{}%

\newcommand*{\addfirstlang}[1]{%
  \TrackLocale{#1}%
  \edef\thislanguage{%
    \TrackedLanguageFromDialect\TrackLangLastTrackedDialect}%
  \IfFileExists{gloss-\thislanguage.ldf}%
  {%
    \xifinlist{\thislanguage}{\langlist}{}%
    {\listxadd{\langlist}{\thislanguage}}%
    \xGlsXtrSetField{\thislabel}{useri}{text\thislanguage}%
    \glsxtrendfor
  }%
  {}%
}

\forglsentries{\thislabel}{%
  \glsxtrifhasfield{translation}{\thislabel}%
  {%
    % requires glossaries-extra v1.24
    \glsxtrforcsvfield{\thislabel}{language}{\addfirstlang}%
  }%
  {}%
}

\forlistloop{\setotherlanguage}{\langlist}

\renewcommand*{\glsxtruserfield}{translation}

\renewcommand*{\glsxtruserparen}[2]{%
  \  
  \glsxtrparen{#1%
  \ifglshasfield{\glsxtruserfield}{#2}{,
   \textenglish{\glscurrentfieldvalue}}{}}%
}

\newcommand*{\tugtextformat}[1]{%
  \glsxtrentryfmt{\glslabel}{#1}%
}

\newcommand*{\tugnameformat}[1]{%
  \glsxtrentryfmt{\glscurrententrylabel}{#1}%
}

\glssetcategoryattribute{tug}{textformat}{tugtextformat}
\glssetcategoryattribute{tug}{glossnamefont}{tugnameformat}

\newcommand{\glsxtrpostnametug}{%
  \ifglshasdesc{\glscurrententrylabel}%
  {\ (\glossentrydesc{\glscurrententrylabel}%
     \glsxtrifhasfield{translation}{\glscurrententrylabel}%
     {, \textenglish{\glscurrentfieldvalue}}%
     {}%
  )}%
  {%
   \glsxtrifhasfield{translation}{\glscurrententrylabel}%
   {\ (\textenglish{\glscurrentfieldvalue})}%
   {}%
  }%
}

\begin{document}
\chapter{Sample}
\section{First Use}
\gls{TUG}. \gls{bgTeX}. \gls{latex-br}. \gls{CTeX}.
\gls{CSTUG}. \gls{DANTE}. \gls{DKTUG}. \gls{EUG}.
\gls{CervanTeX}. \gls{TirantloTeX}. \gls{GUTenberg}.
\gls{UKTUG}. \gls{|\greekmono ɛϕτ|}. \gls{MaTeX}. \gls{ITALIC}.
\gls{ÍsTeX}. \gls{GuIT}. \gls{KTS}. \gls{LTVG}.
\gls{mxTeX}. \gls{NTG}. \gls{NTUG}. \gls{GUST}. \gls{GUTpt}.
\gls{VietTUG}. \gls{LUGSA}.

\section{Next Use}

\gls{TUG}. \gls{bgTeX}. \gls{latex-br}. \gls{CTeX}.
\gls{CSTUG}. \gls{DANTE}. \gls{DKTUG}. \gls{EUG}.
\gls{CervanTeX}. \gls{TirantloTeX}. \gls{GUTenberg}.
\gls{UKTUG}. \gls{|\greekmono ɛϕτ|}. \gls{MaTeX}. \gls{ITALIC}.
\gls{ÍsTeX}. \gls{GuIT}. \gls{KTS}. \gls{LTVG}.
\gls{mxTeX}. \gls{NTG}. \gls{NTUG}. \gls{GUST}. \gls{GUTpt}.
\gls{VietTUG}. \gls{LUGSA}.

\printunsrtglossaries
\end{document}
\end{lstlisting}

\begin{figure}
\figcontents
{\pdftwocol{../examples/sample-usergroups.pdf}{2}}
{\caption{\filefmt{sample-usergroups.pdf}}}
{fig:sample-usergroups.pdf}
\end{figure}

\filesection{sample-multi1.tex}

This example uses \exfile{bacteria.bib},
\exfile{markuplanguages.bib}, \exfile{vegetables.bib}, \exfile{minerals.bib},
\exfile{animals.bib}, \exfile{chemicalformula.bib},
\exfile{baseunits.bib} and \exfile{derivedunits.bib}.
Since there's one or more UTF-8 character, the document requires
UTF-8 support:
\begin{verbatim}
\usepackage[T1]{fontenc}
\usepackage[utf8]{inputenc}
\end{verbatim}

The aim of this example document is to have a separate glossary
(without number lists) for each type of data (bacteria, markup
languages, vegetables, minerals, animals, chemical formula, base
units and derived units) and also an
index listing all referenced entries with number lists as well as
aliased entries that haven't explicitly been used but the
cross-reference term as been indexed. This requires:
\begin{codeenv}
\csopt[recorded and deps and see]{selection}
\end{codeenv}
to ensure the aliased entries are selected.

Since I don't need the default \mainglossary\ (I'm providing
my own custom glossaries) I've used the \styopt{nomain} option to
suppress its automatic creation, but I do want the \code{index}
glossary so I've used the \styopt{index} package option. As with the
other examples, I've used \styopt{nostyles} to suppress the creation
of the default styles and used \styopt{stylemods} to load the
particular style packages that I need and use
\sty{glossaries-extra-stylemods} to patch them. The index needs to
be in an unnumbered chapter, which is the default for book-like
styles, but I want the other glossaries in unnumbered sections so
I've used the \styopt{section} option. I just need to remember to
switch this before displaying the index:
\begin{codeenv}
\cmd{usepackage}\oarg{\styopt{record},\comment{use bib2gls}
 \styopt{section},\comment{use \ics{section*} for glossary headings}
 \styopt{postdot},\comment{insert dot after descriptions in glossaries}
 \styopt{nomain},\comment{don't create 'main' glossary}
 \styopt{index},\comment{create 'index' glossary}
 \styopt{nostyles},\comment{don't load default styles}
\comment{load and patch required style packages:}
 \styopt[list,mcols,tree,bookindex]{stylemods}
}\marg{glossaries-extra}
\end{codeenv}
The remaining glossaries need defining:
\begin{codeenv}
\cs{newglossary*}\marg{bacteria}\marg{Bacteria}
\cs{newglossary*}\marg{markuplanguage}\marg{Markup Languages}
\cs{newglossary*}\marg{vegetable}\marg{Vegetables}
\cs{newglossary*}\marg{mineral}\marg{Minerals}
\cs{newglossary*}\marg{animal}\marg{Animals}
\cs{newglossary*}\marg{chemical}\marg{Chemical Formula}
\cs{newglossary*}\marg{baseunit}\marg{SI Units}
\cs{newglossary*}\marg{derivedunit}\marg{Derived Units}
\end{codeenv}
As with \exfile{sample-bacteria.tex} and \exfile{sample-markuplanguages.tex}
I need to set the abbreviation styles before the abbreviations are
defined:
\begin{codeenv}
\cs{setabbreviationstyle}\oarg{bacteria}\marg{\abbrstyle{long-only-short-only}}
\cs{setabbreviationstyle}\oarg{markuplanguage}\marg{\abbrstyle{long-short-desc}}
\end{codeenv}
Unlike the \exfile{sample-markuplanguages.tex} example, I'm not interested
in tagging the initials in this case, but I still want to change the
way the \field{name} field is set with the
\abbrstyle{long-short-desc} abbreviation style:
\begin{codeenv}
\cs{renewcommand}*\marg{\cs{glsxtrlongshortdescname}}\marg{\comment{}
  \cs{protect}\cs{glsabbrvfont}\marg{\cs{the}\cs{glsshorttok}}\cs{space}
  \cs{glsxtrparen}\marg{\cs{glslongfont}\marg{\cs{the}\cs{glslongtok}}}\comment{}
}
\end{codeenv}
Remember that this also needs to be set before the abbreviations are
defined. The \catattr{textformat} and \catattr{glossnamefont}
attributes may be set after definition:
\begin{codeenv}
\cs{newcommand}\marg{\cmd{bacteriafont}}[1]\marg{\cs{emph}\marg{\idx{param}1}}
\cs{glssetcategoryattribute}\marg{bacteria}\marg{\catattr{textformat}}\marg{bacteriafont}
\cs{glssetcategoryattribute}\marg{bacteria}\marg{\catattr{glossnamefont}}\marg{bacteriafont}
\end{codeenv}
The description font also needs to be set since this will contain
the long form:
\begin{codeenv}
\cs{glssetcategoryattribute}\marg{bacteria}\marg{\catattr{glossdescfont}}\marg{bacteriafont}
\end{codeenv}

The \code{markuplanguage} glossary contains descriptions and some
long names, so it's better suited to the \glostyle{altlist} style,
in which case the descriptions would look better if they started
with a capital letter:
\begin{codeenv}
\cs{glssetcategoryattribute}\marg{markuplanguage}\marg{\catattr{glossdesc}}\marg{firstuc}
\end{codeenv}
Remember that the \glostyle{altlist} style uses the
\env{description} environment, which is governed by the document
class (and may be modified by list-related packages). In this case,
one of the KOMA-Script classes is used, so the list items are
typeset in sans-serif.

There are various ways of dealing with the duplicated data in the
index, such as using the \csopt{secondary} option or having a
separate resource set with a copy \csopt{action}. In this case, I've
decided to use a dual entry system. Since the entries aren't defined
using any dual types, I've used \csopt{entry-type-aliases} to
make \bibgls\ treat them as though they were, and I also need to
alias the custom \atentryfmt{chemical}, \atentryfmt{unit} and 
\atentryfmt{measurement} \glspl{entrytype}:
\begin{codeenv}
\csopt[
  abbreviation=dualindexabbreviation,
  entry=dualindexentry,
  symbol=dualindexsymbol,
  unit=dualindexsymbol,
  measurement=dualindexsymbol,
  chemical=dualindexsymbol
 ]{entry-type-aliases}
\end{codeenv}
Note that I haven't aliased the \atentry{index} types as I only want
these in the index and not replicated in a separate glossary.

The primary entries for the \atentry{dualindexabbreviation} type
ignore the short form. It would be useful to store it. This could be
done by copying the \field{short} field with
\csopt{replicate-fields}. For example,
\csopt[\field{short}=\field{symbol}]{replicate-fields}. However, this will cause the
\field{symbol} field to be set for both the primary and dual
entries, which will cause an unwanted duplication if the dual
entries are displayed using a glossary style that shows the
\field{symbol} field. Another field (such as \field{user1}) could be
used instead or \gls{bibglsnewdualindexabbreviation} could be
defined before \gls{GlsXtrLoadResources}:
\begin{codeenv}
\cs{newcommand}\marg{\gls{bibglsnewdualindexabbreviation}}[7]\marg{\comment{}
 \cs{longnewglossaryentry*}\marg{\idx{param}1}\marg{\comment{}
  \field{name}=\marg{\cs{protect}\gls{bibglsuselongfont}\marg{\idx{param}4}\marg{\cs{glscategory}\marg{\idx{param}2}}},\comment{}
  \field{symbol}=\marg{\cs{protect}\gls{bibglsuseabbrvfont}\marg{\idx{param}5}\marg{\cs{glscategory}\marg{\idx{param}2}}},\comment{}
  \field{category}=\marg{index},\idx{param}3}\marg{}\comment{}
}
\end{codeenv}
However, this will affect all \atentry{dualindexabbreviation} 
\glspl{entrytype}, but it's not necessary for the bacteria abbreviations.
Instead it's simpler to just keep a record of the dual label so that
the short form can be obtained from the dual entry:
\begin{codeenv}
\csopt{dual-field}
\end{codeenv}

By default, the \atentry{dualindexabbreviation} \gls{entrytype} falls back on the
\field{short} field if the \field{name} is omitted. In this case I
want it to fall back on the \field{long} field instead.
\begin{codeenv}
\csopt[long]{abbreviation-name-fallback}
\end{codeenv}
Remember that the sort fallback for abbreviations is still \field{short} (but can be
changed with \csopt{abbreviation-sort-fallback}), but I've changed
the sort fallback for symbols:
\begin{codeenv}
\csopt[name]{symbol-sort-fallback}
\end{codeenv}
I also need to alias the custom fields (especially for those in the
\exfile{chemicalformula.bib}, \exfile{baseunits.bib} and
\exfile{derivedunits.bib} files):
\begin{codeenv}
\csopt[
  \fieldfmt{identifier}=\field{category},
  \fieldfmt{formula}=\field{symbol},
  \fieldfmt{chemicalname}=\field{name},
  \fieldfmt{unitname}=\field{name},
  \fieldfmt{unitsymbol}=\field{symbol},
  \fieldfmt{measurement}=\field{description}
]{field-aliases}
\end{codeenv}
There's a slight problem here. This ensures that the entries defined
in \exfile{chemicalformula.bib} have a \field{name} and
\field{symbol} field, which are swapped round for the dual
(according to the default \csopt{dual-indexsymbol-map}) but these
entries don't have a \field{description} field. Since I'd like to
use the \glostyle{mcolalttreegroup} style, this will end up with
the odd appearance of the formula (stored in the \field{name} field for the
dual) followed by the chemical name (stored in the \field{symbol}
field for the dual) in parenthesis. This is the default \meta{name}
(\meta{symbol}) \meta{description} format for the style. I've fixed
this by locally redefining \ics{glsxtralttreeSymbolDescLocation} for
just that glossary:
\begin{codeenv}
\cs{printunsrtglossary*}\oarg{\printglossopt[chemical]{type},\printglossopt[mcolalttreegroup]{style}}
\marg{\comment{}
  \cs{renewcommand}\cs{glsxtralttreeSymbolDescLocation}[2]\marg{\comment{}
   \ics{glossentrysymbol}\marg{\idx{param}1}\cs{glspostdescription}\ics{glsxtrAltTreePar}
  }\comment{}
  \cs{renewcommand}*\marg{\cs{glstreenamefmt}}[1]\marg{\idx{param}1}\comment{}
  \cs{renewcommand}*\marg{\cs{glstreegroupheaderfmt}}[1]\marg{\cs{textbf}\marg{\idx{param}1}}\comment{}
}
\end{codeenv}
I've also redefined \cs{glstreenamefmt} to prevent the names
appearing in bold, which means I also need to redefine
\cs{glstreegroupheaderfmt} to keep the headers bold.

All the \atentryfmt{dualindex\meta{type}} \glspl{entrytype} provide a primary
entry that behaves like \atentry{index}. The secondary behaves like
\atentryfmt{\meta{type}}. This means that the primaries are
conveniently gathered together with all the unaliased \atentry{index}
entries, so the primary \gls{entrytype} needs to be set to \code{index}:
\begin{codeenv}
\csopt[index]{type}
\end{codeenv}
The dual \gls{entrytype} depends on the entry's category. Since I've
defined my custom glossaries with a label that matches the
custom \fieldfmt{identifier} field, I can both alias this custom field
to the \field{category} field and also set \csopt{dual-type} so that
it matches the category:
\begin{codeenv}
\csopt[\fieldfmt{identifier}=\field{category}]{field-aliases},
\csopt[same as category]{dual-type}
\end{codeenv}
The primary entries (in the \code{index} glossary) need to be sorted
alphabetically, and since the document is in English I'm sorting
according to that language (identified by the language code
\code{en}), but I also want to make sure that all the primary
entries are sorted by the \field{name} field to avoid discrepancies
in the fallback value for the \field{sort} field:
\begin{codeenv}
\csopt[en]{sort},
\csopt[name]{sort-field}
\end{codeenv}
With \csopt[long]{abbreviation-name-fallback} now set, 
this means that \emph{Coxiella burnetii} comes after
\emph{Clostridium tetani} in the index. I haven't changed the sort
field for the dual entries, so in that case the
\csopt{abbreviation-sort-fallback} and \csopt{symbol-sort-fallback}
settings will be used with the duals.  This means that
\emph{C.\ burnetii} is between \emph{C.\ botulinum} and \emph{C.\
perfringens} rather than after \emph{C.\ tetani}.

I'd like to sort the dual entries according to a letter-number rule
(as for the above \exfile{sample-chemical.tex} and \exfile{sample-units3.tex}
examples)
but this would order \qt{b\'ilinite} after \qt{biotite} in the
\code{minerals} glossary, so instead I'm also using the English sort
rule for the duals, but with the numbers padded:
\begin{codeenv}
\csopt[en]{dual-sort},
\csopt[2]{dual-sort-number-pad},
\end{codeenv}
This method doesn't work as well as the method used in
\exfile{sample-chemical.tex} as it doesn't separate the capitals,
digits and \idx!{lowercase} characters in the way that can be achieved with the
letter-number methods. An improvement can be made by changing the
break-points. I could use \csopt[upper-upper]{dual-break-at} but
this would put \qt{seal} before \qt{sea lion} in the \code{animal}
glossary, so instead I've used:
\begin{codeenv}
\csopt[upper-upper-word]{dual-break-at}
\end{codeenv}
This now puts \qt{sea lion} before \qt{seal}. Unfortunately the word
break points will cause a break at the markers used to indicate
positive and negative numbers that are inserted with
\csopt{dual-sort-number-pad}, so these need to be changed to
something that won't cause them to be discarded:
\begin{codeenv}
\csopt[0]{dual-sort-pad-minus},
\csopt[1]{dual-sort-pad-plus}
\end{codeenv}

The document loads \sty{hyperref} which means that all the \cs{gls}
references will create hyperlinks. Since the primaries are in the
index, the default prefixes mean that, for example, 
\code{\cs{gls}\marg{svg}} links to the \qt{scalable vector graphics} item in
the index rather than to the abbreviation \qt{SVG} in the
\code{markuplanguage} glossary. There are two alternatives: change
\code{\cs{gls}\marg{svg}} to \code{\cs{gls}\marg{dual.svg}} or change the 
default prefixes, which is the more convenient approach and is the one used
here:
\begin{codeenv}
\csopt[idx.]{label-prefix},
\csopt[\empty]{dual-prefix}
\end{codeenv}
Now \code{\cs{gls}\marg{svg}} refers to the dual abbreviation \qt{SVG} and
\code{\cs{gls}\marg{idx.svg}} refers to the primary entry \qt{scalable vector
graphics}. Unfortunately this means that the records created with
\code{\cs{gls}\marg{svg}} now refer to the dual abbreviation and will end up
being displayed in the glossary instead of the index. This can be
fixed with:
\begin{codeenv}
\csopt[primary]{combine-dual-locations}
\end{codeenv}
Which transfers the dual entry \glspl{location} to the corresponding
primary.

The other problem is the cross-references in the \field{description}
fields. Since the labels don't start with \qt{\idprefix{dual}} \bibgls\
will assume they refer to the primary entries, which means that
\qt{\idprefixfmt{idx}} (the value of \csopt{label-prefix}) will be
inserted. This means that they'll link to the index rather than the
glossary entry. It also means that the cross-references where the
dual is an abbreviation won't behave like an abbreviation as the
reference is to the primary (non-abbreviation) entry. This can be
fixed by setting \csopt{cs-label-prefix} to the same value as
\csopt{dual-prefix}:
\begin{codeenv}
\csopt[\empty]{cs-label-prefix}
\end{codeenv}
The index is displayed using the \glostyle{bookindex} style. This
doesn't show the description or symbol by default, but it would be
useful to include the symbol in parentheses after the name. This can
be done by redefining \ics{glsxtrbookindexname}:
\begin{codeenv}
\cs{renewcommand}*\marg{\cs{glsxtrbookindexname}}[1]\marg{\comment{}
  \cs{glossentryname}\marg{\idx{param}1}\comment{}
  \cs{ifglshassymbol}\marg{\idx{param}1}\marg{\cs{space}(\cs{glossentrysymbol}\marg{\idx{param}1})}\marg{}\comment{}
}
\end{codeenv}
However the chemical forumlae look a little odd in parentheses
(especially those that contain parenthetical parts) but this can be
fixed by adding a category check:
\begin{codeenv}
\cs{renewcommand}*\marg{\cs{glsxtrbookindexname}}[1]\marg{\comment{}
  \cs{glossentryname}\marg{\idx{param}1}\comment{}
  \cs{ifglshassymbol}\marg{\idx{param}1}\comment{}
  \marg{\comment{}
    \cs{glsifcategory}\marg{\idx{param}1}\marg{chemical}\comment{}
    \marg{, \cs{glossentrysymbol}\marg{\idx{param}1}}\comment{}
    \marg{\cs{space}(\cs{glossentrysymbol}\marg{\idx{param}1})}\comment{}
  }\comment{}
  \marg{}\comment{}
}
\end{codeenv}
Unfortunately \ics{glossentrysymbol} doesn't pick up the
\catattr{glossnamefont} attribute, so if the short form of the
abbreviations is saved in the \field{symbol} field, using one of the
methods discussed above, then the custom \csfmt{bacteriafont} won't
be applied. (As from \sty{glossaries-extra} version 1.42, there is
now the \catattr{glosssymbolfont} attribute that's used by
\ics{glossentrysymbol}.)

A simple solution is to use \ics{glossentrynameother} instead:
\begin{codeenv}
\cs{renewcommand}*\marg{\cs{glsxtrbookindexname}}[1]\marg{\comment{}
  \cs{glossentryname}\marg{\idx{param}1}\comment{}
  \cs{ifglshassymbol}\marg{\idx{param}1}\comment{}
  \marg{\comment{}
    \cs{glsifcategory}\marg{\idx{param}1}\marg{chemical}\comment{}
    \marg{, \cs{glossentrysymbol}\marg{\idx{param}1}}\comment{}
    \marg{\cs{space}(\cs{glossentrynameother}\marg{\idx{param}1}\marg{\field{symbol}})}\comment{}
  }\comment{}
  \marg{}\comment{}
}
\end{codeenv}
However, since I decided not to store the short form in the
\field{symbol} field and just saved the dual entry label instead, I
need to lookup the short form from the dual entry:
\begin{codeenv}
\cs{renewcommand}*\marg{\cs{glsxtrbookindexname}}[1]\marg{\comment{}
  \cs{glossentryname}\marg{\idx{param}1}\comment{}
  \cs{ifglshassymbol}\marg{\idx{param}1}\comment{}
  \marg{\comment{}
    \cs{glsifcategory}\marg{\idx{param}1}\marg{chemical}\comment{}
    \marg{, \ics{glossentrysymbol}\marg{\idx{param}1}}\comment{}
    \marg{\cs{space}(\ics{glossentrynameother}\marg{\idx{param}1}\marg{\field{symbol}})}\comment{}
  }\comment{}
  \marg{\comment{}
    \cs{glsifcategory}\marg{\idx{param}1}\marg{markuplanguage}\comment{}
    \marg{\comment{}
      \cs{glsxtrifhasfield}\marg{\field{short}}\marg{\cs{glsxtrusefield}\marg{\idx{param}1}\marg{\field{dual}}}\comment{}
      \marg{\cs{space}(\cs{glscurrentfieldvalue})}\comment{}
      \marg{}\comment{}
    }\comment{}
    \marg{}\comment{}
  }\comment{}
}
\end{codeenv}
Not all of the markup languages are abbreviations so this uses
\ics{glsxtrifhasfield} to check if the \field{short} field is set.
The dual entry's label is easily obtained because \csopt{dual-field}
has provided the \fieldfmt{dual} internal field and set it to the
corresponding label.

It's sometimes useful for the index to include a reference to the
term's definition. This can be done by making use of 
\idx{glsextrapostnamehook}, which can be redefined before the
glossaries to automatically record each entry:
\begin{codeenv}
\cs{renewcommand}\marg{\cs{glsextrapostnamehook}}[1]\marg{\cs{glsadd}\oarg{\glsaddopt[\encap{hyperbf}]{format}}\marg{\idx{param}1}}
\end{codeenv}
This needs to be redefined to ignore its argument before the index,
to avoid the redundant index record:
\begin{codeenv}
\cs{renewcommand}\marg{\cs{glsextrapostnamehook}}[1]\marg{}
\end{codeenv}
Remember that if any records are added within a glossary, an extra
\LaTeX\ and \bibgls\ call are required to ensure that the \gls{locationlist} 
is correct, so the document build is:
\begin{verbatim}
pdflatex sample-multi1
bib2gls --group sample-multi1
pdflatex sample-multi1
bib2gls --group sample-multi1
pdflatex sample-multi1
\end{verbatim}
The complete document code is listed below. 
The resulting document is shown in \figureref{fig:sample-multi1.pdf}
and \figureref{fig:sample-multi1.pdf2}.

\lstinputlisting[firstline=5]{../examples/sample-multi1.tex}

\begin{figure}
\figcontents
{\pdftwocol{../examples/sample-multi1.pdf}{4}}
{\caption{\filefmt{sample-multi1.pdf} (pages~1 to~4)}}
{fig:sample-multi1.pdf}
\end{figure}

\begin{figure}
\figcontents
{\pdftwocol[5]{../examples/sample-multi1.pdf}{8}}
{\caption{\filefmt{sample-multi1.pdf} (pages~5 to 8)}}
{fig:sample-multi1.pdf2}
\end{figure}

\filesection{sample-multi2.tex}

This example is an alternative approach to
\exfile{sample-multi1.tex}. Instead of using dual \glspl{entrytype} to
define entries that appear in both a glossary and the index, this
example makes use of \csopt{record-label-prefix} to reselect the
recorded entries for the index. This is more complicated but it
allows the entries that have natural word ordering to use a locale
sort method while the entries that are symbolic can use one of the
letter-number sort methods.

This document uses some additional \ext{bib} files to the previous
example, so it has extra glossaries, which all need to be defined:
\begin{codeenv}
\cs{newglossary*}\marg{bacteria}\marg{Bacteria}
\cs{newglossary*}\marg{markuplanguage}\marg{Markup Languages}
\cs{newglossary*}\marg{vegetable}\marg{Vegetables}
\cs{newglossary*}\marg{mineral}\marg{Minerals}
\cs{newglossary*}\marg{animal}\marg{Animals}
\cs{newglossary*}\marg{chemical}\marg{Chemical Formula}
\cs{newglossary*}\marg{baseunit}\marg{SI Units}
\cs{newglossary*}\marg{measurement}\marg{Measurements}
\cs{newglossary*}\marg{film}\marg{Films}
\cs{newglossary*}\marg{book}\marg{Books}
\cs{newglossary*}\marg{person}\marg{People}
\cs{newglossary*}\marg{mediacontrol}\marg{Media Control Symbols}
\cs{newglossary*}\marg{information}\marg{Information Symbols}
\cs{newglossary*}\marg{weather}\marg{Weather Symbols}
\end{codeenv}
Note that this is a total of 15~glossaries (including the
\code{index}). With the basic \ics{cs.makeglossaries} method, this would
require 16 write registers (including the write register used to
create the indexing style file), and a total of $15\times3 + 1 = 46$
associated files. (This doesn't include the standard \ext{aux} file
or the \iext{out} file created by \sty{hyperref}.) With \bibgls, no
additional write registers are required and the number of associated
\bibgls\ files is equal to the number of resource commands plus the
transcript file (in this example, $9+1=10$).

Since this document requires \exfile!{people.bib}, \exfile!{books.bib}
and \exfile!{films.bib} it also requires the files that
supply the definitions of the custom commands
(\exfile!{no-interpret-preamble.bib} and either
\exfile!{interpret-preamble.bib} or \exfile!{interpret-preamble2.bib})
to ensure the custom commands are provided both for the document and
for \bibgls's interpreter.

The first resource set to be loaded simply reads
\exfile{no-interpret-preamble.bib} with the preamble interpreter
switched off:
\begin{codeenv}
\gls{GlsXtrLoadResources}\oarg{
  \csopt[no-interpret-preamble]{src},
  \csopt[false]{interpret-preamble}
}
\end{codeenv}
This ensures that \LaTeX\ can pick up the provided commands and
prevents them from being added to the interpreter.

The \exfile{people.bib} file is the next to be loaded with
\exfile{interpret-preamble.bib}. This is loaded separately from the
other resources as this needs the \field{name} field to be copied to
\field{first} (if not already set), as in the
\exfile{sample-people.tex} file. By having a separate resource set,
this setting doesn't affect the other entries. I've also converted
the date fields so that I can customise the format in the document.
\begin{codeenv}
\gls{GlsXtrLoadResources}\oarg{
 \csopt[interpret-preamble,people]{src},
 \csopt[
   \fieldfmt{identifier}=\field{category},
   \fieldfmt{born}=\field{user1},
   \fieldfmt{died}=\field{user2},
   \fieldfmt{othername}=\field{user3}
 ]{field-aliases},
 \csopt[\field{name}=\marg{\field{first}}]{replicate-fields},
 \csopt[person]{type},
 \csopt[false]{save-locations}
 \csopt[\field{user1},\field{user2}]{date-fields},
 \csopt[d MMM y G]{date-field-format}
}
\end{codeenv}
As with the \exfile{sample-people.tex} document, I need to use the
\longarg{break-space} switch to convert the \idx{nbspchar} to a
normal breakable space so that it matches the given format. I've
loaded the \sty{datetime2} package:\footnote{The
\styoptfmt{en-GB} option to \sty{datetime2} also requires that
\styfmt{datetime2-english} must be installed.}
\begin{verbatim}
\usepackage[en-GB]{datetime2}
\end{verbatim}
so that I can use \ics{DTMdisplaydate} to adjust the formatting:
\begin{codeenv}
\cs{newcommand}*\marg{\gls{bibglsdate}}[7]\marg{\cs{DTMdisplaydate}\marg{\idx{param}1}\marg{\idx{param}2}\marg{\idx{param}3}\marg{\idx{param}4}}
\end{codeenv}
This needs to go before the \idx{resourceset} is loaded. Note that
the \styoptfmt{en-GB} option identifies the document locale as
\code{en-GB} (since there are no language packages loaded).

Note that unlike \exfile{sample-people.tex} which had
\csopt[people]{category}, this document obtains the \field{category} field
from the custom \fieldfmt{identifier} field, which in this case 
has the value \code{person}. This means that the category hooks from
\exfile{sample-people.tex} need to be renamed to reflect the
different category label:
\begin{codeenv}
\cs{newcommand}*\marg{\postlinkhook{person}}{\comment{}
 \cs{glsxtrifwasfirstuse}
 \marg{\comment{}
   \cs{ifglshasfield}\marg{\field{user3}}\marg{\cs{glslabel}}\comment{}
   \marg{\cs{space}(\cs{glscurrentfieldvalue})}\comment{}
   \marg{}\comment{}
 }\comment{}
 \marg{}\comment{}
}
\strut
\cs{newcommand}*\marg{\postnamehook{person}}\marg{\comment{}
 \cs{ifglshasfield}\marg{\field{user3}}\marg{\cs{glscurrententrylabel}}\comment{}
 \marg{\cs{space}(\cs{glscurrentfieldvalue})}\comment{}
 \marg{}\comment{}
}
\strut
\cs{newcommand}*\marg{\postdeschook{person}}\marg{\comment{}
 \cs{ifglshasfield}\marg{\field{user1}}\marg{\cs{glscurrententrylabel}}
 \marg{\comment{born}
   \cs{space}(\cs{glscurrentfieldvalue}\cs{comma}-{}-\cs{comma}\comment{}
     \cs{ifglshasfield}\marg{\field{user2}}\marg{\cs{glscurrententrylabel}}
     \marg{\comment{died}
        \cs{glscurrentfieldvalue}
     }\comment{}
     \marg{}\comment{}
   )\comment{}
 }\comment{}
 \marg{}\comment{}
}
\end{codeenv}
The other \ext{bib} files that require locale sorting can now be
loaded, but remember that the abbreviation style settings must be
set first since this resource set includes abbreviations:
\begin{codeenv}
\cs{setabbreviationstyle}\oarg{bacteria}\marg{\abbrstyle{long-only-short-only}}
\cs{setabbreviationstyle}\oarg{markuplanguage}\marg{\abbrstyle{long-short-desc}}
\strut
\cs{renewcommand}*\marg{\cs{glsxtrlongshortdescname}}\marg{\comment{}
  \cs{protect}\cs{glsabbrvfont}\marg{\cs{the}\cs{glsshorttok}}\cs{space}
  \glsxtrparen{\glslongfont{\the\glslongtok}}%
}
\end{codeenv}
Now the resource set can be loaded:
\begin{codeenv}
\gls{GlsXtrLoadResources}\oarg{
 \csopt[bacteria,markuplanguages,vegetables,
  minerals,animals,books,films]{src},
 \csopt[\fieldfmt{identifier}=\field{category}]{field-aliases},
 \csopt[same as category]{type},
 \csopt[false]{save-locations}
}
\end{codeenv}
The semantic markup command and attributes are as for
\exfile{sample-multi1.tex}:
\begin{codeenv}
\cs{newcommand}\marg{\cmd{bacteriafont}}[1]\marg{\cs{emph}\marg{\idx{param}1}}
\cs{glssetcategoryattribute}\marg{bacteria}\marg{\catattr{textformat}}\marg{bacteriafont}
\cs{glssetcategoryattribute}\marg{bacteria}\marg{\catattr{glossnamefont}}\marg{bacteriafont}
\cs{glssetcategoryattribute}\marg{bacteria}\marg{\catattr{glossdescfont}}\marg{bacteriafont}
\cs{glssetcategoryattribute}\marg{markuplanguage}\marg{\catattr{glossdesc}}\marg{firstuc}
\end{codeenv}
Similarly for the books:
\begin{codeenv}
\cs{newcommand}\marg{\cmd{bookfont}}[1]\marg{\cs{emph}\marg{\idx{param}1}}
\cs{glssetcategoryattribute}\marg{book}\marg{\catattr{textformat}}\marg{bookfont}
\cs{glssetcategoryattribute}\marg{book}\marg{\catattr{glossnamefont}}\marg{bookfont}
\end{codeenv}
(as for \exfile{sample-media.tex}) and for films:
\begin{codeenv}
\cs{newcommand}\marg{\cmd{filmfont}}[1]\marg{\cs{emph}\marg{\idx{param}1}}
\cs{glssetcategoryattribute}\marg{film}\marg{\catattr{textformat}}\marg{filmfont}
\cs{glssetcategoryattribute}\marg{film}\marg{\catattr{glossnamefont}}\marg{filmfont}
\end{codeenv}

Next come the chemical formulae:
\begin{codeenv}
\gls{GlsXtrLoadResources}\oarg{
 \csopt[chemicalformula]{src},
 \csopt[chemical=symbol]{entry-type-aliases},
 \csopt[
   \fieldfmt{identifier}=\field{category},
   \fieldfmt{formula}=\field{name},
   \fieldfmt{chemicalname}=\field{description}
 ]{field-aliases},
 \csopt[chemical]{type},
 \csopt{set-widest},
 \csopt[letternumber-case]{sort},
 \csopt[\field{name}]{symbol-sort-fallback},
 \csopt[false]{save-locations}
}
\end{codeenv}
and the \idxpl{SIunit}, which are now combined into a single
glossary:
\begin{codeenv}
\gls{GlsXtrLoadResources}\oarg{
 \csopt[baseunits,derivedunits]{src},
 \csopt[\fieldfmt{measurement}=\field{symbol},\fieldfmt{unit}=\field{symbol}]{entry-type-aliases},
 \csopt[
   \fieldfmt{unitname}=\field{description},
   \fieldfmt{unitsymbol}=\field{symbol},
   \fieldfmt{measurement}=\field{name}
 ]{field-aliases},
 \csopt[measurement]{category},
 \csopt[measurement]{type},
 \csopt{set-widest},
 \csopt[\field{name}]{symbol-sort-fallback},
 \csopt[false]{save-locations}
}
\end{codeenv}
Here the \field{name} field is obtained from the custom
\fieldfmt{measurement} field. Since this contains a word, the
default locale sort is appropriate. I've locally redefined
\ics{glsxtralttreeSymbolDescLocation} to place the symbol in
parentheses after the description:
\begin{codeenv}
\cs{printunsrtglossary*}\oarg{\printglossopt[measurement]{type},\printglossopt[alttree]{style},\printglossopt{nogroupskip}}
\marg{\comment{}
  \cs{renewcommand}\marg{\cs{glsxtralttreeSymbolDescLocation}}[2]\marg{\comment{}
    \cs{glossentrydesc}\marg{\idx{param}1}\comment{}
    \cs{ifglshassymbol}\marg{\idx{param}1}\marg{\cs{space}(\cs{glossentrysymbol}\marg{\idx{param}1})}\marg{}\comment{}
    \cs{glspostdescription}
    \cs{glsxtrAltTreePar}
  }\comment{}
}
\end{codeenv}

The base units are replicated in the \code{baseunit} glossary, this
time with the \field{name} field obtained from the custom
\fieldfmt{unitsymbol} field. This means that I need to find a way to 
prevent duplicate labels. The simplest method is to use
\csopt{duplicate-label-suffix}:
\begin{codeenv}
\gls{GlsXtrLoadResources}\oarg{
 \csopt[baseunits]{src},
 \csopt[unit=symbol]{entry-type-aliases},
 \csopt[
   \fieldfmt{unitname}=\field{description},
   \fieldfmt{unitsymbol}=\field{name}
 ]{field-aliases},
 \csopt[measurement]{category},
 \csopt[baseunit]{type},
 \csopt[.copy]{duplicate-label-suffix},
 \csopt[\field{name}]{symbol-sort-fallback},
 \csopt[false]{save-locations}
}
\end{codeenv}
I can't use \csopt{set-widest} here as it won't pick up the modified
label and will instead use the label from the original entry.
Instead I've used \cs{glsFindWidestTopLevelName} to find it:
\begin{codeenv}
\cs{printunsrtglossary*}\oarg{\printglossopt[baseunit]{type},\printglossopt[alttree]{style},\printglossopt{nogroupskip}}
\marg{\comment{}
  \cs{glsFindWidestTopLevelName}\oarg{baseunit}\comment{}
}
\end{codeenv}

The text symbols from \exfile{miscsymbols.bib} are all loaded in a
single \igls{resourceset}, where the \field{type} field can be
obtained from the \field{category}, which in turns is obtained from
the custom \fieldfmt{identifier} field. Since \bibgls\ doesn't
recognise any of the symbol commands, I'm sorting according to the
\field{description} field. (Even if \bibgls\ could determine a
Unicode value for each of the symbols, sorting by the description
makes more sense in this case.)
\begin{codeenv}
\gls{GlsXtrLoadResources}\oarg{
 \csopt[miscsymbols]{src},
 \csopt[
   \fieldfmt{identifier}=\field{category},
   \fieldfmt{icon}=\field{name},
   \fieldfmt{icondescription}=\field{description}
 ]{field-aliases},
 \csopt[icon=symbol]{entry-type-aliases},
 \csopt[same as category]{type},
 \csopt[\field{description}]{sort-field},
 \csopt[false]{save-locations},
 \csopt{set-widest}
}
\end{codeenv}

Finally, all recorded and cross-referenced terms are needed for the
index. This includes entries that have already been defined in the
earlier \iglspl{resourceset} (so a guard against duplicates is
necessary) but it also includes entries from the \exfile{terms.bib}
file that haven't yet been dealt with. I'd like the index to start
with a \idx{symbolgroup} containing the icons from
\exfile{miscsymbols.bib}. This needs to be dealt with separately from
the rest of the index to keep them together in a single group:
\begin{codeenv}
\gls{GlsXtrLoadResources}\oarg{
  \csopt[miscsymbols]{src},
  \csopt[recorded no deps]{selection},
  \csopt[.copy]{duplicate-label-suffix},
  \csopt[icon=index]{entry-type-aliases},
  \csopt[
   \fieldfmt{identifier}=\field{category},
   \fieldfmt{icondescription}=\field{symbol},
   \fieldfmt{icon}=\field{name}
  ]{field-aliases},
  \csopt[index]{type},
  \csopt[\field{symbol}]{sort-field},
  \csopt[glssymbols]{group}
}
\end{codeenv}
Since I know that there are no \glsdisp{parententry}{parents} or cross-references in this
set of entries I've used \csopt[recorded no deps]{selection} to skip
the dependency checks. In this \igls{resourceset}, the \field{name}
field has the symbol command (obtained from the custom
\fieldfmt{icon} field), and the \field{symbol} field has the symbol
description (obtained from the custom \fieldfmt{icondescription}
field), which is used as the sort field. I've set the \field{group}
label to \code{glssymbols}, which keeps all these entries in a
single group and the title will be obtained from
\ics{glssymbolsgroupname}.

Before loading the final \igls{resourceset} \ics{glsxtrlongshortdescname} needs 
to be changed
so that the abbreviations using the \abbrstyle{long-short-desc}
style (that is, the abbreviations with the \field{category} set to 
\code{markuplanguage}) have the \field{name} field set to
\meta{long} (\meta{short}):
\begin{codeenv}
\cs{renewcommand}*\marg{\cs{glsxtrlongshortdescname}}\marg{\comment{}
  \cs{protect}\cs{glslongfont}\marg{\cs{the}\cs{glslongtok}}\cs{space}
  \cs{glsxtrparen}\marg{\cs{glsabbrvfont}\marg{\cs{the}\cs{glsshorttok}}}\comment{}
}
\end{codeenv}
The \abbrstyle{long-only-short-only} style has a similar command,
but it was only introduced to \sty{glossaries-extra} version 1.25:
\begin{codeenv}
\cs{renewcommand}*\marg{\ics{glsxtronlyname}}{\comment{}
  \cs{protect}\cs{glsabbrvonlyfont}\marg{\cs{the}\cs{glslongtok}}\comment{}
}
\end{codeenv}

The abbreviations all need to be sorted according to the long form:
\begin{codeenv}
\csopt[\field{long}]{abbreviation-sort-fallback}
\end{codeenv}
The custom \glspl{entrytype} and fields again need to be aliased
\begin{codeenv}
\csopt[
 chemical=index,
 measurement=entry,
 unit=dualentry,
 icon=index
]{entry-type-aliases},
\csopt[
   \fieldfmt{identifier}=\field{category},
   \fieldfmt{formula}=\field{symbol},
   \fieldfmt{chemicalname}=\field{name},
   \fieldfmt{unitname}=\field{description},
   \fieldfmt{unitsymbol}=\field{symbol},
   \fieldfmt{measurement}=\field{name},
   \fieldfmt{icon}=\field{symbol},
   \fieldfmt{icondescription}=\field{name}
 ]{field-aliases}
\end{codeenv}
The chemical formulae and icons are now defined using
\atentry{index} with the \field{name} field set to a word form
(chemical name and icon description). This means
they're appropriate for alphabetical sorting. (Both \atentry{entry}
and \atentry{symbol} require the \field{description} field, which is
why I've aliased \atentryfmt{chemical} and \atentryfmt{icon}
to \atentry{index} here.) The custom \atentryfmt{measurement} 
\gls{entrytype} has a \field{description} field (obtained from
\fieldfmt{unitname}), so that's aliased to \atentry{entry} as again
the \field{name} field is suitable for alphabetical sorting.

I've aliased \atentryfmt{unit} to \atentry{dualentry}
rather than \atentry{symbol} as I want both the unit name and the
measurement in the index and I've combined their \glspl{locationlist}:
\begin{codeenv}
\csopt[both]{combine-dual-locations}
\end{codeenv}
Both primary and dual entries need to go in the \code{index}
glossary:
\begin{codeenv}
\csopt[index]{type},
\csopt[index]{dual-type}
\end{codeenv}

All \ext{bib} files used in the previous \iglspl{resourceset} are needed
as well as the \exfile{terms.bib} file:
\begin{codeenv}
\csopt[terms,bacteria,markuplanguages,vegetables,minerals,
  animals,chemicalformula,baseunits,derivedunits,people,
  films,books,miscsymbols]{src}
\end{codeenv}
but this time I also want to select entries that haven't been
recorded but have a cross-reference to a recorded entry:
\begin{codeenv}
\csopt{selection}[recorded and deps and see]
\end{codeenv}
Again it's necessary to provide a way to avoid duplicate entry
labels, which can be done with
\begin{codeenv}
\csopt[.copy]{duplicate-label-suffix},
\end{codeenv}
as above.  However, this will cause the cross-references (from the
\field{alias} fields) to link to the glossary rather than the index.
This may or may not confuse the reader. For consistency, it may be
more suitable to have the cross-reference in the index link to the
aliased entry in the index rather than in the glossary. I've
therefore instead used:
\begin{codeenv}
 \csopt[idx.]{label-prefix},
 \csopt[idx.]{record-label-prefix},
\end{codeenv}
This means that the entries defined in \exfile{terms.bib} need to be
referenced with this prefix.

All instances of \cs{gls} will link to the original entry, so all
entries except for those in the \exfile{terms.bib} file will link to
the relevant glossary. Those in the \exfile{terms.bib} file will
link to the index. It's possible to disable the hyperlinks for those
entries, but the reader may find it useful to jump to the index to
look up other \glspl{location} for that entry in the document.

To deal with the identical book and film titles, I'm again using the
\field{category} to resolve identical sort values:
\begin{codeenv}
\csopt[\field{category}]{identical-sort-action}
\end{codeenv}
For the people who have a \field{first} field, I've decided that
this would be more appropriate for the index as it's more compact
than the \field{name}, so here I've done the reverse to earlier and copied the
\field{first} field (if supplied) into the \field{name} field, but
since the \field{name} field is already provided the override 
setting needs to be on:
\begin{codeenv}
\csopt{replicate-override},
\csopt[\field{first}=\field{name}]{replicate-fields}
\end{codeenv}
As with \exfile{sample-people.tex} I've provided some custom
commands to make it easier to locally redefine \gls{sortname} and
\gls{sortvonname}:
\begin{codeenv}
\cs{newcommand}*\marg{\cmd{swaptwo}}[2]\marg{\idx{param}2, \idx{param}1}
\cs{newcommand}*\marg{\cmd{swapthree}}[3]\marg{\idx{param}2 \idx{param}3, \idx{param}1}
\end{codeenv}
I've redefined \ics{glsxtrbookindexname} in a similar manner to
\exfile{sample-multi1.tex} but it has some modifications:
\begin{codeenv}
\cs{renewcommand}*\marg{\cs{glsxtrbookindexname}}[1]\marg{\comment{}
  \cs{glossentryname}\marg{\idx{param}1}\comment{}
  \cs{ifglshassymbol}\marg{\idx{param}1}\comment{}
  \marg{\comment{}
    \cs{glsifcategory}\marg{\idx{param}1}\marg{chemical}\comment{}
    \marg{, \cs{glossentrysymbol}\marg{\idx{param}1}}\comment{}
    \marg{\cs{space}(\cs{glossentrynameother}\marg{\idx{param}1}\marg{\field{symbol}})}\comment{}
  }\comment{}
  \marg{\comment{}
    \cs{glsifcategory}\marg{\idx{param}1}\marg{film}\comment{}
    \marg{\cs{cs.space}(film)}\comment{}
    \marg{}\comment{}
  }\comment{}
}
\end{codeenv}
This appends \qt{(film)} to film names. I've chosen this method
rather than using the \idx{postnamehook} as I only want this in the
index and not in the list of films.

For some of the entries that are referenced in the document, I've
appended information in parentheses:
\begin{codeenv}
\cs{gls}\marg{Al2SO43} (\ics{glsdesc}\marg{Al2SO43})
\end{codeenv}
This is all right for odd instances, but if this always needs to be done 
on first use, then it's better to use the \idx{postlinkhook}, which
is what I've done for the icons for comparison:
\begin{codeenv}
\cs{newcommand}*\marg{\postlinkhook{mediacontrol}}\marg{\comment{}
 \ics{glsxtrpostlinkAddDescOnFirstUse}
}
\strut
\cs{newcommand}*\marg{\postlinkhook{information}}\marg{\comment{}
 \cs{glsxtrpostlinkAddDescOnFirstUse}
}
\strut
\cs{newcommand}*\marg{\postlinkhook{weather}}\marg{\comment{}
 \cs{glsxtrpostlinkAddDescOnFirstUse}
}
\end{codeenv}

I've also provided some custom commands to make it easier to reference
entries without worrying about the prefixes:
\begin{codeenv}
\cs{newcommand}\marg{\csfmt{unit}}\marg{\ics{glssymbol}}
\cs{newcommand}\marg{\csfmt{measurement}}\marg{\cs{gls}}
\gls{glsxtrnewgls}\marg{film.}\marg{\csfmt{film}}
\end{codeenv}

As with \exfile{sample-multi1.tex}, it would be useful to include
the page where the entries are defined in their corresponding lists.
Again this can be done by redefining the general purpose
non-category post-name hook \ics{glsextrapostnamehook}:
\begin{codeenv}
\cs{newcommand}*\marg{\cs{glsextrapostnamehook}}[1]\marg{\comment{}
  \cs{glsadd}\oarg{\glsaddopt[\encap{hyperbf}]{format}}\marg{\idx{param}1}\comment{}
}
\end{codeenv}
This needs resetting before the index, since it's redundant to
record an entry in the index. This will require an extra
\bibgls+\LaTeX\ system call as this code can't be performed until
the glossaries have been created.

The complete document code is listed below. The document build is:
\begin{verbatim}
pdflatex sample-multi2
bib2gls --group --break-space sample-multi2
pdflatex sample-multi2
bib2gls --group --break-space sample-multi2
pdflatex sample-multi2
\end{verbatim}
The resulting document is shown in \figureref{fig:sample-multi2.pdf},
\figureref{fig:sample-multi2.pdf2} and \figureref{fig:sample-multi2.pdf3}.

\lstinputlisting[firstline=5]{../examples/sample-multi2.tex}

\begin{figure}
\figcontents
{\pdftwocol{../examples/sample-multi2.pdf}{4}}
{\caption{\filefmt{sample-multi2.pdf} (pages~1 to~4)}}
{fig:sample-multi2.pdf}
\end{figure}

\begin{figure}
\figcontents
{\pdftwocol[5]{../examples/sample-multi2.pdf}{8}}
{\caption{\filefmt{sample-multi2.pdf} (pages~5 to 8)}}
{fig:sample-multi2.pdf2}
\end{figure}

\begin{figure}
\figcontents
{\pdftwocol[9]{../examples/sample-multi2.pdf}{12}}
{\caption{\filefmt{sample-multi2.pdf} (pages~9 and 12)}}
{fig:sample-multi2.pdf3}
\end{figure}

\setupglossaries{indexonlyfirst}
\printstyoptsummary
\printcommandsummary
%\printcommandtopic

\bibliographystyle{plain}
\bibliography{bib2gls-cite}

\renewcommand*{\glsxtrbookindexsubname}[1]{%
  \glsifcategory{#1}{command}%
  {%
    \glsxtrifhasfield{useri}{#1}%
    {\glossentrynameother{#1}{useri}}%
    {\glossentryname{#1}}%
  }%
  {\glossentryname{#1}}%
}
\newcommand{\locationfont}{\small}
\newcommand{\widecrossreffont}{\footnotesize}
\newcommand{\ifwidecrossref}[3]{%
  \dimen0=0pt\relax
  \glsxtrifhasfield*{see}{#1}%
  {\settowidth{\dimen0}{\locationfont\glsxtrusesee{#1}}}%
  {%
    \glsxtrifhasfield*{seealso}{#1}%
    {\settowidth{\dimen0}{\locationfont\glsxtruseseealso{#1}}}%
    {%
      \glsxtrifhasfield*{alias}{#1}%
      {\settowidth{\dimen0}{\locationfont\glsxtrusealias{#1}}}%
      {}%
    }%
  }%
  \ifdim\dimen0>0.4\linewidth #2\else#3\fi
}
\newcommand{\precrossref}{}
\newcommand{\indexdotfill}{\,\cleaders\hbox to .44em{\hss\textcolor{lightgray}{.}\hss}\hfill\,}
\renewcommand*{\glsxtrbookindexprelocation}[1]{%
  \glsxtrifhasfield*{location}{#1}%
  {\nobreak\indexdotfill
   \ifwidecrossref{#1}%
   {\def\precrossref{\widecrossreffont\glstreesubitem}}%
   {\def\precrossref{, }}%
  }%
  {%
    \ifwidecrossref{#1}{\nobreak\def\precrossref{\widecrossreffont\glstreesubitem}}%
    {\def\precrossref{\nobreak\indexdotfill}}%
    \glsxtrprelocation
  }%
}
\renewcommand{\glsxtrbookindexlocation}[2]{%
  \begingroup
  \locationfont
  #2%
  \glsxtrifhasfield{see}{#1}{\precrossref\glsxtrusesee{#1}}%
  {%
    \glsxtrifhasfield{seealso}{#1}%
    {\precrossref\glsxtruseseealso{#1}}%
    {%
      \glsxtrifhasfield{alias}{#1}{\precrossref\glsxtrusealias{#1}}{}%
    }%
  }%
  \endgroup
  \let\precrossref\empty
}
% With \printcommandtopic missing a lot of spawned entries will
% appear in the index without a record. Without the topic list they
% need to be skipped
\renewcommand{\printunsrtglossaryentryprocesshook}[1]{%
 \glsxtrifhasfield*{progenitor}{#1}%
 {%\typeout{SKIPPING [has progenitor] #1}
   \printunsrtglossaryskipentry
 }%
 {%
   \GlsXtrIfFieldEqStr*{originalentrytype}{#1}{topic}%
   {%
     \glsxtrifhasfield*{location}{#1}%
     {}%
     {%
       \ifglshasdesc{#1}{}%
       {%
         %\typeout{SKIPPING [no location] #1}
         \printunsrtglossaryskipentry
       }%
     }%
   }%
   {%
     \GlsXtrIfFieldValueInCsvList*{#1}{category}%
     {abbreviationstyle,packageoption,glossarystyle,package}%
     {%
       \glsxtrifhasfield*{location}{#1}{}%
       {%\typeout{SKIPPING [topic \glscategory{#1} no location] #1}%
        \printunsrtglossaryskipentry
       }%
     }%
     {}%
   }%
 }%
}
\renewcommand*{\unicodecategoryfmt}[1]{\textsf{#1}}
\glsrenewcommand*{\bibglspassim}{}
\newcommand{\glsxtrpostnamecommand}{%
  \glsxtrifhasfield{userii}{\glscurrententrylabel}{ (\glscurrentfieldvalue)}{}%
}
\printunsrtglossary[type=index]
\end{document}
