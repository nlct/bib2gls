% arara: pdflatex
% arara: pdflatex
\documentclass[titlepage=false]{scrreprt}

\usepackage[T1]{fontenc}
\usepackage{alltt}
\usepackage{upquote}
\usepackage{etoolbox}
\usepackage{datetime2}
\usepackage[colorlinks]{hyperref}

\DTMsavetimestamp{creation}{2017-01-20T15:39:00Z}

\IfFileExists{../java/Bib2Gls.java}
{
  \DTMsavefilemoddate{moddate}{../java/Bib2Gls.java}
}
{
  \DTMsavenow{moddate}
}

\providecommand{\frontmatter}{\clearpage\pagenumbering{roman}}
\providecommand{\mainmatter}{\clearpage\pagenumbering{arabic}}

\newcommand{\bibgls}{\appfmt{bib2gls}}

\newcommand{\qt}[1]{``#1''}

\newcommand{\dequals}{%
 \texorpdfstring
 {\discretionary{}{}{}\texttt{=}\discretionary{}{}{}}%
 {=}%
}

\newcommand{\dcomma}{%
 \texorpdfstring
 {\texttt{,}\discretionary{}{}{}}%
 {,}%
}

\newcommand*{\filefmt}[1]{\texorpdfstring{\nolinkurl{#1}}{#1}}
\newcommand*{\metafilefmt}[3]{%
  \filefmt{#1}\discretionary{}{}{}\meta{#2}\discretionary{}{}{}\filefmt{#3}%
}

\newcommand*{\appfmt}[1]{\texorpdfstring{\texttt{#1}}{#1}}
\newcommand*{\styfmt}[1]{\texorpdfstring{\textsf{#1}}{#1}}
\newcommand*{\csfmt}[1]{\texorpdfstring{\texttt{\char`\\#1}}{\string\\#1}}
\newcommand*{\optfmt}[1]{\texorpdfstring{\texttt{#1}}{#1}}
\newcommand*{\fieldfmt}[1]{\texorpdfstring{\texttt{#1}}{#1}}
\newcommand*{\entryfmt}[1]{\texorpdfstring{\texttt{#1}}{#1}}
\newcommand*{\abbrstylefmt}[1]{\texorpdfstring{\textsf{#1}}{#1}}
\newcommand*{\idprefixfmt}[1]{%
 \texorpdfstring{\texttt{#1\spacefactor1000 .}}{#1.}}

\newcommand*{\argor}{\texorpdfstring{\protect\textbar}{\string\|}}

\newcommand*{\meta}[1]{%
 \texorpdfstring{$\langle${\normalfont\emph{#1}}$\rangle$}{#1}%
}

\newcommand*{\oarg}[1]{[#1]}
\newcommand*{\oargm}[1]{\oarg{\meta{#1}}}

\newcommand*{\marg}[1]{\texorpdfstring
 {\char`\{#1\char`\}}
 {\{#1\}}}

\newcommand*{\margm}[1]{\marg{\meta{#1}}}

\newcommand*{\file}[1]{%
 \texorpdfstring
 {\filefmt{#1}\index{#1@\filefmt{#1}}}%
 {#1}%
}

\newcommand*{\ext}[1]{%
 \texorpdfstring
 {\filefmt{.#1}\index{file formats!#1@\filefmt{.#1}}}%
 {.#1}%
}

\newcommand*{\app}[1]{%
 \texorpdfstring
 {\appfmt{#1}\index{#1@\appfmt{#1}}}%
 {#1}%
}

\newcommand*{\sty}[1]{%
  \texorpdfstring
  {\styfmt{#1}\index{#1@\styfmt{#1}}}%
  {#1}%
}

\newcommand*{\abbrstyle}[1]{%
  \texorpdfstring
  {\abbrstylefmt{#1}\index{abbreviation style!#1@\abbrstylefmt{#1}}}%
  {#1}%
}

\newcommand*{\idprefix}[1]{%
 \texorpdfstring
 {\idprefixfmt{#1}\index{#1@\idprefixfmt{#1}}}%
 {#1}%
}

\newcommand*{\cs}[1]{% 
 \texorpdfstring
 {\csfmt{#1}\index{#1@\csfmt{#1}}}
 {\string\\#1}%
}

\newcommand*{\styopt}[2][]{%
  \texorpdfstring%
  {%
    \optfmt{#2\ifblank{#1}{}{\dequals#1}}%
    \index{package options@\optfmt{#2}}%
    \index{#2@\optfmt{#2}}%
  }%
  {#2\ifblank{#1}{}{=#1}}%
}

\newcommand*{\csopt}[2][]{\gencsopt[#1]{glsxtrresourcefile}{#2}}%

\newcommand*{\gencsopt}[3][]{%
  \texorpdfstring%
  {%
    \optfmt{#3\ifblank{#1}{}{\dequals#1}}%
    \index{#2@\csfmt{#2}!#3@\optfmt{#3}}%
  }%
  {#3\ifblank{#1}{}{=#1}}%
}

\newcommand*{\field}[1]{%
 \texorpdfstring
 {\fieldfmt{#1}%
   \index{fields!#1@\fieldfmt{#1}}%
   \index{#1 field@\fieldfmt{#1} field}%
 }%
 {#1}%
}

\newcommand*{\entry}[1]{%
 \texorpdfstring
 {\entryfmt{#1}%
   \index{entry types!#1@\entryfmt{#1}}%
   \index{#1 entry type@\entryfmt{#1} entry type}%
 }%
 {#1}%
}

\newcommand*{\atentry}[1]{%
 \texorpdfstring
 {\entryfmt{@#1}%
   \index{entry types!#1@\entryfmt{#1}}%
   \index{#1 entry type@\entryfmt{#1} entry type}%
 }%
 {@#1}%
}

\newrobustcmd{\longswitch}{\string-{}\string-}

\newcommand*{\longargfmt}[1]{%
 \texorpdfstring{\texttt{\longswitch #1}}%
 {\string-\string-#1}%
}

\newcommand*{\shortargfmt}[1]{%
 \texorpdfstring{\texttt{\string-#1}}%
 {\string-#1}%
}

\newcommand*{\longarg}[1]{%
  \texorpdfstring
  {\longargfmt{#1}\index{command line options!#1@\longargfmt{#1}}}%
  {\string-\string-#1}%
}

\newcommand*{\shortarg}[1]{%
  \texorpdfstring
  {\shortargfmt{#1}\index{command line options!#1@\shortargfmt{#1}}}%
  {\string-#1}%
}

\newenvironment{definition}%
 {\begin{flushleft}\ttfamily\ignorespaces}%
 {\end{flushleft}}

\title{\appfmt{bib2gls}: a command line application to convert
\filefmt{.bib} files to a \filefmt{glossaries-extra.sty} resource file}
\author{Nicola Talbot}
\date{\DTMusedate{moddate}}

\makeatletter
\begingroup
 \let\texorpdfstring\@secondoftwo
 \DTMsetstyle{pdf}
 \protected@edef\x{\endgroup
   \noexpand\hypersetup{%
   pdftitle={\@title},
   pdfauthor={\@author}}%
   \noexpand\pdfinfo{/CreationDate (\DTMuse{creation})
   /ModDate (\DTMuse{moddate})
   }%
 }\x

\newcommand*{\labelarg}[1]{%
 {\def\@currentlabelname{\protect\longarg{#1}}\label{arg.#1}}%
}
\newcommand*{\argref}[1]{\nameref{arg.#1}}

\newcommand*{\labelopt}[1]{%
 {\def\@currentlabelname{\protect\optfmt{#1}}\label{opt.#1}}%
}
\newcommand*{\optref}[2][]{\nameref{opt.#2}\ifblank{#1}{}{\optfmt{\dequals#1}}}

\newcommand*{\labelatentry}[1]{%
 {\def\@currentlabelname{\protect\atentry{#1}}\label{at.#1}}%
}
\newcommand*{\atentryref}[1]{\nameref{at.#1}}

\newcommand*{\labelcs}[1]{%
 {\def\@currentlabelname{\protect\cs{#1}}\label{cs.#1}}%
}
\newcommand*{\csref}[1]{\nameref{cs.#1}}

\makeatother

\begin{document}
\maketitle
\pagenumbering{alph}
\thispagestyle{empty}

\begin{abstract}
The \bibgls\ command line application can be used to extract
glossary information stored in a \filefmt{.bib} file and convert it
into glossary entry definition commands that can be read using
\styfmt{glossaries-extra}'s \csfmt{glsxtrresourcefile} command. When used
in combination with the \optfmt{record} package option, \bibgls\
can select only those entries that have been used in the document,
as well as any dependent entries, which reduces the \TeX\ resources
required by not defining unwanted entries.

Since \bibgls\ can also sort and collate the recorded locations
present in the \filefmt{.aux} file, it can simultaneously by-pass the
need to use \appfmt{makeindex} or \appfmt{xindy}, although \bibgls\ 
can be used together with an external indexing application if required. (For
example, if a custom \appfmt{xindy} rule is needed.)

Note that \bibgls\ is a Java application, so it requires the
Java Runtime Environment (at least JRE~7). Additionally,
\styfmt{glossaries-extra} must be at least version 1.12.
\end{abstract}

\frontmatter

\tableofcontents

\mainmatter
\chapter{Introduction}

If you have extensively used the \styfmt{glossaries} or
\styfmt{glossaries-extra} package, you may have found yourself
creating a large \ext{tex} file containing many definitions that
you frequently use in documents. This file can then simply be
loaded using \cs{input} or \cs{loadglsentries}, but a large file
like this can be difficult to maintain and if the document only
actually uses a small proportion of those entries, the document
build is unnecessarily slow due to the time and resources taken on
defining the unwanted entries.

The aim of \bibgls\ is to allow the entries to be stored in a
\ext{bib} file, which can be maintained using a reference system
such as JabRef. The document build process
can now be analogous to that used with \app{bibtex} (or
\appfmt{biber}), where only those entries that have been recorded in the
document (and possibly their dependent entries) will be extracted
from the \ext{bib} file.

Note that \bibgls\ requires the extension package
\sty{glossaries-extra} and can't be used with just the base
\sty{glossaries} package, since it requires some of the extension
commands. See the \sty{glossaries-extra} user manual for information
on the differences between the basic package and the extended
package, as some of the default settings are different.

\section{Security}

\TeX\ distributions come with two security settings
\texttt{openin\_any} and \texttt{openout\_any} that, respectively,
govern read and write file access (in addition to the operating
system's file permissions). \bibgls\ uses \app{kpsewhich} to
determine these values and honours them.

\section{Localisation}

The messages produced by \bibgls\ are fetched from a resource file
called \metafilefmt{bib2gls-}{lang}{.xml}, where \meta{lang} is a
valid IETF language tag.

The appropriate file is searched for in the following order:
\begin{enumerate}
\item \meta{lang} exactly matches the operating system's locale.
For example, my locale is \texttt{en-GB}, so \bibgls\ will first search
for \filefmt{bib2gls-en-GB.xml}. This file doesn't exist, so it will
try again.

\item If the operating system's locale has an associated script, the
next try is with \meta{lang} set to \meta{lang
code}\texttt{-}\meta{script} where \meta{lang code} is the two
letter ISO language code and \meta{script} is the script code.
For example, if the operating system's locale is \texttt{sr-RS-Latn}
then \bibgls\ will search for \filefmt{bib2gls-sr-Latn.xml} if
\filefmt{bib2gls-sr-RS-Latn.xml} doesn't exist.

\item The final attempt is with \meta{lang} set to just the two
letter ISO language code. For example, \filefmt{bib2gls-en-GB.xml}.
\end{enumerate}

If there is no match, \bibgls\ will fallback on the English resource file
\file{bib2gls-en.xml}.

Note that if you use the \optref[true]{loc-prefix} option, the
textual labels (\qt{Page} and \qt{Pages} in English) will be taken
from the resource file. In the event that the loaded resource file
doesn't match the document language, you will have to manually set
the correct translation (in English, this would be
\optfmt{loc-prefix\dequals\marg{Page,Pages}}).

Currently only \file{bib2gls-en.xml} exists as my language skills aren't up
to translating it. Any volunteers who want to provide other language
resource files would be much appreciated.

\section{Manual Installation}

If you are unable to install \bibgls\ through your \TeX\ package
manager, you can install manually using the instructions below.
Replace \meta{TEXMF} with the path to your local or home TEXMF tree 
(for example, \filefmt{~/texmf}).

Copy the files provided to the following locations:
\begin{itemize}
\item \meta{TEXMF}\filefmt{/scripts/bib2gls/bib2gls.jar}
\item \meta{TEXMF}\filefmt{/scripts/bib2gls/texparserlib.jar}
\item \meta{TEXMF}\filefmt{/scripts/bib2gls/resources/bib2gls-en.xml}
\item \meta{TEXMF}\filefmt{/doc/support/bib2gls/bib2gls.pdf}
\end{itemize}

If you are using a Unix-like system, there's also a bash script
provided called \file{bib2gls.sh}. Either copy it directly to
somewhere on your path without the \ext{sh} extension. For
example:
\begin{verbatim}
cp bib2gls.sh ~/bin/bib2gls
\end{verbatim}
or copy the file to
\meta{TEXMF}\filefmt{/scripts/bib2gls/bib2gls.sh} and create a
symbolic link to it called just \filefmt{bib2gls} from somewhere on
your path. For example:
\begin{verbatim}
cp bib2gls.sh ~/texmf/scripts/bib2gls/
cd ~/bin
ln -s ~/texmf/scripts/bib2gls/bib2gls.sh
\end{verbatim}

Windows users can create a \ext{bat} file that works in a
similar way to the bash script. To do this, create a file called
\file{bib2gls.bat} that contains the following:
\begin{verbatim}
@ECHO OFF
FOR /F %%I IN ('kpsewhich --progname=bib2gls --format=texmfscripts
bib2gls.jar') DO SET JARPATH=%%I
java -Djava.locale.providers=CLDR,JRE -jar "%JARPATH%" %*
\end{verbatim}
Save this file to somewhere on your system's path.

You may need to refresh \TeX's database to ensure that
\app{kpsewhich} can find the \ext{jar} file.

To test that the application has been successfully installed, open a
command prompt or terminal and run the following command:
\begin{verbatim}
bib2gls --version
\end{verbatim}
This should display the version information.

\chapter{Command Line Options}
\label{sec:switches}

\edef\resetsecnumdepth{\noexpand\setcounter{secnumdepth}{\arabic{secnumdepth}}}
\setcounter{secnumdepth}{0}

The syntax of \bibgls\ is:
\begin{alltt}
bib2gls \oargm{options} \meta{filename}
\end{alltt}
where \meta{filename} is the name of the \ext{aux} file. (The
extension may be omitted.) Only one \meta{filename} is permitted.

Available options are listed below.

\section{\longarg{help} (or \shortarg{h})}
\labelarg{help}

Display the help message and quit.

\section{\longarg{version} (or \shortarg{v})}
\labelarg{version}

Display the version information and quit.

\section{\longarg{debug} \oargm{n}}
\labelarg{debug}

Switch on debugging mode. If \meta{n} is present, it must be a
non-negative integer indicating the debugging level. If omitted 1 is
assumed. This option also switches on the verbose mode. A value of 0
is equivalent to \longargfmt{no-debug}.

Note that multiple instances of this switch in a single invocation
can cause some confusion as \bibgls\ performs a quick parse of the
arguments for the first instance of \longarg{debug} or
\longarg{nodebug} or \longarg{silent} before the language resource
file is loaded. Any subsequent use of the switch will be picked up
on the full parse after the language resource file has been loaded.

\section{\longarg{no-debug} (or \longarg{nodebug})}
\labelarg{no-debug}

Switches off the debugging mode.

\section{\longarg{verbose}}
\labelarg{verbose}

Switches on the verbose mode. This writes extra information to the
terminal and transcript file.

\section{\longarg{no-verbose} (or \longarg{noverbose})}
\labelarg{no-verbose}

Switches off the verbose mode. This is the default behaviour.
Some messages are written to the terminal. To completely suppress
all messages (except errors), switch on the silent mode.
For additional information messages, switch on the verbose mode.

\section{\longarg{silent}}
\labelarg{silent}

Suppresses all messages except for errors that would normally be
written to the terminal. Warnings and informational messages are
written to the transcript file, which can be inspected afterwards.

\section{\longarg{log-file} \meta{filename} (or \shortarg{t}
\meta{filename})}
\labelarg{log-file}

Sets the name of the transcript file. By default, the name is the
same as the \ext{aux} file but with a \ext{glg} extension. Note that
if you use \bibgls\ in combination with \app{xindy} or
\app{makeindex}, you will need to change the transcript file name to
prevent interference.

\section{\longarg{dir} \meta{dirname} (or \shortarg{d}
\meta{dirname})}
\labelarg{dir}

By default \bibgls\ assumes that the output files should be written
in the current working directory. The input \ext{.bib} files are assumed to be
either in the current working directory or on \TeX's path (in which
case \app{kpsewhich} will be used to find them).

If your \ext{aux} file isn't in the current working directory (for
example, you have run \TeX\ with \shortargfmt{output-directory})
then you need to take care how you invoke \bibgls.

Suppose I have a file called \filefmt{test-entries.bib} that
contains my entry definitions and a document called
\filefmt{mydoc.tex} that selects the \ext{bib} file using:
\begin{verbatim}
\GlsXtrLoadResources[src={test-entries}]
\end{verbatim}
If I compile this document using
\begin{verbatim}
pdflatex -output-directory tmp mydoc
\end{verbatim}
then the auxiliary file \filefmt{mydoc.aux} will be written to the
\filefmt{tmp} sub-directory. The resource information is listed in
the \ext{aux} file as
\begin{verbatim}
\glsxtr@resource{src={test-entries}}{mydoc}
\end{verbatim}
If I run \bibgls\ from the \filefmt{tmp} directory, then it won't
be able to find the \filefmt{test-entries.bib} file.

If I run \bibgls\ from the same directory as \filefmt{mydoc.tex}
using
\begin{verbatim}
bib2gls tmp/mydoc
\end{verbatim}
then the \ext{aux} file is found and the transcript file is
\filefmt{tmp/mydoc.glg} (since the default is the same as the
\ext{aux} file but with the extension changed to \ext{glg}) but the
output file \filefmt{mydoc.glstex} will be written to the current
directory.

This works fine from \TeX's point of view. The
\ext{glstex} file can be picked up by \cs{GlsXtrLoadResources} but
it may be that you'd rather the \ext{glstex} file was tidied away
into the \filefmt{tmp} directory along with all the other files.
In this case you need to invoke \bibgls\ with the \longarg{dir} or
\shortarg{d} option:
\begin{verbatim}
bib2gls -d tmp mydoc
\end{verbatim}

\section{\longarg{mfirstuc-protection} (or \shortarg{u})}
\labelarg{mfirstuc-protection}

Commands like \cs{Gls} use \cs{makefirstuc} provided by the
\sty{mfirstuc} package. This command has limitations and one of the
things that can break it is the use of a referencing command 
at the start of its argument. The \sty{glossaries-extra} package has
more detail about the problem in the \qt{Nested Links} section of
the user manual. If a glossary field starts with one of these
problematic commands, the recommended method (if the command can't
be replaced) is to insert an empty group in front of it.

For example, the following definition
\begin{verbatim}
\newabbreviation{shtml}{shtml}{\glsps{ssi} enabled \glsps{short}{html}}
\end{verbatim}
will cause a problem for \verb|\Gls{shtml}| on first use.

The above example, would be written in a \ext{bib} file as:
\begin{verbatim}
@abbreviation{shtml,
  short={shtml},
  long={\glsps{ssi} enabled \glsps{html}}
}
\end{verbatim}

With the \longarg{mfirstuc-protection} switch on (the default
behaviour), \bibgls\ will automatically insert an empty group at the
start of the \field{long} field to guard against this problem. A
warning will be written to the transcript.

\section{\longarg{no-mfirstuc-protection}}
\labelarg{no-mfirstuc-protection}

Switches off the \sty{mfirstuc} protection mechanism described
above.

\section{\longarg{mfirstuc-math-protection}}
\labelarg{mfirstuc-math-protection}

This works in the same way as \argref{mfirstuc-protection} but
guards against fields starting with inline maths
(\verb|$|\ldots\verb|$|). For example, if the \field{name} field
starts with \verb|$x$| and the glossary style automatically tries to
convert the first letter of the name to upper case, then this will
cause a problem.

With \longarg{mfirstuc-math-protection} set, \bibgls\ will
automatically insert an empty group at the start of the field and
write a warning in the transcript. This setting is on by default.

\section{\longarg{no-mfirstuc-math-protection}}
\labelarg{no-mfirstuc-math-protection}

Switches off the above.

\section{\longarg{nested-link-check} \texttt{none}\argor\meta{list}}
\labelarg{nested-link-check}

By default, \bibgls\ will parse certain fields for potential nested links.
(See the section \qt{Nested Links} in the \sty{glossaries-extra}
user manual.)

The default set of fields to check are: \field{name}, \field{text},
\field{plural}, \field{first}, \field{firstplural}, \field{long},
\field{longplural}, \field{short}, \field{shortplural} and
\field{symbol}.

You can change this set of fields using
\longarg{nested-link-check} \meta{value} where \meta{value} may be
\optfmt{none} (don't parse any of the fields) or a comma-separated
list of fields to be checked.

\section{\longarg{no-nested-link-check}}
\labelarg{no-nested-link-check}

Equivalent to \longarg{nested-link-check} \optfmt{none}.

\section{\longarg{shortcuts} \meta{value}}
\labelarg{shortcuts}

Some entries may reference another entry within a field, using
commands like \cs{gls}, so \bibgls\ parses the fields for these
commands to determine dependent entries to allow them to be selected
even if they haven't been used within the document.

The \styopt{shortcuts} package option provided by
\styfmt{glossaries-extra} defines various synonyms, such as \cs{ac}
which is equivalent to \cs{gls}. By default the value of the
\styopt{shortcuts} option will be picked up by \bibgls\ when parsing the
\ext{aux} file. This then allows \bibgls\ to additionally search for
those shortcut commands while parsing the fields.

You can override the \styopt{shortcuts} setting using
\longarg{shortcuts} \meta{value} (where \meta{value} may take
any of the allowed values for the \styopt{shortcuts} package option), 
but in general there is little need to use this switch.

\section{\longarg{map-format} \meta{key=val list}}
\labelarg{map-format}

The \styfmt{glossaries} package provides a mapping between field
tags used in control sequences, such as \cs{glsuseri} or
\cs{glsdesc}, and the corresponding key used when defining the
entry, such as \field{user1} or \field{description}. This
information is written to the auxiliary file so that \bibgls\ can
find the relevant commands while parsing the field values for
dependent entries.

If you have defined some extra fields using \cs{glsaddkey}\margm{key} with
analogous commands in the form \csfmt{gls}\meta{field}, then you can
provide a new mapping so that \bibgls\ can find any instances of
\csfmt{gls}\meta{field} when parsing the entries provided in the
\ext{bib} file.

If you have multiple mappings, you can either use a single
\longarg{map-format} with a comma separated list of
\meta{key}=\meta{field} values or you can have multiple instances of 
\longarg{map-format} \meta{key}\texttt{=}\meta{field}.

\section{\longarg{group}}
\labelarg{group}

The \styopt{record} package option automatically creates a new field
called \field{group}. If the \longarg{group} switch is used,
\bibgls\ will try to determine the letter group for each entry and
add it to the \field{group} field. This value will be picked up by
\cs{printunsrtglossary} if letter group headings are required. If
you're not using a glossary style that displays the group headings,
there's no need to use this switch.

The default is \argref{no-group}.

\section{\longarg{no-group}}
\labelarg{no-group}

Don't use the \field{group} field. (Default.)

\chapter{\ext{bib} Format}
\label{sec:bib}

\bibgls\ recognises certain entry types. Any unrecognised types will
be ignored and a warning will be written to the transcript file.
Entries are defined in the usual \ext{bib} format:
\begin{alltt}
@\meta{entry-type}\marg{\meta{id},
  \meta{field-name-1} = \margm{text},
  ...
  \meta{field-name-n} = \margm{text}
}
\end{alltt}
where \meta{entry-type} is the entry type (listed below),
\meta{field-name-1} are the field names (same as the keys available
with \cs{newglossaryentry}) and \meta{id} is a unique label. The
label can't contain any spaces or commas. 

\bibgls\ allows you to insert prefixes to the labels when the data
is read through the \csopt{label-prefix} option. Remember to use
these prefixes when you reference the entries in the document, but
don't include them when you reference them in the \ext{bib} file.
There are some special prefixes that have a particular meaning to
\bibgls: \idprefix{dual} and \idprefix{ext\meta{n}} where
\meta{n} is a positive integer.  In the first case,
\idprefix{dual} references the dual element of a dual entry (see
\atentry{dualentry}). This prefix will be replaced by the value of
the \csopt{dual-prefix} option. The \idprefix{ext\meta{n}} prefix
is used to reference an entry from a different set of resources
(loaded by another \cs{glsxtrresourcefile} command). This prefix is
replaced by the corresponding element of the list supplied by
\csopt{ext-prefixes}.

Avoid non-ASCII characters in the \meta{id} if your document uses the 
\sty{inputenc} package.
You can set the character encoding in the \ext{bib} file using:
\begin{alltt}
\% Encoding: \meta{encoding-name}
\end{alltt}
where \meta{encoding-name} is the name of the character encoding.
For example:
\begin{verbatim}
% Encoding: UTF-8
\end{verbatim}
You can also set the encoding using the \csopt{charset} option, but
it's more efficient to put the encoding comment at the start of the
\ext{bib} file (otherwise \bibgls\ has to search the entire file for
it).

Each entry type may have required fields and optional fields. For the
optional fields, any key recognised by \cs{newglossaryentry} may be
used as a field. However, note that if you add
any custom keys using \cs{glsaddkey} or \cs{glsaddstoragekey}, those
commands must be placed before the first use of
\cs{glsxtrresourcefile} (or \cs{GlsXtrLoadResources}).
Any unrecognised fields will be ignored.

This is more convenient than using \cs{loadglsentries}, which
requires all the keys used in the file to be defined, regardless of
whether or not you actually need them in the document.

If an optional field is missing, \bibgls\ will try to fallback on
another value. The actual fallback value depends in the entry type.

Other entries can be cross-referenced using the \field{see} field or
by using commands like \cs{gls} or \cs{glsxtrp} in any of the
recognised fields. These will automatically be selected if the
\csopt{selection} setting includes dependencies, but you may need to
rebuild the document to ensure the location lists are correct.

The standard \atentry{string}\labelatentry{string} and 
\atentry{preamble}\labelatentry{preamble} types are recognised, so you
can do, for example:
\begin{verbatim}
@string{ssi={server-side includes}}
@string{html={hypertext markup language}}

@abbreviation{shtml,
  short="shtml",
  long= ssi # " enabled " # html,
  see={ssi,html}
}

@abbreviation{html,
  short ="html",
  long  = html
}

@abbreviation{ssi,
  short="ssi",
  long = ssi
}

@preamble{"\providecommand{\mtx}[1]{\boldsymbol{#1}}"}

@entry{matrix,
  name={matrix},
  plural={matrices},
  description={rectangular array of values, denoted $\mtx{M}$}
}
\end{verbatim}

\section{\atentry{entry}}
\labelatentry{entry}

Regular terms are defined by the \atentry{entry} field (such as in
the \texttt{matrix} example above). This requires the
\field{description} field and either \field{name} or \field{parent}.

For example:
\begin{verbatim}
@preamble{"\providecommand{\seealsoname}{see also}
\providecommand{\mtx}[1]{\boldsymbol{#1}}"}

@entry{matrix,
  name={matrix},
  plural={matrices},
  description={rectangular array of values, denoted \gls{M}},
  see={[\seealsoname]{vector}}
}

@entry{M,
  sort={M},
  name={\ensuremath{M}},
  description={a \gls{matrix}}
}

@entry{vector,
  name = "vector",
  description = {column or row of values, denoted \gls{v}},
  see={[\seealsoname]{matrix}}
}

@entry{v,
  sort={v},
  name={\ensuremath{\vec{v}}},
  description={a \gls{vector}}
}
\end{verbatim}

If the \field{sort} field is omitted, \bibgls\ will sort according
to the \field{name} field (or the \field{parent} field if
\field{name} is missing).

Terms defined using \atentry{entry} will be written to the output
file using the command \csref{bibglsnewentry}.

\section{\atentry{symbol}}
\labelatentry{symbol}

The \entry{symbol} entry type is much like \entry{entry}, but it's
designed specifically for symbols, so in the previous example, the
\texttt{M} and \texttt{v} terms would be better defined using the
\atentry{symbol} entry type instead.

Again the required fields are again \field{description} and either
\field{name} or \field{parent}, but in this case if the \field{sort} field
is omitted, the entry's label is used instead. For example:
\begin{verbatim}
@symbol{M,
  name={\ensuremath{M}},
  description={a \gls{matrix}}
}

@symbol{v,
  name={\ensuremath{\vec{v}}},
  description={a \gls{vector}}
}
\end{verbatim}

Terms defined using \atentry{symbol} will be written to the output
file using the command \csref{bibglsnewsymbol}.

\section{\atentry{number}}
\labelatentry{number}

The \entry{number} entry type is like \entry{symbol}, but it's for numbers.
Terms defined using \atentry{number} will be written to the output
file using the command \csref{bibglsnewnumber}.

\section{\atentry{term}}
\labelatentry{term}

The \entry{term} entry type is designed for entries that don't have
a description. Only the label is required. If \field{name} is
omitted, it's assumed to be the same as the label. However, this
means that if the name contains any characters that can't be used in
the label, you will need the \field{name} field. If the \field{sort}
field is omitted, \bibgls\ will use the \field{name} field instead,
if present, otherwise it will use the label.

Example:
\begin{verbatim}
@term{duck}

@term{goose,plural={geese}}

@term{facade,name={fa\c{c}ade}}
\end{verbatim}

Terms defined using \atentry{term} will be written to the output
file using the command \csref{bibglsnewterm}.

\section{\atentry{abbreviation}}
\labelatentry{abbreviation}

The \entry{abbreviation} entry type is designed for abbreviations.
The required fields are \field{short} and \field{long}. If the
\field{sort} key is missing, \bibgls\ will use the value of the
\field{short} field.

Note that you must set the abbreviation style before loading the
resource file to ensure that the abbreviations are defined
correctly, however \bibgls\ has no knowledge of the abbreviation
style so it doesn't know if the \field{description} field must be
included or if the default \field{sort} value isn't simply the value
of the \field{short} field.

You can instruct \bibgls\ to use a specific field for the sort value
using \optref{sort-field} and you can also tell \bibgls\ to ignore
certain fields using the \optref{ignore-fields}, so you can
include a \field{description} field if sometimes you need it and
instruct \bibgls\ to ignore it when you don't want it.

For example:
\begin{verbatim}
@abbreviation{html,
  short ="html",
  long  = {hypertext markup language},
  description={a markup language for creating web pages}
}
\end{verbatim}
If you want the \abbrstyle{long-noshort-desc} style, then you can put
the following in your document (where the \ext{bib} file is called
\filefmt{entries-abbrv.bib}):
\begin{verbatim}
\setabbreviationstyle{long-noshort-desc}
\GlsXtrLoadResources[src={entries-abbrv.bib},sort-field={long}]
\end{verbatim}
Whereas, if you want the \abbrstyle{long-short} style, then you can
instead do:
\begin{verbatim}
\setabbreviationstyle{long-short}
\GlsXtrLoadResources[src={entries-abbrv.bib},ignore-fields={description}]
\end{verbatim}

Terms defined using \atentry{abbreviation} will be written to the output
file using the command \csref{bibglsnewabbreviation}.

\section{\atentry{acronym}}
\labelatentry{acronym}

The \entry{acronym} entry type is like \entry{abbreviation} except that
the term is written to the output file using the command
\csref{bibglsnewacronym}.

\section{\atentry{dualentry}}
\labelatentry{dualentry}

The \entry{dualentry} entry type is similar to \entry{entry} but
actually defines two entries: the primary entry and the dual entry.
The dual entry contains the same information as the primary entry
but some of the fields are swapped around. The dual entry is given
the prefix set by the \csopt{dual-prefix} option.

By default, the \field{name} and \field{description} fields and the
\field{plural} and \field{descriptionplural} fields are swapped.

For example:
\begin{verbatim}
@dualentry{child,
  name={child},
  plural={children},
  description={enfant}
}
\end{verbatim}
Is like
\begin{verbatim}
@entry{child,
  name={child},
  plural={children},
  description={enfant}
  descriptionplural={enfants}
}

@entry{dual.child,
  description={child},
  descriptionplural={children},
  name={enfant}
  plural={enfants}
}
\end{verbatim}
where \idprefix{dual} is replaced by the value of the
\csopt{dual-prefix} option. However, instead of defining the entries
with \csfmt{bibglsnewentry} both the primary and dual entries are
defined using \csref{bibglsnewdualentry}. The \field{category} 
and \field{type} fields can be set for the dual entry using the
\csopt{dual-category} and \csopt{dual-type} options.

If \csopt[combine]{dual-sort} then the dual entries will be sorted
along with the primary entries, otherwise the \csopt{dual-sort}
indicates how to sort the dual entries and the dual entries will be
appended to the end of the \ext{glstex} file. The
\csopt{dual-sort-field} determines what field to use for the sort
value if the dual entries should be sorted separately.

For example:
\begin{verbatim}
\newglossary*{english}{English}
\newglossary*{french}{French}

\GlsXtrLoadResources[
 src           = {entries-dual},% data in entries-dual.bib
 type          = {english},% put primary entries in glossary 'english'
 dual-type     = {french},% put dual entries in glossary 'french'
 category      = {dictionary},% set the primary category to 'dictionary'
 dual-category = {dictionary},% set the dual category to 'dictionary'
 sort          = {en},% sort primary entries according to language 'en'
 dual-sort     = {fr}% sort dual entries according to language 'fr'
]
\end{verbatim}

\chapter{Resource File Options}
\label{sec:resourceopts}

Make sure that you load \sty{glossaries-extra} with the
\styopt{record} package option. This ensures that \bibgls\ can pick
up the required information from the \ext{aux} file. (You may omit
this option if you use \optref[all]{selection} and you don't require
the location lists.)

The \ext{glstex} resource files created by \bibgls\ are loaded in
the document using
\begin{definition}
\cs{glsxtrresourcefile}\oargm{options}\margm{filename}
\end{definition}
where \meta{filename} is the name of the resource file without the
\ext{glstex} extension. There's a shortcut command that uses
\cs{jobname} as the \meta{filename}:
\begin{definition}
\cs{GlsXtrLoadResources}\oargm{options}
\end{definition}
This is equivalent to
\begin{definition}
\cs{glsxtrresourcefile}\oargm{options}\marg{\cs{jobname}}
\end{definition}

You can have multiple \cs{glsxtrresourcefile} commands within your
document, but each \meta{filename} must be unique. For this reason, 
\cs{GlsXtrLoadResources} can only be used once, otherwise \LaTeX\
would attempt to load \cs{jobname}\ext{glstex} multiple times.
\bibgls\ checks for non-unique file names.

The optional argument \meta{options} is a comma-separated 
\meta{key}=\meta{value} list. Allowed options are listed below.
The option list applies only to that specific
\meta{filename}\ext{glstex} and are not carried over to the next
instance of \cs{glsxtrresourcefile}.

If you have multiple \ext{bib} files you can either select them all
using \optref{src} in a single \cs{glsxtrresourcefile} call, if they
all require the same settings, or you can load them separately with
different settings applied.

For example, if the files \filefmt{entries-terms.bib} and
\filefmt{entries-symbols.bib} have the same settings:
\begin{verbatim}
\GlsXtrLoadResources[src={entries-terms,entries-symbols}]
\end{verbatim}
Alternatively, if they have different settings:
\begin{verbatim}
\GlsXtrLoadResources[src={entries-terms}]
\glsxtrresourcefile[sort=use]{entries-symbols}
\end{verbatim}

\section{\csopt{src}=\margm{list}}
\labelopt{src}

If the \csopt{src} option is omitted, the \ext{bib} file is assumed
to be \meta{filename}\ext{bib}. For example:
\begin{verbatim}
\glsxtrresourcefile{entries-symbols}
\end{verbatim}
Indicates that \bibgls\ needs to read the file
\filefmt{entries-symbols.bib} and create the file
\filefmt{entries-symbols.glstex}. If the \ext{bib} file is
different or if you have multiple \ext{bib} files, you need to use
the \csopt{src} option.

The value must be a comma-separated list of the required \ext{bib}
files. These may either be in the current working directory or on
\TeX's path (in which case \app{kpsewhich} will be used to find them).
The \ext{bib} extension may be omitted. Remember that if \meta{list}
contains multiple files it must be grouped to protect the comma
from the \meta{options} list.

For example
\begin{verbatim}
\GlsXtrLoadResources[src={entries-terms,entries-symbols}]
\end{verbatim}
indicates that \bibgls\ must read the files
\filefmt{entries-terms.bib} and \filefmt{entries-symbols.bib} and
create the file obtained from \cs{jobname}\ext{glstex}.

\section{\csopt{sort}=\meta{value}}
\labelopt{sort}

The \csopt{sort} key indicates how entries should be sorted. The
\meta{value} may be one of:
\begin{itemize}
\item \texttt{locale}: sort the entries according to the operating
system's locale (default).
\item \texttt{none} (or \texttt{unsrt}): don't sort the entries.
\item \texttt{use}: sort in order of use. (This order is determined
by the records written to the \ext{aux} file by the \styopt{record}
package option.)
\item \texttt{letter-case}: case-sensitive letter sort.
\item \texttt{letter-nocase}: case-insensitive letter sort.
\item \meta{lang tag}: sort according to the rules of the locale
given by the IETF language tag \meta{lang tag}.
\end{itemize}

For example:
\begin{verbatim}
\GlsXtrLoadResources[src={english-terms},sort={en}]
\glsxtrresourcefile[sort={de-1996}]{german-terms}
\end{verbatim}

\section{\csopt{sort-field}=\margm{label}}
\labelopt{sort-field}

The \csopt{sort-field} key indicates which field provides the sort
value. The default is the \field{sort} field. For example
\begin{verbatim}
\GlsXtrLoadResources[src={entries-terms},sort-label=category,sort=letter-case]
\end{verbatim}
This sorts the entries according to the \field{category} field using
a case-sensitive letter comparison.

\section{\csopt{ignore-fields}=\margm{list}}
\labelopt{ignore-fields}

The \csopt{ignore-fields} key indicates that you want \bibgls\ to
skip the fields listed in supplied the comma-separated \meta{list} of field
labels. Remember that unrecognised fields will always be skipped.

For example, suppose my \ext{bib} file contains
\begin{verbatim}
@abbreviation{html,
  short ="html",
  long  = {hypertext markup language},
  description={a markup language for creating web pages},
  see={[see also]xml}
}
\end{verbatim}
but I want to use the \abbrstyle{short-long} style and I don't want
the cross-referenced term, then I can use
\csopt[\marg{see,description}]{ignore-fields}.

\section{\csopt{type}=\margm{label}}
\labelopt{type}

The \csopt{type} option indicates that all the entries must have the
\field{type} field set to \meta{label}.

For example:
\begin{verbatim}
\usepackage[record,symbols]{glossaries-extra}

\GlsXtrLoadResources[src={entries-symbols},type=symbols]
\end{verbatim}

\section{\csopt{category}=\margm{label}}
\labelopt{category}

The \csopt{category} option indicates that all the entries must have the
\field{category} field set to \meta{label}.

\section{\csopt{label-prefix}=\margm{tag}}
\labelopt{label-prefix}

The \csopt{label-prefix} option prepend \meta{tag} to each
entry's label. If you use this option, make sure that any
cross-references are limited to the \ext{bib} files selected by the
\optref{src} option.

For example, if \filefmt{entries-terms.bib} has entries that
reference terms in \filefmt{entries-symbols.bib}, then
\csopt{label-prefix} can only be used if both \ext{bib} files are
selected with 
\texttt{\csopt{src}\dequals\marg{entries-terms\dcomma
entries-symbols}}. They can be selected with separate instances of
\cs{glsxtrresourcefile} (unless \csopt{label-prefix} is set to the
same value in both cases).

Remember to use this prefix when you reference the terms in the
document when you use commands like \cs{gls}.

\section{\csopt{loc-prefix}=\margm{value}}
\labelopt{loc-prefix}

\section{\csopt{dual-prefix}=\margm{value}}
\labelopt{dual-prefix}

\section{\csopt{dual-type}=\margm{value}}
\labelopt{dual-type}

\section{\csopt{dual-category}=\margm{value}}
\labelopt{dual-category}

\section{\csopt{dual-sort}=\margm{value}}
\labelopt{dual-sort}

\section{\csopt{dual-sort-field}=\margm{value}}
\labelopt{dual-sort-field}

\section{\csopt{ext-prefixes}=\margm{list}}
\labelopt{ext-prefixes}

\section{\csopt{selection}=\meta{value}}
\labelopt{selection}

\chapter{Provided Commands}
\label{sec:bibglscs}

When \bibgls\ writes the entries to the output file, instead of
directly using commands like \cs{newglossaryentry}, it provides
its own commands defined with \cs{providecommand}. This means that
you can customize the way the entries are defined by providing your
own commands before the \ext{glstex} files are loaded.

\section{\cs{bibglsnewentry}}
\labelcs{bibglsnewentry}

\section{\cs{bibglsnewsymbol}}
\labelcs{bibglsnewsymbol}

\section{\cs{bibglsnewnumber}}
\labelcs{bibglsnewnumber}

\section{\cs{bibglsnewterm}}
\labelcs{bibglsnewterm}

\section{\cs{bibglsnewacronym}}
\labelcs{bibglsnewacronym}

\section{\cs{bibglsnewabbreviation}}
\labelcs{bibglsnewabbreviation}

\section{\cs{bibglsnewdualentry}}
\labelcs{bibglsnewdualentry}

\resetsecnumdepth

\end{document}
