% arara: pdflatex
% arara: pdflatex
\documentclass[titlepage=false]{scrreprt}

\usepackage[T1]{fontenc}
\usepackage{alltt}
\usepackage{upquote}
\usepackage{etoolbox}
\usepackage{datetime2}
\usepackage[colorlinks]{hyperref}

\DTMsavetimestamp{creation}{2017-01-20T15:39:00Z}
\DTMsavefilemoddate{moddate}{../java/Bib2Gls.java}

\providecommand{\frontmatter}{\clearpage\pagenumbering{roman}}
\providecommand{\mainmatter}{\clearpage\pagenumbering{arabic}}

\newcommand{\bibgls}{\appfmt{bib2gls}}

\newcommand{\qt}[1]{``#1''}

\newcommand*{\filefmt}[1]{\texorpdfstring{\nolinkurl{#1}}{#1}}
\newcommand*{\metafilefmt}[3]{%
  \filefmt{#1}\discretionary{}{}{}\meta{#2}\discretionary{}{}{}\filefmt{#3}%
}

\newcommand*{\appfmt}[1]{\texorpdfstring{\texttt{#1}}{#1}}
\newcommand*{\styfmt}[1]{\texorpdfstring{\textsf{#1}}{#1}}
\newcommand*{\csfmt}[1]{\texorpdfstring{\texttt{\char`\\#1}}{\string\\#1}}
\newcommand*{\optfmt}[1]{\texorpdfstring{\texttt{#1}}{#1}}
\newcommand*{\fieldfmt}[1]{\texorpdfstring{\texttt{#1}}{#1}}
\newcommand*{\entryfmt}[1]{\texorpdfstring{\texttt{#1}}{#1}}

\newcommand*{\argor}{\texorpdfstring{\protect\textbar}{\string\|}}

\newcommand*{\meta}[1]{%
 \texorpdfstring{$\langle$\normalfont{\emph{#1}}$\rangle$}{#1}%
}

\newcommand*{\oarg}[1]{[#1]}
\newcommand*{\oargm}[1]{\oarg{\meta{#1}}}

\newcommand*{\marg}[1]{\texorpdfstring
 {\char`\{#1\char`\}}
 {\string\{#1\string\}}}

\newcommand*{\margm}[1]{\marg{\meta{#1}}}

\newcommand*{\file}[1]{%
 \texorpdfstring
 {\filefmt{#1}\index{#1@\filefmt{#1}}}%
 {#1}%
}

\newcommand*{\ext}[1]{%
 \texorpdfstring
 {\filefmt{.#1}\index{file formats!#1@\filefmt{.#1}}}%
 {.#1}%
}

\newcommand*{\app}[1]{%
 \texorpdfstring
 {\appfmt{#1}\index{#1@\appfmt{#1}}}%
 {#1}%
}

\newcommand*{\sty}[1]{%
  \texorpdfstring
  {\styfmt{#1}\index{#1@\styfmt{#1}}}%
  {#1}%
}

\newcommand*{\cs}[1]{% 
 \texorpdfstring
 {\csfmt{#1}\index{#1@\csfmt{#1}}}
 {\string\\#1}%
}

\newcommand*{\styopt}[2][]{%
  \texorpdfstring%
  {%
    \optfmt{#2\ifblank{#1}{}{=#1}}%
    \index{package options@\optfmt{#2}}%
    \index{#2@\optfmt{#2}}%
  }%
  {#2\ifblank{#1}{}{=#1}}%
}

\newcommand*{\csopt}[3][]{%
  \texorpdfstring%
  {%
    \optfmt{#3\ifblank{#1}}{{=#1}}%
    \index{#2@\csfmt{#2}!#3@\optfmt{#3}}%
  }%
  {#3\ifblank{#1}{}{=#1}}%
}

\newcommand*{\field}[1]{%
 \texorpdfstring
 {\fieldfmt{#1}%
   \index{fields!#1@\fieldfmt{#1}}%
   \index{#1 field@\fieldfmt{#1} field}%
 }%
 {#1}%
}

\newcommand*{\entry}[1]{%
 \texorpdfstring
 {\entryfmt{#1}%
   \index{entry types!#1@\entryfmt{#1}}%
   \index{#1 entry type@\entryfmt{#1} entry type}%
 }%
 {#1}%
}

\newrobustcmd{\longswitch}{\string-{}\string-}

\newcommand*{\longargfmt}[1]{%
 \texorpdfstring{\texttt{\longswitch #1}}%
 {\string-\string-#1}%
}

\newcommand*{\shortargfmt}[1]{%
 \texorpdfstring{\texttt{\string-#1}}%
 {\string-#1}%
}

\newcommand*{\longarg}[1]{%
  \texorpdfstring
  {\longargfmt{#1}\index{command line options!#1@\longargfmt{#1}}}%
  {\string-\string-#1}%
}

\newcommand*{\shortarg}[1]{%
  \texorpdfstring
  {\shortargfmt{#1}\index{command line options!#1@\shortargfmt{#1}}}%
  {\string-#1}%
}


\title{\appfmt{bib2gls}: a command line application to convert
\filefmt{.bib} files to a \filefmt{glossaries-extra.sty} resource file}
\author{Nicola Talbot}
\date{\DTMusedate{moddate}}

\makeatletter
\begingroup
 \let\texorpdfstring\@secondoftwo
 \DTMsetstyle{pdf}
 \protected@edef\x{\endgroup
   \noexpand\hypersetup{%
   pdftitle={\@title},
   pdfauthor={\@author}}%
   \noexpand\pdfinfo{/CreationDate (\DTMuse{creation})
   /ModDate (\DTMuse{moddate})
   }%
 }\x

\newcommand*{\labelarg}[1]{%
 {\def\@currentlabelname{\protect\longarg{#1}}\label{arg.#1}}%
}
\newcommand*{\argref}[1]{\nameref{arg.#1}}

\newcommand*{\labelopt}[1]{%
 {\def\@currentlabelname{\protect\optfmt{#1}}\label{opt.#1}}%
}
\newcommand*{\optref}[2][]{\nameref{opt.#2}\ifblank{#1}{}{\optfmt{=#1}}}
\makeatother

\begin{document}
\maketitle
\pagenumbering{alph}
\thispagestyle{empty}

\begin{abstract}
The \bibgls\ command line application can be used to extract
glossary information stored in a \filefmt{.bib} file and convert it
into glossary entry definition commands that can be read using
\styfmt{glossaries-extra}'s \csfmt{glsxtrresourcefile} command. When used
in combination with the \optfmt{record} package option, \bibgls\
can select only those entries that have been used in the document,
as well as any dependent entries, which reduces the \TeX\ resources
required by not defining unwanted entries.

Since \bibgls\ can also sort and collate the recorded locations
present in the \filefmt{.aux} file, it can simultaneously by-pass the
need to use \appfmt{makeindex} or \appfmt{xindy}, although \bibgls\ 
can be used together with an external indexing application if required. (For
example, if a custom \appfmt{xindy} rule is needed.)

Note that \bibgls\ is a Java application, so it requires the
Java Runtime Environment (at least JRE~7). Additionally,
\styfmt{glossaries-extra} must be at least version 1.12.
\end{abstract}

\frontmatter

\tableofcontents

\mainmatter
\chapter{Introduction}

If you have extensively used the \styfmt{glossaries} or
\styfmt{glossaries-extra} package, you may have found yourself
creating a large \ext{tex} file containing many definitions that
you frequently use in documents. This file can then simply be
loaded using \cs{input} or \cs{loadglsentries}, but a large file
like this can be difficult to maintain and if the document only
actually uses a small proportion of those entries, the document
build is unnecessarily slow due to the time and resources taken on
defining the unwanted entries.

The aim of \bibgls\ is to allow the entries to be stored in a
\ext{bib} file, which can be maintained using a reference system
such as JabRef. The document build process
can now be analogous to that used with \app{bibtex} (or
\appfmt{biber}), where only those entries that have been recorded in the
document (and possibly their dependent entries) will be extracted
from the \ext{bib} file.

Note that \bibgls\ requires the extension package
\sty{glossaries-extra} and can't be used with just the base
\sty{glossaries} package, since it requires some of the extension
commands. See the \sty{glossaries-extra} user manual for information
on the differences between the basic package and the extended
package, as some of the default settings are different.

\section{Security}

\TeX\ distributions come with two security settings
\texttt{openin\_any} and \texttt{openout\_any} that, respectively,
govern read and write file access (in addition to the operating
system's file permissions). \bibgls\ uses \app{kpsewhich} to
determine these values and honours them.

\section{Localisation}

The messages produced by \bibgls\ are fetched from a resource file
called \metafilefmt{bib2gls-}{lang}{.xml}, where \meta{lang} is a
valid IETF language tag.

The appropriate file is searched for in the following order:
\begin{enumerate}
\item \meta{lang} exactly matches the operating system's locale.
For example, my locale is \texttt{en-GB}, so \bibgls\ will first search
for \filefmt{bib2gls-en-GB.xml}. This file doesn't exist, so it will
try again.

\item If the operating system's locale has an associated script, the
next try is with \meta{lang} set to \meta{lang
code}\texttt{-}\meta{script} where \meta{lang code} is the two
letter ISO language code and \meta{script} is the script code.
For example, if the operating system's locale is \texttt{sr-RS-Latn}
then \bibgls\ will search for \filefmt{bib2gls-sr-Latn.xml} if
\filefmt{bib2gls-sr-RS-Latn.xml} doesn't exist.

\item The final attempt is with \meta{lang} set to just the two
letter ISO language code. For example, \filefmt{bib2gls-en-GB.xml}.
\end{enumerate}

If there is no match, \bibgls\ will fallback on the English resource file
\file{bib2gls-en.xml}.

Note that if you use the \optref[true]{loc-prefix} option, the
textual labels (\qt{Page} and \qt{Pages} in English) will be taken
from the resource file. In the event that the loaded resource file
doesn't match the document language, you will have to manually set
the correct translation (in English, this would be
\optfmt{loc-prefix=\marg{Page,Pages}}).

Currently only \file{bib2gls-en.xml} exists as my language skills aren't up
to translating it. Any volunteers who want to provide other language
resource files would be much appreciated.

\section{Manual Installation}

If you are unable to install \bibgls\ through your \TeX\ package
manager, you can install manually using the instructions below.
Replace \meta{TEXMF} with the path to your local or home TEXMF tree 
(for example, \filefmt{~/texmf}).

Copy the files provided to the following locations:
\begin{itemize}
\item \meta{TEXMF}\filefmt{/scripts/bib2gls/bib2gls.jar}
\item \meta{TEXMF}\filefmt{/scripts/bib2gls/texparserlib.jar}
\item \meta{TEXMF}\filefmt{/scripts/bib2gls/resources/bib2gls-en.xml}
\item \meta{TEXMF}\filefmt{/doc/support/bib2gls/bib2gls.pdf}
\end{itemize}

If you are using a Unix-like system, there's also a bash script
provided called \file{bib2gls.sh}. Either copy it directly to
somewhere on your path without the \ext{sh} extension. For
example:
\begin{verbatim}
cp bib2gls.sh ~/bin/bib2gls
\end{verbatim}
or copy the file to
\meta{TEXMF}\filefmt{/scripts/bib2gls/bib2gls.sh} and create a
symbolic link to it called just \filefmt{bib2gls} from somewhere on
your path. For example:
\begin{verbatim}
cp bib2gls.sh ~/texmf/scripts/bib2gls/
cd ~/bin
ln -s ~/texmf/scripts/bib2gls/bib2gls.sh
\end{verbatim}

Windows users can create a \ext{bat} file that works in a
similar way to the bash script. To do this, create a file called
\file{bib2gls.bat} that contains the following:
\begin{verbatim}
@ECHO OFF
FOR /F %%I IN ('kpsewhich --progname=bib2gls --format=texmfscripts
bib2gls.jar') DO SET JARPATH=%%I
java -Djava.locale.providers=CLDR,JRE -jar "%JARPATH%" %*
\end{verbatim}
Save this file to somewhere on your system's path.

You may need to refresh \TeX's database to ensure that
\app{kpsewhich} can find the \ext{jar} file.

To test that the application has been successfully installed, open a
command prompt or terminal and run the following command:
\begin{verbatim}
bib2gls --version
\end{verbatim}
This should display the version information.

\chapter{Command Line Options}
\label{sec:switches}

\edef\resetsecnumdepth{\noexpand\setcounter{secnumdepth}{\arabic{secnumdepth}}}
\setcounter{secnumdepth}{0}

The syntax of \bibgls\ is:
\begin{alltt}
bib2gls \oargm{options} \meta{filename}
\end{alltt}
where \meta{filename} is the name of the \ext{aux} file. (The
extension may be omitted.) Only one \meta{filename} is permitted.

Available options are listed below.

\section{\longarg{help} (or \shortarg{h})}
\labelarg{help}

Display the help message and quit.

\section{\longarg{version} (or \shortarg{v})}
\labelarg{version}

Display the version information and quit.

\section{\longarg{debug} \oargm{n}}
\labelarg{debug}

Switch on debugging mode. If \meta{n} is present, it must be a
non-negative integer indicating the debugging level. If omitted 1 is
assumed. This option also switches on the verbose mode. A value of 0
is equivalent to \longargfmt{no-debug}.

Note that multiple instances of this switch in a single invocation
can cause some confusion as \bibgls\ performs a quick parse of the
arguments for the first instance of \longarg{debug} or
\longarg{nodebug} or \longarg{silent} before the language resource
file is loaded. Any subsequent use of the switch will be picked up
on the full parse after the language resource file has been loaded.

\section{\longarg{no-debug} (or \longarg{nodebug})}
\labelarg{no-debug}

Switches off the debugging mode.

\section{\longarg{verbose}}
\labelarg{verbose}

Switches on the verbose mode. This writes extra information to the
terminal and transcript file.

\section{\longarg{no-verbose} (or \longarg{noverbose})}
\labelarg{no-verbose}

Switches off the verbose mode. This is the default behaviour.
Some messages are written to the terminal. To completely suppress
all messages (except errors), switch on the silent mode.
For additional information messages, switch on the verbose mode.

\section{\longarg{silent}}
\labelarg{silent}

Suppresses all messages except for errors that would normally be
written to the terminal. Warnings and informational messages are
written to the transcript file, which can be inspected afterwards.

\section{\longarg{log-file} \meta{filename} (or \shortarg{t}
\meta{filename})}
\labelarg{log-file}

Sets the name of the transcript file. By default, the name is the
same as the \ext{aux} file but with a \ext{glg} extension. Note that
if you use \bibgls\ in combination with \app{xindy} or
\app{makeindex}, you will need to change the transcript file name to
prevent interference.

\section{\longarg{dir} \meta{dirname} (or \shortarg{d}
\meta{dirname})}
\labelarg{dir}

By default \bibgls\ assumes that the output files should be written
in the current working directory. The input \ext{.bib} files are assumed to be
either in the current working directory or on \TeX's path (in which
case \app{kpsewhich} will be used to find them).

If your \ext{aux} file isn't in the current working directory (for
example, you have run \TeX\ with \shortargfmt{output-directory})
then you need to take care how you invoke \bibgls.

Suppose I have a file called \filefmt{test-entries.bib} that
contains my entry definitions and a document called
\filefmt{mydoc.tex} that selects the \ext{bib} file using:
\begin{verbatim}
\GlsXtrLoadResources[src={test-entries}]
\end{verbatim}
If I compile this document using
\begin{verbatim}
pdflatex -output-directory tmp mydoc
\end{verbatim}
then the auxiliary file \filefmt{mydoc.aux} will be written to the
\filefmt{tmp} sub-directory. The resource information is listed in
the \ext{aux} file as
\begin{verbatim}
\glsxtr@resource{src={test-entries}}{mydoc}
\end{verbatim}
If I run \bibgls\ from the \filefmt{tmp} directory, then it won't
be able to find the \filefmt{test-entries.bib} file.

If I run \bibgls\ from the same directory as \filefmt{mydoc.tex}
using
\begin{verbatim}
bib2gls tmp/mydoc
\end{verbatim}
then the \ext{aux} file is found and the transcript file is
\filefmt{tmp/mydoc.glg} (since the default is the same as the
\ext{aux} file but with the extension changed to \ext{glg}) but the
output file \filefmt{mydoc.glstex} will be written to the current
directory.

This works fine from \TeX's point of view. The
\ext{glstex} file can be picked up by \cs{GlsXtrLoadResources} but
it may be that you'd rather the \ext{glstex} file was tidied away
into the \filefmt{tmp} directory along with all the other files.
In this case you need to invoke \bibgls\ with the \longarg{dir} or
\shortarg{d} option:
\begin{verbatim}
bib2gls -d tmp mydoc
\end{verbatim}

\section{\longarg{mfirstuc-protection} (or \shortarg{u})}
\labelarg{mfirstuc-protection}

Commands like \cs{Gls} use \cs{makefirstuc} provided by the
\sty{mfirstuc} package. This command has limitations and one of the
things that can break it is the use of a referencing command 
at the start of its argument. The \sty{glossaries-extra} package has
more detail about the problem in the \qt{Nested Links} section of
the user manual. If a glossary field starts with one of these
problematic commands, the recommended method (if the command can't
be replaced) is to insert an empty group in front of it.

For example, the following definition
\begin{verbatim}
\newabbreviation{shtml}{shtml}{\glsps{ssi} enabled \glsps{short}{html}}
\end{verbatim}
will cause a problem for \verb|\Gls{shtml}| on first use.

The above example, would be written in a \ext{bib} file as:
\begin{verbatim}
@abbreviation{shtml,
  short={shtml},
  long={\glsps{ssi} enabled \glsps{html}}
}
\end{verbatim}

With the \longarg{mfirstuc-protection} switch on (the default
behaviour), \bibgls\ will automatically insert an empty group at the
start of the \field{long} field to guard against this problem. A
warning will be written to the transcript.

\section{\longarg{no-mfirstuc-protection}}
\labelarg{no-mfirstuc-protection}

Switches off the \sty{mfirstuc} protection mechanism described
above.

\section{\longarg{mfirstuc-math-protection}}
\labelarg{mfirstuc-math-protection}

This works in the same way as \argref{mfirstuc-protection} but
guards against fields starting with inline maths
(\verb|$|\ldots\verb|$|). For example, if the \field{name} field
starts with \verb|$x$| and the glossary style automatically tries to
convert the first letter of the name to upper case, then this will
cause a problem.

With \longarg{mfirstuc-math-protection} set, \bibgls\ will
automatically insert an empty group at the start of the field and
write a warning in the transcript. This setting is on by default.

\section{\longarg{no-mfirstuc-math-protection}}
\labelarg{no-mfirstuc-math-protection}

Switches off the above.

\section{\longarg{nested-link-check} \texttt{none}\argor\meta{list}}
\labelarg{nested-link-check}

By default, \bibgls\ will parse certain fields for potential nested links.
(See the section \qt{Nested Links} in the \sty{glossaries-extra}
user manual.)

The default set of fields to check are: \field{name}, \field{text},
\field{plural}, \field{first}, \field{firstplural}, \field{long},
\field{longplural}, \field{short}, \field{shortplural} and
\field{symbol}.

You can change this set of fields using
\longarg{nested-link-check} \meta{value} where \meta{value} may be
\optfmt{none} (don't parse any of the fields) or a comma-separated
list of fields to be checked.

\section{\longarg{no-nested-link-check}}
\labelarg{no-nested-link-check}

Equivalent to \longarg{nested-link-check} \optfmt{none}.

\section{\longarg{shortcuts} \meta{value}}
\labelarg{shortcuts}

Some entries may reference another entry within a field, using
commands like \cs{gls}, so \bibgls\ parses the fields for these
commands to determine dependent entries to allow them to be selected
even if they haven't been used within the document.

The \styopt{shortcuts} package option provided by
\styfmt{glossaries-extra} defines various synonyms, such as \cs{ac}
which is equivalent to \cs{gls}. By default the value of the
\styopt{shortcuts} option will be picked up by \bibgls\ when parsing the
\ext{aux} file. This then allows \bibgls\ to additionally search for
those shortcut commands while parsing the fields.

You can override the \styopt{shortcuts} setting using
\longarg{shortcuts} \meta{value} (where \meta{value} may take
any of the allowed values for the \styopt{shortcuts} package option), 
but in general there is little need to use this switch.

\section{\longarg{map-format} \meta{key=val list}}
\labelarg{map-format}

The \styfmt{glossaries} package provides a mapping between field
tags used in control sequences, such as \cs{glsuseri} or
\cs{glsdesc}, and the corresponding key used when defining the
entry, such as \field{user1} or \field{description}. This
information is written to the auxiliary file so that \bibgls\ can
find the relevant commands while parsing the field values for
dependent entries.

If you have defined some extra fields using \cs{glsaddkey}\margm{key} with
analogous commands in the form \csfmt{gls}\meta{field}, then you can
provide a new mapping so that \bibgls\ can find any instances of
\csfmt{gls}\meta{field} when parsing the entries provided in the
\ext{bib} file.

If you have multiple mappings, you can either use a single
\longarg{map-format} with a comma separated list of
\meta{key}=\meta{field} values or you can have multiple instances of 
\longarg{map-format} \meta{key}\texttt{=}\meta{field}.

\section{\longarg{group}}
\labelarg{group}

The \styopt{record} package option automatically creates a new field
called \field{group}. If the \longarg{group} switch is used,
\bibgls\ will try to determine the letter group for each entry and
add it to the \field{group} field. This value will be picked up by
\cs{printunsrtglossary} if letter group headings are required. If
you're not using a glossary style that displays the group headings,
there's no need to use this switch.

The default is \argref{no-group}.

\section{\longarg{no-group}}
\labelarg{no-group}

Don't use the \field{group} field. (Default.)

\resetsecnumdepth
\chapter{\ext{bib} Format}

\end{document}
