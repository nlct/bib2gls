% arara: xelatex
% arara: bib2gls
% arara: bibtex
% arara: xelatex
% arara: xelatex if found ("log", "Rerun")
\documentclass[titlepage=false,fontsize=12pt,captions=tableheading]{scrreprt}

\usepackage[no-math]{fontspec}
\setmainfont{Linux Libertine O}

\newfontface\cyrillicmono{FreeMono}[Scale=MatchLowercase]
\newcommand{\textcyrillicmono}[1]{{\cyrillicmono #1}}

\usepackage[x11names]{xcolor}
\usepackage{upquote}
\usepackage{hologo}
\usepackage{pifont}
\usepackage{graphicx}
\usepackage{datetime2}
\usepackage{listings}

\usepackage{xr-hyper}
\usepackage[hidelinks]{hyperref}
\usepackage[record]{glossaries-extra}

\glsdisablehyper

\lstset{language={[LaTeX]TeX},upquote,basicstyle={\ttfamily},
 frame=lines,rulecolor={\color{lightgray}},
 commentstyle={\color{gray}},
 keywords={name,text,firstplural,first,plural,description,symbol,category},
 keywordstyle={\color{DarkSlateGray4}},
 texcs={newglossaryentry,longnewglossaryentry,newabbreviation,newacronym,
  gls,Gls,glspl,Glspl,glsentrydesc,glsentryname,glsentrysymbol,
  glsnoexpandfields,glsxtrifwasfirstuse,glsxtrpostlinkAddSymbolOnFirstUse,
  glsxtrpostlinkAddDescOnFirstUse,setabbreviationstyle},
 texcsstyle=*{\color{DarkSeaGreen4}}
}

\newcommand{\dhyphen}{%
 \texorpdfstring
 {\discretionary{}{}{}\texttt{-}}%
 {-}%
}

\renewrobustcmd{\-}{%
 \discretionary
 {{\rmfamily\char\ifnum\hyphenchar\font<0
  \defaulthyphenchar\else\hyphenchar\font\fi
 }}%
 {}{}%
}

\setabbreviationstyle[common]{short-nolong}

\GlsXtrLoadResources[
 src={bib2gls},
 save-locations=false,
 entry-type-aliases={dualindexentry=entry}
]

\DTMsavetimestamp{creation}{2017-01-20T15:39:00Z}

\IfFileExists{../java/Bib2Gls.java}
{
  \DTMsavefilemoddate{moddate}{../java/Bib2Gls.java}
}
{
  \DTMsavenow{moddate}
}

\newcommand{\bibgls}{\appfmt{bib2gls}}

\newcommand*{\BibTeX}{\hologo{BibTeX}}
\newcommand*{\eTeX}{\hologo{eTeX}}
\newcommand*{\XeLaTeX}{\hologo{XeLaTeX}}
\newcommand*{\LuaLaTeX}{\hologo{LuaLaTeX}}
\newcommand*{\pdfLaTeX}{\hologo{pdfLaTeX}}

\newcommand*{\ctanfile}[2]{%
 \href{http://mirrors.ctan.org/macros/latex/contrib/#1/#2}{\nolinkurl{#2}}%
}

\newcommand{\qt}[1]{``#1''}

\newcommand{\qtt}[1]{\qt{\,\texttt{#1}\,}}

\newcommand{\incorrect}{\marginpar{\textcolor{red}{\ding{55}}}}
\newcommand{\correct}{\marginpar{\textcolor{green}{\ding{52}}}}

\newenvironment{result}%
{%
 \begin{quotation}%
 \marginpar
  [\raisebox{-2ex}{\ding{43}}]%
  {\raisebox{-2ex}{\reflectbox{\ding{43}}}}%
 \ignorespaces
}
{\end{quotation}\ignorespacesafterend}

\newcommand{\dequals}{%
 \texorpdfstring
 {\discretionary{}{}{}\texttt{=}\discretionary{}{}{}}%
 {=}%
}

\newcommand{\dcomma}{%
 \texorpdfstring
 {\texttt{,}\discretionary{}{}{}}%
 {,}%
}

\newcommand{\dcolon}{%
 \texorpdfstring
 {\texttt{:}\discretionary{}{}{}}%
 {:}%
}

\pdfstringdefDisableCommands{%
  \def\dhyphen{-}%
  \def\dcolon{:}%
  \def\dcomma{,}%
  \def\dequals{,}%
  \let\-\empty
}

\newcommand*{\csfmt}[1]{%
 \texorpdfstring
 {\texttt{\char`\\ #1}}%
 {\string\\#1}%
}

\newcommand*{\appfmt}[1]{\texorpdfstring{\texttt{#1}}{#1}}
\newcommand*{\styfmt}[1]{\texorpdfstring{\textsf{#1}}{#1}}
\newcommand*{\envfmt}[1]{\texorpdfstring{\textsf{#1}}{#1}}
\newcommand*{\optfmt}[1]{\texorpdfstring{\texttt{#1}}{#1}}
\newcommand*{\csoptfmt}[1]{\texorpdfstring{\textcolor{DarkSeaGreen4}{\optfmt{#1}}}{#1}}
\newcommand*{\styoptfmt}[1]{\texorpdfstring{\textcolor{DarkOrchid4}{\optfmt{#1}}}{#1}}
\newcommand*{\fieldfmt}[1]{\texorpdfstring{\texttt{\color{DarkSlateGray4}#1}}{#1}}
\newcommand*{\entryfmt}[1]{\texorpdfstring{\texttt{\color{SteelBlue4}#1}}{#1}}
\newcommand*{\atentryfmt}[1]{\entryfmt{@#1}}
\newcommand*{\abbrstylefmt}[1]{\texorpdfstring{\textsf{#1}}{#1}}
\newcommand*{\glostylefmt}[1]{\texorpdfstring{\textsf{#1}}{#1}}
\newcommand*{\catattrfmt}[1]{\texorpdfstring{\textsf{#1}}{#1}}
\newcommand*{\counterfmt}[1]{\texorpdfstring{\textsf{#1}}{#1}}
\newcommand*{\filefmt}[1]{\texorpdfstring{\texttt{#1}}{#1}}
\newcommand*{\metafilefmt}[3]{%
  \filefmt{#1}\discretionary{}{}{}\meta{#2}\discretionary{}{}{}\filefmt{#3}%
}

\newcommand*{\extfmt}[1]{\filefmt{.#1}}%

\newcommand*{\argor}{\texorpdfstring{\protect\textbar}{|}}

\newrobustcmd*{\texmeta}[1]{{\normalfont\normalcolor$\langle$\emph{#1}$\rangle$}}

\newcommand*{\meta}[1]{%
 \texorpdfstring{\ifmmode\text{\texmeta{#1}}\else\texmeta{#1}\fi}{#1}%
}

\newcommand*{\oarg}[1]{\discretionary{}{}{}[#1]}
\newcommand*{\oargm}[1]{\oarg{\meta{#1}}}

\newcommand*{\marg}[1]{\texorpdfstring
 {\discretionary{}{}{}\char`\{#1\char`\} }%
 {\{#1\}}%
}

\newcommand*{\margm}[1]{\marg{\meta{#1}}}

\newcommand{\switcharg}{}
\newcommand{\switchalt}{}

\makeatletter
\newcommand{\code}[1]{\texorpdfstring{{\ttfamily\obeyspaces #1}}{#1}}
\newenvironment{codeenv}
 {\begin{flushleft}\ttfamily\obeylines\frenchspacing\@vobeyspaces}
 {\end{flushleft}\ignorespacesafterend}
\makeatother

\newcommand{\primary}{\emph}

\newcommand{\pidx}[1][]{\gls[textformat=primary,#1]}
\newcommand{\pidxpl}[1][]{\glspl[textformat=primary,#1]}

\newcommand{\idx}{\gls}
\newcommand{\idxpl}{\glspl}
\newcommand{\Idx}{\Gls}
\newcommand{\Idxpl}{\Glspl}

\newcommand{\ext}{\gls}

\newcommand*{\iext}[1]{%
 \glsxtrtitleorpdforheading{\idx{#1}}{.#1}{\extfmt{#1}}%
}

\newcommand{\sty}{\gls}

\newcommand*{\isty}[1]{%
  \texorpdfstring{\idx{#1}}{#1}%
}

\newcommand*{\env}[1]{%
  \texorpdfstring{\idx{env.#1}}{#1}%
}

\newcommand*{\abbrstyle}[1]{%
  \texorpdfstring{\idx{#1}}{#1}%
}

\newcommand*{\glostyle}[1]{%
  \texorpdfstring{\idx{glostyle.#1}}{#1}%
}

\newcommand*{\catattr}[1]{%
  \texorpdfstring{\idx{#1}}{#1}%
}

\newcommand*{\counter}[1]{%
  \texorpdfstring{\idx{ctr.#1}}{#1}%
}

\newcommand*{\styopt}[2][]{%
  \texorpdfstring%
  {%
    \gls{styopt.#2}\styoptfmt{\ifblank{#1}{}{\dequals\marg{#1}}}%
  }%
  {#2\ifblank{#1}{}{=#1}}%
}

\newcommand*{\keyvallist}{%
 \texorpdfstring
 {key\dequals value list}%
 {key=value list}%
}

\newcommand{\nosecformatdef}[1]{%
  \begin{definition}
   \glsadd{#1}\glsxtrglossentry{#1}%
   \glsentryuseri{#1}%
  \end{definition}\ignorespaces
}

\newcommand*{\cs}{\gls}

\newcommand*{\ics}{\cs}

\newcommand*{\icswithargs}[2][]{\cs{#2}\glsentryuseri{#2}}

\newcommand*{\postdeschook}[2][]{%
 \glslink[#1]{idx.glsxtrpostdesccategory}{\csfmt{glsxtrpostdesc#2}}}

\newcommand*{\postlinkhook}[2][]{%
 \glslink[#1]{idx.glsxtrpostlinkcategory}{\csfmt{glsxtrpostlink#2}}}


\glsxtrnewgls{file.}{\exfile}

\newcommand*{\csopt}[2][]{\gencsopt{#1}{opt}{#2}}%
\newcommand*{\glsopt}[2][]{\gencsopt{#1}{gls}{#2}}%
\newcommand*{\glsaddopt}[2][]{\gencsopt{#1}{glsadd}{#2}}%

\newcommand*{\gencsopt}[3]{%
  \texorpdfstring%
  {%
    \gls{#2.#3}%
    \csoptfmt{\ifblank{#1}{}{\dequals\marg{#1}}}%
  }%
  {#3\ifblank{#1}{}{=#1}}%
}

\newcommand*{\field}[1]{%
 \texorpdfstring
 {\gls{field.#1}}%
 {#1}%
}

\newcommand*{\atentry}[2][]{%
 \texorpdfstring
 {\gls[#1]{entry.#2}}%
 {#2}%
}

\newrobustcmd{\longswitch}{\string-{}\string-}

\newcommand*{\longargfmt}[1]{%
 \texorpdfstring{\texttt{\longswitch #1}}%
 {\string-\string-#1}%
}

\newcommand*{\shortargfmt}[1]{%
 \texorpdfstring{\texttt{\string-#1}}%
 {\string-#1}%
}

\newcommand*{\longarg}[1]{%
  \texorpdfstring
  {\gls{switch.#1}}%
  {\string-\string-#1}%
}

\definecolor{defbackground}{rgb}{1,1,0.75}

\newsavebox\borderedboxcontents
\newlength\borderedboxwidth

\newenvironment{definition}%
{%
  \setlength{\fboxsep}{4pt}\setlength{\fboxrule}{1.25pt}%
  \begin{lrbox}{\borderedboxcontents}%
   \setlength\borderedboxwidth\linewidth
   \addtolength\borderedboxwidth{-2\fboxrule}%
   \addtolength\borderedboxwidth{-2\fboxsep}%
   \begin{minipage}{\borderedboxwidth}
   \flushleft\ttfamily\ignorespaces
}%
{%
   \end{minipage}%
  \end{lrbox}\par\medskip\noindent
  \fcolorbox{black}{defbackground}{\usebox\borderedboxcontents}%
  \medskip\par\noindent
  \ignorespacesafterend
}

\newenvironment{important}{%
  \setlength{\fboxrule}{4pt}%
  \setlength\borderedboxwidth{\linewidth}%
  \addtolength\borderedboxwidth{-2\fboxsep}%
  \addtolength\borderedboxwidth{-2\fboxrule}%
  \begin{lrbox}{\borderedboxcontents}%
    \begin{minipage}{\borderedboxwidth}%
    \raggedright
    \setlength\parindent{1em}%
    \noindent\ignorespaces
}%
{%
    \end{minipage}%
  \end{lrbox}%
  \par\vskip10pt\noindent
  \fcolorbox{red}{white}{\usebox{\borderedboxcontents}}\par\vskip10pt
  \noindent\ignorespacesafterend
}

\newcommand*{\sectionref}[1]{section~\ref{#1}}
\newcommand*{\Sectionref}[1]{Section~\ref{#1}}

\pagestyle{headings}

\newcommand{\glossarytitle}{Index}
\externaldocument{bib2gls}

\newcommand{\addr}[1]{\\\href{https://www.#1/}{\nolinkurl{#1}}}
\title{\bibgls\ and \styfmt{glossaries-extra}: A Guide for Beginners}
\author{Nicola Talbot\addr{dickimaw-books.com}}
\date{\DTMusedate{moddate}}

\makeatletter
\begingroup
 \renewcommand{\addr}[1]{}
 \let\texorpdfstring\@secondoftwo
 \DTMsetstyle{pdf}
 \protected@edef\x{\endgroup
   \noexpand\hypersetup{%
     pdfinfo={
       Title={\@title},
       Author={\@author},
       CreationDate={\DTMuse{creation}},
       ModDate={\DTMuse{moddate}},
     }%
   }%
 }\x

\makeatother

\begin{document}
\maketitle
\pagenumbering{alph}
\thispagestyle{empty}

\begin{abstract}
This document is an introductory guide to \bibgls\ and the
\sty{glossaries-extra} package to help you get started. For
further information, including more complex commands and settings,
see the main \bibgls\ user manual (\filefmt{bib2gls.pdf},
in the same directory as this document),
the \isty{glossaries-extra} user manual,
(distributed with the \sty{glossaries-extra}
package~\cite{glossaries-extra})
and the \isty{glossaries} user manual
(distributed with the \sty{glossaries} package~\cite{glossaries}).

The \sty{glossaries} package is the \emph{base} package. The 
\sty{glossaries-extra} package internally loads the \sty{glossaries} package 
and extends it, providing extra options or modifying the base
commands to increase flexibility. If you want to use \bibgls, you
must load \sty{glossaries-extra}, which provides the interface
required by \bibgls.
\end{abstract}

\clearpage
\pagenumbering{roman}
\tableofcontents

\clearpage
\pagenumbering{arabic}

\chapter{Introduction}
\label{sec:beginintro}

The \sty{glossaries} package provides a way of defining terms,
notation or abbreviations that can then be used in the document.
This ensures consistent naming and formatting. (With the help of the
\sty{hyperref} package, it's also possible to create hyperlinks from
the reference to a place in the document that provides a definition
of the term, but more about that later.) Each entry (term,
notation or abbreviation) is defined using:
\nosecformatdef{newglossaryentry}
Here's a simple example:
\begin{lstlisting}
\documentclass{article}

\usepackage{glossaries}

\newglossaryentry{duck}% label
{% information about this term:
  name={duck},% display name
  description={a waterbird with webbed feet}% description
}

\newglossaryentry{goose}% label
{% information about this term:
  name={goose},% display name
  plural={geese},% plural form
  description={a large waterbird with a long neck, short legs,
   webbed feet and a short broad bill}
}

\begin{document}
The pond contained a \gls{duck} (\glsentrydesc{duck}) and 
a \gls{goose} (\glsentrydesc{goose}). \Glspl{duck} and
\glspl{goose} are fowl.
\end{document}
\end{lstlisting}
The resulting text is:
\begin{result}
The pond contained a \gls{ex1.duck} (\glsentrydesc{ex1.duck}) and 
a \gls{ex1.goose} (\glsentrydesc{ex1.goose}). \Glspl{ex1.duck} and
\glspl{ex1.goose} are fowl.
\end{result}
For convenience, the text produced by commands such as \cs{gls} is called the 
\pidx{link-text} (even if there are no hyperlinks).

The first argument of \gls{newglossaryentry} is a label that
uniquely identifies the term (see \sectionref{sec:labels}). The
second argument is a comma-separated list of
\meta{setting}\dequals\meta{value} assignments.  Each \meta{setting}
is referred to as a \qt{key} in the \isty{glossaries} manual or as a
\qt{field} in the \bibgls\ manual. A list of the available base keys
can be found in the \isty{glossaries} user manual.  The
\isty{glossaries-extra} package provides some additional keys that
are described in the \sty{glossaries-extra} manual. The \bibgls\
user manual summarises all keys (fields) in \sectionref{sec:fields}.

\begin{important}
If the field value contains commas or equal signs the value must be grouped
to hide those characters from the \meta{key}\dequals\meta{value} parser.
\end{important}

The two main keys are \field{name} and \field{description}. The
\field{name} identifies how the term should be displayed in the
glossary (see \sectionref{sec:printgloss}). It also provides the
default singular term, if not explicitly given. The default plural
is obtained by appending \qt{s} to the singular form. If this isn't
correct (as with \qt{geese}), then the plural form can be specified
with the \field{plural} key.

The description (set with the \field{description} key) is usually
only displayed in the glossary, but you can display it in the text
using:
\nosecformatdef{glsentrydesc}
as in the above example. This simply expands to the value of the
\field{description} field (or does nothing if there's no entry
associated with the given label).

The main command used to reference a term is:
\nosecformatdef{gls}
In the above example, it just displays the singular form, but you
can provide alternative text to use the first time a term is
referenced (see \sectionref{sec:firstuse}). The plural form is obtained with
\nosecformatdef{glspl}
If you want to start a sentence with an entry then you can use:
\nosecformatdef{Gls}
for the singular form and
\nosecformatdef{Glspl}
for the plural form.
The \meta{insert} optional argument is provided to insert additional material.
For example:
\begin{lstlisting}
The \gls{goose} liked the \gls{duck}['s] hat.
\end{lstlisting}
which produces (assuming the above definitions):
\begin{result}
The \gls{ex1.goose} liked the \gls{ex1.duck}['s] hat.
\end{result}
In some cases, there may not be a noticeable difference between the
above and the following:
\begin{lstlisting}
The \gls{goose} liked the \gls{duck}'s hat.
\end{lstlisting}
It depends on other settings, such as whether or not hyperlinks have
been enabled. (The inserted material is commonly moved inside the
hyperlink.) Take care if you need a literal open square bracket
following \code{\cs{gls}\margm{label}} as you need to prevent it from being
interpreted as the optional \meta{insert} argument. For example:
\begin{lstlisting}
The \gls{goose} liked the \gls{duck}{['s]} hat.
\end{lstlisting}
which now produces:
\begin{result}
The \gls{ex1.goose} liked the \gls{ex1.duck}{['s]} hat.
\end{result}
An alternative in this case could be to define:
\begin{lstlisting}
\newcommand*{\missing}[1]{[#1]}
\end{lstlisting}
and then use:
\begin{lstlisting}
The \gls{goose} liked the \gls{duck}\missing{'s} hat.
\end{lstlisting}
This conveniently hides the open square bracket from \cs{gls}.

\begin{important}
Commands like \gls{gls} are \idx{robust}. Commands like
\gls{glsentrydesc} are \idx{expandable}. (See \sectionref{sec:robust}.)
If you want the entry to appear in a PDF bookmark, you need to use an 
expandable command to reference it.
\end{important}

There are some helper commands that internally use
\gls{newglossaryentry}, such as \cs{newabbreviation} (described in
\sectionref{sec:abbreviations}) and \cs{glsxtrnewsymbol} (described
in \sectionref{sec:symbols}). If the description contains paragraph
breaks then:
\nosecformatdef{longnewglossaryentry}
is required instead.

\section{Labels}
\label{sec:labels}

The label used to identify the entry can't contain any special
characters, such as \gls{commentchar} (percent), \gls{ampchar}
(ampersand), \gls{param} (hash) or \gls{nbspchar} (tilde). Be
careful of packages that make other characters active (such as
\isty{babel} with its shortcuts). If you are using \isty{inputenc},
this also includes extended Latin characters and characters from
other scripts. If you want to include UTF-8 characters in the label
then you must use a \TeX\ engine with native Unicode support (that
is, \XeLaTeX\ or \LuaLaTeX).

For example, with no UTF-8 support (not even \sty{inputenc}):
\begin{lstlisting}
\newglossaryentry{elite}% label (no UTF-8 support)
{
  name = {{\'e}lite},
  description = {group of people regarded as
  the best of a particular society or organisation}
}
\end{lstlisting}
or with \sty{inputenc}:
\begin{lstlisting}
\newglossaryentry{elite}% label (UTF-8 not natively supported)
{
  name = {élite},
  description = {group of people regarded as
  the best of a particular society or organisation}
}
\end{lstlisting}
Whereas with \XeLaTeX\ or \LuaLaTeX\ you can do:
\begin{lstlisting}
\newglossaryentry{élite}% label (UTF-8 natively supported)
{
  name = {élite},
  description = {group of people regarded as
  the best of a particular society or organisation}
}
\end{lstlisting}

You may have noticed the grouping of the initial (accented) letter
in the \gls{ASCII} example (\verb|{\'e}lite|). This is necessary to
ensure that the first-letter case-changing commands, such as
\ics{Gls}, work. It also used to be required around the
\qtt{é} with \sty{inputenc}, but if you have up-to-date versions of
\sty{glossaries} and \sty{datatool} then it should no longer be
necessary. No special treatment is needed with \XeLaTeX\ or \LuaLaTeX\
where \qtt{é} is a single token.

If you can't use extended characters in the label (because you're
not using \XeLaTeX\ or \LuaLaTeX), then simply stripping the accents
to create an \gls{ASCII} alternative may be sufficient, but take
care if this may cause a conflict. For example:
\begin{lstlisting}
\newglossaryentry{resume}% label
{
  name = {resume},
  description = {continue after an interruption}
}

\newglossaryentry{resumee}% label
{
  name = {r\'esum\'e},
  description = {summary of something or curriculum vitae}
}
\end{lstlisting}
For languages that use a non-Latin script, if you can't or don't
want to use \XeLaTeX\ or \LuaLaTeX, then you need to decide the most
appropriate \gls{ASCII} naming scheme.
For example:
\begin{lstlisting}[escapechar=|]
\newglossaryentry{goose}% using translation for label
{
  name = {|\textcyrillicmono{гусь}|},
  plural = {|\textcyrillicmono{гуси}|},
  description = {...}
}
\end{lstlisting}
or
\begin{lstlisting}[escapechar=|]
\newglossaryentry{hus}% using closest ASCII match for label
{
  name = {|\textcyrillicmono{гусь}|},
  plural = {|\textcyrillicmono{гуси}|},
  description = {...}
}
\end{lstlisting}

In addition to labels identifying entries, there are also labels
that identify other things, such as a glossary, category or letter
group. The same restrictions apply to those labels.

\section{First Use}
\label{sec:firstuse}

Each entry has a \pidx{firstuseflag} (boolean variable) that
determines whether or not the entry has been referenced in the
document.  Commands like \gls{gls} and \gls{glspl} change the flag
to indicate that the entry has been used. Commands like
\gls{glsentrydesc} don't. Here's a modification of the earlier
example document that provides different versions depending on 
whether or not the entry has already been referenced:
\begin{lstlisting}
\documentclass{article}

\usepackage{glossaries}

\newglossaryentry{duck}% label
{% information about this term:
  name   = {Duck (noun)},% display name
  first  = {duck (quack, quack)},% first use singular
  firstplural = {ducks (quack, quack)},% first use plural
  text   = {duck},% subsequent use singular
  description = {a waterbird with webbed feet}% description
}

\newglossaryentry{goose}% label
{% information about this term:
  name   = {Goose (noun, pl.\ geese)},% display name
  first  = {goose (honk, honk)},% first use singular
  firstplural = {geese (honk, honk)},% first use plural
  text   = {goose},% subsequent use singular 
  plural = {geese},% subsequent use plural 
  description={a large waterbird with a long neck, short legs,
   webbed feet and a short broad bill}
}

\begin{document}
The pond contained a \gls{duck}\footnote{\glsentryname{duck}:
\glsentrydesc{duck}} and two 
\glspl{goose}\footnote{\glsentryname{goose}: 
\glsentrydesc{goose}}. \Glspl{duck} and \glspl{goose} are fowl.
\end{document}
\end{lstlisting}
This now produces:
\begin{result}
The pond contained a \gls{ex2.duck}\footnote{\glsentryname{ex2.duck}:
\glsentrydesc{ex2.duck}} and two
\glspl{ex2.goose}\footnote{\glsentryname{ex2.goose}: 
\glsentrydesc{ex2.goose}}. \Glspl{ex2.duck} and \glspl{ex2.goose} are fowl.
\end{result}
This uses:
\nosecformatdef{glsentryname}
which works in a similar way to \gls{glsentrydesc}. In this case,
\gls{glsentryname} simply expands to the value of the \field{name}
key. There's also a case-changing version:
\nosecformatdef{Glsentryname}
which changes the initial character to upper case,
but (unlike \cs{glsentryname}) this command isn't expandable. If,
for example, I had instead set the duck's \field{name} key using:
\begin{lstlisting}
name = {duck (noun)}
\end{lstlisting}
then I would need to use \code{\cs{Glsentryname}\marg{duck}}
instead.

So on \pidx{firstuse}, \cs{gls} uses the value of the \field{first} key
and \cs{glspl} uses the value of the \field{firstplural} key. On
\pidx{subsequentuse}, \cs{gls} uses the value of the \field{text} key and
\cs{glspl} uses the value of the \field{plural} key.

If the first use for a particular group of terms always has the
same pattern (such as following the term with a brief description or
alternative representation), then it's simpler to use one of the
automated methods provided, such as the abbreviation mechanism
(\sectionref{sec:abbreviations}) or changing the formatting
(\sectionref{sec:glsformats}).

\section{Categories}
\label{sec:categories}

The \isty{glossaries-extra} extension package provides the
\field{category} key, which isn't available with just the base
\sty{glossaries} package. The value of this key must be a
label as it's used to construct command names. You can choose
whatever label you like (as long as it conforms to the valid
labelling scheme, described in \sectionref{sec:labels}). If you
don't specify a category, then \gls{newglossaryentry} and
\gls{longnewglossaryentry} assume \code{general}. The helper
commands, such as \gls{newabbreviation}, have different defaults.

The category allows you to apply certain types of formatting. These
are described in more detail in \sectionref{sec:glsformats}. For
abbreviations, the category also governs the abbreviation style.

Consider the following document:
\begin{lstlisting}[moretexcs={glsxtrpostlinkconstant}]
\documentclass{article}

\usepackage{pifont}% provides \ding
\usepackage{glossaries-extra}

\newglossaryentry{fleuron}% label
{
  name = {fleuron},
  symbol = {\ding{167}},
  description = {typographic ornament}
}

\newglossaryentry{pi}% label
{
  name = {pi},
  symbol = {\ensuremath{\pi}},
  category = {constant},
  description = {Archimedes' constant}
}

% post-link hook for 'constant' category:
\newcommand{\glsxtrpostlinkconstant}{% 
 \space (\glsentrysymbol{\glslabel})}

\begin{document}
A \gls{fleuron} and \gls{pi}.
\end{document}
\end{lstlisting}
This produces:
\begin{result}
A fleuron and pi ($\pi$).
\end{result}
The \code{fleuron} entry doesn't have the \field{category} key
explicitly set, so it defaults to \code{general}, but the \code{pi}
entry has the \field{category} set to \code{constant}, so it's
affected by the \idx{postlinkhook} for that category, which in this
case is given by \csfmt{glsxtrpostlinkconstant} (see
\sectionref{sec:postlinkhooks} for further details). This hook is
defined to use:
\nosecformatdef{glsentrysymbol}
which works like \gls{glsentrydesc}, but in this case it expands to
the value of the \field{symbol} key. (The entry label is obtained
from \gls{glslabel}, which is set by \gls{gls} and similar commands.)

This means that \verb|\gls{pi}| is automatically followed by the
symbol in parentheses, but \verb|\gls{fleuron}| isn't because it's
governed by the \code{general} \idx{postlinkhook} instead. Note that
the above is a simple example to demonstrate one of the uses of the
\field{category} field. In general, you're unlikely to need the
symbol appended on every use of \verb|\gls{pi}|. A better definition
is
\begin{lstlisting}[moretexcs={glsxtrpostlinkconstant}]
\newcommand*{\glsxtrpostlinkconstant}{%
 \glsxtrpostlinkAddSymbolOnFirstUse
}
\end{lstlisting}
See \sectionref{sec:postlinkhooks} for further details.

Categories may be assigned \pidxpl{attribute} that can also be used
to modify formatting or styles. Unlike the \idx{postlinkhook}, which
needs to be defined before an entry is used (with commands like
\gls{gls}), some \idxpl{attribute} need to be set before the entry
is defined, so it's best to set them up as soon as possible in the
preamble (after loading \sty{glossaries-extra}).

For example, the \catattr{textformat} attribute may be used to set the 
name of a formatting command (without the leading backslash) as in the
following:
\begin{lstlisting}[morekeywords={textformat}]
\documentclass{article}

\usepackage{xcolor}% provides colour
\usepackage{pifont}% provides \ding
\usepackage{glossaries-extra}

\newcommand{\ornamentfmt}[1]{\textcolor{purple}{#1}}
\newcommand{\constantfmt}[1]{\textcolor{blue}{#1}}

\glssetcategoryattribute{ornament}{textformat}{ornamentfmt}
\glssetcategoryattribute{constant}{textformat}{constantfmt}

\newglossaryentry{fleuron}% label
{
  name = {\ding{167}},
  category = {ornament},
  description = {typographic ornament}
}

\newglossaryentry{pi}% label
{
  name = {\ensuremath{\pi}},
  category = {constant},
  description = {Archimedes' constant}
}

\begin{document}
A \gls{fleuron} and \gls{pi}.
\end{document}
\end{lstlisting}
This produces:
\begin{result}
A \textcolor{purple}{\ding{167}} and \textcolor{blue}{$\pi$}.
\end{result}
In this case, the \code{fleuron} entry is assigned the category
\code{ornament}. The \code{pi} entry still has the category
\code{constant}. I've defined some custom commands to provide the
formatting (\csfmt{ornamentfmt} and \csfmt{constantfmt}), not only
for semantic reasons but also because the \catattr{textformat}
attribute requires that the command identified by the value must
take a single mandatory argument.  This means that it's not possible
to put \code{textcolor} directly in the attribute value.

\section{Spaces}
\label{sec:spaces}

With \LaTeX\ in general, spaces are sometimes significant and
sometimes ignored. When defining entries, any spaces around the
equal sign or comma are ignored. For example, if an entry is defined
as
\begin{lstlisting}
\newglossaryentry{sample}
{
  name = {sample} , description = {an example}
}
\end{lstlisting}
then
\begin{lstlisting}
/\gls{sample}/
\end{lstlisting}
will produce
\begin{result}
/sample/
\end{result}
(no spaces). Similarly with:
\begin{lstlisting}
\newglossaryentry{sample}
{
  name = sample , description = {an example}
}
\end{lstlisting}
However, spaces at the start or end of the value if it's been
enclosed in braces aren't ignored. For example, if the entry is now
defined as:
\begin{lstlisting}[escapechar=|]
\newglossaryentry{sample}
{
  name = { sample } , description = {an example}|\incorrect|
}
\end{lstlisting}
then:
\begin{lstlisting}
/\gls{sample}/
\end{lstlisting}
produces:
\begin{result}
/ sample /
\end{result}
The spaces in this case have been retained. The unstarred version of
\gls{longnewglossaryentry} appends \csfmt{unskip} to the end of the
description, which removes any trailing spaces. The starred version
(only available with \sty{glossaries-extra}) doesn't. In both cases 
any leading spaces are retained. For example,
if the entry is defined as:
\begin{lstlisting}[escapechar=|]
\longnewglossaryentry{sample}{name={sample}}{ an example }|\incorrect|
\end{lstlisting}
then:
\begin{lstlisting}
/\glsentrydesc{sample}/
\end{lstlisting}
produces:
\begin{result}
/ an example/
\end{result}
(trailing space removed), whereas if the entry is defined as:
\begin{lstlisting}[escapechar=|]
\longnewglossaryentry*{sample}{name={sample}}{ an example }|\incorrect|
\end{lstlisting}
then:
\begin{lstlisting}
/\glsentrydesc{sample}/
\end{lstlisting}
produces:
\begin{result}
/ an example /
\end{result}
(leading and trailing spaces retained).

Spaces in labels are significant. For example, in \verb|\gls{ duck }|
the spaces are considered part of the label. If the entry was
actually defined without spaces in the label then the entry
referenced in \verb|\gls{ duck }| won't be found.

\section{Robust, Fragile and Expandable Commands}
\label{sec:robust}

Commands like \gls{gls} are \pidx{robust}. This protects them from
premature expansion in situations that would otherwise break the
command. If content containing a \idx{robust} command is written to
an external file, the \idx{robust} command itself is written instead of its
definition. For example, consider the following document:
\begin{lstlisting}[keywords={},moretexcs={test,section}]
\documentclass{article}
\newcommand{\test}{some sample text}
\begin{document}
\tableofcontents
\section{\test}
\end{document}
\end{lstlisting}
In this case, \csfmt{test} is expandable. Its definition doesn't
contain anything complicated. The \ext{toc} file (which is input by
\csfmt{tableofcontents}) contains the line:
\begin{lstlisting}[keywords={}]
\contentsline {section}{\numberline {1}some sample text}{1}
\end{lstlisting}
So \csfmt{test} has been expanded to its definition when it was
written to the \ext{toc} file. If \csfmt{test} is defined in terms
of another command, that will also be expanded. For example:
\begin{lstlisting}[keywords={},moretexcs={test,section,sample,emph}]
\documentclass{article}
\newcommand{\sample}{\emph{sample}}
\newcommand{\test}{some \sample\ text}
\begin{document}
\tableofcontents
\section{\test}
\end{document}
\end{lstlisting}
The \ext{toc} file now contains:
\begin{lstlisting}[keywords={},moretexcs={emph}]
\contentsline {section}{\numberline {1}some \emph {sample}\ 
text}{1}
\end{lstlisting}
So \csfmt{sample} has also been expanded but neither
\gls{emph} nor \idx{cs.space} (backslash space) have
been expanded. \Idx{robust} commands don't expand. For example:
\begin{lstlisting}
\documentclass{article}

\usepackage{glossaries}

\newglossaryentry{duck}
{
  name={duck},
  description={a waterbird with webbed feet}
}

\begin{document}
\tableofcontents
\section{\Gls{duck}: \glsentrydesc{duck}}
\end{document}
\end{lstlisting}
The \ext{toc} file now contains:
\begin{lstlisting}
\contentsline {section}{\numberline {1}\Gls {duck}: a waterbird 
with webbed feet}{1}
\end{lstlisting}
So \gls{Gls} doesn't expand, and the command itself is written to the
\ext{toc} file, but \gls{glsentrydesc} does expand.

A \pidx{fragile} command is one that breaks (causes an error) when
it's expanded in this type of context. One such command is \gls{footnote}.
For example, the following won't work:
\begin{lstlisting}[moretexcs={footnote},escapechar=|]
\documentclass{article}

\usepackage{glossaries}

\newglossaryentry{duck}
{
  name={duck},
  description={a waterbird with webbed feet}
}

\begin{document}
\tableofcontents
\section{\Gls{duck}\footnote{\glsentrydesc{duck}}}|\incorrect|
\end{document}
\end{lstlisting}
This causes the error:
\begin{verbatim}
! Argument of \@sect has an extra }.
\end{verbatim}
Inserting \gls{protect} before the command prevents the attempted
expansion, which makes the command behave as though it was robust:
\begin{lstlisting}[moretexcs={footnote,protect}]
\section{\Gls{duck}\protect\footnote{\glsentrydesc{duck}}}
\end{lstlisting}
In this case, it's unlikely that you'd want the footnote to appear
in the table of contents, so it would be better to use the optional
argument:
\begin{lstlisting}[moretexcs={footnote},escapechar=|]
\section[Duck]{\Gls{duck}\footnote{\glsentrydesc{duck}}}|\correct|
\end{lstlisting}
Now the \ext{toc} file is just:
\begin{lstlisting}
\contentsline {section}{\numberline {1}Duck}{1}
\end{lstlisting}
If the \field{description} field contains a \idx{fragile} command
then \gls{glsentrydesc} will break in expandable contexts. For
example, the following doesn't work:
\begin{lstlisting}[moretexcs={footnote},escapechar=|]
\documentclass{article}

\usepackage{glossaries}

\newglossaryentry{duck}
{
  name={duck},
  description={a waterbird\footnote{a bird that lives on or 
   near water} with webbed feet}
}

\begin{document}
\tableofcontents
\section{\Gls{duck}: \glsentrydesc{duck}}|\incorrect|
\end{document}
\end{lstlisting}
This is a contrived example. In this case, it would be better to
also define the term \qt{waterbird}: 
\begin{lstlisting}
\documentclass{article}

\usepackage{glossaries}

\newglossaryentry{waterbird}
{
  name={waterbird},
  description={a bird that lives on or near water}
}

\newglossaryentry{duck}
{
  name={duck},
  description={a \gls{waterbird} with webbed feet}
}

\begin{document}
\tableofcontents
\section{\Gls{duck}: \glsentrydesc{duck}}
\end{document}
\end{lstlisting}
The \ext{toc} file now contains:
\begin{lstlisting}
\contentsline {section}{\numberline {1}\Gls {duck}: a \gls 
{waterbird} with webbed feet}{1}
\end{lstlisting}

\begin{important}
The examples in this section are used to illustrate the differences
between \idx{robust}, \idx{fragile} and \idx{expandable} commands.
In general, it's better not to use commands like \gls{gls} in
headings or captions (see \sectionref{sec:headings}) and using
commands like \gls{gls} in field values can be problematic (see
\sectionref{sec:nested}).
\end{important}

By default, most of the field values are expanded when the entry is
defined. This allows for defining entries programmatically, but it
can cause a problem if the value contains any \idx{fragile} commands.
For example:
\begin{lstlisting}[escapechar=|,moretexcs={footnote}]
\documentclass{article}

\usepackage{glossaries}

\newglossaryentry{duck}% label
{
  name = {duck},
  first = {duck\footnote{quack, quack}},|\incorrect|
  description = {a waterbird with webbed feet}
}

\begin{document}
A \gls{duck}.
\end{document}
\end{lstlisting}
This causes the confusing error:
\begin{verbatim}
! Undefined control sequence.
\in@ #1#2->\begingroup \def \in@@ 
\end{verbatim}
In order for this example to work, the \idx{fragile} command must either be
protected:
\begin{lstlisting}[escapechar=|,moretexcs={protect,footnote}]
\newglossaryentry{duck}% label
{
  name = {duck},
  first = {duck\protect\footnote{quack, quack}},|\correct|
  description = {a waterbird with webbed feet}
}
\end{lstlisting}
or the expansion must first be switched off:
\begin{lstlisting}[escapechar=|,moretexcs={footnote}]
\glsnoexpandfields |\correct|
\newglossaryentry{duck}% label
{
  name = {duck},
  first = {duck\footnote{quack, quack}},
  description = {a waterbird with webbed feet}
}
\end{lstlisting}
Since it's not possible to programmatically define entries with
\bibgls, the expansion is automatically switched off as \bibgls\
writes \gls{glsnoexpandfields} to the \ext{glstex} file (although
you can switch this feature off with \longarg{expand-fields}).

The reason why \gls{footnote} didn't cause a problem in the
\field{description} field \emph{when the entry was defined} is that, by
default, expansion isn't performed on the \field{name},
\field{description} and \field{symbol} fields, regardless of whether
or not \gls{glsnoexpandfields} has been used. This only applies to
the point when the entries are being defined. Unprotected
\idx{fragile} commands can still cause a problem if the value is
later used in a problematic context (such as the earlier example
where \gls{glsentrydesc} was used in a section heading).

\chapter{Abbreviations}
\label{sec:abbreviations}

The abbreviation handling provided by the base \isty{glossaries}
package is quite restrictive and only one abbreviation style can be
used for all abbreviations. The \isty{glossaries-extra} package
internally loads the \sty{glossaries} package and extends it,
providing new options and a better abbreviation mechanism that
allows different styles per category.

The base \sty{glossaries} package provides:
\nosecformatdef{newacronym}
The extension package \sty{glossaries-extra} provides:
\nosecformatdef{newabbreviation}
and redefines \gls{newacronym} in terms of \gls{newabbreviation} so
that it effectively behaves like:
\begin{codeenv}
\gls{newabbreviation}\oarg{type=\gls{acronymtype},category=acronym,\meta{\keyvallist}}
\margm{label}\margm{short}\margm{long}
\end{codeenv}
This makes it easier to transfer over from the base \sty{glossaries}
package, but if you use \gls{newacronym} remember that the category 
is set to \code{acronym} instead of \code{abbreviation}, which is
the usual default for \gls{newabbreviation}.

In both cases, \meta{label} is the entry's label used to identify
the abbreviation in commands like \gls{gls}, \meta{short} is the
short form and \meta{long} is the long form. Any additional
settings, such as the \field{category} or \field{description} 
can be set in the optional argument.

The style must always be set before the abbreviations are defined
using:
\nosecformatdef{setabbreviationstyle}
where \meta{category} is the category label and \meta{style-name} is
the name of the style. If the optional argument is omitted,
\code{abbreviation} is assumed. The \sty{glossaries-extra} package
automatically sets the default styles:
\begin{lstlisting}
\setabbreviationstyle{long-short}
\setabbreviationstyle[acronym]{short-nolong}
\end{lstlisting}
This means that if you don't explicitly set the style then any
abbreviation defined with \gls{newacronym} will use the
\abbrstyle{short-nolong} style (unless you change the category in
the optional argument) and other abbreviations will use the
\abbrstyle{long-short} style.

If these styles aren't suitable, then you need to change them. Any
abbreviation that's defined with a category that hasn't been
assigned a style will fallback on the default
style. There are many predefined styles to choose from and they come
with commands to help adjust the formatting. See the
\sty{glossaries-extra} user manual for the complete list. The
\sty{glossaries-extra} package also comes with a sample document
\ctanfile{glossaries-extra/samples}{sample-abbr-styles.pdf}
demonstrating all the predefined styles.

Some of the styles set the \field{description} field (typically to
the \meta{long} form). The styles that end with \code{-desc} don't,
and so that key must be set explicitly in the \meta{\keyvallist} optional part.

\chapter{Symbols}
\label{sec:symbols}

\chapter{Problematic Areas}
\label{sec:problems}

There are some places where the use of commands like \gls{gls} can
cause problems. These are listed below, with workarounds provided.

\section{Headings and Captions}
\label{sec:headings}

\section{Nesting}
\label{sec:nested}

\chapter{Displaying the Definition}
\label{sec:printgloss}

\section{Standalone}

\section{Listing the Terms (Glossary)}

\chapter{Changing the Formatting}
\label{sec:glsformats}

\section{Post-Link Category Hooks}
\label{sec:postlinkhooks}

Extra information can be appended after commands such as \gls{gls}
by defining a \pidxpl{postlinkhook} for the given category. You can
obtain the label of the entry that's just been referenced with:
\nosecformatdef{glslabel}
\Sectionref{sec:categories} gave a simple example, which is
reproduced below with some minor modifications:
\begin{lstlisting}[moretexcs={glsxtrpostlinkconstant,glsxtrpostlinkornament}]
\documentclass{article}

\usepackage{pifont}% provides \ding
\usepackage{glossaries-extra}

\newglossaryentry{fleuron}% label
{
  name = {fleuron},
  symbol = {\ding{167}},
  category = {ornament},
  description = {typographic ornament}
}

\newglossaryentry{pi}% label
{
  name = {pi},
  symbol = {\ensuremath{\pi}},
  category = {constant},
  description = {Archimedes' constant}
}

% post-link hook for 'ornament' category:
\newcommand{\glsxtrpostlinkornament}{% 
 \space (\glsentrydesc{\glslabel})}

% post-link hook for 'constant' category:
\newcommand{\glsxtrpostlinkconstant}{% 
 \space (\glsentrysymbol{\glslabel})}

\begin{document}
A \gls{fleuron} and \gls{pi}. Another \gls{fleuron} and \gls{pi}.
Symbols: \glssymbol{fleuron} and \glssymbol{pi}.
\end{document}
\end{lstlisting}
This produces:
\begin{result}
A fleuron (typographic ornament) and pi ($\pi$). Another fleuron (typographic
ornament) and pi ($\pi$). Symbols: \ding{167} (typographic ornament) and
$\pi$ ($\pi$).
\end{result}
The \idx{postlinkhook} is repeated after every instance of \gls{gls}
or \gls{glssymbol} etc. In the case of the \code{ornament} category,
the description is appended in parentheses and in the case of the
\code{constant} category the symbol is appended. This results in
redundant repetition, especially with \verb|\glssymbol{pi}| which
displays the symbol followed by the symbol in parentheses.

It's more likely that the information only needs to be appended
after the \idx{firstuse}. You can determine if the
\idx{postlinkhook} follows the \idx{firstuse} of the entry using:
\nosecformatdef{glsxtrifwasfirstuse}
For example:
\begin{lstlisting}
\newcommand{\glsxtrpostlinkconstant}{% 
 \glsxtrifwasfirstuse{\space (\glsentrysymbol{\glslabel})}{}%
}
\end{lstlisting}
Commands that don't check or modify the \idx{firstuseflag}, such as
\gls{glssymbol}, always set \gls{glsxtrifwasfirstuse} so that it
expands to \meta{false}. This means that even if \verb|\glssymbol{pi}|
is placed before the first instance of \verb|\gls{pi}| it still
won't be treated as the first use of that entry.

For convenience, there's a shortcut command:
\nosecformatdef{glsxtrpostlinkAddSymbolOnFirstUse}
So an alternative definition is:
\begin{lstlisting}
\newcommand{\glsxtrpostlinkconstant}{% 
  \glsxtrpostlinkAddSymbolOnFirstUse
}
\end{lstlisting}
Similarly, there's a shortcut command for the description:
\nosecformatdef{glsxtrpostlinkAddDescOnFirstUse}
As from \sty{glossaries-extra} v1.31, there's also a shortcut
command that you can use to define the \idx{postlinkhook}:
\nosecformatdef{glsdefpostlink} 
This is just a shortcut for:
\begin{codeenv}
\csfmt{csdef}\marg{glsxtrpostlink\meta{category}}\margm{definition}
\end{codeenv}
So the above document can be changed to:
\begin{lstlisting}
\documentclass{article}

\usepackage{pifont}% provides \ding
\usepackage{glossaries-extra}

\newglossaryentry{fleuron}% label
{
  name = {fleuron},
  symbol = {\ding{167}},
  category = {ornament},
  description = {typographic ornament}
}

\newglossaryentry{pi}% label
{
  name = {pi},
  symbol = {\ensuremath{\pi}},
  category = {constant},
  description = {Archimedes' constant}
}

% post-link hook for 'ornament' category:
\glsdefpostlink{ornament}{% 
  \glsxtrpostlinkAddDescOnFirstUse
}

% post-link hook for 'constant' category:
\glsdefpostlink{constant}{% 
  \glsxtrpostlinkAddSymbolOnFirstUse
}

\begin{document}
Symbols: \glssymbol{fleuron} and \glssymbol{pi}.
A \gls{fleuron} and \gls{pi}. Another \gls{fleuron} and \gls{pi}.
\end{document}
\end{lstlisting}
The result is now:
\begin{result}
Symbols: \ding{167} and $\pi$. A fleuron (typographic ornament) and
pi ($\pi$). Another fleuron and pi.
\end{result}

\section{Post-Name and Post-Description Hooks}
\label{sec:postfieldhooks}

\bibliographystyle{plain}
\bibliography{bib2gls-cite}
\end{document}
